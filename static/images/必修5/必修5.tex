% Generated by GrindEQ Word-to-LaTeX 
\documentclass{article} %%% use \documentstyle for old LaTeX compilers

\usepackage[english]{babel} %%% 'french', 'german', 'spanish', 'danish', etc.
\usepackage{amssymb}
\usepackage{amsmath}
\usepackage{txfonts}
\usepackage{mathdots}
\usepackage[classicReIm]{kpfonts}
\usepackage[dvips]{graphicx} %%% use 'pdftex' instead of 'dvips' for PDF output
\usepackage{ctex}
% You can include more LaTeX packages here 


\begin{document}

%\selectlanguage{english} %%% remove comment delimiter ('%') and select language if required
知识点:数列的概念

难度:1

将正整数的前5个数作如下排列:\textit{①}1,2,3,4,5;\textit{②}5,4,3,2,1;\textit{③}2,1,5,3,4;\textit{④}4,1,5,3,2\textit{.}则可以称为数列的是 (\textit{  })

 A.\textit{①} B.\textit{①②} C.\textit{①②③} D\textit{.①②③④}

 解析:4个都构成数列\textit{.}

 答案:D \\

知识点:数列的概念

难度:1

已知数列$\mathrm{\{}$\textit{a${}_{n}$}$\mathrm{\}}$的通项公式为\textit{a${}_{n}=\frac{1-(-1)^{n+1}}{2}$},则该数列的前4项依次为(\textit{  })

 A.1,0,1,0 B.0,1,0,1

 C.$\frac{1}{2}$,0,$\frac{1}{2}$,0 D.2,0,2,0

 解析:把\textit{n=}1,2,3,4分别代入\textit{a${}_{n}=\frac{1-(-1)^{n+1}}{2}$}中,依次得到0,1,0,1\textit{.}

 答案:B \\

知识点:数列的概念

难度:1
 
\textit{}数列$1,\frac{4}{\sqrt{3}},\frac{9}{\sqrt{5}},\frac{16}{\sqrt{7}},\dots$的一个通项公式是(\textit{  })

 A.\textit{a${}_{n}=\frac{n^2}{\sqrt{2n-1}}$} B.\textit{a${}_{n}=\frac{(n-1)^2}{\sqrt{2n-1}}$}

 C.\textit{a${}_{n}=\frac{n^2}{\sqrt{2n+1}}$} D.\textit{a${}_{n}=\frac{n^2-2n}{\sqrt{2n+1}}$}

 解析:1\textit{=}1${}^{2}$,4\textit{=}2${}^{2}$,9\textit{=}3${}^{2}$,16\textit{=}4${}^{2}$,1\textit{=}2\textit{$\times$}1\textit{-}1,3\textit{=}2\textit{$\times$}2\textit{-}1,5\textit{=}2\textit{$\times$}3\textit{-}1,7\textit{=}2\textit{$\times$}4\textit{-}1,故\textit{a${}_{n}=\frac{n^2}{\sqrt{2n-1}}$}\textit{.}

 答案:A \\

知识点:数列的概念

难度:1

\textit{}已知数列$\mathrm{\{}$\textit{a${}_{n}$}$\mathrm{\}}$的通项公式\textit{a${}_{n}=\frac{1}{n^2-1}$},若\textit{a${}_{k}=\frac{1}{35}$},则\textit{a}${}_{2}$\textit{${}_{k}$=} (\textit{  })

 A.$\frac{1}{99}$ B.99 C.$\frac{1}{143}$ D\textit{.}143

 解析:由\textit{a${}_{k}=\frac{1}{35}$}得$\frac{1}{k^2-1}=\frac{1}{35}$,于是\textit{k=}6(\textit{k=-}6舍去)\textit{.}

因此\textit{a}${}_{2}$\textit{${}_{k}$=a}${}_{12}$\textit{=}$\frac{1}{12^2-1}=\frac{1}{148}$\textit{.}
 
 答案:C \\

知识点:数列的概念

难度:1 
  
已知数列$\frac{1}{2},\frac{2}{3},\frac{3}{4},\frac{4}{5},\cdots$,则三个数0\textit{.}98,0\textit{.}96,0\textit{.}94中属于该数列中的数只有(\textit{  })

 A.1个 B.2个

 C.3个 D.以上都不对

 解析:由已知可得该数列的一个通项公式\textit{a${}_{n}=\frac{n}{n+1}$}\textit{.}令\textit{a${}_{n}$=}0\textit{.}98,解得\textit{n=}49,令\textit{a${}_{n}$=}0\textit{.}96,解得\textit{n=}24,令\textit{a${}_{n}$=}0\textit{.}94,解得\textit{n=}$\frac{47}{3}\mathrm{\notin}$N${}_{+}$\textit{.}故只有0\textit{.}98和0\textit{.}96是该数列中的项\textit{.} 

 答案:B \\

知识点:数列的概念

难度:1
  
已知曲线\textit{y=x}${}^{2}$\textit{+}1,点(\textit{n},\textit{a${}_{n}$})(\textit{n}$\mathrm{\in}$N\textit{${}_{+}$})位于该曲线上,则\textit{a}${}_{10}$\textit{=\underbar{     }.~}

 解析:由题意知\textit{a${}_{n}$=n}${}^{2}$\textit{+}1,因此\textit{a}${}_{10}$\textit{=}10${}^{2}$\textit{+}1\textit{=}101\textit{.}

 答案:101 \\

知识点:数列的概念

难度:1
  
数列$\sqrt{3},3,\sqrt{15},\sqrt{21},3\sqrt{3},\cdots$的一个通项公式是\textit{\underbar{          }.~}

 解析:数列可化为$\sqrt{3},\sqrt{9},\sqrt{15},\sqrt{21},\sqrt{27}$,即$\sqrt{3\times 1},\sqrt{3\times 3},\sqrt{3 \times 5},\sqrt{3 \times 7},\sqrt{3 \times 9},\cdots$,每个根号里面可分解成两数之积,前一个因式为常数3,后一个因式为2\textit{n-}1,故原数列的通项公式为\textit{a${}_{n}=\sqrt{3(2n-1)}=\sqrt{6n-3}$},\textit{n}$\mathrm{\in}$N\textit{${}_{+}$.}

 答案:\textit{a${}_{n}=\sqrt{6n-3}$} \\

知识点:数列的概念

难度:1
  
已知数列$\mathrm{\{}$\textit{a${}_{n}$}$\mathrm{\}}$的通项公式\textit{a${}_{n}=\frac{1}{\sqrt{n}+\sqrt{n+1}}$},则$\sqrt{10}-3$是此数列的第\textit{\underbar{     }}项\textit{.~}

 解析:令$\frac{1}{\sqrt{n}+\sqrt{n+1}}=\sqrt{10}-3$,得$\sqrt{n+1}-\sqrt{n}=\sqrt{10}-3$\textit{-}3,解得\textit{n=}9\textit{.}

 答案:9 \\

知识点:数列的概念

难度:1

写出下列各数列的一个通项公式:

 (1)4,6,8,10,{\dots}

 (2)$\frac{1}{2},\frac{3}{4},\frac{7}{8},\frac{15}{16},\frac{31}{32},\cdots$

 (3)$\frac{2}{3},-1,\frac{10}{7},-\frac{17}{9},\frac{26}{11},-\frac{37}{13},\cdots$

 (4)3,33,333,3 333,{\dots}

解析:

 答案:(1)各项是从4开始的偶数,所以$a_n=2n+2$.

(2)数列中的每一项分子比分母少1,而分母可写成2${}^{1}$,2${}^{2}$,2${}^{3}$,2${}^{4}$,2${}^{5}$,{\dots},2\textit{${}^{n}$},故所求数列的通项公式可写为\textit{a${}_{n}=\frac{2^n-1}{2^n}$}\textit{.}

(3)所给数列中正、负数相间,所以通项中必须含有(\textit{-}1)\textit{${}^{n+}$}${}^{1}$这个因式,忽略负号,将第二项1写成$\frac{5}{5}$,则分母可化为3,5,7,9,11,13,{\dots},均为正奇数,分子可化为$1^2+1,2^2+1,3^2+1,4^2+1,5^2+1,6^2+1,\cdots$,故其通项公式可写为$a_n=(-1)^{n+1}\frac{n^2+1}{2n+1}$

(4)将数列各项写为$\frac{9}{3},\frac{99}{3},\frac{999}{3},\frac{9999}{3},\cdots$,分母都是3,而分子分别是10\textit{-}1,10${}^{2}$\textit{-}1,10${}^{3}$\textit{-}1,10${}^{4}$\textit{-}1,{\dots},所以\textit{a${}_{n}=\frac{1}{3}$}(10\textit{${}^{n}$-}1)\textit{.} \\

知识点:数列的概念

难度:1
 
已知数列$\mathrm{\{}$\textit{a${}_{n}$}$\mathrm{\}}$的通项公式为\textit{a${}_{n}$=}3\textit{n}${}^{2}$\textit{-}28\textit{n.}

 (1)写出数列的第4项和第6项;

 (2)问\textit{-}49是不是该数列的一项?如果是,应是哪一项?68是不是该数列的一项呢?

解析:

 答案:(1)\textit{a}${}_{4}$\textit{=}3\textit{$\times$}16\textit{-}28\textit{$\times$}4\textit{=-}64,

\textit{a}${}_{6}$\textit{=}3\textit{$\times$}36\textit{-}28\textit{$\times$}6\textit{=-}60\textit{.}

(2)设3\textit{n}${}^{2}$\textit{-}28\textit{n=-}49,解得\textit{n=}7或\textit{n=}$\frac{7}{3}$(舍去),\textit{$\therefore$n=}7,即\textit{-}49是该数列的第7项\textit{.}

设3\textit{n}${}^{2}$\textit{-}28\textit{n=}68,解得\textit{n=}$\frac{34}{8}$或\textit{n=-}2\textit{.}

\textit{$\because$}$\frac{34}{3}$$\mathrm{\notin}$N\textit{${}_{+}$},\textit{-}2$\mathrm{\notin}$N\textit{${}_{+}$},

\textit{$\therefore$}68不是该数列的项\textit{.} \\



知识点:数列的概念

难度:2

$2,-\frac{8}{3},4,-\frac{32}{5},\cdots$的通项公式是(\textit{  })

 A.\textit{a${}_{n}$=}2\textit{${}^{n}$}(\textit{n}$\mathrm{\in}$N\textit{${}_{+}$}) B.\textit{a${}_{n}=\frac{(-2)^n}{2n-1}$}(\textit{n}$\mathrm{\in}$N\textit{${}_{+}$})

 C.\textit{a${}_{n}=\frac{(-2)^{n+1}}{n+1}$}(\textit{n}$\mathrm{\in}$N\textit{${}_{+}$}) D.\textit{a${}_{n}=\frac{2n}{2n-1}$}(\textit{n}$\mathrm{\in}$N\textit{${}_{+}$})

 解析:将数列各项改写为$\frac{2^2}{2},-\frac{2^3}{3},\frac{2^4}{4},-\frac{2^5}{5},\cdots$,观察数列的变化规律,可得\textit{a${}_{n}=\frac{(-2)^{n+1}}{n+1}$}(\textit{n}$\mathrm{\in}$N\textit{${}_{+}$})\textit{.}
 
 答案:C \\

知识点:数列的概念

难度:2

 
已知数列$\mathrm{\{}$\textit{a${}_{n}$}$\mathrm{\}}$的通项公式\textit{a${}_{n}=\frac{n}{n+1}$},则\textit{a${}_{n}$}·\textit{a${}_{n+}$}${}_{1}$·\textit{a${}_{n+}$}${}_{2}$等于(\textit{  })

 A.\textit{$ \frac{n}{n+2} $} B.\textit{$ \frac{n}{n+3} $} C.\textit{$ \frac{n+1}{n+2} $} D.\textit{$ \frac{n+1}{n+3} $}

 解析:\textit{$\because$a${}_{n}=\frac{n}{n+1}$},\textit{a${}_{n+}$}${}_{1}=\frac{n+1}{n+2}$,\textit{a${}_{n+}$}${}_{2}=\frac{n+2}{n+3}$\textit{$=\frac{n}{n+3}$},

\textit{$\therefore$a${}_{n}$}·\textit{a${}_{n+}$}${}_{1}$·\textit{a${}_{n+}$}${}_{2}$\textit{=}$\frac{n}{n+3}$\textit{.}

 答案:B \\

知识点:数列的概念

难度:2

 根据下列5个图形中相应点的个数的变化规律,猜测第\textit{n}个图形中有(\textit{  })个点\textit{.}

 \includegraphics*[width=2.82in, height=0.81in, keepaspectratio=false]{image64}

 A.$n^2-n+1$ B.$2n^2-n$

 C.$n^2$ D.2n-1

 解析:观察图中5个图形点的个数分别为$1,1\times 2 + 1,2\times 3 + 1,3\times 4 + 1,4\times 5 + 1$,故第\textit{n}个图形中点的个数为$(n-1)n+1=n^2-n+1$\textit{.}

 答案:A \\

知识点:数列的概念

难度:2

 用火柴棒按下图的方法搭三角形:

 \includegraphics*[width=2.78in, height=0.34in, keepaspectratio=false]{image65}

 按图示的规律搭下去,则所用火柴棒数\textit{a${}_{n}$}与所搭三角形的个数\textit{n}之间的关系式可以是\textit{\underbar{          }.~}

 解析:$\because a_1 = 3, a_2 = 3+2 = 5, a_3 = 3+2+2=7,a_4=3+2+2+2=9,\cdots,\therefore a_n=2n+1$

 答案:$a_n = 2n+1$ \\

知识点:数列的概念

难度:2

 在数列$\frac{\sqrt{5}}{3},\frac{\sqrt{10}}{8},
\frac{\sqrt{17}}{a+b},\frac{\sqrt{a-b}}{24},\cdots$中,有序数对(\textit{a},\textit{b})可以是\textit{\underbar{        }.~}

 解析:从上面的规律可以看出分母的规律是:1\textit{$\times$}3,2\textit{$\times$}4,3\textit{$\times$}5,4\textit{$\times$}6,{\dots},分子的规律是:$5,5+5,5+5+7,5+5+7+9,\cdots$

所以$
\left\{
\begin{array}{l}
a+b = 15, \\
a\cdot b=26.
\end{array}
\right.$,解得$a=\frac{41}{2},b=\frac{11}{2}$

 答案:$(\frac{41}{2},-\frac{11}{2})$ \\

知识点:数列的概念

难度:2

 已知数列$\mathrm{\{}$\textit{a${}_{n}$}$\mathrm{\}}$的通项公式$a_n=a2^n+b$,且\textit{a}${}_{1}$\textit{=-}1,\textit{a}${}_{5}$\textit{=-}31,则\textit{a}${}_{3}$\textit{=\underbar{     }.~}

 解析:由已知得\includegraphics*[width=1.00in, height=0.50in, keepaspectratio=false]{image72}解得\includegraphics*[width=0.50in, height=0.50in, keepaspectratio=false]{image73}

即$a_n = -2^n+1$,于是$a_3=-2^3+1=-7$

 答案:\textit{-}7 \\

知识点:数列的概念

难度:2

 如图,有\textit{m}(\textit{m}$\mathrm{\ge}$2)行(\textit{m+}1)列的士兵队列\textit{.}

\begin{tabular}{|p{0.1in}|p{0.1in}|p{0.1in}|p{0.1in}|p{0.1in}|p{0.1in}|p{0.1in}|p{0.1in}|} \hline 
· & · & · & {\dots} & · & · & · & · \\ \hline 
· & · & · & {\dots} & · & · & · & · \\ \hline 
{\dots} & {\dots} & {\dots} & {\dots} & {\dots} & {\dots} & {\dots} & {\dots} \\ \hline 
· & · & · & {\dots} & · & · & · & · \\ \hline 
· & · & · & {\dots} & · & · & · & · \\ \hline 
· & · & · & {\dots} & · & · & · & · \\ \hline 
\end{tabular}



 

 (1)写出一个数列,用它表示当\textit{m}分别为2,3,4,5,6,{\dots}时队列中的士兵人数;

 (2)写出(1)中数列的第5,6项,用\textit{a}${}_{5}$,\textit{a}${}_{6}$表示;

 (3)若把(1)中的数列记为$\mathrm{\{}$\textit{a${}_{n}$}$\mathrm{\}}$,求该数列的通项公式\textit{a${}_{n}$};

 (4)求\textit{a}${}_{10}$,并说明\textit{a}${}_{10}$所表示的实际意义\textit{.}

 

解析:

 答案:(1)当\textit{m=}2时,表示2行3列,人数为6;

当\textit{m=}3时,表示3行4列,人数为12,依此类推,故所求数列为6,12,20,30,42,{\dots}\textit{.}

(2)队列的行数比数列的序号大1,因此第5项表示的是6行7列,第6项表示7行8列,故\textit{a}${}_{5}$\textit{=}42,\textit{a}${}_{6}$\textit{=}56\textit{.}

(3)根据对数列的前几项的观察、归纳,猜想数列的通项公式\textit{.}

前4项分别为6\textit{=}2\textit{$\times$}3,12\textit{=}3\textit{$\times$}4,20\textit{=}4\textit{$\times$}5,30\textit{=}5\textit{$\times$}6\textit{.}因此$a_n=(n+1)(n=1)$

(4)由(3)知\textit{a}${}_{10}$\textit{=}11\textit{$\times$}12\textit{=}132,\textit{a}${}_{10}$表示11行12列的士兵队列中士兵的人数\textit{.} \\

知识点:数列的概念

难度:2

 在数列$\mathrm{\{}$\textit{a${}_{n}$}$\mathrm{\}}$中,\textit{a}${}_{1}$\textit{=}2,\textit{a}${}_{17}$\textit{=}66,通项公式是关于\textit{n}的一次函数\textit{.}

 (1)求数列$\mathrm{\{}$\textit{a${}_{n}$}$\mathrm{\}}$的通项公式;

 (2)求\textit{a}${}_{2017}$;

 (3)是否存在\textit{m},\textit{k}$\mathrm{\in}$N${}_{+}$,满足\textit{a${}_{m}$+a${}_{m+}$}${}_{1}$\textit{=a${}_{k}$}?若存在,求出\textit{m},\textit{k}的值,若不存在,说明理由\textit{.}

解析:

 答案:(1)设$a_n=kn+b$(\textit{k}$\mathrm{\neq}$0),则由\textit{a}${}_{1}$\textit{=}2,\textit{a}${}_{17}$\textit{=}66得,

$\left\{
\begin{array}{l}
k+b=2, \\
17k+b=66.
\end{array}
\right.$解得$\left\{
\begin{array}{l}
k=4, \\
b=-2.
\end{array}
\right.$

所以\textit{a${}_{n}$=}4\textit{n-}2\textit{.}

(2)\textit{a}${}_{2017}$\textit{=}4\textit{$\times$}2017\textit{-}2\textit{=}8 066\textit{.}

(3)由\textit{a${}_{m}$+a${}_{m+}$}${}_{1}$\textit{=a${}_{k}$},得4\textit{m-}2\textit{+}4(\textit{m+}1)\textit{-}2\textit{=}4\textit{k-}2,

整理后可得4\textit{m=}2\textit{k-}1,

因为\textit{m},\textit{k}$\mathrm{\in}$N${}_{+}$,所以4\textit{m}是偶数,2\textit{k-}1是奇数,

故不存在\textit{m},\textit{k}$\mathrm{\in}$N${}_{+}$,使等式4\textit{m=}2\textit{k-}1成立,

即不存在\textit{m},\textit{k}$\mathrm{\in}$N${}_{+}$,使\textit{a${}_{m}$+a${}_{m+}$}${}_{1}$\textit{=a${}_{k}$.} \\

知识点:数列的概念

难度:1

 数列$\{n^2-4n+3\}$的图像是(\textit{  })

 A.一条直线

 B.一条直线上的孤立的点

 C.一条抛物线

 D.一条抛物线上的孤立的点

 解析:$a_n = n^2-4n+3$是关于\textit{n}的二次函数,故其图像是抛物线$y=x^2-4x+3$上一群孤立的点\textit{.}

 答案:D \\

知识点:数列的概念

难度:1

 已知数列$\mathrm{\{}$\textit{a${}_{n}$}$\mathrm{\}}$的通项公式是$a_n = \frac{2n}{3n+1}$,则这个数列是 (\textit{  })

 \textit{                }

 A.递增数列 B.递减数列

 C.摆动数列 D.常数列

 解析:$\because a_{n+1}-a_n = \frac{2(n+1)}{3(n+1)+1}-\frac{2n}{3n+1}$

$=\frac{2}{[3(n+1)+1](3n+1)}>0$,

$\therefore a_{n+1} > a_n$,

\textit{$\therefore$}数列$\mathrm{\{}$\textit{a${}_{n}$}$\mathrm{\}}$是递增数列\textit{.}

 答案:A \\

知识点:数列的概念

难度:1

 若数列$\mathrm{\{}$\textit{a${}_{n}$}$\mathrm{\}}$的通项公式$a_n=\frac{3n-5}{3n-14}$,则在数列$\mathrm{\{}$\textit{a${}_{n}$}$\mathrm{\}}$的前20项中,最大项和最小项分别是(\textit{  })

 A.\textit{a}${}_{1}$,\textit{a}${}_{20}$ B.\textit{a}${}_{20}$,\textit{a}${}_{1}$ C.\textit{a}${}_{5}$,\textit{a}${}_{4}$ D.\textit{a}${}_{4}$,\textit{a}${}_{5}$

 解析:由于$a_n=\frac{3n-5}{3n-14}=\frac{3n-14+9}{3n-14}=1+\frac{3}{n-\frac{14}{3}}$,因此当1$\mathrm{\le}$\textit{n}$\mathrm{\le}$4时,$\mathrm{\{}$\textit{a${}_{n}$}$\mathrm{\}}$是递减的,且\textit{a}${}_{1}$\textit{$>$}0\textit{$>$a}${}_{2}$\textit{$>$a}${}_{3}$\textit{$>$a}${}_{4}$;当5$\mathrm{\le}$\textit{n}$\mathrm{\le}$20时,\textit{a${}_{n}$$>$}0,且$\mathrm{\{}$\textit{a${}_{n}$}$\mathrm{\}}$也是递减的,即\textit{a}${}_{5}$\textit{$>$a}${}_{6}$\textit{$>$}{\dots}\textit{$>$a}${}_{20}$\textit{$>$}0,因此最大的是\textit{a}${}_{5}$,最小的是\textit{a}${}_{4}$\textit{.}

 答案:C \\

知识点:数列的概念

难度:1

 已知$\mathrm{\{}$\textit{a${}_{n}$}$\mathrm{\}}$的通项公式$a_n = n^2+3kn$,且$\mathrm{\{}$\textit{a${}_{n}$}$\mathrm{\}}$是递增数列,则实数\textit{k}的取值范围是(\textit{  })

 A.\textit{k}$\mathrm{\ge}$\textit{-}1 B.$k>-\frac{2}{3}$ C.$k\le -\frac{2}{3}$ D.\textit{k$>$-}1

 解析:因为$\mathrm{\{}$\textit{a${}_{n}$}$\mathrm{\}}$是递增数列,所以\textit{a${}_{n+}$}${}_{1}$\textit{$>$a${}_{n}$}对\textit{n}$\mathrm{\in}$N\textit{${}_{+}$}恒成立\textit{.}即$(n+1)^2+3k(n+1)>n^2+3kn$,整理得$k>-\frac{2n+1}{3}$,当\textit{n=}1时,$-\frac{2n+1}{3}$取最大值\textit{-}1,故\textit{k$>$-}1\textit{.}

 答案:D \\

知识点:数列的概念

难度:1

 给定函数\textit{y=f}(\textit{x})的图像,对任意\textit{a${}_{n}$}$\mathrm{\in}$(0,1),由关系式\textit{a${}_{n+}$}${}_{1}$\textit{=f}(\textit{a${}_{n}$})得到的数列$\mathrm{\{}$\textit{a${}_{n}$}$\mathrm{\}}$满足\textit{a${}_{n+}$}${}_{1}$\textit{$>$a${}_{n}$}(\textit{n}$\mathrm{\in}$N\textit{${}_{+}$}),则该函数的图像是(\textit{  })

 \includegraphics*[width=2.68in, height=0.85in, keepaspectratio=false]{image87}

 解析:由\textit{a${}_{n+}$}${}_{1}$\textit{$>$a${}_{n}$}可知数列$\mathrm{\{}$\textit{a${}_{n}$}$\mathrm{\}}$为递增数列,又由\textit{a${}_{n+}$}${}_{1}$\textit{=f}(\textit{a${}_{n}$})\textit{$>$a${}_{n}$}可知,当\textit{x}$\mathrm{\in}$(0,1)时,\textit{y=f}(\textit{x})的图像在直线\textit{y=x}的上方\textit{.}

 答案:A \\

知识点:数列的概念

难度:1

 已知数列$\mathrm{\{}$\textit{a${}_{n}$}$\mathrm{\}}$的通项公式是$a_n=\frac{an}{bn+1}$,其中\textit{a},\textit{b}均为正常数,则\textit{a${}_{n+}$}${}_{1}$与\textit{a${}_{n}$}的大小关系是\textit{\underbar{ }.~}

 解析:$\because a_{n+1}-a_n = \frac{a(n+1)}{b(n+1)+1}-\frac{an}{bn+1}$

$=\frac{a}{[b(n+1)](bn+1)}>0$,

\textit{$\therefore$a${}_{n+}$}${}_{1}$\textit{-a${}_{n}$$>$}0,故\textit{a${}_{n+}$}${}_{1}$\textit{$>$a${}_{n}$.}

 答案:\textit{a${}_{n+}$}${}_{1}$\textit{$>$a${}_{n}$} \\

知识点:数列的概念

难度:1

 已知数列$\mathrm{\{}$\textit{a${}_{n}$}$\mathrm{\}}$的通项公式为$a_n=2n^2-5n+2$,则数列$\mathrm{\{}$\textit{a${}_{n}$}$\mathrm{\}}$的最小值是\textit{\underbar{     }.~}

 解析:$\because a_n = 2n^2-5n+2 = 2(n-\frac{5}{4})^2-\frac{9}{8}$,

\textit{$\therefore$}当\textit{n=}1时,\textit{a${}_{n}$}最小,最小为\textit{a}${}_{1}$\textit{=-}1\textit{.}

 答案:\textit{-}1 \\

知识点:数列的概念

难度:1

 已知数列$\mathrm{\{}$\textit{a${}_{n}$}$\mathrm{\}}$满足$a_{n+1}=\left\{
\begin{array}{l}
2a_n(0 <a_n <\frac{1}{2}), \\
2a_n-1(\frac{1}{2}\le a_n <1),
\end{array}\right.$,若$a_1=\frac{6}{7}$,则$a_{2017}$\textit{=\underbar{ }.~}

 解析:$a_1=\frac{6}{7},a_2 =2a_1-1= \frac{5}{7},a_3=2a_2-1=\frac{3}{7},a_4=2a_3=\frac{6}{7},\cdots$,所以$\mathrm{\{}$\textit{a${}_{n}$}$\mathrm{\}}$是周期为3的周期数列,于是$a_{2017}=a_{672\times 3+1}=a_1=\frac{6}{7}$\textit{.}

 答案:$\frac{6}{7}$ \\

知识点:数列的概念

难度:1

 已知数列$\mathrm{\{}$\textit{a${}_{n}$}$\mathrm{\}}$的通项公式为$a_n=n^2-21n+20$\textit{.}

 (1)\textit{-}60是否是该数列中的项,若是,求出项数;该数列中有小于0的项吗?有多少项?

 (2)\textit{n}为何值时,\textit{a${}_{n}$}有最小值?并求出最小值\textit{.}

解析:

 答案:(1)令$n^2-21n+20=-60$,得\textit{n=}5或\textit{n=}16\textit{.}

所以数列的第5项,第16项都为\textit{-}60\textit{.}

由$n^2-21n+20<0$,得1\textit{$<$n$<$}20,所以共有18项小于0\textit{.}

(2)由$a_n = n^2-21n+20=(n-\frac{21}{2})^2-\frac{361}{4}$,可知对称轴方程为$n=\frac{21}{5}=10.5$\textit{.}又\textit{n}$\mathrm{\in}$N\textit{${}_{+}$},故\textit{n=}10或\textit{n=}11时,\textit{a${}_{n}$}有最小值,其最小值为$11^2-21\times 11+20=-90$\textit{.} \\

知识点:数列的概念

难度:1

 已知函数$f(x) = \frac{1-2x}{x+1}$(\textit{x}$\mathrm{\ge}$1),构造数列\textit{a${}_{n}$=f}(\textit{n})(\textit{n}$\mathrm{\in}$N${}_{+}$)\textit{.}

 (1)求证:\textit{a${}_{n}$$>$-}2;

 (2)数列$\mathrm{\{}$\textit{a${}_{n}$}$\mathrm{\}}$是递增数列还是递减数列?为什么?

解析:

 答案:(1)证明由题意可知$a_n = \frac{1-2n}{n+1}=\frac{3-2(n+1)}{n+1}=\frac{3}{n+1}-2$\textit{.}

\textit{$\because$n}$\mathrm{\in}$N${}_{+}$,\textit{$\therefore$}$\frac{3}{n+1}>0$,$\therefore a_n = \frac{3}{n+1}-2 > -2$\textit{.}

 \eqref{GrindEQ__2_}解递减数列\textit{.}

理由如下:由(1)知,$a_n = \frac{3}{n+1}-2$\textit{.}

$\because a_{n+1}-a_n = \frac{3}{(n+1)+1}-\frac{3}{n+1}$

$=\frac{3n+3-3n-6}{(n+1)(n+2)} = \frac{-3}{(n+1)(n+2)}<0$,

即\textit{a${}_{n+}$}${}_{1}$\textit{$<$a${}_{n}$},\textit{$\therefore$}数列$\mathrm{\{}$\textit{a${}_{n}$}$\mathrm{\}}$是递减数列\textit{.} \\

知识点:数列的概念

难度:2

 若函数\textit{f}(\textit{x})满足$f(1)=1,f(n+1)=f(n)+3$(\textit{n}$\mathrm{\in}$N\textit{${}_{+}$}),则\textit{f}(\textit{n})是(\textit{  })

 A.递增数列 B.递减数列

 C.常数列 D.不能确定

 解析:\textit{$\because$}$f(n+1)-f(n)=3$(\textit{n}$\mathrm{\in}$N\textit{${}_{+}$}),

\textit{$\therefore$f}$f(n+1)>f(n)$,

\textit{$\therefore$f}(\textit{n})是递增数列\textit{.}

 答案:A \\

知识点:数列的概念

难度:2

 设函数$f(x)=\left\{
\begin{array}{l}
(3-a)x-3,x\le 7, \\
a^{x-6},x>7. 
\end{array}
\right.$,数列$\mathrm{\{}$\textit{a${}_{n}$}$\mathrm{\}}$满足\textit{a${}_{n}$=f}(\textit{n}),\textit{n}$\mathrm{\in}$N\textit{${}_{+}$},且数列$\mathrm{\{}$\textit{a${}_{n}$}$\mathrm{\}}$是递增数列,则实数\textit{a}的取值范围是(\textit{  })

 A.(1,3) B.(2,3) C.$(\frac{9}{4},2)$ D.(1,2)

 答案:B \\

知识点:数列的概念

难度:2

 若数列$\mathrm{\{}$\textit{a${}_{n}$}$\mathrm{\}}$的通项公式为$a_n=7(\frac{3}{4})^{2n-2}-3(\frac{3}{4})^{n-1}$,则数列$\mathrm{\{}$\textit{a${}_{n}$}$\mathrm{\}}$的(\textit{  })

 A.最大项为\textit{a}${}_{5}$,最小项为\textit{a}${}_{6}$

 B.最大项为\textit{a}${}_{6}$,最小项为\textit{a}${}_{7}$

 C.最大项为\textit{a}${}_{1}$,最小项为\textit{a}${}_{6}$

 D.最大项为\textit{a}${}_{7}$,最小项为\textit{a}${}_{6}$

 解析:令$t=(\frac{3}{4})^{n-1}$,\textit{n}$\mathrm{\in}$N\textit{${}_{+}$},则\textit{t}$\mathrm{\in}$(0,1],且$(\frac{3}{4})^{2n-2}=[(\frac{3}{4})^{n-1}]^2=t^2$.从而$a_n = 7t^2-3t=7(t-\frac{3}{14})^2-\frac{9}{28}$.

又函数\textit{f}(\textit{t})\textit{=}7\textit{t}${}^{2}$\textit{-}3\textit{t}在$(0,\frac{3}{14}]$上是减少的,在$[\frac{3}{14},1]$上是增加的,所以\textit{a}${}_{1}$是最大项,\textit{a}${}_{6}$是最小项\textit{.}故选C\textit{.}

 答案:C \\

知识点:数列的概念

难度:2

 若数列$\mathrm{\{}$\textit{a${}_{n}$}$\mathrm{\}}$的通项公式为$a_n=-2n^2+13n$,关于该数列,有以下四种说法:

 \textit{①}该数列有无限多个正数项;\textit{②}该数列有无限多个负数项;\textit{③}该数列的最大值就是函数$f(x)=-2x^2+13x$的最大值;\textit{④-}70是该数列中的一项\textit{.}

 其中正确的说法有\textit{\underbar{      }.}(填序号)~

 解析:令$-2n^2+13n>0$,得$0<n<\frac{13}{2}$,故数列$\mathrm{\{}$\textit{a${}_{n}$}$\mathrm{\}}$中有6项是正数项,有无限个负数项,所以\textit{①}错,\textit{②}正确;当\textit{n=}3时,数列$\mathrm{\{}$\textit{a${}_{n}$}$\mathrm{\}}$取到最大值,而当\textit{x=}3\textit{.}25时,函数\textit{f}(\textit{x})取到最大值,所以\textit{③}错;令\textit{-}2\textit{n}${}^{2}$\textit{+}13\textit{n=-}70,得\textit{n=}10或$n=-\frac{7}{2}$(舍去),即\textit{-}70是该数列的第10项,所以\textit{④}正确\textit{.}
 
 答案:\textit{②④}   \\

知识点:数列的概念

难度:2

 若数列$\{n(n+4)(\frac{2}{3})^n\}$中的最大项是第\textit{k}项,则\textit{k=\underbar{     }.~}

 解析:已知数列最大项为第\textit{k}项,则有

$\left\{
\begin{array}{l}
k(k+4)(\frac{2}{3})^k\ge (k+1)(k+5)(\frac{2}{3})^{k+1}, \\
k(k+4)(\frac{2}{3})^k\ge (k-1)(k+3)(\frac{2}{3})^{k-1}
\end{array}
\right.$

即$\left\{
\begin{array}{l}
k^2\ge 10, \\
k^2-2k-9\le 0.
\end{array}
\right.$由\textit{k}$\mathrm{\in}$N\textit{${}_{+}$}可得\textit{k=}4\textit{.}

 答案:4 \\

知识点:数列的概念

难度:2

 已知数列$\mathrm{\{}$\textit{a${}_{n}$}$\mathrm{\}}$满足$a_n = \frac{1}{n+1}+\frac{1}{n+2}+\frac{1}{n+3}+\cdots + \frac{1}{2n}$\textit{.}

 (1)数列$\mathrm{\{}$\textit{a${}_{n}$}$\mathrm{\}}$是递增数列还是递减数列?为什么?

 (2)证明:$a_n \le \frac{1}{2}$对一切正整数恒成立\textit{.}

解析:

 答案:(1)因为$a_n = \frac{1}{n+1}+\frac{1}{n+2}+\frac{1}{n+3}+\cdots+\frac{1}{2n}$,

所以$a_{n+1} = \frac{1}{(n+1)+1}+\frac{1}{(n+1)+2}+\frac{1}{(n+1)+3}+\cdots +\frac{1}{2(n+1)}$

$=\frac{1}{n+2}+\frac{1}{n+3}+\frac{1}{n+4}+\cdots +\frac{1}{2n}+\frac{1}{2n+1}+\frac{1}{2n+2}$

所以$a_{n+1}-a_n = \frac{1}{2n+1}+\frac{1}{2n+2}-\frac{1}{n-1}=\frac{1}{2n+1}-\frac{1}{2n+2}$,

又\textit{n}$\mathrm{\in}$N\textit{${}_{+}$},所以$\frac{1}{2n+1}>\frac{1}{2n+2}$\textit{.}

所以\textit{a${}_{n+}$}${}_{1}$\textit{-a${}_{n}$$>$}0\textit{.}

所以数列$\mathrm{\{}$\textit{a${}_{n}$}$\mathrm{\}}$是递增数列\textit{.}

 (2)证明由(1)知数列$\mathrm{\{}$\textit{a${}_{n}$}$\mathrm{\}}$是递增数列,所以数列的最小项为$a_1 = \frac{1}{2}$,所以$a_n\ge a_1=\frac{1}{2}$,即$a_n \ge \frac{1}{2}$对一切正整数恒成立\textit{.} \\

知识点:数列的概念

难度:2

 已知数列$\mathrm{\{}$\textit{a${}_{n}$}$\mathrm{\}}$的通项公式为\textit{a${}_{n}$=n}${}^{2}$\textit{-n-}30\textit{.}

(1)求数列的前三项,60是此数列的第几项?

 (2)\textit{n}为何值时,\textit{a${}_{n}$=}0,\textit{a${}_{n}$$>$}0,\textit{a${}_{n}$$<$}0?

 (3)该数列前\textit{n}项和\textit{S${}_{n}$}是否存在最值?说明理由\textit{.}

解析:

  答案:(1)由\textit{a${}_{n}$=n}${}^{2}$\textit{-n-}30,得\textit{a}${}_{1}$\textit{=}1\textit{-}1\textit{-}30\textit{=-}30,\textit{a}${}_{2}$\textit{=}2${}^{2}$\textit{-}2\textit{-}30\textit{=-}28,\textit{a}${}_{3}$\textit{=}3${}^{2}$\textit{-}3\textit{-}30\textit{=-}24\textit{.}

设\textit{a${}_{n}$=}60,则\textit{n}${}^{2}$\textit{-n-}30\textit{=}60\textit{.}

解得\textit{n=}10或\textit{n=-}9(舍去),即60是此数列的第10项\textit{.}

(2)令\textit{n}${}^{2}$\textit{-n-}30\textit{=}0,解得\textit{n=}6或\textit{n=-}5(舍去)\textit{.}

\textit{$\therefore$}当\textit{n=}6时,\textit{a${}_{n}$=}0\textit{.}

令\textit{n}${}^{2}$\textit{-n-}30\textit{$>$}0,解得\textit{n$>$}6或\textit{n$<$-}5(舍去)\textit{.}

\textit{$\therefore$}当\textit{n$>$}6(\textit{n}$\mathrm{\in}$N\textit{${}_{+}$})时,\textit{a${}_{n}$$>$}0\textit{.}

令\textit{n}${}^{2}$\textit{-n-}30\textit{$<$}0,解得\textit{-}5\textit{$<$n$<$}6\textit{.}

又\textit{n}$\mathrm{\in}$N\textit{${}_{+}$},\textit{$\therefore$}0\textit{$<$n$<$}6,

\textit{$\therefore$}当0\textit{$<$n$<$}6(\textit{n}$\mathrm{\in}$N\textit{${}_{+}$})时,\textit{a${}_{n}$$<$}0\textit{.}

(3)由$a_n = n^2-n-30=(n-\frac{1}{2})^2-30-\frac{1}{4}$(\textit{n}$\mathrm{\in}$N\textit{${}_{+}$}),知$\mathrm{\{}$\textit{a${}_{n}$}$\mathrm{\}}$是递增数列,

且\textit{a}${}_{1}$\textit{$<$a}${}_{2}$\textit{$<$}{\dots}\textit{$<$a}${}_{5}$\textit{$<$a}${}_{6}$\textit{=}0\textit{$<$a}${}_{7}$\textit{$<$a}${}_{8}$\textit{$<$a}${}_{9}$\textit{$<$}{\dots},

故\textit{S${}_{n}$}存在最小值\textit{S}${}_{5}$\textit{=S}${}_{6}$,\textit{S${}_{n}$}不存在最大值\textit{.} \\

知识点:等差数列的概念

难度:1

 若$\mathrm{\{}$\textit{a${}_{n}$}$\mathrm{\}}$是等差数列,则下列数列中也成等差数列的是 (\textit{  })

 \textit{                }

 A.$\{a_n^2\}$ B.$\{\frac{1}{a_n}\}$ C.$\mathrm{\{}$3\textit{a${}_{n}$}$\mathrm{\}}$ D.$\{|a_n|\}$

 解析:设$\mathrm{\{}$\textit{a${}_{n}$}$\mathrm{\}}$的公差为\textit{d},则3\textit{a${}_{n+}$}${}_{1}$\textit{-}3\textit{a${}_{n}$=}3(\textit{a${}_{n+}$}${}_{1}$\textit{-a${}_{n}$})\textit{=}3\textit{d}是常数,故$\mathrm{\{}$3\textit{a${}_{n}$}$\mathrm{\}}$一定成等差数列\textit{.}

$\{a_n^2\},\{\frac{1}{a_n}\},\{|a_n|\}$都不一定是等差数列,例如当$\mathrm{\{}$\textit{a${}_{n}$}$\mathrm{\}}$为$\mathrm{\{}$3,1,\textit{-}1,\textit{-}3$\mathrm{\}}$时\textit{.}

 答案:C \\

知识点:等差数列的概念

难度:1

 在等差数列$\mathrm{\{}$\textit{a${}_{n}$}$\mathrm{\}}$中,$a_1+a_5 = 10$,\textit{a}${}_{4}$\textit{=}7,则数列$\mathrm{\{}$\textit{a${}_{n}$}$\mathrm{\}}$的公差为(\textit{  })

 A.1 B.2 C.3 D.4

 解析:$\because a_1+a_5 = 10=a_1+a_1+4d=2(a_1+2d)=2a_3$,

$\therefore a_3 = 5$.故$d =a_4-a_3 = 7-5=2$\textit{.}

 答案:B \\

知识点:等差数列的概念

难度:1

 已知$\mathrm{\{}$\textit{a${}_{n}$}$\mathrm{\}}$是首项\textit{a}${}_{1}$\textit{=}2,公差为\textit{d=}3的等差数列,若\textit{a${}_{n}$=}2 018,则序号\textit{n}等于(\textit{  })

 A.670 B.671 C.672 D.673

 解析:\textit{$\because$a}${}_{1}$\textit{=}2,\textit{d=}3,$\therefore a_n = 2+3(n-1) = 3n-1$\textit{.}

令3\textit{n-}1\textit{=}2018,解得\textit{n=}673\textit{.}

 答案:D \\

知识点:等差数列的概念

难度:1

 等差数列$\mathrm{\{}$\textit{a${}_{n}$}$\mathrm{\}}$中,\textit{a}${}_{1}$\textit{=}8,\textit{a}${}_{5}$\textit{=}2,如果在每相邻两项间各插入一个数,使之成为新的等差数列,那么新的等差数列的公差是(\textit{  })

 A.$\frac{3}{4}$ B.$-\frac{3}{4}$ C.$-\frac{6}{7}$ D.-1

 解析:设新数列\textit{a}${}_{1}$,\textit{b}${}_{1}$,\textit{a}${}_{2}$,\textit{b}${}_{2}$,\textit{a}${}_{3}$,\textit{b}${}_{3}$,\textit{a}${}_{4}$,\textit{b}${}_{4}$,\textit{a}${}_{5}$,{\dots},公差为\textit{d},则$a_5 = a_1 +8d$,所以$d = \frac{a_5 - a_1}{8} = \frac{2 -8 }{8}=-\frac{6}{8} = -\frac{3}{4}$\textit{.}故选B\textit{.}
 
 答案:B \\

知识点:等差数列的概念

难度:1

 已知点(\textit{n},\textit{a${}_{n}$})(\textit{n}$\mathrm{\in}$N\textit{${}_{+}$})都在直线3\textit{x-y-}24\textit{=}0上,则在数列$\mathrm{\{}$\textit{a${}_{n}$}$\mathrm{\}}$中有(\textit{  })

 A.$a_7+a_9>0$ B.$a_7+a_9<0$

 C.$a_7+a_9=0$ D.$a_7 \cdot a_9 = 0$

 解析:\textit{$\because$}(\textit{n},\textit{a${}_{n}$})在直线3\textit{x-y-}24\textit{=}0,\textit{$\therefore$a${}_{n}$=}3\textit{n-}24\textit{.}

\textit{$\therefore$a}${}_{7}$\textit{=}3\textit{$\times$}7\textit{-}24\textit{=-}3,\textit{a}${}_{9}$\textit{=}3\textit{$\times$}9\textit{-}24\textit{=}3,

$\therefore a_7+a_9=0$\textit{.}

 答案:C \\

知识点:等差数列的概念

难度:1

 在等差数列$\mathrm{\{}$\textit{a${}_{n}$}$\mathrm{\}}$中,若\textit{a}${}_{1}$\textit{=}7,\textit{a}${}_{7}$\textit{=}1,则\textit{a}${}_{5}$\textit{=\underbar{     }.~}

 答案:3 \\

知识点:等差数列的概念

难度:1

 在等差数列$\mathrm{\{}$\textit{a${}_{n}$}$\mathrm{\}}$中,已知\textit{a}${}_{5}$\textit{=}10,\textit{a}${}_{12}$\textit{$>$}31,则公差\textit{d}的取值范围是\textit{\underbar{        }.~}

 解析:设此数列的首项为\textit{a}${}_{1}$,公差为\textit{d},

由已知得$\left\{
\begin{array}{l}
a_1+4d=10, ① \\
a_1+11d>31,②. 
\end{array}
\right.$

\textit{②-①},得7\textit{d$>$}21,所以\textit{d$>$}3\textit{.}

 答案:\textit{d$>$}3 \\

知识点:等差数列的概念

难度:1

 在数列$\mathrm{\{}$\textit{a${}_{n}$}$\mathrm{\}}$中,\textit{a}${}_{1}$\textit{=}3,且对任意大于1的正整数\textit{n},点$(\sqrt{a_n},\sqrt{a_{n-1}})$在直线$x-y-\sqrt{3} = 0$上,则数列$\mathrm{\{}$\textit{a${}_{n}$}$\mathrm{\}}$的通项公式为\textit{a${}_{n}$=\underbar{     }.~}

 解析:由题意知$\sqrt{a_n} - \sqrt{a_{n-1}} = \sqrt{3}(n \ge 2)$,

$\therefore \{\sqrt{a_n}\}$是以$\sqrt{a_1}$为首项,以$\sqrt{3}$为公差的等差数列,

$\therefore \sqrt{a_n} = \sqrt{a_1} + (n-1)d = \sqrt{3}+\sqrt{3}(n-1) = \sqrt{3} n$\textit{n.}

\textit{$\therefore$a${}_{n}$=}3\textit{n}${}^{2}$\textit{.}

 答案:3\textit{n}${}^{2}$ \\

知识点:等差数列的概念

难度:1

 已知数列$\mathrm{\{}$\textit{a${}_{n}$}$\mathrm{\}}$,$\mathrm{\{}$\textit{b${}_{n}$}$\mathrm{\}}$满足$\{\frac{1}{a_n+b_n}\}$是等差数列,且\textit{b${}_{n}$=n}${}^{2}$,\textit{a}${}_{2}$\textit{=}5,\textit{a}${}_{8}$\textit{=}8,则\textit{a}${}_{9}$\textit{=\underbar{       }.~}

 解析:由题意得$\frac{1}{a_2+b_2} = \frac{1}{9}, \frac{1}{a_8+b_8} = \frac{1}{72}$,

因为$\{\frac{1}{a_n+b_n}\}$是等差数列,所以可得该等差数列的公差$d=-\frac{7}{72\times 6}$,

所以$\frac{1}{a_9+b_9} = \frac{1}{72}-\frac{7}{72\times 6} = -\frac{1}{432}$,所以$a_9 = -513$.

 答案:$-513$ \\

知识点:等差数列的概念

难度:1

 如果在等差数列$\mathrm{\{}$3\textit{n-}1$\mathrm{\}}$的每相邻两项之间插入三项后使它们构成一个新的等差数列,那么新数列的第29项是原数列的第\textit{\underbar{     }}项\textit{.~}

 解析:设\textit{a${}_{n}$=}3\textit{n-}1,公差为\textit{d}${}_{1}$,新数列为$\mathrm{\{}$\textit{b${}_{n}$}$\mathrm{\}}$,公差为\textit{d}${}_{2}$,\textit{a}${}_{1}$\textit{=}2,\textit{b}${}_{1}$\textit{=}2,\textit{d}${}_{1}$\textit{=a${}_{n}$-a${}_{n-}$}${}_{1}$\textit{=}3,$d_2 = \frac{d-1}{4} = \frac{3}{4}$,则$b_n = 2+\frac{3}{4}(n-1) = \frac{3}{4}n+\frac{5}{4},b_{29} = 23 $,令$a_n = 23$,即$3n-1=23$.故$n=8$.

 答案:8 \\

知识点:等差数列的概念

难度:1

 若一个数列$\{a_n\}$满足$a_n+a_{n-1} = h$,其中\textit{h}为常数,\textit{n}$\mathrm{\ge}$2且\textit{n}$\mathrm{\in}$N\textit{${}_{+}$},则称数列$\mathrm{\{}$\textit{a${}_{n}$}$\mathrm{\}}$为等和数列,\textit{h}为公和\textit{.}已知等和数列$\mathrm{\{}$\textit{a${}_{n}$}$\mathrm{\}}$中,\textit{a}${}_{1}$\textit{=}1,\textit{h=-}3,则\textit{a}${}_{2\ 016}$\textit{=\underbar{     }.~}

 解析:易知$a_n=\left\{
\begin{array}{l}
1, n为奇数 \\
-4,n为偶数
\end{array}
\right.$,\textit{$\therefore$a}${}_{2\ 016}$\textit{=-}4\textit{.}

 答案:\textit{-}4 \\

知识点:等差数列的概念

难度:1

 已知\textit{a},\textit{b},\textit{c}成等差数列,且它们的和为33,又lg(\textit{a-}1),lg(\textit{b-}5),lg(\textit{c-}6)也构成等差数列,求\textit{a},\textit{b},\textit{c}的值\textit{.}

解析:

 答案:由已知,得$\left\{
\begin{array}{l}
2b=a+c \\
a+b+c=33 \\
2\lg(b-5)=\lg(a-1)+\lg(c-6)
\end{array}
\right.$

$\therefore \left\{
\begin{array}{l}
b=11 \\
a+c=22 \\
(b-5)^2=(a-1)(c-6)
\end{array}
\right.$

解得\textit{a=}4,\textit{b=}11,\textit{c=}18或\textit{a=}13,\textit{b=}11,\textit{c=}9\textit{.} \\

知识点:等差数列的概念

难度:1

 已知无穷等差数列$\mathrm{\{}$\textit{a${}_{n}$}$\mathrm{\}}$,首项\textit{a}${}_{1}$\textit{=}3,公差\textit{d=-}5,依次取出项的序号被4除余3的项组成数列$\mathrm{\{}$\textit{b${}_{n}$}$\mathrm{\}}$\textit{.}

 (1)求\textit{b}${}_{1}$和\textit{b}${}_{2}$;

 (2)求$\mathrm{\{}$\textit{b${}_{n}$}$\mathrm{\}}$的通项公式;

 (3)$\mathrm{\{}$\textit{b${}_{n}$}$\mathrm{\}}$中的第110项是$\mathrm{\{}$\textit{a${}_{n}$}$\mathrm{\}}$的第几项?

解析:

 答案:(1)\textit{$\because$a}${}_{1}$\textit{=}3,\textit{d=-}5,$\therefore a_n = 3+(n-1)(-5) = 8-5n$

\textit{$\because$}数列$\mathrm{\{}$\textit{a${}_{n}$}$\mathrm{\}}$中项的序号被4除余3的项依次是第3项,第7项,第11项,{\dots},

\textit{$\therefore$}$\mathrm{\{}$\textit{b${}_{n}$}$\mathrm{\}}$的首项\textit{b}${}_{1}$\textit{=a}${}_{3}$\textit{=-}7,\textit{b}${}_{2}$\textit{=a}${}_{7}$\textit{=-}27\textit{.}

(2)设$\mathrm{\{}$\textit{a${}_{n}$}$\mathrm{\}}$中的第\textit{m}项是$\mathrm{\{}$\textit{b${}_{n}$}$\mathrm{\}}$的第\textit{n}项,即\textit{b${}_{n}$=a${}_{m}$},

则$m = 3+4(n-1) = 4n-1$,

\textit{$\therefore$b${}_{n}$=a${}_{m}$=a}${}_{4}$\textit{${}_{n-}$}${}_{1}$\textit{=}8\textit{-}5(4\textit{n-}1)\textit{=}13\textit{-}20\textit{n}(\textit{n}$\mathrm{\in}$N\textit{${}_{+}$})\textit{.$\therefore$}$\mathrm{\{}$\textit{b${}_{n}$}$\mathrm{\}}$的通项公式为\textit{b${}_{n}$=}13\textit{-}20\textit{n}(\textit{n}$\mathrm{\in}$N\textit{${}_{+}$})\textit{.}

(3)\textit{b}${}_{110}$\textit{=}13\textit{-}20\textit{$\times$}110\textit{=-}2187,设它是$\mathrm{\{}$\textit{a${}_{n}$}$\mathrm{\}}$中的第\textit{m}项,则8\textit{-}5\textit{m=-}2187,则\textit{m=}439\textit{.} \\

知识点:等差数列的概念

难度:1

	 已知数列$\mathrm{\{}$\textit{a${}_{n}$}$\mathrm{\}}$满足$a_1 = \frac{1}{5}$,且当\textit{n$>$}1,\textit{n}$\mathrm{\in}$N\textit{${}_{+}$}时,有$\frac{a_{n-1}}{a_n} = \frac{2a_{n-1}+1}{1-2a_n}$,设$b_n = \frac{1}{a_n}$,\textit{n}$\mathrm{\in}$N\textit{${}_{+}$.}

 (1)求证:数列$\mathrm{\{}$\textit{b${}_{n}$}$\mathrm{\}}$为等差数列\textit{.}

 (2)试问\textit{a}${}_{1}$\textit{a}${}_{2}$是否是数列$\mathrm{\{}$\textit{a${}_{n}$}$\mathrm{\}}$中的项?如果是,是第几项?如果不是,请说明理由\textit{.}

解析:

	 答案:(1)当\textit{n$>$}1,\textit{n}$\mathrm{\in}$N\textit{${}_{+}$}时,$\frac{a_{n-1}}{a_n} = \frac{2a_{n-1}+1}{1-2a_n} \Leftrightarrow \frac{1-2a_n}{a_n} = \frac{2a_{n-1}+1}{a_{n-1}} \Leftrightarrow \frac{1}{a_n}-2 = 2+\frac{1}{a_{n-1}} \Leftrightarrow \frac{1}{a_n} -\frac{1}{a_{n-1} = 4} \Leftrightarrow b_n - b_{n-1} = 4$,且$b_1=\frac{1}{a_n } = 5$

\textit{$\therefore$}$\mathrm{\{}$\textit{b${}_{n}$}$\mathrm{\}}$是等差数列,且公差为4,首项为5\textit{.}

(2)由\eqref{GrindEQ__1_}知$b_n = b_1 + (n-1)d = 5 + 4(n-1) = 4n+1$

$\therefore a_n = \frac{1}{b_n} = \frac{1}{4n+1}$,\textit{n}$\mathrm{\in}$N\textit{${}_{+}$.}

$\therefore a_1 = \frac{1}{5}, a_2 = \frac{1}{9}$,$\therefore a_1 a_2 = \frac{1}{45}$

令$a_n = \frac{1}{4n+1} = \frac{1}{45}$,$\therefore n = 11$,即$a_1 a_2 = a_{11}$

\textit{$\therefore$a}${}_{1}$\textit{a}${}_{2}$是数列$\mathrm{\{}$\textit{a${}_{n}$}$\mathrm{\}}$中的项,是第11项\textit{.} \\

知识点:等差数列的概念

难度:1

 已知等差数列$\mathrm{\{}$\textit{a${}_{n}$}$\mathrm{\}}$中,$a_7+a_9 = 16$,\textit{a}${}_{4}$\textit{=}1,则\textit{a}${}_{12}$的值是 (\textit{  })

 \textit{                }

 A.15 B.30 C.31 D.64

 解析:\textit{$\because$}$\mathrm{\{}$\textit{a${}_{n}$}$\mathrm{\}}$是等差数列,$\therefore a_7 + a_9 = a_4 + a_{12}$,

$\therefore a_{12} = 16 -1= 15$
  
 答案:A \\

知识点:等差数列的概念

难度:1

 已知$\{a_n\}$为等差数列,$a_1+a_3+a_5 = 105, a_2+a_4+a_6 = 99$,则\textit{a}${}_{20}$等于(\textit{  })

 A.\textit{-}1 B.1 C.3 D.7

 解析:$\because a_1 + a_3 + a_5 = 105$,$\therefore 3a_3=105$,

解得\textit{a}${}_{3}$\textit{=}35,同理由$a_2+a_4+a_6 = 99$,得\textit{a}${}_{4}$\textit{=}33\textit{.}

\textit{$\because$d=a}${}_{4}$\textit{-a}${}_{3}$\textit{=}33\textit{-}35\textit{=-}2,

$\therefore a_{20} = a_4 + (20-4)d = 33+16\times (-2) = 1$

 答案:B \\

知识点:等差数列的概念

难度:1

 若$\mathrm{\{}$\textit{a${}_{n}$}$\mathrm{\}}$是等差数列,则下列数列中仍为等差数列的有 (\textit{  })

 \textit{①}$\{a_n+3\}$\textit{② }$\{a_n^2\}$\textit{ ③}$\mathrm{\{}$\textit{a${}_{n+}$}${}_{1}$\textit{-a${}_{n}$}$\mathrm{\}}$\textit{ ④}$\mathrm{\{}$2\textit{a${}_{n}$}$\mathrm{\}}$\textit{ ⑤} $\{2a_n+n\}$

 A.1个 B.2个 C.3个 D.4个

 解析:根据等差数列的定义判断,若$\mathrm{\{}$\textit{a${}_{n}$}$\mathrm{\}}$是等差数列,则$\{a_n+3\},\{a_{n+1}-a_n\},\{2a_n\},\{2a_n+n\}$均为等差数列,而$\{a_n^2\}$不一定是等差数列\textit{.}

 答案:D \\

知识点:等差数列的概念

难度:1

 已知等差数列$\mathrm{\{}$\textit{a${}_{n}$}$\mathrm{\}}$满足$a_1+a_2+a_3+\cdots+a_{101} = 0$,则有 (\textit{  })

 A.$a_1 + a_{101} > 0$ B.$a_2 + a_{101} < 0$

 C.$a_3 +a_{100} \le 0$ D.\textit{a}${}_{51}$\textit{=}0

 解析:由题设$a_1+a_2+a_3+\cdots+a_{101} = 101a_{51} = 0$,得\textit{a}${}_{51}$\textit{=}0\textit{.}

 答案:D \\

知识点:等差数列的概念

难度:1

 若等差数列的前三项依次是$x-1, x+1, 2x+3$,则其通项公式为(\textit{  })

 A.\textit{a${}_{n}$=}2\textit{n-}5 B.\textit{a${}_{n}$=}2\textit{n-}3 C.\textit{a${}_{n}$=}2\textit{n-}1 D.$a_n = 2n+1$

 解析:$\because x-1, x+1, 2x+3$是等差数列的前三项,

$\therefore 2(x+1)=x-1+2x+3$,解得\textit{x=}0\textit{.}

\textit{$\therefore$a}${}_{1}$\textit{=x-}1\textit{=-}1,\textit{a}${}_{2}$\textit{=}1,\textit{a}${}_{3}$\textit{=}3,\textit{$\therefore$d=}2\textit{.}

$\therefore a_n = -1+2(n-1) = 2n-3$,故选B\textit{.}

 答案:B \\

知识点:等差数列的概念

难度:1

 在等差数列$\mathrm{\{}$\textit{a${}_{n}$}$\mathrm{\}}$中,$a_1+a_4+a_7=39,a_2+a_5+a_8=33$,则$a_3+a_6+a_9$\textit{=\underbar{    }.~}

 解析:由等差数列的性质,

得$(a_1+a_4+a_7)+(a_3+a_6+a_9) = 2(a_2+a_5+a_8)$,

即$39+(a_3+a_6+a_9) = 2\times 33$,

故$a_3+a_6+a_9 = 66-39 = 27$

 答案:27 \\

知识点:等差数列的概念

难度:1

 若lg 2,lg(2\textit{${}^{x}$-}1),lg(2\textit{${}^{x}$+}3)成等差数列,则\textit{x}的值是\textit{\underbar{      }.~}

 解析:由题意,知2lg(2\textit{${}^{x}$-}1)\textit{=}lg 2\textit{+}lg(2\textit{${}^{x}$+}3),

则(2\textit{${}^{x}$-}1)${}^{2}$\textit{=}2(2\textit{${}^{x}$+}3),即(2\textit{${}^{x}$})${}^{2}$\textit{-}4·2\textit{${}^{x}$-}5\textit{=}0,

\textit{$\therefore$}(2\textit{${}^{x}$-}5)(2\textit{${}^{x}$+}1)\textit{=}0,\textit{$\therefore$}2\textit{${}^{x}$=}5,\textit{$\therefore$x=}log${}_{2}$5\textit{.}

 答案:log${}_{2}$5 \\

知识点:等差数列的概念

难度:1

 已知一个等差数列由三个数构成,这三个数之和为9,平方和为35,则这三个数构成的等差数列为\textit{\underbar{              }.~}

 答案:1,3,5或5,3,1 \\

知识点:等差数列的概念

难度:1

 在等差数列$\{a_n\}$中,$a_1+a_4+a_7=15,a_2a_4a_6=45$,求数列$\{a_n\}$的通项公式.

解析:

 答案:$a_1+a_7=2a_4=a_2+a_6$,

$\therefore a_1+a_4+a_7=3a_4=15$, $\therefore a_4=5$

$\therefore a_2+a_6=10, a_2a_6=9$

\textit{$\therefore$a}${}_{2}$,\textit{a}${}_{6}$是方程$x^2-10x+9 = 0$的两根\textit{.}

$\therefore \left\{
\begin{array}{l}
a_2=1 \\
a_6=9
\end{array}
\right.$ 或$\left\{
\begin{array}{l}
a_2=9 \\
a_6=1
\end{array}
\right.$

若\textit{a}${}_{2}$\textit{=}1,\textit{a}${}_{6}$\textit{=}9,则$d = \frac{a_6 - a_2}{6-2} = 2$,\textit{$\therefore$a${}_{n}$=}2\textit{n-}3\textit{.}

若\textit{a}${}_{2}$\textit{=}9,\textit{a}${}_{6}$\textit{=}1,则$d = \frac{a_6 -a_2}{6-2} = -2$,\textit{$\therefore$a${}_{n}$=}13\textit{-}2\textit{n.}

\textit{$\therefore$}数列$\mathrm{\{}$\textit{a${}_{n}$}$\mathrm{\}}$的通项公式为\textit{a${}_{n}$=}2\textit{n-}3或\textit{a${}_{n}$=}13\textit{-}2\textit{n.} \\

知识点:等差数列的概念

难度:1

 已知\textit{f}(\textit{x})\textit{=x}${}^{2}$\textit{-}2\textit{x-}3,等差数列$\mathrm{\{}$\textit{a${}_{n}$}$\mathrm{\}}$中,\textit{a}${}_{1}$\textit{=f}(\textit{x-}1),$a_2=-\frac{3}{2}$,\textit{a}${}_{3}$\textit{=f}(\textit{x}),求:

 (1)\textit{x}的值;

 (2)通项\textit{a${}_{n}$.}

解析:

 答案:(1)由\textit{f}(\textit{x})\textit{=x}${}^{2}$\textit{-}2\textit{x-}3,得\textit{a}${}_{1}$\textit{=f}(\textit{x-}1)\textit{=}(\textit{x-}1)${}^{2}$\textit{-}2(\textit{x-}1)\textit{-}3\textit{=x}${}^{2}$\textit{-}4\textit{x},\textit{a}${}_{3}$\textit{=x}${}^{2}$\textit{-}2\textit{x-}3,

又因为$\mathrm{\{}$\textit{a${}_{n}$}$\mathrm{\}}$为等差数列,所以$2a_2=a_1+a_3$,即$-3=x^2-4x+x^2-2x-3$,解得\textit{x=}0或\textit{x=}3\textit{.}

(2)当\textit{x=}0时,$a_1=0,d=a_2-a_1=-\frac{3}{2}$,

此时$a_n = a_1+(n-1)d=-\frac{3}{2}(n-1)$;

当\textit{x=}3时,$a_1=-3,d=a_2-a_1=\frac{3}{2}$,

此时$a_n = a_1+(n-1)d=\frac{3}{2}(n-3)$ \\

知识点:等差数列的概念

难度:2

 在数列$\mathrm{\{}$\textit{a${}_{n}$}$\mathrm{\}}$中,若\textit{a}${}_{2}$\textit{=}2,\textit{a}${}_{6}$\textit{=}0,且数列$\{\frac{1}{a_n+1}\}$是等差数列,则\textit{a}${}_{4}$等于(\textit{  })

 A.$\frac{1}{2}$ B.$\frac{1}{3}$ C.$\frac{1}{4}$ D.$\frac{1}{6}$

 解析:令$b_n = \frac{1}{a_n+1}$,则$b_2 = \frac{1}{a_2+1}=\frac{1}{3},b_6=\frac{1}{a_6+1}+1$

由题意知$\mathrm{\{}$\textit{b${}_{n}$}$\mathrm{\}}$是等差数列,

$\therefore b_6 - b_2 = (6-2)d = 4d = \frac{2}{3}, \therefore d = \frac{1}{6}$

$\therefore b_4=b_2+2d=\frac{1}{3}+2\times \frac{1}{6} = \frac{2}{3}$
 
$\because b_4 = \frac{1}{a_4+1}, \therefore a_4 = \frac{1}{2}$

 答案:A \\

知识点:等差数列的概念

难度:2

 已知数列$\mathrm{\{}$\textit{a${}_{n}$}$\mathrm{\}}$为等差数列,且$a_1+a_7+a_{13}=4\pi$,则$\tan(a_2+a_{12})$的值为(\textit{  })

 A.$\sqrt{3}$ B.$\pm \sqrt{3}$ C.$-\frac{\sqrt{3}}{3}$ D.$-\sqrt{3}$

 解析:\textit{$\because$}$\mathrm{\{}$\textit{a${}_{n}$}$\mathrm{\}}$为等差数列,$\therefore a_1+a_7+a_{13} = 3a_7 = 4\pi$

$\therefore a_7 = \frac{4\pi}{3}, \tan(a_2+a_{12}) = \tan(2a_7) = \tan\frac{8\pi}{3}=-\sqrt{3}$

 答案:D \\

知识点:等差数列的概念

难度:2

 《九章算术》``竹九节''问题:现有一根9节的竹子,自上而下各节的容积成等差数列,上面4节的容积共3升,下面3节的容积共4升,则第5节的容积为(\textit{  })

 A.1升 B.$\frac{67}{66}$升 C.$\frac{47}{44}$升 D.$\frac{37}{38}$升

 解析:设所构成的等差数列$\mathrm{\{}$\textit{a${}_{n}$}$\mathrm{\}}$的首项为\textit{a}${}_{1}$,公差为\textit{d},

由题意得$\left\{
\begin{array}{l}
a_1+a_2+a_3+a_4=3 \\
a_7+a_8+a_9=4
\end{array}
\right.$ 即$\left\{
\begin{array}{l}
4a_1+6d=3, \\
3a_1+21d=4.
\end{array}
\right.$

解得$\left\{
\begin{array}{l}
a_1=\frac{13}{22} \\
d=\frac{7}{66}
\end{array}
\right.$,所以$a_5 = a_1+4d = \frac{67}{66}$

 答案:B \\

知识点:等差数列的概念

难度:2

 在等差数列$\mathrm{\{}$\textit{a${}_{n}$}$\mathrm{\}}$中,如果$a_2+a_5+a_8=9$,那么关于\textit{x}的方程$x^2+(a_4+a_6)x+10=0$(\textit{  })

 A.无实根 B.有两个相等实根

 C.有两个不等实根 D.不能确定有无实根

 解析:$\because a_4+a_6=a_2+a_8=2a_5$,即$3a_5=9, \therefore a_5 = 3$

又$a_4+a_6 = 2a_5 = 6$,

$\therefore$关于x的方程为$x^2+6x+10=0$,则判别式$\Delta =6^2-4\times 10 < 0,\therefore$无实数解.

 答案:A \\

知识点:等差数列的概念

难度:2

 已知log\textit{${}_{a}$b},\textit{-}1,log\textit{${}_{b}$a}成等差数列,且\textit{a},\textit{b}为关于\textit{x}的方程\textit{x}${}^{2}$\textit{-cx+d=}0的两根,则\textit{d=\underbar{     }.~}

 解析:由已知,得log\textit{${}_{a}$b+}log\textit{${}_{b}$a=-}2,即$\frac{\lg b }{\lg a}+\frac{\lg a}{\lg b} = -2$,从而有$(\lg a+\lg b)^2 = 0$,可得$\lg a = -\lg b = \lg \frac{1}{b}$,即$ab = 1$

故由根与系数的关系得$d = ab = 1$

 答案:1 \\

知识点:等差数列的概念

难度:2

 已知方程$(x^2-2x+m)(x^2-2x+n)=0$的四个根组成一个首项为$\frac{1}{4}$的等差数列,则$|m-n|$=\underbar{     }

 解析:由题意设这4个根为$\frac{1}{4},\frac{1}{4}+d, \frac{1}{4}+2d, \frac{1}{4}+3d$

可得$\frac{1}{4}+(\frac{1}{4}+3d) = 2, \therefore d = \frac{1}{2}$

\textit{$\therefore$}这4个根依次为$\frac{1}{4},\frac{3}{4},\frac{5}{4},\frac{7}{4}$

$\therefore n = \frac{1}{4} \times \frac{7}{4}=\frac{7}{16}, m = \frac{3}{4}\times \frac{5}{4}=\frac{15}{16}$或$n = \frac{15}{16}, m= \frac{7}{16}, \therefore |m-n| = \frac{1}{2}$

 答案:$\frac{1}{2}$ \\

知识点:等差数列的概念

难度:2

 两个等差数列5,8,11,{\dots}和3,7,11,{\dots}都有100项,那么它们共有多少相同的项?

解析:

 答案:在数列$\mathrm{\{}$\textit{a${}_{n}$}$\mathrm{\}}$中,\textit{a}${}_{1}$\textit{=}5,公差\textit{d}${}_{1}$\textit{=}8\textit{-}5\textit{=}3\textit{.}

$\therefore a_n = a_1+(n-1)d_1=3n+2$

在数列$\mathrm{\{}$\textit{b${}_{n}$}$\mathrm{\}}$中,\textit{b}${}_{1}$\textit{=}3,公差\textit{d}${}_{2}$\textit{=}7\textit{-}3\textit{=}4,

$\therefore b_n = b_1+(n-1)d=4n-1$

令\textit{a${}_{n}$=b${}_{m}$},则$3n+2=4m-1,\therefore n=\frac{4m}{3}-1$

\textit{$\because$m},\textit{n}$\mathrm{\in}$N\textit{${}_{+}$},\textit{$\therefore$m=}3\textit{k}(\textit{k}$\mathrm{\in}$N\textit{${}_{+}$}),

又$\left\{
\begin{array}{l}
0<m\le 100, \\
0<n\le 100
\end{array}
\right.$,解得0\textit{$<$m}$\mathrm{\le}$75\textit{.}

\textit{$\therefore$}0\textit{$<$}3\textit{k}$\mathrm{\le}$75,\textit{$\therefore$}0\textit{$<$k}$\mathrm{\le}$25,\textit{$\therefore$k=}1,2,3,{\dots},25\textit{.}

\textit{$\therefore$}两个数列共有25个公共项\textit{.} \\

知识点:等差数列的概念

难度:2

 已知数列$\mathrm{\{}$\textit{a${}_{n}$}$\mathrm{\}}$中,$a_1 = \frac{3}{5},a_na_{n-1}+1=2a_{n-1}$(\textit{n}$\mathrm{\ge}$2,\textit{n}$\mathrm{\in}$N\textit{${}_{+}$})\textit{.}数列$\mathrm{\{}$\textit{b${}_{n}$}$\mathrm{\}}$中,$b_n=\frac{1}{a_{n}-1}$(\textit{n}$\mathrm{\in}$N\textit{${}_{+}$})\textit{.}

 (1)求证:$\mathrm{\{}$\textit{b${}_{n}$}$\mathrm{\}}$是等差数列;

 (2)求数列$\mathrm{\{}$\textit{a${}_{n}$}$\mathrm{\}}$的通项公式,并求其最大、最小项\textit{.}

解析:

 答案:(1)由$a_na_{n-1}+1=2a_{n-1}$,得\textit{a${}_{n}$a${}_{n-}$}${}_{1}$\textit{-a${}_{n-}$}${}_{1}$\textit{=a${}_{n-}$}${}_{1}$\textit{-}1,

$\therefore \frac{1}{a_n-1}=\frac{a_{n-1}}{a_{n-1}-1}=b_n$,又$b_{n-1}=\frac{1}{a_{n-1}-1}$

$\therefore b_n-b_{n-1}=\frac{a_{n-1}}{a_{n-1}-1}-\frac{1}{a_{n-1}-1}=1(n\ge 2,n\in N_{+})$

$\because b_1=\frac{1}{a_1-1}=-\frac{5}{2}$

\textit{$\therefore$}数列$\mathrm{\{}$\textit{b${}_{n}$}$\mathrm{\}}$是以$-\frac{5}{2}$为首项,1为公差的等差数列\textit{.}

 (2)由(1)知\textit{b${}_{n}$=n-}3\textit{.}5,

又由$b_n=\frac{1}{a_n-1}$得$a_n=1+\frac{1}{b_n}=1+\frac{1}{n-3.5}$

点(\textit{n},\textit{a${}_{n}$})在函数$y=\frac{1}{x-3.5}+1$的图像上\textit{.}

显然,在区间(3\textit{.}5,\textit{+$\infty$})上,$y=\frac{1}{x-3.5}+1$递减且\textit{y$>$}1;在区间(0,3\textit{.}5)上,$y=\frac{1}{x-3.5}+1$递减且\textit{y$<$}1\textit{.}

因此,当\textit{n=}4时,\textit{a${}_{n}$}取得最大值3;当\textit{n=}3时,\textit{a${}_{n}$}取得最小值\textit{-}1\textit{.} \\
 
知识点:等差数列的前n项和

难度:1

 设\textit{S${}_{n}$}是等差数列$\mathrm{\{}$\textit{a${}_{n}$}$\mathrm{\}}$的前\textit{n}项和,已知\textit{a}${}_{2}$\textit{=}3,\textit{a}${}_{6}$\textit{=}11,则\textit{S}${}_{7}$等于(\textit{  })

 \textit{                }

 A.13 B.35 C.49 D.63

 解析:$S_7=\frac{7(a_1+a_7)}{2}=\frac{7(a_2+a_6)}{2}=\frac{7\times (3+11)}{2}=49$
 
 答案:C \\

知识点:等差数列的前n项和

难度:1

 设\textit{S${}_{n}$}是等差数列$\mathrm{\{}$\textit{a${}_{n}$}$\mathrm{\}}$的前\textit{n}项和,\textit{S}${}_{5}$\textit{=}10,则\textit{a}${}_{3}$的值为 (\textit{  })

 A.$\frac{6}{5}$ B.1 C.2 D.3

 解析:$\because S_5=\frac{5(a_1+a_5)}{2}=5a_3$,

$\therefore a_3=\frac{1}{5}S_5=\frac{1}{5}\times 10 = 2$

 答案:C \\

知识点:等差数列的前n项和

难度:1

 已知数列$\mathrm{\{}$\textit{a${}_{n}$}$\mathrm{\}}$的通项公式为\textit{a${}_{n}$=}2\textit{n-}37,则\textit{S${}_{n}$}取最小值时\textit{n}的值为(\textit{  })

 A.17 B.18 C.19 D.20

 解析:由$\left\{
\begin{array}{l}
a_n\le 0 \\
a_{n+1}\ge 0
\end{array}
\right.$得$\left\{
\begin{array}{l}
2n-37 \le 0, \\
2(n+1)-37\ge 0
\end{array}
\right.$,$\therefore \frac{35}{2}\le n\le \frac{37}{2}$

\textit{$\because$n}$\mathrm{\in}$N\textit{${}_{+}$},\textit{$\therefore$n=}18\textit{.$\therefore$S}${}_{18}$最小,此时\textit{n=}18\textit{.}

 答案:B \\

知识点:等差数列的前n项和

难度:1

 等差数列$\mathrm{\{}$\textit{a${}_{n}$}$\mathrm{\}}$的前\textit{n}项和为\textit{S${}_{n}$}(\textit{n=}1,2,3,{\dots}),若当首项\textit{a}${}_{1}$和公差\textit{d}变化时,$a_5+a_8+a_{11}$是一个定值,则下列选项中为定值的是(\textit{  })

 A.\textit{S}${}_{17}$ B.\textit{S}${}_{18}$ C.\textit{S}${}_{15}$ D.\textit{S}${}_{14}$

 解析:由$a_5+a_8+a_{11}=3a_8$是定值,可知\textit{a}${}_{8}$是定值,所以$S_{15}=\frac{15(a_1+a_{15})}{2}=15a_8$是定值\textit{.}

 答案:C \\

知识点:等差数列的前n项和

难度:1

 若两个等差数列$\mathrm{\{}$\textit{a${}_{n}$}$\mathrm{\}}$,$\mathrm{\{}$\textit{b${}_{n}$}$\mathrm{\}}$的前\textit{n}项和分别为\textit{A${}_{n}$}与\textit{B${}_{n}$},且满足$\frac{A_n}{B_n}=\frac{7n+1}{4n+27}$(\textit{n}$\mathrm{\in}$N\textit{${}_{+}$}),则$\frac{a_{11}}{b_{11}}$的值是(\textit{  })

 A.$\frac{7}{4}$ B.$\frac{3}{2}$ C.$\frac{4}{3}$ D.$\frac{78}{71}$

 解析:$\because \frac{a_n}{b_n}=\frac{A_{2n-1}}{B_{2n-1}}=\frac{7(2n-1)+1}{4(2n-1)+27}=\frac{14n-6}{8n+28}$,

$\therefore \frac{a_{11}}{b_{11}}=\frac{14\times 11-6}{8\times 11+23}=\frac{148}{111}=\frac{4}{3}$

 答案:C \\

知识点:等差数列的前n项和

难度:1

 已知$\mathrm{\{}$\textit{a${}_{n}$}$\mathrm{\}}$是等差数列,\textit{S${}_{n}$}为其前\textit{n}项和,\textit{n}$\mathrm{\in}$N\textit{${}_{+}$.}若\textit{a}${}_{3}$\textit{=}16,\textit{S}${}_{20}$\textit{=}20,则\textit{S}${}_{10}$的值为\textit{\underbar{     }.~}

 解析:设等差数列$\mathrm{\{}$\textit{a${}_{n}$}$\mathrm{\}}$的首项为\textit{a}${}_{1}$,公差为\textit{d.}

$\because a_3=a_1+2d,S_{20}=20a_1+\frac{20\times 19}{2}d=20$,

$\therefore \left\{
\begin{array}{l}
a_1+2d=16, \\
2a_1+19d=2
\end{array}
\right.$

解得\textit{d=-}2,\textit{a}${}_{1}$\textit{=}20,

$\therefore S_{10}=10a_1+\frac{10\times 9}{2}d=200-90=110$

 答案:110 \\

知识点:等差数列的前n项和

难度:1

 在等差数列$\mathrm{\{}$\textit{a${}_{n}$}$\mathrm{\}}$中,前\textit{n}项和为\textit{S${}_{n}$},若\textit{a}${}_{9}$\textit{=}3\textit{a}${}_{5}$,则$\frac{S_{17}}{S_9}$\underbar{     }.

 解析:$\because S_{17}=17a_9,S_9=9a_5$,

$\therefore \frac{S_{17}}{S_9}=\frac{17a_9}{9a_5}=\frac{17}{9}\times 3=\frac{17}{3}$

 答案:$\frac{17}{3}$ \\

知识点:等差数列的前n项和

难度:1

 已知某等差数列共有10项,其奇数项之和为15,偶数项之和为30,则其公差等于\textit{\underbar{     }.~}

 解析:设公差为\textit{d},则有5\textit{d=S}${}_{\textrm{偶}}$\textit{-S}${}_{\textrm{奇}}$\textit{=}30\textit{-}15\textit{=}15,于是\textit{d=}3\textit{.}

 答案:3 \\

知识点:等差数列的前n项和

难度:1

 若等差数列$\mathrm{\{}$\textit{a${}_{n}$}$\mathrm{\}}$的公差\textit{d$<$}0,且$a_2\cdot a_4=12,a_2+a_4=8$

 (1)求数列$\mathrm{\{}$\textit{a${}_{n}$}$\mathrm{\}}$的首项\textit{a}${}_{1}$和公差\textit{d};

 (2)求数列$\mathrm{\{}$\textit{a${}_{n}$}$\mathrm{\}}$的前10项和\textit{S}${}_{10}$的值\textit{.}

解析:

 答案:(1)由题意知$(a_1+d)(a_1+3d)=12,(a_1+d)+(a_1+3d)=8$,且\textit{d$<$}0,解得\textit{a}${}_{1}$\textit{=}8,\textit{d=-}2\textit{.}

(2)$S_{10}=10\times a_1+\frac{10\times 9}{2}d=-10$ \\

知识点:等差数列的前n项和

难度:1

 已知数列$\mathrm{\{}$\textit{a${}_{n}$}$\mathrm{\}}$是首项为23,公差为整数的等差数列,且前6项均为正,从第7项开始变为负\textit{.}

 求:(1)此等差数列的公差\textit{d};

 (2)设前\textit{n}项和为\textit{S${}_{n}$},求\textit{S${}_{n}$}的最大值;

 (3)当\textit{S${}_{n}$}是正数时,求\textit{n}的最大值\textit{.}

解析:

 答案:(1)\textit{$\because$}数列$\mathrm{\{}$\textit{a${}_{n}$}$\mathrm{\}}$首项为23,前6项均为正,从第7项开始变为负,

$\therefore a_6=a_1+5d=23+5d>0,a_7=a_1+6d=23+6d<0$,解得$-\frac{23}{5}<d<-\frac{23}{6}$,又\textit{d}$\mathrm{\in}$Z,\textit{$\therefore$d=-}4\textit{.}

(2)\textit{$\because$d$<$}0,\textit{$\therefore$}$\mathrm{\{}$\textit{a${}_{n}$}$\mathrm{\}}$是递减数列\textit{.}

又\textit{a}${}_{6}$\textit{$>$}0,\textit{a}${}_{7}$\textit{$<$}0,\textit{$\therefore$}当\textit{n=}6时,\textit{S${}_{n}$}取得最大值,

即$S_6=6\times 23+\frac{6\times 5}{2}\times (-4)=78$

(3)$S_n=23n+\frac{n(n-1)}{2}\times (-4)>0$,整理得$n(25-2n)>0,\therefore0<n<\frac{25}{2}$,又\textit{n}$\mathrm{\in}$N\textit{${}_{+}$},\textit{$\therefore$n}的最大值为12\textit{.} \\

知识点:等差数列的前n项和

难度:2

 设数列$\mathrm{\{}$\textit{a${}_{n}$}$\mathrm{\}}$为等差数列,公差\textit{d=-}2,\textit{S${}_{n}$}为其前\textit{n}项和,若\textit{S}${}_{10}$\textit{=S}${}_{11}$,则\textit{a}${}_{1}$\textit{=}(\textit{  })

 A.18 B.20 C.22 D.24

 解析:因为$S_{11}-S_{10}=a_{11}=0,a_{11}=a_1+10d=a_1+10\times (-2)=0$,所以\textit{a}${}_{1}$\textit{=}20\textit{.}

 答案:B \\

知识点:等差数列的前n项和

难度:2

 (2017全国1高考)记\textit{S${}_{n}$}为等差数列$\mathrm{\{}$\textit{a${}_{n}$}$\mathrm{\}}$的前\textit{n}项和\textit{.}若$a_4+a_5=24,S_6=48$,则$\mathrm{\{}$\textit{a${}_{n}$}$\mathrm{\}}$的公差为(\textit{  })

 A\textit{.}1 B\textit{.}2 C\textit{.}4 D\textit{.}8

 解析:设首项为\textit{a}${}_{1}$,公差为\textit{d},则$a_4+a_5=a_1+3d+a_1+4d=24,S_6=6a_1+\frac{6\times 5}{2}d=48$,联立可得$\left\{
\begin{array}{l}
2a_1+7d=24, ① \\
6a_1+15d=48, ②
\end{array}
\right.$,\textit{①$\times$}3\textit{-②},得(21\textit{-}15)\textit{d=}24,即6\textit{d=}24,所以\textit{d=}4\textit{.}
 
 答案:C \\

知识点:等差数列的前n项和

难度:2

 等差数列$\mathrm{\{}$\textit{a${}_{n}$}$\mathrm{\}}$的前\textit{n}项和记为\textit{S${}_{n}$},若$a_2+a_4+a_{15}$的值为一个确定的常数,则下列各数中也是常数的是(\textit{  })

 A.\textit{S}${}_{7}$ B.\textit{S}${}_{8}$ C.\textit{S}${}_{13}$ D.\textit{S}${}_{15}$

 解析:$\because a_2+a_4+a_{15}=3a_1+18d=3(a_1+6d)=3a_7$为常数,

$\therefore S_{13}=\frac{13(a_1+a_{13})}{2}=13a_7$为常数\textit{.}

 答案:C \\

知识点:等差数列的前n项和

难度:2

 若等差数列$\mathrm{\{}$\textit{a${}_{n}$}$\mathrm{\}}$的通项公式是\textit{a${}_{n}$=}1\textit{-}2\textit{n},其前\textit{n}项和为\textit{S${}_{n}$},则数列$\{\frac{S_n}{n}\}$的前11项和为 (\textit{  })

 A.\textit{-}45 B.\textit{-}50 C.\textit{-}55 D.\textit{-}66

 解析:$\because S_n=\frac{(a_1+a_n)n}{2},\therefore \frac{S_n}{n}=\frac{a_1+a_n}{2}=-n$,

$\therefore \{\frac{S_n}{n}\}$的前11项和为$-(1+2+3+\cdots +11)=-66$.故选D\textit{.}

 答案:D \\

知识点:等差数列的前n项和

难度:2

 已知等差数列$\mathrm{\{}$\textit{a${}_{n}$}$\mathrm{\}}$前9项的和等于前4项的和\textit{.}若$a_1=1,a_k+a_4=0$,则\textit{k=\underbar{     }.~}

 解析:设等差数列$\mathrm{\{}$\textit{a${}_{n}$}$\mathrm{\}}$的公差为\textit{d},则$a_n=1+(n-1)d$,

$\because S_4=S_9, \therefore a_5+a_6+a_7+a_8+a_9=0$

$\therefore a_7=0, \therefore 1+6d=0,d=-\frac{1}{6}$

又$a_4=1+3\times (-\frac{1}{6})=\frac{1}{2},a_k=1+(k-1)d$,

由$a_k+a_4=0$,得$\frac{1}{2}+1+(k-1)d=0$,将$d=\frac{1}{6}$代入,可得\textit{k=}10\textit{.}

 答案:10 \\

知识点:等差数列的前n项和

难度:2

 已知数列$\mathrm{\{}$\textit{a${}_{n}$}$\mathrm{\}}$为等差数列,其前\textit{n}项和为\textit{S${}_{n}$},且$1+\frac{a_{11}}{a_{10}}<0$若\textit{S${}_{n}$}存在最大值,则满足\textit{S${}_{n}$$>$}0的\textit{n}的最大值为\textit{\underbar{     }.~}

 解析:因为\textit{S${}_{n}$}有最大值,所以数列$\mathrm{\{}$\textit{a${}_{n}$}$\mathrm{\}}$单调递减,又$\frac{a_{11}}{a_{10}}<-1$,所以\textit{a}${}_{10}$\textit{$>$}0,\textit{a}${}_{11}$\textit{$<$}0,且$a_{10}+a_{11}<0$

所以$S_{19}=19\times \frac{a_1+a_{19}}{2}=19a_{10}>0,S_{20}=20\times \frac{a_1+a_{20}}{2}=10\times(a_{10}+a_{11})<0$,

故满足\textit{S${}_{n}$$>$}0的\textit{n}的最大值为19\textit{.}

 答案:19 \\

知识点:等差数列的前n项和

难度:2

 在等差数列$\mathrm{\{}$\textit{a${}_{n}$}$\mathrm{\}}$中,\textit{a}${}_{1}$\textit{=-}60,\textit{a}${}_{17}$\textit{=-}12,求数列$\{|a_n|\}$的前\textit{n}项和\textit{.}

解析:

  答案:数列$\mathrm{\{}$\textit{a${}_{n}$}$\mathrm{\}}$的公差$d=\frac{a_{17}-a_1}{17-1}=\frac{-12-(-60)}{17-1}=3$,

$\therefore a_n=a_1+(n-1)d=-60+(n-1)\times 3=3n-63$

由\textit{a${}_{n}$$<$}0得3\textit{n-}63\textit{$<$}0,

解得\textit{n$<$}21\textit{.}

\textit{$\therefore$}数列$\mathrm{\{}$\textit{a${}_{n}$}$\mathrm{\}}$的前20项是负数,第20项以后的项都为非负数\textit{.}

设$S_n,S_n^{'}$分别表示数列$\{a_n\},\{|a_n|\}$的前\textit{n}项和,

当$n\le 20$时,$S_n^{'}=-S_n=-[-60n+\frac{n(n-1)}{2}\times 3]=-\frac{3}{2}n^2+\frac{123}{2}n$;

当\textit{n$>$}20时,$S_n^{'}=-S_{20}+(S_n-S_{20})=S_n-2S_{20}=-60n+\frac{n(n-1)}{2}\times 3-2\times (-60\times 20+\frac{20\times 19}{2}\times 3)=\frac{3}{2}n^2-\frac{123}{2}n+1260$

$\therefore$数列$\{|a_n|\}$的前\textit{n}项和

$S_n^{'}=\left\{
\begin{array}{l}
-\frac{3}{2}n^2+\frac{123}{2}n(n\le 20), \\
\frac{3}{2}n^2-\frac{123}{2}n+1260(n>20).
\end{array}
\right.$ \\

知识点:等差数列的前n项和

难度:2

 设等差数列$\mathrm{\{}$\textit{a${}_{n}$}$\mathrm{\}}$的前\textit{n}项和为\textit{S${}_{n}$},且\textit{a}${}_{5}$\textit{+a}${}_{13}$\textit{=}34,\textit{S}${}_{3}$\textit{=}9\textit{.}

 (1)求数列$\mathrm{\{}$\textit{a${}_{n}$}$\mathrm{\}}$的通项公式及前\textit{n}项和公式;

 (2)设数列$\mathrm{\{}$\textit{b${}_{n}$}$\mathrm{\}}$的通项公式为$b_n=\frac{a_n}{a_n+t}$,问:是否存在正整数\textit{t},使得\textit{b}${}_{1}$,\textit{b}${}_{2}$,\textit{b${}_{m}$}(\textit{m}$\mathrm{\ge}$3,\textit{m}$\mathrm{\in}$N)成等差数列?若存在,求出\textit{t}和\textit{m}的值;若不存在,请说明理由\textit{.}

解析:

 答案:(1)设等差数列$\mathrm{\{}$\textit{a${}_{n}$}$\mathrm{\}}$的公差为\textit{d},

因为\textit{a}${}_{5}$\textit{+a}${}_{13}$\textit{=}34,\textit{S}${}_{3}$\textit{=}9,

所以$\left\{
\begin{array}{l}
a_1+4d+a_1+12d=34, \\
a_1+a_1+d+a_1+2d=9
\end{array}
\right.$

整理得$\left\{
\begin{array}{l}
a_1+8d=17, \\
a_1+d=3
\end{array}
\right.$解得$\left\{
\begin{array}{l}
a_1=1, \\
d=2
\end{array}
\right.$

所以$a_n=1+(n-1)\times 2=2n-1$,

$S_n=n\times 1+\frac{n(n-1)}{2}\times 2=n^2$

(2)由\eqref{GrindEQ__1_}知$b_n=\frac{2n-1}{2n-1+t}$,

所以$b_1=\frac{1}{1+t},b_2=\frac{3}{3+t},b_m\frac{2m-1}{2m-1+t}$

若\textit{b}${}_{1}$,\textit{b}${}_{2}$,\textit{b${}_{m}$}(\textit{m}$\mathrm{\ge}$3,\textit{m}$\mathrm{\in}$N)成等差数列,

则$2b_2=b_1+b_m$,

所以$\frac{6}{3+t}=\frac{1}{1+t}+\frac{2m-1}{2m-1+t}$,

即$6(1+t)(2m-1+t)=(3+t)(2m-1+t)+(2m-1)(1+t)(3+t)$,

整理得$(m-3)t^2-(m+1)t=0$,

因为\textit{t}是正整数,所以$(m-3)t-(m+1)=0,m=3$时显然不成立,所以$t=\frac{m+1}{m-3}=\frac{m-3+4}{m-3}=1+\frac{4}{m-3}$

又因为\textit{m}$\mathrm{\ge}$3,\textit{m}$\mathrm{\in}$N,

所以\textit{m=}4或5或7,

当\textit{m=}4时,\textit{t=}5;

当\textit{m=}5时,\textit{t=}3;

当\textit{m=}7时,\textit{t=}2\textit{.}

所以存在正整数\textit{t},使得\textit{b}${}_{1}$,\textit{b}${}_{2}$,\textit{b${}_{m}$}(\textit{m}$\mathrm{\ge}$3,\textit{m}$\mathrm{\in}$N)成等差数列\textit{.} \\

知识点:数列的概念

难度:1

 已知数列$\{a_n\}$的前$n$项和$S_n=\frac{1}{n}$的值等于(\textit{  })

 \textit{                }

 A.$\frac{1}{20}$ B.$-\frac{1}{20}$ C.$\frac{1}{30}$ D.$-\frac{1}{30}$

 解析:$a_5=S_5-S_4=\frac{1}{5}-\frac{1}{4}=-\frac{1}{20}$

 答案:B \\

知识点:数列的概念

难度:1

 已知等差数列$\mathrm{\{}$\textit{a${}_{n}$}$\mathrm{\}}$的前\textit{n}项和为\textit{S${}_{n}$},\textit{a}${}_{5}$\textit{=}5,\textit{S}${}_{5}$\textit{=}15,则数列$\{\frac{1}{a_na_{n+1}}\}$的前100项和为(\textit{  })

 A.$\frac{100}{101}$ B.$\frac{99}{101}$ C.$\frac{99}{100}$ D.$\frac{101}{100}$

 解析:$\because S_5=\frac{5(a_1+a_5)}{2}=\frac{5(a_1+5}{2}=15, \therefore a_1=1$,

$\therefore d=\frac{a_5-a_1}{5-1}=\frac{5-1}{5-1}=1$,

$\therefore a_n=1+(n-1)\times 1=n$,

$\therefore \frac{1}{a_na_{n+1}}=\frac{1}{n(n+1)}$

设$\{\frac{1}{a_na_{n+1}}\}$的前\textit{n}项和为\textit{T${}_{n}$},

则$T_{100}=\frac{1}{1\times 2}+\frac{1}{2\times 3}+\cdots +\frac{1}{100\times 101}$

$=1-\frac{1}{2}+\frac{1}{2}-\frac{1}{3}+\cdots+\frac{1}{100}-\frac{1}{101}=1-\frac{1}{101}=\frac{100}{101}$

 答案:A \\


知识点:数列的概念

难度:1

 设$\mathrm{\{}$\textit{a${}_{n}$}$\mathrm{\}}$(\textit{n}$\mathrm{\in}$N\textit{${}_{+}$})是等差数列,\textit{S${}_{n}$}是其前\textit{n}项和,且\textit{S}${}_{5}$\textit{$<$S}${}_{6}$,\textit{S}${}_{6}$\textit{=S}${}_{7}$\textit{$>$S}${}_{8}$,则下列结论错误的是(\textit{  })

 A.\textit{d$<$}0 

 B.\textit{a}${}_{7}$\textit{=}0

 C.\textit{S}${}_{9}$\textit{$>$S}${}_{5}$ 

 D.\textit{S}${}_{6}$和\textit{S}${}_{7}$均为\textit{S${}_{n}$}的最大值

 解析:由\textit{S}${}_{5}$\textit{$<$S}${}_{6}$得$a_1+a_2+\cdots+a_5<a_1+a_2+\cdots +a_6,\therefore a_6>0$

又$S_6=S_7,\therefore a_1+a_2+\cdots+a_6=a_1+a_2+\cdots+a_6+a_7,\therefore a_7=0$,故B正确;

同理由\textit{S}${}_{7}$\textit{$>$S}${}_{8}$,得\textit{a}${}_{8}$\textit{$<$}0,

又\textit{d=a}${}_{7}$\textit{-a}${}_{6}$\textit{$<$}0,故A正确;

由C选项中\textit{S}${}_{9}$\textit{$>$S}${}_{5}$,即$a_6+a_7+a_8+a_9>0$,

可得$2(a_7+a_8)>0$

而由\textit{a}${}_{7}$\textit{=}0,\textit{a}${}_{8}$\textit{$<$}0,知2(\textit{a}${}_{7}$\textit{+a}${}_{8}$)\textit{$>$}0不可能成立,故C错误;

\textit{$\because$S}${}_{5}$\textit{$<$S}${}_{6}$,\textit{S}${}_{6}$\textit{=S}${}_{7}$\textit{$>$S}${}_{8}$,\textit{$\therefore$S}${}_{6}$与\textit{S}${}_{7}$均为\textit{S${}_{n}$}的最大值,故D正确\textit{.}故选C\textit{.}

 答案:C \\

知识点:数列的概念

难度:1

 数列$\{\frac{1}{(n+1)^2-1}\}$的前\textit{n}项和\textit{S${}_{n}$}为(\textit{  })

 A.$\frac{n+1}{2(n+2)}$

 B.$\frac{3}{4}-\frac{n+1}{2(n+2)}$

 C.$\frac{3}{4}-\frac{1}{2}(\frac{1}{n+1}+\frac{1}{n+2})$

 D.$\frac{3}{2}-\frac{1}{n+1}-\frac{1}{n+2}$

 解析:$\frac{1}{(n+1)^2-1}=\frac{1}{n^2+2n}=\frac{1}{2}(\frac{1}{n}-\frac{1}{n+2})$,

于是$S_n=\frac{1}{2}(1-\frac{1}{3}+\frac{1}{2}-\frac{1}{4}+\frac{1}{3}-\frac{1}{5}+\cdots +\frac{1}{n}-\frac{1}{n+2})=\frac{3}{4}-\frac{1}{2}(\frac{1}{n+1}+\frac{1}{n+2})$

 答案:C \\


知识点:数列的概念

难度:1

 设函数\textit{f}(\textit{x})满足$f(n+1)=\frac{2f(n)+n}{2}$(\textit{n}$\mathrm{\in}$N\textit{${}_{+}$}),且$f(1)=2$,则$f(20)$为(\textit{  })

 A.95 B.97 C.105 D.192

 解析:$\because f(n+1)=f(n)+\frac{n}{2}$,

$\therefore f(n+1)-f(n)=\frac{n}{2}$

$\therefore f(2)-f(1)=\frac{1}{2}$,

$f(3)-f(2)=\frac{2}{2}$,

{\dots}{\dots}

$f(20)-f(19)=\frac{19}{2}$,

$\therefore f(20)-f(1)=\frac{1+2+\cdots +19}{2}=\frac{\frac{(1+19)\times 19}{2}}{2}=95$

又$f(1)=2, \therefore f(20)=97$

 答案:B \\

知识点:数列的概念

难度:1

 已知数列$\mathrm{\{}$\textit{a${}_{n}$}$\mathrm{\}}$的前\textit{n}项和\textit{S${}_{n}$=n}${}^{2}$\textit{-}9\textit{n},第\textit{k}项满足5\textit{$<$a${}_{k}$$<$}8,则\textit{k=\underbar{     }.~}

 解析:\textit{a${}_{n}$=S${}_{n}$-S${}_{n-}$}${}_{1}$\textit{=}(\textit{n}${}^{2}$\textit{-}9\textit{n})\textit{-}[(\textit{n-}1)${}^{2}$\textit{-}9(\textit{n-}1)]\textit{=}2\textit{n-}10(\textit{n}$\mathrm{\ge}$2),又\textit{a}${}_{1}$\textit{=S}${}_{1}$\textit{=-}8符合上式,所以\textit{a${}_{n}$=}2\textit{n-}10\textit{.}

令5\textit{$<$}2\textit{k-}10\textit{$<$}8,解得$\frac{15}{2}<k<9$

又\textit{k}$\mathrm{\in}$N\textit{${}_{+}$},所以\textit{k=}8\textit{.}

 答案:8 \\

知识点:数列的概念

难度:1

 设数列$\mathrm{\{}$\textit{a${}_{n}$}$\mathrm{\}}$的前\textit{n}项和为\textit{S${}_{n}$},$S_n=\frac{a_1(3^n-1)}{2}$,且\textit{a}${}_{4}$\textit{=}54,则\textit{a}${}_{1}$\textit{=\underbar{     }.~}

 解析:因为$a_4=S_4-S_3=\frac{a_1(3^4-1)}{2}-\frac{a_1(3^3-1)}{2}$,

所以27\textit{a}${}_{1}$\textit{=}54,解得\textit{a}${}_{1}$\textit{=}2\textit{.}

 答案:2 \\

知识点:数列的概念

难度:1

 数列$1,\frac{1}{1+2},\frac{1}{1+2+3},\cdots, \frac{1}{1+2+3+\cdots+n}$的前\textit{n}项和\textit{S${}_{n}$=\underbar{       }.~}

 解析:因为$\frac{1}{1+2+3+\cdots+n}=\frac{1}{\frac{n(n+1)}{2}}=\frac{2}{n(n+1)}=2(\frac{1}{n}-\frac{1}{n+1})$

所以$S_n=1+\frac{1}{1+2}+\frac{1}{1+2+3}+\cdots+\frac{1}{1+2+3+\cdots+n}=2[(1-\frac{1}{2})+(\frac{1}{2}-\frac{1}{3})+(\frac{1}{3}-\frac{1}{4})+\cdots+(\frac{1}{n}-\frac{1}{n+1})]=2(1-\frac{1}{n+1})=\frac{2n}{n+1}$

 答案:$\frac{2n}{n+1}$ \\

知识点:数列的概念

难度:1

 正项数列$\mathrm{\{}$\textit{a${}_{n}$}$\mathrm{\}}$满足]$a_n^2-(2n-1)a_n-2n=0$

 (1)求数列$\mathrm{\{}$\textit{a${}_{n}$}$\mathrm{\}}$的通项公式\textit{a${}_{n}$};

 (2)令$b_n=\frac{1}{(n+1)a_n}$,求数列$\mathrm{\{}$\textit{b${}_{n}$}$\mathrm{\}}$的前\textit{n}项和\textit{T${}_{n}$.}

解析:

 答案:(1)由$a_n^2-(2n-1)a_n-2n=0$,

得$(a_n-2n)(a_n+1)=0$,即\textit{a${}_{n}$=}2\textit{n}或\textit{a${}_{n}$=-}1,

由于$\mathrm{\{}$\textit{a${}_{n}$}$\mathrm{\}}$是正项数列,故\textit{a${}_{n}$=}2\textit{n.}

(2)由(1)知\textit{a${}_{n}$=}2\textit{n},所以$b_n=\frac{1}{(n+1)a_n}=\frac{1}{2n(n+1)}=\frac{1}{2}(\frac{1}{n}-\frac{1}{n+1})$,

故$T_n=\frac{1}{2}(1-\frac{1}{2}+\frac{1}{2}-\frac{1}{3}+\cdots +\frac{1}{n}-\frac{1}{n+1})=\frac{1}{2}(1-\frac{1}{n+1})=\frac{n}{2(n+1)}$ \\

知识点:数列的概念

难度:1

 已知等差数列$\mathrm{\{}$\textit{a${}_{n}$}$\mathrm{\}}$的前\textit{n}项和为\textit{S${}_{n}$},\textit{n}$\mathrm{\in}$N\textit{${}_{+}$},且$a_3+a_6=4,S_5=-5$

 (1)求\textit{a${}_{n}$};

 (2)若$T_n=|a_1|+|a_2|+|a_3|+\cdots+|a_n|$,求\textit{T}${}_{5}$的值和\textit{T${}_{n}$}的表达式\textit{.}

解析:

 答案:(1)设$\mathrm{\{}$\textit{a${}_{n}$}$\mathrm{\}}$的首项为\textit{a}${}_{1}$,公差为\textit{d},易由$a_3+a_6=4,S_5=-5$得出\textit{a}${}_{1}$\textit{=-}5,\textit{d=}2\textit{.}

\textit{$\therefore$a${}_{n}$=}2\textit{n-}7\textit{.}

(2)当\textit{n}$\mathrm{\ge}$4时,\textit{a${}_{n}$=}2\textit{n-}7\textit{$>$}0;当\textit{n}$\mathrm{\le}$3时,\textit{a${}_{n}$=}2\textit{n-}7\textit{$<$}0,

$\therefore T_5=-(a_1+a_2+a_3)+a_4+a_5=13$

当1$\mathrm{\le}$\textit{n}$\mathrm{\le}$3时,$T_n=-(a_1+a_2+\cdots+a_n)=-n^2+6n$;

当\textit{n}$\mathrm{\ge}$4时,$T_n=-(a_1+a_2+a_3)+a_4+a_5+\cdots+a_n=n^2-6n+18$

综上所述,$T_n=\left\{
\begin{array}{l}
-n^2+6n,1\le n\le 3, \\
n^2-6n+18,n\ge 4
\end{array}
\right.$ \\

知识点:数列的概念

难度:2

 若等差数列$\mathrm{\{}$\textit{a${}_{n}$}$\mathrm{\}}$的通项公式为$a_n=2n+1$,则由$b_n=\frac{a_1+a_2+\cdots+a_n}{n}$所确定的数列$\mathrm{\{}$\textit{b${}_{n}$}$\mathrm{\}}$的前\textit{n}项之和是(\textit{  })

 A.$n(n+2)$ B.$\frac{1}{2}n(n+4)$

 C.$\frac{1}{2}n(n+5)$ D.$\frac{1}{2}n(n+6)$

 解析:由题意知$a_1+a_2+\cdots+a_n=\frac{n(3+2n+1)}{2}=n(n+2),\therefore b_n=\frac{n(n+2)}{n}=n+2$,于是数列$\mathrm{\{}$\textit{b${}_{n}$}$\mathrm{\}}$的前\textit{n}项和$S_n=\frac{n(3+n+2)}{2}=\frac{1}{2}n(n+5)$

 答案:C \\

知识点:数列的概念

难度:2

 已知一个等差数列共\textit{n}项,且其前四项之和为21,末四项之和为67,前\textit{n}项和为286,则项数\textit{n}为(\textit{  })

 A.24 B.26 C.25 D.28

 解析:设该等差数列为$\mathrm{\{}$\textit{a${}_{n}$}$\mathrm{\}}$,

由题意,得$a_1+a_2+a_3+a_4=21$,

$a_n+a_{n-1}+a_{n-2}+a_{n-3}=67$,

又$a_1+a_n=a_2+a_{n-1}=a_3+a_{n-2}=a_4+a_{n-3}$,

$\therefore 4(a_1+a_n)=21+67=88, \therefore a_1+a_n=22$

$\therefore S_n=\frac{n(a_1+a_n)}{2}=11n=286, \therefore n=26$
 
 答案:B \\

知识点:数列的概念

难度:2

 已知数列$\mathrm{\{}$\textit{a${}_{n}$}$\mathrm{\}}$满足$a_1=1,a_n=a_{n-1}+2n$(\textit{n}$\mathrm{\ge}$2),则\textit{a}${}_{7}$\textit{=} (\textit{  })

 A.53 B.54 C.55 D.109

 解析:$\because a_n=a_{n-1}+2n,\therefore a_n-a_{n-1}=2n$

$\therefore a_2-a_1=4, a_3-a_2=6, a_4-a_3=8,\cdots, a_n-a_{n-1}=2n(n\ge 2)$

$\therefore a_n=1+4+6+\cdots+2n=1+\frac{(n-1)(4+2n)}{2}=n^2-n+1$

$a_7=7^2+7-1=55$

 答案:C \\

知识点:数列的概念

难度:2

 已知数列$\{a_n\}$为$\frac{1}{2},\frac{1}{3}+\frac{2}{3},\frac{1}{4}+\frac{2}{4}+\frac{3}{4},\cdots,\frac{1}{10}+\frac{2}{10}+\frac{3}{10}+\cdots+\frac{9}{10},\cdots$,如果$b_n=\frac{1}{a_na_{n+1}}$,那么数列$\{b_n\}$的前\textit{n}项和\textit{S${}_{n}$}为(\textit{  })

 A.$\frac{n}{n+1}$ B.$\frac{4n}{n+1}$ C.$\frac{3n}{n+1}$ D.$\frac{5n}{n+1}$

 解析:$\because a_n=\frac{1+2+3+\cdots+n}{n+1}=\frac{n}{2}$,

$\therefore b_n=\frac{1}{a_na_{n+1}}=\frac{4}{n(n+1)}=4(\frac{1}{n}-\frac{1}{n-1})$,

$\therefore Sn=4[(1-\frac{1}{2})+(\frac{1}{2}-\frac{1}{3})+\cdots+(\frac{1}{n-1}-\frac{1}{n})+(\frac{1}{n}-\frac{1}{n+1})]=4(1-\frac{1}{n+1})=\frac{4n}{n+1}$

 答案:B \\

知识点:数列的概念

难度:2

 已知数列$\mathrm{\{}$\textit{a${}_{n}$}$\mathrm{\}}$的前\textit{n}项和为$S_n=n^2+n+1$,则\textit{a${}_{n}$=\underbar{          }.~}

 解析:当\textit{n=}1时,\textit{a}${}_{1}$\textit{=S}${}_{1}$\textit{=}3;

当\textit{n}$\mathrm{\ge}$2时,$a_n=S_n-S_{n-1}=n^2+n+1-[(n-1)^2+(n-1)+1]=2n$

此时,当\textit{n=}1时,2\textit{n=}2$\mathrm{\neq}$3\textit{.}

所以$a_n=\left\{
\begin{array}{l}
3(n=1), \\
2n(n\ge 2)
\end{array}
\right.$

 答案:$\left\{
\begin{array}{l}
3(n=1), \\
2n(n\ge 2)
\end{array}
\right.$ \\

知识点:数列的概念

难度:2

 设\textit{S${}_{n}$}是等差数列$\mathrm{\{}$\textit{a${}_{n}$}$\mathrm{\}}$的前\textit{n}项和,已知\textit{S}${}_{6}$\textit{=}36,\textit{S${}_{n}$=}324,若\textit{S${}_{n-}$}${}_{6}$\textit{=}144(\textit{n$>$}6),则数列的项数\textit{n}为\textit{\underbar{     }.~}

 解析:由题意可知$\left\{
\begin{array}{l}
a_1+a_2+\cdots+a_6=36, ① \\
a_n+a_{n-1}+\cdots +a_{n-5}=324-144, ②
\end{array}
\right.$

由①+②,得$(a_1+a_n)+(a_2+a_{n-1})+\cdots+(a_6+a_{n-5})=216,\therefore 6(a_1+a_n)=216,\therefore a_1+a_n=36$

$\therefore S_n = \frac{n(a_1+a_n)}{2}=18n=324, \therefore n=18$

 答案:18 \\

知识点:数列的概念

难度:2

 设数列$\mathrm{\{}$\textit{a${}_{n}$}$\mathrm{\}}$的前$S_n,a_1=1,a_n=\frac{S_n}{n}+2(n-1)$(\textit{n}$\mathrm{\in}$N\textit{${}_{+}$})\textit{.}

 (1)求证:数列$\mathrm{\{}$\textit{a${}_{n}$}$\mathrm{\}}$为等差数列,并求\textit{a${}_{n}$}与\textit{S${}_{n}$};

 (2)是否存在自然数\textit{n},使得$S_1+\frac{S_2}{2}+\frac{S_3}{3}+\cdots+\frac{S_n}{n}-(n-1)^2=2019$?若存在,求出\textit{n}的值;若不存在,请说明理由\textit{.}

解析:

 答案:(1)由$a_n=\frac{S_n}{n}+2(n-1)$,

得$S_n=na_n-2n(n-1)(n\in N_{+})$

当\textit{n}$\mathrm{\ge}$2时,\textit{a${}_{n}$=S${}_{n}$-S${}_{n-}$}${}_{1}$\textit{=na${}_{n}$-}(\textit{n-}1)\textit{a${}_{n-}$}${}_{1}$\textit{-}4(\textit{n-}1),即\textit{a${}_{n}$-a${}_{n-}$}${}_{1}$\textit{=}4,

故数列$\mathrm{\{}$\textit{a${}_{n}$}$\mathrm{\}}$是以1为首项,4为公差的等差数列\textit{.}

于是,$a_n=4n-3,S_n=\frac{(a_1+a_n)n}{2}=2n^2-n$

 (2)存在自然数\textit{n}使得$S_1+\frac{S_2}{2}+\frac{S_3}{3}+\cdots+\frac{S_n}{n}-(n-1)^2=2019$成立\textit{.}理由如下:

由(1),得$\frac{S_n}{n}=2n-1$(\textit{n}$\mathrm{\in}$N\textit{${}_{+}$}),

所以$S_1+\frac{S_2}{2}+\frac{S_3}{3}+\cdots+\frac{S_n}{n}-(n-1)^2=1+3+5+7+\cdots+(2n-1)-(n-1)^2=n^2-(n-1)^2=2n-1$

令2\textit{n-}1\textit{=}2019,得\textit{n=}1010,

所以存在满足条件的自然数\textit{n}为1010\textit{.} \\

知识点:数列的概念

难度:2

 数列$\mathrm{\{}$\textit{a${}_{n}$}$\mathrm{\}}$的前\textit{n}项和\textit{S${}_{n}$=}100\textit{n-n}${}^{2}$(\textit{n}$\mathrm{\in}$N\textit{${}_{+}$})\textit{.}

 (1)求证$\mathrm{\{}$\textit{a${}_{n}$}$\mathrm{\}}$是等差数列;

 (2)设$b_n=\{|a_n|\}$,求数列$\mathrm{\{}$\textit{b${}_{n}$}$\mathrm{\}}$的前\textit{n}项和\textit{.}

解析:

 答案:(1)\textit{a${}_{n}$=S${}_{n}$-S${}_{n-}$}${}_{1}$

\textit{=}(100\textit{n-n}${}^{2}$)\textit{-}[100(\textit{n-}1)\textit{-}(\textit{n-}1)${}^{2}$]

\textit{=}101\textit{-}2\textit{n}(\textit{n}$\mathrm{\ge}$2)\textit{.}

\textit{$\because$a}${}_{1}$\textit{=S}${}_{1}$\textit{=}100\textit{$\times$}1\textit{-}1${}^{2}$\textit{=}99\textit{=}101\textit{-}2\textit{$\times$}1,

\textit{$\therefore$}数列$\mathrm{\{}$\textit{a${}_{n}$}$\mathrm{\}}$的通项公式为\textit{a${}_{n}$=}101\textit{-}2\textit{n}(\textit{n}$\mathrm{\in}$N\textit{${}_{+}$})\textit{.}

又\textit{a${}_{n+}$}${}_{1}$\textit{-a${}_{n}$=-}2为常数,\textit{$\therefore$}数列$\mathrm{\{}$\textit{a${}_{n}$}$\mathrm{\}}$是首项\textit{a}${}_{1}$\textit{=}99,公差\textit{d=-}2的等差数列\textit{.}

 (2)令\textit{a${}_{n}$=}101\textit{-}2\textit{n}$\mathrm{\ge}$0,得\textit{n}$\mathrm{\le}$50\textit{.}5\textit{.}

\textit{$\because$n}$\mathrm{\in}$N\textit{${}_{+}$},\textit{$\therefore$n}$\mathrm{\le}$50(\textit{n}$\mathrm{\in}$N\textit{${}_{+}$})\textit{.}

\textit{①}当1$\mathrm{\le}$\textit{n}$\mathrm{\le}$50时\textit{a${}_{n}$$>$}0,此时$b_n=\{|a_n|\}=a_n$,

\textit{$\therefore$}$\mathrm{\{}$\textit{b${}_{n}$}$\mathrm{\}}$的前\textit{n}项和\textit{S${}_{n}$'=}100\textit{n-n}${}^{2}$;

\textit{②}当\textit{n}$\mathrm{\ge}$51时\textit{a${}_{n}$$<$}0,此时$b_n=\{|a_n|\}=-a_n$,

由$b_{51}+b_{52}+\cdots+b_n=-(a_{51}+a_{52}+\cdots+a_n)=-(S_n-S_{50})=S_{50}-S_n$,

得数列$\mathrm{\{}$\textit{b${}_{n}$}$\mathrm{\}}$的前\textit{n}项和为

$S_n^{'}=S_{50}+(S_{50}-S_n)=2S_{50}-S_n=2\times 2500-(100n-n^2)=5000-100n+n^2$

由\textit{①②}得数列$\mathrm{\{}$\textit{b${}_{n}$}$\mathrm{\}}$的前\textit{n}项和为

$S_n^{'}=\left\{
\begin{array}{l}
100n-n^2(1\le n\le 50, n\in N_{+}) \\
5000-100n+n^2(n\ge 51, n\in N_{+})
\end{array}
\right.$ \\

知识点:等比数列的概念

难度:1

 若$\mathrm{\{}$\textit{a${}_{n}$}$\mathrm{\}}$是等比数列,则下列数列不是等比数列的是 (\textit{  })

 \textit{                }

 A.$\{a_n+1\}$ B.$\{\frac{1}{a_n}\}$ C.$\{4a_n\}$ D.$\{a_n^2\}$

 答案:A \\

知识点:等比数列的概念

难度:1

 在等比数列$\mathrm{\{}$\textit{a${}_{n}$}$\mathrm{\}}$中,2\textit{a}${}_{4}$\textit{=a}${}_{6}$\textit{-a}${}_{5}$,则公比是(\textit{  })

 A.0 B.1或2

 C.\textit{-}1或2 D.\textit{-}1或\textit{-}2

 解析:设公比为\textit{q}(\textit{q}$\mathrm{\neq}$0),由已知得2\textit{a}${}_{1}$\textit{q}${}^{3}$\textit{=a}${}_{1}$\textit{q}${}^{5}$\textit{-a}${}_{1}$\textit{q}${}^{4}$,

\textit{$\therefore$}2\textit{=q}${}^{2}$\textit{-q},\textit{$\therefore$q}${}^{2}$\textit{-q-}2\textit{=}0,

\textit{$\therefore$q=-}1或\textit{q=}2\textit{.}
 
 答案:C \\

知识点:等比数列的概念

难度:1

 若一个等比数列的首项为$\frac{9}{4}$,末项为$\frac{2}{3}$,公比为$\frac{2}{3}$,则这个数列的项数为(\textit{  })

 A.3 B.4 C.5 D.6

 解析:在等比数列中,

$\because \frac{2}{3}=\frac{9}{4}\cdot (\frac{2}{3})^{n-1}=(\frac{2}{3})^{n-3}$,

\textit{$\therefore$n-}3\textit{=}1,即\textit{n=}4,故选B.

 答案:B \\

知识点:等比数列的概念

难度:1

 若数列$\mathrm{\{}$\textit{a${}_{n}$}$\mathrm{\}}$满足\textit{a${}_{n+}$}${}_{1}$\textit{=}4\textit{a${}_{n}$+}6(\textit{n}$\mathrm{\in}$N\textit{${}_{+}$})且\textit{a}${}_{1}$\textit{$>$}0,则下列数列是等比数列的是(\textit{  })

 A.$\{a_n+6\}$ B.$\{a_n+1\}$

 C.$\{a_n+3\}$ D.$\{a_n+2\}$

 解析:由\textit{a${}_{n+}$}${}_{1}$\textit{=}4\textit{a${}_{n}$+}6可得$a_{n+1}+2=4a_n+8=4(a_n+2)$,因为\textit{a}${}_{1}$\textit{$>$}0,

所以\textit{a${}_{n}$$>$}0,从而$a_n+2$(\textit{n}$\mathrm{\in}$N\textit{${}_{+}$}),因此$\frac{a_{n+1}+2}{a_n+2}=4$,故$\{a_n+2\}$是等比数列\textit{.}

 答案:D \\

知识点:等比数列的概念

难度:1

 在等比数列$\mathrm{\{}$\textit{a${}_{n}$}$\mathrm{\}}$中,若\textit{a}${}_{5}$·\textit{a}${}_{6}$·\textit{a}${}_{7}$\textit{=}3,\textit{a}${}_{6}$·\textit{a}${}_{7}$·\textit{a}${}_{8}$\textit{=}24,则\textit{a}${}_{7}$·\textit{a}${}_{8}$·\textit{a}${}_{9}$的值等于(\textit{  })

 A.48 B.72 C.144 D.192

 解析:设公比为\textit{q},由\textit{a}${}_{6}$·\textit{a}${}_{7}$·\textit{a}${}_{8}$\textit{=a}${}_{5}$·\textit{a}${}_{6}$·\textit{a}${}_{7}$·\textit{q}${}^{3}$,

得$q^3=\frac{24}{3}=8$

所以\textit{a}${}_{7}$·\textit{a}${}_{8}$·\textit{a}${}_{9}$\textit{=a}${}_{6}$·\textit{a}${}_{7}$·\textit{a}${}_{8}$·\textit{q}${}^{3}$\textit{=}24\textit{$\times$}8\textit{=}192\textit{.}

 答案:D \\

知识点:等比数列的概念

难度:1

 数列$\mathrm{\{}$\textit{a${}_{n}$}$\mathrm{\}}$是公差不为0的等差数列,且\textit{a}${}_{1}$,\textit{a}${}_{3}$,\textit{a}${}_{7}$为等比数列$\mathrm{\{}$\textit{b${}_{n}$}$\mathrm{\}}$的连续三项,则数列$\mathrm{\{}$\textit{b${}_{n}$}$\mathrm{\}}$的公比为(\textit{  })

 A.$\sqrt{2}$ B.4 C.2 D.$\frac{1}{2}$

 解析:\textit{$\because$a}${}_{1}$,\textit{a}${}_{3}$,\textit{a}${}_{7}$为等比数列$\mathrm{\{}$\textit{b${}_{n}$}$\mathrm{\}}$中的连续三项,

$\therefore a_3^2=a_1\cdot a_7$

设$\mathrm{\{}$\textit{a${}_{n}$}$\mathrm{\}}$的公差为\textit{d},则\textit{d}$\mathrm{\neq}$0,

$\therefore (a_1+2d)^2=a_1(a_1+6d), \therefore a_1=2d$

\textit{$\therefore$}公比$q=\frac{a_3}{a_1}=\frac{4d}{2d}=2$,故选C\textit{.}

 答案:C \\

知识点:等比数列的概念

难度:1

 (2017全国3高考)设等比数列$\mathrm{\{}$\textit{a${}_{n}$}$\mathrm{\}}$满足$a_1+a_2=-1,a_1-a_3=-3$,则\textit{a}${}_{4}$\textit{=\underbar{     }.~}

 解析:设$\mathrm{\{}$\textit{a${}_{n}$}$\mathrm{\}}$的公比为\textit{q},则由题意,得$\left\{
\begin{array}{l}
a_1(1+q)=-1 \\
a_1(1-q^2)=-3
\end{array}
\right.$,解得$\left\{
\begin{array}{l}
a_1=1, \\
q=-2
\end{array}
\right.$故\textit{a}${}_{4}$\textit{=a}${}_{1}$\textit{q}${}^{3}$\textit{=-}8\textit{.}

 答案:\textit{-}8 \\

知识点:等比数列的概念

难度:1

 设数列$\mathrm{\{}$\textit{a${}_{n}$}$\mathrm{\}}$是等比数列,公比\textit{q=}2,则$\frac{2a_1+a_2}{2a_3+a_4}$的值是\textit{\underbar{     }.~}

 解析:\textit{$\because$q=}2,\textit{$\therefore$}2\textit{a}${}_{1}$\textit{=a}${}_{2}$,2\textit{a}${}_{3}$\textit{=a}${}_{4}$,

$\therefore \frac{2a_1+a_2}{2a_3+a_4}=\frac{2a_2}{2a_4}=\frac{1}{q^2}=\frac{1}{4}$

 答案:$\frac{1}{4}$ \\

知识点:等比数列的概念

难度:1

 已知数列$\mathrm{\{}$\textit{a${}_{n}$}$\mathrm{\}}$满足\textit{a}${}_{9}$\textit{=}1,\textit{a${}_{n+}$}${}_{1}$\textit{=}2\textit{a${}_{n}$}(\textit{n}$\mathrm{\in}$N\textit{${}_{+}$}),则\textit{a}${}_{5}$\textit{=\underbar{     }.~}

 解析:由\textit{a${}_{n+}$}${}_{1}$\textit{=}2\textit{a${}_{n}$}(\textit{n}$\mathrm{\in}$N\textit{${}_{+}$})知,数列$\mathrm{\{}$\textit{a${}_{n}$}$\mathrm{\}}$是公比$q=\frac{a_{n+1}}{a_n}=2$的等比数列\textit{.}

所以$a_5=a_1q^4=\frac{a_1q^8}{q^4}=\frac{a_9}{q^4}=\frac{1}{16}$

 答案:$\frac{1}{16}$ \\

知识点:等比数列的概念

难度:1

 若数列$\mathrm{\{}$\textit{a${}_{n}$}$\mathrm{\}}$为等差数列,且\textit{a}${}_{2}$\textit{=}3,\textit{a}${}_{5}$\textit{=}9,则数列$\{(\frac{1}{2})^{a_n}\}$一定是\textit{\underbar{      }}数列(填``等差''或``等比'')\textit{.~}

 解析:设$\mathrm{\{}$\textit{a${}_{n}$}$\mathrm{\}}$的公差为\textit{d},则$\left\{
\begin{array}{l}
a_1+d=3, \\
a_1+4d=9.
\end{array}
\right.$

解得$\left\{
\begin{array}{l}
a_1=1, \\
d=2.
\end{array}
\right.$于是\textit{a${}_{n}$=}2\textit{n-}1,

从而$(\frac{1}{2})^{a_n}=(\frac{1}{2})^{2n-1}=2\cdot (\frac{1}{4})^n$,

设$b_n=2\cdot(\frac{1}{4})^n$,则$\frac{b_{n+1}}{b_n}=\frac{1}{4}$,

故$\{(\frac{1}{2})^{a_n}\}$一定是等比数列\textit{.}

 答案:等比 \\

知识点:等比数列的概念

难度:1

 在等比数列$\mathrm{\{}$\textit{a${}_{n}$}$\mathrm{\}}$中,\textit{a}${}_{1}$·\textit{a}${}_{9}$\textit{=}256,$a_4+a_6=40$,则公比\textit{q=\underbar{ }.~}

 解析:$\because a_1a_9=a_1^2 q^8,a_4a_6=a_1q^3\cdot a_1q^5=a_1^2 q^8$,

\textit{$\therefore$a}${}_{1}$\textit{a}${}_{9}$\textit{=a}${}_{4}$\textit{a}${}_{6}$\textit{.}

可得方程组$\left\{
\begin{array}{l}
a_4+a_6=40, \\
a_4\cdot a_6=256
\end{array}
\right.$

解得$\left\{
\begin{array}{l}
a_4=32, \\
a_6=8
\end{array}
\right.$ 或$\left\{
\begin{array}{l}
a_4=8, \\
a_6=32
\end{array}
\right.$

$q^2=\frac{a_6}{a_4}=\frac{8}{32}=\frac{1}{4}$或$q^2=\frac{32}{8}=4$

$q=\pm \frac{1}{2}$或$q=\pm 2$

 答案:$-2,2,-\frac{1}{2},\frac{1}{2}$ \\

知识点:等比数列的概念

难度:1

 在等比数列$\mathrm{\{}$\textit{a${}_{n}$}$\mathrm{\}}$中,已知\textit{a}${}_{1}$\textit{=}2,\textit{a}${}_{4}$\textit{=}16\textit{.}

 (1)求数列$\mathrm{\{}$\textit{a${}_{n}$}$\mathrm{\}}$的通项公式;

 (2)若\textit{a}${}_{3}$,\textit{a}${}_{5}$分别为等差数列$\mathrm{\{}$\textit{b${}_{n}$}$\mathrm{\}}$的第3项和第5项,试求数列$\mathrm{\{}$\textit{b${}_{n}$}$\mathrm{\}}$的通项公式\textit{.}

解析:

 答案:(1)设$\mathrm{\{}$\textit{a${}_{n}$}$\mathrm{\}}$的公比为\textit{q}(\textit{q}$\mathrm{\neq}$0),由已知得16\textit{=}2·\textit{q}${}^{3}$,解得\textit{q=}2,\textit{$\therefore$a${}_{n}$=a}${}_{1}$·\textit{q${}^{n-}$}${}^{1}$\textit{=}2\textit{$\times$}2\textit{${}^{n-}$}${}^{1}$\textit{=}2\textit{${}^{n}$.}

(2)由\eqref{GrindEQ__1_}得\textit{a}${}_{3}$\textit{=}8,\textit{a}${}_{5}$\textit{=}32,则\textit{b}${}_{3}$\textit{=}8,\textit{b}${}_{5}$\textit{=}32,

设$\mathrm{\{}$\textit{b${}_{n}$}$\mathrm{\}}$的公差为\textit{d},

则有$\left\{
\begin{array}{l}
b_1+2d=8, \\
b_1+4d=32
\end{array}
\right.$

解得$\left\{
\begin{array}{l}
b_1=-16, \\
d=12.
\end{array}
\right.$

$\therefore -16+12(n-1)=12n-28$ \\

知识点:等比数列的概念

难度:1

 已知关于\textit{x}的二次方程$a_nx^2-a_{n+1}x+1=0$(\textit{n}$\mathrm{\in}$N\textit{${}_{+}$})的两根\textit{$\alpha$},\textit{$\beta$}满足$6\alpha-2\alpha \beta+6\beta=3$,且\textit{a}${}_{1}$\textit{=}1\textit{.}

 (1)试用\textit{a${}_{n}$}表示\textit{a${}_{n+}$}${}_{1}$;

 (2)求证:数列$\{a_n-\frac{2}{3}\}$为等比数列;

 (3)求数列$\mathrm{\{}$\textit{a${}_{n}$}$\mathrm{\}}$的通项公式\textit{.}

解析:

 答案:(1)因为\textit{$\alpha$},\textit{$\beta$}是方程$a_nx^2-a_{n+1}x+1=0$(\textit{n}$\mathrm{\in}$N\textit{${}_{+}$})的两根,

所以$\left\{
\begin{array}{l}
\alpha +\beta=\frac{a_{n+1}}{a_n}, \\
\alpha \beta =\frac{1}{a_n}.
\end{array}
\right.$

又因为$6\alpha-2\alpha \beta +6\beta=3$,所以6\textit{a${}_{n+}$}${}_{1}$\textit{-}3\textit{a${}_{n}$-}2\textit{=}0\textit{.}

所以$a_{n+1}=\frac{1}{2}a_n+\frac{1}{3}$

 (2)因为$a_{n+1}=\frac{1}{2}a_n+\frac{1}{3} \Rightarrow a_{n+1}-\frac{2}{3}=\frac{1}{2}a_n-\frac{1}{3}\Rightarrow \frac{a_{n+1}-\frac{2}{3}}{a_n-\frac{2}{3}}=\frac{1}{2}$为常数,且$a_1-\frac{2}{3}=\frac{1}{3}$

所以$\{a_n-\frac{2}{3}\}$为等比数列\textit{.}

 (3)令$b_n=a_n-\frac{2}{3}$,则$\mathrm{\{}$\textit{b${}_{n}$}$\mathrm{\}}$为等比数列,公比为$\frac{1}{2}$,首项$b_1=a_1-\frac{2}{3}=\frac{1}{3}$,

所以$b_n=\frac{1}{3}\cdot (\frac{1}{2})^{n-1}$

所以$a_n=b_n+\frac{2}{3}=\frac{1}{3}\cdot (\frac{1}{2})^{n-1}+\frac{2}{3}$

所以数列$\mathrm{\{}$\textit{a${}_{n}$}$\mathrm{\}}$的通项公式为$a_n=\frac{1}{3}\cdot (\frac{1}{2})^{n-1}+\frac{2}{3}$ \\

知识点:等比数列的概念

难度:1

 容积为\textit{a} L(\textit{a$>$}1)的容器盛满酒精后倒出1 L,然后加满水,再倒出1 L混合溶液后又用水加满,如此继续下去,问第\textit{n}次操作后溶液的浓度是多少?当\textit{a=}2时,至少应倒出几次后才可能使酒精浓度低于10\%?

解析:

 答案:开始的浓度为1,操作一次后溶液的浓度是$a_1=1-\frac{1}{a}$

设操作\textit{n}次后溶液的浓度是\textit{a${}_{n}$},则操作$n+1$次后溶液的浓度是$a_{n+1}=a_n(1-\frac{1}{a})$

所以$\mathrm{\{}$\textit{a${}_{n}$}$\mathrm{\}}$构成以$a_1=1-\frac{1}{a}$为首项,$q=1-\frac{1}{a}$为公比的等比数列\textit{.}

所以$a_n=(1-\frac{1}{a})^n$,即第\textit{n}次操作后溶液的浓度是$(1-\frac{1}{a})^n$

当\textit{a=}2时,由$a_n=(\frac{1}{2})^n<\frac{1}{10}$,得\textit{n}$\mathrm{\ge}$4\textit{.}

因此,至少应倒4次后才可以使酒精浓度低于10\%\textit{.} \\

知识点:等比数列的概念

难度:1

 在等比数列$\mathrm{\{}$\textit{a${}_{n}$}$\mathrm{\}}$中,\textit{a}${}_{5}$\textit{=}3,则\textit{a}${}_{2}$·\textit{a}${}_{8}$\textit{=}(\textit{  })

 \textit{ } \textit{      } \textit{      } \textit{   }

 A.3 B.6 C.8 D.9

 解析:$a_2\cdot a_8 =a_5^2=3^2=9$

 答案:D \\

知识点:等比数列的概念

难度:1

 若1,\textit{a}${}_{1}$,\textit{a}${}_{2}$,4成等差数列,1,\textit{b}${}_{1}$,\textit{b}${}_{2}$,\textit{b}${}_{3}$,4成等比数列,则$\frac{a_1-a_2}{b_2}$的值等于(\textit{  })

 A.$-\frac{1}{2}$ B.$\frac{1}{2}$ C.$\pm \frac{1}{2}$ D.$\frac{1}{4}$

 解析:$\because b_2^2=1\times 4 = 4$,\textit{$\therefore$b}${}_{2}$\textit{=}2或\textit{b}${}_{2}$\textit{=-}2(舍去)\textit{.}

又$a_2-a_1=\frac{4-1}{4-1}=1,\therefore \frac{a_1-a_2}{b_2}=\frac{-1}{2}=-\frac{1}{2}$

 答案:A \\

知识点:等比数列的概念

难度:1

 若互不相等的实数\textit{a},\textit{b},\textit{c}成等差数列,\textit{c},\textit{a},\textit{b}成等比数列,且$a+3b+c=10$,则\textit{a}等于(\textit{  })

 A.4 B.2 C.\textit{-}2 D.\textit{-}4

 解析:由$\left\{
\begin{array}{l}
2b=a+c, \\
a^2=bc, \\
a+3b+c=10
\end{array}
\right.$, 解得\textit{a=-}4或\textit{a=}2\textit{.}

又当\textit{a=}2时,\textit{b=}2,\textit{c=}2,与题意不符,故\textit{a=-}4\textit{.}

 答案:D \\

知识点:等比数列的概念

难度:1

 在等比数列$\mathrm{\{}$\textit{a${}_{n}$}$\mathrm{\}}$中,\textit{a}${}_{1}$\textit{=}1,公比\textit{{\textbar}q{\textbar}}$\mathrm{\neq}$1\textit{.}若\textit{a${}_{m}$=a}${}_{1}$\textit{a}${}_{2}$\textit{a}${}_{3}$\textit{a}${}_{4}$\textit{a}${}_{5}$,则\textit{m=}(\textit{  })

 A.9 B.10 C.11 D.12

 解析:因为$\mathrm{\{}$\textit{a${}_{n}$}$\mathrm{\}}$是等比数列,所以$a_1a_5=a_2a_4=a_3^2$,

于是$a_1a_2a_3a_4a_5=a_3^5$

从而$a_m=a_3^5=(q^2)^5=q^{10}=1\times q^{11-1}$,故\textit{m=}11\textit{.}

 答案:C \\

知识点:等比数列的概念

难度:1

 在正项等比数列$\mathrm{\{}$\textit{a${}_{n}$}$\mathrm{\}}$中,$\frac{1}{a_2a_4}+\frac{2}{a_4^2}+\frac{1}{a_4a_6}=81$,则$\frac{1}{a_3}+\frac{1}{a_5}$等于(\textit{  })

 A.$\frac{1}{9}$ B.3 C.6 D.9

 解析:$\because \frac{1}{a_2a_4}+\frac{2}{a_4^2}+\frac{1}{a_4a_6}=81$,

$\therefore \frac{1}{a_3^2}+\frac{1}{a_3a_5}+\frac{1}{a_5^2}=81, \therefore (\frac{1}{a_3}+\frac{1}{a_5})^2=81$

\textit{$\because$}数列各项都是正数,$\therefore \frac{1}{a_3}+\frac{1}{a_5}=9$

 答案:D \\

知识点:等比数列的概念

难度:1

 在等差数列$\mathrm{\{}$\textit{a${}_{n}$}$\mathrm{\}}$中,公差\textit{d}$\mathrm{\neq}$0,且\textit{a}${}_{1}$,\textit{a}${}_{3}$,\textit{a}${}_{9}$成等比数列,则$\frac{a_1+a_3+a_9}{a_2+a_4+a_{10}}$\textit{=\underbar{    }.~}

 解析:由题意知\textit{a}${}_{3}$是\textit{a}${}_{1}$和\textit{a}${}_{9}$的等比中项,

$a_3^2=a_1a_9, \therefore (a_1+2d)^2=a_1(a_1+8d)$,得\textit{a}${}_{1}$\textit{=d},

$\therefore \frac{a_1+a_3+a_9}{a_2+a_4+a_{10}}=\frac{13d}{16d}=\frac{13}{16}$

 答案:$\frac{13}{16}$ \\

知识点:等比数列的概念

难度:1

 在1和100之间插入\textit{n}个正数,使这$n+2$个数成等比数列,则插入的这\textit{n}个正数的积为\textit{\underbar{     }.~}

 解析:设插入的\textit{n}个正数为\textit{a}${}_{1}$,\textit{a}${}_{2}$,{\dots},\textit{a${}_{n}$.}

设\textit{M=}1·\textit{a}${}_{1}$·\textit{a}${}_{2}$·{\dots}·\textit{a${}_{n}$}·100,则

\textit{M=}100·\textit{a${}_{n}$}·\textit{a${}_{n-}$}${}_{1}$·{\dots}·\textit{a}${}_{1}$·1,

\textit{$\therefore$M}${}^{2}$\textit{=}(1\textit{$\times$}100)\textit{${}^{n+}$}${}^{2}$\textit{=}100\textit{${}^{n+}$}${}^{2}$,$\therefore 100^{\frac{n+2}{2}}=10^{n+2}$,

\textit{$\therefore$a}${}_{1}$·\textit{a}${}_{2}$·{\dots}·\textit{a${}_{n}$=}10\textit{${}^{n}$.}

 答案:10\textit{${}^{n}$} \\

知识点:等比数列的概念

难度:1

 在表格中,每格填上一个数字后,使每一横行成等差数列,每一纵行成等比数列,所有公比相等,则\textit{a+b+c}的值为\textit{\underbar{     }.~}

\begin{tabular}{|p{0.7in}|p{0.1in}|p{0.7in}|p{0.1in}|p{0.8in}|} \hline 
\textit{a} &  &  &  &  \\ \hline 
 &  & \textit{b} &  & 6 \\ \hline 
1 &  & 2 &  &  \\ \hline 
 &  &  &  & \textit{c} \\ \hline 
\end{tabular}

 

 解析:设公比为\textit{q},由题意知$q=\frac{2}{b},q^2=\frac{c}{6}$

第四行最后一个数为$\frac{c}{q}=\frac{c}{\frac{2}{b}}=\frac{bc}{2}$

因为每一行成等差数列,所以$2\times 2=1+\frac{bc}{2}$,即\textit{bc=}6\textit{.}

因为$\frac{4}{b^2}=\frac{c}{6}$,所以$\left\{
\begin{array}{l}
bc=6, \\
b^2 c=24.
\end{array}
\right.$

所以$\left\{
\begin{array}{l}
b=4, \\
c=\frac{3}{2}
\end{array}
\right.$,所以$q=\frac{2}{b}=\frac{2}{4}=\frac{1}{2}$

又$\frac{1}{a}=q^3=(\frac{1}{2})^3$,所以$a=8,a+b+c=\frac{27}{2}$

 答案:$\frac{27}{2}$ \\

知识点:等比数列的概念

难度:1

 三个互不相等的实数成等差数列,如果适当排列这三个数,又可成为等比数列,且这三个数的和为6,求这三个数\textit{.}

解析:

 答案:由题意,这三个数成等差数列,可设这三个数分别为\textit{a-d},\textit{a},\textit{a+d}(\textit{d}$\mathrm{\neq}$0),\textit{$\therefore$a-d+a+a+d=}6,\textit{$\therefore$a=}2,

\textit{$\therefore$}这三个数分别为2\textit{-d},2,2\textit{+d.}

若2\textit{-d}为等比中项,则有(2\textit{-d})${}^{2}$\textit{=}2(2\textit{+d})\textit{.}

解得\textit{d=}6或\textit{d=}0(舍去),

此时三个数分别为\textit{-}4,2,8;

若2\textit{+d}是等比中项,则有(2\textit{+d})${}^{2}$\textit{=}2(2\textit{-d}),

解得\textit{d=-}6或\textit{d=}0(舍去),此时三个数分别为8,2,\textit{-}4\textit{.} \\

知识点:等比数列的概念

难度:1

 已知等比数列$\mathrm{\{}$\textit{b${}_{n}$}$\mathrm{\}}$与数列$\mathrm{\{}$\textit{a${}_{n}$}$\mathrm{\}}$满足$b_n=3^{a_n}$(\textit{n}$\mathrm{\in}$N${}_{+}$)\textit{.}

 (1)判断$\mathrm{\{}$\textit{a${}_{n}$}$\mathrm{\}}$是何种数列;

 (2)若$a_8+a_{13}=m$,求\textit{b}${}_{1}$·\textit{b}${}_{2}$·{\dots}·\textit{b}${}_{20}$\textit{.}

解析:

 答案:(1)设数列$\mathrm{\{}$\textit{b${}_{n}$}$\mathrm{\}}$的公比为\textit{q},则\textit{q$>$}0\textit{.}

$\because b_n=3^{a_n}, \therefore b_1=3^{a_1}$,

$\therefore b_n=3^{a_1}\cdot q^{n-1},\therefore 3^{a_1}\cdot q^{n-1}=3^{a_n}$\textit{. ①}

将两边取以3为底的对数得$a_n=\log(3^{a_1}\cdot q^{n-1})=a_1+(n-1)\log_3q=\log_3 b_1+(n-1)\log_3 q$

\textit{$\therefore$}数列$\mathrm{\{}$\textit{a${}_{n}$}$\mathrm{\}}$是以log${}_{3}$\textit{b}${}_{1}$为首项,log${}_{3}$\textit{q}为公差的等差数列\textit{.}

(2)$\because a_1+a_{20}=a_8+a_{13}=m$,

$\therefore a_1+a_2+\cdots +a_{20}=\frac{20(a_1+a_{20})}{2}=10m$,

$\therefore b_1\cdot b_2\cdot \cdots \cdot b_{20}=3^{a_1}\cdot 3^{a_2}\cdot \cdots \cdot 3^{a_{20}} = 3^{a_1+a_2+\cdot +a_{20}}=3^{10m}$ \\

知识点:等比数列的概念

难度:2

 已知0\textit{$<$a$<$b$<$c},且\textit{a},\textit{b},\textit{c}成等比数列,\textit{n}为大于1的整数,则log\textit{${}_{a}$n},log\textit{${}_{b}$n},log\textit{${}_{c}$n}(\textit{  })

 A.成等差数列 B.成等比数列

 C.各项倒数成等差数列 D.以上都不对

 解析:\textit{$\because$a},\textit{b},\textit{c}成等比数列,\textit{$\therefore$b}${}^{2}$\textit{=ac},又$\frac{1}{\log_an}+\frac{1}{\log_c n}=\log_na+\log_nc=\log_nac=\log_nb^2=2\log_nb=\frac{2}{\log_b n}$,

\textit{$\therefore$}log\textit{${}_{a}$n},log\textit{${}_{b}$n},log\textit{${}_{c}$n}的各项倒数成等差数列\textit{.}

故选C\textit{.}

 答案:C \\

知识点:等比数列的概念

难度:2

 一个等比数列的前三项的积为3,最后三项的积为9,且所有项的积为729,则该数列的项数是(\textit{  })

 A.13 B.12 C.11 D.10

 解析:设该等比数列为$\mathrm{\{}$\textit{a${}_{n}$}$\mathrm{\}}$,其前\textit{n}项积为\textit{T${}_{n}$},则由已知得\textit{a}${}_{1}$·\textit{a}${}_{2}$·\textit{a}${}_{3}$\textit{=}3,\textit{a${}_{n-}$}${}_{2}$·\textit{a${}_{n-}$}${}_{1}$·\textit{a${}_{n}$=}9,(\textit{a}${}_{1}$·\textit{a${}_{n}$})${}^{3}$\textit{=}3\textit{$\times$}9\textit{=}3${}^{3}$,\textit{$\therefore$a}${}_{1}$·\textit{a${}_{n}$=}3,

又\textit{T${}_{n}$=a}${}_{1}$·\textit{a}${}_{2}$·{\dots}·\textit{a${}_{n-}$}${}_{1}$·\textit{a${}_{n}$},\textit{T${}_{n}$=a${}_{n}$}·\textit{a${}_{n-}$}${}_{1}$·{\dots}·\textit{a}${}_{2}$·\textit{a}${}_{1}$,

$\therefore T_n^2=(a_1\cdot a_2)^n$,即729${}^{2}$\textit{=}3\textit{${}^{n}$},\textit{$\therefore$n=}12\textit{.}

 答案:B \\

知识点:等比数列的概念

难度:2

 在等比数列$\mathrm{\{}$\textit{a${}_{n}$}$\mathrm{\}}$中,$|a_1|=1,a_5=-8a_2$,且\textit{a}${}_{5}$\textit{$>$a}${}_{2}$,则\textit{a${}_{n}$}等于(\textit{  })

 A.(\textit{-}2)\textit{${}^{n-}$}${}^{1}$ B.\textit{-}(\textit{-}2)\textit{${}^{n-}$}${}^{1}$

 C.\textit{$\pm$}(\textit{-}2)\textit{${}^{n-}$}${}^{1}$ D.\textit{-}(\textit{-}2)\textit{${}^{n}$}

 解析:$\because |a_1|=1,\therefore a_1=1$或\textit{a}${}_{1}$\textit{=-}1\textit{.}

\textit{$\because$a}${}_{5}$\textit{=-}8\textit{a}${}_{2}$\textit{=a}${}_{2}$·\textit{q}${}^{3}$,\textit{$\therefore$q}${}^{3}$\textit{=-}8,\textit{$\therefore$q=-}2\textit{.}

又\textit{a}${}_{5}$\textit{$>$a}${}_{2}$,即\textit{a}${}_{2}$\textit{q}${}^{3}$\textit{$>$a}${}_{2}$,\textit{$\therefore$a}${}_{2}$\textit{$<$}0\textit{.}

而\textit{a}${}_{2}$\textit{=a}${}_{1}$\textit{q=a}${}_{1}$·(\textit{-}2)\textit{$<$}0,\textit{$\therefore$a}${}_{1}$\textit{=}1\textit{.}

故\textit{a${}_{n}$=a}${}_{1}$·(\textit{-}2)\textit{${}^{n-}$}${}^{1}$\textit{=}(\textit{-}2)\textit{${}^{n-}$}${}^{1}$\textit{.}

 答案:A \\

知识点:等比数列的概念

难度:2

 已知等比数列$\mathrm{\{}$\textit{a${}_{n}$}$\mathrm{\}}$满足\textit{a${}_{n}$$>$}0,\textit{n=}1,2,{\dots},且\textit{a}${}_{5}$·\textit{a}${}_{2}$\textit{${}_{n-}$}${}_{5}$\textit{=}2${}^{2}$\textit{${}^{n}$}(\textit{n}$\mathrm{\ge}$3),则当\textit{n}$\mathrm{\ge}$1时,$\log_2a_1+\log_2a_3+\cdots +\log_2a_{2n-1}$\textit{=}(\textit{  })

 A.\textit{n}(2\textit{n-}1) \textit{ }B.$(n+1)^2$C.\textit{n}${}^{2}$ D.(\textit{n-}1)${}^{2}$

 解析:由等比数列的性质可得$a_n^2=a_5\cdot a_{2n-5}=2^{2n}=(2^n)^2$,

\textit{$\because$a${}_{n}$$>$}0,\textit{$\therefore$a${}_{n}$=}2\textit{${}^{n}$},故数列首项\textit{a}${}_{1}$\textit{=}2,公比\textit{q=}2,

故$\log_2 a_1+\log_2a_3+\cdots +\log_2a_{2n-1}=\log_2(a_1\cdot a_3\cdot \cdots 
\cdot a_{2n-1})=\log_2[(a_1)^nq^{0+2+4+\cdots+2n-2}]$

$=\log_2[2^n\cdot 2^{\frac{n(0+2n-2)}{2}}]=\log_2(2^{n+n^2-n})=\log_2(2^{n^2})=n^2$,故选C\textit{.}

 答案:C \\

知识点:等比数列的概念

难度:2

 在数列$\mathrm{\{}$\textit{a${}_{n}$}$\mathrm{\}}$中,\textit{a}${}_{1}$\textit{=}2,当\textit{n}为奇数时,$a_{n+1}=a_n+2$;当\textit{n}为偶数时,$a_{n+1}=2a_{n-1}$,则\textit{a}${}_{12}$\textit{=}(\textit{  })

 A.32 C.34 C.66 D.64

 解析:依题意,\textit{a}${}_{1}$,\textit{a}${}_{3}$,\textit{a}${}_{5}$,\textit{a}${}_{7}$,\textit{a}${}_{9}$,\textit{a}${}_{11}$构成以2为首项,2为公比的等比数列,故\textit{a}${}_{11}$\textit{=a}${}_{1}$\textit{$\times$}2${}^{5}$\textit{=}64,\textit{a}${}_{12}$\textit{=a}${}_{11}$\textit{+}2\textit{=}66\textit{.}故选C\textit{.}

 答案:C \\

知识点:等比数列的概念

难度:2

 在等比数列$\mathrm{\{}$\textit{a${}_{n}$}$\mathrm{\}}$中,已知\textit{a}${}_{9}$\textit{=-}2,则此数列的前17项之积为\textit{\underbar{     }.~}

 解析:$\because a_1a_2a_3\cdots a_{17}=(a_1\cdot a_{17})(a_2\cdot a_{16})\cdots a_9=a_9^2\cdot a_9^2\cdots a_9=a_9^{17}=(-2)^{17}=-2^{17}$

 答案:$-2^{17}$ \\

知识点:等比数列的概念

难度:2

 已知数列$\mathrm{\{}$\textit{a${}_{n}$}$\mathrm{\}}$是公差不为零的等差数列,且\textit{a}${}_{5}$,\textit{a}${}_{8}$,\textit{a}${}_{13}$是等比数列$\mathrm{\{}$\textit{b${}_{n}$}$\mathrm{\}}$中相邻的三项,若\textit{b}${}_{2}$\textit{=}5,求数列$\mathrm{\{}$\textit{b${}_{n}$}$\mathrm{\}}$的通项公式\textit{.}

解析:

 答案:\textit{$\because$}$\mathrm{\{}$\textit{a${}_{n}$}$\mathrm{\}}$是等差数列,

$\therefore a_5=a_1+4d,a_8=a_1+7d,a_{13}=a_1+12d$

\textit{$\because$a}${}_{5}$,\textit{a}${}_{8}$,\textit{a}${}_{13}$是等比数列$\mathrm{\{}$\textit{b${}_{n}$}$\mathrm{\}}$中相邻的三项,

$\therefore a_8^2=a_5a_{13}$,即(\textit{a}${}_{1}$\textit{+}7\textit{d})${}^{2}$\textit{=}(\textit{a}${}_{1}$\textit{+}4\textit{d})(\textit{a}${}_{1}$\textit{+}12\textit{d}),

解得\textit{d=}2\textit{a}${}_{1}$\textit{.}

$\therefore q=\frac{a_8}{a_5}\frac{5}{3},b_2=b_1q=5,\frac{5}{3}b_1=5,b_1=3$,

$\therefore b_n=3\cdot (\frac{5}{3})^{n-1}$ \\

知识点:等比数列的概念

难度:2

 已知两个等比数列$\mathrm{\{}$\textit{a${}_{n}$}$\mathrm{\}}$,$\mathrm{\{}$\textit{b${}_{n}$}$\mathrm{\}}$满足\textit{a}${}_{1}$\textit{=a}(\textit{a$>$}0),\textit{b}${}_{1}$\textit{-a}${}_{1}$\textit{=}1,\textit{b}${}_{2}$\textit{-a}${}_{2}$\textit{=}2,\textit{b}${}_{3}$\textit{-a}${}_{3}$\textit{=}3\textit{.}

 (1)若\textit{a=}1,求数列$\mathrm{\{}$\textit{a${}_{n}$}$\mathrm{\}}$的通项公式;

 (2)若数列$\mathrm{\{}$\textit{a${}_{n}$}$\mathrm{\}}$唯一,求\textit{a}的值\textit{.}

解析:

 答案:(1)设$\mathrm{\{}$\textit{a${}_{n}$}$\mathrm{\}}$的公比为\textit{q},则$b_1=1+a_1=1+a=2,b_2=2+aq=2+q,b_3=3+aq^2=3+q^2$

由\textit{b}${}_{1}$,\textit{b}${}_{2}$,\textit{b}${}_{3}$成等比数列,得$(2+q)^2=2(3+q^2)$,即$q^2-4q+2=0$,解得$q_1=2+\sqrt{2},q_2=2-\sqrt{2}$,

故$\mathrm{\{}$\textit{a${}_{n}$}$\mathrm{\}}$的通项公式为$a_n=(2+\sqrt{2})^{n-1}$或$a_n=(2-\sqrt{2})^{n-1}$

(2)设$\mathrm{\{}$\textit{a${}_{n}$}$\mathrm{\}}$的公比为\textit{q},则由(2\textit{+aq})${}^{2}$\textit{=}(1\textit{+a})·(3\textit{+aq}${}^{2}$),得\textit{aq}${}^{2}$\textit{-}4\textit{aq+}3\textit{a-}1\textit{=}0,由\textit{a$>$}0得,$\Delta$\textit{=}4\textit{a}${}^{2}$\textit{+}4\textit{a$>$}0,故方程\textit{aq}${}^{2}$\textit{-}4\textit{aq+}3\textit{a-}1\textit{=}0有两个不同的实根\textit{.}又$\mathrm{\{}$\textit{a${}_{n}$}$\mathrm{\}}$唯一,故方程必有一根为0,代入上式得$a=\frac{1}{3}$ \\

知识点:等比数列的前n项和

难度:1

 设$\mathrm{\{}$\textit{a${}_{n}$}$\mathrm{\}}$是公比为正数的等比数列,若\textit{a}${}_{1}$\textit{=}1,\textit{a}${}_{5}$\textit{=}16,则数列$\mathrm{\{}$\textit{a${}_{n}$}$\mathrm{\}}$前7项的和为(\textit{  })

 \textit{                }

 A.63 B.64 C.127 D.128

 解析:设公比为\textit{q}(\textit{q$>$}0),则1·\textit{q}${}^{4}$\textit{=}16,解得\textit{q=}2(\textit{q=-}2舍去)\textit{.}于是$S_7=\frac{1-2^7}{1-2}=127$

 答案:C \\

知识点:等比数列的前n项和

难度:1

 设\textit{S${}_{n}$}为等比数列$\mathrm{\{}$\textit{a${}_{n}$}$\mathrm{\}}$的前\textit{n}项和,已知3\textit{S}${}_{3}$\textit{=a}${}_{4}$\textit{-}2,3\textit{S}${}_{2}$\textit{=a}${}_{3}$\textit{-}2,则公比\textit{q}等于(\textit{  })

 A.3 B.4 C.5 D.6

 解析:由题意知,$\left\{
\begin{array}{l}
3S_3=a_4-2, \\
3S_2=a_2-2
\end{array}
\right.$

两式相减,得3\textit{a}${}_{3}$\textit{=a}${}_{4}$\textit{-a}${}_{3}$,

即4\textit{a}${}_{3}$\textit{=a}${}_{4}$,则$q=\frac{a_4}{a_3}=4$

 答案:B \\

知识点:等比数列的前n项和

难度:1

 若数列$\mathrm{\{}$\textit{a${}_{n}$}$\mathrm{\}}$的前\textit{n}项和\textit{S${}_{n}$=a${}^{n}$-}1(\textit{a}$\mathrm{\in}$R,且\textit{a}$\mathrm{\neq}$0),则此数列是(\textit{  })

 A.等差数列

 B.等比数列

 C.等差数列或等比数列

 D.既不是等差数列,也不是等比数列

 解析:当\textit{n=}1时,\textit{a}${}_{1}$\textit{=S}${}_{1}$\textit{=a-}1;

当\textit{n}$\mathrm{\ge}$2时,\textit{a${}_{n}$=S${}_{n}$-S${}_{n-}$}${}_{1}$\textit{=}(\textit{a${}^{n}$-}1)\textit{-}(\textit{a${}^{n-}$}${}^{1}$\textit{-}1)

\textit{=a${}^{n}$-a${}^{n-}$}${}^{1}$\textit{=a${}^{n-}$}${}^{1}$(\textit{a-}1)\textit{.}

当\textit{a-}1\textit{=}0,即\textit{a=}1时,该数列为等差数列,当\textit{a}$\mathrm{\neq}$1时,该数列为等比数列\textit{.}

 答案:C \\

知识点:等比数列的前n项和

难度:1

 公比\textit{q}$\mathrm{\neq}$\textit{-}1的等比数列的前3项,前6项,前9项的和分别为\textit{S}${}_{3}$,\textit{S}${}_{6}$,\textit{S}${}_{9}$,则下面等式成立的是(\textit{  })

 A.$S_3+S_6=S_9$ B.$S_6^2=S_3\cdot S_9$

 C.$S_3+S_6-S_9=S_6^2$ D.$S_3^2+S_6^2=S_3(S_6+S_9)$

 解析:由题意知\textit{S}${}_{3}$,\textit{S}${}_{6}$\textit{-S}${}_{3}$,\textit{S}${}_{9}$\textit{-S}${}_{6}$也成等比数列\textit{.}

\textit{$\therefore$}(\textit{S}${}_{6}$\textit{-S}${}_{3}$)${}^{2}$\textit{=S}${}_{3}$(\textit{S}${}_{9}$\textit{-S}${}_{6}$),

整理得$S_3^2+S_6^2=S_3(S_6+S_9)$

 答案:D \\

知识点:等比数列的前n项和

难度:1

 已知$\mathrm{\{}$\textit{a${}_{n}$}$\mathrm{\}}$是首项为1的等比数列,\textit{S${}_{n}$}是$\mathrm{\{}$\textit{a${}_{n}$}$\mathrm{\}}$的前\textit{n}项和,且9\textit{S}${}_{3}$\textit{=S}${}_{6}$,则数列$\{\frac{1}{a_n}\}$的前5项和为(\textit{  })

 A.$\frac{15}{8}$或5 B.$\frac{31}{16}$或5 C.$\frac{31}{16}$ D.$\frac{15}{8}$

 解析:设$\mathrm{\{}$\textit{a${}_{n}$}$\mathrm{\}}$的公比为\textit{q.}由9\textit{S}${}_{3}$\textit{=S}${}_{6}$知\textit{q}$\mathrm{\neq}$1,

于是$\frac{9a_1(1-q^3)}{1-q}=\frac{a_1(1-q^6)}{1-q}$,整理得$q^6-9q^3+8=0$,所以$q^3=8$或$q^3=1$(舍去),于是\textit{q=}2\textit{.}

从而$\{\frac{1}{a_n}\}$是首项为$\frac{1}{1}=1$,公比为$\frac{1}{2}$的等比数列\textit{.}

其前5项的和$S=\frac{1-(\frac{1}{2})^5}{1-\frac{1}{2}}=\frac{31}{16}$

 答案:C \\

知识点:等比数列的前n项和

难度:1

 设等比数列$\mathrm{\{}$\textit{a${}_{n}$}$\mathrm{\}}$的前\textit{n}项和为\textit{S${}_{n}$},若\textit{a}${}_{1}$\textit{=}1,\textit{S}${}_{6}$\textit{=}4\textit{S}${}_{3}$,则\textit{a}${}_{4}$\textit{=\underbar{     }.~}

 解析:设等比数列$\mathrm{\{}$\textit{a${}_{n}$}$\mathrm{\}}$的公比为\textit{q},很明显$q\ne 1$,则$\frac{1-q^6}{1-q}=4\cdot \frac{1-q^3}{1-q}$,解得\textit{q}${}^{3}$\textit{=}3,所以\textit{a}${}_{4}$\textit{=a}${}_{1}$\textit{q}${}^{3}$\textit{=}3\textit{.}

 答案:3 \\

知识点:等比数列的前n项和

难度:1

 已知$\lg x+\lg x^2+\cdots +\lg x^{10}=110$,则$\lg x+\lg^2 x+\cdots +\lg^{10}x$\textit{x=\underbar{      }.~}

 答案:2046 \\

知识点:等比数列的前n项和

难度:1

 已知在等比数列$\mathrm{\{}$\textit{a${}_{n}$}$\mathrm{\}}$中,$a_2=2,a_5=\frac{1}{4}$,则$a_1a_2+a_2a_3+\cdots +a_na_{n+1}$\textit{=\underbar{          }.~}

 解析:设数列$\mathrm{\{}$\textit{a${}_{n}$}$\mathrm{\}}$的公比为\textit{q},由$a_2=2,a_5=a_2q^3=\frac{1}{4}$,

得$q=\frac{1}{2},\therefore a_1=\frac{a_2}{q}=4$

$\because \frac{a_na_{n+1}}{a_{n-1}a_n}=\frac{a_{n+1}}{a_{n-1}}=\frac{a_{n-1}q^2}{a_{n-1}}=q^2=\frac{1}{4}$为常数(\textit{n}$\mathrm{\ge}$2),

\textit{$\therefore$}数列$\mathrm{\{}$\textit{a${}_{n}$a${}_{n+}$}${}_{1}$$\mathrm{\}}$是以\textit{a}${}_{1}$\textit{a}${}_{2}$\textit{=}4\textit{$\times$}2\textit{=}8为首项,以$\frac{1}{4}$为公比的等比数列,

$\therefore a_1a_2+a_2a_3+\cdots +a_na_{n+1}$

$=\frac{8\times [1-(\frac{1}{4})^n]}{1-\frac{1}{4}}=\frac{32}{3}(1-4^{-n})$

 答案:$\frac{32}{3}(1-4^{-n})$ \\

知识点:等比数列的前n项和

难度:1

 (2017北京高考)已知等差数列$\mathrm{\{}$\textit{a${}_{n}$}$\mathrm{\}}$和等比数列$\mathrm{\{}$\textit{b${}_{n}$}$\mathrm{\}}$满足$a_1=b_1=1,a_2+a_4=1-,b_2b_4=a_5$

 (1)求$\mathrm{\{}$\textit{a${}_{n}$}$\mathrm{\}}$的通项公式;

 (2)求和:$b_1+b_3+b_5+\cdots+b_{2n-1}$

解析:

 答案:(1)设等差数列$\mathrm{\{}$\textit{a${}_{n}$}$\mathrm{\}}$的公差为\textit{d.}

因为$a_2+a_4=10$,所以$2a_1+4d=10$

解得\textit{d=}2\textit{.}所以\textit{a${}_{n}$=}2\textit{n-}1\textit{.}

(2)设等比数列$\mathrm{\{}$\textit{b${}_{n}$}$\mathrm{\}}$的公比为\textit{q.}

因为\textit{b}${}_{2}$\textit{b}${}_{4}$\textit{=a}${}_{5}$,所以\textit{b}${}_{1}$\textit{qb}${}_{1}$\textit{q}${}^{3}$\textit{=}9\textit{.}

解得\textit{q}${}^{2}$\textit{=}3\textit{.}所以\textit{b}${}_{2}$\textit{${}_{n-}$}${}_{1}$\textit{=b}${}_{1}$\textit{q}${}^{2}$\textit{${}^{n-}$}${}^{2}$\textit{=}3\textit{${}^{n-}$}${}^{1}$\textit{.}

从而$b_1+b_3+b_5+\cdots+b_{2n-1}=1+3+3^2+\cdots+3^{n-1}=\frac{3^n-1}{2}$ \\

知识点:等比数列的前n项和

难度:1

 已知等差数列$\mathrm{\{}$\textit{a${}_{n}$}$\mathrm{\}}$满足\textit{a${}_{n+}$}${}_{1}$\textit{$>$a${}_{n}$}(\textit{n}$\mathrm{\in}$N\textit{${}_{+}$}),\textit{a}${}_{1}$\textit{=}1,该数列的前三项分别加上1,1,3后顺次成为等比数列$\mathrm{\{}$\textit{b${}_{n}$}$\mathrm{\}}$的前三项\textit{.}

 (1)求数列$\mathrm{\{}$\textit{a${}_{n}$}$\mathrm{\}}$,$\mathrm{\{}$\textit{b${}_{n}$}$\mathrm{\}}$的通项公式;

 (2)设$T_n=\frac{a_1}{b_1}+\frac{a_2}{b_2}+\cdots+\frac{a_n}{b_n}$(\textit{n}$\mathrm{\in}$N\textit{${}_{+}$}),求\textit{T${}_{n}$.}

解析:

 答案:(1)设\textit{d},\textit{q}分别为等差数列$\mathrm{\{}$\textit{a${}_{n}$}$\mathrm{\}}$的公差、等比数列$\mathrm{\{}$\textit{b${}_{n}$}$\mathrm{\}}$的公比,由题意知,$a_1=1,a_2=1+d,a_3=1+2d$,分别加上1,1,3得$2,2+d,4+2d$,

$\therefore (2+d)^2=2(4+2d), \therefore d=\pm 2$

$\because a_{n+1}>a_n, \therefore d>0, \therefore d=2$

\textit{$\therefore$a${}_{n}$=}2\textit{n-}1(\textit{n}$\mathrm{\in}$N\textit{${}_{+}$})\textit{.}由此可得\textit{b}${}_{1}$\textit{=}2,\textit{b}${}_{2}$\textit{=}4,\textit{b}${}_{3}$\textit{=}8,\textit{$\therefore$q=}2\textit{.$\therefore$b${}_{n}$=}2\textit{${}^{n}$}(\textit{n}$\mathrm{\in}$N\textit{${}_{+}$})\textit{.}

(2)$\because T_n=\frac{a_1}{b_1}+\frac{a_2}{b_2}+\cdots+\frac{a_n}{b_n}$

$=\frac{1}{2}+\frac{3}{2^2}+\frac{5}{2^3}+\cdots+\frac{2n-1}{2^n}$,\textit{ ①}

$\therefore \frac{1}{2}T_n=\frac{1}{2^2}+\frac{3}{2^3}+\frac{5}{2^4}+\cdots+\frac{2n-1}{2^{n+1}}$,\textit{ ②}

由\textit{①-②}得$\frac{1}{2}T_n=\frac{1}{2}+\frac{1}{2}+\frac{1}{2^2}+\frac{1}{2^3}+\cdots+\frac{1}{2^{n-1}}-\frac{2n-1}{2^{n+1}}$,

$\therefore T_n=1+\frac{1-\frac{1}{2^{n-1}}}{1-\frac{1}{2}}-\frac{2n-1}{2^n}$

$=3-\frac{1}{2^{n-2}}=3-\frac{2n+3}{2^n}$ \\

知识点:等比数列的前n项和

难度:2

 已知等比数列$\mathrm{\{}$\textit{a${}_{n}$}$\mathrm{\}}$的前\textit{n}项和为\textit{S${}_{n}$},则下列一定成立的是(\textit{  })

 A.若\textit{a}${}_{3}$\textit{$>$}0,则\textit{a}${}_{2017}$\textit{$<$}0 B.若\textit{a}${}_{4}$\textit{$>$}0,则\textit{a}${}_{2016}$\textit{$<$}0

 C.若\textit{a}${}_{3}$\textit{$>$}0,则\textit{S}${}_{2017}$\textit{$>$}0 D.若\textit{a}${}_{4}$\textit{$>$}0,则\textit{S}${}_{2016}$\textit{$>$}0

 解析:若\textit{a}${}_{3}$\textit{$>$}0,则\textit{a}${}_{3}$\textit{=a}${}_{1}$\textit{q}${}^{2}$\textit{$>$}0,因此\textit{a}${}_{1}$\textit{$>$}0,当公比\textit{q$>$}0时,任意\textit{n}$\mathrm{\in}$N\textit{${}_{+}$},\textit{a${}_{n}$$>$}0,故有\textit{S}${}_{2017}$\textit{$>$}0,当公比\textit{q$<$}0时,\textit{q}${}^{2017}$\textit{$<$}0,则$S_{2017}=\frac{a_1(1-q^{2017})}{1-q}>0$,故答案为C\textit{.}

 答案:C \\

知识点:等比数列的前n项和

难度:2

 已知数列前\textit{n}项的和\textit{S${}_{n}$=}2\textit{${}^{n}$-}1,则此数列奇数项的前\textit{n}项的和是(\textit{  })

 A.$\frac{1}{3}(2^{n+1}-1)$ B.$\frac{1}{3}(2^{n+1}-2)$

 C.$\frac{1}{3}(2^{2n}-1)$ D.$\frac{1}{3}(2^{2n}-2)$

 解析:由\textit{S${}_{n}$=}2\textit{${}^{n}$-}1知当\textit{n=}1时,\textit{a}${}_{1}$\textit{=}2${}^{1}$\textit{-}1\textit{=}1\textit{.}

当\textit{n}$\mathrm{\ge}$2时,\textit{a${}_{n}$=S${}_{n}$-S${}_{n-}$}${}_{1}$\textit{=}2\textit{${}^{n-}$}${}^{1}$,当\textit{n=}1时也适合,

\textit{$\therefore$a${}_{n}$=}2\textit{${}^{n-}$}${}^{1}$\textit{.}

\textit{$\therefore$}奇数项的前\textit{n}项和为$S_n=\frac{1-4^n}{1-4}=\frac{1}{3}(4^n-1)=\frac{1}{3}(2^{2n}-1)$

 答案:C \\

知识点:等比数列的前n项和

难度:2

 等比数列$\mathrm{\{}$\textit{a${}_{n}$}$\mathrm{\}}$的前\textit{n}项和为\textit{S${}_{n}$},已知\textit{S}${}_{1}$,2\textit{S}${}_{2}$,3\textit{S}${}_{3}$成等差数列,则数列$\mathrm{\{}$\textit{a${}_{n}$}$\mathrm{\}}$的公比为\textit{\underbar{     }.~}

 解析:由\textit{S}${}_{1}$,2\textit{S}${}_{2}$,3\textit{S}${}_{3}$成等差数列知4\textit{S}${}_{2}$\textit{=S}${}_{1}$\textit{+}3\textit{S}${}_{3}$,

即$4(a_1+a_2)=a_1+3(a_1+a_2+a_3)$,整理得3\textit{a}${}_{3}$\textit{-a}${}_{2}$\textit{=}0,$\therefore \frac{a_3}{a_2}=\frac{1}{3}$,则数列$\mathrm{\{}$\textit{a${}_{n}$}$\mathrm{\}}$的公比为$\frac{1}{3}$

 答案:$\frac{1}{3}$ \\

知识点:等比数列的前n项和

难度:2

 设数列$\mathrm{\{}$\textit{x${}_{n}$}$\mathrm{\}}$满足$\lg x^{n+1}=1+\lg x_n$(\textit{n}$\mathrm{\in}$N\textit{${}_{+}$}),且$x_1+x_2+\cdots +x_{100}=100$,则$x_{101}+x_{102}+\cdots+x_{200}$\textit{=\underbar{     }.~}

 解析:由$\lg x_{n+1}=1+\lg x_n$,

得lg \textit{x${}_{n+}$}${}_{1}$\textit{=}lg(10\textit{x${}_{n}$}),即$\frac{x_{n+1}}{x_n}=10$

故$x_{101}+x_{102}+\cdots+x_{200}=q^{100}(x_1+x_2+\cdots+x_{100})=10^{100}\times 100=10^{102}$

 答案:$10^{102}$ \\

知识点:等比数列的前n项和

难度:2

 已知等比数列$\mathrm{\{}$\textit{a${}_{n}$}$\mathrm{\}}$是递增数列,\textit{S${}_{n}$}是$\mathrm{\{}$\textit{a${}_{n}$}$\mathrm{\}}$的前\textit{n}项和\textit{.}若\textit{a}${}_{1}$,\textit{a}${}_{3}$是方程$x^2-5x+4=0$的两个根,则\textit{S}${}_{6}$\textit{=\underbar{     }.~}

 解析:\textit{$\because$x}${}^{2}$\textit{-}5\textit{x+}4\textit{=}0的两根为1和4,

又$\mathrm{\{}$\textit{a${}_{n}$}$\mathrm{\}}$为递增数列,\textit{$\therefore$a}${}_{1}$\textit{=}1,\textit{a}${}_{3}$\textit{=}4,\textit{q=}2\textit{.}

$\therefore S_6=\frac{1\times (1-2^6)}{1-2}=63$
 
 答案:63 \\

知识点:等比数列的前n项和

难度:2

 数列$\mathrm{\{}$\textit{a${}_{n}$}$\mathrm{\}}$的前\textit{n}项和记为\textit{S${}_{n}$},\textit{a}${}_{1}$\textit{=t},点(\textit{S${}_{n}$},\textit{a${}_{n+}$}${}_{1}$)在直线$y=3x+1$上,\textit{n}$\mathrm{\in}$N\textit{${}_{+}$.}

 (1)当实数\textit{t}为何值时,数列$\mathrm{\{}$\textit{a${}_{n}$}$\mathrm{\}}$是等比数列;

 (2)在\eqref{GrindEQ__1_}的结论下,设\textit{b${}_{n}$=}log${}_{4}$\textit{a${}_{n+}$}${}_{1}$,\textit{c${}_{n}$=a${}_{n}$+b${}_{n}$},\textit{T${}_{n}$}是数列$\mathrm{\{}$\textit{c${}_{n}$}$\mathrm{\}}$的前\textit{n}项和,求\textit{T${}_{n}$.}

解析:

 答案:(1)\textit{$\because$}点(\textit{S${}_{n}$},\textit{a${}_{n+}$}${}_{1}$)在直线\textit{y=}3\textit{x+}1上,

\textit{$\therefore$a${}_{n+}$}${}_{1}$\textit{=}3\textit{S${}_{n}$+}1,\textit{a${}_{n}$=}3\textit{S${}_{n-}$}${}_{1}$\textit{+}1(\textit{n$>$}1,且\textit{n}$\mathrm{\in}$N\textit{${}_{+}$}),\textit{a${}_{n+}$}${}_{1}$\textit{-a${}_{n}$=}3(\textit{S${}_{n}$-S${}_{n-}$}${}_{1}$)\textit{=}3\textit{a${}_{n}$},

\textit{$\therefore$a${}_{n+}$}${}_{1}$\textit{=}4\textit{a${}_{n}$},\textit{n$>$}1,\textit{a}${}_{2}$\textit{=}3\textit{S}${}_{1}$\textit{+}1\textit{=}3\textit{a}${}_{1}$\textit{+}1\textit{=}3\textit{t+}1,

\textit{$\therefore$}当\textit{t=}1时,\textit{a}${}_{2}$\textit{=}4\textit{a}${}_{1}$,数列$\mathrm{\{}$\textit{a${}_{n}$}$\mathrm{\}}$是等比数列\textit{.}

(2)在\eqref{GrindEQ__1_}的结论下,\textit{a${}_{n+}$}${}_{1}$\textit{=}4\textit{a${}_{n}$},\textit{a${}_{n+}$}${}_{1}$\textit{=}4\textit{${}^{n}$},\textit{b${}_{n}$=}log${}_{4}$\textit{a${}_{n+}$}${}_{1}$\textit{=n},\textit{c${}_{n}$=a${}_{n}$+b${}_{n}$=}4\textit{${}^{n-}$}${}^{1}$\textit{+n},

$T_n=c_1+c_2+\cdots+c_n=(4^0+1)+(4^1+2)+\cdots+(4^{n-1}+n)$

$=(1+4+4^2+\cdots+4^{n-1})+(1+2+3+\cdots+n)$

$=\frac{4^n-1}{3}+\frac{n(n+1)}{2}$ \\

知识点:等比数列的前n项和

难度:2

 设数列$\mathrm{\{}$\textit{b${}_{n}$}$\mathrm{\}}$的前\textit{n}项和为\textit{S${}_{n}$},且\textit{b${}_{n}$=}2\textit{-}2\textit{S${}_{n}$},数列$\mathrm{\{}$\textit{a${}_{n}$}$\mathrm{\}}$为等差数列,且\textit{a}${}_{5}$\textit{=}14,\textit{a}${}_{7}$\textit{=}20\textit{.}

 (1)求数列$\mathrm{\{}$\textit{b${}_{n}$}$\mathrm{\}}$的通项公式;

 (2)若\textit{c${}_{n}$=a${}_{n}$}·\textit{b${}_{n}$}(\textit{n=}1,2,3{\dots}),\textit{T${}_{n}$}为数列$\mathrm{\{}$\textit{c${}_{n}$}$\mathrm{\}}$的前\textit{n}项和,求\textit{T${}_{n}$.}

解析:

 答案:(1)由\textit{b${}_{n}$=}2\textit{-}2\textit{S${}_{n}$},令\textit{n=}1,

则\textit{b}${}_{1}$\textit{=}2\textit{-}2\textit{S}${}_{1}$,又\textit{S}${}_{1}$\textit{=b}${}_{1}$,所以$b_1=\frac{2}{3}$

当\textit{n}$\mathrm{\ge}$2时,由\textit{b${}_{n}$=}2\textit{-}2\textit{S${}_{n}$}及\textit{b${}_{n-}$}${}_{1}$\textit{=}2\textit{-}2\textit{S${}_{n-}$}${}_{1}$,

可得\textit{b${}_{n}$-b${}_{n-}$}${}_{1}$\textit{=-}2(\textit{S${}_{n}$-S${}_{n-}$}${}_{1}$)\textit{=-}2\textit{b${}_{n}$},即$\frac{b_n}{b_{n-1}}=\frac{1}{3}$

所以$\mathrm{\{}$\textit{b${}_{n}$}$\mathrm{\}}$是以$\frac{2}{3}$为首项,$\frac{1}{3}$为公比的等比数列,

于是$b_n=\frac{2}{3^n}$

(2)由数列$\mathrm{\{}$\textit{a${}_{n}$}$\mathrm{\}}$为等差数列,公差$d=\frac{1}{2}(a_7-a_5)=3$,可得\textit{a${}_{n}$=}3\textit{n-}1\textit{.}从而$c_n=a_n\cdot b_n=2(3n-1)\cdot \frac{1}{3^n}$,

所以$T_n=2[2\cdot \frac{1}{3}+5\cdot \frac{1}{3^2}+8\cdot \frac{1}{3^3}+\cdots +(3n-1)\cdot \frac{1}{3^n}]$,\textit{ ①}

$\frac{1}{3}T_n=2[2\cdot \frac{1}{3^2}+5\cdot \frac{1}{3^3}+\cdots +(3n-4)\cdot \frac{1}{3^n}+(3n-1)\cdot \frac{1}{3^{n+1}}]$,\textit{.②}

\textit{①-②}得,

$\frac{2}{3}T_n=2[2\cdot \frac{1}{3}+3\cdot \frac{1}{3^2}+3\cdot \frac{1}{3^3}+\cdots +3\cdot \frac{1}{3^n}-(3n-1)\frac{1}{3^{n+1}}]$

$=2\left\{2\cdot \frac{1}{3}+3\cdot \frac{\frac{1}{3^2}[1-(\frac{1}{3})^{n-1}]}{1-\frac{1}{3}}-\frac{3n-1}{3^{n+1}}\right\}$

$=\frac{7}{3}-(\frac{1}{3})^{n-1}-\frac{2(3n-1)}{3^{n+1}}$

$T_n=\frac{7}{2}-\frac{1}{2\cdot 3^{n-2}}-\frac{3n-1}{3^n}$ \\


知识点:数列的应用

难度:1

 现有200根相同的钢管,把它们堆成正三角形垛,要使剩余的钢管尽可能少,则剩余钢管的根数为(\textit{  })

 \textit{                }

A.9 B.10 C.19 D.29

 解析:$\because \frac{n(n+1)}{2}<200$,而满足$\frac{n(n+1)}{2}<200$时,\textit{n}可取的最大值为19\textit{.}当\textit{n=}19时,$\frac{n(n+1)}{2}=190$,\textit{$\therefore$}200\textit{-}190\textit{=}10\textit{.}

 答案:B \\

知识点:数列的应用

难度:1

 银行一年定期的年利率为\textit{r},三年定期的年利率为\textit{q},为吸引长期资金,鼓励储户存三年定期存款,则\textit{q}的值应略大于(\textit{  })

 A.$\sqrt{(1+r)^3-1}$ B.$\frac{1}{3}[(1+r)^3-1]$

 C.$(1+r)^3-1$ D.\textit{r}

 解析:设储户存\textit{a}元,存一年定期并自动转存,三年后的本利和为$a(1+r)^3$元\textit{.}三年定期的本利和为$a(1+3q)$元\textit{.}为鼓励储户存三年定期,则$a(1+3q)>a(1+r)^3$,即$q>\frac{1}{3}[(1+r)^3-1]$

 答案:B \\

知识点:数列的应用

难度:1

 某运输卡车从材料工地运送电线杆到500 m以外的公路,沿公路一侧每隔50 m埋一根电线杆,又知每次最多只能运3根,要完成运载20根电线杆的任务,最佳方案是使运输卡车运行(\textit{  })

 A.11700 m B.14600 m

 C.14500 m D.14000 m

 解析:由近往远运送,第一次运两根,以后每次运三根,这种运法最佳,由近往远运送,每次来回行走的米数构成一个等差数列,记为$\mathrm{\{}$\textit{a${}_{n}$}$\mathrm{\}}$,则\textit{a}${}_{1}$\textit{=}1 100,\textit{d=}300,\textit{n=}7,

故$S_7=7\times 1100+\frac{7\times 6}{2}\times 300=1400$

 答案:D \\

知识点:数列的应用

难度:1

 某林厂现在的森林木材存量是1 800万立方米,木材以每年25\%的增长率生长,而每年要砍伐固定的木材量为\textit{x}万立方米,为达到经两次砍伐后木材存量增加50\%的目标,则\textit{x}的值是(\textit{  })

 A.40 B.45 C.50 D.55

 解析:经过一次砍伐后,木材存量为$1800(1+25\%)-x=2250-x$;

经过两次砍伐后,木材存量为$(2250-x)\times (1+25\%)-x=2812.5-2.25x$

由题意应有$2812.5-2.25x=1800\times (1+50\%)$,

解得$x=50$

 答案:C \\

知识点:数列的应用

难度:1

 一个卷筒纸,其内圆直径为4 cm,外圆直径为12 cm,一共卷了60层,若把各层都视为一个同心圆,$\piup$取3\textit{.}14,则这个卷筒纸的长度约为\textit{\underbar{     }}m(精确到个位)\textit{.~}

 解析:\textit{$\because$}纸的厚度相同,\textit{$\therefore$}各层同心圆直径成等差数列\textit{.}

$l=\pi d_1+\pi d_2+\cdots +\pi d_{60}=60\pi \cdot \frac{4+12}{2}=480\pi =1507.2(cm)\approx 15(m)$

 答案:15 \\

知识点:数列的应用

难度:1

 一种专门占据内存的计算机病毒开始时占据内存2 kB,然后每3分钟自身复制一次,复制后所占内存是原来的2倍,那么开机后\textit{\underbar{   }}分,该病毒占据64 MB(1 MB\textit{=}2${}^{10}$ kB)\textit{.~}

 解析:由题意可得每3分病毒占的内存容量构成一个等比数列,设病毒占据64 MB时自身复制了\textit{n}次,即2\textit{$\times$}2\textit{${}^{n}$=}64\textit{$\times$}2${}^{10}$\textit{=}2${}^{16}$,解得\textit{n=}15,从而复制的时间为15\textit{$\times$}3\textit{=}45(分)\textit{.}

 答案:45 \\

知识点:数列的应用

难度:1

 甲、乙两人于同一天分别携款1万元到银行储蓄,甲存5年定期储蓄,年利率为2\textit{.}88\%,乙存一年定期储蓄,年利率为2\textit{.}25\%,并在每年到期时将本息续存一年期定期储蓄,按规定每次计息时,储户须交纳20\%作为利息税\textit{.}若存满五年后两人同时从银行中取出存款,则甲、乙所得利息之差为\textit{\underbar{     }}元\textit{.~}

 解析:由已知甲所得本息和$a=10000+10000\times 2.88\%\times 5\times 80\%$,而乙实际上年利率在去掉利息税后为$\frac{4}{5}\times 2.25\%$,故乙所得本息和应为$b=10000\times (1+\frac{4}{5}\times 2.25\%)^5$,经计算\textit{a-b}$\mathrm{\approx}$219\textit{.}01(元)\textit{.}

 答案:219\textit{.}01 \\

知识点:数列的应用

难度:1

 某地区有荒山2 200亩,从2015年开始每年年初在荒山上植树造林,第一年植树100亩,以后每一年比上一年多植树50亩(假定全部成活)\textit{.}则至少需要几年可将荒山全部绿化?

解析:

 答案:设第\textit{n}年植树造林\textit{a${}_{n}$}亩,数列$\mathrm{\{}$\textit{a${}_{n}$}$\mathrm{\}}$的前\textit{n}项和为\textit{S${}_{n}$},

则数列$\mathrm{\{}$\textit{a${}_{n}$}$\mathrm{\}}$为等差数列,其中\textit{a}${}_{1}$\textit{=}100,\textit{d=}50,

$\therefore a_n=100+50\times (n-1)=50(n+1)$,

$\therefore S_n=na_1+\frac{n(n-1)}{2}d=100n+\frac{n(n-1)}{2}\times 50=25(n^2+3n)$

要将荒山全部绿化,只要\textit{S${}_{n}$}$\mathrm{\ge}$2 200,

即$25(n^2+3n)\ge 2200$,

$\therefore n^2+3n-8\times 11\ge 0$,得\textit{n}$\mathrm{\ge}$8,

故至少需要8年可将荒山全部绿化\textit{.} \\

知识点:数列的应用

难度:1

 为了加强环保建设,提高社会效益和经济效益,长沙市计划用若干年更换一万辆燃油型公交车,每更换一辆新车,则淘汰一辆旧车,更换的新车为电力型车和混合动力型车\textit{.}今年年初投入了电力型公交车128辆,混合动力型公交车400辆,计划以后电力型车每年的投入量比上一年增加50\%,混合动力型车每年比上一年多投入\textit{a}辆\textit{.}

 (1)求经过\textit{n}年,该市被更换的公交车总数\textit{S}(\textit{n});

 (2)若该市计划用7年的时间完成全部更换,求\textit{a}的最小值\textit{.}

解析:

 答案:(1)设\textit{a${}_{n}$},\textit{b${}_{n}$}分别为第\textit{n}年投入的电力型公交车、混合动力型公交车的数量,依题意知,数列$\mathrm{\{}$\textit{a${}_{n}$}$\mathrm{\}}$是首项为128,公比为$1+50\%=\frac{3}{2}$的等比数列,数列$\mathrm{\{}$\textit{b${}_{n}$}$\mathrm{\}}$是首项为400,公差为\textit{a}的等差数列\textit{.}

所以数列$\mathrm{\{}$\textit{a${}_{n}$}$\mathrm{\}}$的前\textit{n}项和$S_n=\frac{128\times [1-(\frac{3}{2})^n]}{1-\frac{3}{2}}=256[(\frac{3}{2})^n-1]$

数列$\mathrm{\{}$\textit{b${}_{n}$}$\mathrm{\}}$的前\textit{n}项和$T_n=400n+\frac{n(n-1)}{2}a$

所以经过\textit{n}年,该市更换的公交车总数

$S(n)=S_n+T_n=256[(\frac{3}{2})^n-1]+400n+\frac{n(n-1)}{2}a$

(2)若用7年的时间完成全部更换,

则\textit{S}\eqref{GrindEQ__7_}$\mathrm{\ge}$10 000,

即$256[(\frac{3}{2})^7-1]+400\times 7+\frac{7\times 6}{2}a\ge 10000$,

即21\textit{a}$\mathrm{\ge}$3 082,所以$a\ge \frac{3082}{21}$

又\textit{a}$\mathrm{\in}$N\textit{${}_{+}$},所以\textit{a}的最小值为147\textit{.} \\

知识点:数列的应用

难度:2

 通过测量知道,温度每降低6 $\mathtt{{}^\circ\!{C}}$,某电子元件的电子数目就减少一半\textit{.}已知在零下34 $\mathtt{{}^\circ\!{C}}$时,该电子元件的电子数目为3个,则在室温27 $\mathtt{{}^\circ\!{C}}$时,该元件的电子数目接近 (\textit{  })

 A.860个 B.1 730个 C.3 072个 D.3 900个

 解析:由题设知,该元件的电子数目变化为等比数列,且\textit{a}${}_{1}$\textit{=}3,\textit{q=}2,由$27-(-34)=61,\frac{61}{6}=10\frac{1}{6}$,可得\textit{a}${}_{11}$\textit{=}3·2${}^{10}$\textit{=}3072,故选C\textit{.}

 答案:C \\

知识点:数列的应用

难度:2

 现存入银行8万元,年利率为2\textit{.}50 \%,若采用1年期自动转存业务,则5年末的本利和是(\textit{  })万元\textit{.}

 A.8\textit{$\times$}1\textit{.}025${}^{3}$ B.8\textit{$\times$}1\textit{.}025${}^{4}$

 C.8\textit{$\times$}1\textit{.}025${}^{5}$ D.8\textit{$\times$}1\textit{.}025${}^{6}$

 解析:定期自动转存属于复利计算问题,5年末的本利和为8\textit{$\times$}(1\textit{+}2\textit{.}50\%)${}^{5}$\textit{=}8\textit{$\times$}1\textit{.}025${}^{5}$(万元)\textit{.}

 答案:C \\

知识点:数列的应用

难度:2

 某企业在2016年年初贷款\textit{M}万元,年利率为\textit{m},从该年年末开始,每年偿还的金额都是\textit{a}万元,并恰好在10年间还清,则\textit{a}的值等于(\textit{  })

 A.$\frac{M(1+m)^{10}}{(1+m)^{10}-1}$ B.$\frac{Mm}{(1+m)^{10}}$

 C.$\frac{Mm(1+m)^{10}}{(1+m)^{10}-1}$ D.$\frac{Mm(1+m)^{10}}{(1+m)^{10}-1}$

 解析:由已知条件和分期付款公式可得,$a[(1+m)^9+(1+m)^8+\cdots+(1+m)+1]=M(1+m)^{10}$,

则$a=\frac{Mm(1+m)^{10}}{(1+m)^{10}-1}$

 答案:C \\

知识点:数列的应用

难度:2

 商家通常依据``乐观系数准则''确定商品销售价格,即根据商品的最低销售限价\textit{a},最高销售限价\textit{b}(\textit{b$>$a})以及实数\textit{x}(0\textit{$<$x$<$}1)确定实际销售价格\textit{c=a+x}(\textit{b-a})\textit{.}这里,\textit{x}被称为乐观系数\textit{.}经验表明,最佳乐观系数\textit{x}恰好使得(\textit{c-a})是(\textit{b-c})和(\textit{b-a})的等比中项\textit{.}据此可得,最佳乐观系数\textit{x}的值等于\textit{\underbar{     }.~}

 答案:$\frac{-1+\sqrt{5}}{2}$ \\

知识点:数列的应用

难度:2

 已知某火箭在点火第一秒通过的路程为2 km,以后每秒通过的路程都增加2 km,在达到离地面240 km的高度时,火箭与飞船分离,则这一过程大约需要的时间是\textit{\underbar{     }}秒\textit{.~}

 解析:设每一秒通过的路程依次为\textit{a}${}_{1}$,\textit{a}${}_{2}$,\textit{a}${}_{3}$,{\dots},\textit{a${}_{n}$},则数列$\mathrm{\{}$\textit{a${}_{n}$}$\mathrm{\}}$是首项\textit{a}${}_{1}$\textit{=}2,公差\textit{d=}2的等差数列\textit{.}

由求和公式得$na_1+\frac{n(n-1)d}{2}=240$,

即$2n+n(n-1)=240$,解得\textit{n=}15\textit{.}

 答案:15 \\

知识点:数列的应用

难度:2

 某企业进行技术改造,有两种方案,甲方案:一次性贷款10万元,第一年便可获利1万元,以后每年比前一年增加30\%的利润;乙方案:每年贷款1万元,第一年可获利1万元,以后每年比前一年增加5千元\textit{.}两种方案使用期都是10年,到期一次性归还本息\textit{.}若银行两种形式的贷款都按年息5\%的复利计算,试比较两种方案中,哪种纯获利更多?(取1\textit{.}05${}^{10}$$\mathrm{\approx}$1\textit{.}629,1\textit{.}3${}^{10}$$\mathrm{\approx}$13\textit{.}786,1\textit{.}5${}^{10}$$\mathrm{\approx}$57\textit{.}665)

解析:

 答案:\textit{①}甲方案获利:$1+(1+30\%)+(1+30\%)^2+\cdots+(1+30\%)^9=\frac{1.3^{10}-1}{0.3}\approx 42.62$(万元),

银行贷款本息:$10(1+5\%)^{10}\approx 16.29$(万元),

故甲方案纯获利:42\textit{.}62\textit{-}16\textit{.}29\textit{=}26\textit{.}33(万元)\textit{.}

\textit{②}乙方案获利:$1+(1+0.5)+(1+2\times 0.5)+\cdots +(1+9\times 0.5)=10\times 1+\frac{10\times 9}{2}\times 0.5=32.5$(万元),

银行本息和:$1.05\times [1+(1+5\%)+(1+5\%)^2+\cdots+(1+5\%)^9]=1.05\times \frac{1.05^{10}-1}{0.05}\approx 13.21$(万元)\textit{.}

故乙方案纯获利:32\textit{.}50\textit{-}13\textit{.}21\textit{=}19\textit{.}29(万元)\textit{.}

综上所述,甲方案纯获利更多\textit{.} \\

知识点:数列的应用

难度:2

 某企业在第1年年初购买一台价值为120万元的设备\textit{M},\textit{M}的价值在使用过程中逐年减少\textit{.}从第2年到第6年,每年年初\textit{M}的价值比上年年初减少10万元;从第7年开始,每年年初\textit{M}的价值为上年年初的75\%\textit{.}

 (1)求第\textit{n}年年初设备\textit{M}的价值\textit{a${}_{n}$}的表达式;

 (2)设$A_n=\frac{a_1+a_2+\cdots+a_n}{n}$,若\textit{A${}_{n}$}大于80万元,则\textit{M}继续使用,否则须在第\textit{n}年年初对\textit{M}更新\textit{.}证明:须在第9年年初对设备\textit{M}更新\textit{.}

解析:

 答案:(1)当\textit{n}$\mathrm{\le}$6时,数列$\mathrm{\{}$\textit{a${}_{n}$}$\mathrm{\}}$是首项为120,公差为\textit{-}10的等差数列,

\textit{a${}_{n}$=}120\textit{-}10(\textit{n-}1)\textit{=}130\textit{-}10\textit{n.}

当\textit{n}$\mathrm{\ge}$7时,数列$\mathrm{\{}$\textit{a${}_{n}$}$\mathrm{\}}$是以\textit{a}${}_{7}$为首项,$\frac{3}{4}$为公比的等比数列,又$a_7=70\times \frac{3}{4}$,

所以$70\times \frac{3}{4}\times (\frac{3}{4})^{n-7}=70\times (\frac{3}{4})^{n-6}$

因此,第\textit{n}年年初,\textit{M}的价值\textit{a${}_{n}$}的表达式为

$a_n=\left\{
\begin{array}{l}
130-10n, n\le 6, \\
70\times (\frac{3}{4})^{n-6}, n\ge7
\end{array}
\right.$

 (2)设\textit{S${}_{n}$}表示数列$\mathrm{\{}$\textit{a${}_{n}$}$\mathrm{\}}$的前\textit{n}项和,由等差及等比数列的求和公式得,

当1$\mathrm{\le}$\textit{n}$\mathrm{\le}$6时,\textit{S${}_{n}$=}120\textit{n-}5\textit{n}(\textit{n-}1),

\textit{A${}_{n}$=}120\textit{-}5(\textit{n-}1)\textit{=}125\textit{-}5\textit{n.}

当\textit{n}$\mathrm{\ge}$7时,$S_n=S_6+(a_7+a_8+\cdots+a_n)$

$=570+70\times \frac{3}{4}\times 4\times [1-(\frac{3}{4})^{n-6}]$

$=780-210\times (\frac{3}{4})^{n-6}$,

$A_n=\frac{780-210\times (\frac{3}{4})^{n-6}}{n}$

易知$\mathrm{\{}$\textit{A${}_{n}$}$\mathrm{\}}$是递减数列,

又$A_8=\frac{780-210\times (\frac{3}{4})^{8-6}}{8}=82\frac{47}{64}>80$,

$A_9=\frac{780-210\times (\frac{3}{4})^{9-6}}{9}=76\frac{79}{96}<80$,

所以须在第9年年初对设备\textit{M}更新\textit{.} \\

知识点:正弦定理

难度:1

 在$\mathrm{\vartriangle}$\textit{ABC}中,若$\frac{\sin A}{a}=\frac{\cos B}{b}$,则\textit{B}的值为(\textit{  })

 \textit{                }

 A.30$\mathrm{{}^\circ}$ B.45$\mathrm{{}^\circ}$ C.60$\mathrm{{}^\circ}$ D.90$\mathrm{{}^\circ}$

 解析:因为$\frac{\sin A}{a}=\frac{\sin B}{b}$,所以$\frac{\cos B}{b} = \frac{\sin B}{b}$,

所以cos \textit{B=}sin \textit{B},从而tan \textit{B=}1,

又0$\mathrm{{}^\circ}$\textit{$<$B$<$}180$\mathrm{{}^\circ}$,所以\textit{B=}45$\mathrm{{}^\circ}$\textit{.}

 答案:B \\

知识点:正弦定理

难度:1

 在$\mathrm{\vartriangle}$\textit{ABC}中,若\textit{B=}45$\mathrm{{}^\circ}$,\textit{C=}60$\mathrm{{}^\circ}$,\textit{c=}1,则最短边的边长是(\textit{  })

 A.$\frac{\sqrt{6}}{3}$ B.$\frac{\sqrt{6}}{2}$ C.$\frac{1}{2}$ D.$\frac{\sqrt{3}}{2}$

 解析:由已知得\textit{A=}75$\mathrm{{}^\circ}$,所以\textit{B}最小,故最短边是\textit{b.}

由$\frac{c}{\sin C}=\frac{b}{\sin B}$,得$b=\frac{c\sin B}{\sin C}=\frac{\sin 45^{\circ}}{\sin 60^{\circ}}=\frac{\sqrt{6}}{3}$
 
 答案:A \\

知识点:正弦定理

难度:1

 在$\mathrm{\vartriangle}$\textit{ABC}中,若$b=8,c=8\sqrt{3},S_{\vartriangle ABC}=16\sqrt{3}$,则\textit{A}等于 (\textit{  })

 A.30$\mathrm{{}^\circ}$ B.60$\mathrm{{}^\circ}$

 C.30$\mathrm{{}^\circ}$或150$\mathrm{{}^\circ}$ D.60$\mathrm{{}^\circ}$或120$\mathrm{{}^\circ}$

 解析:由三角形面积公式得$\frac{1}{2}\times 8\times 8\sqrt{3}\cdot \sin A=16\sqrt{3}$,

于是$\sin A=\frac{1}{2}$,所以\textit{A=}30$\mathrm{{}^\circ}$或\textit{A=}150$\mathrm{{}^\circ}$\textit{.}

 答案:C \\

知识点:正弦定理

难度:1

 下列条件判断三角形解的情况,正确的是(\textit{  })

 A.\textit{a=}8,\textit{b=}16,\textit{A=}30$\mathrm{{}^\circ}$有两解

 B.\textit{b=}9,\textit{c=}20,\textit{B=}60$\mathrm{{}^\circ}$有一解

 C.\textit{a=}15,\textit{b=}2,\textit{A=}90$\mathrm{{}^\circ}$无解

 D.\textit{a=}30,\textit{b=}25,\textit{A=}150$\mathrm{{}^\circ}$有一解

 解析:对于$A,\sin B=\frac{b}{a}\sin A=1$,所以\textit{B=}90$\mathrm{{}^\circ}$,有一解;

对于B,$\sin C=\frac{c}{b}\sin B=\frac{10}{9}\sqrt{3}>1$,所以无解;

对于C,$\sin B=\frac{b}{a}\sin A=\frac{2}{15}<1$,

又\textit{A=}90$\mathrm{{}^\circ}$,所以有一解;

对于D,$\sin B=\frac{b}{a}\sin A=\frac{5}{12}<1$,又\textit{A=}150$\mathrm{{}^\circ}$,

所以有一解\textit{.}

 答案:D \\

知识点:正弦定理

难度:1

 在$\mathrm{\vartriangle}$\textit{ABC}中,角\textit{A},\textit{B},\textit{C}所对的边分别为\textit{a},\textit{b},\textit{c},若$A:B=1:2$,且$a:b=1:\sqrt{3}$,则cos 2\textit{B}的值是(\textit{  })

 A.$-\frac{1}{2}$ B.$\frac{1}{2}$ C.$-\frac{\sqrt{3}}{2}$ D.$\frac{\sqrt{3}}{2}$

 解析:由已知得$\frac{a}{b}=\frac{\sin A}{\sin B}=\frac{\sin A}{\sin 2A}=\frac{\sin A}{2\sin A\cos A}=\frac{1}{2\cos A}=\frac{1}{\sqrt{3}}$,所以$\cos A=\frac{\sqrt{3}}{2}$,解得\textit{A=}30$\mathrm{{}^\circ}$,\textit{B=}60$\mathrm{{}^\circ}$,所以$\cos 2B=\cos 120^{\circ}=\frac{1}{2}$

 答案:A \\

知识点:正弦定理

难度:1

 在$\mathrm{\vartriangle}$\textit{ABC}中,若$a=\sqrt{2},A=45^{\circ}$,则$\mathrm{\vartriangle}$\textit{ABC}的外接圆半径为\textit{\underbar{     }.~}

 解析:因为$2R=\frac{a}{\sin A}=\frac{\sqrt{2}}{\sin 45^{\circ}}=2$,所以\textit{R=}1\textit{.}

 答案:1 \\

知识点:正弦定理

难度:1

 在$\mathrm{\vartriangle}$\textit{ABC}中,角\textit{A},\textit{B},\textit{C}所对的边分别为\textit{a},\textit{b},\textit{c.}已知$A=\frac{\pi}{6},a=1,b=\sqrt{3}$,则\textit{B=\underbar{     }.~}

 解析:由正弦定理得$\frac{a}{\sin A}=\frac{b}{\sin B}$,即$\frac{1}{\sin \frac{\pi}{6}}=\frac{\sqrt{3}}{\sin B}$,解得$\sin B=\frac{\sqrt{3}}{2}$,又因为\textit{b$>$a},所以$B=\frac{\pi}{3}$或$B=\frac{2\pi}{3}$

 答案:$\frac{\pi}{3}$或$\frac{2\pi}{3}$ \\

知识点:正弦定理

难度:1

 在$\mathrm{\vartriangle}$\textit{ABC}中,若$\sin A=2\sin B\cos C,\sin^2 A=\sin^2 B+\sin^2 C$,则$\mathrm{\vartriangle}$\textit{ABC}的形状是\textit{\underbar{     }.~}

 解析:由$\sin^2 A=\sin^2 B+\sin^2 C$,利用正弦定理,

得$a^2=b^2+c^2$,故$\mathrm{\vartriangle}$\textit{ABC}是直角三角形,且\textit{A=}90$\mathrm{{}^\circ}$,

所以$B+C=90^{\circ},B=90^{\circ}-C$,所以sin \textit{B=}cos \textit{C.}

由sin \textit{A=}2sin \textit{B}cos \textit{C},可得1\textit{=}2sin${}^{2}$\textit{B},

所以$\sin^2 B=\frac{1}{2},\sin B=\frac{\sqrt{2}}{2}$,所以\textit{B=}45$\mathrm{{}^\circ}$,\textit{C=}45$\mathrm{{}^\circ}$\textit{.}

所以$\mathrm{\vartriangle}$\textit{ABC}为等腰直角三角形\textit{.}

 答案:等腰直角三角形 \\

知识点:正弦定理

难度:1

 在$\mathrm{\vartriangle}$\textit{ABC}中,$\sin (C-A)=1,\sin B=\frac{1}{3}$

 (1)求sin \textit{A}的值;

 (2)设$AC=\sqrt{6}$,求$\mathrm{\vartriangle}$\textit{ABC}的面积\textit{.}

解析:

 答案:(1)由sin(\textit{C-A})\textit{=}1,\textit{-}$\piup$\textit{$<$C-A$<$}$\piup$,知$C=A+\frac{\pi}{2}$

又$A+B+C=\pi$,所以$2A+B=\frac{\pi}{2}$,

即$2A=\frac{\pi}{2}-B,0<A<\frac{\pi}{4}$

故cos 2\textit{A=}sin \textit{B},即$1-2\sin^2 A=\frac{1}{3},\sin A=\frac{\sqrt{3}}{3}$

(2)由\eqref{GrindEQ__1_}得$\cos A=\frac{\sqrt{6}}{3},\sin C=\sin (A+\frac{\pi}{2})=\cos A$

又由正弦定理,得$\frac{BC}{\sin A}=\frac{AC}{\sin B},BC=\frac{AC\sin A}{\sin B}=3\sqrt{2}$,

所以$S_{\vartriangle ABC}=\frac{1}{2}AC\cdot BC\cdot \sin C=\frac{1}{2}AC\cdot BC\cdot \cos A=3\sqrt{2}$ \\

知识点:正弦定理

难度:1

 在$\mathrm{\vartriangle}$\textit{ABC}中,角\textit{A},\textit{B},\textit{C}的对边分别为\textit{a},\textit{b},\textit{c},角\textit{A},\textit{B},\textit{C}成等差数列\textit{.}

 (1)求cos \textit{B}的值;

 (2)边\textit{a},\textit{b},\textit{c}成等比数列,求sin \textit{A}sin \textit{C}的值\textit{.}

解析:

 答案:(1)因为角\textit{A},\textit{B},\textit{C}成等差数列,所以$2B=A+C$

又$A+B+C=\pi$,所以$B=\frac{\pi}{3}$,所以$\cos B=\frac{1}{2}$

(2)因为边\textit{a},\textit{b},\textit{c}成等比数列,

所以\textit{b}${}^{2}$\textit{=ac},根据正弦定理得sin${}^{2}$\textit{B=}sin \textit{A}sin \textit{C},

所以$\sin A\sin C=\sin^2 B=(\sin \frac{\pi}{3})^2=\frac{3}{4}$ \\

知识点:正弦定理

难度:2


 已知在$\mathrm{\vartriangle}$\textit{ABC}中,\textit{a=x},\textit{b=}2,\textit{B=}45$\mathrm{{}^\circ}$,若三角形有两解,则\textit{x}的取值范围是(\textit{  })

 A.\textit{x$>$}2 B.\textit{x$<$}2

 C.$2<x<2\sqrt{2}$ D.$2<x<2\sqrt{3}$

 解析:由题设条件可知$\left\{
\begin{array}{l}
x>2, \\
x\sin 45^{\circ} <2
\end{array}
\right.$, 解得$2<x<2\sqrt{2}$.

 答案:C \\

知识点:正弦定理

难度:2

 在$\mathrm{\vartriangle}$\textit{ABC}中,内角\textit{A},\textit{B},\textit{C}所对的边分别是\textit{a},\textit{b},\textit{c.}若3\textit{a=}2\textit{b},则$\frac{2\sin^2 B-\sin^2 A}{\sin^2 A}$的值为(\textit{  })

 A.$\frac{1}{9}$ B.$\frac{1}{3}$ C.$1$ D.$\frac{7}{2}$

 解析:因为3\textit{a=}2\textit{b},所以$b=\frac{3}{2}a$

由正弦定理可知$\frac{2\sin^2 B-\sin^2 A}{\sin^2 A}=\frac{2b^2-a^2}{a^2}=\frac{2\times \frac{9}{4}a^2-a^2}{a^2}=\frac{7}{2}$

 答案:D \\

知识点:正弦定理

难度:2

 在$\mathrm{\vartriangle}$\textit{ABC}中,角\textit{A},\textit{B},\textit{C}所对的边分别为\textit{a},\textit{b},\textit{c},已知$a=2\sqrt{3},c=2\sqrt{2},1+\frac{\tan A}{\tan B}=\frac{2c}{b}$,则\textit{C=}(\textit{  })

 A.$\frac{\pi}{6}$ B.$\frac{\pi}{4}$ C.$\frac{\pi}{4}$或$\frac{3}{4}$ D.$\frac{\pi}{3}$

 解析:由$1+\frac{\tan A}{\tan B}=\frac{2c}{b}$得$\frac{\sin (A+B)}{\cos A\sin B}=\frac{2\sin C}{\sin B}$,从而$\cos A=\frac{1}{2}$,所以$A=\frac{\pi}{3}$,由正弦定理得$\frac{2\sqrt{3}}{\frac{\sqrt{3}}{2}}$,解得$\sin C=\frac{\sqrt{2}}{2}$,又\textit{C}$\mathrm{\in}$(0,$\piup$),所以$C=\frac{\pi}{4}$或$C=\frac{3\pi}{4}$(舍去),选B\textit{.}

 答案:B \\

知识点:正弦定理

难度:2

 设\textit{a},\textit{b},\textit{c}三边分别是$\mathrm{\vartriangle}$\textit{ABC}中三个内角\textit{A},\textit{B},\textit{C}所对应的边,则直线$x\sin(\pi -A)+ay+c=0$与$bx-y\cos(\frac{\pi}{2}-B)+\sin C=0$的位置关系是(\textit{  })

 A.平行 B.重合

 C.垂直 D.相交但不垂直

 解析:由已知得$k_1=-\frac{\sin A}{a},k_2=\frac{b}{\sin B}$,因为$\frac{a}{\sin A}=\frac{b}{\sin B}$,所以$k_1\cdot k_2=-\frac{\sin A}{a}\cdot \frac{b}{\sin B}=-\frac{\sin B}{b}\cdot \frac{b}{\sin B}=-1$,所以两直线垂直,故选C\textit{.}

 答案:C \\

知识点:正弦定理

难度:2

 已知在锐角三角形\textit{ABC}中,\textit{A=}2\textit{B},\textit{a},\textit{b},\textit{c}所对的角分别为\textit{A},\textit{B},\textit{C},则$\frac{a}{b}$的取值范围是\textit{\underbar{     }.~}

 解析:在锐角三角形\textit{ABC}中,\textit{A},\textit{B},\textit{C}均小于90$\mathrm{{}^\circ}$,

所以$\left\{
\begin{array}{l}
0^{\circ} < B < 90^{\circ} , \\
0^{\circ} < 2B < 90^{\circ}, \\
0^{\circ} < 180^{\circ} -3B < 90^{\circ}
\end{array}
\right.$, 所以30$\mathrm{{}^\circ}$\textit{$<$B$<$}45$\mathrm{{}^\circ}$\textit{.}

由正弦定理得$\frac{a}{b}=\frac{\sin A}{\sin B}=\frac{\sin 2B}{\sin B}=2\cos B\in (\sqrt{2},\sqrt{3})$,

故$\frac{a}{b}$的取值范围是$(\sqrt{2},\sqrt{3})$.

 答案:$(\sqrt{2},\sqrt{3})$ \\

知识点:正弦定理

难度:2

 在$\mathrm{\vartriangle}$\textit{ABC}中,已知$\sin B\cdot \sin C=\cos^2 \frac{A}{2},A=120^{\circ},a=12$,则$\mathrm{\vartriangle}$\textit{ABC}的面积为\textit{\underbar{     }.~}

 解析:因为$\sin B\cdot \sin C=\cos^2 \frac{A}{2}$,所以$\sin B\cdot \sin C=\frac{\cos A+1}{2}$,所以$2\sin B\sin C=\cos A+1$

又因为$A+B+C=\pi$,所以$\cos A=\cos(\pi -B-C)=-\cos(B+C)=-\cos B\cdot \cos C+\sin B\cdot \sin C$,

所以$2\sin B\sin C=-\cos B\cdot \cos C+\sin B\cdot \sin C+1$,

所以$\cos B\cdot \cos C+\sin B\cdot \sin C=\cos(B-C)=1$

因为\textit{B},\textit{C}为$\mathrm{\vartriangle}$\textit{ABC}的内角,所以\textit{B=C.}

因为\textit{A=}120$\mathrm{{}^\circ}$,所以\textit{B=C=}30$\mathrm{{}^\circ}$\textit{.}

由正弦定理得,$b=\frac{a\sin B}{\sin A}=\frac{12\times \frac{1}{2}}{\frac{\sqrt{3}}{2}}=4\sqrt{3}$,

所以$S_{\vartriangle ABC}=\frac{1}{2}ab\sin C=\frac{1}{2}\times 12\times 4\sqrt{3}\times 
\frac{1}{2}=12\sqrt{3}$

 答案:$12\sqrt{3}$ \\

知识点:正弦定理

难度:2

 $\mathrm{\vartriangle}$\textit{ABC}的三个内角\textit{A},\textit{B},\textit{C}的对边分别是\textit{a},\textit{b},\textit{c},若$a^2=b(b+c)$,求证:\textit{A=}2\textit{B.}

解析:

 答案:由已知及正弦定理得,$\sin^2 A=\sin^2 B+\sin B\cdot \sin C$,

因为$A+B+C=\pi$,所以$\sin C=\sin (A+B)$,

所以$\sin^2 A=\sin^2 B+\sin B\cdot \sin(A+B)$,

所以$\sin^2 A-\sin^2 B=\sin B\cdot \sin(A+B)$

因为$\sin^2 A-\sin^2 B=\sin^2 A(\sin^2 B+\cos^2 B)-\sin^2 B(\sin^2 A+\cos^2 A)=\sin^2 A\cos^2 B-\cos^2 A\sin^2 B$

$=(\sin A\cos B+\cos A\sin B)(\sin A\cos B-\cos A\sin B)$

$=\sin(A+B)\cdot \sin (A-B)$,

所以$\sin(A+B)\cdot \sin(A-B)=\sin B\cdot \sin(A+B)$

因为\textit{A},\textit{B},\textit{C}为$\mathrm{\vartriangle}$\textit{ABC}的三个内角,所以$\sin(A+B)\ne 0$,

所以sin(\textit{A-B})\textit{=}sin \textit{B},所以只能有\textit{A-B=B},即\textit{A=}2\textit{B.} \\

知识点:正弦定理

难度:2

 在$\mathrm{\vartriangle}$\textit{ABC}中,\textit{a},\textit{b},\textit{c}分别是角\textit{A},\textit{B},\textit{C}所对的边,已知$\cos B=\frac{a}{2c}$,

 (1)判断$\mathrm{\vartriangle}$\textit{ABC}的形状;

 (2)若$\sin B=\frac{\sqrt{3}}{3},b=3$,求$\mathrm{\vartriangle}$\textit{ABC}的面积\textit{.}

解析:

 答案:(1)因为$\cos B=\frac{a}{2c},\frac{a}{\sin A}=\frac{c}{\sin C}$,

所以$\cos B=\frac{\sin A}{2\sin C}$,所以$\sin A=2\cos B\sin C$

又$\sin A=\sin[\pi-(B+C)]$

$=\sin(B+C)=\sin B\cos C+\cos B\sin C$,

所以$\sin B\cos C+\cos B\sin C=2\cos B\sin C$

所以$\sin B\cos C-\cos B\sin C=\sin(B-C)=0$

所以在$\mathrm{\vartriangle}$\textit{ABC}中,\textit{B=C},所以$\mathrm{\vartriangle}$\textit{ABC}为等腰三角形\textit{.}

(2)因为\textit{C=B},所以$0<B<\frac{\pi}{2},c=b=3$

因为$\sin B=\frac{\sqrt{3}}{3}$,所以$\cos B=\frac{\sqrt{6}}{3}$

所以$\sin A=\sin[\pi-(B+C)]=\sin(B+C)$

$=\sin 2B=2\sin B\cos B=\frac{2\sqrt{2}}{3}$,

所以$S_{\vartriangle ABC}=\frac{1}{2}bc\sin A=\frac{1}{2}\times 3\times 3\times \frac{2\sqrt{2}}{3}=3\sqrt{2}$ \\


知识:余弦定理

难度:1

 题目:在$\mathrm{\vartriangle}$\textit{ABC}中,已知\textit{a=}2,\textit{b=}3,cos \textit{C=}$\frac{1}{3}$,则边\textit{c}长为 (\textit{  })



 A.2 B.3 C.$\sqrt{11}$ D.$\sqrt{17}$

 解析:因为$ c^2=a^2+b^2-2ab\cos C=2^2+3^2-2\times2\times3\times\frac{1}{3}=9$,所以\textit{c=}3\textit{.}

 答案:B

知识:余弦定理

难度:1

 题目:在$\mathrm{\vartriangle}$\textit{ABC}中,若\textit{C=}60$\mathrm{{}^\circ}$,\textit{c}${}^{2}$\textit{=ab},则三角形的形状为 (\textit{  })

 A.直角三角形 B.等腰三角形

 C.等边三角形 D.钝角三角形

 解析:因为在$\mathrm{\vartriangle}$\textit{ABC}中,\textit{C=}60$\mathrm{{}^\circ}$,\textit{c}${}^{2}$\textit{=ab},所以\textit{c}${}^{2}$\textit{=a}${}^{2}$\textit{+b}${}^{2}$\textit{-}2\textit{ab}cos \textit{C=a}${}^{2}$\textit{+b}${}^{2}$\textit{-ab=ab},所以\textit{a=b},所以\textit{a=b=c},所以三角形的形状为等边三角形,故选C\textit{.}

 答案:C

知识:余弦定理

难度:1

 题目:已知$\mathrm{\vartriangle}$\textit{ABC}的三边满足\textit{a}${}^{2}$\textit{+b}${}^{2}$\textit{=c}${}^{2}$\textit{-}$\sqrt{3}$\textit{ab},则$\mathrm{\vartriangle}$\textit{ABC}的最大内角为(\textit{  })

 A.60$\mathrm{{}^\circ}$ B.90$\mathrm{{}^\circ}$ C.120$\mathrm{{}^\circ}$ D.150$\mathrm{{}^\circ}$

 解析:由已知得,$c^2=a^2+b^2+\sqrt{3}ab$,所以\textit{c$>$a},\textit{c$>$b},故\textit{C}为最大内角\textit{.}由cos \textit{C=}$\frac{a^2+b^2-c^2}{2ab}=-\frac{\sqrt{2}}{2}$,得\textit{C=}150\textit{${}^\circ$},故选D\textit{.}

 答案:D

知识:余弦定理

难度:1

 题目:在$\mathrm{\vartriangle}$\textit{ABC}中,若\textit{a=}1,\textit{B=}45$\mathrm{{}^\circ}$,\textit{S}${}_{\vartriangle }$\textit{${}_{ABC}$=}2,则$\mathrm{\vartriangle}$\textit{ABC}外接圆的直径为(\textit{  })

 A.$4\sqrt{3}$ B\textit{.}6 C.$5\sqrt{2}$ D.$6\sqrt{2}$

 解析:因为\textit{S}${}_{\vartriangle }$\textit{${}_{ABC}$=}$\frac{1}{2}$\textit{ac}sin \textit{B=}$\frac{1}{2}$·\textit{c}·sin 45$\mathrm{{}^\circ}$\textit{=}$\frac{\sqrt{2}}{4}$\textit{c=}2,

所以$c=4\sqrt{2}$\textit{.}

由余弦定理得$b^2=a^2+c^2-2ac\cos B=1+32-2\times 1\times4\sqrt{2}\times\frac{\sqrt{2}}{2}$,所以\textit{b=}5\textit{.}

所以$\mathrm{\vartriangle}$\textit{ABC}外接圆直径2\textit{R=}$\frac{b}{\sin B}$\textit{=}$5\sqrt{2}$

 答案:C

知识:余弦定理

难度:1

 题目:已知在$\mathrm{\vartriangle}$\textit{ABC}中,\textit{a}比\textit{b}大2,\textit{b}比\textit{c}大2,最大角的正弦值是$\frac{\sqrt{2}}{2}$,则$\mathrm{\vartriangle}$\textit{ABC}的面积是(\textit{  })

 A.$\frac{15\sqrt{3}}{4}$ B.$\frac{15}{4}$ C.$\frac{21\sqrt{3}}{4}$ D.$\frac{35\sqrt{3}}{4}$

 解析:因为\textit{a=b+}2,\textit{b=c+}2,所以\textit{a=c+}4,\textit{A}为最大角,所以sin \textit{A=}$\frac{\sqrt{2}}{2}$

又\textit{A$>$B$>$C},所以\textit{A=}120$\mathrm{{}^\circ}$,

所以cos \textit{A=-}$\frac{1}{2}$,即$\frac{b^2+c^2-a^2}{2bc}=-\frac{1}{2}$,

所以${(c+2)}^2+c^2-{(c+4)}^2=-c(c+2)$,解得\textit{c=}3\textit{.}

所以\textit{a=}7,\textit{b=}5,\textit{c=}3,\textit{A=}120$\mathrm{{}^\circ}$\textit{.}

\textit{S}${}_{\vartriangle }$\textit{${}_{ABC}$=}$\frac{1}{2}$\textit{bc}sin \textit{A=}$\frac{\sqrt{3}}{2}$\textit{$\times$}5\textit{$\times$}3\textit{$\times$}$\frac{\sqrt{3}}{2}=\frac{15\sqrt{3}}{4}$

 答案:A

知识:余弦定理

难度:1

 题目:在$\mathrm{\vartriangle}$\textit{ABC}中,内角\textit{A},\textit{B},\textit{C}的对边分别为\textit{a},\textit{b},\textit{c},若\textit{c=}2\textit{a},\textit{b=}4,cos \textit{B=}$\frac{1}{4}$,则\textit{c=\underbar{     }.~}

 解析:因为cos \textit{B=}$\frac{1}{4}$,由余弦定理得4${}^{2}$\textit{=a}${}^{2}$\textit{+}(2\textit{a})${}^{2}$\textit{-}2\textit{a$\times$}2\textit{a$\times$}$\frac{1}{4}$,解得\textit{a=}2,所以\textit{c=}4\textit{.}

 答案:4

知识:余弦定理

难度:1

 题目:设$\mathrm{\vartriangle}$\textit{ABC}的内角\textit{A},\textit{B},\textit{C}所对边长分别为\textit{a},\textit{b},\textit{c},且3\textit{b}${}^{2}$\textit{+}3\textit{c}${}^{2}$\textit{-}3\textit{a}${}^{2}$\textit{=}4$\sqrt{2}$\textit{bc},则sin \textit{A}的值为\textit{\underbar{     }.~}

 解析:由已知得$b^2+c^2-a^2=\frac{4\sqrt{2}}{3}bc$,于是cos \textit{A=}$\frac{\frac{4\sqrt{2}}{3}bc}{2bc}=\frac{2\sqrt{2}}{3}$,从而sin \textit{A=}$\sqrt{1-{\cos A}^2}=\frac{1}{3}$

 答案:$\frac{1}{3}$

知识:余弦定理

难度:1

 题目:已知在$\mathrm{\vartriangle}$\textit{ABC}中,\textit{AB=}7,\textit{BC=}5,\textit{CA=}6,则$\vec{BA}\times\vec{BC}$\textit{=\underbar{     }.~}

 解析:在$\mathrm{\vartriangle}$\textit{ABC}中,分别用\textit{a},\textit{b},\textit{c}表示边\textit{BC},\textit{CA},\textit{AB},

则$\vec{BA}\times\vec{BC}$\textit{=ca}·cos \textit{B=ca}·$\frac{a^2+c^2-b^2}{2ac}$

\textit{=}$\frac{1}{2}$(\textit{a}${}^{2}$\textit{$+$c}${}^{2}$\textit{-b}${}^{2}$)\textit{=}$\frac{1}{2}$(5${}^{2}$\textit{$+$}7${}^{2}$\textit{-}6${}^{2}$)\textit{=}19\textit{.}

 答案:19

知识:余弦定理

难度:1

 题目:设$\mathrm{\vartriangle}$\textit{ABC}的内角\textit{A},\textit{B},\textit{C}所对的边分别为\textit{a},\textit{b},\textit{c},且\textit{a+c=}6,\textit{b=}2,cos \textit{B=}$\frac{7}{9}$\textit{.}

 (1)求\textit{a},\textit{c}的值;

 (2)求sin(\textit{A-B})的值\textit{.}

 答案:(1)由\textit{b}${}^{2}$\textit{=a}${}^{2}$\textit{$+$c}${}^{2}$\textit{-}2\textit{ac}cos \textit{B},

得\textit{b}${}^{2}$\textit{=}(\textit{a$+$c})${}^{2}$\textit{-}2\textit{ac}(1\textit{$+$}cos \textit{B}),又\textit{b=}2,\textit{a$+$c=}6,cos \textit{B=}$\frac{7}{9}$,所以\textit{ac=}9,解得\textit{a=}3,\textit{c=}3\textit{.}

(2)在$\mathrm{\vartriangle}$\textit{ABC}中,sin \textit{B=}$\sqrt{1-{\cos B}^2}=\frac{4\sqrt{2}}{9}$,

由正弦定理得sin \textit{A=}$\frac{a\sin B}{b}=\frac{2\sqrt{2}}{3}$

因为\textit{a=c},所以\textit{A}为锐角,

所以cos \textit{A=}$\sqrt{1-{\sin A}^2}=\frac{1}{3}$\textit{.}

因此sin(\textit{A-B})\textit{=}sin \textit{A}cos \textit{B-}cos \textit{A}sin \textit{B=}$\frac{10\sqrt{2}}{27}$\textit{.}

知识点:余弦定理

难度:1

 题目:已知在$\mathrm{\vartriangle}$\textit{ABC}中,三个内角\textit{A},\textit{B},\textit{C}所对的边分别为\textit{a},\textit{b},\textit{c},向量p\textit{=}(sin \textit{A-}cos \textit{A},1\textit{-}sin \textit{A}),q\textit{=}(2\textit{$+$}2sin \textit{A},sin \textit{A$+$}cos \textit{A}),p与q是共线向量,且$\frac{\pi}{6}$$\mathrm{\le}$\textit{A}$\mathrm{\le}$$\frac{\pi}{2}$\textit{.}

 (1)求角\textit{A}的大小;

 (2)若sin \textit{C=}2sin \textit{B},且\textit{a=}$\sqrt{3}$,试判断$\mathrm{\vartriangle}$\textit{ABC}的形状,并说明理由\textit{.}

 答案:(1)因为p$\mathrm{\parallel}$q,所以(sin \textit{A-}cos \textit{A})(sin \textit{A$+$}cos \textit{A})\textit{-}2(1\textit{-}sin \textit{A})(1\textit{$+$}sin \textit{A})\textit{=-}cos 2\textit{A-}2cos${}^{2}$\textit{A=}0,所以1\textit{$+$}2cos 2\textit{A=}0,所以cos 2\textit{A=-}$\frac{1}{2}$

因为$\frac{\pi}{6}$$\mathrm{\le}$\textit{A}$\mathrm{\le}$$\frac{\pi}{2}$,所以$\frac{\pi}{3}$$\mathrm{\le}$2\textit{A}$\mathrm{\le}$$\piup$,所以2\textit{A=}$\frac{2\pi}{3}$,所以\textit{A=}$\frac{\pi}{3}$

(2)$\mathrm{\vartriangle}$\textit{ABC}是直角三角形\textit{.}理由如下:

由cos \textit{A=}$\frac{1}{2}$,\textit{a=}$\sqrt{3}$及余弦定理得\textit{b}${}^{2}$\textit{$+$c}${}^{2}$\textit{-bc=}3\textit{.}

又sin \textit{C=}2sin \textit{B},由正弦定理得\textit{c=}2\textit{b.}

联立可得$
\begin{cases}
b^2+c^2-bc=3,\\
c=2b
\end{cases}$,$
\begin{cases}
b=1,\\
c=2
\end{cases}$

所以\textit{a}${}^{2}$\textit{$+$b}${}^{2}$\textit{=}($\sqrt{3}$)${}^{2}$\textit{$+$}1${}^{2}$\textit{=}4\textit{=c}${}^{2}$,所以$\mathrm{\vartriangle}$\textit{ABC}是直角三角形\textit{.}



知识点:余弦定理

难度:2

 题目:在$\mathrm{\vartriangle}$\textit{ABC}中,若$\mathrm{\vartriangle}$\textit{ABC}的面积\textit{S=}$\frac{1}{4}$(\textit{a}${}^{2}$\textit{$+$b}${}^{2}$\textit{-c}${}^{2}$),则\textit{C=}(\textit{  })

 A.$\frac{\pi}{2}$ B.$\frac{\pi}{6}$ C.$\frac{\pi}{3}$ D.$\frac{\pi}{2}$

 解析:由\textit{S=}$\frac{1}{4}$(\textit{a}${}^{2}$\textit{+b}${}^{2}$\textit{-c}${}^{2}$),得$\frac{1}{2}$\textit{ab}sin \textit{C=}$\frac{1}{4}$\textit{$\times$}2\textit{ab}cos \textit{C},

所以tan \textit{C=}1,又\textit{C}$\mathrm{\in}$(0,$\piup$),所以\textit{C=}$\frac{\pi}{4}$

 答案:A

知识点:余弦定理

难度:2

 题目:在$\mathrm{\vartriangle}$\textit{ABC}中,若sin \textit{A-}sin \textit{A}·cos \textit{C=}cos \textit{A}sin \textit{C},则$\mathrm{\vartriangle}$\textit{ABC}的形状是(\textit{  })

 A.正三角形 B.等腰三角形

 C.直角三角形 D.等腰直角三角形

 解析:由正弦定理、余弦定理,知sin \textit{A-}sin \textit{A}cos \textit{C=}cos \textit{A}sin \textit{C}可化为$a(1-\frac{a^2+b^2-c^2}{2ab})=\frac{b^2+c^2-a^2}{abc}\times c c$,整理,得\textit{a=b},所以$\mathrm{\vartriangle}$\textit{ABC}是等腰三角形,选B\textit{.}

 答案:B

知识点:余弦定理

难度:2

 题目:已知$\mathrm{\vartriangle}$\textit{ABC}各角的对边分别为\textit{a},\textit{b},\textit{c},满足$\frac{b}{a+c}+\frac{c}{a+b}$$\mathrm{\ge}$1,则角\textit{A}的范围是(\textit{  })

 A.$( 0,\frac{\pi}{3} \rbrack$ B.$(0,\frac{\pi}{6}\rbrack$

 C.$\lbrack \frac{\pi}{3},\pi)$ D.$\lbrack \frac{\pi}{6},\pi)$

 解析:将不等式$\frac{b}{a+c}+\frac{c+a+b}{den}$$\mathrm{\ge}$1两边同乘以$(a+c)(a+b)$整理得,\textit{b}${}^{2}$\textit{$+$c}${}^{2}$\textit{-a}${}^{2}$$\mathrm{\ge}$\textit{bc},所以cos \textit{A=}$\frac{b^2+c^2-a^2}{2bc}\ge \frac{bc}{2bc}=\frac{1}{2}$,所以0\textit{$<$A}$\mathrm{\le}$$\frac{\pi}{3}$,故选A\textit{.}

 答案:A

知识点:余弦定理

难度:2


 题目:在$\mathrm{\vartriangle}$\textit{ABC}中,若边长和内角满足\textit{a}${}^{2}$\textit{-b}${}^{2}$\textit{=}$\sqrt{3}$\textit{bc},$\frac{\sin (A+B)}{\sin B}=2\sqrt{3}=2\sqrt{3}$,则\textit{A=\underbar{     }.~}

 解析:因为$\frac{\sin (A+B)}{\sin B}=\frac{\sin C}{\sin B}=\frac{c}{b}=2\sqrt{3}$,

所以\textit{c=}2$\sqrt{3}$\textit{b.}

又\textit{a}${}^{2}$\textit{-b}${}^{2}$\textit{=}$\sqrt{3}$\textit{bc},所以$\cos A=\frac{b^2+c^2-a^2}{2bc}=\frac{c^2-\sqrt{3}bc}{2bc}=\frac{12b^2-6b^2}{4\sqrt{3}b^2}=\frac{\sqrt{3}}{2}$,又\textit{A}$\mathrm{\in}$(0,$\piup$),所以\textit{A=}$\frac{\pi}{6}$\textit{.}

 答案:$\frac{\pi}{6}$

知识点:余弦定理

难度:2


 题目:已知在$\mathrm{\vartriangle}$\textit{ABC}中,三个内角\textit{A},\textit{B},\textit{C}所对边分别为\textit{a=}3,\textit{b=}4,\textit{c=}6,则\textit{bc}cos \textit{A$+$ac}cos \textit{B$+$ab}cos \textit{C}的值为\textit{\underbar{     }.~}

 解析:
\begin{align}
\notag
&bc\cos A+ac\cos B+ab\cos C\\
&=bc\cdot \frac{b^2+c^2-a^2}{2bc}+ac\cdot \frac{a^2+c^2-b^2}{2ac}+ab\cdot \frac{a^2+b^2-c^2}{2ab}\\
&=\frac{1}{2}(b^2+c^2-a^2+a^2+c^2-b^2+a^2+b^2-c^2)\\
&=\frac{1}{2}(a^2+b^2+c^2)=\frac{61}{2}
\end{align}

 答案:$\frac{61}{2}$

知识点:余弦定理

难度:2


 题目:已知点\textit{O}是$\mathrm{\vartriangle}$\textit{ABC}的重心,内角\textit{A},\textit{B},\textit{C}所对的边分别为\textit{a},\textit{b},\textit{c},且$2a\cdot \vec{OA}+b\cdot\vec{OB}+\frac{2\sqrt{3}}{3}c\cdot\vec{OC}=0$,则角\textit{C}的大小是\textit{\underbar{     }.~}

 解析:因为点\textit{O}是$\mathrm{\vartriangle}$\textit{ABC}的重心,所以$\vec{OA}+\vec{OB}+\vec{OC}=0$,

又因为$2a\cdot \vec{OA}+b\cdot\vec{OB}+\frac{2\sqrt{3}}{3}c\cdot\vec{OC}=0$,所以$2a=b=\frac{2\sqrt{3}}{3}c=k(k>0)$,从而\textit{a=}$\frac{1}{2}$,\textit{b=k},\textit{c=}$\frac{\sqrt{3}}{2}$\textit{k},由余弦定理得$\cos C=\frac{a^2+b^2-c^2}{2ab}=\frac{\frac{k^2}{4}+k^2\cdot \frac{3}{4}k^2}{2\cdot \frac{k}{2}\cdot k}=\frac{1}{2}$,又因为\textit{C}$\mathrm{\in}$(0,$\piup$),所以\textit{C=}$\frac{\pi}{3}$,所以角\textit{C}的大小是$\frac{\pi}{3}$\textit{.}

 答案:$\frac{\pi}{3}$

知识点:余弦定理

难度:2


 题目:在$\mathrm{\vartriangle}$\textit{ABC}中,角\textit{A},\textit{B},\textit{C}的对边分别为\textit{a},\textit{b},\textit{c},tan \textit{C=}$3\sqrt{7}$

 (1)求$\cos$ \textit{C}的值;

 (2)若$\vec{CB}\cdot\vec{CA}=\frac{5}{2}$,且$a+b=9$,求\textit{c.}

 答案:(1)因为tan \textit{C=}$3\sqrt{7}$,所以$\frac{\sin C}{\cos C}=3\sqrt{7}$,

又因为${\sin C}^2+{\cos C}^2=1$,解得$\cos C\pm\frac{1}{3}$,

由tan \textit{C$>$}0知,\textit{C}为锐角,所以$\cos=\frac{1}{3}$

(2)由$\vec{CB}\cdot\vec{CA}=\frac{3}{2}$,得$ab\cos C=\frac{5}{2}$,即\textit{ab=}20\textit{.}

又因为$a+b=9$,则$a^2+2ab+b^2=81$,所以\textit{a}${}^{2}$\textit{$+$b}${}^{2}$\textit{=}41\textit{.}

由余弦定理得,\textit{c}${}^{2}$\textit{=a}${}^{2}$\textit{$+$b}${}^{2}$\textit{-}2\textit{ab}cos \textit{C=}41\textit{-}2\textit{$\times$}20\textit{$\times$}$\frac{1}{8}$\textit{=}36,故\textit{c=}6\textit{.}

知识点:余弦定理

难度:2

 题目:在$\mathrm{\vartriangle}$\textit{ABC}中,\textit{a},\textit{b},\textit{c}分别是角\textit{A},\textit{B},\textit{C}的对边,且$\frac{\cos B}{\cos C}=\frac{b}{2a+c}$

 (1)求角\textit{B}的大小;

 (2)若\textit{b=}$\sqrt{13}$,$a+c=4$,求$\mathrm{\vartriangle}$\textit{ABC}的面积\textit{.}

 答案:(1)由余弦定理知,cos \textit{B=}$\frac{a^2+c^2-b^2}{2ac}$,cos \textit{C=}$\frac{a^2+b^2-c^2}{2ab}$

将上式代入$\frac{\cos B}{\cos C}=-\frac{b}{2a+c}$,得$\frac{a^2+c^2-b^2}{2ac}\cdot\frac{2ab}{a^2+b^2-c^2}=-\frac{b}{2a+c}$,整理得\textit{a}${}^{2}$\textit{$+$c}${}^{2}$\textit{-b}${}^{2}$\textit{=-ac.}

所以$\cos B=\frac{a^2+c^2-b^2}{2ac}=-\frac{ac}{2ac}=\frac{1}{2}$

因为\textit{B}为三角形的内角,所以$B=\frac{2\pi}{3}$

(2)将$b=\sqrt{13}$,$a+c=4$,$B=\frac{2\pi}{3}$代入$b^2=a^2+c^2-2ac\cos B$,即\textit{b}${}^{2}$\textit{=}(\textit{a$+$c})${}^{2}$\textit{-}2\textit{ac-}2\textit{ac}cos \textit{B}得,

$13=16-2ac(1-\frac{1}{2})$,所以\textit{ac=}3\textit{.}

所以\textit{S}${}_{\vartriangle }$\textit{${}_{ABC}$=}$\frac{1}{2}$\textit{ac}sin \textit{B=}$\frac{3\sqrt{3}}{4}$


 \eject 2.2 三角形中的几何计算

知识点:解三角形

难度:1

 题目:在$\mathrm{\vartriangle}$\textit{ABC}中,若\textit{A=}105$\mathrm{{}^\circ}$,\textit{B=}30$\mathrm{{}^\circ}$,\textit{BC=}$\frac{\sqrt{6}}{2}$,则角\textit{B}的平分线的长是(\textit{  })

 

 A.$\frac{\sqrt{3}}{2}$ B.$2\sqrt{2}$ C.1 D.$\sqrt{2}$

 解析:设角\textit{B}的平分线与\textit{AC}交于点\textit{D},则在$\mathrm{\vartriangle}$\textit{BCD}中,$\mathrm{\angle}$\textit{BDC=}120$\mathrm{{}^\circ}$,$\mathrm{\angle}$\textit{BCD=}45$\mathrm{{}^\circ}$,\textit{BC=}$\frac{\sqrt{6}}{2}$,由正弦定理可知\textit{BD=}1\textit{.}

 答案:C

知识点:解三角形

难度:1

 题目:在$\mathrm{\vartriangle}$\textit{ABC}中,若\textit{AC=}$\sqrt{7}$,\textit{BC=}2,\textit{B=}60$\mathrm{{}^\circ}$,则\textit{BC}边上的高等于(\textit{  })

 A.$\frac{\sqrt{3}}{2}$ B.$\frac{3\sqrt{3}}{2}$

 C.$\frac{\sqrt{3}+\sqrt{6}}{2}$ D.$\frac{\sqrt{3}+\sqrt{39}}{4}$

 解析:如图,在$\mathrm{\vartriangle}$\textit{ABC}中,由余弦定理可知,\textit{AC}${}^{2}$\textit{=AB}${}^{2}$\textit{$+$BC}${}^{2}$\textit{-}2\textit{AB}·\textit{BC}cos \textit{B},

 \includegraphics*[width=0.83in, height=1.02in, keepaspectratio=false]{image962}

即7\textit{=AB}${}^{2}$\textit{$+$}4\textit{-}2\textit{$\times$}2\textit{$\times$AB$\times$}$\frac{1}{2}$\textit{.}

整理得\textit{AB}${}^{2}$\textit{-}2\textit{AB-}3\textit{=}0\textit{.}

解得\textit{AB=}3或\textit{AB=-}1(舍去)\textit{.}

故\textit{BC}边上的高\textit{AD=AB}·sin \textit{B=}3\textit{$\times$}sin 60$\mathrm{{}^\circ}$\textit{=}$\frac{3\sqrt{3}}{2}$\textit{.}

 答案:B

知识点:解三角形

难度:1

 题目:若$\mathrm{\vartriangle}$\textit{ABC}的周长等于20,面积是 $10\sqrt{3}$,\textit{A=}60$\mathrm{{}^\circ}$,则\textit{BC}边的长是(\textit{  })

 A.5 B.6 C.7 D.8

 解析:在$\mathrm{\vartriangle}$\textit{ABC}中,分别用\textit{a},\textit{b},\textit{c}表示边\textit{BC},\textit{CA},\textit{AB.}依题意及面积公式$S=\frac{1}{2}bc\sin A$,得$10\sqrt{3}=\frac{1}{2}bc\times\sin 60^\circ$,即\textit{bc=}40\textit{.}

又周长为20,所以$a+b+c=20$,$b+c=20-a$

由余弦定理,得\textit{a}${}^{2}$\textit{=b}${}^{2}$\textit{$+$c}${}^{2}$\textit{-}2\textit{bc}cos \textit{A=b}${}^{2}$\textit{$+$c}${}^{2}$\textit{-}2\textit{bc}cos 60$\mathrm{{}^\circ}$\textit{=b}${}^{2}$\textit{$+$c}${}^{2}$\textit{-bc=}(\textit{b$+$c})${}^{2}$\textit{-}3\textit{bc},

所以\textit{a}${}^{2}$\textit{=}(20\textit{-a})${}^{2}$\textit{-}120,解得\textit{a=}7\textit{.}

 答案:C

知识点:解三角形

难度:2

 题目:在$\mathrm{\vartriangle}$\textit{ABC}中,角\textit{A},\textit{B},\textit{C}所对的边分别为\textit{a},\textit{b},\textit{c}且满足\textit{c}sin \textit{A=a}cos \textit{C.}当$\sqrt{3}\sin A-\cos (B+\frac{\pi}{4})$取最大值时,\textit{A}的大小为(\textit{  })

 A.$\frac{\pi}{3}$ B.$\frac{\pi}{4}$

 C.$\frac{\pi}{6}$ D.$\frac{2\pi}{3}$

 解析:由正弦定理得sin \textit{C}sin \textit{A=}sin \textit{A}cos \textit{C.}

因为0\textit{$<$A$<$}$\piup$,所以sin \textit{A$>$}0,

从而sin \textit{C=}cos \textit{C.}

又cos \textit{C}$\mathrm{\neq}$0,所以tan \textit{C=}1,则\textit{C=}$\frac{\pi}{4}$,

所以\textit{B=}$\frac{3\pi}{4}-A$

于是
\begin{align}
\notag
\sqrt{3}\sin A-\cos (B+\frac{\pi}{4})&=\sqrt{3}\sin A-\cos (\pi-A)\\\notag
&=\sqrt{3}\sin A+\cos A\\\notag
&=2\sin A(A+\frac{\pi}{6})
\end{align}


因为$0<A<\frac{3\pi}{4}$,所以$\frac{\pi}{6}<A+\frac{\pi}{6}<\frac{11\pi}{12}$,所以当$A+\frac{\pi}{6}=\frac{\pi}{2}$,

即$A=\frac{\pi}{3}$时,$2\sin (A+\frac{\pi}{6})$取最大值2\textit{.}

 答案:A

知识点:解三角形

难度:1

 题目:在$\mathrm{\vartriangle}$\textit{ABC}中,若\textit{C=}60$\mathrm{{}^\circ}$,\textit{c=}$2\sqrt{2}$,周长为$2(1+\sqrt{2}+\sqrt{3})$,则\textit{A}为(\textit{  })

 A.30$\mathrm{{}^\circ}$ B.45$\mathrm{{}^\circ}$

 C.45$\mathrm{{}^\circ}$或75$\mathrm{{}^\circ}$ D.60$\mathrm{{}^\circ}$

 解析:根据正弦定理,得
\begin{align}
\notag
2R&=\frac{a+b+c}{\sin A+\sin B+\sin C}\\\notag
&=\frac{2(1+\sqrt{2}+\sqrt{3})}{\sin A+\sin B+\sin C}\\\notag
&=\frac{C}{\sin C}=\frac{2\sqrt{2}}{\sin 60^{\circ}}=\frac{4\sqrt{6}}{3}
\end{align}
,所以$\sin A+\sin B+\sin 60^{\circ}=\frac{\sqrt{3}}{2\sqrt{2}}+\frac{3}{2}+\frac{3}{2\sqrt{2}}$,所以$\sin A+\sin B=\frac{3+\sqrt{3}}{2\sqrt{2}}$,即
\begin{align}
\notag
\sin A+\sin (A+C)=\frac{3+\sqrt{3}}{2\sqrt{2}}
&\Rightarrow \sin (A+60^{\circ})+\sin A=\frac{3+\sqrt{3}}{2\sqrt{2}}\\\notag
&\Rightarrow\sqrt{3}\sin (A+30^{\circ})=\frac{\sqrt{3}(\sqrt{3}+1)}{2\sqrt{2}}\\\notag
&\Rightarrow\sin (A+30^{\circ})=\frac{\sqrt{6}+\sqrt{2}}{4}
\end{align}
,所以\textit{A+}30$\mathrm{{}^\circ}$\textit{=}75$\mathrm{{}^\circ}$或\textit{A+}30$\mathrm{{}^\circ}$\textit{=}105$\mathrm{{}^\circ}$,所以\textit{A=}45$\mathrm{{}^\circ}$或\textit{A=}75$\mathrm{{}^\circ}$\textit{.}

 答案:C

知识点:解三角形

难度:1

 题目:已知三角形的一边长为7,这条边所对的角为60$\mathrm{{}^\circ}$,另两边之比为3\textit{$:$}2,则这个三角形的面积是\textit{\underbar{     }.~}

 解析:设另两边分别为3\textit{x},2\textit{x},则

cos 60$\mathrm{{}^\circ}$\textit{=}$\frac{9x^2+4x^2-49}{12x^2}$,解得$x=\sqrt{7}$,

故两边长为$3\sqrt{7}$和$2\sqrt{7}$,

所以$S=\frac{1}{2}\times3\sqrt{7}\times2\sqrt{7}\times\sin 60^{\circ}=\frac{21\sqrt{3}}{2}$

 答案:$\frac{21\sqrt{3}}{2}$

知识点:解三角形

难度:1

 题目:已知在$\mathrm{\vartriangle}$\textit{ABC}中,\textit{AC=}2,\textit{AB=}3,$\mathrm{\angle}$\textit{BAC=}60$\mathrm{{}^\circ}$,\textit{AD}是$\mathrm{\vartriangle}$\textit{ABC}的角平分线,则\textit{AD=\underbar{    }.~}

 解析:如图,\textit{S}${}_{\vartriangle }$\textit{${}_{ABC}$=S}${}_{\vartriangle }$\textit{${}_{ABD}+$S}${}_{\vartriangle }$\textit{${}_{ACD}$},

 \includegraphics*[width=1.00in, height=0.81in, keepaspectratio=false]{image1007}

所以$\frac{1}{2}$\textit{$\times$}3\textit{$\times$}2sin 60$\mathrm{{}^\circ}$\textit{=}$\frac{1}{2}$\textit{$\times$}3\textit{AD}sin 30$\mathrm{{}^\circ}$\textit{$+$}$\frac{1}{2}$\textit{$\times$}2\textit{AD$\times$}sin 30$\mathrm{{}^\circ}$,所以$AD=\frac{6\sqrt{3}}{5}$

 答案:$\frac{6\sqrt{3}}{5}$

知识点:解三角形

难度:2

 题目:在$\mathrm{\vartriangle}$\textit{ABC}中,若\textit{AB=a},\textit{AC=b},$\mathrm{\vartriangle}$\textit{BCD}为等边三角形,则当四边形\textit{ABDC}的面积最大时,$\mathrm{\angle}$\textit{BAC=\underbar{     }.~}

 \includegraphics*[width=0.82in, height=0.85in, keepaspectratio=false]{image1013}

 解析:设$\mathrm{\angle}$\textit{BAC=$\theta$},则\textit{BC}${}^{2}$\textit{=a}${}^{2}$\textit{+b}${}^{2}$\textit{-}2\textit{ab}cos \textit{$\theta$.S}${}_{\textrm{四}\textrm{边}\textrm{形}}$\textit{${}_{ABDC}$=S}${}_{\vartriangle }$\textit{${}_{ABC}+$S}${}_{\vartriangle }$\textit{${}_{BCD}$=}$\frac{1}{2}$\textit{ab}sin \textit{$\theta+$}$\frac{\sqrt{3}}{4}$\textit{BC}${}^{2}$\textit{=}$\frac{\sqrt{3}}{4}$(\textit{a}${}^{2}$\textit{$+$b}${}^{2}$)\textit{$+$ab}·sin(\textit{$\theta$-}60$\mathrm{{}^\circ}$),即当$\mathrm{\angle}$\textit{BAC=$\theta$=}150$\mathrm{{}^\circ}$时,\textit{S}${}_{\textrm{四}\textrm{边}\textrm{形}}$\textit{${}_{ABDC}$}取得最大值\textit{.}

 答案:150$\mathrm{{}^\circ}$

知识点:解三角形

难度:2

 题目:已知$\mathrm{\vartriangle}$\textit{ABC}的一个内角为120$\mathrm{{}^\circ}$,并且三边长构成公差为4的等差数列,则$\mathrm{\vartriangle}$\textit{ABC}的面积为\textit{\underbar{     }.~}

 解析:设三角形的三边依次为\textit{a-}4,\textit{a},\textit{a$+$}4,可得\textit{a$+$}4的边所对的角为120$\mathrm{{}^\circ}$\textit{.}

由余弦定理得(\textit{a$+$}4)${}^{2}$\textit{=a}${}^{2}$\textit{$+$}(\textit{a-}4)${}^{2}$\textit{-}2\textit{a}(\textit{a-}4)·cos 120$\mathrm{{}^\circ}$,则\textit{a=}10,所以三边长为6,10,14,

\textit{S}${}_{\vartriangle }$\textit{${}_{ABC}$=}$\frac{1}{2}$\textit{$\times$}6\textit{$\times$}10\textit{$\times$}sin 120$\mathrm{{}^\circ}$\textit{=}$15\sqrt{3}$

 答案:$15\sqrt{3}$

知识点:解三角形

难度:2

 题目:已知$\mathrm{\vartriangle}$\textit{ABC}的重心为\textit{G},角\textit{A},\textit{B},\textit{C}所对的边分别为\textit{a},\textit{b},\textit{c},若$2a\vec{GA}+\sqrt{3}b\vec{GB}+3c\vec{GC}=0$,则sin \textit{A$:$}sin \textit{B$:$}sin \textit{C=\underbar{     }.~}

 解析:因为\textit{G}是$\mathrm{\vartriangle}$\textit{ABC}的重心,所以$\vec{GA}+\vec{GB}+\vec{GC}=0$,又$2a\vec{GA}+\sqrt{3}b\vec{GB}+3c\vec{GC}=0$,所以$2a\vec{GA}+\sqrt{3}b\vec{GB}-3c(\vec{GA}+\vec{GB})=0$,即$(2a-3c)\vec{GA}+(\sqrt{3}b-3c)\vec{GB}=0$,则$
\begin{cases}
2a-3c=0,\\
\sqrt{3}b-3c=0.
\end{cases}$所以\textit{a$:$b$:$c=}3\textit{$:$}2$\sqrt{3}$\textit{$:$}2,由正弦定理,得sin \textit{A$:$}sin \textit{B$:$}sin \textit{C=}3\textit{$:$}2$\sqrt{3}$\textit{$:$}2\textit{.}

 答案:3\textit{$:$}2$\sqrt{3}$\textit{$:$}2

知识点:解三角形

难度:1

 题目:$\mathrm{\vartriangle}$\textit{ABC}的内角\textit{A},\textit{B},\textit{C}的对边分别为\textit{a},\textit{b},\textit{c.}已知$\sin (A+C)=8{\sin \frac{B}{2}}^2$

 (1)求$\cos B$;

 (2)若$a+c=6$,$\mathrm{\vartriangle}$\textit{ABC}的面积为2,求\textit{b.}

 答案:(1)由题设及$A+B+C=$$\piup$,得sin \textit{B=}8sin${}^{2}$$\frac{B}{2}$,

故sin \textit{B=}4(1\textit{-}cos \textit{B})\textit{.}

上式两边平方,整理得17cos${}^{2}$\textit{B-}32cos \textit{B+}15\textit{=}0,

解得cos \textit{B=}1(舍去),cos \textit{B=}$\frac{15}{17}$

(2)由cos \textit{B=}$\frac{15}{17}$得sin \textit{B=}$\frac{8}{17}$,

故\textit{S}${}_{\vartriangle }$\textit{${}_{ABC}$=}$\frac{1}{2}$\textit{ac}sin \textit{B=}$\frac{4}{17}$\textit{ac.}

又\textit{S}${}_{\vartriangle }$\textit{${}_{ABC}$=}2,则\textit{ac=}$\frac{17}{2}$\textit{.}

由余弦定理及$a+c=6$得

\begin{align}
\notag
b^2&=a^2+c^2-2ac\cos B\\\notag
&={(a+c)}^2-2ac(1+\cos B)\\\notag
&=36-2\times\frac{17}{2}\times(1+\frac{15}{17})\\\notag
&=4
\end{align}

所以\textit{b=}2\textit{.}

知识点:

难度:1

 题目:$\mathrm{\vartriangle}$\textit{ABC}的内角\textit{A},\textit{B},\textit{C}的对边分别为\textit{a},\textit{b},\textit{c.}已知$\sin A+\sqrt{3}\cos A=0$,$a=2\sqrt{7}$,$b=2$

 (1)求\textit{c};

 (2)设\textit{D}为\textit{BC}边上一点,且\textit{AD}$\mathrm{\bot}$\textit{AC},求$\mathrm{\vartriangle}$\textit{ABD}的面积\textit{.}

 答案:(1)由已知可得$\tan A=-\sqrt{3}$,所以$A=\frac{2\pi}{3}$

在$\mathrm{\vartriangle}$\textit{ABC}中,由余弦定理得$28=4+c^2-4c\cos\frac{2\pi}{3}$,

即\textit{c}${}^{2}$\textit{$+$}2\textit{c-}24\textit{=}0\textit{.}解得\textit{c=-}6(舍去),\textit{c=}4\textit{.}

(2)由题设可得$\mathrm{\angle}$\textit{CAD=}$\frac{\pi}{2}$,

所以$\mathrm{\angle}$\textit{BAD=}$\mathrm{\angle}$\textit{BAC-}$\mathrm{\angle}$\textit{CAD=}$\frac{\pi}{6}$\textit{.}故$\mathrm{\vartriangle}$\textit{ABD}面积与$\mathrm{\vartriangle}$\textit{ACD}面积的比值为$\frac{1}{2}\cdot AB\cdot AD\cdot \sin\frac{\pi}{6} : \frac{1}{2}\cdot AC\cdot AD=1$

又$\mathrm{\vartriangle}$\textit{ABC}的面积为$\frac{1}{2}$4\textit{$\times$}2sin$\mathrm{\angle}$\textit{BAC=}2$\sqrt{3}$,所以$\mathrm{\vartriangle}$\textit{ABD}的面积为$\sqrt{3}$




知识点:解三角形的实际应用

难度:1

题目:
\includegraphics*[width=0.77in, height=0.59in, keepaspectratio=false]{image1056}

 如图所示,为了测量某湖泊两侧\textit{A},\textit{B}间的距离,某同学首先选定了与\textit{A},\textit{B}不共线的一点\textit{C},然后给出了四种测量方案:($\mathrm{\vartriangle}$\textit{ABC}的角\textit{A},\textit{B},\textit{C}所对的边分别记为\textit{a},\textit{b},\textit{c})

 \textit{①}测量\textit{A},\textit{C},\textit{b ②}测量\textit{a},\textit{b},\textit{C ③}测量\textit{A},\textit{B},\textit{a ④}测量\textit{a},\textit{b},\textit{B}

 则一定能确定\textit{A},\textit{B}间距离的所有方案的序号为(\textit{  })


 A.\textit{①②③} B.\textit{②③④} C.\textit{①③④} D.\textit{①②③④}

 解析:已知三角形的两角及一边,可以确定三角形,故\textit{①③}正确;已知两边及夹角,可以确定三角形,故\textit{②}正确;已知两边与其中一边的对角,满足条件的三角形可能有一个或两个,故\textit{④}错误\textit{.}故选A\textit{.}

 答案:A

知识点:解三角形的实际应用

难度:1

 题目:已知某路边一树干被台风吹断后,树尖与地面成45$\mathrm{{}^\circ}$角,树干也倾斜为与地面成75$\mathrm{{}^\circ}$角,树干底部与树尖着地处相距20 m,则折断点与树干底部的距离是(\textit{  })m\textit{.}



 A.$\frac{20\sqrt{6}}{3}$ B.$10\sqrt{6}$

 C.$\frac{10\sqrt{6}}{3}$ D.$20\sqrt{2}$

 \includegraphics*[width=0.97in, height=0.89in, keepaspectratio=false]{image1061}

 解析:如图,设树干底部为\textit{O},树尖着地处为\textit{B},折断点为\textit{A},则$\mathrm{\angle}$\textit{ABO=}45$\mathrm{{}^\circ}$,$\mathrm{\angle}$\textit{AOB=}75$\mathrm{{}^\circ}$,

所以$\mathrm{\angle}$\textit{OAB=}60$\mathrm{{}^\circ}$\textit{.}

由正弦定理知,$\frac{AO}{\sin 45^{\circ}}=\frac{20}{\sin 60^{\circ}}$,

所以$AO=\frac{20\sin 45^{\circ}}{\sin 60^{\circ}}=\frac{20\sqrt{6}}{3}(m)$

 答案:A

知识点:解三角形的实际应用

难度:1

 题目:已知一艘船以4 km/h的速度与水流方向成120$\mathrm{{}^\circ}$的方向航行,已知河水流速为2 km/h,则经过$\sqrt{3}h$,该船实际航程为(\textit{  })

 A\textit{.}$2\sqrt{15}$km B\textit{.}6 km

 C\textit{.}$2\sqrt{21}$km D\textit{.}8 km

 \includegraphics*[width=1.17in, height=1.01in, keepaspectratio=false]{image1067}

 解析:如图,因为$|\vec{OA}|=2km/h$,$|\vec{OB}|=4km/h$,$\angle AOB=120^{\circ}$,

所以$\mathrm{\angle}$\textit{OAC=}60$\mathrm{{}^\circ}$,
$|\vec{OC}|=\sqrt{2^2+4^2-2\times2\times4\cos60^{\circ}}=2\sqrt{3}(km/h)$

经过$\sqrt{3}$ h,该船的实际航程为$2\sqrt{3}\times\sqrt{3}=6(km)$

 答案:B

知识点:解三角形的实际应用

难度:2

 题目:甲船在\textit{B}岛的正南方10 km处,且甲船以4 km/h的速度向正北方向航行,同时乙船自\textit{B}岛出发以6 km/h的速度向北偏东60$\mathrm{{}^\circ}$的方向行驶,当甲、乙两船相距最近时它们航行的时间是(\textit{  })

 A.$\frac{150}{7}$ min B.$\frac{15}{7}$ h C.21\textit{.}5 min D.2\textit{.}15 h

 \includegraphics*[width=0.90in, height=1.17in, keepaspectratio=false]{image1077}

 解析:如图,设经过\textit{x} h后甲船处于点\textit{P}处,乙船处于点\textit{Q}处,两船的距离为\textit{s},则在$\mathrm{\vartriangle}$\textit{BPQ}中,\textit{BP=}(10\textit{-}4\textit{x}) km,\textit{BQ=}6\textit{x} km,$\mathrm{\angle}$\textit{PBQ=}120$\mathrm{{}^\circ}$,由余弦定理可知\textit{s}${}^{2}$\textit{=PQ}${}^{2}$\textit{=BP}${}^{2}$\textit{+BQ}${}^{2}$\textit{-}2\textit{BP}·\textit{BQ}·cos$\mathrm{\angle}$\textit{PBQ},即\textit{s}${}^{2}$\textit{=}(10\textit{-}4\textit{x})${}^{2}$\textit{+}(6\textit{x})${}^{2}$\textit{-}2(10\textit{-}4\textit{x})·6\textit{x}·cos 120$\mathrm{{}^\circ}$\textit{=}28\textit{x}${}^{2}$\textit{-}20\textit{x+}100\textit{.}

当$x=-\frac{-20}{2\times20}=\frac{5}{14}$时,\textit{s}最小,此时$\frac{5}{14}$ h\textit{=}$\frac{150}{7}$ min\textit{.}

 答案:A

知识点:解三角形的实际应用

难度:2

 题目:已知一货轮航行到\textit{M}处,测得灯塔\textit{S}在货轮的北偏东15$\mathrm{{}^\circ}$,与灯塔\textit{S}相距20海里,随后货轮按北偏西30$\mathrm{{}^\circ}$的方向航行30分后,又测得灯塔在货轮的东北方向,则货轮的速度为(\textit{  })

 A.20($\sqrt{2}+\sqrt{6}$)海里\textit{/}时 B.20($\sqrt{6}-\sqrt{2}$)海里\textit{/}时

 C.20($\sqrt{6}+\sqrt{3}$)海里\textit{/}时 D.20($\sqrt{6}-\sqrt{3}$)海里\textit{/}时

 解析:设货轮航行30分后到达\textit{N}处,

由题意可知$\mathrm{\angle}$\textit{NMS=}45$\mathrm{{}^\circ}$,$\mathrm{\angle}$\textit{MNS=}105$\mathrm{{}^\circ}$,

则$\mathrm{\angle}$\textit{MSN=}180$\mathrm{{}^\circ}$\textit{-}105$\mathrm{{}^\circ}$\textit{-}45$\mathrm{{}^\circ}$\textit{=}30$\mathrm{{}^\circ}$\textit{.}

而\textit{MS=}20海里,在$\mathrm{\vartriangle}$\textit{MNS}中,

由正弦定理得$\frac{MN}{\sin 30^{\circ}}=\frac{MS}{\sin 105^{\circ}}$,

即
\begin{align}
\notag
MN&=\frac{20\sin 30^{\circ}}{\sin 105^{\circ}}=\frac{10}{\sin(60^{\circ}+45^{\circ})}\\\notag
&=\frac{10}{\sin 60^{\circ}\cos 45^{\circ}+\cos 60^{\circ}\sin 45^{\circ}}\\\notag
&=\frac{10}{\frac{\sqrt{6}+\sqrt{2}}{4}}\\\notag
&=10(\sqrt{6}-\sqrt{2})\text{海里}
\end{align}

故货轮的速度为$10(\sqrt{6}-\sqrt{2})\div \frac{1}{2}=20(\sqrt{6}-\sqrt{2})$(海里\textit{/}时)\textit{.}

 答案:B

知识点:解三角形的实际应用

难度:1

 题目:飞机沿水平方向飞行,在\textit{A}处测得正前下方地面目标\textit{C}的俯角为30$\mathrm{{}^\circ}$,向前飞行10 000 m到达\textit{B}处,此时测得正前下方目标\textit{C}的俯角为75$\mathrm{{}^\circ}$,这时飞机与地面目标的水平距离为(\textit{  })

 A.2 500($\sqrt{3}$\textit{-}1) m B.5 000$\sqrt{2}$ m

 C.4 000 m D.4 000$\sqrt{2}$ m

 解析:如图,$\mathrm{\angle}$\textit{BAC=}30$\mathrm{{}^\circ}$,$\mathrm{\angle}$\textit{DBC=}75$\mathrm{{}^\circ}$,\textit{AB=}10 000 m,

 \includegraphics*[width=1.45in, height=1.03in, keepaspectratio=false]{image1096}

所以$\mathrm{\angle}$\textit{ACB=}45$\mathrm{{}^\circ}$\textit{.}

由正弦定理,得$\frac{10000}{\sin 45^{\circ}}=\frac{BC}{\sin 30^{\circ}}$,

又cos75$\mathrm{{}^\circ}$\textit{=}$\frac{BD}{BC}$,

所以\textit{BD=}$\frac{10000\cdot\sin 30^{\circ}}{\sin 45^{\circ}}$·cos 75$\mathrm{{}^\circ}$\textit{=}2 500($\sqrt{3}$\textit{-}1)(m)\textit{.}

 答案:A

知识点:解三角形的实际应用

难度:1

 题目:台风中心从\textit{A}地以20 km/h的速度向东北方向移动,离台风中心30 km内的地区为危险区,城市\textit{B}在\textit{A}的正东40 km处,\textit{B}城市处于危险区内的持续时间为(\textit{  })

 A.0\textit{.}5 h B.1 h 

 C.1\textit{.}5 h D.2 h

 解析:设\textit{t} h后,\textit{B}市处于危险区内,则由余弦定理得(20\textit{t})${}^{2}$\textit{+}40${}^{2}$\textit{-}2\textit{$\times$}20\textit{t$\times$}40cos 45$\mathrm{{}^\circ}$$\mathrm{\le}$30${}^{2}$\textit{.}

化简得4\textit{t}${}^{2}$\textit{-}8$\sqrt{2}$\textit{t+}7$\mathrm{\le}$0,所以\textit{t}${}_{1}$\textit{+t}${}_{2}$\textit{=}2$\sqrt{2}$,\textit{t}${}_{1}$·\textit{t}${}_{2}$\textit{=}$\frac{7}{4}$

从而$|t_1-t_2|=\sqrt{{(t_1+t_2)}^2-4t_1t_2}=1$

 答案:B

知识点:解三角形的实际应用

难度:1

题目: 
\includegraphics*[width=0.82in, height=1.00in, keepaspectratio=false]{image1105}

 如图,已知海岸线上有相距5 n mile的两座灯塔\textit{A},\textit{B},灯塔\textit{B}位于灯塔\textit{A}的正南方向\textit{.}海上停泊着两艘轮船,甲位于灯塔\textit{A}的北偏西75$\mathrm{{}^\circ}$方向,与\textit{A}相距3$\sqrt{2}$ n mile的\textit{D}处;乙船位于灯塔\textit{B}的北偏西60$\mathrm{{}^\circ}$方向,与\textit{B}相距5 n mile的\textit{C}处,则两艘船之间的距离为\textit{\underbar{     }}n mile\textit{.~}

 解析:连接\textit{AC},\textit{BC=AB=}5 n mile,$\mathrm{\angle}$\textit{ABC=}60$\mathrm{{}^\circ}$,

所以$\mathrm{\vartriangle}$\textit{ABC}为等边三角形,所以\textit{AC=}5 n mile,

且$\mathrm{\angle}$\textit{DAC=}180$\mathrm{{}^\circ}$\textit{-}75$\mathrm{{}^\circ}$\textit{-}60$\mathrm{{}^\circ}$\textit{=}45$\mathrm{{}^\circ}$\textit{.}

在$\mathrm{\vartriangle}$\textit{ACD}中,由余弦定理得\textit{CD}${}^{2}$\textit{=}(3$\sqrt{2}$)${}^{2}$\textit{+}5${}^{2}$\textit{-}2\textit{$\times$}3$\sqrt{2}$\textit{$\times$}5\textit{$\times$}cos 45$\mathrm{{}^\circ}$\textit{=}13,所以\textit{CD=}$\sqrt{13}$ n mile\textit{.}

故两艘船之间的距离为$\sqrt{13}$ n mile\textit{.}

 答案:$\sqrt{13}$

知识点:解三角形的实际应用

难度:1

题目:
\includegraphics*[width=0.85in, height=1.06in, keepaspectratio=false]{image1112}

 如图,山顶上有一座电视塔,在塔顶\textit{B}处测得地面上一点\textit{A}的俯角\textit{$\alpha$=}60$\mathrm{{}^\circ}$,在塔底\textit{C}处测得点\textit{A}的俯角\textit{$\beta$=}45$\mathrm{{}^\circ}$\textit{.}已知塔高60 m,则山高为\textit{\underbar{     }.~}

 解析:在$\mathrm{\vartriangle}$\textit{ABC}中,\textit{BC=}60 m,$\mathrm{\angle}$\textit{BAC=}15$\mathrm{{}^\circ}$,$\mathrm{\angle}$\textit{ABC=}30$\mathrm{{}^\circ}$,

由正弦定理,得\textit{AC=}$\frac{60\sin 30^{\circ}}{\sin 15^{\circ}}$\textit{=}30($\sqrt{6}+\sqrt{2}$)(m)\textit{.}

所以\textit{CD=AC}·sin 45$\mathrm{{}^\circ}$\textit{=}30($\sqrt{3}+1$)(m)\textit{.}

 答案:30($\sqrt{3}+1$)m

知识点:解三角形的实际应用

难度:2

题目:
\includegraphics*[width=1.42in, height=0.97in, keepaspectratio=false]{image1117}

 如图,为测量山高\textit{MN},选择\textit{A}和另一座山的山顶\textit{C}为测量观测点\textit{.}从点\textit{A}测得点\textit{M}的仰角$\mathrm{\angle}$\textit{MAN=}60$\mathrm{{}^\circ}$,点\textit{C}的仰角$\mathrm{\angle}$\textit{CAB=}45$\mathrm{{}^\circ}$及$\mathrm{\angle}$\textit{MAC=}75$\mathrm{{}^\circ}$,从点\textit{C}测得$\mathrm{\angle}$\textit{MCA=}60$\mathrm{{}^\circ}$\textit{.}已知山高\textit{BC=}50 m,则山高\textit{MN=\underbar{     }} m\textit{.~}

 解析:在Rt$\mathrm{\vartriangle}$\textit{ABC}中,$\mathrm{\angle}$\textit{CAB=}45$\mathrm{{}^\circ}$,\textit{BC=}50 m,所以\textit{AC=}50$\sqrt{2}$ m\textit{.}

在$\mathrm{\vartriangle}$\textit{AMC}中,$\mathrm{\angle}$\textit{MAC=}75$\mathrm{{}^\circ}$,$\mathrm{\angle}$\textit{MCA=}60$\mathrm{{}^\circ}$,从而$\mathrm{\angle}$\textit{AMC=}45$\mathrm{{}^\circ}$,

由正弦定理得,$\frac{AC}{\sin45^{\circ}}=\frac{AN}{\sin 60^{\circ}}$,因此\textit{AM=}50$\sqrt{3}$ m\textit{.}

在Rt$\mathrm{\vartriangle}$\textit{MNA}中,\textit{AM=}50$\sqrt{3}$ m,$\mathrm{\angle}$\textit{MAN=}60$\mathrm{{}^\circ}$,由$\frac{MN}{AN}$\textit{=}sin 60$\mathrm{{}^\circ}$,得\textit{MN=}50$\sqrt{3}\times\frac{\sqrt{3}}{2}$\textit{=}75(m)\textit{.}

 答案:75

知识点:解三角形的实际应用

难度:1

题目:
\includegraphics*[width=1.52in, height=0.74in, keepaspectratio=false]{image1124}

 如图,\textit{CM},\textit{CN}为某公园景观湖畔的两条木栈道,$\mathrm{\angle}$\textit{MCN=}120$\mathrm{{}^\circ}$\textit{.}现拟在两条木栈道的\textit{A},\textit{B}两处设置观景台,记\textit{BC=a},\textit{AC=b},\textit{AB=c}(单位:百米)\textit{.}

 (1)若\textit{a},\textit{b},\textit{c}成等差数列,且公差为4,求\textit{b}的值;

 (2)已知\textit{AB=}12,记$\mathrm{\angle}$\textit{ABC=$\theta$},试用\textit{$\theta$}表示观景路线\textit{A-C-B}的长,并求观景路线\textit{A-C-B}长的最大值\textit{.}

 答案:(1)因为\textit{a},\textit{b},\textit{c}成等差数列,且公差为4,

所以\textit{a=b-}4,\textit{c=b$+$}4,

因为$\mathrm{\angle}$\textit{MCN=}120$\mathrm{{}^\circ}$,

所以由余弦定理得,(\textit{b$+$}4)${}^{2}$\textit{=}(\textit{b-}4)${}^{2}$\textit{$+$b}${}^{2}$\textit{-}2\textit{b}(\textit{b-}4)cos 120$\mathrm{{}^\circ}$,解得\textit{b=}10\textit{.}

(2)由题意,得$\frac{AC}{\sin\theta}=\frac{BC}{\sin(60^{\circ}-\theta)}=\frac{12}{\sin120^{\circ}}$,

所以$AC+8\sqrt{3}\sin\theta$ ,$BC=8\sqrt{3}\sin(60^{\circ}-\theta)$,

所以观景路线\textit{A-C-B}的长
$AC+BC=8\sqrt{3}\sin\theta+8\sqrt{3}\sin(60^{\circ}-\theta)=8\sqrt{3}\sin(60^{\circ}+\theta)(0^{\circ}<\theta<60^{\circ})$

所以当\textit{$\theta$=}30$\mathrm{{}^\circ}$时,观景路线\textit{A-C-B}长的最大值为$8\sqrt{3}$百米\textit{.}

知识点:解三角形的实际应用

难度:1

题目:
\includegraphics*[width=1.14in, height=0.64in, keepaspectratio=false]{image1133}

 如图,一艘船由西向东航行,测得某岛\textit{M}的方位角为\textit{$\alpha$},前进5 km后测得此岛的方位角为\textit{$\beta$.}已知该岛周围3 km内有暗礁,现该船继续东行\textit{.}

 (1)若\textit{$\alpha$=}2\textit{$\beta$=}60$\mathrm{{}^\circ}$,问该船有无触礁危险?

 (2)当\textit{$\alpha$}与\textit{$\beta$}满足什么条件时,该船没有触礁的危险?

 答案:(1)设岛\textit{M}到直线\textit{AB}的距离\textit{MC}为\textit{d} km,则

\textit{AC=d}tan \textit{$\alpha$} km,\textit{BC=d}tan \textit{$\beta$} km\textit{.}

由\textit{AC-BC=AB},

得\textit{d}tan \textit{$\alpha$-d}tan \textit{$\beta$=}5,\textit{d=}$\frac{5}{\tan \alpha\cdot\tan\beta}$.

当\textit{$\alpha$=}2\textit{$\beta$=}60$\mathrm{{}^\circ}$时,\textit{d=}$\frac{5}{\sqrt{3}\cdot\frac{\sqrt{3}}{3}}=\frac{5\sqrt{3}}{2}>3$,

所以此时没有触礁的危险\textit{.}

(2)方法一:要使船没有触礁危险,只要使\textit{d$>$}3,

即$\frac{5}{\tan\alpha\cdot\tan\beta}$.

因为0\textit{$<$$\beta$$<$$\alpha$$<\frac{\pi}{2}$},所以tan \textit{$\alpha$-}tan \textit{$\beta$$>$}0,

所以tan \textit{$\alpha$-}tan \textit{$\beta$$<\frac{5}{3}$},

所以当\textit{$\alpha$},\textit{$\beta$}满足tan \textit{$\alpha$-}tan \textit{$\beta$$<\frac{5}{3}$}时,该船没有触礁的危险\textit{.}

方法二:设\textit{CM=x} km,由$\frac{AB}{\sin\angle AMB}=\frac{BM}{\sin\angle MAB}$,

即$\frac{5}{\sin(\alpha-\beta)}=\frac{x}{\cos\alpha\cdot\cos\beta}$,解得$x=\frac{5\cos\alpha\cdot\cos\beta}{\sin(\alpha-\beta)}$,

所以当$\frac{5\cos\alpha\cdot\cos\beta}{\sin(\alpha-\beta)}>3$时没有触礁危险\textit{.}

知识点:解三角形的实际应用

难度:2

 13\textit{.}某海军护航舰艇在某海域执行护航任务时,收到某渔船在航行中发出的求救信号,海军舰艇在\textit{A}处获悉后,立即测出该渔船在方位角为45$\mathrm{{}^\circ}$、距离\textit{A}为10 n mile的\textit{C}处,并测得渔船正沿方位角为105$\mathrm{{}^\circ}$的方向,以9 n mile/h的速度航行,海军舰艇立即以21 n mile/h的速度前去营救,试问舰艇应按照怎样的航向前进?并求出靠近渔船所用的时间(角度精确到0\textit{.}1$\mathrm{{}^\circ}$,时间精确到1 min)\textit{.}

 \includegraphics*[width=1.22in, height=0.89in, keepaspectratio=false]{image1145}


答案:如图,设舰艇从\textit{A}处靠近渔船所用的时间为\textit{x} h,

则\textit{AB=}21\textit{x} n mile,

\textit{BC=}9\textit{x} n mile,

\textit{AC=}10 n mile,

$\mathrm{\angle}$\textit{ACB=}$\mathrm{\angle}$1\textit{+}$\mathrm{\angle}$2\textit{=}45$\mathrm{{}^\circ}$\textit{+}(180$\mathrm{{}^\circ}$\textit{-}105$\mathrm{{}^\circ}$)\textit{=}120$\mathrm{{}^\circ}$,

根据余弦定理可得\textit{AB}${}^{2}$\textit{=AC}${}^{2}$\textit{$+$BC}${}^{2}$\textit{-}2\textit{AC}·\textit{BC}·cos 120$\mathrm{{}^\circ}$,

即(21\textit{x})${}^{2}$\textit{=}10${}^{2}$\textit{$+$}(9\textit{x})${}^{2}$\textit{-}2\textit{$\times$}10\textit{$\times$}9\textit{x}cos 120$\mathrm{{}^\circ}$,

亦即36\textit{x}${}^{2}$\textit{-}9\textit{x-}10\textit{=}0,

解得\textit{x}${}_{1}$\textit{=}$\frac{2}{3}$,\textit{x}${}_{2}$\textit{=-}$\frac{5}{12}$(舍去),

所以\textit{AB=}14 n mile,\textit{BC=}6 n mile\textit{.}

由余弦定理可得
$\cos\angle BAC=\frac{{AB}^2+{AC}^2\cdot{BC}^2}{2AB\cdot AC}=\frac{14^2+10^2\cdot6^2}{2\times14\times10}\approx0.9286$
,所以$\mathrm{\angle}$\textit{BAC}$\mathrm{\approx}$21\textit{.}8$\mathrm{{}^\circ}$,

所以方位角为45$\mathrm{{}^\circ}$\textit{+}21\textit{.}8$\mathrm{{}^\circ}$\textit{=}66\textit{.}8$\mathrm{{}^\circ}$,又因为$\frac{2}{3}$ h\textit{=}40 min,所以舰艇应以北偏东66\textit{.}8$\mathrm{{}^\circ}$的方向航行,靠近渔船需要40 min\textit{.}


知识点:不等式基本性质

难度:1

 
题目:大桥桥头竖立的``限重40吨''的警示牌,是指示司机要安全通过该桥,应使车和货的总重量\textit{T}(吨)满足关系为(\textit{  })

 \textit{      } \textit{      } \textit{     }

 A.\textit{T$<$}40 B.\textit{T$>$}40

 C.\textit{T}$\mathrm{\le}$40 D.\textit{T}$\mathrm{\ge}$40

 答案:C


知识点:不等式基本性质

难度:1

题目:
把下列各题中的``\textit{=}''全部改成``\textit{$<$}'',结论仍然成立的是 (\textit{  })

 A.如果\textit{a=b},\textit{c=d},那么\textit{a-c=b-d}

 B.如果\textit{a=b},\textit{c=d},那么\textit{ac=bd}

 C.如果\textit{a=b},\textit{c=d},且\textit{cd}$\mathrm{\neq}$0,那么\textit{$\frac{a}{c} = \frac{b}{d}$}

 D.如果\textit{a=b},那么\textit{a}${}^{3}$\textit{=b}${}^{3}$

 解析:由不等式性质知只有D选项仍然成立,即若\textit{a$<$b},则\textit{a}${}^{3}$\textit{$<$b}${}^{3}$\textit{.}

 答案:D

知识点:不等式基本性质

难度:1


 题目:
若\textit{a$>$b},则下列各式正确的是(\textit{  })

 A.\textit{a}lg \textit{x$>$b}lg \textit{x} B.\textit{ax}${}^{2}$\textit{$>$bx}${}^{2}$

 C.\textit{a}${}^{2}$\textit{$>$b}${}^{2}$ D.\textit{a}·2\textit{${}^{x}$$>$b}·2\textit{${}^{x}$}

 解析:对任意的\textit{x},2\textit{${}^{x}$$>$}0\textit{.}又因为\textit{a$>$b},所以\textit{a}·2\textit{${}^{x}$$>$b}·2\textit{${}^{x}$.}

 答案:D

知识点:不等式基本性质

难度:1

 题目: 若\textit{a$>$b$>$c},则\textit{$\frac{1}{b-c}+\frac{1}{c-a}$}的值为(\textit{  })

 A.正数 B.负数

 C.非正数 D.非负数

 解析:因为\textit{a$>$b$>$c},所以\textit{b-c$>$}0,\textit{c-a$<$}0,\textit{b-a$<$}0\textit{.}

所以\textit{$\frac{1}{b-c}+\frac{1}{c-a}=\frac{b-a}{(b-c)(c-a)}$}\textit{.}

 答案:A

知识点:不等式基本性质

难度:1

 题目:若\textit{$\alpha$},\textit{$\beta$}满足\textit{$-\frac{\pi}{2}<$$\alpha$$<$$\beta$$<\frac{pi}{2}$},则2\textit{$\alpha$-$\beta$}的取值范围是(\textit{  })

 A.\textit{-}$\piup$$\mathrm{\le}$2\textit{$\alpha$-$\beta$$<$}0 B.\textit{-}$\piup$\textit{$<$}2\textit{$\alpha$-$\beta$$<$}$\piup$

 C.\textit{-}\textit{$\frac{3\pi}{2}<$}2\textit{$\alpha$-$\beta$$<\frac{\pi}{2}$} D.0\textit{$<$}2\textit{$\alpha$-$\beta$$<$}$\piup$

 解析:由\textit{-}\textit{$\frac{\pi}{2}<$$\alpha$$<$$\beta$$<\frac{\pi}{2}$},得\textit{-}$\piup$\textit{$-\frac{\pi}{2}<$$\alpha$-$\beta$$<\frac{\pi}{2}$}0,\textit{-}\textit{$\frac{\pi}{2}<$$\alpha$$<\frac{\pi}{2}$}\textit{.}

所以\textit{-}\textit{$\frac{3\pi}{2}<$$\alpha$+}(\textit{$\alpha$-$\beta$})\textit{$<\frac{\pi}{2}$},

即\textit{-}\textit{$\frac{3\pi}{2}<$}2\textit{$\alpha$-$\beta$$<\frac{\pi}{2}$}\textit{.}

 答案:C


知识点:不等式基本性质

难度:1

题目: 若1\textit{$<$a$<$}3,\textit{-}4\textit{$<$b$<$}2,则\textit{$a-|b|$}的取值范围是\textit{\underbar{     }.~}

 解析:因为\textit{-}4\textit{$<$b$<$}2,所以0$\mathrm{\le}$\textit{$|b|<$}4,

所以\textit{-}4\textit{$<-|b|$}$\mathrm{\le}$0\textit{.}

又因为1\textit{$<$a$<$}3,所以\textit{-}3\textit{$<a-|b|<$}3\textit{.}

 答案:(\textit{-}3,3)

知识点:不等式基本性质

难度:1

题目: 
已知1\textit{$<$a$<$b},比较大小:log\textit{${}_{a}$b\underbar{     }}log\textit{${}_{b}$a}(填``\textit{$>$}''``\textit{$<$}''或``\textit{=}'')\textit{.~}

 解析:log\textit{${}_{b}a=\frac{1}{\log_ab}$},因为1\textit{$<$a$<$b},所以log\textit{${}_{a}$b$>$}1\textit{.}

所以log\textit{${}_{b}$a$<$}1,

所以log\textit{${}_{a}$b$>$}log\textit{${}_{b}$a.}

 答案:\textit{$>$}

知识点:不等式基本性质

难度:1

题目:
已知\textit{a$>$b$>$c$>$d$>$}0,且\textit{a},\textit{b},\textit{c},\textit{d}成等差数列,则$\lg\frac{a}{b}$, $\lg\frac{b}{c}$,$\lg\frac{c}{d}$的大小顺序为\textit{\underbar{     }.~}

 解析:因为\textit{a},\textit{b},\textit{c},\textit{d}成等差数列,

所以2\textit{b=a+c},2\textit{c=b+d.}

所以$\frac{a}{b}-\frac{b}{c}=\frac{ac-b^2}{bc}=\frac{ac-(\frac{a+c}{2})^2}{bc}=-\frac{(a-c)^2}{4bc}<0 $

所以$ \frac{a}{b}< \frac{b}{c}$\textit{.}

同理$ \frac{b}{c}< \frac{c}{d}$,所以$ 0<\frac{a}{b}<\frac{b}{c}<\frac{c}{d}$,

所以$\lg\frac{a}{b}< \lg\frac{b}{c}< \lg\frac{c}{d}$\textit{.}

 答案$\lg\frac{a}{b}< \lg\frac{b}{c}< \lg\frac{c}{d}$

知识点:不等式基本性质

难度:1

题目: 
若\textit{a}$\mathrm{\neq}$\textit{-}1,且\textit{a}$\mathrm{\in}$R,试比较$\frac{1}{1+a}$与1\textit{-a}的大小\textit{.}

 答案:因为$\frac{1}{1+a}-(1-a)=\frac{a^2}{1+a}$,

所以当\textit{a$>$-}1且\textit{a}$\mathrm{\neq}$0时,\textit{$\frac{1}{1+a}>$}1\textit{-a};

当\textit{a$<$-}1时,\textit{$\frac{1}{1+a}<$}1\textit{-a};

当\textit{a=}0时,\textit{$\frac{1}{1+a}$=}1\textit{-a.}

 
用锤子以均匀的力敲击铁钉进入木板,随着铁钉的深入,铁钉所受的阻力会越来越大,每次敲击后铁钉进入木板的长度满足后一次为前一次的$\frac{1}{k}$\textit{.}已知一个铁钉受击三次后全部进入木板,且第一次受击后铁钉进入木板的部分是钉长的$\frac{4}{7}$,请从这个实例中提炼出一个不等式组\textit{.}

 解由题意知,第二次受击后铁钉没有全部进入木板;第三次受击后铁钉全部进入木板,所以
\[
\begin{cases}
\frac{4}{7}+\frac{4}{7k}<1,\\
\frac{4}{7}+\frac{4}{7k}+\frac{4}{7k^2} \ge 1.
\end{cases}
\]



知识点:不等式基本性质

难度:2

 题目: 设\textit{a},\textit{b}$\mathrm{\in}$R,若\textit{$a-|b|>$}0,则下列不等式正确的是(\textit{  })

 A.\textit{b-a$>$}0 B.\textit{a}${}^{3}$\textit{$+$b}${}^{3}$\textit{$<$}0

 C.\textit{b$+$a$<$}0 D.\textit{a}${}^{2}$\textit{-b}${}^{2}$\textit{$>$}0

 解析:利用赋值法,令\textit{a=}1,\textit{b=}0,排除A,B,C,故选D\textit{.}

 答案:D

知识点:不等式基本性质

难度:2

 题目: 如果\textit{a$>$}0,且\textit{a}$\mathrm{\neq}$1,\textit{M=}log\textit{${}_{a}$}(\textit{a}${}^{3}$\textit{$+$}1),\textit{N=}log\textit{${}_{a}$}(\textit{a}${}^{2}$\textit{$+$}1),那么\textit{M},\textit{N}的大小关系为(\textit{  })

 A.\textit{M$>$N} B.\textit{M$<$N}

 C.\textit{M=N} D.无法确定


 解析:当\textit{a$>$}1时,\textit{a}${}^{3}$\textit{$+$}1\textit{$>$a}${}^{2}$\textit{$+$}1,\textit{y=}log\textit{${}_{a}$x}是增加的,

所以log\textit{${}_{a}$}(\textit{a}${}^{3}$\textit{$+$}1)\textit{$>$}log\textit{${}_{a}$}(\textit{a}${}^{2}$\textit{$+$}1)\textit{.}

当0\textit{$<$a$<$}1时,\textit{a}${}^{3}$\textit{$+$}1\textit{$<$a}${}^{2}$\textit{$+$}1,\textit{y=}log\textit{${}_{a}$x}是减少的\textit{.}

所以log\textit{${}_{a}$}(\textit{a}${}^{3}$\textit{$+$}1)\textit{$>$}log\textit{${}_{a}$}(\textit{a}${}^{2}$\textit{$+$}1)\textit{.}

故选A\textit{.}

 答案:A

知识点:不等式基本性质

难度:2

 题目: 下列不等式:\textit{①}${}^{2}$\textit{$+$}3\textit{$>$}2\textit{x}(\textit{x}$\mathrm{\in}$R);\textit{②a}${}^{3}$\textit{$+$b}${}^{3}$$\mathrm{\ge}$\textit{a}${}^{2}$\textit{b$+$ab}${}^{2}$(\textit{a},\textit{b}$\mathrm{\in}$R);\textit{③}\textit{a}${}^{2}$\textit{$+$b}${}^{2}$$\mathrm{\ge}$2(\textit{a-b-}1)中,正确的个数为 (\textit{  })

 A.0 B.1 C.2 D.3

 解析:对于\textit{①},\textit{x}${}^{2}$\textit{$+$}3\textit{-}2\textit{x=}(\textit{x-}1)${}^{2}$\textit{$+$}2\textit{$>$}0恒成立,故\textit{①}正确;

对于\textit{②},\textit{a}${}^{3}$\textit{+b}${}^{3}$\textit{-a}${}^{2}$\textit{b-ab}${}^{2}$\textit{=a}${}^{2}$(\textit{a-b})\textit{+b}${}^{2}$(\textit{b-a})\textit{=}(\textit{a-b})(\textit{a}${}^{2}$\textit{-b}${}^{2}$)\textit{=}(\textit{a-b})${}^{2}$(\textit{a+b}),由\textit{a},\textit{b}$\mathrm{\in}$R,得(\textit{a-b})${}^{2}$$\mathrm{\ge}$0,而\textit{a+b$>$}0或\textit{a+b=}0或\textit{a+b$<$}0,故\textit{②}不正确;

对于\textit{③},\textit{a}${}^{2}$\textit{+b}${}^{2}$\textit{-}2\textit{a+}2\textit{b+}2\textit{=a}${}^{2}$\textit{-}2\textit{a+}1\textit{+b}${}^{2}$\textit{+}2\textit{b+}1\textit{=}(\textit{a-}1)${}^{2}$\textit{+}(\textit{b+}1)${}^{2}$$\mathrm{\ge}$0,故\textit{③}正确,故选C\textit{.}

 答案:C

知识点:不等式基本性质

难度:2

 题目:
下列各式中,对任何实数\textit{x}都成立的一个式子是(\textit{  })

 A.lg(\textit{x}${}^{2}$\textit{$+$}1)$\mathrm{\ge}$lg 2\textit{x} B.\textit{x}${}^{2}$\textit{$+$}1\textit{$>$}2\textit{x}

 C.$\frac{1}{x^2+1}$
$\mathrm{\le}$1 D.\textit{x$+$}$\frac{1}{x}$$\mathrm{\ge}$2

 解析:A中\textit{x$>$}0;B中当\textit{x=}1时,\textit{x}${}^{2}$\textit{$+$}1\textit{=}2\textit{x};C中对任意\textit{x},\textit{x}${}^{2}$\textit{$+$}1$\mathrm{\ge}$1恒成立,故$\frac{1}{x^2+1}$$\mathrm{\le}$1恒成立;D中当\textit{x$<$}0时,\textit{x$+\frac{1}{x}$}\textit{$<$}0\textit{.}

 答案:C


知识点:不等式基本性质

难度:2


 题目:
g糖水中有\textit{a}(\textit{b$>$a$>$}0) g糖,若再添加\textit{m}(\textit{m$>$}0) g糖,则糖水就变甜了,根据这一事实可以提炼的一个不等式是\textit{\underbar{          }.~}

 解析:由题意知原有的\textit{b} g糖水中,再添加\textit{m}(\textit{m$>$}0)g糖后,糖水变甜了,说明糖水中的糖的质量分数变大了,则$\frac{a+m}{b+m}>\frac{a}{b}$(\textit{b$>$a$>$}0,\textit{m$>$}0)\textit{.}

 答案:$\frac{a+m}{b+m}>\frac{a}{b}$(\textit{b$>$a$>$}0,\textit{m$>$}0)

知识点:不等式基本性质

难度:2

 题目: 给出三个条件:\textit{①}\textit{ac}${}^{2}$\textit{$>$bc}${}^{2}$;\textit{②}$\frac{a}{c}>\frac{b}{c}$;\textit{③}\textit{a}${}^{2}$\textit{$>$b}${}^{2}$,其中能推出\textit{a$>$b}的条件有\textit{\underbar{     }}个\textit{.~}

 解析:只有\textit{①}能推出\textit{a$>$b.}

 答案:1

知识点:不等式基本性质

难度:2

 题目:
已知1$\mathrm{\le}$\textit{a$+$b}$\mathrm{\le}$4,\textit{-}1$\mathrm{\le}$\textit{a-b}$\mathrm{\le}$2,求4\textit{a-}2\textit{b}的取值范围\textit{.}

 答案:令4\textit{a-}2\textit{b=x}(\textit{a$+$b})\textit{$+$y}(\textit{a-b}),

所以4\textit{a-}2\textit{b=}(\textit{x$+$y})\textit{a$+$}(\textit{x-y})\textit{b.}

所以
\[
\begin{cases}
x+y=4,\\
x-y=-2.
\end{cases}
\]

所以
\[
\begin{cases}
x=1,\\
y=3.
\end{cases}
\]


所以
\[
\begin{cases}
1\leqslant a+b\leqslant 4,\\
-3 \leqslant 3(a-b) \leqslant 6.
\end{cases}
\]
所以\textit{-}2$\mathrm{\le}$4\textit{a-}2\textit{b}$\mathrm{\le}$10\textit{.}

知识点:不等式基本性质

难度:2

 题目: 已知\textit{a$>$}0,\textit{b$>$}0,且\textit{a}$\mathrm{\neq}$\textit{b},比较$\frac{a^2}{b}+\frac{b^2}{a}$与\textit{a$+$b}的大小\textit{.}

 答案:因为


\begin{align}
\notag
(\frac{a^2}{b}+\frac{b^2}{a}) - (a+b) = \frac{a^2}{b}-b+\frac{b^2}{a}-a\\\notag
=\frac{a^2-b^2}{b}+\frac{b^2-a^2}{a}\\\notag
=(a^2-b^2)(\frac{1}{b}-\frac{1}{a})\\\notag
=(a^2-b^2)\frac{a-b}{ab}\\\notag
=\frac{(a-b)^2(a+b)}{ab}
\end{align}
,

又\textit{a$>$}0,\textit{b$>$}0,\textit{a}$\mathrm{\neq}$\textit{b},

所以(\textit{a-b})${}^{2}$\textit{$>$}0,\textit{a+b$>$}0,\textit{ab$>$}0\textit{.}

所以$(\frac{a^2}{b}+\frac{b^2}{a})-(a+b)>0$\textit{.}

所以$\frac{a^2}{b}+\frac{b^2}{a}>a+b$

 
实数\textit{a},\textit{b},\textit{c},\textit{d}满足下列三个条件:\textit{①d$>$c};\textit{②a+b=c+d};\textit{③a+d$<$b+c.}请将\textit{a},\textit{b},\textit{c},\textit{d}按照从小到大的顺序排列,并证明你的结论\textit{.}

 解结论是:\textit{a$<$c$<$d$<$b.}

证明如下:因为\textit{a+d$<$b+c},

所以\textit{d-b$<$c-a. ①}

又因为\textit{a+b=c+d},所以\textit{c-a=b-d. ②}

所以由\textit{①②},得
\[
\begin{cases}
d-b<b-d,\\
a-c<c-a.
\end{cases}
\Rightarrow
\begin{cases}
d<b,\\
a<c.
\end{cases}
\]

由\textit{d$>$c},得\textit{a$<$c$<$d$<$b.}



知识点:一元二次不等式的解法

难度:1

 题目:当\textit{$0<t<1$}时,不等式$ (x-t)(x-\frac{1}{t})$的解集为 (\textit{  })

 

 A.$\mathrm{\{}$ ${x|\frac{1}{t}<x<t}$ $\mathrm{\}}$ B.$\mathrm{\{}$$\lbrace x|x>\frac{1}{t}$或$x<t\rbrace$ $\mathrm{\}}$

 C.$\mathrm{\{}$$\lbrace x|x<\frac{1}{t}$或$x>t\rbrace$ $\mathrm{\}}$ D.$\mathrm{\{}$${x|t<x<\frac{1}{t}}$ $\mathrm{\}}$

 解析:因为\textit{t}$\mathrm{\in}$(0,1),所以$\frac{1}{t}>t$

所以由$(x-t)(x-\frac{1}{t})>0$,得$x>\frac{1}{t}$或$x<t$

 答案:B

知识点:一元二次不等式的解法

难度:1

 题目:
已知一元二次不等式\textit{f}(\textit{x})\textit{$<$}0的解集为$\mathrm{\}}$$x|x<-1$或$x>\frac{1}{2}$$\mathrm{\}}$,则\textit{f}(10\textit{${}^{x}$})\textit{$>$}0的解集为(\textit{ })

 A.$\mathrm{\{}$\textit{x$ |$x$<$-}1或\textit{x$>$-}lg 2$\mathrm{\}}$

 B.$\mathrm{\{}$\textit{x$ |$-}1\textit{$<$x$<$-}lg 2$\mathrm{\}}$

 C.$\mathrm{\{}$\textit{x$ |$x$>$-}lg 2$\mathrm{\}}$

 D.$\mathrm{\{}$\textit{x$ |$x$<$-}lg 2$\mathrm{\}}$

 解析:由题意可知\textit{f}(\textit{x})\textit{$>$}0的解集为${x|-1<x<\frac{1}{2}}$,因为$0<10^x<\frac{1}{2}$,所以$x<\lg\frac{1}{2=-\lg2}$

 答案:D

知识点:一元二次不等式的解法

难度:2

 题目:设集合\textit{A=}$\mathrm{\{}$\textit{x$ |$}6\textit{$+$}5\textit{x-x}${}^{2}$\textit{$>$}0$\mathrm{\}}$,\textit{B=}$\mathrm{\{}$\textit{x$ |$a}${}^{2}$\textit{-x}${}^{2}$\textit{$<$}0$\mathrm{\}}$,若\textit{A}$\mathrm{\cap}$\textit{B=}$\mathrm{\varnothing }$,则\textit{a}的取值范围是(\textit{  })

 A.$\mathrm{\{}$\textit{a$ |$a}$\mathrm{\ge}$6$\mathrm{\}}$ B.$\mathrm{\{}$\textit{a$ |$a$>$}6$\mathrm{\}}$

 C.$\mathrm{\{}$\textit{a$ |$a}$\mathrm{\le}$\textit{-}6或\textit{a}$\mathrm{\ge}$6$\mathrm{\}}$ D.$\mathrm{\{}$\textit{a$ |$a}$\mathrm{\le}$\textit{-}6$\mathrm{\}}$

 解析:由6\textit{$+$}5\textit{x-x}${}^{2}$\textit{$>$}0,得\textit{x}${}^{2}$\textit{-}5\textit{x-}6\textit{$<$}0,解得\textit{-}1\textit{$<$x$<$}6\textit{.}

由\textit{a}${}^{2}$\textit{-x}${}^{2}$\textit{$<$}0,得\textit{x$>$$|a|$}或\textit{x$<$-$|a|$.}

由\textit{A}$\mathrm{\cap}$\textit{B=}$\mathrm{\varnothing }$,得\textit{$|a|$}$\mathrm{\ge}$6,所以\textit{a}$\mathrm{\ge}$6或\textit{a}$\mathrm{\le}$\textit{-}6\textit{.}

 答案:C

知识点:一元二次不等式的解法

难度:1

 题目: 若对一切实数\textit{x},不等式\textit{x}${}^{2}$\textit{+a$|a|+$}1$\mathrm{\ge}$0恒成立,则实数\textit{a}的取值范围是(\textit{  })

 A.(\textit{-$\infty$},\textit{-}2] B.[\textit{-}2,2]

 C.[\textit{-}2,\textit{$+\infty$}) D.[0,\textit{$+\infty$})

 解析:令\textit{t={$|x|$}},则\textit{t}$\mathrm{\ge}$0,所以\textit{t}${}^{2}$\textit{$+at+$}1$\mathrm{\ge}$0对\textit{t}$\mathrm{\ge}$0恒成立,当\textit{a}$\mathrm{\ge}$0时,显然不等式恒成立\textit{.}

当\textit{a$<$}0时,\textit{y=t}${}^{2}$\textit{$+at+$}1在[0,\textit{$+\infty$})上的最小值为$ 1-\frac{a^2}{4}$,由题意得$ 1-\frac{a^2}{4}\ge 0$,解得\textit{-}2$\mathrm{\le}$\textit{a}$\mathrm{\le}$2,所以\textit{-}2$\mathrm{\le}$\textit{a$<$}0\textit{.}

综上,\textit{a}$\mathrm{\ge}$\textit{-}2,故选C\textit{.}

 答案:C

知识点:一元二次不等式的解法

难度:2

 题目:已知不等式\textit{ax}${}^{2}$\textit{$+bx+c>$}0的解集是(\textit{-$\infty$},\textit{-}1)$\mathrm{\cup}$(3,\textit{$+\infty$}),则对函数\textit{f}(\textit{x})\textit{=ax}${}^{2}$\textit{$+bx+c$},下列不等式成立的是(\textit{  })

 A.$f(4)>f(0)>f(1)$ B.$f(4)>f(1)>f(0)$

 C.$f(0)>f(1)>f(4)$ D.$f(0)>f(4)>f(1)$

 解析:由题意知\textit{-}1,3是方程\textit{ax}${}^{2}$\textit{$+bx+c=$}0的两根,且\textit{a$>$}0,所以
\[
\begin{cases}
-1+3 = -\frac{b}{a},\\
-1 \times 3 = \frac{c}{a}.
\end{cases}
\]

所以
\[
\begin{cases}
\frac{b}{a} = -2,\\
\frac{c}{a}=-3.
\end{cases}
\]


对二次函数\textit{f}(\textit{x})\textit{=ax}${}^{2}$\textit{$+bx+c$}来说,其图像的对称轴为$ x = -\frac{b}{2a}=1$,且开口向上\textit{.}

由于$  |4-1|>|1-0|$,所以\textit{f}\eqref{4}\textit{$>$f}(0)\textit{$>$f}\eqref{1}\textit{.}

 答案:A


知识点:一元二次不等式的解法

难度:1


 题目: 函数\textit{y=}log${}_{3}$(9\textit{-x}${}^{2}$)的定义域为\textit{A},值域为\textit{B},则\textit{A}$\mathrm{\cap}$\textit{B=\underbar{     }.~}

 答案:(\textit{-}3,2]


知识点:一元二次不等式的解法

难度:1

 题目: 二次函数\textit{y=ax}${}^{2}$\textit{$+bx+c$}(\textit{x}$\mathrm{\in}$R)的部分对应值如下表:

\begin{tabular}{|p{0.2in}|p{0.3in}|p{0.3in}|p{0.3in}|p{0.3in}|p{0.3in}|p{0.3in}|p{0.2in}|p{0.2in}|} \hline 
	\textit{x} & \textit{-}3 & \textit{-}2 & \textit{-}1 & 0 & 1 & 2 & 3 & 4 \\ \hline 
	\textit{y} & 6 & 0 & \textit{-}4 & \textit{-}6 & \textit{-}6 & \textit{-}4 & 0 & 6 \\ \hline 
\end{tabular}

 则不等式\textit{ax}${}^{2}$\textit{$+bx+c>$}0的解集是\textit{\underbar{       }.~}

 解析:由表格知,一元二次方程\textit{ax}${}^{2}$\textit{$+bx+c=$}0的两个根为\textit{x}${}_{1}$\textit{=-}2,\textit{x}${}_{2}$\textit{=}3,且抛物线开口向上,所以\textit{ax}${}^{2}$\textit{$+bx+c>$}0的解集为$\mathrm{\{}$\textit{x$ |$x$<$-}2或\textit{x$>$}3$\mathrm{\}}$\textit{.}

 答案:$\mathrm{\{}$\textit{x$ |$x$<$-}2或\textit{x$>$}3$\mathrm{\}}$

知识点:一元二次不等式的解法

难度:1

 
题目:若关于\textit{x}的不等式$ -\frac{1}{2}x^2+2x>mx$的解集是$\mathrm{\{}$\textit{x$ |$}0\textit{$<$x$<$}2$\mathrm{\}}$,则实数\textit{m}的值是\textit{\underbar{     }.~}

 解析:由已知得,0和2是方程$ -\frac{1}{2}x^2+2x-mx=0 $的两根,代入得\textit{m=}1\textit{.}

 答案:1

知识点:一元二次不等式的解法

难度:1

 题目: 若不等式(\textit{a-}2)\textit{x}${}^{2}$\textit{-}2(\textit{a-}2)\textit{x-}4\textit{$<$}0的解集为R,则实数\textit{a}的取值范围是\textit{\underbar{     }.~}

 解析:当\textit{a-}2\textit{=}0,即\textit{a=}2时,不等式化为\textit{-}4\textit{$<$}0,显然恒成立;

当\textit{a-}2$\mathrm{\neq}$0时,由题意得
\[
\begin{cases}
a-2<0,\\
\bigtriangleup=4{(a-2)}^2+16(a-2)<0.
\end{cases}
\]

解得\textit{-}2\textit{$<$a$<$}2\textit{.}

综上所述,\textit{a}$\mathrm{\in}$(\textit{-}2,2]\textit{.}

 答案:(\textit{-}2,2]

知识点:一元二次不等式的解法

难度:1

 题目:已知$ f(x)=x^2-(a+\frac{1}{a})+1$\textit{.}

 (1)当\textit{$a=\frac{1}{2}$}时,解不等式\textit{f}(\textit{x})$\mathrm{\le}$0\textit{.}

 (2)若\textit{a$>$}0,解关于\textit{x}的不等式\textit{f}(\textit{x})$\mathrm{\le}$0\textit{.}

 答案:(1)当\textit{$a=\frac{1}{2}$}时,有不等式$f(x)=x^2-\frac{1}{2}x+1\le 0$,

所以$ (x-\frac{1}{2})(x-2)\le 0$,所以$ \frac{1}{2}\le x\le 2$\textit{.}

所以不等式的解集为$\mathrm{\{}$$ x | \frac{1}{2} \le x \le 2$$\mathrm{\}}$\textit{.}

(2)不等式$f(x)=(x-\frac{1}{a})(x-a)\le 0$,

当0\textit{$<$a$<$}1时,$\frac{1}{2}$\textit{$>$a},所以不等式的解集为$\mathrm{\{}$$x|a \le x \le \frac{1}{a}$$\mathrm{\}}$;

当\textit{a$>$}1时,$\frac{1}{2}<a$,所以不等式的解集为$\mathrm{\{}$$x | \frac{1}{a}\le x \le a$$\mathrm{\}}$;

当\textit{a=}1时,不等式的解集为$\mathrm{\{}$\textit{x$ |$x=}1$\mathrm{\}}$\textit{.}


知识点:一元二次不等式的解法

难度:1

 题目:
已知关于\textit{x}的不等式(\textit{a}${}^{2}$\textit{-}4)\textit{x}${}^{2}$\textit{+}(\textit{a+}2)\textit{x-}1$\mathrm{\ge}$0的解集是空集,求实数\textit{a}的取值范围\textit{.}

 答案:当\textit{a}${}^{2}$\textit{-}4\textit{=}0时,\textit{a=$\pm$}2,当\textit{a=-}2时,解集为$\mathrm{\varnothing }$;

当\textit{a=}2时,解集为$\mathrm{\{} x|x \ge \frac{1}{4} \mathrm{\}}$,不符合题意,舍去\textit{.}

当\textit{a}${}^{2}$\textit{-}4$\mathrm{\neq}$0时,要使解集为$\mathrm{\varnothing }$,

则有
\[
\begin{cases}
a^2-4 <0,\\
\bigtriangleup <0.
\end{cases}
\]
解得\textit{-}2\textit{$<$a$<\frac{6}{5}$}\textit{.}

综上,\textit{a}的取值范围是$[ -2, \frac{6}{5} )$.

知识点:一元二次不等式的解法

难度:2

 题目:若不等式组
\[
\begin{cases}
x^2-x-2>0,\\
2x^2+(2k+5)x+5k<0
\end{cases}
\]

 的整数解只有\textit{-}2,求\textit{k}的取值范围\textit{.}

 答案:因为\textit{x}${}^{2}$\textit{-x-}2\textit{$>$}0,所以\textit{x$>$}2或\textit{x$<$-}1\textit{.}

又2\textit{x}${}^{2}$\textit{+}(2\textit{k+}5)\textit{x+}5\textit{k$<$}0,

所以(2\textit{x+}5)(\textit{x+k})\textit{$<$}0\textit{. ①}

当\textit{k$>\frac{5}{2}$}时,\textit{-k$<-\frac{5}{2}$},

由\textit{①}得\textit{-k$<$x$<-\frac{5}{2}$}\textit{$<$-}2,此时\textit{-}2$\mathrm{\notin} (-k, -\frac{5}{2})$;

当\textit{$k=\frac{5}{2}$}时,\textit{①}的解集为空集;

当\textit{k$<\frac{5}{2}$}时,\textit{-}\textit{$\frac{5}{2}<$-k},由\textit{①}得\textit{-}\textit{$\frac{5}{2}<$x$<$-k},

所以
\[
\begin{cases}
x<-1, \\
-\frac{5}{2} < x <-k
\end{cases}
\] 或
\[
\begin{cases}
x>2,\\
-\frac{5}{2} < x < -k.
\end{cases}
\]

因为原不等式组只有整数解\textit{-}2,

所以
\[
\begin{cases}
k <\frac{5}{2},\\
-k > -2,\\
-k \le 3.
\end{cases}
\]
所以\textit{-}3$\mathrm{\le}$\textit{k$<$}2\textit{.}

综上,\textit{k}的取值范围是[\textit{-}3,2)\textit{.}


知识点:一元二次不等式的解法

难度:1

 题目:函数$ f(x)=\lg\frac{1-x}{x-4}$的定义域为(\textit{  })

 

 A.(1,4) B.[1,4)

 C.(\textit{-$\infty$},1)$\mathrm{\cup}$(4,\textit{$+\infty$}) D.(\textit{-$\infty$},1]$\mathrm{\cup}$(4,\textit{$+\infty$})

 解析:依题意应有$f(x)=\lg\frac{1-x}{x-4}>0 $,即(\textit{x-}1)(\textit{x-}4)\textit{$<$}0,所以1\textit{$<$x$<$}4\textit{.}

 答案:A

知识点:一元二次不等式的解法

难度:1

 题目:已知\textit{a}${}_{1}$\textit{$>$a}${}_{2}$\textit{$>$a}${}_{3}$\textit{$>$}0,则使得(1\textit{-a${}_{i}$x})${}^{2}$\textit{$<$}1(\textit{i=}1,2,3)都成立的\textit{x}取值范围是(\textit{  })

 A.$(0,\frac{1}{a_1})$ B.$(0,\frac{2}{a_1})$

 C.$(0,\frac{1}{a_3})$ D.$(0,\frac{2}{a_3})$

 解析:由(1\textit{-a${}_{i}$x})${}^{2}$\textit{$<$}1,得\textit{a${}_{i}$x}(\textit{a${}_{i}$x-}2)\textit{$<$}0,

又\textit{a${}_{i}$$>$}0,所以$x(x-\frac{2}{a_1})$,解得$ 0<x<\frac{2}{a_1}$,

要使上式对\textit{a}${}_{1}$,\textit{a}${}_{2}$,\textit{a}${}_{3}$都成立,则0\textit{$<$x$<\frac{2}{a_1}$}\textit{.}故选B\textit{.}

 答案:B

知识点:一元二次不等式的解法

难度:1

 题目:
不等式\textit{x$>\frac{1}{x}$}的解集是(\textit{  })

 A.(1,\textit{$+\infty$}) B.(\textit{-$\infty$},\textit{-}1)$\mathrm{\cup}$(1,\textit{$+\infty$})

 C.(\textit{-}1,0)$\mathrm{\cup}$(1,\textit{$+\infty$}) D.(\textit{-$\infty$},\textit{-}1)$\mathrm{\cup}$(0,1)

 解析:因为\textit{x$>\frac{1}{x}$},所以$x-\frac{1}{x}=\frac{x^2-1}{x}$,

即\textit{x}(\textit{x}${}^{2}$\textit{-}1)\textit{=x}(\textit{x+}1)(\textit{x-}1)\textit{$>$}0\textit{.}

 \includegraphics*[width=1.17in, height=0.32in, keepaspectratio=false]{image1274}

画出示意图如图\textit{.}

所以解集为(\textit{-}1,0)$\mathrm{\cup}$(1,\textit{$+\infty$})\textit{.}

 答案:C

知识点:一元二次不等式的解法

难度:1

 题目:对任意\textit{a}$\mathrm{\in}$[\textit{-}1,1],都有函数$ f(x)=x^2+(a-4)x+4-2a$的值恒大于零,则\textit{x}的取值范围是(\textit{  })

 A.1\textit{$<$x$<$}3 B.\textit{x$<$}1或\textit{x$>$}3

 C.1\textit{$<$x$<$}2 D.\textit{x$<$}1或\textit{x$>$}2

 解析:设$ g(a)=(x-2)a+(x^2-4x+4),g(a)>0$恒成立,且\textit{a}$\mathrm{\in}$[\textit{-}1,1],

所以
\[
\begin{cases}
g(1)=x^2-3x+2>0,\\
g(-1)=x^2-5x+6>0,
\end{cases}
\]

所以
\[
\begin{cases}
x<1 \text{或} x>2,\\
x<2 \text{或} x>3,
\end{cases}
\]

 答案:B

知识点:一元二次不等式的解法

难度:1

 题目: 若关于\textit{x}的不等式\textit{x}${}^{2}$\textit{$+px+q<$}0的解集为$\mathrm{\{}$\textit{x$ |$}1\textit{$<$x$<$}2$\mathrm{\}}$,则关于\textit{x}的不等式$\frac{x^2+px+q}{x^2-5x-6}$\textit{$>$}0的解集为(\textit{  })

 A.(1,2)

 B.(\textit{-$\infty$},\textit{-}1)$\mathrm{\cup}$(6,\textit{$+\infty$})

 C.(\textit{-}1,1)$\mathrm{\cup}$(2,6)

 D.(\textit{-$\infty$},\textit{-}1)$\mathrm{\cup}$(1,2)$\mathrm{\cup}$(6,\textit{$+\infty$})

 解析:由已知得,\textit{x}${}^{2}$\textit{$+px+q=$}(\textit{x-}1)(\textit{x-}2),

所以$\frac{x^2+px+q}{x^2-5x-6} $\textit{$>$}0,即$\frac{(x-1)(x-2)}{(x+1)(x-6)}$\textit{$>$}0,

等价于$(x-1)(x-2)(x+1)(x-6)>0$,

解得\textit{x$<$-}1或1\textit{$<$x$<$}2或\textit{x$>$}6\textit{.}

 答案:D

知识点:一元二次不等式的解法

难度:2


 题目:
不等式$ \frac{{(x-2)}^2(x-3)}{x+1}<0$\textit{$<$}0的解集为\textit{\underbar{     }.~}

 解析:不等式等价于(\textit{x-}2)${}^{2}$(\textit{x-}3)(\textit{x$+$}1)\textit{$<$}0,如图,用穿针引线法易得\textit{-}1\textit{$<$x$<$}3,且\textit{x}$\mathrm{\neq}$2\textit{.}

 \includegraphics*[width=1.43in, height=0.34in, keepaspectratio=false]{image1281}

 答案:(\textit{-}1,2)$\mathrm{\cup}$(2,3)

知识点:一元二次不等式的解法

难度:2

 题目:已知$\frac{ax}{x-1}$\textit{$<$}1的解集为$\mathrm{\{}$\textit{x$|$x$<$}1或\textit{x$>$}2$\mathrm{\}}$,则实数\textit{a}的值为\textit{\underbar{     }.~}

 解析:因为$\frac{ax}{x-1}$\textit{$<$}1,所以$\frac{ax-x+1}{x-1}$\textit{$<$}0,

即[(\textit{a-}1)\textit{x$+$}1](\textit{x-}1)\textit{$<$}0\textit{.}

又不等式$\frac{ax}{x-1}$\textit{$<$}1的解集为$\mathrm{\{}$\textit{x$|$x$<$}1或\textit{x$>$}2$\mathrm{\}}$,

所以\textit{a-}1\textit{$<$}0,所以$x+\frac{1}{a-1})(x-1)>0$\textit{.}

所以\textit{-}$\frac{1}{a-1}$\textit{=}2,所以\textit{a=}$\frac{1}{2}$\textit{.}

 答案:$\frac{1}{2}$

知识点:一元二次不等式的解法

难度:1

 题目:如果关于\textit{x}的方程$x^2+(m-1)x+m^2-2=0$的两个实根一个小于\textit{-}1,另一个大于1,那么实数\textit{m}的取值范围是\textit{\underbar{     }.~}

 解析:令$f(x)=x^2+(m-1)x+m^2-2$,则
\[
\begin{cases}
f(1)<0,\\
f(-1)<0.
\end{cases}
\]


所以
\[
\begin{cases}
m^2+m-2<0,\\
m^2-m<0.
\end{cases}
\]

 答案:(0,1)

知识点:一元二次不等式的解法

难度:1

 题目:某商家一月至五月累计销售额达3 860万元,预测六月销售额为500万元,七月销售额比六月递增\textit{x}\%,八月销售额比七月递增\textit{x}\%,九、十月销售总额与七、八月销售总额相等\textit{.}若一月至十月销售总额至少达7 000万元,则\textit{x}的最小值是\textit{\underbar{     }.~}

 解析:由题意得,$3860+500+[500(1+x\%)+500{(1+x\%)}^2] \times 2 \ge 7000$,化简得(\textit{x}\%)${}^{2}$\textit{$+$}3·\textit{x}\%\textit{-}0\textit{.}64$\mathrm{\ge}$0,

解得\textit{x}\%$\mathrm{\ge}$0\textit{.}2或\textit{x}\%$\mathrm{\le}$\textit{-}3\textit{.}2(舍去),

所以\textit{x}$\mathrm{\ge}$20,即\textit{x}的最小值为20\textit{.}

 答案:20

知识点:一元二次不等式的解法

难度:2

 题目:解不等式\textit{.}

 (1)$\frac{x-1}{x-2}$$\mathrm{\ge}$0;\textit{ }(2)$\frac{2x-1}{3-4x}$\textit{$>$}1\textit{.}

 答案:(1)原不等式等价于
\[
\begin{cases}
(x-1)(x-2)\ge 0,\\
x-2 \ne 0.
\end{cases}
\]
解得\textit{x}$\mathrm{\le}$1或\textit{x$>$}2,所以原不等式的解集为$\mathrm{\{}$\textit{x$|$x}$\mathrm{\le}$1或\textit{x$>$}2$\mathrm{\}}$\textit{.}

(2)原不等式可改写为$\frac{2x-1}{4x-3}+1<0$,即$\frac{6x-4}{4x-3}<0$,

所以(6\textit{x-}4)(4\textit{x-}3)\textit{$<$}0,所以\textit{$\frac{2}{3}<$x$<\frac{3}{4}$}\textit{.}

所以原不等式的解集为$\mathrm{\{}$\textit{x$|\frac{2}{3}<x<\frac{3}{4}$}$\mathrm{\}}$\textit{.}

知识点:一元二次不等式的解法

难度:3

 题目:解关于\textit{x}的不等式$\frac{1}{x-1}$\textit{$>$a.}

 答案:将原不等式移项、通分化为$\frac{ax-(a+1)}{x-1}$\textit{$<$}0\textit{.}

若\textit{a$>$}0,有$\frac{a+1}{a}$\textit{$>$}1,则原不等式的解集为$\mathrm{\{}$\textit{x$|1<x<\frac{a+1}{a}$}$\mathrm{\}}$;

若\textit{a=}0,有$\frac{a+1}{a}$\textit{$<$}0,则原不等式的解集为$\mathrm{\{}$\textit{x$|$x$>$}1$\mathrm{\}}$;

若\textit{a$<$}0,有$\frac{a+1}{a}$\textit{$<$}1,则原不等式的解集为$\mathrm{\{}$\textit{x$|x<\frac{a+1}{a} \text{或} x>1$}$\mathrm{\}}$\textit{.}

综上所述,

当\textit{a$>$}0时,原不等式的解集为$\mathrm{\{}$\textit{x$|1<x<\frac{a+1}{a}$}$\mathrm{\}}$;

当\textit{a=}0时,原不等式的解集为$\mathrm{\{}$\textit{x$|$x$>$}1$\mathrm{\}}$;

当\textit{a$<$}0时,原不等式的解集为$\mathrm{\{}$\textit{x$|x<\frac{a+1}{a} \text{或} x>1$}$\mathrm{\}}$\textit{.}

知识点:一元二次不等式的解法

难度:2

 题目:若不等式$\frac{x^2-8x+20}{mx^2+2(m+1)x+9m+4}$\textit{$>$}0对任意实数\textit{x}恒成立,求\textit{m}的取值范围\textit{.}

 答案:由于\textit{x}${}^{2}$\textit{-}8\textit{x$+$}20\textit{=}(\textit{x-}4)${}^{2}$\textit{$+$}4\textit{$>$}0恒成立,

因此原不等式对任意实数\textit{x}恒成立等价于$mx^2+2(m+1)x+9m+4>0$对\textit{x}$\mathrm{\in}$R恒成立\textit{.}

(1)当\textit{m=}0时,不等式化为2\textit{x$+$}4\textit{$>$}0,不满足题意\textit{.}

(2)当\textit{m}$\mathrm{\neq}$0时,应有
\[
\begin{cases}
m>0,\\
\bigtriangleup = {[2(m+1)]}^2-4m(9m+4)<0
\end{cases}
\]
解得\textit{m$>\frac{1}{4}$}\textit{.}

综上,实数\textit{m}的取值范围是$(\frac{1}{4},+\infty)$\textit{.}


知识点:基本不等式

难度:1

 题目:已知\textit{x},\textit{y}$\mathrm{\in}$R,下列不等关系正确的是(\textit{  })

 

 A.$x^2+y^2 \ge 2 |xy|$ B.$x^2+y^2 \le 2 |xy|$

 C.$x^2+y^2 > 2 |xy|$ D.$x^2+y^2 < 2 |xy|$

 解析:$x^2+y^2={|x|}^2+{|y|}^2 \ge 2 |x||y|=2|xy|$

当且仅当$|x|=|y|$时等号成立\textit{.}

 答案:A

知识点:基本不等式

难度:1

 
题目:若\textit{x$>$}0,\textit{y$>$}0,且$\sqrt{2xy} \ge \frac{x+2y}{2}$,则必有(\textit{  })

 A.2\textit{x=y} B.\textit{x=}2\textit{y} C.\textit{x=y} D.\textit{x=}4\textit{y}

 解析:因为\textit{x$>$}0,\textit{y$>$}0,所以$\frac{x+2y}{2}\ge \sqrt{2xy}$,即$\frac{x+2y}{2}\ge \sqrt{2xy}$\textit{.}又$\sqrt{2xy\ge\frac{x+2y}{2}}$,所以必有$\sqrt{2xy}=\frac{x+2y}{2}$,所以\textit{x=}2\textit{y.}

 答案:B

知识点:基本不等式

难度:1

 题目:如果正数\textit{a},\textit{b},\textit{c},\textit{d}满足\textit{$a+b=cd=$}4,那么(\textit{  })

 A.\textit{ab}$\mathrm{\le}$\textit{c$+$d},且等号成立时\textit{a},\textit{b},\textit{c},\textit{d}的取值唯一

 B.\textit{ab}$\mathrm{\ge}$\textit{c$+$d},且等号成立时\textit{a},\textit{b},\textit{c},\textit{d}的取值唯一

 C.\textit{ab}$\mathrm{\le}$\textit{c$+$d},且等号成立时\textit{a},\textit{b},\textit{c},\textit{d}的取值不唯一

 D.\textit{ab}$\mathrm{\ge}$\textit{c$+$d},且等号成立时\textit{a},\textit{b},\textit{c},\textit{d}的取值不唯一

 解析:因为$a+b=cd=4$,$a+b\ge 2\sqrt{ab}$,所以$\sqrt{ab}\le 2$,所以\textit{ab}$\mathrm{\le}$4,当且仅当\textit{a=b=}2时,等号成立\textit{.}

又\textit{cd}$\mathrm{\le}\frac{{(c+d)}^2}{4}$,所以$\frac{{(c+d)}^2}{4}$$\mathrm{\ge}$4,所以\textit{c$+$d}$\mathrm{\ge}$4,当且仅当\textit{c=d=}2时,等号成立\textit{.}所以\textit{ab}$\mathrm{\le}$\textit{c+d},当且仅当\textit{a=b=c=d=}2时,等号成立,故选A\textit{.}

 答案:A

知识点:基本不等式

难度:1

 题目:已知0\textit{$<$a$<$b},且\textit{$a+b=$}1,则下列不等式中,正确的是(\textit{  })

 A.log${}_{2}$\textit{a$>$}0 B.2\textit{${}^{a-b}$$<\frac{1}{2}$}

 C.$2^{\frac{a}{b}+\frac{b}{a}}<\frac{1}{2}$ D.log${}_{2}$\textit{$a+$}log${}_{2}$\textit{b$<$-}2

 解析:因为0\textit{$<$a$<$b},且\textit{$a+b=$}1,

所以\textit{ab$<{(\frac{a+b}{2})}^2=\frac{1}{4}$},

所以log${}_{2}$\textit{a+}log${}_{2}$\textit{b=}log${}_{2}$(\textit{ab})\textit{$<$}log${}_{2}\frac{1}{2}$\textit{=-}2\textit{.}

 答案:D

知识点:基本不等式

难度:1

 题目:若\textit{a$>$}0,\textit{b$>$}0,则$\sqrt{\frac{a^2+b^2}{2}}$与$\frac{a+b}{2}$的大小关系是\textit{\underbar{     }.~}

 解析:因为
$\frac{a^2+b^2}{2}=\frac{a^2+b^2+a^2+b^2}{4}\ge \frac{a^2+b^2+2ab}{4}=\frac{{(a+b)}^2}{4}$
,所以
$\sqrt{\frac{a^2+b^2}{2}}\ge \frac{a+b}{2}$
,当且仅当\textit{a=b$>$}0时,等号成立\textit{.}

 答案:$\sqrt{\frac{a^2+b^2}{2}}\ge \frac{a+b}{2}$

知识点:基本不等式

难度:1

 
题目:设\textit{a$>$}0,\textit{b$>$}0,给出下列不等式:

 (1)$(a+\frac{1}{a})(b+\frac{1}{b})$$\mathrm{\ge}$4;

 (2)$(a+b)(\frac{1}{a}+\frac{1}{b})\mathrm{\ge}$4;

 (3)\textit{a}${}^{2}$\textit{$+$}9\textit{$>$}6\textit{a};

 (4)$a^2+1+\frac{1}{a^2+1}>2$\textit{.}

 其中正确的是\textit{\underbar{     }.~}

 解析:因为$a+\frac{1}{a}\ge 2\sqrt{a\cdot \frac{1}{a}}=2$,$b+\frac{1}{b}\ge 2\sqrt{b\cdot \frac{1}{b}}=2$

所以$(a+\frac{1}{a})(b+\frac{1}{b})$$\mathrm{\ge}$4,当且仅当\textit{a=}1,\textit{b=}1时,等号成立,所以(1)正确;

因为$(a+b)(\frac{1}{a}+\frac{1}{b})=1+1+\frac{b}{a}+\frac{a}{b} \ge 2+2\cdot \sqrt{\frac{b}{a}\cdot \frac{a}{b}}=4$,当且仅当\textit{a=b$>$}0时,等号成立,所以(2)正确;

因为$a^2+9 \ge 2\sqrt{a^2\cdot9}=6a$,当且仅当\textit{a=}3时,等号成立,所以当\textit{a=}3时,\textit{a}${}^{2}$\textit{$+$}9\textit{=}6\textit{a},所以(3)不正确;

因为$a^2+1+\frac{1}{a^2+1}\ge 2\sqrt{(a^2+1)\cdot\frac{1}{a^2+1}}$,

当且仅当$a^2+1=\frac{1}{a^2+1}$,即\textit{a=}0时,等号成立,又\textit{a$>$}0,所以等号不成立,所以(4)正确\textit{.}

 答案:(1)(2)(4)

知识点:基本不等式

难度:1

 题目:若\textit{a},\textit{b}为正实数,\textit{a}$\mathrm{\neq}$\textit{b},\textit{x},\textit{y}$\mathrm{\in}$(0,\textit{+$\infty$}),则$\frac{a^2}{x}+\frac{b^2}{y}\ge \frac{{(a+b)}^2}{x+y}$,当且仅当$\frac{a}{x}=\frac{b}{y}$时取等号,利用以上结论,函数$f(x)=\frac{2}{x}+\frac{9}{1-2x}(x\in (0,\frac{1}{2}))$取得最小值时,\textit{x}的值为\textit{\underbar{     }.~}

 解析:由题意可$f(x)=\frac{4}{2x}+\frac{9}{1-2x}\ge \frac{{(2+3)}^2}{2x+(1-2x)}$,当且仅当$\frac{2}{2x}=\frac{3}{1-2x}$时,等号成立,解得$x=\frac{1}{5}$\textit{.}

 答案:$x=\frac{1}{5}$

知识点:基本不等式

难度:1

 题目:若实数\textit{x},\textit{y}满足$x^2+y^2+xy=1$,求\textit{$x+y$}的最大值\textit{.}

 答案:由$x^2+y^2+xy=1$可得${(x+y)}^2=xy+1$,

又$xy \le {(\frac{x+y}{2})}^2$,

所以(\textit{x+y})${}^{2}$$\mathrm{\le}{(\frac{x+y}{2})}^2+1$,整理得$\frac{3}{4}{(x+y)}^2 \le 1$,

当且仅当\textit{x=y}时取等号\textit{.}

所以$x+y \in [-\frac{2\sqrt{3}}{3}, \frac{2\sqrt{3}}{3}]$\textit{.}

所以\textit{$x+y$}的最大值为$\frac{2\sqrt{3}}{3}$\textit{.}

知识点:基本不等式

难度:1

 题目:已知\textit{a$>$}0,\textit{b$>$}0,\textit{$a+b=$}1,求证:$\sqrt{a+\frac{1}{2}}+\sqrt{b+\frac{1}{2}}$$\mathrm{\le}$2\textit{.}

 答案:因为$\sqrt{a+\frac{1}{2}}=\sqrt{1\cdot(a+\frac{1}{2})} \le \frac{1+a+\frac{1}{2}}{2}=\frac{3}{4}+\frac{a}{2}$、,当且仅当$a=\frac{1}{2}$时取等号,

同理$\sqrt{b+\frac{1}{2}}\le \frac{3}{4}+\frac{b}{2}$,当且仅当$b=\frac{1}{2}$时取等号\textit{.}

所以$\sqrt{a+\frac{1}{2}}+\sqrt{b+\frac{1}{2}}\le \frac{3}{4}+\frac{a}{2}+\frac{3}{4}+\frac{b}{2}=\frac{3}{2}+\frac{1}{2}(a+b)=\frac{3}{2}+\frac{1}{2}=2$,当且仅当$\frac{1}{2}$时取等号\textit{.}

所以$\sqrt{a+\frac{1}{2}}+\sqrt{b+\frac{1}{2}}$$\mathrm{\le}$2\textit{.}

知识点:基本不等式

难度:2

 题目:已知$m>0$,$n>0$,$\alpha=m+\frac{1}{m}$,$\beta=n+\frac{1}{n}$,$m$,$n$的等差中项为1,则$\alpha+\beta$的最小值为(\textit{  })

 A.3 B.4 C.5 D.6

 解析:由已知得,$m+n=2$,所以$\alpha+\beta=m+\frac{1}{m}+n+\frac{1}{n}=(m+n)+\frac{m+n}{mn}=2+\frac{2}{mn}$

因为\textit{m$>$}0,\textit{n$>$}0,所以\textit{mn}$\mathrm{\le}{(\frac{m+n}{2})}^2$\textit{=}1\textit{.}

所以$\alpha+\beta \ge 2+ \frac{2}{1}=4$\textit{.}

当且仅当\textit{m=n=}1时,等号成立\textit{.}

所以\textit{$\alpha+\beta$}的最小值为4\textit{.}

 答案:B

知识点:基本不等式

难度:2

 题目: 给出下列四个命题:\textit{①}若\textit{a$<$b},则\textit{a}${}^{2}$\textit{$<$b}${}^{2}$;\textit{②}若\textit{a}$\mathrm{\ge}$\textit{b$>$-}1,则$\frac{a}{1+a} \ge \frac{b}{1+b}$;\textit{③}若正整数\textit{m}和\textit{n}满足\textit{m$<$n},则$\sqrt{m(n-m)} \le \frac{n}{2}$;\textit{④}若\textit{x$>$}0,且\textit{x}$\mathrm{\neq}$1,则$\ln x+ \frac{1}{\ln x} \ge 2$,其中真命题的序号是(\textit{  })

 A.\textit{①②} B.\textit{②③} C.\textit{①④} D.\textit{②④}

 解析:当\textit{a=-}2,\textit{b=}1时,\textit{a$<$b},但\textit{a}${}^{2}$\textit{$>$b}${}^{2}$,故\textit{①}不成立;

对于\textit{②},$\frac{a}{1+a}-\frac{b}{1+b}=\frac{a(1+b)-b(1+a)}{(1+a)(1+b)}=\frac{a-b}{(1+a)(1+b)}$,因为\textit{a}$\mathrm{\ge}$\textit{b$>$-}1,所以$\frac{a}{1+a}-\frac{b}{1+b}$$\mathrm{\ge}$0,故\textit{②}正确;

对于\textit{③},$\sqrt{m(n-m)} \le \frac{m+n\cdot m}{2}= \frac{n}{2}$(\textit{m$<$n},且\textit{m},\textit{n}为正整数),当且仅当\textit{m=n-m},即\textit{m=$\frac{n}{2}$}时,等号成立,故\textit{③}正确;

对于\textit{④},当0\textit{$<$x$<$}1时,ln \textit{x$<$}0,故\textit{④}不成立\textit{.}故选B\textit{.}

 答案:B

知识点:基本不等式

难度:2

 题目: 在算式4\textit{$\times$}□\textit{$+$}$\mathrm{\vartriangle}$\textit{=}30的□、$\mathrm{\vartriangle}$中,分别填入一个正整数使算式成立,并使填入的正整数的倒数之和最小,则这两个正整数构成的数对(□,$\mathrm{\vartriangle}$)应为(\textit{  })

 A.(4,14) B.(6,6) C.(3,18) D.(5,10)

 解析:可设□中的正整数为\textit{x},$\mathrm{\vartriangle}$中的正整数为\textit{y},则由已知可得4\textit{x+y=}30\textit{.}

因为$\frac{1}{x}+\frac{1}{y}=\frac{1}{30}(\frac{4x+y}{x}+\frac{4x+y}{y})=\frac{1}{30}(5+\frac{y}{x}+\frac{4x}{y})\ge (5+2\sqrt{\frac{y}{x}\cdot \frac{4x}{y}})=\frac{3}{10}$,当且仅当$\frac{y}{x}=\frac{4x}{y}$,即\textit{y=}2\textit{x}时,等号成立,又4\textit{$x+y=$}30,所以\textit{x=}5,\textit{y=}10,故选D\textit{.}

 答案:D

知识点:基本不等式

难度:2

 题目:当\textit{x$>$}3时,$x+\frac{1}{x-3} \ge a$恒成立,则\textit{a}的最大值为\textit{\underbar{     }.~}

 解析:因为\textit{x$>$}3,所以$x+\frac{1}{x-3}=x-3+\frac{1}{x-3}+3 \ge 2\sqrt{(x-3)\cdot \frac{1}{x-3}}+3=5$

当且仅当\textit{x-}3\textit{=}$\frac{1}{x-3}$,即\textit{x=}4时,等号成立\textit{.}

所以由题意可知\textit{a}$\mathrm{\le}$5\textit{.}

 答案:5

知识点:基本不等式

难度:2

 题目:若\textit{a$>$}1,0\textit{$<$b$<$}1,则log\textit{${}_{a}$$b+$}log\textit{${}_{b}$a}的取值范围是\textit{\underbar{    }.~}

 解析:因为\textit{a$>$}1,0\textit{$<$b$<$}1,所以log\textit{${}_{a}$b$<$}0,log\textit{${}_{b}$a$<$}0,

所以\textit{-}(log\textit{${}_{a}$b$+$}log\textit{${}_{b}$a})\textit{=}(\textit{-}log\textit{${}_{a}$b})\textit{$+$}(\textit{-}log\textit{${}_{b}$a})$\mathrm{\ge}$2,

当且仅当\textit{-}log\textit{${}_{a}$b=-}log\textit{${}_{b}$a},即\textit{a$>$}1,0\textit{$<$b$<$}1,\textit{ab=}1时,等号成立\textit{.}所以log\textit{${}_{a}$b$+$}log\textit{${}_{b}$a}$\mathrm{\le}$\textit{-}2\textit{.}

 答案:(\textit{-$\infty$},\textit{-}2]

知识点:基本不等式

难度:2

 题目:已知\textit{a},\textit{b},\textit{c}为不全相等的正数,求证:$\frac{b+c\cdot a}{a}+\frac{c+a\cdot b}{b}+\frac{a+b\cdot c}{c}$\textit{$>$}3\textit{.}

 答案:
\begin{align}
\notag
\frac{b+c\cdot a}{a}+\frac{c+a\cdot b}{b}+\frac{a+b\cdot c}{c}
&=\frac{b}{a}+\frac{c}{a}+\frac{c}{b}+\frac{a}{b}+\frac{a}{c}+\frac{b}{c}-3\\ \notag
&=(\frac{b}{a}+\frac{a}{b})+(\frac{c}{a}+\frac{a}{c})+(\frac{c}{b}+\frac{b}{c})-3	\notag
\end{align}


因为\textit{a$>$}0,\textit{b$>$}0,\textit{c$>$}0,

所以$\frac{a}{b}+\frac{b}{a}$$\mathrm{\ge}$2,$\frac{c}{a}+\frac{a}{c}$$\mathrm{\ge}$2,$\frac{c}{b}+\frac{b}{c}$$\mathrm{\ge}$2\textit{.}

又\textit{a},\textit{b},\textit{c}不全相等,

所以$\frac{b}{a}+\frac{c}{a}+\frac{c}{b}+\frac{a}{b}+\frac{a}{c}+\frac{b}{c}$\textit{$>$}6\textit{.}

所以$\frac{b}{a}+\frac{c}{a}+\frac{c}{b}+\frac{a}{b}+\frac{a}{c}+\frac{b}{c}$\textit{-}3\textit{$>$}6\textit{-}3\textit{=}3\textit{.}

故原不等式成立\textit{.}

知识点:基本不等式

难度:2

 题目:已知\textit{a$>$b$>$c},且$\frac{1}{a-b}+\frac{1}{b-c} \ge \frac{n}{a-c}$恒成立\textit{.}求\textit{n}的最大值\textit{.}

 答案:因为$\frac{1}{a-b}+\frac{1}{b-c} \ge \frac{n}{a-c}$,\textit{a$>$b$>$c},

所以$(a-c)(\frac{1}{a-b}+\frac{1}{b-c}) \ge n$

又
\begin{align}
\notag
(a-c)(\frac{1}{a-b}+\frac{1}{b-c})&=(a-b+b-c)(\frac{1}{a-b}+\frac{1}{b-c})\\ \notag
&=	2+\frac{b-c}{a-b}+\frac{a-b}{b-c}\ge 2+2\sqrt{\frac{b-c}{a-b}\cdot \frac{a-b}{b-c}}=4 \\ \notag
&=2+\frac{b-c}{a-b}+ \frac{a-b}{b-c} \ge 2+2\sqrt{\frac{b-c}{a-b}\cdot \frac{a-b}{b-c}}=4
\end{align}


当且仅当\textit{a-b=b-c},即\textit{a+c=}2\textit{b}时,等号成立\textit{.}

由$\frac{1}{a-b}+\frac{1}{b-c} \ge \frac{n}{a-c}$恒成立,得\textit{n}$\mathrm{\le}$4,所以\textit{n}的最大值为4\textit{.}





知识点:基本不等式

难度:1

 题目:若\textit{a$>$}0,\textit{b$>$}0,且ln(\textit{a+b})\textit{=}0,则$\frac{1}{a}+\frac{1}{b}$的最小值是 (\textit{  })

 

 A.$\frac{1}{4}$B.1 C.4 D.8

 解析:由\textit{a$>$}0,\textit{b$>$}0,ln(\textit{a+b})\textit{=}0,得
\[
\begin{cases}
a>0,\\
b>0,\\
a+b=1.
\end{cases}
\]


所以$ \frac{1}{a}+\frac{1}{b}=\frac{a+b}{a}+\frac{a+b}{b}=2+\frac{b}{a}+\frac{a}{b}\ge 2+2\sqrt{\frac{1}{a}\cdot \frac{1}{b}}=4$


当且仅当\textit{a=b=}$\frac{1}{2}$时,等号成立\textit{.}

所以$\frac{1}{a}+\frac{1}{b}$的最小值为4\textit{.}

 答案:C

知识点:基本不等式

难度:1

 题目:若\textit{x$>$}4,则函数$y=-x+\frac{1}{4-x}$(\textit{  })

 A.有最大值\textit{-}6 B.有最小值6

 C.有最大值\textit{-}2 D.有最小值2

 解析:因为\textit{x$>$}4,所以\textit{x-}4\textit{$>$}0\textit{.}

所以$y=-x+\frac{1}{4-x}=-[(x-4)+\frac{1}{4-x}]-4 \le -2-4=-6$,当且仅当$x-4=\frac{1}{4-x}$,即\textit{x=}5时,等号成立\textit{.}

 答案:A

知识点:基本不等式

难度:1

 题目:已知\textit{x$>$}1,\textit{y$>$}1,且$\frac{1}{4}$ln \textit{x},$\frac{1}{4}$,ln \textit{y}成等比数列,则\textit{xy}有(\textit{  })

 A.最小值e B.最小值$\sqrt{e}$

 C.最大值e D.最大值$\sqrt{e}$

 解析:因为\textit{x$>$}1,\textit{y$>$}1,且$\frac{1}{4}$ln \textit{x},$\frac{1}{4}$,ln \textit{y}成等比数列,

所以$\frac{1}{4}$ln \textit{x}·ln \textit{y=}${(\frac{1}{4})}^2$\textit{.}

所以$\frac{1}{4}$\textit{=}ln \textit{x}·ln \textit{y}$\mathrm{\le} {(\frac{\ln x +\ln y}{2})}^2$,当且仅当\textit{x=y=}$\sqrt{e}$时,等号成立,所以ln \textit{x$+$}ln \textit{y}$\mathrm{\ge}$1,即ln \textit{xy}$\mathrm{\ge}$1,所以\textit{xy}$\mathrm{\ge}$e\textit{.}

 答案:A

知识点:基本不等式

难度:1

 题目:已知函数$f(x)=|\lg x|$,若\textit{a}$\mathrm{\neq}$\textit{b},且\textit{f}(\textit{a})\textit{=f}(\textit{b}),则\textit{$a+b$}的取值范围是(\textit{  })

 A.(1,\textit{$+\infty$}) B.[1,\textit{$+\infty$}) 

 C.(2,\textit{$+\infty$}) D.[2,\textit{$+\infty$})

 解析:由已知得$|\lg a|=|\lg b|$,\textit{a$>$}0,\textit{b$>$}0,

所以lg \textit{a=}lg \textit{b}或lg \textit{a=-}lg \textit{b.}

因为\textit{a}$\mathrm{\neq}$\textit{b},所以lg \textit{a=}lg \textit{b}不成立,

所以只有lg \textit{a=-}lg \textit{b},

即lg \textit{a$+$}lg \textit{b=}0,所以\textit{ab=}1,\textit{b=}$\frac{1}{a}$\textit{.}

又\textit{a$>$}0,\textit{a}$\mathrm{\neq}$\textit{b},所以\textit{$a+b=a+\frac{1}{a}$}\textit{$>$}2\textit{.}故选C\textit{.}

 答案:C

知识点:基本不等式

难度:1

 题目:若log${}_{4}$(3\textit{a$+$}4\textit{b})\textit{=}log${}_{2}\sqrt{ab}$,则\textit{$a+b$}的最小值是(\textit{  })

 A.6\textit{$+$}2$\sqrt{3}$ B.7\textit{$+$}2$\sqrt{3}$

 C.6\textit{$+$}4$\sqrt{3}$ D.7\textit{$+$}4$\sqrt{3}$

 解析:由题意得
\[
\begin{cases}
\sqrt{ab}>0,\\
3a+4b>0.
\end{cases}
\]

所以
\[
\begin{cases}
a>0,\\
b>0.
\end{cases}
\]


又log${}_{4}$(3\textit{a$+$}4\textit{b})\textit{=}log${}_{2}\sqrt{ab}$,

所以log${}_{4}$(3\textit{a$+$}4\textit{b})\textit{=}log${}_{4}$\textit{ab.}

所以3\textit{a$+$}4\textit{b=ab},所以$\frac{4}{a}+\frac{3}{b}$\textit{=}1\textit{.}

所以$a+b=(a+b)(\frac{4}{a}+\frac{3}{b})=7+\frac{3a}{b}+\frac{4b}{a} \ge 7+2\sqrt{\frac{3a}{b}+\frac{4b}{a}}=7+4\sqrt{3}$,当且仅当$\frac{3a}{b}=\frac{4b}{a}$,即$a=4+2\sqrt{3},b=3+2\sqrt{3}$时取等号,故选D\textit{.}

 答案:D

知识点:基本不等式

难度:1

 题目:若正数\textit{a},\textit{b},\textit{c}满足$c^2+4bc+2ac+8ab=8$,则$a+2b+c$的最小值为(\textit{  })

 A.$\sqrt{3}$ B.2$\sqrt{3}$ C.2 D.2$\sqrt{3}$

 解析:方法一:$c^2+4bc+2ac+8ab=(c+2a)(c+4b)=8$,

因为\textit{a},\textit{b},\textit{c}均为正数,所以由基本不等式得$(c+2a)(c+4b) \le {(\frac{c+2a+c+4b}{den})}^2$,所以$a+2b+c \ge 2\sqrt{2}$

当且仅当$c+2a=c+4b$,即\textit{a=}2\textit{b}时,等号成立\textit{.}

方法二:${(a+2b+c)}^2=a^2+4b^2+c^2+4ab+2ac+4bc$,

因为$c^2+4bc+2ac+8ab=(c+2a)(c+4b)=8$,所以${(a+2b+c)}^2=a^2+4b^2-4ab+8={(a-2b)}^2+8 \ge 8$,所以$a+2b+c \ge 2\sqrt{2}$

 答案:D

知识点:基本不等式

难度:2

 题目:若直线$\frac{x}{a}+\frac{y}{b}$\textit{=}1(\textit{a$>$}0,\textit{b$>$}0)过点(1,2),则2\textit{a$+$b}的最小值为\textit{\underbar{     }.~}

 解析:\textit{$\because$}直线$\frac{x}{a}+\frac{y}{b}$\textit{=}1过点(1,2),\textit{$\therefore$}$\frac{1}{a}+\frac{2}{b}$\textit{=}1\textit{.}

\textit{$\because$a$>$}0,\textit{b$>$}0,\textit{$\therefore$}$2a+b=(2a+b)(\frac{1}{a}+\frac{2}{b})=4+(\frac{b}{a}+\frac{4a}{b}) \ge 4+2\sqrt{\frac{b}{a}\cdot \frac{4a}{b}}=8$

当且仅当\textit{b=}2\textit{a}时``\textit{=}''成立\textit{.}

 答案:8

知识点:基本不等式

难度:2

 题目:若\textit{a},\textit{b}$\mathrm{\in}$R,\textit{ab$>$}0,则$\frac{a^4+4b^4+1}{ab}$的最小值为\textit{\underbar{     }.~}

 解析:\textit{$\because$a},\textit{b}$\mathrm{\in}$R,且\textit{ab$>$}0,

\textit{$\therefore$}
$\frac{a^4+4b^4+1}{ab} \ge \frac{4a^2b^2+1}{ab}=4ab+\frac{1}{ab}\ge 4$

(当且仅当
$
\begin{cases}
a^2=2b^2,\\
4ab=\frac{1}{ab}
\end{cases}$
即
$
\begin{cases}
a^2=\frac{\sqrt{2}}{2},\\
b^2=\frac{\sqrt{2}}{4}
\end{cases}$
时取等号)

 答案:4

知识点:基本不等式

难度:1

 题目:已知\textit{x$>$}0,\textit{y$>$}0,且$\frac{2}{x}+\frac{1}{y}$\textit{=}1,若\textit{x$+$}2\textit{y$>$m}${}^{2}$\textit{$+$}2\textit{m}恒成立,则实数\textit{m}的取值范围是\textit{\underbar{     }.~}

 解析:因为\textit{x$>$}0,\textit{y$>$}0,且$\frac{2}{x}+\frac{1}{y}$\textit{=}1,

所以$x+2y=(x+2y)(\frac{2}{x}+\frac{1}{y})=4+(\frac{4y}{x}+\frac{x}{y}) \ge 4+2\sqrt{\frac{4y}{x} \cdot \frac{x}{y}}=8$

,当且仅当$\frac{4y}{x} = \frac{x}{y}$,即\textit{x=}4,\textit{y=}2时,\textit{x$+$}2\textit{y}取得最小值8,

所以\textit{m}${}^{2}$\textit{$+$}2\textit{m$<$}8,解得\textit{-}4\textit{$<$m$<$}2\textit{.}

 答案:(\textit{-}4,2)

知识点:基本不等式

难度:1

 题目:已知正常数\textit{a},\textit{b}和正变数\textit{x},\textit{y},满足$a+b=10$,$\frac{a}{x}+\frac{b}{y}=1$,$x+y$的最小值为18,求\textit{a},\textit{b}的值\textit{.}

 答案:由已知得
$x+y=(x+y)\cdot =(x+y)\cdot (\frac{a}{x}+\frac{b}{y})= a+b+\frac{ay}{x}+\frac{bx}{y} \ge a+b+2\sqrt{ab}={(\sqrt{a}+\sqrt{b})}^2$

当且仅当
$
\begin{cases}
\frac{a}{x}+\frac{b}{y}=1,\\
\frac{ay}{x}+\frac{bx}{y},
\end{cases}
$
即
$
\begin{cases}
x=a+\sqrt{ab},\\
y=b+\sqrt{ab}
\end{cases}
$
时等号成立,所以\textit{x$+$y}的最小值为${(\sqrt{a}+\sqrt{b})}^2=18$

又\textit{a$+$b=}10,所以\textit{ab=}16\textit{.}

所以\textit{a},\textit{b}是方程\textit{x}${}^{2}$\textit{-}10\textit{x$+$}16\textit{=}0的两根,所以\textit{a=}2,\textit{b=}8或\textit{a=}8,\textit{b=}2\textit{.}

知识点:基本不等式

难度:2

题目:已知\textit{a},\textit{b}都是正实数,且\textit{a$+$b=}1\textit{.}

 (1)求证:$\frac{1}{a}+\frac{1}{b}$$\mathrm{\ge}$4;

 (2)求${(a+\frac{1}{a})}^2+{(b+\frac{1}{b})}^2$的最小值\textit{.}

 答案: (1)证明因为\textit{a$+$b=}1,\textit{a$>$}0,\textit{b$>$}0,

所以$\frac{1}{a}+\frac{1}{b}=(\frac{1}{a}+\frac{1}{b})(a+b)=2+\frac{b}{a}+\frac{a}{b} \ge 2+2\sqrt{\frac{b}{a}\cdot \frac{a}{b}}=4$

当且仅当\textit{a=b=}$\frac{1}{2}$时,等号成立\textit{.}所以$\frac{1}{a}+\frac{1}{b}$$\mathrm{\ge}$4\textit{.}

 (2)解因为\textit{a$+$b=}1,\textit{a$>$}0,\textit{b$>$}0,

所以${(a+\frac{1}{a})}^2+{(b+\frac{1}{b})}^2 \ge 2\cdot {(\frac{a+\frac{1}{a}+b+\frac{1}{b}}{2})}^2=\frac{1}{2}{(1+\frac{1}{ab})}^2$

又$\frac{a+b}{2} \ge \sqrt{ab}$,所以0\textit{$<$ab}$\mathrm{\le}\frac{1}{4}$,即$\frac{1}{ab}$$\mathrm{\ge}$4,

所以$1+\frac{1}{ab} \ge 5$

所以${(a+\frac{1}{a})}^2+{(b+\frac{1}{b})}^2 \ge \frac{25}{2}$,当且仅当\textit{a=b=}$\frac{1}{2}$时,等号成立\textit{.}

所以${(a+\frac{1}{a})}^2+{(b+\frac{1}{b})}^2$的最小值为$\frac{25}{2}$\textit{.}

知识点:基本不等式

难度:1

 题目:某单位用2 160万元购得一块空地,计划在该块地上建造一栋至少10层,每层2 000平方米的楼房\textit{.}经测算,若将楼房建为\textit{x}(\textit{x}$\mathrm{\ge}$10)层,则每平方米的平均建筑费用为560\textit{$+$}48\textit{x}(单位:元)\textit{.}为了使楼房每平方米的平均综合费用最少,该楼房应建为多少层?

 注:平均综合费用\textit{=}平均建筑费用\textit{+}平均购地费用,平均购地费用\textit{=}$\frac{购地总费用}{建筑总面积}$

 答案:设楼房每平方米的平均综合费用为\textit{f}(\textit{x})元,

则\textit{f}(\textit{x})\textit{=}560\textit{$+$}48\textit{x$+$}$\frac{2160\times 10000}{2000x}$

\textit{=}560\textit{$+$}48\textit{x$+\frac{10800}{x}$}(\textit{x}$\mathrm{\ge}$10,\textit{x}$\mathrm{\in}$N\textit{${}_{+}$}),

所以\textit{f}(\textit{x})\textit{=}560\textit{$+$}48\textit{x$+$}$\frac{10800}{x}$

$\mathrm{\ge}$560\textit{$+$}2$\sqrt{48x\frac{10800}{x}}$\textit{=}2 000,

当且仅当48\textit{x=}$\frac{10800}{x}$,即\textit{x=}15时,等号成立\textit{.}

因此,当\textit{x=}15时,\textit{f}(\textit{x})取最小值2 000\textit{.}

答:为了使楼房每平方米的平均综合费用最少,该楼房应建为15层\textit{.}

知识点:二元一次不等式(组)的解集

难度:1

 题目:不等式2\textit{x+y+}1\textit{$<$}0表示的平面区域在直线2\textit{x+y+}1\textit{=}0的(\textit{  })

 \textit{      } \textit{      } \textit{     }

 A.右上方 B.右下方 C.左上方 D.左下方

 答案:D

知识点:二元一次不等式(组)的解集

难度:1

 题目:不等式组
$
\begin{cases}
x-y+5 \ge 0,\\
0 \le x \le 3,\\
x+y \ge
\end{cases}$
表示的平面区域是(\textit{  })

 A.矩形 B.三角形

 C.直角梯形 D.等腰梯形

 \includegraphics*[width=1.00in, height=1.13in, keepaspectratio=false]{image1508}

 解析:画出平面区域(如图阴影部分),该区域是等腰梯形\textit{.}

 答案:D

知识点:二元一次不等式(组)的解集

难度:1

 题目:直线2\textit{x+y-}10\textit{=}0与不等式组
$
\begin{cases}
x \ge 0,\\
y \ge 0,\\
x-y \ge -2,\\
4x+3y \le 20.
\end{cases}$
表示的平面区域的公共点有(\textit{  })

 A.0个 B.1个

 C.2个 D.无数个

 \includegraphics*[width=0.93in, height=1.11in, keepaspectratio=false]{image1510}

 解析:如图所示,不等式组表示的平面区域为阴影部分,直线与阴影只有一个公共点(5,0)\textit{.}

 答案:B

知识点:二元一次不等式(组)的解集

难度:1

 题目:若不等式组
$
\begin{cases}
x \le 1,\\
y \le 3,\\
2x-y + \lambda -1 \ge 0
\end{cases}$
表示的平面区域经过四个象限,则实数\textit{$\lambda$}的取值范围是(\textit{  })

 A.(\textit{-$\infty$},4) B.[1,2]

 C.(1,4) D.(1,\textit{$+\infty$})

 答案:D

知识点:二元一次不等式(组)的解集

难度:1

 题目:若点\textit{A}(3,3),\textit{B}(2,\textit{-}1)在直线\textit{x$+$y-a=}0的两侧,则\textit{a}的取值范围是\textit{\underbar{   }.~}

 解析:由题意得(3\textit{$+$}3\textit{-a})(2\textit{-}1\textit{-a})\textit{$<$}0,解得1\textit{$<$a$<$}6\textit{.}

 答案:(1,6)

知识点:二元一次不等式(组)的解集

难度:1

 题目:若用三条直线\textit{x$+$}2\textit{y=}2,2\textit{x$+$y=}2,\textit{x-y=}3围成一个三角形,则三角形内部区域(不包括边界)可用不等式(组)表示为\textit{\underbar{     }.~}

 答案:
$
\begin{cases}
x+2y <2,\\
2x +y >2,\\
x-y <3
\end{cases}$


知识点:二元一次不等式(组)的解集

难度:1

 题目:若不等式组
$
\begin{cases}
x-y+5 \ge 0,\\
y \ge a,\\
0 \le x \le 2
\end{cases}$
表示的平面区域是一个三角形,则\textit{a}的取值范围是\textit{\underbar{     }.~}

 \includegraphics*[width=1.25in, height=1.04in, keepaspectratio=false]{image1514}

 解析:如图,当直线\textit{y=a}位于直线\textit{y=}5和\textit{y=}7之间(不含\textit{y=}7)时满足条件,故\textit{a}的取值范围应是5$\mathrm{\le}$\textit{a$<$}7\textit{.}

 答案:[5,7)

知识点:二元一次不等式(组)的解集

难度:1

 题目:设\textit{f}(\textit{x})\textit{=x}${}^{2}$\textit{$+ax+b$},若1$\mathrm{\le}$\textit{f}(\textit{-}1)$\mathrm{\le}$2,2$\mathrm{\le}$\textit{f(1)}$\mathrm{\le}$4,试求点(\textit{a},\textit{b})构成的平面区域的面积\textit{.}

 答案:\textit{f}(\textit{-}1)\textit{=}1\textit{-a$+$b},\textit{f(1)}\textit{=}1\textit{$+a+b$},

由
$
\begin{cases}
1 \le f(-1) \le 2,\\
2 \le f(1) \le 4.
\end{cases}$


得不等式组
$
\begin{cases}
1 \le 1-a+b \le 2,\\
2 \le 1+a+b \le 4,
\end{cases}$

即
$
\begin{cases}
a-b \le 0,\\
a-b \ge -1,\\
a+b \ge 1,\\
a+b \le 3.
\end{cases}$


 \includegraphics*[width=1.11in, height=1.00in, keepaspectratio=false]{image1519}

作出不等式组表示的平面区域(如图阴影部分所示)\textit{.}

可知平面区域为矩形\textit{ABCD},$|AB|=\frac{2}{\sqrt{2}}=\sqrt{2}$,$|BC|=\frac{1}{\sqrt{2}}=\frac{\sqrt{2}}{2}$,

所以所求区域面积为$\sqrt{2}\times \frac{\sqrt{2}}{2}$

知识点:二元一次不等式(组)的解集

难度:1

题目:某工厂生产甲、乙两种产品,需要经过金工和装配两个车间加工,有关数据如下表:

\begin{tabular}{|p{0.8in}|p{0.7in}|p{0.2in}|p{0.2in}|p{1.3in}|} \hline 
	\multicolumn{2}{|p{1in}|}{加工时间\textit{/}(小时\textit{/}件)} & \multicolumn{2}{|p{0.4in}|}{产品} & 总有效工时\textit{/}小时 \\ \hline 
	\multicolumn{2}{|p{1in}|}{} & 甲 & 乙 &  \\ \hline 
	车间 & 金工 & 4 & 3 & 480 \\ \hline 
	& 装配 & 2 & 5 & 500 \\ \hline 
\end{tabular}


 列出满足生产条件的数学关系式,并画出相应的平面区域\textit{.}

 答案:设分别生产甲、乙两种产品\textit{x}件和\textit{y}件,于是满足条件
$
\begin{cases}
4x+3y \le 480,\\
2x + 5y \le 500,\\
x \in N,\\
y \in N,
\end{cases}$


所以满足的生产条件是图中阴影部分中的整数点\textit{.}

 \includegraphics*[width=1.90in, height=1.27in, keepaspectratio=false]{image1524}


知识点:二元一次不等式(组)的解集

难度:2

 题目:在平面直角坐标系中,若点\textit{A}(\textit{-}2,\textit{t})在直线\textit{x-}2\textit{y$+$}4\textit{=}0的上方,则\textit{t}的取值范围是(\textit{  })

 A.(\textit{-$\infty$},1) B.(1,\textit{$+\infty$})

 C.(\textit{-}1,\textit{$+\infty$}) D.(0,1)

 \includegraphics*[width=1.45in, height=0.95in, keepaspectratio=false]{image1525}

 解析:在直线方程\textit{x-}2\textit{y$+$}4\textit{=}0中,令\textit{x=-}2,则\textit{y=}1,则点(\textit{-}2,1)在直线\textit{x-}2\textit{y$+$}4\textit{=}0上,又点(\textit{-}2,\textit{t})在直线\textit{x-}2\textit{y$+$}4\textit{=}0的上方,由图可知,\textit{t}的取值范围是\textit{t$>$}1,故选B\textit{.}

 答案:B

知识点:二元一次不等式(组)的解集

难度:2

 题目:若不等式组$
\begin{cases}
x\ge 0,\\
x+3y \ge 4,\\
3x +y \le 4
\end{cases}$所表示的平面区域被直线\textit{$y=kx+\frac{4}{3}$}分为面积相等的两部分,则\textit{k}的值是(\textit{  })

 A.$\frac{7}{3}$ B.$\frac{3}{7}$ C.$\frac{4}{3}$ D.$\frac{3}{4}$

 \includegraphics*[width=1.33in, height=1.03in, keepaspectratio=false]{image1532}

 解析:不等式组表示的平面区域是如图所示阴影部分的$\mathrm{\vartriangle}$\textit{ABC.}

由$
\begin{cases}
x+3y=4,\\
3x+y=4.
\end{cases}$得\textit{A}(1,1),

又\textit{B}(0,4),\textit{C}$(0,\frac{4}{3})$,

所以\textit{S}${}_{\vartriangle }$\textit{${}_{ABC}$=}$\frac{1}{2}\times (4-\frac{4}{3})$\textit{$\times$}1\textit{=}$\frac{4}{3}$

设\textit{$y=kx+\frac{4}{3}$}与3\textit{x$+$y=}4的交点为\textit{D}(\textit{x${}_{D}$},\textit{y${}_{D}$}),

则\textit{S}${}_{\vartriangle }$\textit{${}_{BCD}$=}$\frac{1}{2}$\textit{S}${}_{\vartriangle }$\textit{${}_{ABC}$=}$\frac{2}{3}$,

所以\textit{x${}_{D}$=}$\frac{1}{2}$,所以\textit{y${}_{D}$=}$\frac{5}{2}$,

所以$\frac{5}{2}=k\times \frac{1}{2}+\frac{4}{3}$,所以$k=\frac{7}{3}$

 答案:A

知识点:二元一次不等式(组)的解集

难度:2

 题目:已知点(1,2)和点(\textit{-}1,3)在直线2\textit{x$+$ay-}1\textit{=}0的同一侧,则实数\textit{a}的取值范围是\textit{\underbar{        }.~}

 解析:因为(2\textit{a$+$}1)(3\textit{a-}3)\textit{$>$}0,

所以\textit{a$<-\frac{1}{2}$}或$a>1$

 答案:$(-\infty,-\frac{1}{2}) \cup (1,+\infty)$

知识点:二元一次不等式(组)的解集

难度:2

 题目:若区域\textit{A}为不等式组$
\begin{cases}
x \le 0,\\
y \ge 0,\\
y-x \le 2
\end{cases}
$表示的平面区域,则当\textit{a}从\textit{-}2连续变化到1时,动直线\textit{$x+y=a$}扫过\textit{A}中的那部分区域的面积为\textit{\underbar{     }.~}

 \includegraphics*[width=1.13in, height=1.03in, keepaspectratio=false]{image1549}

 解析:如图,区域\textit{A}表示的平面区域为$\mathrm{\vartriangle}$\textit{OBC}内部及其边界组成的图形,当\textit{a}从\textit{-}2连续变化到1时扫过的区域为四边形\textit{ODEC}所围成的区域\textit{.}

又\textit{D}(0,1),\textit{B}(0,2),\textit{E}$(-\frac{1}{2},\frac{3}{2})$,\textit{C}(\textit{-}2,0)\textit{.}

所以$S_{\text{四边形ODEC}}=S_{\triangle OBC}-S_{\triangle BDE}=2-\frac{1}{4}=\frac{7}{4}$

 答案:$\frac{7}{4}$

知识点:二元一次不等式(组)的解集

难度:2

 题目:以原点为圆心的圆全部在不等式组$
\begin{cases}
x-3y+6 \ge 0,\\
x-y +2 \ge 0
\end{cases}$表示的平面区域的内部,则圆的面积的最大值为\textit{\underbar{     }.~}

 解析:根据条件画出平面区域如图中阴影所示,要使以原点为圆心的圆面积最大,则圆与直线\textit{x-y$+$}2\textit{=}0相切\textit{.}此时半径$r = \frac{|0-0+2|}{\sqrt{2}}=\sqrt{2}$,此时圆面积为$S=\pi{(\sqrt{2})}^2=2\pi$

 \includegraphics*[width=1.68in, height=0.92in, keepaspectratio=false]{image1556}

 答案:$2\piup$

知识点:二元一次不等式(组)的解集

难度:2

 题目:若不等式组$
\begin{cases}
x-y \ge 0,\\
2x+y \le 2,\\
y \ge 0,\\
x+y \le a.
\end{cases}$表示的平面区域是一个三角形,则\textit{a}的取值范围是\textit{\underbar{       }.~}

 \includegraphics*[width=0.95in, height=0.98in, keepaspectratio=false]{image1559}

 解析:不等式表示的平面区域如图,

当\textit{x+y=a}过\textit{A}$(\frac{2}{3},\frac{2}{3})$时,表示的区域是$\mathrm{\vartriangle}$\textit{AOB},此时\textit{a=}$\frac{4}{3}$

当\textit{a$>\frac{4}{3}$}时,表示区域是三角形\textit{.}

当\textit{x$+$y=a}过\textit{B}(1,0)时,表示的区域是$\mathrm{\vartriangle}$\textit{DOB},此时\textit{a=}1;当0\textit{$<$a$<$}1时,表示区域是三角形;当\textit{a$<$}0时,不表示任何区域,当1\textit{$<$a$<\frac{4}{3}$}时,表示区域是四边形\textit{.}

故当0\textit{$<$a}$\mathrm{\le}$1或$a \ge \frac{4}{3}$时,表示的平面区域为三角形\textit{.}

 答案:$(0,1]\cup [ \frac{4}{3},+\infty)$

知识点:二元一次不等式(组)的解集

难度:2

 题目:已知点\textit{P}(1,\textit{-}2)及其关于原点对称点均在不等式2\textit{$x+by+$}1\textit{$>$}0表示的平面区域内,求\textit{b}的取值范围\textit{.}

 答案:点\textit{P}(1,\textit{-}2)关于原点对称点为\textit{P'}(\textit{-}1,2),

由题意知$
\begin{cases}
2-2b+1 >0,\\
-2+2b +1>0,
\end{cases}$解得$\frac{1}{2}<b<\frac{3}{2}$

故满足条件的\textit{b}的取值范围$(\frac{1}{2},\frac{3}{2})$

知识点:二元一次不等式(组)的解集

难度:2

 题目:一个小型家具厂计划生产两种类型的桌子A和B\textit{.}每类桌子都要经过打磨、着色、上漆三道工序\textit{.}桌子A需要10 min打磨,6 min着色,6 min上漆;桌子B需要5 min打磨,12 min着色,9 min上漆\textit{.}如果一个工人每天打磨和上漆分别至多工作450 min,着色每天至多工作480 min,请列出满足生产条件的数学关系式,并在直角坐标系中画出每天生产两类桌子数量的允许范围\textit{.}

 答案:设家具厂每天生产A类桌子\textit{x}张,B类桌子\textit{y}张\textit{.}

对于A类\textit{x}张桌子需要打磨10\textit{x} min,着色6\textit{x} min,上漆6\textit{x} min;对于B类\textit{y}张桌子需要打磨5\textit{y} min,着色12\textit{y} min,上漆9\textit{y} min\textit{.}

所以题目中包含的限制条件为$
\begin{cases}
10x+5y \le 450,\\
6x +12y \le 480,\\
6x+9y \le 450,\\
x \in N,\\
y \in N.
\end{cases}$

上述条件表示的平面区域如图中阴影部分所示,每天生产两类桌子数量的允许范围为阴影内的整数点\textit{.}

 \includegraphics*[width=1.65in, height=1.18in, keepaspectratio=false]{image1571}

知识点:简单线性规划

难度:1

 题目:若\textit{x},\textit{y}满足$
\begin{cases}
x \le 3,\\
x+y \ge 2,\\
y \le x.
\end{cases}$则\textit{x$+$}2\textit{y}的最大值为(\textit{  })

 

 A.1 B.3 C.5 D.9

 解析:由题意画出可行域(如图)\textit{.}

 \includegraphics*[width=1.71in, height=1.53in, keepaspectratio=false]{image1573}

设\textit{z=x$+$}2\textit{y},则\textit{z=x$+$}2\textit{y}表示斜率为$-\frac{1}{2}$的一组平行线,当过点\textit{C}(3,3)时,目标函数取得最大值\textit{z}${}_{max}$\textit{=}3\textit{$+$}2\textit{$\times$}3\textit{=}9\textit{.}故选D\textit{.}

 答案:D

知识点:简单线性规划

难度:1

 题目:已知\textit{x},\textit{y}满足约束条件$
\begin{cases}
x-2y+5 \le 0,\\
x+3\ge 0,\\
y \le 2
\end{cases}$则\textit{z=x$+$}2\textit{y}的最大值是(\textit{  })

 A.\textit{-}3 B.\textit{-}1 

 C.1 D.3

 解析:可行域为如图所示阴影部分(包括边界)\textit{.}

 \includegraphics*[width=1.78in, height=1.46in, keepaspectratio=false]{image1576}

把\textit{z=x$+$}2\textit{y}变形为$y=-\frac{1}{2}x+\frac{1}{2}z$,作直线\textit{l}${}_{0}$:$y=-\frac{1}{2}x$并向上平移,当直线过点\textit{A}时,\textit{z}取最大值,易求点\textit{A}的坐标为(\textit{-}1,2),所以\textit{z}${}_{max}$\textit{=-}1\textit{$+$}2\textit{$\times$}2\textit{=}3\textit{.}

 答案:D

知识点:简单线性规划

难度:1

 题目:已知在平面直角坐标系\textit{xOy}内的区域\textit{D}由不等式组$
\begin{cases}
0 \le x \le \sqrt{2},\\
y \le 2,\\
x \le \sqrt{2}y
\end{cases}$给定\textit{.}若\textit{M}(\textit{x},\textit{y})为\textit{D}上的动点,点\textit{A}的坐标为$(\sqrt{2},1)$,则$z=\vec{OM}\cdot\vec{OA}$的最大值为(\textit{  })

 A.$4\sqrt{2}$ B.$3\sqrt{2}$ C.4 D.3

 解析:画出可行域,而$z=\vec{OM}\cdot\vec{OA}=\sqrt{2}(x+y)$,所以$y=-\sqrt{2}x+z$令\textit{l}${}_{0}$:$y=-\sqrt{2}x$,将\textit{l}${}_{0}$平移到过点$(\sqrt{2},2)$时,截距\textit{z}有最大值,故\textit{z}${}_{max}$$=\sqrt{2}\times\sqrt{2}+2=4$

 \includegraphics*[width=1.54in, height=1.36in, keepaspectratio=false]{image1590}

 答案:C

知识点:简单线性规划

难度:1

 题目:已知\textit{x},\textit{y}满足$
\begin{cases}
x+y \le 4,\\
2x + y \ge 3,\\
0 \le x\le 3,\\
y \ge 1.
\end{cases}$则点\textit{P}(\textit{x},\textit{y})到直线$x+y=-2$的距离的最小值为(\textit{  })

 A.$\sqrt{2}$ B.$2\sqrt{2}$ C.$\frac{\sqrt{2}}{2}$ D.$\frac{5\sqrt{2}}{2}$

 \includegraphics*[width=1.33in, height=1.24in, keepaspectratio=false]{image1596}

 解析:不等式组

 $
\begin{cases}
x+y \le 4,\\
2x+y \ge 3,\\
0 \le x \le 3,\\
y \ge 1.
\end{cases}$所表示的可行域如图阴影部分\textit{.}

其中点\textit{P}(1,1)到直线的距离最短,其最小值为$\frac{2+2}{\sqrt{2}}=2\sqrt{2}$.故选B\textit{.}

 答案:B

知识点:简单线性规划

难度:1

 题目:若点(\textit{x},\textit{y})位于曲线$y=|x-1|$与\textit{y=}2所围成的封闭区域,则2\textit{x-y}的最小值为\textit{\underbar{     }.~}

 解析:由
$
y=|x-1|=
\begin{cases}
x-1,x\ge 1,\\
-x + 1, x<1
\end{cases}$及\textit{y=}2画出可行域如图阴影部分\textit{.}

 \includegraphics*[width=1.56in, height=1.06in, keepaspectratio=false]{image1601}

令2\textit{x-y=z},则\textit{y=}2\textit{x-z},画直线\textit{l}${}_{0}$:\textit{y=}2\textit{x}并平移到过点\textit{A}(\textit{-}1,2)时,\textit{-z}最大,即\textit{z}${}_{min}$\textit{=}2\textit{$\times$}(\textit{-}1)\textit{-}2\textit{=-}4\textit{.}

 答案:\textit{-}4

知识点:简单线性规划

难度:1

 题目:若变量\textit{x},\textit{y}满足约束条件$
\begin{cases}
3 \le 2x +y \le 9,\\
6 \le x-y \le 9.
\end{cases}$则\textit{z=x$+$}2\textit{y}的最小值为\textit{\underbar{     }.~}

 \includegraphics*[width=1.24in, height=1.68in, keepaspectratio=false]{image1603}

 解析:根据$
\begin{cases}
3 \le 2x+y \le 9,\\
6\le x-y \le 9
\end{cases}$得可行域如图,根据$z=x+2y$,得$y=-\frac{x}{2}+\frac{z}{2}$,平移直线$y=-\frac{1}{2}$,在点\textit{M}处\textit{z}取得最小值\textit{.}

由$
\begin{cases}
x-y=9,\\
2x+y=3
\end{cases}$得
 $
\begin{cases}
x=4,\\
y=-5
\end{cases}$此时\textit{z}${}_{min}$$=4+2\times(-5)=-6$

 答案:\textit{-}6

知识点:简单线性规划

难度:1

题目:若实数\textit{x},\textit{y}满足$
\begin{cases}
x-y+1\ge0,\\
x+y \ge 0,\\
x\le 0
\end{cases}$则\textit{z=}3\textit{${}^{x+}$}${}^{2}$\textit{${}^{y}$}的最小值为\textit{\underbar{     }.~}

 \includegraphics*[width=1.20in, height=1.11in, keepaspectratio=false]{image1610}

 解析:不等式组所表示的可行域如图阴影部分\textit{.}

令$t=x+2y$,则当直线$y=-\frac{1}{2}x+\frac{1}{2}t$经过原点\textit{O}(0,0)时,$\frac{1}{2}t$取最小值,即\textit{t}的最小值为0,则\textit{z=}3\textit{${}^{x+}$}${}^{2}$\textit{${}^{y}$}的最小值为3${}^{0}$\textit{=}1\textit{.}

 答案:1

知识点:简单线性规划

难度:1

题目:若实数\textit{x},\textit{y}满足不等式组$
\begin{cases}
x \ge 1,\\
x-y +1 \le 0,\\
2x-y-2\le 0
\end{cases}$,则${(x+2)}^2+{(y+1)}^2$的最小值为\textit{\underbar{     }.~}

 解析:画出不等式组表示的平面区域,如图阴影部分\textit{.}

 \includegraphics*[width=1.50in, height=1.39in, keepaspectratio=false]{image1616}

$\sqrt{{(x+2)}^2+{(y+1)}^2}$表示可行域内的点\textit{D}(\textit{x},\textit{y})与定点\textit{M}(\textit{-}2,\textit{-}1)间的距离\textit{.}显然当点\textit{D}在点\textit{A}(1,2)时,$|DM|$最小,这时$|DM|=3\sqrt{2}$,故${(x+2)}^2+{(y+1)}^2$的最小值是18\textit{.}

 答案:18

知识点:简单线性规划

难度:1

题目:已知\textit{x},\textit{y}满足约束条件$
\begin{cases}
x+y\le 6,\\
5x+9y \le 45,\\
x\ge 0,y\ge 0.
\end{cases}$求\textit{z=}5\textit{x-}8\textit{y}的最大值\textit{.}

 答案:作出不等式组$
\begin{cases}
x+y\le 6,\\
5x+9y \le 45,\\
x\ge 0,y\ge 0.
\end{cases}$
表示的可行域,如图阴影部分\textit{.}

 \includegraphics*[width=1.74in, height=1.10in, keepaspectratio=false]{image1621}

作直线\textit{l}${}_{0}$:5\textit{x-}8\textit{y=}0,平移直线\textit{l}${}_{0}$,由图可知,当直线平移到经过\textit{A}点时,\textit{z}取最大值\textit{.}解方程组$
\begin{cases}
x+y=6,\\
y=0
\end{cases}$得\textit{A}(6,0),所以\textit{z}${}_{max}$\textit{=}5\textit{$\times$}6\textit{-}8\textit{$\times$}0\textit{=}30\textit{.}

知识点:简单线性规划

难度:1

 题目:已知\textit{-}4$\mathrm{\le}$\textit{a-b}$\mathrm{\le}$\textit{-}1,\textit{-}1$\mathrm{\le}$4\textit{a-b}$\mathrm{\le}$5,求9\textit{a-b}的取值范围\textit{.}

 答案:如图所示,令\textit{a=x},\textit{b=y},\textit{z=}9\textit{a-b},即已知\textit{-}4$\mathrm{\le}$\textit{x-y}$\mathrm{\le}$\textit{-}1,\textit{-}1$\mathrm{\le}$4\textit{x-y}$\mathrm{\le}$5,求\textit{z=}9\textit{x-y}的取值范围,画出不等式表示的可行域如图阴影部分\textit{.}

 \includegraphics*[width=1.43in, height=1.40in, keepaspectratio=false]{image1624}

由\textit{z=}9\textit{x-y},得\textit{y=}9\textit{x-z},当直线过点\textit{A}时,\textit{z}取最大值,当直线过点\textit{B}时,\textit{z}取最小值\textit{.}

由$
\begin{cases}
4x-y=5,\\
x-y=-4
\end{cases}$得\textit{A}(3,7),

由$
\begin{cases}
4x-y=-1,\\
x-y=-1
\end{cases}$得\textit{B}(0,1),

所以\textit{z}${}_{max}$\textit{=}9\textit{$\times$}3\textit{-}7\textit{=}20,\textit{z}${}_{min}$\textit{=-}1,

所以9\textit{a-b}的取值范围是[\textit{-}1,20]\textit{.}

知识点:简单线性规划

难度:2

题目:在约束条件$
\begin{cases}
y \le x,\\
y \ge \frac{1}{2}x,\\
x+y \le 1
\end{cases}$下,目标函数$z=x+\frac{1}{2}y$的最大值为(\textit{  })

 A.$\frac{1}{4}$ B.$\frac{3}{4}$ C.$\frac{5}{6}$ D.$\frac{5}{3}$

 解析:由$z=x+\frac{1}{2}y$,得$y=-2x+2z$

 \includegraphics*[width=1.32in, height=1.29in, keepaspectratio=false]{image1634}

作出可行域如图阴影部分,平移直线\textit{y=-}2\textit{x$+$}2\textit{z},当直线经过点\textit{C}时,直线\textit{y=-}2\textit{x$+$}2\textit{z}在\textit{y}轴上的截距最大,此时\textit{z}最大\textit{.}

由$
\begin{cases}
y=\frac{1}{2}x,\\
x+y=1
\end{cases}$解得点\textit{C}坐标为$(\frac{2}{3},\frac{1}{3})$,代入$z=x+\frac{1}{2}y$,得$z=\frac{2}{3}+\frac{1}{2}\times\frac{1}{3}=\frac{5}{6}$

 答案:C

知识点:简单线性规划

难度:2

题目:已知\textit{x},\textit{y}满足约束条件$
\begin{cases}
x\ge 0,\\
y \ge 0,\\
x+y \ge 1
\end{cases}$则${(x+3)}^2+y^2$的最小值为(\textit{  })

 A.$\sqrt{10}$ B.$2\sqrt{2}$ C.8 D.10

 \includegraphics*[width=1.33in, height=0.90in, keepaspectratio=false]{image1642}

 解析:画出可行域(如图)\textit{.}

${(x+3)}^2+y^2$表示点\textit{A}(\textit{-}3,0)与可行域内点(\textit{x},\textit{y})间距离的平方\textit{.}显然$|AC|$长度最小,

所以${|AC|}^2={(0+3)}^2+{(1-0)}^2=10$

 答案:D

知识点:简单线性规划

难度:2

题目:若关于\textit{x},\textit{y}的不等式组$
\begin{cases}
2x-y+1 >0,\\
x+m<0,\\
y-m>0
\end{cases}$表示的平面区域内存在点\textit{P}(\textit{x}${}_{0}$,\textit{y}${}_{0}$),满足\textit{x}${}_{0}$\textit{-}2\textit{y}${}_{0}$\textit{=}2\textit{.}则\textit{m}的取值范围是(\textit{  })

 A.$(-\infty,\frac{4}{3})$ B.$(-\infty,\frac{1}{3})$

 C.$(-\infty,-\frac{2}{3})$ D.$(-\infty,-\frac{5}{3})$

 \includegraphics*[width=1.33in, height=1.01in, keepaspectratio=false]{image1648}

 解析:由线性约束条件可画出如图所示的可行域,要使可行域内存在点\textit{P}(\textit{x}${}_{0}$,\textit{y}${}_{0}$),使\textit{x}${}_{0}$\textit{-}2\textit{y}${}_{0}$\textit{=}2成立,只需点\textit{A}(\textit{-m},\textit{m})在直线\textit{x-}2\textit{y-}2\textit{=}0的下方即可,

即\textit{-m-}2\textit{m-}2\textit{$>$}0,解得$m<-\frac{2}{3}$.故选C\textit{.}

 答案:C

知识点:简单线性规划

难度:2

题目:设不等式组$
\begin{cases}
x \ge 1,\\
x-2y+3 \ge0,\\
y \ge x
\end{cases}$所表示的平面区域是\textit{$\mathit{\Omega}$}${}_{1}$,平面区域\textit{$\mathit{\Omega}$}${}_{2}$与\textit{$\mathit{\Omega}$}${}_{1}$关于直线3\textit{x-}4\textit{y-}9\textit{=}0对称\textit{.}对于\textit{$\mathit{\Omega}$}${}_{1}$中的任意点\textit{A}与\textit{$\mathit{\Omega}$}${}_{2}$中的任意点\textit{B},则$|AB|$的最小值为(\textit{  })

 A.$\frac{28}{5}$ B.4 C.$\frac{12}{5}$ D.2

 解析:如图所示\textit{.}由约束条件作出可行域,得\textit{D}(1,1),\textit{E}(1,2),\textit{C}(3,3)\textit{.}

 \includegraphics*[width=1.63in, height=1.25in, keepaspectratio=false]{image1653}

要求$|AB|$${}_{min}$,可通过求可行域内的点到直线3\textit{x-}4\textit{y-}9\textit{=}0距离最小值的2倍来求得\textit{.}

经分析,点\textit{D}(1,1)到直线3\textit{x-}4\textit{y-}9\textit{=}0的距离$d=\frac{|3\times1+4\times1-9|}{5}=2$最小,故$|AB|$${}_{min}$\textit{=}4\textit{.}

 答案:B

知识点:简单线性规划

难度:2

题目:已知实数\textit{x},\textit{y}满足不等式组$
\begin{cases}
x-y+2 \ge 0,\\
x+y-4\ge0,\\
2x-y-5 \le0
\end{cases}$若目标函数\textit{z=y-ax}取得最大值时的唯一解是(1,3),则实数\textit{a}的取值范围为(\textit{  })

 A.(\textit{-$\infty$},\textit{-}1) B.(0,1)

 C.[1,\textit{$+\infty$}) D.(1,\textit{$+\infty$})

 解析:作出不等式组对应的平面区域如图阴影部分所示,由\textit{z=y-ax},得$y=ax+z$,要使目标函数$y=ax+z$仅在点(1,3)处取最大值,则只需直线\textit{y=ax+z}仅在点\textit{B}(1,3)处的截距最大,由图像可知\textit{a$>$k${}_{BD}$},因为\textit{k${}_{BD}$=}1,所以\textit{a$>$}1,即\textit{a}的取值范围是$(1,+\infty)$.

 \includegraphics*[width=1.70in, height=1.61in, keepaspectratio=false]{image1657}

 答案:D

知识点:简单线性规划

难度:2

题目:设实数\textit{x},\textit{y}满足$
\begin{cases}
x-y-2 \le 0,\\
x+ 2y-5 \ge 0,\\
y-2 \le 0
\end{cases}$,则$z=\frac{y}{x}+\frac{x}{y}$的取值范围是\textit{\underbar{      }.~}

 解析:令$k=\frac{y}{x}$,则\textit{y=kx.}因为\textit{x}$\mathrm{\neq}$0,所以\textit{k}存在,直线\textit{y=kx}恒过原点,而在可行域$
\begin{cases}
x-y-2 \le 0,\\
x+2y -5 \ge 0,\\
y-2 \le 0
\end{cases}$中,当直线过边界点(1,2)时,斜率有最大值,\textit{k=}2;当直线过边界点(3,1)时,斜率有最小值,$k=\frac{1}{3}$,所以斜率\textit{k}的取值范围是$[\frac{1}{3},2]$,又$z=\frac{y}{x}+\frac{x}{y}=k+\frac{1}{k}$,利用函数单调性的定义可知$k \in [\frac{1}{3},1]$时,$z=k+\frac{1}{4}$为减函数;\textit{k}$\mathrm{\in}$[1,2]时,$z=k+\frac{1}{4}$为增函数,可得\textit{z}的取值范围为$[2,\frac{10}{3}]$.

 答案:$[2,\frac{10}{3}]$

知识点:简单线性规划

难度:2

题目:若\textit{x},\textit{y}满足约束条件$
\begin{cases}
x+y \ge 1,\\
-x +y \le1,\\
2x-y \le 2
\end{cases}
$
 (1)求目标函数$z=\frac{1}{2}x-y+\frac{1}{2}$的最值;

 (2)求点\textit{P}(\textit{x},\textit{y})到直线\textit{y=-x-}2的距离的最大值\textit{.}

 答案:(1)根据约束条件,作出可行域如图,则直线$x+y=1$,$-x+y=1$,$2x-y=2$的交点分别为\textit{A}(3,4),\textit{B}(0,1),\textit{C}(1,0)\textit{.}

 \includegraphics*[width=1.74in, height=1.31in, keepaspectratio=false]{image1675}

平移直线$\frac{1}{2}x-y+\frac{1}{2}=0$,由图像可知过点\textit{A}时,\textit{z}取得最小值,

\textit{z}${}_{min}$$=\frac{1}{2}\times3-4+\frac{1}{2}=-2$,

过点\textit{C}时,\textit{z}取得最大值,\textit{z}${}_{max}$$+\frac{1}{2}+\frac{1}{2}=1$.

故\textit{z}的最大值为1,最小值为\textit{-}2\textit{.}

(2)由图像可知,所求的最大值即是点\textit{A}到直线$x+y+2=0$的距离,则$d=\frac{|3+4+2|}{\sqrt{1+1}}=\frac{9\sqrt{2}}{2}$

 
\begin{tabular}{c|cccc}
	
	1&2&5& & \\
	2&0&2&3&3\\
	3&1&2&3& 
	
\end{tabular}

知识点:简单线性规划

难度:2

题目:\includegraphics*[width=0.95in, height=0.16in, keepaspectratio=false]{image1683}导学号33194074在直角坐标系\textit{xOy}中,\textit{O}为坐标原点,点\textit{M}的横、纵坐标分别为茎叶图中的中位数和众数,若点\textit{N}(\textit{x},\textit{y})的坐标满足$
\begin{cases}
x^2+y^2 \le 4,\\
2x-y\ge 0,\\
y\ge 0
\end{cases}$,求$\vec{OM}\cdot\vec{ON}$的最大值\textit{.}

 答案:由茎叶图可得中位数为23,众数为23,所以点\textit{M}为(23,23),

所以$\vec{OM}\cdot\vec{ON}=23x+23y$设$z=23x+23y$

作出不等式组对应的平面区域如图\textit{.}

 \includegraphics*[width=1.10in, height=1.15in, keepaspectratio=false]{image1686}

作一平行于\textit{z=}23\textit{x+}23\textit{y}的直线,当直线和圆相切时,\textit{z=}23\textit{x+}23\textit{y}取得最大值\textit{.}

由圆心到直线的距离$d=\frac{|z|}{\sqrt{23^2+23^2}}=\frac{|z|}{23\sqrt{2}}=2$,

解得$|z|=46\sqrt{2}$
所以$z=46\sqrt{2}$或$z=-46\sqrt{2}$(舍去),

故$\vec{OM}\cdot\vec{ON}$的最大值是$46\sqrt{2}$


知识点:简单线性规划

难度:1


 \includegraphics*[width=1.10in, height=1.02in, keepaspectratio=false]{image1693}

 题目:已知点(\textit{x},\textit{y})构成的平面区域如图阴影部分,\textit{z=mx+y}(\textit{m}为常数)在平面区域内取得最大值的最优解有无数多个,则\textit{m}的值为 (\textit{  })

 A.$-\frac{7}{20}$ B.$\frac{7}{20}$

 C.$\frac{1}{2}$ D.$\frac{7}{20}$或$\frac{1}{2}$


解析:观察平面区域可知直线$y=-mx+z$与直线\textit{AC}重合,则$-m=k_{AC}=\frac{\frac{22}{5}-3}{1-5}=-\frac{7}{20}$,解得$m=\frac{7}{20}$

 答案:B

知识点:简单线性规划

难度:1

 \includegraphics*[width=1.33in, height=1.27in, keepaspectratio=false]{image1701}

 题目:如图,目标函数\textit{z=ax-y}的可行域为四边形\textit{OACB}(含边界),若\textit{C}$(\frac{2}{3},\frac{4}{5})$是该目标函数\textit{z=ax-y}唯一的最优解,则\textit{a}的取值范围是(\textit{  })

 

 A.$(-\frac{10}{3},-\frac{5}{12})$

 B.$(-\frac{12}{5},-\frac{3}{10})$

 C.$(\frac{3}{10},\frac{12}{5})$

 D.$(\frac{12}{5},\frac{3}{10})$

 解析:最优解为点\textit{C},则目标函数表示的直线斜率在直线\textit{BC}与\textit{AC}的斜率之间\textit{.}

因为$K_{BC}=-\frac{3}{10}$,$k_{AC}=-\frac{12}{5}$,所以$a \in (-\frac{12}{5},-\frac{3}{10})$

 答案:B

知识点:简单线性规划

难度:1

 题目:若直线\textit{y=}2\textit{x}上存在点(\textit{x},\textit{y})满足约束条件$
\begin{cases}
x+y-3 \le 0,\\
x-2y-3\le 0,\\
x \ge m.
\end{cases}$则实数\textit{m}的最大值为\textit{\underbar{     }.~}

 解析:由约束条件作出其可行域如图\textit{.}

 \includegraphics*[width=1.43in, height=1.39in, keepaspectratio=false]{image1711}

由图可知,当直线\textit{x=m}过直线\textit{y=}2\textit{x}与\textit{x$+$y-}3\textit{=}0的交点(1,2)时,\textit{m}取得最大值,此时\textit{m=}1\textit{.}

 答案:1

知识点:简单线性规划

难度:1

题目:某公司租赁甲、乙两种设备生产A,B两类产品,甲种设备每天能生产A类产品5件和B类产品10件,乙种设备每天能生产A类产品6件和B类产品20件\textit{.}已知设备甲每天的租赁费为200元,设备乙每天的租赁费为300元\textit{.}现该公司至少要生产A类产品50件,B类产品140件,则所需租赁费最少为\textit{\underbar{     }}元\textit{.~}

 解析:设甲种设备需要生产\textit{x}天,乙种设备需要生产\textit{y}天,此时该公司所需租赁费为\textit{z}元,

则\textit{z=}200\textit{x$+$}300\textit{y.}

又因为
$
\begin{cases}
5x+6y \ge 50,\\
10x + 20y \ge 140,\\
x \in N,\\
y \in N
\end{cases}$
即
$
\begin{cases}
x+\frac{6}{5}y \ge10,\\
x+2y \ge 14,\\
x \in N,\\
y \in N.
\end{cases}$
画出该不等式组表示的平面区域,如图阴影部分所示\textit{.}

 \includegraphics*[width=1.65in, height=1.38in, keepaspectratio=false]{image1713}

解$
\begin{cases}
x+\frac{6}{5}y=10,\\
x+2y=14.
\end{cases}$
得
$
\begin{cases}
x=4,\\
y=5
\end{cases}$即点\textit{A}(4,5)\textit{.}

由$z=200x+300y$,

得直线$y=-\frac{2}{3}x+\frac{z}{300}$过点\textit{A}(4,5)时,

\textit{z=}200\textit{x$+$}300\textit{y}取得最小值,为2 300元\textit{.}

 答案:2 300

知识点:简单线性规划

难度:1

题目:设不等式组$
\begin{cases}
x+y-11 \ge 0,\\
3x-y+3\ge0,\\
5x-3y+9\le 0
\end{cases}$表示的平面区域为\textit{D.}若指数函数\textit{y=a${}^{x}$}的图像上存在区域\textit{D}上的点,则\textit{a}的取值范围是\textit{\underbar{     }.~}

 解析:画出可行域如图阴影部分,易知当\textit{a}$\mathrm{\in}$(0,1)时不符合题意,故\textit{a$>$}1\textit{.}

 \includegraphics*[width=1.45in, height=1.14in, keepaspectratio=false]{image1719}

由$
\begin{cases}
x+y-11=0,\\
3x-y+3=0
\end{cases}$得交点\textit{A}(2,9)\textit{.}

由图像可知,当\textit{y=a${}^{x}$}的图像经过该交点\textit{A}时,\textit{a}取最大值,此时\textit{a}${}^{2}$\textit{=}9,所以\textit{a=}3\textit{.}

故\textit{a}$\mathrm{\in}$(1,3]\textit{.}

 答案:(1,3]

知识点:简单线性规划

难度:1

 题目:某养鸡场有1万只鸡,用动物饲料和谷物饲料混合喂养\textit{.}每天每只鸡平均吃混合饲料0\textit{.}5 kg,其中动物饲料不能少于谷物饲料的$\frac{1}{5}$\textit{.}动物饲料每千克0\textit{.}9元,谷物饲料每千克0\textit{.}28元,饲料公司每周仅保证供应谷物饲料50 000 kg,问饲料怎样混合,才使成本最低?

 \includegraphics*[width=0.97in, height=0.85in, keepaspectratio=false]{image1722}

 答案:设每周需用谷物饲料\textit{x} kg,动物饲料\textit{y} kg,每周总的饲料费用为\textit{z}元,则$
\begin{cases}
x+y \ge 3500,\\
y \ge \frac{1}{5}x,\\
0 \le x \le 50000,\\
y \ge 0
\end{cases}$

而$z=0.28x+0.9y$,如图,作出不等式组所表示的平面区域,即可行域\textit{.}作一组平行直线$0.28x+0.9y=t$其中经过可行域内的点\textit{A}时,\textit{z}最小,又直线\textit{x$+$y=}35 000和直线$y=\frac{1}{5}x$的交点\textit{A}$(\frac{87500}{3},\frac{17500}{3})$

即$x=\frac{87500}{3}$,$y=\frac{17500}{3}$时,饲料费用最低\textit{.}

答:谷物饲料和动物饲料应按5\textit{$:$}1的比例混合,此时成本最低\textit{.}


知识点:简单线性规划

难度:2

题目:某学校用800元购买A,B两种教学用品,A种用品每件100元,B种用品每件160元,两种用品至少各买一件,要使剩下的钱最少,A,B两种用品应各买的件数为 (\textit{  })

 

 A.1件,4件 B.3件,3件

 C.4件,2件 D.不确定

 解析:设买A种用品\textit{x}件,B种用品\textit{y}件,剩下的钱为\textit{z}元,

则$
\begin{cases}
x\ge 1, x\in N,\\
y \ge 1, y \in N,\\
100x + 160 \le 800.
\end{cases}$

求\textit{z=}800\textit{-}100\textit{x-}160\textit{y}取得最小值时的整数解(\textit{x},\textit{y}),用图解法求得整数解为(3,3)\textit{.}

 答案:B


知识点:简单线性规划

难度:2

题目:已知\textit{x},\textit{y}满足条件$
\begin{cases}
y \ge 0,\\
y \le x,\\
2x + y +k \le 0
\end{cases}$(\textit{k}为常数),若目标函数$z=x+3y$的最大值为8,则\textit{k=}(\textit{  })

 A.\textit{-}16 B.\textit{-}6 C.$-\frac{8}{3}$ D.6

 解析:由$z=x+3y$得$y=-\frac{1}{3}x+\frac{z}{3}$

先作出$
\begin{cases}
y \ge 0,\\
y \le x
\end{cases}$的图像,因为目标函数$z=x+3y$的最大值为8,

所以直线$2x+y+k=0$过直线$x+3y=8$与直线\textit{y=x}的交点\textit{A},由$
\begin{cases}
x+3y=8,\\
y=x
\end{cases}$解得\textit{A}(2,2),代入直线$2x+y+k=0$,得\textit{k=-}6\textit{.}故选B\textit{.}

 \includegraphics*[width=1.89in, height=1.22in, keepaspectratio=false]{image1735}

 答案:B

 \includegraphics*[width=1.29in, height=0.86in, keepaspectratio=false]{image1736}


知识点:简单线性规划

难度:2

题目:已知在图中的可行域内(阴影部分,且包括边界),目标函数$z=x+ay$取得最小值的最优解有无数个,则\textit{a}的值为(\textit{  })

 A.\textit{-}3 B.3

 C.\textit{-}1 D.1

 解析:当\textit{a=}0时,\textit{z=x.}仅当直线\textit{x=z}过点\textit{A}(1,1)时,

目标函数\textit{z}有最小值1,与题意不符\textit{.}

当\textit{a$>$}0时,$y=-\frac{1}{a}x+\frac{z}{a}$

斜率$k=-\frac{1}{a}<0$,仅当直线$z=x+ay$过点\textit{A}(1,1)时,直线在\textit{y}轴的截距最小,此时\textit{z}也最小,

与目标函数取得最小值的最优解有无数个矛盾\textit{.}

当$a<0$时,$y=-\frac{1}{a}+\frac{z}{a}$,斜率$k=-\frac{1}{a}>0$,

为使目标函数\textit{z}取得最小值的最优解有无数个,

当且仅当斜率$-\frac{1}{a}=k_{AO}$,即$-\frac{1}{a}=\frac{1}{3}$,故$a=-3$

 答案:A


知识点:简单线性规划

难度:2

题目:已知点\textit{M}在不等式组$
\begin{cases}
x-2 \le 0,\\
3x+4y\ge 4,\\
y-3 \le 0
\end{cases}$所表示的平面区域上,点\textit{N}在曲线$x^2+y^2+4x+3=0$上,则$|MN|$的最小值是(\textit{  })

 A.$\frac{1}{2}$ B.1 C.$\frac{2\sqrt{10}}{3}-1$ D.$\frac{2\sqrt{10}}{3}$

 解析:如图,画出平面区域(阴影部分所示),

 \includegraphics*[width=1.56in, height=1.42in, keepaspectratio=false]{image1750}

由圆心\textit{C}(\textit{-}2,0)向直线3\textit{x+}4\textit{y-}4\textit{=}0作垂线,圆心\textit{C}(\textit{-}2,0)到直线3\textit{x+}4\textit{y-}4\textit{=}0的距离为$\frac{|3\times(-2)+4\times0-4|}{\sqrt{3^2+4^2}}=2$,又圆的半径为1,所以可求得$|MN|$的最小值是1\textit{.}故选B\textit{.}

 答案:B


知识点:简单线性规划

难度:2

题目:毕业庆典活动中,某班团支部决定组织班里48名同学去水上公园坐船观赏风景,于是先派一人去了解船只的租金情况,看到的租金价格如下表,则他们合理设计租船方案后,所付租金最少为\textit{\underbar{     }}元\textit{.~}

\begin{tabular}{|p{0.4in}|p{1.5in}|p{1.3in}|} \hline 
	船型 & 每只船限载人数 & 租金\textit{/}(元\textit{/}只) \\ \hline 
	大船 & 5 & 12 \\ \hline 
	小船 & 3 & 8 \\ \hline 
\end{tabular}



 解析:设租大船\textit{x}只,小船\textit{y}只,

则$
\begin{cases}
x\in N,\\
y \in N,\\
5x+3y \ge 48.
\end{cases}$租金$z=12x+8y$,

 \includegraphics*[width=1.15in, height=1.07in, keepaspectratio=false]{image1753}

作出可行域如图,由图可知,当直线$z=12x+8y$经过点(9\textit{.}6,0)时,\textit{z}取最小值,但\textit{x},\textit{y}$\mathrm{\in}$N,所以当\textit{x=}9,\textit{y=}1时,\textit{z}${}_{min}$\textit{=}116\textit{.}

 答案:116

 
知识点:简单线性规划

难度:2

题目:铁矿石A和B的含铁率\textit{a},冶炼每万吨铁矿石的CO${}_{2}$的排放量\textit{b}及每万吨铁矿石的价格\textit{c}如表:

\begin{tabular}{|p{0.2in}|p{0.5in}|p{0.9in}|p{1.2in}|} \hline 
	& \textit{a} & \textit{b/}万吨 & \textit{c/}百万元 \\ \hline 
	A & 50\% & 1 & 3 \\ \hline 
	B & 70\% & 0\textit{.}5 & 6 \\ \hline 
\end{tabular}



 某冶炼厂至少要生产1\textit{.}9万吨铁,若要求CO${}_{2}$的排放量不超过2万吨,则购买铁矿石的最少费用为\textit{\underbar{     }}百万元\textit{.~}

 解析:设购买铁矿石A,B分别为\textit{x}万吨和\textit{y}万吨,

购买铁矿石的费用为\textit{z}百万元,

则$
\begin{cases}
0.5x+0.7y \ge 1.9,\\
x+0.5y \le 2,\\
x \ge 0,\\
y \ge 0
\end{cases}$

 \includegraphics*[width=1.27in, height=1.39in, keepaspectratio=false]{image1755}

目标函数\textit{z=}3\textit{x+}6\textit{y},作出可行域如图\textit{.}

由$
\begin{cases}
0.5x+0.7y=1.9,\\
x+0.5y=2
\end{cases}$
得$
\begin{cases}
x=1,\\
y=2.
\end{cases}$

记\textit{P}(1,2),当目标函数$z=3x+6y$过点\textit{P}(1,2)时,\textit{z}取到最小值15\textit{.}

 答案:15


知识点:简单线性规划

难度:2

题目:电视台播放甲、乙两套连续剧,每次播放连续剧时,需要播放广告\textit{.}已知每次播放甲、乙两套连续剧时,连续剧播放时长、广告播放时长、收视人次如下表所示:

\begin{tabular}{|p{0.2in}|p{1.3in}|p{1.1in}|p{0.7in}|} \hline 
	& 连续剧播放时长\newline (分钟) & 广告播放时长\newline (分钟) & 收视人次\newline (万) \\ \hline 
	甲 & 70 & 5 & 60 \\ \hline 
	乙 & 60 & 5 & 25 \\ \hline 
\end{tabular}



 已知电视台每周安排的甲、乙连续剧的总播放时间不多于600分钟,广告的总播放时间不少于30分钟,且甲连续剧播放的次数不多于乙连续剧播放次数的2倍\textit{.}分别用\textit{x},\textit{y}表示每周计划播出的甲、乙两套连续剧的次数\textit{.}

 (1)用\textit{x},\textit{y}列出满足题目条件的数学关系式,并画出相应的平面区域;

 (2)问电视台每周播出甲、乙两套连续剧各多少次,才能使总收视人次最多?

 答案:(1)由已知,\textit{x},\textit{y}满足的数学关系式为$
\begin{cases}
70x+60y\le 600,\\
5x+5y \ge 30,\\
x \le 2y,\\
x\ge 0,\\
y \ge 0
\end{cases}$
即
$
\begin{cases}
7x+6y \le 60,\\
x+y \ge 6,\\
x-2y \le 0,\\
x \ge 0,\\
y \ge 0.
\end{cases}$

 该二元一次不等式组所表示的平面区域为图1中的阴影部分:

(2)设总收视人次为\textit{z}万,则目标函数为$z=60x+25y$

考虑$z=60x+25y$,将它变形为$y=-\frac{12}{5}x+\frac{z}{25}$,这是斜率为$-\frac{12}{5}$,随\textit{z}变化的一族平行直线\textit{.}

$-\frac{z}{25}$为直线在\textit{y}轴上的截距,当$\frac{z}{25}$取得最大值时,\textit{z}的值最大\textit{.}

 \includegraphics*[width=1.58in, height=1.71in, keepaspectratio=false]{image1765}

 图1

 \includegraphics*[width=1.71in, height=1.71in, keepaspectratio=false]{image1766}

 图2



又因为\textit{x},\textit{y}满足约束条件,所以由图2可知,当直线$z=60x+25y$经过可行域上的点\textit{M}时,截距$\frac{z}{25}$最大,即\textit{z}最大\textit{.}

解方程组$
\begin{cases}
7x+6y=60,\\
x-2y=0.
\end{cases}$得点\textit{M}的坐标为(6,3)\textit{.}

所以,电视台每周播出甲连续剧6次,乙连续剧3次时才能使总收视人次最多\textit{.}

\end{document}

