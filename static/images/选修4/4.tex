% Generated by GrindEQ Word-to-LaTeX 
\documentclass{article} %%% use \documentstyle for old LaTeX compilers

\usepackage[english]{babel} %%% 'french', 'german', 'spanish', 'danish', etc.
\usepackage{amssymb}
\usepackage{amsmath}
\usepackage{txfonts}
\usepackage{mathdots}
\usepackage[classicReIm]{kpfonts}
\usepackage[dvips]{graphicx} %%% use 'pdftex' instead of 'dvips' for PDF output
\usepackage{ctex} 
\usepackage{comment}

\begin{document}

%\selectlanguage{english} %%% remove comment delimiter ('%') and select language if required

知识:基本不等式

难度:1

题目:下列命题中不正确的是(  )

A.若$\sqrt[3]{a}\mathrm{>}\sqrt[3]{b}$,则a$\mathrm{>}$b   B.若a$\mathrm{>}$b,c$\mathrm{>}$d,则$\frac{a}{d}\mathrm{>}\frac{b}{c}$

C.若a$\mathrm{>}$b$\mathrm{>}$0,c$\mathrm{>}$d$\mathrm{>}$0,则$\frac{a}{d}\mathrm{>}\frac{b}{c}$   D.若a$\mathrm{>}$b$\mathrm{>}$0,ac$\mathrm{>}$bd,则c$\mathrm{>}$d

解析:当a$\mathrm{>}$b$\mathrm{>}$0,ac$\mathrm{>}$ad时,c,d的大小关系不确定.

答案:D 



知识:基本不等式

难度:1

题目:已知a$\mathrm{>}$b$\mathrm{>}$c,则下列不等式正确的是(  )

A.ac$\mathrm{>}$bc           B.ac${}^{2}$$\mathrm{>}$bc${}^{2}$

C.b(a-b)$\mathrm{>}$c(a-b)   D.{\textbar}ac{\textbar}$\mathrm{>}${\textbar}bc{\textbar}

解析:a$\mathrm{>}$b$\mathrm{>}$c$\mathrm{\Rightarrow }$a-b$\mathrm{>}$0$\mathrm{\Rightarrow }$(a-b)b$\mathrm{>}$(a-b)c.

答案:C



知识:基本不等式

难度:1

题目:如果a$\mathrm{<}$b$\mathrm{<}$0,那么下列不等式成立的是(  )

A.$\frac{1}{a}\mathrm{<}\frac{1}{b}$   B.ab$\mathrm{<}$b${}^{2}$

C.-ab$\mathrm{<}$-a${}^{2}$   D.-$\frac{1}{a}\mathrm{<}$-$\frac{1}{b}$

解析:对于A项,由a$\mathrm{<}$b$\mathrm{<}$0,得b-a$\mathrm{>}$0,ab$\mathrm{>}$0,故$\frac{1}{a}$-$\frac{1}{b}$=$\frac{b-a}{ab}\mathrm{>}$0,$\frac{1}{a}\mathrm{>}\frac{1}{b}$,故A项错误;对于B项,由a$\mathrm{<}$b$\mathrm{<}$0,得b(a-b)$\mathrm{>}$0,ab$\mathrm{>}$b${}^{2}$,故B项错误;对于C项,由a$\mathrm{<}$b$\mathrm{<}$0,得a(a-b)$\mathrm{>}$0,a${}^{2}$$\mathrm{>}$ab,即-ab$\mathrm{>}$-a${}^{2}$,故C项错误;对于D项,由a$\mathrm{<}$b$\mathrm{<}$0,得a-b$\mathrm{<}$0,ab$\mathrm{>}$0,故-$\frac{1}{a}$-$(-\frac{1}{b})$=$\frac{a-b}{ab}\mathrm{<}$0,-$\frac{1}{a}\mathrm{<}$-$\frac{1}{b}$成立,故D项正确.

答案:D



知识:基本不等式

难度:1

题目:若a>0>b>-a,c<d<0,则下列结论:①ad>bc;②$\frac{a}{d}$+$\frac{b}{c}$<0;③a-c>b-d;④a(d-c)>b(d-c)中,成立的个数是(  )

A.1   B.2

C.3   D.4

解析:$\mathrm{\because}$a>0>b,c<d<0,$\mathrm{\therefore}$ad<0,bc>0,$\mathrm{\therefore}$ad<bc,故①不成立.$\mathrm{\because}$a>0>b>-a,$\mathrm{\therefore}$a>-b>0,$\mathrm{\because}$c<d<0,$\mathrm{\therefore}$-c>-d>0,$\mathrm{\therefore}$a(-c)>(-b)(-d),$\mathrm{\therefore}$ac+bd<0,$\mathrm{\therefore}$$\frac{a}{d}$+$\frac{b}{c}$=$\frac{ac+bd}{cd}$<0,故②成立.$\mathrm{\because}$c<d,$\mathrm{\therefore}$-c>-d,$\mathrm{\because}$a>b,$\mathrm{\therefore}$a+(-c)>b+(-d),a-c>b-d,故③成立.$\mathrm{\because}$a>b,d-c>0,$\mathrm{\therefore}$a(d-c)>b(d-c),故④成立.成立的个数为3.

答案:C



知识:基本不等式

难度:1

题目:给出四个条件:

①b$\mathrm{>}$0$\mathrm{>}$a;②0$\mathrm{>}$a$\mathrm{>}$b;③a$\mathrm{>}$0$\mathrm{>}$b;④a$\mathrm{>}$b$\mathrm{>}$0.

能得出$\frac{1}{a}$<$\frac{1}{b}$成立的有\_\_\_\_\_\_\_\_(填序号).

解析:由$\frac{1}{a}$$\mathrm{<}$$\frac{1}{b}$,得$\frac{1}{a}$-$\frac{1}{b}$$\mathrm{<}$0,$\frac{b-a}{ab}$$\mathrm{<}$0,故①②④可推得$\frac{1}{a}$$\mathrm{<}$$\frac{1}{b}$成立.

答案:①②④



知识:基本不等式

难度:1

题目:设a$\mathrm{>}$b$\mathrm{>}$1,c$\mathrm{<}$0,给出下列三个结论:①$\frac{c}{a}$$\mathrm{>}$$\frac{c}{b}$;②a${}^{c}$$\mathrm{<}$b${}^{c}$;③log${}_{b}$(a-c)$\mathrm{>}$log${}_{a}$(b-c).

其中所有的正确结论的序号是\_\_\_\_\_\_\_\_.

解析:由a$\mathrm{>}$b$\mathrm{>}$1,c$\mathrm{<}$0,得$\frac{1}{a}$$\mathrm{<}$$\frac{1}{b}$,$\frac{c}{a}$$\mathrm{>}$$\frac{c}{b}$;幂函数y=x${}^{c}$(c$\mathrm{<}$0)是减函数,所以a${}^{c}$$\mathrm{<}$b${}^{c}$;因为a-c$\mathrm{>}$b-c,所以log${}_{b}$(a-c)$\mathrm{>}$log${}_{a}$(a-c)$\mathrm{>}$log${}_{a}$(b-c),①②③均正确.

答案:①②③



知识:基本不等式

难度:1

题目:已知-1$\mathrm{<}$x+y$\mathrm{<}$4且2$\mathrm{<}$x-y$\mathrm{<}$3,则z=2x-3y的取值范围是\_\_\_\_\_\_\_\_.

解析:设z=2x-3y=m(x+y)+n(x-y),即2x-3y=(m+n)x+(m-n)y.

$\mathrm{\therefore}$$\left\{\begin{array}{r}
m+n=2\\
m-n=-3
\end{array} \right.$
解得$\left\{\begin{array}{r} m=-\frac{1}{2}\\ n=\frac{5}{2} \end{array}\right.
$$\mathrm{\therefore}$2x-3y=-$\frac{1}{2}$(x+y)+$\frac{5}{2}$(x-y).

$\mathrm{\because}$-1$\mathrm{<}$x+y$\mathrm{<}$4,2$\mathrm{<}$x-y$\mathrm{<}$3,

$\mathrm{\therefore}$-2$\mathrm{<}$-$\frac{1}{2}$(x+y)$\mathrm{<}$$\frac{1}{2}$,5$\mathrm{<}\frac{5}{2}$(x-y)$\mathrm{<}$$\frac{15}{2}$.

由不等式同向可加性,得3$\mathrm{<}$-$\frac{1}{2}$(x+y)+$\frac{5}{2}$(x-y)$\mathrm{<}$8,即3$\mathrm{<}$z$\mathrm{<}$8.

答案:(3,8)



知识:基本不等式

难度:1

题目:若a$\mathrm{>}$0,b$\mathrm{>}$0,求证:$\frac{b^2}{a}+\frac{a^2}{b}\mathrm{\ge}$a+b.

解析:

证明:$\mathrm{\because}$$\frac{b^2}{a}$+$\frac{a^2}{b}$-a-b=(a-b)$(\frac{a}{b}-\frac{b}{a})$=$\frac{(a-b)^2(a+b)}{ab}$,

(a-b)${}^{2}$$\mathrm{\ge}$0恒成立,且已知a>0,b>0,

$\mathrm{\therefore}$a+b$\mathrm{>}$0,ab$\mathrm{>}$0.$\mathrm{\therefore}$$\frac{(a-b)^2(a+b)}{ab}$$\mathrm{\ge}$0.$\mathrm{\therefore}$$\frac{b^2}{a}$+$\frac{a^2}{b}$$\mathrm{\ge}$a+b.



知识:基本不等式

难度:1

题目:已知-6$\mathrm{<}$a$\mathrm{<}$8,2$\mathrm{<}$b$\mathrm{<}$3,分别求2a+b,a-b,$\frac{a}{b}$的取值范围.

解析:

解:$\mathrm{\because}$-6$\mathrm{<}$a$\mathrm{<}$8,$\mathrm{\therefore}$-12$\mathrm{<}$2a$\mathrm{<}$16.

又2$\mathrm{<}$b$\mathrm{<}$3,$\mathrm{\therefore}$-10$\mathrm{<}$2a+b$\mathrm{<}$19.

$\mathrm{\because}$2$\mathrm{<}$b$\mathrm{<}$3,$\mathrm{\therefore}$-3$\mathrm{<}$-b$\mathrm{<}$-2.

又$\mathrm{\because}$-6$\mathrm{<}$a$\mathrm{<}$8,$\mathrm{\therefore}$-9$\mathrm{<}$a-b$\mathrm{<}$6.

$\mathrm{\because}$2$\mathrm{<}$b$\mathrm{<}$3,$\mathrm{\therefore}$$\frac{1}{3}$$\mathrm{<}$$\frac{1}{b}$$\mathrm{<}$$\frac{1}{2}$.

①当0$\mathrm{\le}$a$\mathrm{<}$8时,0$\mathrm{\le}$$\frac{a}{b}$$\mathrm{<}$4;

②当-6<a<0时,-3<$\frac{a}{b}$<0.

综合①②得-3$\mathrm{<}$$\frac{a}{b}$$\mathrm{<}$4.

$\mathrm{\therefore}$2a+b,a-b,$\frac{a}{b}$的取值范围分别为(-10,19),(-9,6),(-3,4).



知识:基本不等式

难度:2

题目:已知a$\mathrm{>}$0,a$\mathrm{\neq}$1.

(1)比较下列各式大小.

①a${}^{2}$+1与a+a;②a${}^{3}$+1与a${}^{2}$+a;

③a${}^{5}$+1与a${}^{3}$+a${}^{2}$.

(2)探讨在m,n$\mathrm{\in}$N${}_{\textrm{+}}$条件下,a${}^{m}$${}^{\textrm{+}}$${}^{n}$+1与a${}^{m}$+a${}^{n}$的大小关系,并加以证明.

解析:

解:(1)由题意,知a$\mathrm{>}$0,a$\mathrm{\neq}$1,

①a${}^{2}$+1-(a+a)=a${}^{2}$+1-2a=(a-1)${}^{2}$$\mathrm{>}$0.

$\mathrm{\therefore}$a${}^{2}$+1$\mathrm{>}$a+a.

②a${}^{3}$+1-(a${}^{2}$+a)=a${}^{2}$(a-1)-(a-1)

=(a+1)(a-1)${}^{2}$>0,$\mathrm{\therefore}$a${}^{3}$+1$\mathrm{>}$a${}^{2}$+a,

③a${}^{5}$+1-(a${}^{3}$+a${}^{2}$)

=a${}^{3}$(a${}^{2}$-1)-(a${}^{2}$-1)=(a${}^{2}$-1)(a${}^{3}$-1).

当a$\mathrm{>}$1时,a${}^{3}$$\mathrm{>}$1,a${}^{2}$$\mathrm{>}$1,$\mathrm{\therefore}$(a${}^{2}$-1)(a${}^{3}$-1)$\mathrm{>}$0.

当0$\mathrm{<}$a$\mathrm{<}$1时,0$\mathrm{<}$a${}^{3}$$\mathrm{<}$1,0$\mathrm{<}$a${}^{2}$$\mathrm{<}$1,

$\mathrm{\therefore}$(a${}^{2}$-1)(a${}^{3}$-1)$\mathrm{>}$0,即a${}^{5}$+1$\mathrm{>}$a${}^{3}$+a${}^{2}$.

(2)根据(1)可得a${}^{m}$${}^{\textrm{+}}$${}^{n}$+1>a${}^{m}$+a${}^{n}$.证明如下:

a${}^{m}$${}^{\textrm{+}}$${}^{n}$+1-(a${}^{m}$+a${}^{n}$)=a${}^{m}$(a${}^{n}$-1)+(1-a${}^{n}$)=(a${}^{m}$-1)(a${}^{n}$-1).

当a$\mathrm{>}$1时,a${}^{m}$$\mathrm{>}$1,a${}^{n}$$\mathrm{>}$1,$\mathrm{\therefore}$(a${}^{m}$-1)(a${}^{n}$-1)$\mathrm{>}$0.

当0$\mathrm{<}$a$\mathrm{<}$1时,0$\mathrm{<}$a${}^{m}$$\mathrm{<}$1,0$\mathrm{<}$a${}^{n}$$\mathrm{<}$1,

$\mathrm{\therefore}$(a${}^{m}$-1)(a${}^{n}$-1)$\mathrm{>}$0.

综上可知(a${}^{m}$-1)(a${}^{n}$-1)$\mathrm{>}$0,即a${}^{m}$${}^{\textrm{+}}$${}^{n}$+1$\mathrm{>}$a${}^{m}$+a${}^{n}$.



知识:基本不等式

难度:1

题目:下列不等式中,正确的个数是(  )

①若a,b$\mathrm{\in}$R,则$\frac{a+b}{2}$$\mathrm{\ge}$$\sqrt{ab}$;

②若x$\mathrm{\in}$R,则x${}^{2}$+2+$\frac{1}{x^2+2}$$\mathrm{\ge}$2;

③若x$\mathrm{\in}$R,则x${}^{2}$+1+$\frac{1}{x^2+1}$$\mathrm{\ge}$2;

④若a,b为正实数,则$\frac{\sqrt{a}+\sqrt{b}}{2}$$\mathrm{\ge}$$\sqrt{ab}$.

A.0            B.1  

C.2  D.3

解析:显然①不正确,③正确;虽然x${}^{2}$+2=$\frac{1}{x^2+2}$无解,但x${}^{2}$+2+$\frac{1}{x^2+2}$$\mathrm{>}$2成立,故②正确;④不正确,如a=1,b=4.

答案:C



知识:基本不等式

难度:1

题目:已知a$\mathrm{>}$0,b$\mathrm{>}$0,a,b的等差中项是$\frac{1}{2}$,且$\alpha$=a+$\frac{1}{a}$,$\beta$=b+$\frac{1}{b}$,则$\alpha$+$\beta$的最小值是(  )

A.3    B.4  

C.5    D.6

解析:$\mathrm{\because}$a+b=2$\mathrm{\times}$$\frac{1}{2}$=1,a$\mathrm{>}$0,b$\mathrm{>}$0,

$\mathrm{\therefore}$$\alpha$+$\beta$=a+$\frac{1}{a}$+b+$\frac{1}{b}$=1+$\frac{1}{ab}$$\mathrm{\ge}$1+$\frac{1}{(\frac{a+b}{2})^2}$=5,

当且仅当a=b=$\frac{1}{2}$时,等号成立.

答案:C



知识:基本不等式

难度:1

题目:已知不等式(x+y)$(\frac{1}{x}+\frac{a}{y})$$\mathrm{\ge}$9对任意的正实数x,y恒成立,则正实数a的最小值为(  )

A.2   B.4  

C.6   D.8

解析: (x+y)$(\frac{1}{x}+\frac{a}{y})$=1+a+$\frac{y}{x}$+$\frac{ax}{y}$$\mathrm{\ge}$1+a+2$\sqrt{a}$=($\sqrt{a}$+1)${}^{2}$(x,y,a$\mathrm{>}$0),当且仅当y=x时取等号,所以$(x+y)\cdot(\frac{1}{x}+\frac{a}{y})$的最小值为($\sqrt{a}$+1)${}^{2}$,于是($\sqrt{a}$+1)${}^{2}$$\mathrm{\ge}$9恒成立,所以a$\mathrm{\ge}$4,故选B.

答案:B



知识:基本不等式

难度:1

题目:要制作一个容积为4 m${}^{3}$,高为1 m的无盖长方体容器.已知该容器的底面造价是每平方米20元,侧面造价是每平方米10元,则该容器的最低总造价是(  )

A.80元  B.120元  

C.160元  D.240元

解析:设底面矩形的长和宽分别为a m,b m,则ab=4.容器的总造价为20ab+2(a+b)$\mathrm{\times}$10=80+20(a+b)$\mathrm{\ge}$80+40$\sqrt{ab}$=160(元)(当且仅当a=b=2时,等号成立).

答案:C



知识:基本不等式

难度:1

题目:已知函数f(x)=4x+$\frac{a}{x}$(x>0,a>0)在x=3时取得最小值,则a=\_\_\_\_\_\_\_\_.

解析:$\mathrm{\because}$x>0,a>0,

$\mathrm{\therefore}$f(x)=4x+$\frac{a}{x}$$\mathrm{\ge}$2$\sqrt{4x\cdot\frac{a}{x}}$=4$\sqrt{a}$,当且仅当4x=$\frac{a}{x}$时等号成立,此时a=4x${}^{2}$,由已知x=3时函数取得最小值,

$\mathrm{\therefore}$a=4$\mathrm{\times}$9=36.

答案:36



知识:基本不等式

难度:1

题目:若$log_{\sqrt{2}}x$+$log_{\sqrt{2}}y$=4,则x+y的最小值是\_\_\_\_\_\_\_\_.

解析:由题意知x$\mathrm{>}$0,y$\mathrm{>}$0,$log_{\sqrt{2}}{xy}$=4,得xy=4,

$\mathrm{\therefore}$x+y$\mathrm{\ge}$2$\sqrt{xy}$=4(当且仅当x=y时,等号成立).

答案:4



知识:基本不等式

难度:1

题目:y=$\frac{3+x+x^2}{x+1}$(x$\mathrm{>}$0)的最小值是\_\_\_\_\_\_\_\_.

解析:$\mathrm{\because}$x$\mathrm{>}$0,

$\mathrm{\therefore}$y=$\frac{3+x+x^2}{x+1}$=$\frac{3}{x+1}$+x+1-1$\mathrm{\ge}$2$\sqrt{3}$-1.

当且仅当x+1=$\sqrt{3}$时,等号成立.

答案:2$\sqrt{3}$-1



知识:基本不等式

难度:1

题目:已知a,b是正数,求证:

(1) $\sqrt{\frac{a^2+b^2}{2}}$$\mathrm{\ge}$$\frac{a+b}{2}$; (2)$\sqrt{ab}$$\mathrm{\ge}$$\frac{2}{\frac{1}{a}+\frac{1}{b}}$.

证明:(1)左边= $\sqrt{\frac{a^2+b^2+a^2+b^2}{4}}$

$\mathrm{\ge}$ $\sqrt{\frac{a^2+b^2+2ab}{4}}$=$\sqrt{\frac{(a+b)^2}{4}}$=$\frac{a+b}{2}$=右边,

原不等式成立.

(2)右边=$\frac{2}{\frac{1}{a}+\frac{1}{b}}$$\mathrm{\le}$$\frac{2}{2\sqrt{\frac{1}{ab}}}$=$\sqrt{ab}$=左边,

原不等式成立.



知识:基本不等式

难度:1

题目:设x$\mathrm{>}$0,y$\mathrm{>}$0且x+y=4,要使不等式$\frac{1}{x}$+$\frac{4}{y}$$\mathrm{\ge}$m恒成立,求实数m 的取值范围.

解析:

解:由x$\mathrm{>}$0,y$\mathrm{>}$0且x+y=4,得$\frac{x+y}{4}$=1,

$\mathrm{\therefore}$$\frac{1}{x}$+$\frac{4}{y}$=$\frac{x+y}{4}\cdot(\frac{1}{x}+\frac{4}{y})$

=$\frac{1}{4}(1+\frac{y}{x}+\frac{4x}{y}+4)$

=$\frac{1}{4}(5+\frac{y}{x}+\frac{4x}{y})$

$\mathrm{\ge}$$\frac{1}{4}(5+2\sqrt{\frac{y}{x}\cdot\frac{4x}{y}})$=$\frac{9}{4}$.

当且仅当$\frac{y}{x}$=$\frac{4x}{y}$时,等号成立.

即y=2x($\mathrm{\because}$x$\mathrm{>}$0,y$\mathrm{>}$0,$\mathrm{\therefore}$y=-2x舍去).

此时,结合x+y=4,解得x=$\frac{4}{3}$,y=$\frac{8}{3}$.

$\mathrm{\therefore}$$\frac{1}{x}$+$\frac{4}{y}$的最小值为$\frac{9}{4}$,$\mathrm{\therefore}$m$\mathrm{\le}$$\frac{9}{4}$,

$\mathrm{\therefore}$m的取值范围为$(-\infty,\frac{9}{4}]$.



知识:基本不等式

难度:2

题目:如图,建立平面直角坐标系xOy,x轴在地平面上,y轴垂直于地平面,单位长度为1千米,某炮位于坐标原点.已知炮弹发射后的轨迹在方程y=kx-$\frac{1}{20}$(1+k${}^{2}$)x${}^{2}$(k>0)表示的曲线上,其中k与发射方向有关.炮的射程是指炮弹落地点的横坐标.

(1)求炮的最大射程.

(2)设在第一象限有一飞行物(忽略其大小),其飞行高度为3.2千米,试问它的横坐标a不超过多少时,炮弹可以击中它?请说明理由.

\includegraphics*[width=1.89in, height=0.78in, keepaspectratio=false]{image1}

解析:

解:(1)令y=0,得kx-$\frac{1}{20}$(1+k${}^{2}$)x${}^{2}$=0.

由实际意义和题设条件知x>0,k>0,

故x=$\frac{20k}{1+k^2}$=$\frac{20}{k+\frac{1}{k}}$$\mathrm{\le}$$\frac{20}{2}$=10,

当且仅当k=1时取等号.

所以炮的最大射程为10千米.

(2)因为a>0,所以炮弹可击中飞行物,

即存在k>0,使3.2=ka-$\frac{1}{20}$(1+k${}^{2}$)a${}^{2}$成立,

即关于k的方程a${}^{2}$k${}^{2}$-20ak+a${}^{2}$+64=0有正根

$\mathrm{\Rightarrow }$$\Delta$=(-20a)${}^{2}$-4a${}^{2}$(a${}^{2}$+64)$\mathrm{\ge}$0

$\mathrm{\Rightarrow }$a$\mathrm{\le}$6.

所以当a不超过6(千米)时,可击中飞行物.




知识:几何平均不等式

难度:1

题目:已知x为正数,下列各题求得的最值正确的是(  )

A.y=x${}^{2}$+2x+$\frac{4}{x^3}$$\mathrm{\ge}$3$\sqrt[3]{x^2\cdot2x\cdot\frac{4}{x^3}}$=6,$\mathrm{\therefore}$y${}_{min}$=6.

B.y=2+x+$\frac{1}{x}$$\mathrm{\ge}$$3\sqrt[3]{2\cdot x\cdot\frac{1}{x}}$=$3\sqrt[3]{2}$,$\mathrm{\therefore}$y${}_{min}$=$3\sqrt[3]{2}$.

C.y=2+x+$\frac{1}{4}$$\mathrm{\ge}$4,$\mathrm{\therefore}$y${}_{min}$=4.

D.y=x(1-x)(1-2x)$\mathrm{\le}$$\frac{1}{3}[\frac{3x+(1-x)+(1-2x)}{3}]^3$=$\frac{8}{81}$,

$\mathrm{\therefore}$y${}_{max}$=$\frac{8}{81}$.

解析:A、B、D在使用不等式a+b+c$\mathrm{\ge}$3$\sqrt[3]{abc}$(a,b,c$\mathrm{\in}$R${}_{\textrm{+}}$)和abc$\mathrm{\le}$$(\frac{a+b+c}{3})^3$(a,b,c$\mathrm{\in}$R${}_{\textrm{+}}$)都不能保证等号成立,最值取不到.

C中,$\mathrm{\because}$x$\mathrm{>}$0,$\mathrm{\therefore}$y=2+x+$\frac{1}{x}$=2+(x+$\frac{1}{x}$)$\mathrm{\ge}$2+2=4,

当且仅当x=$\frac{1}{x}$,即x=1时,等号成立.

答案:C



知识:几何平均不等式

难度:1

题目:已知a,b,c为正数,则$\frac{a}{b}$+$\frac{b}{c}$+$\frac{c}{a}$有(  )

A.最小值3        

B.最大值3  

C.最小值2  

D.最大值2

解析:$\frac{a}{b}$+$\frac{b}{c}$+$\frac{c}{a}$$\mathrm{\ge}$3$\sqrt[3]{\frac{a}{b}x\frac{b}{c}x\frac{c}{a}}$=3,

当且仅当$\frac{a}{b}$=$\frac{b}{c}$=$\frac{c}{a}$,即a=b=c时,等号成立.

答案:A



知识:几何平均不等式

难度:1

题目:若log${}_{x}$y=-2,则x+y的最小值是(  )

A. $\frac{3\sqrt[3]{2}}{2}$ B.$\frac{8\sqrt{3}}{3}$  C.$\frac{3\sqrt{3}}{2}$   D.$\frac{2\sqrt{2}}{3}$ 

解析:由log${}_{x}$y=-2,得y=$\frac{1}{x^2}$.而x+y=x+$\frac{1}{x^2}$=$\frac{x}{2}$+$\frac{x}{2}$+$\frac{1}{x^2}$$\mathrm{\ge}$3$\sqrt[3]{\frac{x}{2}\cdot\frac{x}{2}\cdot\frac{1}{x^2}}$=$3\sqrt[3]{\frac{1}{4}}$=$\frac{3\sqrt[3]{2}}{2}$,当且仅当$\frac{x}{2}$=$\frac{1}{x^2}$,即x=$\sqrt[3]{2}$时,等号成立.

答案:A



知识:几何平均不等式

难度:1

题目:已知圆柱的轴截面周长为6,体积为V,则下列不等式总成立的是(  )

A.V$\mathrm{\ge}$$\pi$    B.V$\mathrm{\le}$$\pi$  

C.V$\mathrm{\ge}$$\frac{1}{8}\pi$    D.V$\mathrm{\le}$$\frac{1}{8}\pi$

解析:设圆柱底面半径为r,则圆柱的高h=$\frac{6-4r}{2}$,所以圆柱的体积为V=$\pi$r${}^{2}$·h=$\pi$r${}^{2}$·$\frac{6-4r}{2}$=$\pi$r${}^{2}$(3-2r)$\mathrm{\le}$$\pi$$(\frac{r+r+3-2r}{3})^3$=$\pi$.

当且仅当r=3-2r,即r=1时,等号成立.

答案:B



知识:几何平均不等式

难度:1

题目:若a$\mathrm{>}$2,b$\mathrm{>}$3,则a+b+$\frac{1}{(a-2)(b-3)}$的最小值为\_\_\_\_\_\_\_\_.

解析:$\mathrm{\because}$a$\mathrm{>}$2,b$\mathrm{>}$3,$\mathrm{\therefore}$a-2$\mathrm{>}$0,b-3$\mathrm{>}$0,

则a+b+$\frac{1}{(a-2)(b-3)}$

=(a-2)+(b-3)+$\frac{1}{(a-2)(b-3)}$+5

$\mathrm{\ge}$3$\sqrt[3]{(a-2)x(b-3)x\frac{1}{(a-2)(b-3)}}$+5=8.

当且仅当a-2=b-3=$\frac{1}{(a-2)(b-3)}$,即a=3,b=4时,等号成立.

答案:8



知识:几何平均不等式

难度:1

题目:设0$\mathrm{<}$x$\mathrm{<}$1,则x(1-x)${}^{2}$的最大值为 \_\_\_\_\_\_\_\_.

解析:$\mathrm{\because}$0$\mathrm{<}$x$\mathrm{<}$1,$\mathrm{\therefore}$1-x$\mathrm{>}$0.

故x(1-x)${}^{2}$=$\mathrm{\times}$2x(1-x)(1-x)$\mathrm{\le}$$\frac{1}{2}[\frac{2x+(1-x)+(1+x)}{3}]^3$

=$\frac{1}{2}\mathrm{\times}\frac{8}{27}$=$\frac{4}{27}$(当且仅当x=$\frac{1}{3}$时,等号成立).

答案:$\frac{4}{27}$



知识:几何平均不等式

难度:1

题目:已知关于x的不等式2x+$\frac{1}{(x-a)^2}$$\mathrm{\ge}$7在x$\mathrm{\in}$(a,+$\mathrm{\infty}$)上恒成立,则实数a的最小值为\_\_\_\_\_\_\_\_.

解析:2x+$\frac{1}{(x-a)^2}$=(x-a)+(x-a)+$\frac{1}{(x-a)^2}$+2a.

$\mathrm{\because}$x-a$\mathrm{>}$0,

$\mathrm{\therefore}$2x+$\frac{1}{(x-a)^2}$$\mathrm{\ge}$3$\sqrt[3]{(x-a)(x-a)\frac{1}{(x-a)^2}}$+2a=3+2a,当且仅当x-a=$\frac{1}{(x-a)^2}$即x=a+1时,等号成立.

$\mathrm{\therefore}$2x+$\frac{1}{(x-a)^2}$的最小值为3+2a.

由题意可得3+2a$\mathrm{\ge}$7,得a$\mathrm{\ge}$2.

答案:2



知识:几何平均不等式

难度:1

题目:设a,b,c$\mathrm{\in}$R${}_{\textrm{+}}$,求证:

(a+b+c)$(\frac{1}{a+b}+\frac{1}{b+c}+\frac{1}{a+c})$$\mathrm{\ge}\frac{9}{2}$.

解析:

证明:$\mathrm{\because}$a,b,c$\mathrm{\in}$R${}_{\textrm{+}}$,

$\mathrm{\therefore}$2(a+b+c)=(a+b)+(b+c)+(c+a)$\mathrm{\ge}$3$\sqrt[3]{(a+b)(b+c)(c+a)}$$\mathrm{>}$0.

$\frac{1}{a+b}$+$\frac{1}{b+c}$+$\frac{1}{a+c}$$\mathrm{\ge}$3$\sqrt[3]{\frac{1}{a+b}\cdot\frac{1}{b+c}\cdot\frac{1}{a+c}}$$\mathrm{>}$0,

$\mathrm{\therefore}$(a+b+c)$\frac{1}{a+b}+frac{1}{b+c}+\frac{1}{a+c}$$\mathrm{\ge}\frac{9}{2}$.

当且仅当a=b=c时,等号成立.



知识:几何平均不等式

难度:1

题目:已知正数a,b,c满足abc=1,求(a+2)(b+2)·(c+2)的最小值.

解析:

解:因为(a+2)(b+2)(c+2)=(a+1+1)(b+1+1)(c+1+1)

$\mathrm{\ge}$3·$\sqrt[3]{a}$·3·$\sqrt[3]{b}$·3·$\sqrt[3]{c}$=27·$\sqrt[3]{abc}$=27,

当且仅当a=b=c=1时,等号成立.

所以(a+2)(b+2)(c+2)的最小值为27.



知识:几何平均不等式

难度:2

题目:已知a,b,c均为正数,证明:a${}^{2}$+b${}^{2}$+c${}^{2}$+$(\frac{1}{a}+\frac{2}{b}+\frac{1}{c})^2$$\mathrm{\ge}$6$\sqrt{3}$,并确定a,b,c为何值时,等号成立.

解析:

证明:法一:因为a,b,c均为正数,由平均值不等式,得

a${}^{2}$+b${}^{2}$+c${}^{2}$$\mathrm{\ge}$3(abc)$\frac{2}{3}$,①

$\frac{1}{a}$+$\frac{1}{b}$+$\frac{1}{c}$$\mathrm{\ge}$3(abc)-$\frac{1}{3}$,

所以$(\frac{1}{a}$+$\frac{1}{b}$+$\frac{1}{c})^2$$\mathrm{\ge}$9(abc)-$\frac{2}{3}$.②

故a${}^{2}$+b${}^{2}$+c${}^{2}$+$(\frac{1}{a}$+$\frac{1}{b}$+$\frac{1}{c})^2$$\mathrm{\ge}$3(abc)+9(abc)-$\frac{2}{3}$.

又3(abc)$\frac{2}{3}$+9(abc)-$\frac{2}{3}$$\mathrm{\ge}$2$\sqrt{27}$=6$\sqrt{3}$,③

所以原不等式成立.

当且仅当a=b=c时,①式和②式等号成立.

当且仅当3(abc)$\frac{2}{3}$=9(abc)-$\frac{2}{3}$时,③式等号成立.

即当且仅当a=b=c=3$\frac{1}{4}$时,原式等号成立.

法二:因为a,b,c均为正数,由基本不等式,得

a${}^{2}$+b${}^{2}$$\mathrm{\ge}$2ab,b${}^{2}$+c${}^{2}$$\mathrm{\ge}$2bc,c${}^{2}$+a${}^{2}$$\mathrm{\ge}$2ac,

所以a${}^{2}$+b${}^{2}$+c${}^{2}$$\mathrm{\ge}$ab+bc+ac,①

同理$\frac{1}{a^2}$+$\frac{1}{b^2}$+$\frac{1}{c^2}$$\mathrm{\ge}$$\frac{1}{ab}$+$\frac{1}{bc}$+$\frac{1}{ac}$,②

故a${}^{2}$+b${}^{2}$+c${}^{2}$+$(\frac{1}{a}$+$\frac{1}{b}$+$\frac{1}{c})^2$$\mathrm{\ge}$ab+bc+ac+$\frac{3}{ab}$+$\frac{3}{bc}$+$\frac{3}{ac}$$\mathrm{\ge}$6$\sqrt{3}$,③

所以原不等式成立.

当且仅当a=b=c时,①式和②式等号成立;当且仅当a=b=c,(ab)${}^{2}$=(bc)${}^{2}$=(ac)${}^{2}$=3时,③式等号成立,即当且仅当a=b=c=3$\frac{1}{4}$时,原式等号成立.





知识:绝对值三角不等式

难度:1

题目:对于{\textbar}a{\textbar}-{\textbar}b{\textbar}$\mathrm{\le}${\textbar}a+b{\textbar}$\mathrm{\le}${\textbar}a{\textbar}+{\textbar}b{\textbar},下列结论正确的是(  )

A.当a,b异号时,左边等号成立

B.当a,b同号时,右边等号成立

C.当a+b=0时,两边等号均成立

D.当a+b$\mathrm{>}$0时,右边等号成立;当a+b$\mathrm{<}$0时,左边等号成立

解析:当a,b异号且{\textbar}a{\textbar}$\mathrm{>}${\textbar}b{\textbar}时左边等号才成立,A不正确,显然B正确;当a+b=0时,右边等号不成立,C不正确,D显然不正确.

答案:B



知识:绝对值三角不等式

难度:1

题目:不等式$\frac{|a+b|}{|a|+|b|}$$\mathrm{<}$1成立的充要条件是(  )

A.a,b都不为零

B.ab$\mathrm{<}$0

C.ab为非负数

D.a,b中至少有一个不为零

解析:原不等式即为{\textbar}a+b{\textbar}$\mathrm{<}${\textbar}a{\textbar}+{\textbar}b{\textbar}$\mathrm{\Leftrightarrow }$a${}^{2}$+b${}^{2}$+2ab$\mathrm{<}$a${}^{2}$+b${}^{2}$+2{\textbar}ab{\textbar}$\mathrm{\Leftrightarrow }$ab$\mathrm{<}$0.

答案:B



知识:绝对值三角不等式

难度:1

题目:已知a,b,c$\mathrm{\in}$R,且a$\mathrm{>}$b$\mathrm{>}$c,则有(  )

A.{\textbar}a{\textbar}$\mathrm{>}${\textbar}b{\textbar}$\mathrm{>}${\textbar}c{\textbar}     B.{\textbar}ab{\textbar}$\mathrm{>}${\textbar}bc{\textbar}

C.{\textbar}a+b{\textbar}$\mathrm{>}${\textbar}b+c{\textbar}  D.{\textbar}a-c{\textbar}$\mathrm{>}${\textbar}a-b{\textbar}

解析:$\mathrm{\because}$a,b,c$\mathrm{\in}$R,且a$\mathrm{>}$b$\mathrm{>}$c,令a=2,b=1,c=-6.

$\mathrm{\therefore}${\textbar}a{\textbar}=2,{\textbar}b{\textbar}=1,{\textbar}c{\textbar}=6,{\textbar}b{\textbar}$\mathrm{<}${\textbar}a{\textbar}$\mathrm{<}${\textbar}c{\textbar},故排除A.

又{\textbar}ab{\textbar}=2,{\textbar}bc{\textbar}=6,{\textbar}ab{\textbar}$\mathrm{<}${\textbar}bc{\textbar},故排除B.

又{\textbar}a+b{\textbar}=3,{\textbar}b+c{\textbar}=5,{\textbar}a+b{\textbar}$\mathrm{<}${\textbar}b+c{\textbar},排除C.

而{\textbar}a-c{\textbar}={\textbar}2-(-6){\textbar}=8,{\textbar}a-b{\textbar}=1,$\mathrm{\therefore}${\textbar}a-c{\textbar}$\mathrm{>}${\textbar}a-b{\textbar}.

答案:D



知识:绝对值三角不等式

难度:1

题目:设{\textbar}a{\textbar}$\mathrm{<}$1,{\textbar}b{\textbar}$\mathrm{<}$1,则{\textbar}a+b{\textbar}+{\textbar}a-b{\textbar}与2的大小关系是(  )

A.{\textbar}a+b{\textbar}+{\textbar}a-b{\textbar}$\mathrm{>}$2

B.{\textbar}a+b{\textbar}+{\textbar}a-b{\textbar}$\mathrm{<}$2

C.{\textbar}a+b{\textbar}+{\textbar}a-b{\textbar}=2

D.不可能比较大小

解析:当(a+b)(a-b)$\mathrm{\ge}$0时,{\textbar}a+b{\textbar}+{\textbar}a-b{\textbar}={\textbar}(a+b)+(a-b){\textbar}=2{\textbar}a{\textbar}$\mathrm{<}$2.

当(a+b)(a-b)$\mathrm{<}$0时,{\textbar}a+b{\textbar}+{\textbar}a-b{\textbar}={\textbar}(a+b)-(a-b){\textbar}=2{\textbar}b{\textbar}$\mathrm{<}$2.

答案:B



知识:绝对值三角不等式

难度:1

题目:不等式{\textbar}x-1{\textbar}-{\textbar}x-2{\textbar}$\mathrm{<}$a恒成立,则a的取值范围为\_\_\_\_\_\_\_\_.

解析:若使不等式{\textbar}x-1{\textbar}-{\textbar}x-2{\textbar}$\mathrm{<}$a恒成立,只需a$\mathrm{>}$({\textbar}x-1{\textbar}-{\textbar}x-2{\textbar})${}_{max}$.

因为{\textbar}x-1{\textbar}-{\textbar}x-2{\textbar}$\mathrm{\le}${\textbar}x-1-(x-2){\textbar}=1,

故a$\mathrm{>}$1.故a的取值范围为(1,+$\mathrm{\infty}$).

答案:(1,+$\mathrm{\infty}$)



知识:绝对值三角不等式

难度:1

题目:设a,b$\mathrm{\in}$R,{\textbar}a-b{\textbar}>2,则关于实数x的不等式{\textbar}x-a{\textbar}+{\textbar}x-b{\textbar}>2的解集是\_\_\_\_\_\_\_\_.

解析:$\mathrm{\because}${\textbar}x-a{\textbar}+{\textbar}x-b{\textbar}={\textbar}a-x{\textbar}+{\textbar}x-b{\textbar}$\mathrm{\ge}${\textbar}(a-x)+(x-b){\textbar}={\textbar}a-b{\textbar}>2,

$\mathrm{\therefore}${\textbar}x-a{\textbar}+{\textbar}x-b{\textbar}>2对x$\mathrm{\in}$R恒成立,故解集为(-$\mathrm{\infty}$,+$\mathrm{\infty}$).

答案:(-$\mathrm{\infty}$,+$\mathrm{\infty}$)



知识:绝对值三角不等式

难度:1

题目:下列四个不等式:

①log${}_{x}$10+lg x$\mathrm{\ge}$2(x$\mathrm{>}$1);

②{\textbar}a-b{\textbar}$\mathrm{<}${\textbar}a{\textbar}+{\textbar}b{\textbar};

③$|\frac{b}{a}+\frac{a}{b}|$$\mathrm{\ge}$2(ab$\mathrm{\neq}$0);

④{\textbar}x-1{\textbar}+{\textbar}x-2{\textbar}$\mathrm{\ge}$1.其中恒成立的是\_\_\_\_\_\_(把你认为正确的序号都填上).

解析:log${}_{x}$10+lg x=$\frac{1}{lgx}$+lg x$\mathrm{\ge}$2,①正确;ab$\mathrm{\le}$0时,{\textbar}a-b{\textbar}={\textbar}a{\textbar}+{\textbar}b{\textbar},②不正确;

$\mathrm{\because}$ab$\mathrm{\neq}$0时,$\frac{b}{a}$与$\frac{a}{b}$同号,

$\mathrm{\therefore}$$|\frac{b}{a}+\frac{a}{b}|$=$|\frac{b}{a}|$+$|\frac{a}{b}|$$\mathrm{\ge}$2,③正确;

由{\textbar}x-1{\textbar}+{\textbar}x-2{\textbar}的几何意义知{\textbar}x-1{\textbar}+{\textbar}x-2{\textbar}$\mathrm{\ge}$1恒成立,④正确.

综上可知①③④正确.

答案:①③④



知识:绝对值三角不等式

难度:1

题目:已知x,y$\mathrm{\in}$R,且{\textbar}x+y{\textbar}$\mathrm{\le}\frac{1}{6分母}$,{\textbar}x-y{\textbar}$\mathrm{\le}\frac{1}{4}$,求证:{\textbar}x+5y{\textbar}$\mathrm{\le}$1.

解析:

证明:{\textbar}x+5y{\textbar}={\textbar}3(x+y)-2(x-y){\textbar}.

由绝对值不等式的性质,得

{\textbar}x+5y{\textbar}={\textbar}3(x+y)-2(x-y){\textbar}$\mathrm{\le}${\textbar}3(x+y){\textbar}+{\textbar}2(x-y){\textbar}

=3{\textbar}x+y{\textbar}+2{\textbar}x-y{\textbar}$\mathrm{\le}$3$\mathrm{\times}\frac{1}{6}$+2$\mathrm{\times}\frac{1}{4}$=1,即{\textbar}x+5y{\textbar}$\mathrm{\le}$1.



知识:绝对值三角不等式

难度:1

题目:设f(x)=x${}^{2}$-x+b,{\textbar}x-a{\textbar}$\mathrm{<}$1,求证:{\textbar}f(x)-f(a){\textbar}$\mathrm{<}$2({\textbar}a{\textbar}+1).

解析:

证明:$\mathrm{\because}$f(x)-f(a)=x${}^{2}$-x-a${}^{2}$+a=(x-a)(x+a-1),

{\textbar}f(x)-f(a){\textbar}={\textbar}(x-a)(x+a-1){\textbar}

={\textbar}x-a{\textbar}{\textbar}x+a-1{\textbar}$\mathrm{<}${\textbar}x+a-1{\textbar}

={\textbar}(x-a)+2a-1{\textbar}$\mathrm{\le}${\textbar}x-a{\textbar}+{\textbar}2a-1{\textbar}

$\mathrm{\le}${\textbar}x-a{\textbar}+2{\textbar}a{\textbar}+1$\mathrm{<}$2{\textbar}a{\textbar}+2=2({\textbar}a{\textbar}+1),

$\mathrm{\therefore}${\textbar}f(x)-f(a){\textbar}$\mathrm{<}$2({\textbar}a{\textbar}+1).



知识:绝对值三角不等式

难度:2

题目:设函数y={\textbar}x-4{\textbar}+{\textbar}x-3{\textbar}.求:

(1)y的最小值;

(2)使y$\mathrm{<}$a有解的a的取值范围;

(3)使y$\mathrm{\ge}$a恒成立的a的最大值.

解析:

解:(1)y={\textbar}x-4{\textbar}+{\textbar}x-3{\textbar}={\textbar}x-4{\textbar}+{\textbar}3-x{\textbar}

$\mathrm{\ge}${\textbar}(x-4)+(3-x){\textbar}=1,

$\mathrm{\therefore}$y${}_{min}$=1.

(2)由(1)知y$\mathrm{\ge}$1,要使y$\mathrm{<}$a有解,$\mathrm{\therefore}$a$\mathrm{>}$1,即a的取值范围为(1,+$\mathrm{\infty}$).

(3)要使y$\mathrm{\ge}$a恒成立,只要y的最小值1$\mathrm{\ge}$a即可,

$\mathrm{\therefore}$a${}_{max}$=1.



知识:绝对值三角不等式

难度:1

题目:不等式{\textbar}x+1{\textbar}$\mathrm{>}$3的解集是(  )

A.$\mathrm{\{}$x{\textbar}x$\mathrm{<}$-4或x$\mathrm{>}$2$\mathrm{\}}$  B.$\mathrm{\{}$x{\textbar}-4$\mathrm{<}$x$\mathrm{<}$2$\mathrm{\}}$

C.$\mathrm{\{}$x{\textbar}x$\mathrm{<}$-4或x$\mathrm{\ge}$2$\mathrm{\}}$  D.$\mathrm{\{}$x{\textbar}-4$\mathrm{\le}$x$\mathrm{<}$2$\mathrm{\}}$

解析:{\textbar}x+1{\textbar}$\mathrm{>}$3,则x+1$\mathrm{>}$3或x+1$\mathrm{<}$-3,因此x$\mathrm{<}$-4或x$\mathrm{>}$2.

答案:A



知识:绝对值三角不等式

难度:1

题目:满足不等式{\textbar}x+1{\textbar}+{\textbar}x+2{\textbar}$\mathrm{<}$5的所有实数解的集合是(  )

A.(-3,2)   B.(-1,3)   C.(-4,1)   D.(-$\frac{3}{2}$,$\frac{7}{2}$)

解析:{\textbar}x+1{\textbar}+{\textbar}x+2{\textbar}表示数轴上一点到-2,-1两点的距离和,根据-2,-1之间的距离为1,可得到-2,-1距离和为5的点是-4,1.因此{\textbar}x+1{\textbar}+{\textbar}x+2{\textbar}$\mathrm{<}$5解集是(-4,1).

答案:C



知识:绝对值三角不等式

难度:1

题目:不等式1$\mathrm{\le}${\textbar}2x-1{\textbar}$\mathrm{<}$2的解集为(  )

A.$(-\frac{1}{2},0)\mathrm{\cup}[1,\frac{3}{2}]$ 

B.$(-\frac{1}{2},0]\mathrm{\cup}[1,\frac{3}{2}]$

C.$(-\frac{1}{2},0]\mathrm{\cup}(1,\frac{3}{2}]$ 

D.$(-\frac{1}{2},0]\mathrm{\cup}[1,\frac{3}{2})$

解析:由1$\mathrm{\le}${\textbar}2x-1{\textbar}$\mathrm{<}$2,得1$\mathrm{\le}$2x-1$\mathrm{<}$2或-2$\mathrm{<}$2x-1$\mathrm{\le}$-1,因此-$\frac{1}{2}$$\mathrm{<}$x$\mathrm{\le}$0或1$\mathrm{\le}$x$\mathrm{<}\frac{3}{2}$.

答案:D



知识:绝对值三角不等式

难度:1

题目:若关于x的不等式{\textbar}x-1{\textbar}+{\textbar}x+m{\textbar}>3的解集为R,则实数m的取值范围是(  )

A.(-$\mathrm{\infty}$,-4)$\mathrm{\cup}$(2,+$\mathrm{\infty}$)  B.(-$\mathrm{\infty}$,-4)$\mathrm{\cup}$(1,+$\mathrm{\infty}$)

C.(-4,2)  D.(-4,1)

解析:由题意知,不等式{\textbar}x-1{\textbar}+{\textbar}x+m{\textbar}>3恒成立,即函数f(x)={\textbar}x-1{\textbar}+{\textbar}x+m{\textbar}的最小值大于3,根据绝对值不等式的性质可得{\textbar}x-1{\textbar}+{\textbar}x+m{\textbar}$\mathrm{\ge}${\textbar}(x-1)-(x+m){\textbar}={\textbar}m+1{\textbar},故只要满足{\textbar}m+1{\textbar}>3即可,所以m+1>3或m+1<-3,解得m>2或m<-4,故实数m的取值范围是(-$\mathrm{\infty}$,-4)$\mathrm{\cup}$(2,+$\mathrm{\infty}$).

答案:A



知识:绝对值三角不等式

难度:1

题目:不等式{\textbar}x+2{\textbar}$\mathrm{\ge}${\textbar}x{\textbar}的解集是\_\_\_\_\_\_\_\_.

解析:$\mathrm{\because}$不等式两边是非负实数,$\mathrm{\therefore}$不等式两边可以平方,两边平方,得(x+2)${}^{2}$$\mathrm{\ge}$x${}^{2}$,

$\mathrm{\therefore}$x${}^{2}$+4x+4$\mathrm{\ge}$x${}^{2}$,即x$\mathrm{\ge}$-1,

$\mathrm{\therefore}$原不等式的解集为$\mathrm{\{}$x{\textbar}x$\mathrm{\ge}$-1$\mathrm{\}}$.

答案:$\mathrm{\{}$x{\textbar}x$\mathrm{\ge}$-1$\mathrm{\}}$



知识:绝对值三角不等式

难度:1

题目:不等式{\textbar}2x-1{\textbar}-x$\mathrm{<}$1的解集是\_\_\_\_\_\_\_\_\_\_.

解析:原不等式等价于{\textbar}2x-1{\textbar}$\mathrm{<}$x+1$\mathrm{\Leftrightarrow }$-x-1$\mathrm{<}$2x-1$\mathrm{<}$x+1$\mathrm{\Leftrightarrow }$$\left\{\begin{array}{r}
3x>0\\
x<2
\end{array} \right.$$\mathrm{\Leftrightarrow }$0$\mathrm{<}$x$\mathrm{<}$2.

答案:$\mathrm{\{}$x{\textbar}0$\mathrm{<}$x$\mathrm{<}$2$\mathrm{\}}$



知识:绝对值三角不等式

难度:1

题目:已知函数f(x)={\textbar}x+1{\textbar}+{\textbar}x-2{\textbar}-{\textbar}a${}^{2}$-2a{\textbar},若函数f(x)的图象恒在x轴上方,则实数a的取值范围为\_\_\_\_\_\_\_\_.

解析:因为{\textbar}x+1{\textbar}+{\textbar}x-2{\textbar}$\mathrm{\ge}${\textbar}x+1-(x-2){\textbar}=3,

所以f(x)的最小值为3-{\textbar}a${}^{2}$-2a{\textbar}.

由题意,得{\textbar}a${}^{2}$-2a{\textbar}<3,解得-1$\mathrm{<}$a$\mathrm{<}$3.

答案:(-1,3)



知识:绝对值三角不等式

难度:1

题目:解不等式:{\textbar}x${}^{2}$-2x+3{\textbar}$\mathrm{<}${\textbar}3x-1{\textbar}.

解析:

解:原不等式$\mathrm{\Leftrightarrow }$(x${}^{2}$-2x+3)${}^{2}$$\mathrm{<}$(3x-1)${}^{2}$

$\mathrm{\Leftrightarrow }$(x${}^{2}$+x+2)(x${}^{2}$-5x+4)$\mathrm{<}$0

$\mathrm{\Leftrightarrow }$x${}^{2}$-5x+4$\mathrm{<}$0(因为x${}^{2}$+x+2恒大于0)$\mathrm{\Leftrightarrow }$1$\mathrm{<}$x$\mathrm{<}$4.

所以原不等式的解集是$\mathrm{\{}$x{\textbar}1$\mathrm{<}$x$\mathrm{<}$4$\mathrm{\}}$.



知识:绝对值三角不等式

难度:1

题目:解关于x的不等式{\textbar}2x-1{\textbar}$\mathrm{<}$2m-1(m$\mathrm{\in}$R).

解析:

解:若2m-1$\mathrm{<}$0,即m$\mathrm{\le}\frac{1}{2}$,则{\textbar}2x-1{\textbar}$\mathrm{<}$2m-1恒不成立,此时,原不等式无解;若2m-1$\mathrm{>}$0,即m$\mathrm{>}\frac{1}{2}$,

则-(2m-1)$\mathrm{<}$2x-1$\mathrm{<}$2m-1,

所以1-m$\mathrm{<}$x$\mathrm{<}$m.

综上所述:

当m$\mathrm{\le}\frac{1}{2}$时,原不等式的解集为$\mathrm{\emptyset}$;

当m$\mathrm{>}\frac{1}{2}$时,原不等式的解集为$\mathrm{\{}$x{\textbar}1-m$\mathrm{<}$x$\mathrm{<}$m$\mathrm{\}}$.



知识:绝对值三角不等式

难度:2

题目:已知函数f(x)={\textbar}2x-1{\textbar}+{\textbar}2x+a{\textbar},g(x)=x+3.

(1)当a=-2时,求不等式f(x)<g(x)的解集;

(2)设a>-1,且当x$\mathrm{\in}[-\frac{a}{2},\frac{1}{2})$时,f(x)$\mathrm{\le}$g(x),求a的取值范围.

解析:

解:(1)当a=-2时,不等式f(x)<g(x)化为{\textbar}2x-1{\textbar}+{\textbar}2x-2{\textbar}-x-3<0.

设函数y={\textbar}2x-1{\textbar}+{\textbar}2x-2{\textbar}-x-3,则

y=$\left\{\begin{array}{r}
-5x,x<\frac{1}{2}\\
-x-2,\frac{1}{2}\le x\le1\\
3x-6,x>1
\end{array} \right.$

其图象如图所示.

\includegraphics*[width=1.07in, height=1.16in, keepaspectratio=false]{image2}

从图象可知,当且仅当x$\mathrm{\in}$(0,2)时,y<0,

所以原不等式的解集是$\mathrm{\{}$x{\textbar}0<x<2$\mathrm{\}}$.

(2)当x$\mathrm{\in}[-\frac{a}{2},\frac{1}{2})$时,f(x)=1+a.

不等式f(x)$\mathrm{\le}$g(x)化为1+a$\mathrm{\le}$x+3,

所以x$\mathrm{\ge}$a-2对x$\mathrm{\in}[-\frac{a}{2},\frac{1}{2})$都成立.

故-$\frac{a}{2}\mathrm{\ge}$a-2,即a$\mathrm{\le}\frac{4}{3}$.

从而a的取值范围是$(-1,\frac{4}{3}]$.



 知识:比较法解不等式

 难度:1

 题目:下列命题:

①当b$\mathrm{>}$0时,a$\mathrm{>}$b$\mathrm{\Leftrightarrow }$$\frac{a}{b}\mathrm{>}$1;

②当b$\mathrm{>}$0时,a$\mathrm{<}$b$\mathrm{\Leftrightarrow }$$\frac{a}{b}\mathrm{<}$1;

③当a$\mathrm{>}$0,b$\mathrm{>}$0时,$\frac{a}{b}\mathrm{>}$1$\mathrm{\Leftrightarrow }$a$\mathrm{>}$b;

④当ab$\mathrm{>}$0时,$\frac{a}{b}\mathrm{>}$1$\mathrm{\Leftrightarrow }$a$\mathrm{>}$b.

其中是真命题的有(  )

A.①②③  B.①②④  

C.④          D.①②③④

解析:只有④不正确.如a=-2,b=-1时,$\frac{a}{b}$=2$\mathrm{>}$1,但a$\mathrm{<}$b.

答案:A



 知识:比较法解不等式

 难度:1

 题目:若x,y$\mathrm{\in}$R,记w=x${}^{2}$+3xy,u=4xy-y${}^{2}$,则(  )

A.w$\mathrm{>}$u  B.w$\mathrm{<}$u  C.w$\mathrm{\ge}$u  D.无法确定

 解析:$\mathrm{\because}$w-u=x${}^{2}$-xy+y${}^{2}$=$(x-\frac{y}{2})^2$+$\frac{3y^2}{4}$$\mathrm{\ge}$0,

$\mathrm{\therefore}$w$\mathrm{\ge}$u.

 答案:C



 知识:比较法解不等式

 难度:1

 题目:a,b都是正数,P=$\frac{\sqrt{a}+\sqrt{b}}{\sqrt{2}}$,Q=$\sqrt{a+b}$,则P,Q的大小关系是(  )

A.P$\mathrm{>}$Q  

B.P$\mathrm{<}$Q  

C.P$\mathrm{\ge}$Q  

D.P$\mathrm{\le}$Q

 解析:$\mathrm{\because}$a,b都是正数,$\mathrm{\therefore}$P$\mathrm{>}$0,Q$\mathrm{>}$0.

$\mathrm{\therefore}$P${}^{2}$-Q${}^{2}$=$(\frac{\sqrt{a}+\sqrt{b}}{\sqrt{2}})^2$-$(\sqrt{a+b})^2$

=$\frac{-(\sqrt{a}-\sqrt{b})^2}{2}$$\mathrm{\le}$0(当且仅当a=b时取等号),

$\mathrm{\therefore}$P${}^{2}$-Q${}^{2}$$\mathrm{\le}$0,$\mathrm{\therefore}$P$\mathrm{\le}$Q.

 答案:D



 知识:比较法解不等式

 难度:1

 题目:在$\mathrm{\vartriangle}$ABC中,sin Asin C$\mathrm{<}$cos Acos C,则$\mathrm{\vartriangle}$ABC(  )

A.一定是锐角三角形  B.一定是直角三角形

C.一定是钝角三角形  D.不确定

 解析: 由sin Asin C$\mathrm{<}$cos Acos C,得

cos Acos C-sin Asin C$\mathrm{>}$0,即cos(A+C)$\mathrm{>}$0,

所以A+C是锐角,从而B$\mathrm{>}\frac{\pi}{2}$,

故$\mathrm{\vartriangle}$ABC一定是钝角三角形.

 答案:D



 知识:比较法解不等式

 难度:1

 题目:若0$\mathrm{<}$x$\mathrm{<}$1,则$\frac{1}{x}$与$\frac{1}{x^2}$的大小关系是\_\_\_\_\_\_\_\_.

 解析:$\frac{1}{x}-\frac{1}{x^2}$=$\frac{x-1}{x^2}$.

因为0$\mathrm{<}$x$\mathrm{<}$1,所以$\frac{1}{x}$-$\frac{1}{x^2}\mathrm{<}$0,

所以$\frac{1}{x}\mathrm{<}\frac{1}{x^2}$.

 答案:$\frac{1}{x}\mathrm{<}\frac{1}{x^2}$



 知识:比较法解不等式

 难度:1

 题目:$\frac{2a}{1+a^2}$与1的大小关系为\_\_\_\_\_\_\_\_.

 解析:$\frac{2a}{1+a^2}$-1=$\frac{2a-1-a^2}{1+a^2}$=-$\frac{(1-a)^2}{1+a^2}$$\mathrm{\le}$0.

 答案:$\frac{2a}{1+a^2}\mathrm{\le}$1



 知识:比较法解不等式

 难度:1

 题目:设a$\mathrm{>}$b$\mathrm{>}$0,x=$\sqrt{a+b}$-$\sqrt{a}$,y=$\sqrt{a}$-$\sqrt{a-b}$,则x,y的大小关系是\_\_\_\_\_\_\_\_.

 解析:$\mathrm{\because}$$\frac{x}{y}$=$\frac{\sqrt{a+b}-\sqrt{a}}{\sqrt{a}-\sqrt{a-b}}$=$\frac{\sqrt{a}+\sqrt{a-b}}{\sqrt{a}+\sqrt{a+b}}$$\mathrm{<}$$\frac{\sqrt{a}+\sqrt{a+b}}{\sqrt{a}+\sqrt{a+b}}$=1,且x$\mathrm{>}$0,y$\mathrm{>}$0,

$\mathrm{\therefore}$x$\mathrm{<}$y.

 答案:x$\mathrm{<}$y



 知识:比较法解不等式

 难度:1

 题目:已知x,y$\mathrm{\in}$R, 求证:sin x+sin y$\mathrm{\le}$1+sin xsin y.

 解析:

 证明:$\mathrm{\because}$sin x+sin y-1-sin xsin y

=sin x(1-sin y)-(1-sin y)=(1-sin y)(sin x-1).

$\mathrm{\because}$-1$\mathrm{\le}$sin x$\mathrm{\le}$1,-1$\mathrm{\le}$sin y$\mathrm{\le}$1,

$\mathrm{\therefore}$1-sin y$\mathrm{\ge}$0,sin x-1$\mathrm{\le}$0,$\mathrm{\therefore}$(1-sin y)(sin x-1)$\mathrm{\le}$0,

即sin x+sin y$\mathrm{\le}$1+sin xsin y.



 知识:比较法解不等式

 难度:1

 题目:已知a$\mathrm{<}$b$\mathrm{<}$c,求证:a${}^{2}$b+b${}^{2}$c+c${}^{2}$a$\mathrm{<}$ab${}^{2}$+bc${}^{2}$+ca${}^{2}$.

 解析:

 证明:因为a$\mathrm{<}$b$\mathrm{<}$c,所以a-b$\mathrm{<}$0,b-c$\mathrm{<}$0,a-c$\mathrm{<}$0,

所以(a${}^{2}$b+b${}^{2}$c+c${}^{2}$a)-(ab${}^{2}$+bc${}^{2}$+ca${}^{2}$)

=(a${}^{2}$b-ca${}^{2}$)+(b${}^{2}$c-bc${}^{2}$)+(ac${}^{2}$-ab${}^{2}$)

=a${}^{2}$(b-c)+bc(b-c)-a(b-c)(b+c)

=(b-c)=(b-c)(a-b)(a-c)$\mathrm{<}$0,

所以a${}^{2}$b+b${}^{2}$c+c${}^{2}$a$\mathrm{<}$ab${}^{2}$+bc${}^{2}$+ca${}^{2}$.



 知识:比较法解不等式

 难度:2

 题目:已知a$\mathrm{>}$2,求证:log${}_{a}$(a-1)$\mathrm{<}$log${}_{(a}$${}_{\textrm{+}}$${}_{1)}$a.

 解析:

 证明:$\mathrm{\because}$a$\mathrm{>}$2,

$\mathrm{\therefore}$a-1$\mathrm{>}$1,

$\mathrm{\therefore}$log${}_{a}$(a-1)$\mathrm{>}$0,log${}_{(a}$${}_{\textrm{+}}$${}_{1)}$a$\mathrm{>}$0.

由于$\frac{log_a(a-1)}{log_{a+1}a}$=log${}_{a}$(a-1)·log${}_{a}$(a+1)$\mathrm{<}$$[\frac{log_a(a-1)+log_a(a+1)}{2}]^2$=$[\frac{log_a(a^2-1)}{2}]^2$.

$\mathrm{\because}$a$\mathrm{>}$2,$\mathrm{\therefore}$0$\mathrm{<}$log${}_{a}$(a${}^{2}$-1)$\mathrm{<}$log${}_{a}$a${}^{2}$=2.

$\mathrm{\therefore}$$[\frac{log_a(a^2-1)}{2}]^2$$\mathrm{<}$$[\frac{log_aa^2}{2}]^2$=1,即$\frac{log_a(a-1)}{log_{a+1}a}\mathrm{<}$1.

$\mathrm{\because}$log${}_{(a}$${}_{\textrm{+}}$${}_{1)}$a$\mathrm{>}$0,$\mathrm{\therefore}$log${}_{a}$(a-1)$\mathrm{<}$log${}_{(a}$${}_{\textrm{+}}$${}_{1)}$a.





 

 知识:综合法解不等式,分析法解不等式

 难度:1

 题目:设a,b$\mathrm{\in}$R${}_{\textrm{+}}$,A=$\sqrt{a}$+$\sqrt{b}$,B=$\sqrt{a+b}$,则A,B的大小关系是(  )

A.A$\mathrm{\ge}$B  B.A$\mathrm{\le}$B  C.A>B  D.A<B

 解析: A${}^{2}$=($\sqrt{a}$+$\sqrt{b}$)${}^{2}$=a+2$\sqrt{ab}$+b,B${}^{2}$=a+b,所以A${}^{2}$$\mathrm{>}$B${}^{2}$.又A>0,B>0,$\mathrm{\therefore}$A>B.

 答案:C



 知识:综合法解不等式,分析法解不等式

 难度:1

 题目:a,b$\mathrm{\in}$R${}_{\textrm{+}}$,那么下列不等式中不正确的是(  )

A.$\frac{a}{b}$+$\frac{b}{a}$$\mathrm{\ge}$2  B.$\frac{b^2}{a}$+$\frac{a^2}{b}$$\mathrm{\ge}$a+b

C.$\frac{b}{a^2}$+$\frac{a}{b^2}$$\mathrm{\le}$$\frac{a+b}{ab}$  D.$\frac{1}{a^2}$+$\frac{1}{b^2}$$\mathrm{\ge}$$\frac{2}{ab}$

 解析: A项满足基本不等式;B项可等价变形为(a-b)${}^{2}$(a+b)$\mathrm{\ge}$0,正确;B选项中不等式的两端同除以ab,不等式方向不变,所以C选项不正确;D选项是A选项中不等式的两端同除以ab得到的,正确.

 答案:C



 知识:综合法解不等式,分析法解不等式

 难度:1

 题目:设a=$\sqrt{2}$,b=$\sqrt{7}$-$\sqrt{3}$,c=$\sqrt{6}$-$\sqrt{2}$,那么a,b,c的大小关系是(  )

A.a$\mathrm{>}$b$\mathrm{>}$c  B.a$\mathrm{>}$c$\mathrm{>}$b  C.b$\mathrm{>}$a$\mathrm{>}$c  D.b$\mathrm{>}$c$\mathrm{>}$a

 解析: 由已知,可得出a=$\frac{4}{2\sqrt{2}}$,b=$\frac{4}{\sqrt{7}+\sqrt{3}}$,c=$\frac{4}{\sqrt{6}+\sqrt{2}}$,

$\mathrm{\because}$$\sqrt{7}$+$\sqrt{3}$$\mathrm{>}$$\sqrt{6}$+$\sqrt{2}$$\mathrm{>}$2$\sqrt{2}$,$\mathrm{\therefore}$b$\mathrm{<}$c$\mathrm{<}$a.

 答案:B



 知识:综合法解不等式,分析法解不等式

 难度:1

 题目:设$\frac{1}{3}\mathrm{<}$$(\frac{1}{3})^b$$\mathrm{<}$$(\frac{1}{3})^a$$\mathrm{<}$1,则(  )

A.a${}^{a}$$\mathrm{<}$a${}^{b}$$\mathrm{<}$b${}^{a}$  B.a${}^{a}$$\mathrm{<}$b${}^{a}$$\mathrm{<}$a${}^{b}$  C.a${}^{b}$$\mathrm{<}$a${}^{a}$$\mathrm{<}$b${}^{a}$  D.a${}^{b}$$\mathrm{<}$b${}^{a}$$\mathrm{<}$a${}^{a}$

 解析:$\mathrm{\because}$$\frac{1}{3}\mathrm{<}$$(\frac{1}{3})^b$$\mathrm{<}$$(\frac{1}{3})^a$$\mathrm{<}$1,

$\mathrm{\therefore}$0$\mathrm{<}$a$\mathrm{<}$b$\mathrm{<}$1,$\mathrm{\therefore}$$\frac{a^a}{a^b}$=a${}^{a}$${}^{\textrm{-}}$${}^{b}$$\mathrm{>}$1,

$\mathrm{\therefore}$a${}^{b}$$\mathrm{<}$a${}^{a}$,$\frac{a^a}{a^b}$=$(\frac{a}{b})^a$.$\mathrm{\because}$0$\mathrm{<}$$\frac{a}{b}$$\mathrm{<}$1,a$\mathrm{>}$0,

$\mathrm{\therefore}$$(\frac{a}{b})^a$$\mathrm{<}$1,$\mathrm{\therefore}$a${}^{a}$$\mathrm{<}$b${}^{a}$,$\mathrm{\therefore}$a${}^{b}$$\mathrm{<}$a${}^{a}$$\mathrm{<}$b${}^{a}$.

 答案:B

 

 知识:综合法解不等式,分析法解不等式

 难度:1

 题目:若$\frac{1}{a}$<$\frac{1}{a}$<0,则下列不等式:

①a+b<ab;②{\textbar}a{\textbar}>{\textbar}b{\textbar};③a<b;④$\frac{b}{a}$+$\frac{a}{b}$>2,

其中正确的有\_\_\_\_\_\_\_\_(填序号).

 解析:$\mathrm{\because}$$\frac{1}{a}$<$\frac{1}{b}$<0,$\mathrm{\therefore}$b<a<0.

$\mathrm{\therefore}$$\left\{\begin{array}{r}
a+b<0\\
ab>0\\
|b|>|a|
\end{array} \right.$故①正确,②③错误.

$\mathrm{\because}$a,b同号且a$\mathrm{\neq}$b,$\mathrm{\therefore}$$\frac{b}{a}$,$\frac{a}{b}$均为正,

$\mathrm{\therefore}$$\frac{b}{a}$+$\frac{a}{b}$>$2\sqrt{\frac{b}{a}\cdot\frac{a}{b}}$ =2,故④正确.

 答案:①④



 知识:综合法解不等式,分析法解不等式

 难度:1

 题目:已知a$\mathrm{>}$0,b$\mathrm{>}$0,若P是a,b的等差中项,Q是a,b的正的等比中项,$\frac{1}{R}$是$\frac{1}{a}$,$\frac{1}{b}$的等差中项,则P,Q,R按从大到小的顺序排列为\_\_\_\_\_\_\_\_.

 解析:$\mathrm{\because}$P=$\frac{a+b}{2}$,Q=$\sqrt{ab}$,$\frac{2}{R}$=$\frac{1}{a}$+$\frac{1}{b}$,

$\mathrm{\therefore}$R=$\frac{2ab}{a+b}$$\mathrm{\le}$Q=$\sqrt{ab}$$\mathrm{\le}$P=$\frac{a+b}{2}$,

当且仅当a=b时,等号成立.

 答案:P$\mathrm{\ge}$Q$\mathrm{\ge}$R

 

 知识:综合法解不等式,分析法解不等式

 难度:1

 题目:设a$\mathrm{>}$b$\mathrm{>}$c,且$\frac{1}{a-b}$+$\frac{1}{b-c}$$\mathrm{\ge}$$\frac{m}{a-c}$恒成立,则m的取值范围是\_\_\_\_\_\_\_\_.

 解析
$\mathrm{\because}$a$\mathrm{>}$b$\mathrm{>}$c,$\mathrm{\therefore}$a-b$\mathrm{>}$0,b-c$\mathrm{>}$0,a-c$\mathrm{>}$0.

$\therefore$, 原不等式等价于$\frac{a-c}{a-b}+\frac{a-c}{b-c}\ge m$

$\frac{a-c}{a-b}+\frac{a-c}{b-c}=\frac{(a-b)+(b-c)}{a-b}+\frac{(a-b)+(b-c)}{b-c}=2+\frac{b-c}{a-b}+\frac{a-b}{b-c}\ge 2+2\sqrt{\frac{b-c}{a-b}\cdot\frac{a-b}{b-c}}=4$

当且仅当$\frac{b-c}{a-b}=\frac{a-b}{b-c}$, 即$2b=a+c$或$a=c$(舍去)时, 等式成立,

$\therefore m\le 4.$

 答案:(-$\mathrm{\infty}$,4]



 知识:综合法解不等式,分析法解不等式

 难度:1

 题目:已知a,b,c均为正实数,且b${}^{2}$=ac.

 求证:a${}^{4}$+b${}^{4}$+c${}^{4}$$\mathrm{>}$(a${}^{2}$-b${}^{2}$+c${}^{2}$)${}^{2}$.

 解析:

 证明:要证a${}^{4}$+b${}^{4}$+c${}^{4}$$\mathrm{>}$(a${}^{2}$-b${}^{2}$+c${}^{2}$)${}^{2}$成立,

只需证a${}^{4}$+b${}^{4}$+c${}^{4}$$\mathrm{>}$a${}^{4}$+b${}^{4}$+c${}^{4}$-2a${}^{2}$b${}^{2}$+2a${}^{2}$c${}^{2}$-2b${}^{2}$c${}^{2}$,

即证a${}^{2}$b${}^{2}$+b${}^{2}$c${}^{2}$-a${}^{2}$c${}^{2}$$\mathrm{>}$0.$\mathrm{\because}$b${}^{2}$=ac,

故只需证(a${}^{2}$+c${}^{2}$)ac-a${}^{2}$c${}^{2}$$\mathrm{>}$0.

$\mathrm{\because}$a$\mathrm{>}$0,c$\mathrm{>}$0,故只需证a${}^{2}$+c${}^{2}$-ac$\mathrm{>}$0.

又$\mathrm{\because}$a${}^{2}$+c${}^{2}$$\mathrm{\ge}$2ac>ac,$\mathrm{\therefore}$a${}^{2}$+c${}^{2}$-ac$\mathrm{>}$0显然成立,

$\mathrm{\therefore}$原不等式成立.



 知识:综合法解不等式,分析法解不等式

 难度:1

 题目:已知a$\mathrm{>}$0,b$\mathrm{>}$0,c$\mathrm{>}$0,且a,b,c不全相等,

 求证:$\frac{bc}{a}$+$\frac{ac}{b}$+$\frac{ab}{c}$$\mathrm{>}$a+b+c.

 解析:

 证明:因为a,b,c$\mathrm{\in}$(0,+$\mathrm{\infty}$),所以$\frac{bc}{a}$+$\frac{ac}{b}$$\mathrm{\ge}$$2\sqrt{\frac{bc}{a}\cdot\frac{ac}{b}}$=2c.

同理$\frac{ac}{b}$+$\frac{ab}{c}$$\mathrm{\ge}$2a,$\frac{ab}{c}$+$\frac{bc}{a}$$\mathrm{\ge}$2b.因为a,b,c不全相等,

所以上述三个不等式中至少有一个等号不成立,三式相加,得$2(\frac{bc}{a}+\frac{ac}{b}+\frac{ab}{c})$$\mathrm{>}$2(a+b+c),即$\frac{bc}{a}$+$\frac{ac}{b}$+$\frac{ab}{c}$$\mathrm{>}$a+b+c.



 知识:综合法解不等式,分析法解不等式

 难度:2

 题目:设实数x,y满足y+x${}^{2}$=0,0$\mathrm{<}$a$\mathrm{<}$1,

 求证:log${}_{a}$(a${}^{x}$+a${}^{y}$)$\mathrm{<}$$\frac{1}{8}$+log${}_{a}$2.

 解析:

 证明:因为a${}^{x}$$\mathrm{>}$0,a${}^{y}$$\mathrm{>}$0,所以a${}^{x}$+a${}^{y}$$\mathrm{\ge}$$2\sqrt{a^{x+y}}$=$2\sqrt{ax-x^2}$ .

因为x-x${}^{2}$=x(1-x)$\mathrm{\le}$$[\frac{x+(1-x)}{2}]^2$=$\frac{1}{4}$,

又因为0$\mathrm{<}$a$\mathrm{<}$1,所以ax-x${}^{2}$$\mathrm{\ge}$a$\frac{1}{4}$,当x=$\frac{1}{2}$时,等式成立.

但当x=$\frac{1}{2}$时,a${}^{x}$$\mathrm{\neq}$a-x${}^{2}$,所以 $\sqrt{ax-x^2}$$\mathrm{>}$a$\frac{1}{8}$,

所以a${}^{x}$+a${}^{y}$>2a$\frac{1}{8}$.又因为0$\mathrm{<}$a$\mathrm{<}$1,

所以log${}_{a}$(a${}^{x}$+a${}^{y}$)$\mathrm{<}$log${}_{a}$2a$\frac{1}{8}$,即log${}_{a}$(a${}^{x}$+a${}^{y}$)$\mathrm{<}$log${}_{a}$2+$\frac{1}{8}$.




 

 知识:放缩法解不等式,反证法解不等式

 难度:1

 题目:设a,b,c$\mathrm{\in}$R${}_{\textrm{+}}$,P=a+b-c,Q=b+c-a,R=c+a-b,则``PQR>0''是``P,Q,R同时大于零''的(  )

A.充分而不必要条件  B.必要而不充分条件

C.充要条件  D.既不充分也不必要条件

 解析:必要性是显然成立的;当PQR>0时,若P,Q,R不同时大于零,则其中两个为负,一个为正,不妨设P>0,Q<0,R<0,则Q+R=2c<0,这与c>0矛盾,即充分性也成立.

 答案:C





 知识:放缩法解不等式,反证法解不等式

 难度:1

 题目:若{\textbar}a-c{\textbar}$\mathrm{<}$h,{\textbar}b-c{\textbar}$\mathrm{<}$h,则下列不等式一定成立的是(  )

A.{\textbar}a-b{\textbar}$\mathrm{<}$2h  

B.{\textbar}a-b{\textbar}$\mathrm{>}$2h

C.{\textbar}a-b{\textbar}$\mathrm{<}$h  

D.{\textbar}a-b{\textbar}$\mathrm{>}$h

 解析:{\textbar}a-b{\textbar}={\textbar}(a-c)-(b-c){\textbar}$\mathrm{\le}${\textbar}a-c{\textbar}+{\textbar}b-c{\textbar}$\mathrm{<}$2h.

 答案:A



 知识:放缩法解不等式,反证法解不等式

 难度:1

 题目:设x,y都是正实数,且xy-(x+y)=1,则(  )

A.x+y$\mathrm{\ge}$2($\sqrt{2}$+1)  

B.xy$\mathrm{\le}$$\sqrt{2}$+1

C.x+y$\mathrm{\le}$($\sqrt{2}$+1)${}^{2}$ 

D.xy$\mathrm{\ge}$2($\sqrt{2}$+1)

 解析:由已知(x+y)+1=xy$\mathrm{\le}$$(\frac{x+y}{2})^2$,

$\mathrm{\therefore}$(x+y)${}^{2}$-4(x+y)-4$\mathrm{\ge}$0.

$\mathrm{\because}$x,y都是正实数,

$\mathrm{\therefore}$x$\mathrm{>}$0,y$\mathrm{>}$0,$\mathrm{\therefore}$x+y$\mathrm{\ge}$$2\sqrt{2}$+2=2($\sqrt{2}$+1).

 答案:A

 

 知识:放缩法解不等式,反证法解不等式

 难度:1

 题目:对``a,b,c是不全相等的正数'',给出下列判断:

①(a-b)${}^{2}$+(b-c)${}^{2}$+(c-a)${}^{2}$$\mathrm{\neq}$0;

②a$\mathrm{>}$b与a$\mathrm{<}$b及a$\mathrm{\neq}$c中至少有一个成立;

③a$\mathrm{\neq}$c,b$\mathrm{\neq}$c,a$\mathrm{\neq}$b不能同时成立.

其中判断正确的个数为(  )

A.0  B.1  

C.2  D.3

 解析:若(a-b)${}^{2}$+(b-c)${}^{2}$+(c-a)${}^{2}$=0,则a=b=c,与已知矛盾,故①对;当a$\mathrm{>}$b与a$\mathrm{<}$b及a$\mathrm{\neq}$c都不成立时,有a=b=c,不符合题意,故②对;③显然不正确.

 答案:C

 

 知识:放缩法解不等式,反证法解不等式

 难度:1

 题目:若要证明``a,b至少有一个为正数'',用反证法证明时作的反设应为\_\_\_\_\_\_\_\_.

 答案:a,b中没有任何一个为正数(或a$\mathrm{\le}$0且b$\mathrm{\le}$0)

 

 知识:放缩法解不等式,反证法解不等式

 难度:1

 题目:lg9·lg11与1的大小关系是\_\_\_\_\_\_\_\_.

 解析:$\mathrm{\because}$lg 9>0,lg 11>0,

$\mathrm{\therefore}$$\sqrt{lg9\cdot lg11}$<$\frac{lg9+lg11}{2}$=$\frac{lg99}{2}$<$\frac{lg100}{2}$=1,

$\mathrm{\therefore}$lg 9·lg 11<1.

 答案:lg 9·lg 11<1

 

 知识:放缩法解不等式,反证法解不等式

 难度:1

 题目:设x>0,y>0,A=$\frac{x+y}{1+x+y}$,B=$\frac{x}{1+x}$+$\frac{y}{1+y}$,则A,B的大小关系是\_\_\_\_\_\_\_\_.

 解析:A=$\frac{x}{1+x+y}$+$\frac{y}{1+x+y}$<$\frac{x}{1+x}$+$\frac{y}{1+y}$=B.

 答案:A<B

 

 知识:放缩法解不等式,反证法解不等式

 难度:1

 题目:实数a,b,c,d满足a+b=c+d=1,且ac+bd$\mathrm{>}$1.求证:a,b,c,d中至少有一个是负数.

 解析:

 证明:假设a,b,c,d都是非负数.

由a+b=c+d=1知a,b,c,d$\mathrm{\in}$.

从而ac$\mathrm{\le}$$\sqrt{ac}$$\mathrm{\le}$$\frac{a+c}{2}$,bd$\mathrm{\le}$$\sqrt{bd}$$\mathrm{\le}$$\frac{b+d}{2}$,

$\mathrm{\therefore}$ac+bd$\mathrm{\le}$$\frac{a+c+b+d}{2}$=1,

即ac+bd$\mathrm{\le}$1,与已知ac+bd$\mathrm{>}$1矛盾,

$\mathrm{\therefore}$a,b,c,d中至少有一个是负数.

 

 知识:放缩法解不等式,反证法解不等式

 难度:1

 题目:已知a${}_{n}$=$\sqrt{1\times2}$+$\sqrt{2\times3}$+$\sqrt{3\times4}$+{$\dots$}+$\sqrt{n(n+1)}$(n$\mathrm{\in}$N${}^{*}$).

 求证:$\frac{n(n+1)}{2}$$\mathrm{<}$a${}_{n}$$\mathrm{<}$$\frac{n(n+2)}{2}$.

 解析:

 证明:$\mathrm{\because}$$\sqrt{n(n+1)}$=$\sqrt{n^2+n}$,

$\sqrt{n(n+1)}\mathrm{\therefore}$$\mathrm{>}$n,

$\mathrm{\therefore}$a${}_{n}$=$\sqrt{1\times2}$+$\sqrt{2\times3}$+$\sqrt{3\times4}$+{$\dots$}+$\sqrt{n(n+1)}$$\mathrm{>}$1+2+3+{$\dots$}+n=$\frac{n(n+1)}{2}$.

$\mathrm{\because}$$\sqrt{n(n+1)}$$\mathrm{<}$$\frac{n+(n+1)}{2}$,

$\mathrm{\therefore}$a${}_{n}$$\mathrm{<}$$\frac{1+2}{2}$+$\frac{2+3}{2}$+$\frac{3+4}{2}$+{$\dots$}+$\frac{n+(n+1)}{2}$

=$\frac{n}{2}$+(1+2+3+{$\dots$}+n)=$\frac{n(n+2)}{2}$.

综上得$\frac{n(n+1)}{2}\mathrm{<}$a${}_{n}$$\mathrm{<}\frac{n(n+2)}{2}$.



 知识:放缩法解不等式,反证法解不等式

 难度:2

 题目:已知f(x)=ax${}^{2}$+bx+c,若a+c=0,f(x)在上的最大值为2,最小值为-$\frac{5}{2}$.

 求证:a$\mathrm{\neq}$0且$|\frac{b}{a}|\mathrm{<}$2.

 解析:

 证明:假设a=0或$|\frac{b}{a}|\mathrm{\ge}$2.

①当a=0时,由a+c=0,得f(x)=bx,显然b$\mathrm{\neq}$0.

由题意得f(x)=bx在上是单调函数,

所以f(x)的最大值为{\textbar}b{\textbar},最小值为-{\textbar}b{\textbar}.

由已知条件得{\textbar}b{\textbar}+(-{\textbar}b{\textbar})=2-$\frac{5}{2}$=-$\frac{1}{2}$,

这与{\textbar}b{\textbar}+(-{\textbar}b{\textbar})=0相矛盾,所以a$\mathrm{\neq}$0.

②当$|\frac{b}{a}|\mathrm{\ge}$2时,由二次函数的对称轴为x=-$\frac{b}{2a}$,

知f(x)在上是单调函数,故其最值在区间的端点处取得 .

所以$\left\{\begin{array}{r}
f(1)=a+b+c=2\\
f(-1)=a-b+c=-\frac{5}{2}
\end{array} \right.$或$\left\{\begin{array}{r}
f(1)=a-b+c=-\frac{5}{2}\\
f(-1)=a+b+c=2
\end{array} \right.$

又a+c=0,则此时b无解,所以$|\frac{b}{a}|\mathrm{<}$2.

由①②,得a$\mathrm{\neq}$0且$|\frac{b}{a}|\mathrm{<}$2.




 知识:二维柯西不等式

 难度:1

 题目:已知x,y$\mathrm{\in}$R${}_{\textrm{+}}$,且xy=1,则$(1+\frac{1}{x})(1+\frac{1}{y})$的最小值为(  )

A.4           B.2  

C.1           D.$\frac{1}{4}$

 解析: 
$(1+\frac{1}{x})(1+\frac{1}{y})$
=$[1^2+(\frac{1}{\sqrt{x}})^2]$·$[1^2+(\frac{1}{\sqrt{y}})^2]$

$\mathrm{\ge}$$(1\times1+\frac{1}{\sqrt{x}\times\frac{1}{\sqrt{y}}})^2$=$(1+\frac{1}{\sqrt{xy}})^2$=2${}^{2}$=4.

 答案:A

 

 知识:二维柯西不等式

 难度:1

 题目:若a,b$\mathrm{\in}$R,且a${}^{2}$+b${}^{2}$=10,则a-b的取值范围是(  )

A.[$-2\sqrt{5}$,$2\sqrt{5}$]   B.[-$2\sqrt{10}$,$2\sqrt{10}$]

C.[-$\sqrt{10}$,$\sqrt{10}$]   D.($-\sqrt{5}$,$\sqrt{5}$)

 解析: (a${}^{2}$+b${}^{2}$)$\mathrm{\ge}$(a-b)${}^{2}$,

$\mathrm{\because}$a${}^{2}$+b${}^{2}$=10,$\mathrm{\therefore}$(a-b)${}^{2}$$\mathrm{\le}$20.

$\mathrm{\therefore}$-2$\sqrt{5}$$\mathrm{\le}$a-b$\mathrm{\le}$2$\sqrt{5}$.

 答案:A

 

 知识:二维柯西不等式

 难度:1

 题目:已知x+y=1,那么2x${}^{2}$+3y${}^{2}$的最小值是(  )

A.  $\frac{5}{6}$  B.$\frac{6}{5}$    C.$\frac{25}{36}$    D.$\frac{36}{25}$

 解析: (2x${}^{2}$+3y${}^{2}$)$\mathrm{\ge}$($\sqrt{6}$x+$\sqrt{6}$y)${}^{2}$=[$\sqrt{6}$(x+y)]${}^{2}$=6,

当且仅当x=$\frac{3}{5}$,y=$\frac{2}{5}$时,等号成立,即2x${}^{2}$+3y${}^{2}$$\mathrm{\ge\frac{6}{5}}$.

 答案:B

 

 知识:二维柯西不等式

 难度:1

 题目:函数y=$\sqrt{x-5}$+$2\sqrt{6-x}$的最大值是(  )

A. $\sqrt{3}$  B.  $\sqrt{5}$

C.3               D.5

 解析: 根据柯西不等式,知y=1$\mathrm{\times}$$\sqrt{x-5}$+2$\mathrm{\times}$$\sqrt{6-x}$$\mathrm{\le}$$\sqrt{1^2+2^2}$

 $\mathrm{\times}$$\sqrt{(\sqrt{x-5})^2+(\sqrt{6-x})^2}$=$\sqrt{5}$,当且仅当x=$\frac{26}{5}$时,等号成立.

 答案:B

 

 知识:二维柯西不等式

 难度:1

 题目:设xy$\mathrm{>}$0,则$(x^2+\frac{4}{y^2})$·$(y^2+\frac{1}{x^2})$的最小值为\_\_\_\_\_\_\_\_.

 解析:原式=$[x^2+(\frac{2}{y})^2][(\frac{1}{x})^2+y^2]$

$\mathrm{\ge}$$(x\cdot\frac{1}{x}+\frac{2}{y}\cdot y)^2$=9(当且仅当xy=$\sqrt{2}$时,等号成立).

 答案:9

 

 知识:二维柯西不等式

 难度:1

 题目:

 设实数x,y满足3x${}^{2}$+2y${}^{2}$$\mathrm{\le}$6,则P=2x+y的最大值为\_\_\_\_\_\_\_\_.

 解析:由柯西不等式,得(2x+y)${}^{2}$$\mathrm{\le}$$(3x^2+2y^2)$·$[(\frac{2}{\sqrt{3}})^2+(\frac{1}{\sqrt{2}})^2]$=(3x${}^{2}$+2y${}^{2}$)·$(\frac{4}{3}+\frac{1}{2})$$\mathrm{\le}$6$\mathrm{\times}$$\frac{11}{6}$=11,当且仅当x=$\frac{4}{\sqrt{11}}$,y=$\frac{3}{\sqrt{11}}$时,等号成立,

于是2x+y$\mathrm{\le}\sqrt{11}$.

 答案:

 

 知识:二维柯西不等式

 难度:1

 题目:

 函数f(x)=$\sqrt{2-x^2}$+$\sqrt{2x^2-1}$的最大值为\_\_\_\_\_\_\_\_.

 解析:因题意得函数有意义时x满足$\frac{1}{2}\mathrm{\le}$x${}^{2}$$\mathrm{\le}$2.

由柯西不等式,得$[f(x)]^2$=$[\sqrt{2-x^2}+\sqrt{2(x^2-\frac{1}{2})}]^2$

$\mathrm{\le}$(1+2)$(2-x^2+x^2-\frac{1}{2})$=$\frac{9}{2}$,$\mathrm{\therefore}$f(x)$\mathrm{\le}\frac{3\sqrt{2}}{2}$,

当且仅当2-x${}^{2}$=,即x${}^{2}$=时,等号成立.

 答案:$\frac{3\sqrt{2}}{2}$

 

 知识:二维柯西不等式

 难度:1

 题目:已知$\theta$为锐角,a,b $\mathrm{\in}$R${}_{\textrm{+}}$.

 求证:(a+b)${}^{2}$$\mathrm{\le}$$\frac{a^2}{\cos^2\theta}$+$\frac{b^2}{\sin^2\theta}$.

 解析:

 证明:设m=$(\frac{a}{\cos\theta},\frac{b}{\sin\theta})$,n=(cos $\theta$,sin $\theta$),

则{\textbar}a+b{\textbar}=$|\frac{a}{\cos\theta}\cdot\cos\theta+\frac{b}{\sin\theta}\cdot\sin\theta|$

={\textbar}m·n{\textbar}$\mathrm{\le}${\textbar}m{\textbar}{\textbar}n{\textbar}=$\sqrt{(\frac{a}{\cos\theta})^2+(\frac{b}{\sin\theta})^2}$ ·$\sqrt{1}$

=$\sqrt{\frac{a^2}{\cos^2\theta}+\frac{b^2}{\sin^2\theta}}$,$\mathrm{\therefore}$(a+b)${}^{2}$$\mathrm{\le}$$\frac{a^2}{\cos^2\theta}$+$\frac{b^2}{\sin^2\theta}$.

 

 知识:二维柯西不等式

 难度:1

 题目:解方程:$\sqrt{4x+3}$+$2\sqrt{1-2x}$ =$\sqrt{15}$.

 解析:

 解:15=$(\sqrt{2}\cdot\sqrt{2x+\frac{3}{2}}+2\sqrt{1-2x})^2$

$\mathrm{\le}$$(\sqrt{2}^2+2^2)$·$[(\sqrt{2x+\frac{3}{2}})^2+(\sqrt{1-2x})^2]$

=$6(2x+\frac{3}{2}+1-2x)$=6$\mathrm{\times}\frac{5}{2}$=15.

其中等号成立的充要条件是$\frac{\sqrt{2x+\frac{3}{2}}}{\sqrt{2}}$=$\frac{\sqrt{1-2x}}{2}$,

解得x=-$\frac{1}{3}$.

 

 知识:二维柯西不等式

 难度:2

 题目:试求函数f(x)=3cos x+4$\sqrt{1+\sin^2 x}$的最大值,并求出相应的x的值.

 解析:

 解:设m=(3,4),

$n=(\cos x, \sqrt{1+\sin^2x})$

则f(x)=3cos x+4 $\sqrt{1+\sin^2x}$

=|m·n|$\mathrm{\le}$|m|·|n|

=$\sqrt{\cos^2 x+1+\sin^2 x}\cdot \sqrt{3^2+4^2}$=5$\sqrt{2}$,

当且仅当m$//$n时,上式取等号.

此时,$3\sqrt{1+\sin^2 x}-4\cos x=0$

解得sin x=$\frac{\sqrt{7}}{5}$,cos x=$\frac{3\sqrt{2}}{5}$.

故当sin x=$\frac{\sqrt{7}}{5}$,cos x=$\frac{3\sqrt{2}}{5}$时,

f(x)=$3\cos x+4\sqrt{1+\sin^2 x}$取得最大值5$\sqrt{2}$.

 

 知识:二维柯西不等式

 难度:2

 题目:设a=(-2,1,2),|b|=6,则a·b的最小值为(  )

A.18            B.6  

C.-18  D.12

 解析: |a·b|$\mathrm{\le}$|a||b|,

$\mathrm{\therefore}$|a·b|$\mathrm{\le}$18.

$\mathrm{\therefore}$-18$\mathrm{\le}$a·b$\mathrm{\le}$18,当a,b反向时,a,b最小,最小值-18.

 答案:C

 

 知识:二维柯西不等式

 难度:2

 题目:已知$a_1^2+a_2^2+\cdots+a_n^2=1, x_1^2+x_2^2+\cdots+x_n^2$,则a${}_{1}$x${}_{1}$+a${}_{2}$x${}_{2}$+{$\dots$}+a${}_{n}$x${}_{n}$的最大值是(  )

A.1  B.2  

C.3  D.4

 解析: (a${}_{1}$x${}_{1}$+a${}_{2}$x${}_{2}$+{$\dots$}+a${}_{n}$x${}_{n}$)${}^{2}$$\mathrm{\le}(a_1^2+a_2^2+\cdots +a_n^2)(x_1^2+x_2^2+\cdots +x_n^2)$=1$\mathrm{\times}$1=1,当且仅当$\frac{x_1}{a_1}$=$\frac{x_2}{a_2}$={$\dots$}=$\frac{x_n}{a_n}$=1时取等号,$\mathrm{\therefore}$a${}_{1}$x${}_{1}$+a${}_{2}$x${}_{2}$+{$\dots$}+a${}_{n}$x${}_{n}$的最大值是1.

 答案:A

 

 知识:二维柯西不等式

 难度:1

 题目:已知a${}^{2}$+b${}^{2}$+c${}^{2}$+d${}^{2}$=5,则ab+bc+cd+ad的最小值为(  )

A.5  B.-5  

C.25  D.-25

 解析: (ab+bc+cd+da)${}^{2}$$\mathrm{\le}$(a${}^{2}$+b${}^{2}$+c${}^{2}$+d${}^{2}$)·(b${}^{2}$+c${}^{2}$+d${}^{2}$+a${}^{2}$)=25,当且仅当a=b=c=d=$\mathrm{\pm}\frac{\sqrt{5}}{2}$时,等号成立,$\mathrm{\therefore}$ab+bc+cd+bd的最小值为-5.

 答案:B

 

 知识:二维柯西不等式

 难度:1

 题目:已知x,y,z$\mathrm{\in}$R,且x-2y-3z=4,则x${}^{2}$+y${}^{2}$+z${}^{2}$的最小值为(  )

A.$\frac{8}{7}$   B.$\frac{7}{8}$  

C.$\frac{4}{7}$   D.$\frac{7}{4}$

 解析: 由柯西不等式,得${}^{2}$$\mathrm{\le}$(x${}^{2}$+y${}^{2}$+z${}^{2}$),即(x-2y-3z)${}^{2}$$\mathrm{\le}$14(x${}^{2}$+y${}^{2}$+z${}^{2}$),

即16$\mathrm{\le}$14(x${}^{2}$+y${}^{2}$+z${}^{2}$),所以x${}^{2}$+y${}^{2}$+z${}^{2}$$\mathrm{\ge}\frac{8}{7}$.

当且仅当$x=\frac{y}{-2}=\frac{z}{-3}=\frac{2}{7}$时,等号成立,

即x${}^{2}$+y${}^{2}$+z${}^{2}$的最小值为$\frac{8}{7}$.

 答案:A



 知识:二维柯西不等式

 难度:1

 题目:已知2x+3y+z=8,则x${}^{2}$+y${}^{2}$+z${}^{2}$取得最小值时,x,y,z形成的点(x,y,z)=\_\_\_\_\_\_\_\_.

 解析:由柯西不等式,得(2${}^{2}$+3${}^{2}$+1${}^{2}$)(x${}^{2}$+y${}^{2}$+z${}^{2}$)$\mathrm{\ge}$(2x+3y+z)${}^{2}$,即x${}^{2}$+y${}^{2}$+z${}^{2}$$\mathrm{\ge}\frac{8^2}{14}=\frac{32}{7}$.

当且仅当$\frac{x}{2}=\frac{y}{3}=z$时,等号成立.又2x+3y+z=8,

解得x=$\frac{8}{7}$,y=$\frac{12}{7}$,z=$\frac{4}{7}$,所求点为$(\frac{8}{7}, \frac{12}{7}, \frac{4}{7})$.

 答案:$(\frac{8}{7}, \frac{12}{7}, \frac{4}{7})$

 

 知识:二维柯西不等式

 难度:1

 题目:已知实数x,y,z满足x+2y+z=1,则x${}^{2}$+4y${}^{2}$+z${}^{2}$的最小值为\_\_\_\_\_\_\_\_.

解析:由柯西不等式,得(x${}^{2}$+4y${}^{2}$+z${}^{2}$)(1+1+1)$\mathrm{\ge}$(x+2y+z)${}^{2}$.

$\mathrm{\because}$x+2y+z=1,$\mathrm{\therefore}$3(x${}^{2}$+4y${}^{2}$+z${}^{2}$)$\mathrm{\ge}$1,

即x${}^{2}$+4y${}^{2}$+z${}^{2}$$\mathrm{\ge}\frac{1}{3}$.当且仅当x=2y=z=$\frac{1}{3}$,

即x=$\frac{1}{3}$,y=$\frac{1}{6}$,z=$\frac{1}{3}$时,等号成立,

故x${}^{2}$+4y${}^{2}$+z${}^{2}$的最小值为$\frac{1}{3}$.

 答案:

 

 知识:二维柯西不等式

 难度:1

 题目:已知a,b,c$\mathrm{\in}$R${}_{\textrm{+}}$且a+b+c=6,则$\sqrt{2a}+\sqrt{2b+1}+\sqrt{2c+3}$的最大值为\_\_\_\_\_\_\_\_.

 解析:由柯西不等式,得$(\sqrt{2a}+\sqrt{2b+1}+\sqrt{2c+3})^2=$(1$\mathrm{\times}\sqrt{2a}$+1$\mathrm{\times}\sqrt{2b+1}$+1$\mathrm{\times}\sqrt{2c+3}$)${}^{2}$$\mathrm{\le}$(1${}^{2}$+1${}^{2}$+1${}^{2}$)(2a+2b+1+2c+3)=3(2$\mathrm{\times}$6+4)=48.

当且仅当$\sqrt{2a}=\sqrt{2b+1}=\sqrt{2c+3}$,

即2a=2b+1=2c+3时,等号成立.

又a+b+c=6,$\mathrm{\therefore}$a=$\frac{8}{3}$,b=$\frac{13}{6}$,c=$\frac{7}{6}$时,

$\sqrt{2a}+\sqrt{2b+1}+\sqrt{2c+3}$取得最大值4$\sqrt{3}$.

 答案:4$\sqrt{3}$

 

 知识:二维柯西不等式

 难度:1

 题目:在$\mathrm{\vartriangle}$ABC中,设其各边长为a,b,c,外接圆半径为R,求证:$(a^2+b^2+c^2)(\frac{1}{\sin^2A}+\frac{1}{\sin^2B}+\frac{1}{\sin^2C})\ge 36R^2$.

 解析:

 证明:$\mathrm{\because}\frac{a}{\sin A}=\frac{b}{\sin B}=\frac{c}{\sin C}=2R$,

$(a^2+b^2+c^2)(\frac{1}{\sin^2A}+\frac{1}{\sin^2B}+\frac{1}{\sin^2C})\ge (\frac{a}{\sin A}+\frac{b}{\sin B}+\frac{c}{\sin C})^2=36R^2$


 

 知识:二维柯西不等式

 难度:1

 题目:求实数x,y的值使得(y-1)${}^{2}$+(x+y-3)${}^{2}$+(2x+y-6)${}^{2}$取到最小值.

 解析:

 解:$a=y-1, b=x+y, c=2x+y-6$, 可得$a-2b+c=-1$, 

则原式$=a^2+b^2+c^2=a^2+\frac{1}{4}b^2+\frac{1}{4}b^2+\frac{1}{4}b^2+\frac{1}{4}b^2+c^2\ge \frac{1}{6}(a-\frac{1}{2}b-\frac{1}{2}b-\frac{1}{2}b-\frac{1}{2}b+c)^2=\frac{1}{6}(a-2b+c)^2=\frac{1}{6}$,

取等条件$a=-\frac{1}{2}b=c$, 即$y-1=-\frac{1}{2}(x+y-3)=2x+y-6$,

$\left\{\begin{array}{l} y-1=-\frac{1}{2}(x+y-3)\\ y-1=2x+y-6 \end{array}\right.$

解得:$\left\{\begin{array}{l} x=\frac{5}{2}\\ y=\frac{5}{6} \end{array}\right.$

$\therefore x=\frac{5}{2}, y=\frac{5}{6}$, 此时最小值为$\frac{1}{6}$.

 

 知识:二维柯西不等式

 难度:3

 题目:已知不等式|a-2|$\mathrm{\le}$x${}^{2}$+2y${}^{2}$+3z${}^{2}$对满足x+y+z=1的一切实数x,y,z都成立,求实数a的取值范围.

 解析:

 解:由柯西不等式,得$[1^2+(\frac{1}{\sqrt{2}})^2+(\frac{1}{\sqrt{3}})^2]\mathrm{\ge}$(x+y+z)${}^{2}$.

又因为x+y+z=1,所以x${}^{2}$+2y${}^{2}$+3z${}^{2}$$\mathrm{\ge}\frac{6}{11}$.

当且仅当$\frac{x}{1}=\frac{\sqrt{2}y}{\frac{1}{\sqrt{2}}}=\frac{\sqrt{3}z}{\frac{1}{\sqrt{3}}}$,即x=$\frac{6}{11}$,y=$\frac{3}{11}$,z=$\frac{2}{11}$时取等号,则|a-2|$\mathrm{\le}\frac{6}{11}$,所以实数a的取值范围为$[\frac{16}{11}, \frac{28}{11}]$.

 

 知识:排序不等式

 难度:1

 题目:有一有序数组,其顺序和为A,反序和为B,乱序和为C,则它们的大小关系为(  )

A.A$\mathrm{\ge}$B$\mathrm{\ge}$C         B.A$\mathrm{\ge}$C$\mathrm{\ge}$B

C.A$\mathrm{\le}$B$\mathrm{\le}$C  D.A$\mathrm{\le}$C$\mathrm{\le}$B

 解析:由排序不等式,顺序和$\mathrm{\ge}$乱序和$\mathrm{\ge}$反序和知:A$\mathrm{\ge}$C$\mathrm{\ge}$B.

 答案:B

 

 知识:排序不等式

 难度:1

 题目:若A=$x_1^2+x_2^2+\cdots+x_n^2$,B=x${}_{1}$x${}_{2}$+x${}_{2}$x${}_{3}$+{$\dots$}+x${}_{n}$${}_{\textrm{-}}$${}_{1}$x${}_{n}$+x${}_{n}$x${}_{1}$,其中x${}_{1}$,x${}_{2}$,{$\dots$},x${}_{n}$都是正数,则A与B的大小关系为(  )

A.A$\mathrm{>}$B  B.A$\mathrm{<}$B  C.A$\mathrm{\ge}$B  D.A$\mathrm{\le}$B

 解析:序列$\mathrm{\{}$x${}_{n}$$\mathrm{\}}$的各项都是正数,不妨设0<x${}_{1}$$\mathrm{\le}$x${}_{2}$$\mathrm{\le}${$\dots$}$\mathrm{\le}$x${}_{n}$,则x${}_{2}$,x${}_{3}$,{$\dots$},x${}_{n}$,x${}_{1}$为序列$\mathrm{\{}$x${}_{n}$$\mathrm{\}}$ 的一个排列.由排序原理,得x${}_{1}$x${}_{1}$+x${}_{2}$x${}_{2}$+{$\dots$}+x${}_{n}$x${}_{n}$$\mathrm{\ge}$x${}_{1}$x${}_{2}$+x${}_{2}$x${}_{3}$+{$\dots$}+x${}_{n}$x${}_{1}$,即$x_1^2+x_2^2+\cdots+x_n^2\ge x_1x_2+x_2x_3+\cdots+x_nx_1$.

 答案:C

 

 知识:排序不等式

 难度:1

 题目:锐角三角形中,设P=$\frac{a+b+c}{2}$,Q=acos C+bcos B+ccos A,则P,Q的关系为(  )

A.P$\mathrm{\ge}$Q  B.P=Q  C.P$\mathrm{\le}$Q  D.不能确定

 解析:不妨设A$\mathrm{\ge}$B$\mathrm{\ge}$C,则a$\mathrm{\ge}$b$\mathrm{\ge}$c,cos A$\mathrm{\le}$cos B$\mathrm{\le}$cos C,

则由排序不等式有Q=acos C+bcos B+ccos A

$\mathrm{\ge}$acos B+bcos C+ccos A

=R(2sin Acos B+2sin Bcos C+2sin Ccos A)

=R

=R(sin C+sin A+sin B)=P=$\frac{a+b+c}{2}$.

 答案:C

 

 知识:排序不等式

 难度:1

 题目:儿子过生日要老爸买价格不同的礼品1件、2件及3件,现在选择商店中单价为13元、20元和10元的礼品,至少要花\_\_\_\_\_\_\_\_元.(  )

A.76  B.20  C.84  D.96

解析:设a${}_{1}$=1(件),a${}_{2}$=2(件),a${}_{3}$=3(件),b${}_{1}$=10(元),b${}_{2}$=13(元),b${}_{3}$=20(元),则由排序原理反序和最小知至少要花a${}_{1}$b${}_{3}$+a${}_{2}$b${}_{2}$+a${}_{3}$b${}_{1}$=1$\mathrm{\times}$20+2$\mathrm{\times}$13+3$\mathrm{\times}$10=76(元).

 答案:A



 知识:排序不等式

 难度:1

 题目:已知两组数1,2,3和4,5,6,若c${}_{1}$,c${}_{2}$,c${}_{3}$是4,5,6的一个排列,则1c${}_{1}$+2c${}_{2}$+3c${}_{3}$的最大值是\_\_\_\_\_\_\_\_,最小值是\_\_\_\_\_\_\_\_.

 解析:由反序和$\mathrm{\le}$乱序和$\mathrm{\le}$顺序和知,顺序和最大,反序和最小,故最大值为32,最小值为28.

 答案:32 28

 

 知识:排序不等式

 难度:1

 题目:有4人各拿一只水桶去接水,设水龙头注满每个人的水桶分别需要5 s、4 s、3 s、7 s,每个人接完水后就离开,则他们总的等候时间最短为\_\_\_\_\_\_\_\_s.

 解析:由题意知,等候的时间最短为3$\mathrm{\times}$4+4$\mathrm{\times}$3+5$\mathrm{\times}$2+7=41.

 答案:41

 

 知识:排序不等式

 难度:1

 题目:在Rt$\mathrm{\vartriangle}$ABC中,$\mathrm{\angle}$C为直角,A,B所对的边分别为a,b,则aA+bB与$\frac{\pi}{4}$(a+b)的大小关系为\_\_\_\_\_\_\_\_.

 解析:不妨设a$\mathrm{\ge}$b$\mathrm{>}$0,则A$\mathrm{\ge}$B$\mathrm{>}$0,由排序不等式

$\left.\begin{array}{l} aA+bB\ge aB+bA\\ aA+bB=aA+bB \end{array}\right\}\mathrm{\Rightarrow }$2(aA+bB)$\mathrm{\ge}$a(A+B)+b(A+B) $\frac{\pi}{2}$=(a+b),

$\mathrm{\therefore}$aA+bB$\mathrm{\ge}\frac{\pi}{4}$(a+b).

 答案:aA+bB$\mathrm{\ge}\frac{\pi}{4}$(a+b)

 

 知识:排序不等式

 难度:1

 题目:设a,b,c都是正数,求证:a+b+c$\mathrm{\le}\frac{a^4+b^4+c^4}{abc}$.

 解析:

 证明:由题意不妨设a$\mathrm{\ge}$b$\mathrm{\ge}$c$\mathrm{>}$0.

由不等式的性质,知a${}^{2}$$\mathrm{\ge}$b${}^{2}$$\mathrm{\ge}$c${}^{2}$,ab$\mathrm{\ge}$ac$\mathrm{\ge}$bc.

根据排序原理,得a${}^{2}$bc+ab${}^{2}$c+abc${}^{2}$$\mathrm{\le}$a${}^{3}$c+b${}^{3}$a+c${}^{3}$b.①

又由不等式的性质,知a${}^{3}$$\mathrm{\ge}$b${}^{3}$$\mathrm{\ge}$c${}^{3}$,且a$\mathrm{\ge}$b$\mathrm{\ge}$c.

再根据排序不等式,得

a${}^{3}$c+b${}^{3}$a+c${}^{3}$b$\mathrm{\le}$a${}^{4}$+b${}^{4}$+c${}^{4}$.②

由①②及不等式的传递性,得

a${}^{2}$bc+ab${}^{2}$c+abc${}^{2}$$\mathrm{\le}$a${}^{4}$+b${}^{4}$+c${}^{4}$.

两边同除以abc得证原不等式成立.

 

 知识:排序不等式

 难度:1

 题目:设a,b,c为任意正数,求$\frac{a}{b+c}+\frac{b}{c+a}+\frac{c}{a+b}$的最小值.

 解析:

 解:不妨设a$\mathrm{\ge}$b$\mathrm{\ge}$c,

则a+b$\mathrm{\ge}$a+c$\mathrm{\ge}$b+c,$\frac{1}{b+c}\mathrm{\ge}\frac{1}{c+a}\mathrm{\ge}\frac{1}{a+b}$.

由排序不等式,得

$\frac{a}{b+c}+\frac{b}{c+a}+\frac{c}{a+b}\mathrm{\ge}\frac{b}{b+c}+\frac{c}{c+a}+\frac{a}{a+b}$,

$\frac{a}{b+c}+\frac{b}{c+a}+\frac{c}{a+b}\mathrm{\ge}\frac{c}{b+c}+\frac{a}{c+a}+\frac{b}{a+b}$,

以上两式相加,得2$(\frac{a}{b+c}+\frac{b}{c+a}+\frac{c}{a+b})\mathrm{\ge}$3,

$\mathrm{\therefore}\frac{a}{b+c}+\frac{b}{c+a}+\frac{c}{a+b}\mathrm{\ge}\frac{3}{2}$,

即当且仅当a=b=c时,

$\frac{a}{b+c}+\frac{b}{c+a}+\frac{c}{a+b}$的最小值为$\frac{3}{2}$.

 知识:排序不等式

 难度:2

 题目:设x,y,z为正数,求证:

x+y+z$\mathrm{\le}\frac{x^2+y^2}{2z}+\frac{y^2+z^2}{2x}+\frac{z^2+x^2}{2y}$.

 解析:

 证明:由于不等式关于x,y,z对称,

不妨设0$\mathrm{<}$x$\mathrm{\le}$y$\mathrm{\le}$z,于是x${}^{2}$$\mathrm{\le}$y${}^{2}$$\mathrm{\le}$z${}^{2}$,$\frac{1}{z}\mathrm{\le}\frac{1}{y}\mathrm{\le}\frac{1}{x}$,

由排序原理:反序和$\mathrm{\le}$乱序和,得

x${}^{2}\cdot\frac{1}{x}$+y${}^{2}\cdot\frac{1}{z}$+z${}^{2}\cdot\frac{1}{z}\mathrm{\le}$x${}^{2}\cdot\frac{1}{z}$+y${}^{2}\cdot\frac{1}{x}$+z${}^{2}\cdot\frac{1}{y}$,

x${}^{2}\cdot\frac{1}{x}$+y${}^{2}\cdot\frac{1}{z}$+z${}^{2}\cdot\frac{1}{z}\mathrm{\le}$x${}^{2}\cdot\frac{1}{y}$+y${}^{2}\cdot\frac{1}{z}$+z${}^{2}\cdot\frac{1}{x}$,

将上面两式相加,得2(x+y+z)$\mathrm{\le}\frac{x^2+y^2}{z}+\frac{y^2+z^2}{x}+\frac{z^2+x^2}{y}$,于是x+y+z$\mathrm{\le}\frac{x^2+y^2}{2z}+\frac{y^2+z^2}{2x}+\frac{z^2+x^2}{2y}$.







 知识:数学归纳法及应用

 难度:1

 题目:数学归纳法证明中,在验证了n=1时命题正确,假定n=k时命题正确,此时k的取值范围是  (  )

A.k$\mathrm{\in}$N          B.k$\mathrm{>}$1,k$\mathrm{\in}$N${}^{*}$

C.k$\mathrm{\ge}$1,k$\mathrm{\in}$N${}^{*}$  D.k$\mathrm{>}$2,k$\mathrm{\in}$N${}^{*}$

 解析:数学归纳法是证明关于正整数n的命题的一种方法,所以k是正整数;因为第一步是递推的基础,所以k大于等于1.

 答案:C

 

 知识:数学归纳法及应用

 难度:1

 题目:用数学归纳法证明1+2+3+{$\dots$}+n${}^{3}$=$\frac{n^6+n^3}{2}$,则当n=k+1时,左端应在n=k的基础上加上(  )

A.k${}^{3}$+1  

B.(k+1)${}^{3}$

C.  $\frac{(k+1)^6+(k+1)^3}{2}$

D.(k${}^{3}$+1)+(k${}^{3}$+2)+(k${}^{3}$+3)+{$\dots$}+(k+1)${}^{3}$

 解析: 当n=k时,等式左端=1+2+{$\dots$}+k${}^{3}$.

当n=k+1时,等式左端=1+2+{$\dots$}+k${}^{3}$+(k${}^{3}$+1)+(k${}^{3}$+2)+(k${}^{3}$+3)+{$\dots$}+(k+1)${}^{3}$,故选D.

 答案:D

 

 知识:数学归纳法及应用

 难度:1

 题目:设f(n)=$\frac{1}{n+1}+\frac{1}{n+2}+\frac{1}{n+3}+\cdots+\frac{1}{2n}$(n$\mathrm{\in}$N${}^{*}$),那么f(n+1)-f(n)等于(  )

A.$\frac{1}{2n+1}$   B.$\frac{1}{2n+2}$

C.$\frac{1}{2n+1}+\frac{1}{2n+2}$  D.$\frac{1}{2n+1}-\frac{1}{2n+2}$

 解析:因为f(n)=$\frac{1}{n+1}+\frac{1}{n+2}+\frac{1}{n+3}+\cdots+\frac{1}{2n}$,

所以f(n+1)=$\frac{1}{n+2}+\frac{1}{n+3}+\frac{1}{n+3}+\cdots+\frac{1}{2n}+\frac{1}{2n+1}+\frac{1}{2n+2}$

所以f(n+1)-f(n)=$\frac{1}{2n+1}+\frac{1}{2n+2}-\frac{1}{n-1}=\frac{1}{2n+1}-\frac{1}{2n+2}$.

 答案:D

 

 知识:数学归纳法及应用

 难度:1

 题目:某同学回答``用数学归纳法证明$\sqrt{n^2+n}\mathrm{<}$n+1(n$\mathrm{\in}$N${}^{*}$)''的过程如下:

 证明:(1)当n=1时,显然命题是正确的.

(2)假设n=k时,有$\sqrt{k(k+1)}\mathrm{<}$k+1,那么当n=k+1时,$\sqrt{(k+1)^2+k+1}=\sqrt{k^2+3k+2}\mathrm{<}\sqrt{k^2+4k+4}$=(k+1)+1,

所以当n=k+1时命题是正确的.

由(1)(2)可知对于n$\mathrm{\in}$N${}^{*}$,命题都是正确的.

以上证法是错误的,错误在于(  )

A.从k到k+1的推理过程没有使用归纳假设

B.归纳假设的写法不正确

C.从k到k+1的推理不严密

D.当n=1时,验证过程不具体

 解析: 证明  $\sqrt{(k+1)^2+k+1}\mathrm{<}$(k+1)+1时进行了一般意义的放大,而没有使用归纳假设$\sqrt{k(k+1)}\mathrm{<}$k+1.

 答案:A

 

 知识:数学归纳法及应用

 难度:1

 题目:数列$\mathrm{\{}$a${}_{n}$$\mathrm{\}}$中,已知a${}_{1}$=1,当n$\mathrm{\ge}$2时,a${}_{n}$-a${}_{n}$${}_{\textrm{-}}$${}_{1}$=2n-1,依次计算a${}_{2}$,a${}_{3}$,a${}_{4}$后,猜想a${}_{n}$的表达式是\_\_\_\_\_\_\_\_.

 解析:计算出a${}_{1}$=1,a${}_{2}$=4,a${}_{3}$=9,a${}_{4}$=16.可猜想a${}_{n}$=n${}^{2}$.

 答案:a${}_{n}$=n${}^{2}$

 

 知识:数学归纳法及应用

 难度:1

 题目:用数学归纳法证明``1$\mathrm{\times}$4+2$\mathrm{\times}$7+3$\mathrm{\times}$10+{$\dots$}+n(3n+1)=n(n+1)${}^{2}$,n$\mathrm{\in}$N${}^{*}$''时,若n=1,则左端应为\_\_\_\_\_\_\_\_.

 解析:n=1时,左端应为1$\mathrm{\times}$4=4.

 答案:4

 

 知识:数学归纳法及应用

 难度:1

 题目:记凸k边形的内角和为f(k),则凸k+1边形的内角和f(k+1)=f(k)+\_\_\_\_\_\_\_\_.

 解析:由凸k边形变为凸k+1边形时,增加了一个三角形图形,故f(k+1)=f(k)+$\pi$.

 答案:$\pi$

 

 知识:数学归纳法及应用

 难度:1

 题目:用数学归纳法证明:1·(n${}^{2}$-1${}^{2}$)+2·(n${}^{2}$-2${}^{2}$)+{$\dots$}+n(n${}^{2}$-n${}^{2}$)=$\frac{1}{4}$n${}^{2}$(n-1)(n+1).

 解析:

 证明:①当n=1时,左边=1·(1${}^{2}$-1${}^{2}$)=0,右边=$\frac{1}{4}\mathrm{\times}$1${}^{2}$$\mathrm{\times}$0$\mathrm{\times}$2=0,所以左边=右边,n=1时,等式成立.

②假设n=k(k$\mathrm{\ge}$1,k$\mathrm{\in}$N${}^{*}$)时,等式成立,即

1·(k${}^{2}$-1${}^{2}$)+2·(k${}^{2}$-2${}^{2}$)+{$\dots$}+k·(k${}^{2}$-k${}^{2}$)=$\frac{1}{4}$k${}^{2}$(k-1)·(k+1),所以当n=k+1时,左边=1·+2·+{$\dots$}+k·+(k+1)=+=$\frac{1}{4}$k${}^{2}$(k-1)(k+1)+$\frac{k(k+1)}{2}$·(2k+1)=$\frac{1}{4}$k(k+1)·

=$\frac{1}{4}$k(k+1)(k${}^{2}$+3k+2)=$\frac{1}{4}(k+1)^2k(k+2)$,

即n=k+1时,等式成立,

根据①与②可知等式对n$\mathrm{\in}$N${}^{*}$都成立.

 

 知识:数学归纳法及应用

 难度:1

 题目:

 求证:a${}^{n}$${}^{\textrm{+}}$${}^{2}$+(a+1)${}^{2n}$${}^{\textrm{+}}$${}^{1}$能被a${}^{2}$+a+1整除,n$\mathrm{\in}$N${}^{*}$.

 解析:

 证明:①当n=1时,

a${}^{3}$+(a+1)${}^{3}$==(2a+1)(a${}^{2}$+a+1).

结论成立.

②假设当n=k时,结论成立,

即a${}^{k}$${}^{\textrm{+}}$${}^{2}$+(a+1)${}^{2k}$${}^{\textrm{+}}$${}^{1}$能被a${}^{2}$+a+1整除,

那么n=k+1时,

有a${}^{(k}$${}^{\textrm{+}}$${}^{1)}$${}^{\textrm{+}}$${}^{2}$+(a+1)${}^{2(k}$${}^{\textrm{+}}$${}^{1)}$${}^{\textrm{+}}$${}^{1}$=a·a${}^{k}$${}^{\textrm{+}}$${}^{2}$+(a+1)${}^{2}$(a+1)${}^{2k}$${}^{\textrm{+}}$${}^{1}$

=a+(a+1)${}^{2}$(a+1)${}^{2k}$${}^{\textrm{+}}$${}^{1}$-a(a+1)${}^{2k}$${}^{\textrm{+}}$${}^{1}$

=a+(a${}^{2}$+a+1)(a+1)${}^{2k}$${}^{\textrm{+}}$${}^{1}$.

因为a${}^{k}$${}^{\textrm{+}}$${}^{2}$+(a+1)${}^{2k}$${}^{\textrm{+}}$${}^{1}$,a${}^{2}$+a+1均能被a${}^{2}$+a+1整除,

所以a${}^{(k}$${}^{\textrm{+}}$${}^{1)}$${}^{\textrm{+}}$${}^{2}$+(a+1)${}^{2(k}$${}^{\textrm{+}}$${}^{1)}$${}^{\textrm{+}}$${}^{1}$能被a${}^{2}$+a+1整除,

即当n=k+1时,结论也成立.

由①②可知,原结论成立.



 知识:数学归纳法及应用

 难度:2

 题目:有n个圆,任意两个圆都相交于两点,任意三个圆不相交于同一点,求证这n个圆将平面分成f(n)=n${}^{2}$-n+2个部分(n$\mathrm{\in}$N${}^{*}$).

 解析:

 证明:①当n=1时,一个圆将平面分成两个部分,且f(1)=1-1+2=2,

所以n=1时命题成立.

②假设n=k(k$\mathrm{\ge}$1)时命题成立.

即k个圆把平面分成f(k)=k${}^{2}$-k+2个部分.

则n=k+1时,在k+1个圆中任取一个圆O,剩下的k个圆将平面分成f(k)个部分,而圆O与k个圆有2k个交点,这2k个点将圆O分成2k段弧,每段弧将原平面一分为二,

故得f(k+1)=f(k)+2k=k${}^{2}$-k+2+2k

=(k+1)${}^{2}$-(k+1)+2,$\mathrm{\therefore}$当n=k+1时,命题成立.

综合①②可知,对一切n$\mathrm{\in}$N${}^{*}$,命题成立.

 

 知识:数学归纳法及应用

 难度:1

 题目:用数学归纳法证明``对于任意x$\mathrm{>}$0和正整数n,都有x${}^{n}$+x${}^{n}$${}^{\textrm{-}}$${}^{2}$+x${}^{n}$${}^{\textrm{-}}$${}^{4}$+{$\dots$}+$\frac{1}{x^{n-4}}$+$\frac{1}{x^{n-2}}$+$\frac{1}{x^n}$$\mathrm{\ge}$n+1''时,需验证的使命题成立的最小正整数值n${}_{0}$应为(  )

A.1         B.2

C.1,2  D.以上答案均不正确

 解析:需验证n${}_{0}$=1时,x+$\frac{1}{x}\mathrm{\ge}$1+1成立.

 答案:A



 知识:数学归纳法及应用

 难度:1

 题目:用数学归纳法证明``2${}^{n}$$\mathrm{>}$n${}^{2}$+1对于n$\mathrm{\ge}$n${}_{0}$的正整数n都成立''时,第一步证明中的起始值n${}_{0}$应取(  )

A.2  B.3  C.5  D.6

 解析: n取1,2,3,4时不等式不成立,起始值为5.

 答案:C

 

 知识:数学归纳法及应用

 难度:1

 题目:用数学归纳法证明``1+$\frac{1}{2}$+$\frac{1}{2}$+$\frac{1}{3}${$\dots$}+$\frac{1}{2^n-1}\mathrm{<}$n(n$\mathrm{\in}$N${}^{*}$,n$\mathrm{>}$1)''时,由n=k(k$\mathrm{>}$1)不等式成立,推证n=k+1时,左边应增加的项数是(  )

A.2${}^{k}$${}^{\textrm{-}}$${}^{1}$  B.2${}^{k}$-1  C.2${}^{k}$  D.2${}^{k}$+1

 解析:由n=k到n=k+1,应增加的项数为(2${}^{k}$${}^{\textrm{+}}$${}^{1}$-1)-(2${}^{k}$-1)=2${}^{k}$${}^{\textrm{+}}$${}^{1}$-2${}^{k}$=2${}^{k}$项.

 答案:C

 

 知识:数学归纳法及应用

 难度:1

 题目:设f(x)是定义在正整数集上的函数,且f(x)满足``当f(k)$\mathrm{\ge}$k${}^{2}$成立时,总可推出f(k+1)$\mathrm{\ge}$(k+1)${}^{2}$成立''.那么,下列命题总成立的是(  )

A.若f(1)$\mathrm{<}$1成立,则f\eqref{GrindEQ__10_}$\mathrm{<}$100成立

B.若f(2)$\mathrm{<}$4成立,则f(1)$\mathrm{\ge}$1成立

C.若f(3)$\mathrm{\ge}$9成立,则当k$\mathrm{\ge}$1时,均有f(k)$\mathrm{\ge}$k${}^{2}$成立

D.若f(4)$\mathrm{\ge}$16成立,则当k$\mathrm{\ge}$4时,均有f(k)$\mathrm{\ge}$k${}^{2}$成立

 解析:选项A、B与题设中不等号方向不同,故A、B错;选项C中,应该是k$\mathrm{\ge}$3时,均有f(k)$\mathrm{\ge}$k${}^{2}$成立;选项D符合题意.

 答案:D

 

 知识:数学归纳法及应用

 难度:1

 题目:证明$\frac{n+2}{2}\mathrm{<}$1+$\frac{1}{2}$+$\frac{1}{3}$+{$\dots$}+$\frac{1}{2n}\mathrm{<}$n+1(n$\mathrm{>}$1),当n=2时,要证明的式子为\_\_\_\_\_\_\_\_.

 解析:当n=2时,要证明的式子为2$\mathrm{<}$1$\frac{1}{2}$+$\frac{1}{3}+\frac{1}{4}\mathrm{<}$3.

 答案:2$\mathrm{<}$1$+\frac{1}{2}+\frac{1}{3}+\frac{1}{3}+\frac{1}{4}\mathrm{<}$3



 知识:数学归纳法及应用

 难度:1

 题目:利用数学归纳法证明``$(1+\frac{1}{3})(1+\frac{1}{5})$\dots$(1+\frac{1}{2n-1})\mathrm{>}\frac{\sqrt{2n+1}}{2}$''时,n的最小取值n${}_{0}$为\_\_\_\_\_\_\_\_.

 解析:左边为(n-1)项的乘积,故n${}_{0}$=2.

 答案:2

 

 知识:数学归纳法及应用

 难度:1

 题目:设a,b均为正实数(n$\mathrm{\in}$N${}^{*}$),已知M=(a+b)${}^{n}$,N=a${}^{n}$+na${}^{n}$${}^{\textrm{-}}$${}^{1}$b,则M,N的大小关系为\_\_\_\_\_\_\_\_(提示:利用贝努利不等式,令x=$\frac{b}{a}$).

 解析:当n=1时,M=a+b=N.当n=2时,M=(a+b)${}^{2}$,N=a${}^{2}$+2ab$\mathrm{<}$M.当n=3时,M=(a+b)${}^{3}$,N=a${}^{3}$+3a${}^{2}$b$\mathrm{<}$M.归纳得M $\mathrm{\ge}$N.

 答案:M $\mathrm{\ge}$N

 

 知识:数学归纳法及应用

 难度:1

 题目:用数学归纳法证明,对任意n$\mathrm{\in}$N${}^{*}$,有

(1+2+{$\dots$}+n)$(1+\frac{1}{2}+\frac{1}{3}+\cdots+\frac{1}{n})\mathrm{\ge}$n${}^{2}$.

 解析:

 证明:①当n=1时,左边=右边,不等式成立.

当n=2时,左边=(1+2)$(1+\frac{1}{2})$=$\frac{9}{2}\mathrm{>}$2${}^{2}$,不等式成立.

②假设当n=k(k$\mathrm{\ge}$2)时不等式成立,

即(1+2+{$\dots$}+k)$(1+\frac{1}{2}+\cdots+\frac{1}{k})\mathrm{\ge}$k${}^{2}$.

则当n=k+1时,有

左边=$[(1+2+\cdots+k+(k+1))(1+\frac{1}{2}+\cdots+\frac{1}{k}+\frac{1}{k+1})]$

=(1+2+{$\dots$}+k)$(1+\frac{1}{2}+\cdots+\frac{1}{k})$+(1+2+{$\dots$}+k)·$\frac{1}{k+1}$+(k+1)$\mathrm{\times}(1+\frac{1}{2}+\cdots+\frac{1}{k})$+1

$\mathrm{\ge}$k${}^{2}$+$\frac{k}{2}$+1+(k+1)$(1+\frac{1}{2}+\cdots+\frac{1}{k})$.

$\mathrm{\because}$当k$\mathrm{\ge}$2时,1+$\frac{1}{2}$+{$\dots$}+$\frac{1}{k}\mathrm{\ge}$1+$\frac{1}{2}$=$\frac{3}{2}$,

$\mathrm{\therefore}$左边$\mathrm{\ge}$k${}^{2}$+$\frac{k}{2}$+1+(k+1)$\mathrm{\times}\frac{3}{2}$

=k${}^{2}$+2k+1+   $\frac{3}{2}\mathrm{\ge}$(k+1)${}^{2}$.

这就是说当n=k+1时,不等式成立.

由①②可知当n$\mathrm{\ge}$1时,不等式成立.

 

 知识:数学归纳法及应用

 难度:1

 题目:设数列$\mathrm{\{}$a${}_{n}$$\mathrm{\}}$满足a${}_{n}$${}_{\textrm{+}}$${}_{1}$=$a_0^2$-na${}_{n}$+1,n=1,2,3{$\dots$}.

(1)当a${}_{1}$=2时,求a${}_{2}$,a${}_{3}$,a${}_{4}$,并由此猜想出a${}_{n}$的一个通项公式;

(2)当a$\mathrm{\ge}$3时,证明对所有的n$\mathrm{\ge}$1,有a${}_{n}$$\mathrm{\ge}$n+2.

 解析:

 解:(1)由a${}_{1}$=2,得a${}_{2}$=$a_0^2$-a${}_{1}$+1=3;

由a${}_{2}$=3,得a${}_{3}$=a$_2^2$-2a${}_{2}$+1=4;

由a${}_{3}$=4,得a${}_{4}$=$a_3^2$-3a${}_{3}$+1=5.

由此猜想a${}_{n}$的一个通项公式:a${}_{n}$=n+1(n$\mathrm{\ge}$1).

(2)证明:用数学归纳法证明.

①当n=1,a${}_{1}$$\mathrm{\ge}$3=1+2,不等式成立.

②假设当n=k时不等式成立,

即a${}_{k}$$\mathrm{\ge}$k+2.

那么,当n=k+1时,a${}_{k}$${}_{\textrm{+}}$${}_{1}$=a${}_{k}$(a${}_{k}$-k)+1$\mathrm{\ge}$(k+2)(k+2-k)+1$\mathrm{\ge}$k+3,

也就是说,当n=k+1时,a${}_{k}$${}_{\textrm{+}}$${}_{1}$$\mathrm{\ge}$(k+1)+2.

根据①和②,对于所有n$\mathrm{\ge}$1,有a${}_{n}$$\mathrm{\ge}$n+2.





 知识:数学归纳法及应用

 难度:1

 题目:设a$\mathrm{\in}$R,f(x)=$\frac{a\cdot 2^x+a-2}{2^x+1}$是奇函数.

(1)求a的值;

(2)如果g(n)=$\frac{n}{n+1}$(n$\mathrm{\in}$N${}^{*}$),试比较f(n)与g(n)的大小(n$\mathrm{\in}$N${}^{*}$).

 解析:

 解:(1)$\mathrm{\because}$f(x)是定义在R上的奇函数,

$\mathrm{\therefore}$f(0)=0.故a=1.

(2)f(n)-g(n)=$\frac{2^n-1}{2^n+1}-\frac{n}{n+1}=\frac{2^n-2n-1}{(2^n+1)(n+1)}$.

只要比较2${}^{n}$与2n+1的大小.

当n=1,2时,f(n)$\mathrm{<}$g(n);

当n$\mathrm{\ge}$3时,2${}^{n}$$\mathrm{>}$2n+1,f(n)$\mathrm{>}$g(n).

下面证明,n$\mathrm{\ge}$3时,2${}^{n}$$\mathrm{>}$2n+1,即f(x)$\mathrm{>}$g(x).

①n=3时,2${}^{3}$$\mathrm{>}$2$\mathrm{\times}$3+1,显然成立,

②假设n=k(k$\mathrm{\ge}$3,k$\mathrm{\in}$N${}^{*}$)时,2${}^{k}$$\mathrm{>}$2k+1,

那么n=k+1时,2${}^{k}$${}^{\textrm{+}}$${}^{1}$=2$\mathrm{\times}$2${}^{k}$$\mathrm{>}$2(2k+1).

2(2k+1)-=4k+2-2k-3=2k-1$\mathrm{>}$0($\mathrm{\because}$k$\mathrm{\ge}$3),有2${}^{k}$${}^{\textrm{+}}$${}^{1}$$\mathrm{>}$2(k+1)+1.

$\mathrm{\therefore}$n=k+1时,不等式也成立.

由①②可以判定,n$\mathrm{\ge}$3,n$\mathrm{\in}$N${}^{*}$时,2${}^{n}$$\mathrm{>}$2n+1.

$\mathrm{\therefore}$n=1,2时,f(n)$\mathrm{<}$g(n);

当n$\mathrm{\ge}$3,n$\mathrm{\in}$N${}^{*}$时,f(n)$\mathrm{>}$g(n).

 

 知识:曲边梯形的面积

 难度:1

 题目:和式$\sum\limits_{i=1}^5$(\textit{y${}_{i}$}+1)可表示为(  )

A.(\textit{y}${}_{1}$+1)+(\textit{y}${}_{5}$+1)

B.\textit{y}${}_{1}$+\textit{y}${}_{2}$+\textit{y}${}_{3}$+\textit{y}${}_{4}$+\textit{y}${}_{5}$+1

C.\textit{y}${}_{1}$+\textit{y}${}_{2}$+\textit{y}${}_{3}$+\textit{y}${}_{4}$+\textit{y}${}_{5}$+5

D.(\textit{y}${}_{1}$+1)(\textit{y}${}_{2}$+1){$\dots$}(\textit{y}${}_{5}$+1)

 解析: $\sum\limits_{i=1}^5$(\textit{y${}_{i}$}+1)=(\textit{y}${}_{1}$+1)+(\textit{y}${}_{2}$+1)+(\textit{y}${}_{3}$+1)+(\textit{y}${}_{4}$+1)+(\textit{y}${}_{5}$+1)=\textit{y}${}_{1}$+\textit{y}${}_{2}$+\textit{y}${}_{3}$+\textit{y}${}_{4}$+\textit{y}${}_{5}$+5,故选C.

 答案: C



 知识:曲边梯形的面积

 难度:1

 题目:在求由\textit{x}=\textit{a},\textit{x}=\textit{b}(\textit{a}$\mathrm{<}$\textit{b}),\textit{y}=\textit{f}(\textit{x})(\textit{f}(\textit{x})$\mathrm{\ge}$0)及\textit{y}=0围成的曲边梯形的面积\textit{S}时,在区间[\textit{a},\textit{b}]上等间隔地插入\textit{n}-1个分点,分别过这些分点作\textit{x}轴的垂线,把曲边梯形分成\textit{n}个小曲边梯形,下列说法中正确的个数是(  )

①\textit{n}个小曲边梯形的面积和等于\textit{S};

②\textit{n}个小曲边梯形的面积和小于\textit{S};

③\textit{n}个小曲边梯形的面积和大于\textit{S};

④\textit{n}个小曲边梯形的面积和与\textit{S}之间的大小关系无法确定

A.1个       B.2个   

C.3个       D.4个

 解析: \textit{n}个小曲边梯形是所给曲边梯形等距离分割得到的,因此其面积和为\textit{S}.$\mathrm{\therefore}$①正确,②③④错误,故应选A.

 答案: A



 知识:曲边梯形的面积

 难度:1

 题目:在``近似代替''中,函数\textit{f}(\textit{x})在区间[\textit{x${}_{i}$},\textit{x${}_{i}$}${}_{\textrm{+}}$${}_{1}$]上的近似值等于(  )

A.只能是左端点的函数值\textit{f}(\textit{x${}_{i}$})

B.只能是右端点的函数值\textit{f}(\textit{x${}_{i}$}${}_{\textrm{+}}$${}_{1}$)

C.可以是该区间内任一点的函数值\textit{f}(\textit{$\xi$${}_{i}$})(\textit{$\xi$${}_{i}$}$\mathrm{\in}$[\textit{x${}_{i}$},\textit{x${}_{i}$}${}_{\textrm{+}}$${}_{1}$])

D.以上答案均不正确

 解析: 由求曲边梯形面积的``近似代替''知,C正确,故应选C.

 答案: C

 知识:曲边梯形的面积

 难度:1

 题目:(2010·惠州高二检测)求由抛物线\textit{y}=2\textit{x}${}^{2}$与直线\textit{x}=0,\textit{x}=\textit{t}(\textit{t}$\mathrm{>}$0),\textit{y}=0所围成的曲边梯形的面积时,将区间[0,\textit{t}]等分成\textit{n}个小区间,则第\textit{i}-1个区间为(  )

A.$[\frac{i-1}{n}, \frac{i}{n}]$     B.$[\frac{i}{n}, \frac{i+1}{n}]$

C.$[\frac{t(i-1)}{n}, \frac{ti}{n}]$      D.$[\frac{t(i-2)}{n}, \frac{t(i-1)}{n}]$

 解析: 在[0,\textit{t}]上等间隔插入(\textit{n}-1)个分点,把区间[0,\textit{t}]等分成\textit{n}个小区间,每个小区间的长度均为$\frac{t}{n}$,故第\textit{i}-1个区间为$[\frac{t(i-2)}{n}, \frac{t(i-1)}{n}]$,故选D.

 答案: D



 知识:曲边梯形的面积

 难度:1

 题目:由直线\textit{x}=1,\textit{y}=0,\textit{x}=0和曲线\textit{y}=\textit{x}${}^{3}$所围成的曲边梯形,将区间4等分,则曲边梯形面积的近似值(取每个区间的右端点)是(  )

A.$\frac{1}{19}$        B.$\frac{111}{256}$  

C.$\frac{110}{270}$        D.$\frac{25}{64}$

 解析: \textit{s}=$[(\frac{1}{4})^3+(\frac{2}{4})^3+(\frac{3}{4})^3+1^3]\mathrm{\times}\frac{1}{4}$=$\frac{1^3+2^3+3^3+4^3}{4^4}$=$\frac{25}{64}$.

 答案: D



 知识:曲边梯形的面积

 难度:1

 题目:在等分区间的情况下,\textit{f}(\textit{x})=$\frac{1}{1+x^2}$(\textit{x}$\mathrm{\in}$[0,2])及\textit{x}轴所围成的曲边梯形面积和式的极限形式正确的是(  )

A.$\lim\limits_{n\rightarrow \infty}\sum\limits_{i=1}^n[\frac{1}{1+(\frac{i}{n})^2}\cdot\frac{2}{n}]$

B.$\lim\limits_{n\rightarrow \infty}\sum\limits_{i=1}^n[\frac{1}{1+(\frac{2i}{n})^2}\cdot\frac{2}{n}]$

C.$\lim\limits_{n\rightarrow \infty}\sum\limits_{i=1}^n[\frac{1}{1+i^2}\cdot\frac{1}{n}]$

D.$\lim\limits_{n\rightarrow \infty}\sum\limits_{i=1}^n[\frac{1}{1+(\frac{i}{n})^2}\cdot n]$

 解析: 将区间[0,2]进行\textit{n}等分每个区间长度为$\frac{2}{n}$,故应选B.

 答案: B



 知识:曲边梯形的面积

 难度:1

 题目:求直线\textit{x}=0,\textit{x}=2,\textit{y}=0与曲线\textit{y}=\textit{x}${}^{2}$所围成曲边梯形的面积.

 解析: 按分割,近似代替,求和,取极限四个步骤进行.

 解: 将区间[0,2]分成\textit{n}个小区间,则第\textit{i}个小区间为$[\frac{2(i-1)}{n}, \frac{2i}{n}]$.

第\textit{i}个小区间的面积$\Delta$\textit{S${}_{i}$}=\textit{f}$(\frac{2(t-1)}{n})\cdot\frac{2}{n}$,

$\mathrm{\therefore}$\textit{S${}_{n}$}=$\sum\limits_{i=1}^nf(\frac{2(i-1)}{n})\cdot\frac{2}{n}$

=$\frac{2}{n}\sum\limits_{i=1}^n\frac{4(i-1)^2}{n^2}=\frac{8}{n^3}\sum\limits_{i=1}^n(i-1)^2$

=$\frac{8}{n^3}$[0${}^{2}$+1${}^{2}$+2${}^{2}$+{$\dots$}+(\textit{n}-1)${}^{2}$]

=$\frac{8}{n^3}\cdot\frac{(n-1)n(2n-1)}{6}$·

=$\frac{8(n-1)(2n-1)}{6n^2}$.

\textit{S}=$\lim\limits_{n\rightarrow \infty}$\textit{S${}_{n}$}=$\lim\limits_{n\rightarrow \infty} \frac{8(n-1)(2n-1)}{6n^2}$ =$\frac{8}{3}$,

$\mathrm{\therefore}$所求曲边梯形面积为$\frac{8}{3}$.



 知识:行驶路程

 难度:1

 题目:汽车以速度\textit{v}做匀速直线运动时,经过时间\textit{t}所行驶的路程\textit{s}=\textit{vt}.如果汽车做变速直线运动,在时刻\textit{t}的速度为\textit{v}(\textit{t})=\textit{t}${}^{2}$+2(单位:km/h),那么它在1$\mathrm{\le}$\textit{t}$\mathrm{\le}$2(单位:h)这段时间行驶的路程是多少?

 解析: 汽车行驶路程类似曲边梯形面积,根据曲边梯形面积思想,求和后再求极限值.

 解: 将区间[1,2]等分成\textit{n}个小区间,第\textit{i}个小区间为$[1+\frac{i-1}{n}, 1+\frac{i}{n}]$.

$\mathrm{\therefore}$$\Delta$\textit{s${}_{i}$}=\textit{f}$(1+\frac{i-1}{n})\cdot\frac{1}{n}$.

\textit{s${}_{n}$}=$\sum\limits_{i=1}^nf(1+\frac{i-1}{n})\cdot\frac{1}{n}$

=$\frac{1}{n}\sum\limits_{i=1}^n[(1+\frac{i-1}{n})^2+2]$

=$\frac{1}{n}\sum\limits_{i=1}^n[\frac{(i-1)^2}{n}+\frac{2(i-1)}{n}+3]$

=$\frac{1}{n}$3\textit{n}+$\frac{1}{n^2}$[0${}^{2}$+1${}^{2}$+2${}^{2}$+{$\dots$}+(\textit{n}-1)${}^{2}$]+$\frac{1}{n}$[0+2+4+6+{$\dots$}+2(\textit{n}-1)]

=3$+\frac{(n-1)(2n-1)}{6n^2}+\frac{n-1}{n}$.

\textit{s}=\textit{s${}_{n}$}=$\lim\limits_{n\rightarrow \infty} s_n$ =$\lim\limits_{n\rightarrow \infty}[3+\frac{(n-1)(2n-1)}{6n^2}+\frac{n-1}{n}]=\frac{13}{3}$.

$\mathrm{\therefore}$这段时间行驶的路程为$\frac{13}{3}$km.



 知识:行驶路程

 难度:1

 题目:求物体自由落体的下落距离:已知自由落体的运动速度\textit{v}=\textit{gt},求在时间区间[0,\textit{t}]内物体下落的距离.

 解析: 选定区间$\mathrm{\to}$分割$\mathrm{\to}$近似代替$\mathrm{\to}$求和$\mathrm{\to}$求极限

 解: (1)分割:将时间区间[0,\textit{t}]分成\textit{n}等份.

把时间[0,\textit{t}]分成\textit{n}个小区间$[\frac{i-1}{n}t, \frac{it}{n}]$(\textit{i}=1,2,{$\dots$},\textit{n}),

每个小区间所表示的时间段$\Delta$\textit{t}=$\frac{it}{n}$-$\frac{i-1}{n}$\textit{t}=$\frac{t}{n}$,在各小区间物体下落的距离记作$\Delta$\textit{s${}_{i}$}(\textit{i}=1,2,{$\dots$},\textit{n}).

(2)近似代替:在每个小区间上以匀速运动的路程近似代替变速运动的路程.

在$[\frac{i-1}{n}t, \frac{it}{n}]$上任取一时刻\textit{$\xi$${}_{i}$}(\textit{i}=1,2,{$\dots$},\textit{n}),可取\textit{$\xi$${}_{i}$}使\textit{v}(\textit{$\xi$${}_{i}$})=$g\frac{(i-1)}{n}t$近似代替第\textit{i}个小区间上的速度,因此在每个小区间上自由落体$\Delta$\textit{t}=$\frac{t}{n}$内所经过的距离可近似表示为$\Delta$\textit{s${}_{i}$}$\mathrm{\approx}$\textit{g}$(\frac{i-1}{n}t)\cdot\frac{t}{n}$(\textit{i}=1,2,{$\dots$},\textit{n}).

(3)求和:\textit{s${}_{n}$}=$\sum\limits_{i=1}^n$\textit{s${}_{i}$}

=$\sum\limits_{i=1}^ng(\frac{i-1}{n}\cdot t)\cdot\frac{t}{n}$

=$\frac{gt^2}{n^2}$[0+1+2+{$\dots$}+(\textit{n}-1)]

=$\frac{1}{2}$\textit{gt}${}^{2}(1-\frac{1}{n})$.

(4)取极限:\textit{s}= $\lim\limits_{n\rightarrow \infty}$\textit{gt}${}^{2}(1-\frac{1}{n})$=\textit{gt}${}^{2}$.

 

 

 知识:定积分的概念

 难度:1

 题目:定积分$\int_{1}^{3}$(-3)d\textit{x}等于(  )

A.-6          B.6

C.-3       D.3

 解析: 由积分的几何意义可知$\int_1^3$(-3)d\textit{x}表示由\textit{x}=1,\textit{x}=3,\textit{y}=0及\textit{y}=-3所围成的矩形面积的相反数,故$\int_1^3$(-3)d\textit{x}=-6.

 答案:A

 

 知识:定积分的概念

 难度:1

 题目:定积分$\int_a^b$\textit{f}(\textit{x})d\textit{x}的大小(  )

A.与\textit{f}(\textit{x})和积分区间[\textit{a},\textit{b}]有关,与\textit{$\xi$${}_{i}$}的取法无关

B.与\textit{f}(\textit{x})有关,与区间[\textit{a},\textit{b}]以及\textit{$\xi$${}_{i}$}的取法无关

C.与\textit{f}(\textit{x})以及\textit{$\xi$${}_{i}$}的取法有关,与区间[\textit{a},\textit{b}]无关

D.与\textit{f}(\textit{x})、区间[\textit{a},\textit{b}]和\textit{$\xi$${}_{i}$}的取法都有关

 解析: 由定积分定义及求曲边梯形面积的四个步骤知A正确.

 答案: A



 知识:定积分的概念

 难度:1

 题目:下列说法成立的个数是(  )

①$\int_a^b$\textit{f}(\textit{x})d\textit{x}=$\sum\limits_{i=1}^n$f(\textit{$\xi$${}_{i}$})$\frac{b-a}{n}$

②$\int_a^b$\textit{f}(\textit{x})d\textit{x}等于当\textit{n}趋近于+$\mathrm{\infty}$时,\textit{f}(\textit{$\xi$${}_{i}$})·$\frac{b-a}{n}$无限趋近的值

③$\int_a^b$\textit{f}(\textit{x})d\textit{x}等于当\textit{n}无限趋近于+$\mathrm{\infty}$时,$\sum\limits_{i=1}^n$f(\textit{$\xi$${}_{i}$})$\frac{b-a}{n}$无限趋近的常数

④$\int_a^b$\textit{f}(\textit{x})d\textit{x}可以是一个函数式子

A.1        B.2

C.3        D.4

 解析: 由$\int_a^b$\textit{f}(\textit{x})d\textit{x}的定义及求法知仅③正确,其余不正确.故应选A.

 答案: A



 知识:定积分的概念

 难度:1

 题目:已知$\int_1^3$\textit{f}(\textit{x})d\textit{x}=56,则(  )

A.$\int_1^2$\textit{f}(\textit{x})d\textit{x}=28      B.$\int_2^3$\textit{f}(\textit{x})d\textit{x}=28

C.$\int_1^2$2\textit{f}(\textit{x})d\textit{x}=56      D.$\int_1^2$\textit{f}(\textit{x})d\textit{x}+$\int_2^3$\textit{f}(\textit{x})d\textit{x}=56

 解析: 由\textit{y}=\textit{f}(\textit{x}),\textit{x}=1,\textit{x}=3及\textit{y}=0围成的曲边梯形可分拆成两个:由\textit{y}=\textit{f}(\textit{x}),\textit{x}=1,\textit{x}=2及\textit{y}=0围成的曲边梯形知由\textit{y}=\textit{f}(\textit{x}),\textit{x}=2,\textit{x}=3及\textit{y}=0围成的曲边梯形.

$\mathrm{\therefore}\int_1^3$\textit{f}(\textit{x})d\textit{x}=$\int_1^2$\textit{f}(\textit{x})d\textit{x}+$\int_2^3$\textit{f}(\textit{x})d\textit{x}

即$\int_1^2$\textit{f}(\textit{x})d\textit{x}+$\int_2^3$\textit{f}(\textit{x})d\textit{x}=56.

故应选D.

 答案: D



 知识:定积分的概念

 难度:1

 题目:已知$\int_a^b$\textit{f}(\textit{x})d\textit{x}=6,则$\int_a^b$6\textit{f}(\textit{x})d\textit{x}等于(  )

A.6        B.6(\textit{b}-\textit{a})

C.36        D.不确定

 解析: $\mathrm{\because}$$\int_a^b$\textit{f}(\textit{x})d\textit{x}=6,

$\mathrm{\therefore}$在$\int_a^b$6\textit{f}(\textit{x})d\textit{x}中曲边梯形上、下底长变为原来的6倍,由梯形面积公式,知$\int_a^b$6\textit{f}(\textit{x})d\textit{x}=6$\int_a^b$\textit{f}(\textit{x})d\textit{x}=36.故应选C.

 答案:C



 知识:定积分的概念

 难度:1

 题目:设\textit{f}(\textit{x})=$\left\{\begin{array}{l} x^2, (x\ge 0)\\ 2^x, (x<0) \end{array}\right.$则$\int_{-1}^1$\textit{f}(\textit{x})d\textit{x}的值是(  )

A. $\int_{-1}^{1}x^2$d$x$   B.$\int_{-1}^{1}2^x$d$x$
C. $\int_{-1}^{0}x^2$d$x$+$\int_{0}^{1}2^x$d$x$    D.$\int_{-1}^{0}2^x$d$x$+$\int_{0}^{1}x^2$d$x$


 解析: 由定积分性质(3)求\textit{f}(\textit{x})在区间[-1,1]上的定积分,可以通过求\textit{f}(\textit{x})在区间[-1,0]与[0,1]上的定积分来实现,显然D正确,故应选D.

 答案: D



 知识:定积分的概念

 难度:1

 题目:下列命题不正确的是(  )

A.若\textit{f}(\textit{x})是连续的奇函数,则$\int_{-a}^{a}f(x)$d$x$=0

B.若\textit{f}(\textit{x})是连续的偶函数,则$\int_{-a}^{a}f(x)$d$x$=$2\int_0^a f(x)$d$x$

C.若\textit{f}(\textit{x})在[\textit{a},\textit{b}]上连续且恒正,则$\int_a^b$\textit{f}(\textit{x})d\textit{x}$\mathrm{>}$0

D.若\textit{f}(\textit{x})在[\textit{a},\textit{b})上连续且$\int_a^b$\textit{f}(\textit{x})d\textit{x}$\mathrm{>}$0,则\textit{f}(\textit{x})在[\textit{a},\textit{b})上恒正

 解析: 本题考查定积分的几何意义,对A:因为\textit{f}(\textit{x})是奇函数,所以图象关于原点对称,所以\textit{x}轴上方的面积和\textit{x}轴下方的面积相等,故积分是0,所以A正确.对B:因为\textit{f}(\textit{x})是偶函数,所以图象关于\textit{y}轴对称,故图象都在\textit{x}轴下方或上方且面积相等,故B正确.C显然正确.D选项中\textit{f}(\textit{x})也可以小于0,但必须有大于0的部分,且\textit{f}(\textit{x})$\mathrm{>}$0的曲线围成的面积比\textit{f}(\textit{x})$\mathrm{<}$0的曲线围成的面积大.

 答案: D



 知识:定积分的概念

 难度:1

 题目:利用定积分的有关性质和几何意义可以得出定积分$\int_{-1}^1$[(tan\textit{x})${}^{11}$+(cos\textit{x})${}^{21}$]d\textit{x}=

(  )

A.2$\int_0^1$[(tan\textit{x})${}^{11}$+(cos\textit{x})${}^{21}$]d\textit{x}

B.0

C.2$\int_0^1$(cos\textit{x})${}^{21}$d\textit{x}

D.2

 解析: $\mathrm{\because}$\textit{y}=tan\textit{x}为[-1,1]上的奇函数,

$\mathrm{\therefore}$\textit{y}=(tan\textit{x})${}^{11}$仍为奇函数,而\textit{y}=(cos\textit{x})${}^{21}$是偶函数,

$\mathrm{\therefore}$原式=$\int_{-1}^{1}$(cos\textit{x})${}^{21}$d\textit{x}=2$\int_0^1$(cos\textit{x})${}^{21}$d\textit{x}.故应选C.

 答案: C



 知识:定积分的概念

 难度:1

 题目:设\textit{f}(\textit{x})是[\textit{a},\textit{b}]上的连续函数,则$\int_a^b$\textit{f}(\textit{x})d\textit{x}-$\int_a^b$\textit{f}(\textit{t})d\textit{t}的值(  )

A.小于零       B.等于零

C.大于零       D.不能确定

 解析: $\int_a^b$\textit{f}(\textit{x})d\textit{x}和$\int_a^b$\textit{f}(\textit{t})d\textit{t}都表示曲线\textit{y}=\textit{f}(\textit{x})与\textit{x}=\textit{a},\textit{x}=\textit{b}及\textit{y}=0围成的曲边梯形面积,不因曲线中变量字母不同而改变曲线的形状和位置.所以其值为0.

 答案:B



 知识:定积分的概念

 难度:1

 题目:由\textit{y}=sin\textit{x},\textit{x}=0,\textit{x}=$\frac{\pi}{2}$,\textit{y}=0所围成的图形的面积可以写成\_\_\_\_\_\_\_\_.

 解析: 由定积分的几何意义可得.

 答案:  $\int_0^{\frac{\pi}{2}}\sin x$d$x$



 

 知识:定积分的概念

 难度:1

 \includegraphics*[width=1.01in, height=1.08in, keepaspectratio=false]{image7}题目:(2\textit{x}-4)d\textit{x}=\_\_\_\_\_\_\_\_.

 解析: 如图\textit{A}(0,-4),\textit{B}(6,8)

\textit{S}${}_{\mathrm{\vartriangle }}$\textit{${}_{AOM}$}=$\frac{1}{2}\mathrm{\times}$2$\mathrm{\times}$4=4

\textit{S}${}_{\mathrm{\vartriangle }}$\textit{${}_{MBC}$}=$\frac{1}{2}\mathrm{\times}$4$\mathrm{\times}$8=16

$\mathrm{\therefore}\int_0^6$(2\textit{x}-4)d\textit{x}=16-4=12.

 答案: 12



 知识:定积分的概念

 难度:1

 题目:(2010·新课标全国理,13)设\textit{y}=\textit{f}(\textit{x})为区间[0,1]上的连续函数,且恒有0$\mathrm{\le}$\textit{f}(\textit{x})$\mathrm{\le}$1,可以用随机模拟方法近似计算积分$\int_0^1$\textit{f}(\textit{x})d\textit{x}.先产生两组(每组\textit{N}个)区间[0,1]上的均匀随机数\textit{x}${}_{1}$,\textit{x}${}_{2}$,{$\dots$},\textit{x${}_{N}$}和\textit{y}${}_{1}$,\textit{y}${}_{2}$,{$\dots$},\textit{y${}_{N}$},由此得到\textit{N}个点(\textit{x${}_{i}$},\textit{y${}_{i}$})(\textit{i}=1,2,{$\dots$},\textit{N}).再数出其中满足\textit{y${}_{i}$}$\mathrm{\le}$\textit{f}(\textit{x${}_{i}$})(\textit{i}=1,2,{$\dots$},\textit{N})的点数\textit{N}${}_{1}$,那么由随机模拟方法可得积分$\int_0^1$\textit{f}(\textit{x})d\textit{x}的近似值为\_\_\_\_\_\_\_\_.


 \includegraphics*[width=0.98in, height=1.03in, keepaspectratio=false]{image8}


 解析: 因为0$\mathrm{\le}$\textit{f}(\textit{x})$\mathrm{\le}$1且由积分的定义知:$\int_0^1$\textit{f}(\textit{x})d\textit{x}是由直线\textit{x}=0,\textit{x}=1及曲线\textit{y}=\textit{f}(\textit{x})与\textit{x}轴所围成的面积,又产生的随机数对在如图所示的正方形内,正方形面积为1,且满足\textit{y${}_{i}$}$\mathrm{\le}$\textit{f}(\textit{x${}_{i}$})的有\textit{N}${}_{1}$个点,即在函数\textit{f}(\textit{x})的图象上及图象下方有\textit{N}${}_{1}$个点,所以用几何概型的概率公式得:\textit{f}(\textit{x})在\textit{x}=0到\textit{x}=1上与\textit{x}轴围成的面积为$\frac{N_1}{N}\mathrm{\times}$1=$\frac{N_1}{N}$,即$\int_0^1$\textit{f}(\textit{x})d\textit{x}=$\frac{N_1}{N}$.

 答案:$\frac{N_1}{N}$

 

 知识:定积分的概念

 难度:1

 题目:利用定积分的几何意义,说明下列等式.

 (1)$\int_0^12x$d$x$=1; (2)$\int_{-1}^{1}\sqrt{1-x^2}$d$x=\frac{\pi}{2}$

 解析: 

 解: (1)$\int_0^1$2\textit{x}d\textit{x}表示由直线\textit{y}=2\textit{x},直线\textit{x}=0,\textit{x}=1,\textit{y}=0所围成的图形的面积,如图所示,阴影部分为直角三角形,所以\textit{S}${}_{\mathrm{\vartriangle }}$=$\frac{1}{2}\mathrm{\times}$1$\mathrm{\times}$2=1,故$\int_0^1$2\textit{x}d\textit{x}=1.

\includegraphics*[width=1.57in, height=1.27in, keepaspectratio=false]{image10}

(2)$\int_{-1}^{1}\sqrt{1-x^2}$\textit{x}表示由曲线\textit{y}=$\sqrt{1-x^2}$,直线\textit{x}=-1,\textit{x}=1,\textit{y}=0所围成的图形面积(而\textit{y}=$\sqrt{1-x^2}$表示圆\textit{x}${}^{2}$+\textit{y}${}^{2}$=1在\textit{x}轴上面的半圆),如图所示阴影部分,所以\textit{S}${}_{\textrm{半}\textrm{圆}}$=$\frac{\pi}{2}$,

故$\int_{-1}^{1}\sqrt{1-x^2}$d$x$=$\frac{\pi}{2}$

\includegraphics*[width=1.85in, height=1.25in, keepaspectratio=false]{image12}

 知识:定积分的概念

 难度:1

 题目:利用定积分的性质求$\int_{-1}^{1}(\frac{2x}{x^4+1}+\sin^3x+x^2-\frac{e^x-1}{e^x+1})$d\textit{x}.

 解析:

 解: \textit{y}=$\frac{2x}{x^4+1}$,\textit{y}=sin${}^{3}$\textit{x}均为[-1,1]上的奇函数,而对于\textit{f}(\textit{x})=$\frac{e^x-1}{e^x+1}$,

$\mathrm{\because}$\textit{f}(-\textit{x})=$\frac{e^{-1}-1}{e^{-1}+1}$=$\frac{1-e^x}{1+e^x}$=-\textit{f}(\textit{x}),

此函数为奇函数.

$\therefore\int_{-1}^{1}(\frac{2x}{x^4+1})$d$x$=0, $\int_{-1}^{1}\sin^3x$d$x=0$, $\int_{-1}^{1}\frac{e^x-1}{e^x+1}$d$x=0$

$\therefore$原式$=\int_{-1}^{1}x^2$d$x=2\int_0^1x^2$d$x$

$\mathrm{\because}$\textit{S}=$\sum\limits_{i=1}^{n}\frac{1}{n}\cdot(\frac{i}{n})^2$=$\frac{1}{n^3}\sum\limits_{i=1}^{n}(i)^2$

=$\frac{1}{n^3}\cdot\frac{1}{6}$\textit{n}(\textit{n}+1)(2\textit{n}+1)

=$\frac{1}{6}(2+\frac{3}{n}+\frac{1}{n^2})$

$\mathrm{\therefore}$\textit{S}=$\lim\limits_{n\rightarrow \infty}\frac{1}{6}(2+\frac{3}{n}+\frac{1}{n^2})$=$\frac{1}{3}$

即2$\int_{0}^{1}$\textit{x}${}^{2}$d\textit{x}=2$\mathrm{\times}\frac{1}{3}$=$\frac{2}{3}$

$\int_{-1}^{1}(\frac{2x}{x^4+1}+\sin^3x+x^2-\frac{e^x-1}{e^x+1})$d$x=\frac{2}{3}$



 知识:定积分的概念

 难度:1

 题目:已知函数\textit{f}(\textit{x})=$\left\{\begin{array}{l} x^3, x\in [-2, 2)\\ 2x, x\in[2, \pi)\\ \cos x, x\in[\pi, 2\pi] \end{array}\right.$,求\textit{f}(\textit{x})在区间[-2,2$\pi$]上的积分.

 解析:

 解: 由定积分的几何意义知

$\int_{-2}^{2}x^3$d$x$=0

$\int_{2}^{\pi}2x$d$x=\frac{(\pi-2)(2\pi+4)}{2}=\pi^2-4$

$\int_{\pi}^{2\pi}\cos x$d$x=0$, 由定积分的性质得

$\int_{-2}^{2\pi}f(x)$d$x=\int_{-2}^{2}x^3$d$x+\int_{2}^\pi 2x$d$x+\int_{\pi}^{2\pi}\cos x$d$x$

=$\pi$${}^{2}$-4.

 

 知识:定积分的概念

 难度:1

 题目:利用定积分的定义计算$\int_a^b$\textit{x}d\textit{x}.

 解析:

 解: (1)分割:将区间[\textit{a},\textit{b}]\textit{n}等分,则每一个小区间长为$\Delta$\textit{x${}_{i}$}=$\frac{b-a}{n}$(\textit{i}=1,2,{$\dots$},\textit{n}).

(2)近似代替:在小区间[\textit{x${}_{i}$}${}_{\textrm{-}}$${}_{1}$,\textit{x${}_{i}$}]上取点:\textit{$\xi$${}_{i}$}=\textit{a}+$\frac{i(b-a)}{n}$(\textit{i}=1,2,{$\dots$},\textit{n}).

\textit{I${}_{i}$}=\textit{f}(\textit{$\xi$${}_{i}$})·$\Delta$\textit{x${}_{i}$}=$[a+\frac{i(b-a)}{n}]\cdot\frac{b-a}{n}$.

(3)求和:\textit{I${}_{n}$}=$\sum\limits_{i=1}^nf$(\textit{$\xi$${}_{i}$})·$\Delta$\textit{x${}_{i}$}

=$\sum\limits_{i=1}^n[a+\frac{i(b-a)}{n}]\cdot\frac{b-a}{n}$

=$\frac{b-a}{n}\sum\limits_{i=1}^n[a+\frac{i(b-a)}{n}]$

=$\frac{b-a}{n}[\sum\limits_{i=1}^n a+\sum\limits_{i=1}^n \frac{i(b-a)}{n}]$

=$\frac{b-a}{n}(na+\frac{b-a}{n}\sum\limits_{i=1}^n i)$

=(\textit{b}-\textit{a})$(a+\frac{b-a}{n^2}\cdot\frac{n(n+1)}{2})$

(4)求极限:$\int_a^b$\textit{x}d\textit{x}=$\lim\limits_{n\rightarrow \infty}$\textit{I${}_{n}$}

=$\lim\limits_{n\rightarrow \infty}$(\textit{b}-\textit{a})$[a+\frac{b-a}{n}(1+\frac{1}{n})]$

=(\textit{b}-\textit{a})$(a+\frac{b}{2}-\frac{a}{2})$=$\frac{1}{2}$(\textit{b}${}^{2}$-\textit{a}${}^{2}$).

 

 知识:微积分定理

 难度:1

 题目:   $\int_{-1}^{1}$|\textit{x}|d\textit{x}等于(  )

A.$\int_{-1}^1$\textit{x}d\textit{x}       B.$\int_{-1}^{1}$d\textit{x}

C.$\int_{-1}^0$(-\textit{x})d\textit{x}+$\int_{0}^{1}$\textit{x}d\textit{x}    D.$\int_{-1}^{0}$\textit{x}d\textit{x}+$\int_{0}^{1}$(-\textit{x})d\textit{x}

 解析: $\mathrm{\because}$|\textit{x}|=$\left\{\begin{array}{l} x, (x\ge 0)\\ -x, (x<0) \end{array}\right.$

$\mathrm{\therefore}$$\int_{-1}^{1}$|\textit{x}|d\textit{x}=$\int_{-1}^{0}$|\textit{x}|d\textit{x}+$\int_{0}^{1}$|\textit{x}|d\textit{x}

=$\int_{-1}^{0}$(-\textit{x})d\textit{x}+$\int_{0}^{1}$\textit{x}d\textit{x},故应选C.

 答案: C

 

 知识:微积分定理

 难度:1

 题目:设\textit{f}(\textit{x})=$\left\{\begin{array}{l} x^2, (x\le x<0)\\2-x, (1\le x\le 2)\end{array}\right.$,则$\int_0^1$\textit{f}(\textit{x})d\textit{x}等于(  )

A.$\frac{3}{4}$        B.$\frac{4}{5}$

C.$\frac{5}{6}$        D.不存在

 解析: $\int_0^2$\textit{f}(\textit{x})d\textit{x}=$\int_0^1$\textit{x}${}^{2}$d\textit{x}+$\int_1^2$(2-\textit{x})d\textit{x}

取\textit{F}${}_{1}$(\textit{x})=\textit{x}${}^{3}$,\textit{F}${}_{2}$(\textit{x})=2\textit{x}-\textit{x}${}^{2}$,

则\textit{F}$'$${}_{1}$(\textit{x})=\textit{x}${}^{2}$,\textit{F}$'$${}_{2}$(\textit{x})=2-\textit{x}

$\mathrm{\therefore}$$\int_0^2$\textit{f}(\textit{x})d\textit{x}=\textit{F}${}_{1}$(1)-\textit{F}${}_{1}$(0)+\textit{F}${}_{2}$(2)-\textit{F}${}_{2}$(1)

=$\frac{1}{3}$-0+2$\mathrm{\times}$2-$\frac{1}{2}\mathrm{\times}$2${}^{2}$-$(2\times 1-\frac{1}{2}\times 1^2)$=$\frac{5}{6}$.故应选C.

 答案:C



 知识:微积分定理

 难度:1

 题目:$\int_a^b$\textit{f}$'$(3\textit{x})d\textit{x}=(  )

A.\textit{f}(\textit{b})-\textit{f}(\textit{a})      B.\textit{f}(3\textit{b})-\textit{f}(3\textit{a})

C.$\frac{1}{3}$[\textit{f}(3\textit{b})-\textit{f}(3\textit{a})]      D.3[\textit{f}(3\textit{b})-\textit{f}(3\textit{a})]

 解析: $\mathrm{\because}[\frac{1}{3}]f(3x)$$'$=\textit{f}$'$(3\textit{x})

$\mathrm{\therefore}$取\textit{F}(\textit{x})=$\frac{1}{3}$\textit{f}(3\textit{x}),则

$\int_a^b$\textit{f}$'$(3\textit{x})d\textit{x}=\textit{F}(\textit{b})-\textit{F}(\textit{a})=$\frac{1}{3}$[\textit{f}(3\textit{b})-\textit{f}(3\textit{a})].故应选C.

 答案: C



 知识:微积分定理

 难度:1

 题目:$\int_0^3$|\textit{x}${}^{2}$-4|d\textit{x}=(  )

A.$\frac{21}{3}$        B.$\frac{22}{3}$

C.$\frac{23}{3}$        D.$\frac{25}{3}$

 解析: $\int_{0}^{3}$|\textit{x}${}^{2}$-4|d\textit{x}=$\int_0^2$(4-\textit{x}${}^{2}$)d\textit{x}+$\int_2^3$(\textit{x}${}^{2}$-4)d\textit{x}

=$(4x-\frac{1}{3}x^3)|_0^2+(\frac{1}{3}x^3-4x)|_2^3=\frac{23}{3}$

 答案: C



 知识:微积分定理

 难度:1

 题目:$\int_0^{\frac{\pi}{3}}(1-2\sin^2 \frac{\theta}{2})$d$\theta$的值为(    )

A.$-\frac{\sqrt{3}}{2}$       B.$-\frac{1}{2}$

C. $\frac{1}{2}$       D.$\frac{\sqrt{3}}{2}$

 解析: $\mathrm{\because}$1-2sin${}^{2}\frac{\theta}{2}$=cos\textit{$\theta$}

$\therefore\int_0^{\frac{\pi}{3}}(1-2\sin^2\frac{\theta}{2})$d$\theta=\int_0^{\frac{\pi}{3}}\cos\theta$d$\theta$

$=\sin\theta|_0^{\frac{\pi}{3}}=\frac{\sqrt{3}}{2}$

 答案: D



 知识:微积分定理

 难度:1

 题目:函数\textit{F}(\textit{x})=$\int_0^x$cos\textit{t}d\textit{t}的导数是(  )

A.cos\textit{x}       B.sin\textit{x}

C.-cos\textit{x}       D.-sin\textit{x}

 解析:\textit{F}(\textit{x})=$\int_0^x$cos\textit{t}d\textit{t}=sin\textit{t}$|_0^x$=sin\textit{x}-sin0=sin\textit{x}.

所以\textit{F}$'$(\textit{x})=cos\textit{x},故应选A.

 答案: A



 知识:微积分定理

 难度:1

 题目:若$\int_0^k$(2\textit{x}-3\textit{x}${}^{2}$)d\textit{x}=0,则\textit{k}=(  )

A.0        B.1

C.0或1      D.以上都不对

 解析: $\int_0^k$(2\textit{x}-3\textit{x}${}^{2}$)d\textit{x}=(\textit{x}${}^{2}$-\textit{x}${}^{3}$)$|_0^k$=\textit{k}${}^{2}$-\textit{k}${}^{3}$=0,

$\mathrm{\therefore}$\textit{k}=0或1.

 答案: C



 知识:微积分定理

 难度:1

 题目:函数\textit{F}(\textit{x})=$\int_0^x$\textit{t}(\textit{t}-4)d\textit{t}在[-1,5]上(  )

A.有最大值0,无最小值

B.有最大值0和最小值$-\frac{32}{3}$

C.有最小值$-\frac{32}{3}$,无最大值

D.既无最大值也无最小值

 解析: \textit{F}(\textit{x})=$\int_0^x$(\textit{t}${}^{2}$-4\textit{t})d\textit{t}=$(\frac{1}{3}t^3-2t^2)|_0^x$=\textit{x}${}^{3}$-2\textit{x}${}^{2}$(-1$\mathrm{\le}$\textit{x}$\mathrm{\le}$5).

\textit{F}$'$(\textit{x})=\textit{x}${}^{2}$-4\textit{x},由\textit{F}$'$(\textit{x})=0得\textit{x}=0或\textit{x}=4,列表如下:

%\begin{tabular}{|p{0.4in}|p{0.5in}|p{0.5in}|p{0.4in}|p{0.5in}|p{0.4in}|} \hline 
%\textit{x} & (-1,0)\textit{} & 0\textit{} & (0,4)\textit{} & 4\textit{} & (4,5) \\ \hline 
%\textit{F}$'$(\textit{x})\textit{} & +\textit{} & 0\textit{} & -\textit{} & 0\textit{} & + \\ \hline 
%\textit{F}(\textit{x})\textit{} & $\nearrow$\textit{} & 极大值\textit{} & $\searrow$ & 极小值\textit{} & $\nearrow$ \\ \hline 
%\end{tabular}

\includegraphics*[width=11cm, height=2cm, keepaspectratio=false]{image}

可见极大值\textit{F}(0)=0,极小值\textit{F}(4)=$-\frac{32}{3}$.

又\textit{F}(-1)=$-\frac{7}{3}$,\textit{F}(5)=$-\frac{25}{3}$

$\mathrm{\therefore}$最大值为0,最小值为$-\frac{32}{3}$.

 答案: B



 知识:微积分定理

 难度:1

 题目:计算定积分:

①$\int_{-1}^{1}$\textit{x}${}^{2}$d\textit{x}=\_\_\_\_\_\_\_\_

②$\int_2^3(3x-\frac{2}{x^2})$d\textit{x}=\_\_\_\_\_\_\_\_

③$\int_0^2$|\textit{x}${}^{2}$-1|d\textit{x}=\_\_\_\_\_\_\_\_

④$\int_{-\frac{\pi}{2}}^{0}$|sin\textit{x}|d\textit{x}=\_\_\_\_\_\_\_\_

 解析: ①$\int_{-1}^{1}$\textit{x}${}^{2}$d\textit{x}=$\frac{1}{3}$\textit{x}${}^{3}|_{-1}^1$=$\frac{2}{3}$.

②$\int_2^3(3x-\frac{2}{x^2})$d\textit{x}=$(\frac{3}{2}x^2+\frac{2}{x})|_2^3$=$\frac{43}{6}$.

③$\int_0^2$|\textit{x}${}^{2}$-1|d\textit{x}=$\int_0^1$(1-\textit{x}${}^{2}$)d\textit{x}+$\int_1^2$(\textit{x}${}^{2}$-1)d\textit{x}

$=(x-\frac{1}{3}x^3)|_0^1+(\frac{1}{3}x^3-x)|_1^2=2$

④$\int_{-\frac{\pi}{2}}^{0}|\sin x|$d$x=\int_{-\frac{\pi}{2}}^{0}(-\sin x)$d$x=\cos x|_{-\frac{\pi}{2}}^0=1$


 答案: $\frac{2}{3}; \frac{43}{6}; 2; 1$



 知识:微积分定理

 难度:1

 题目:$\int_0^{\frac{\pi}{2}}(\sin\frac{x}{2}+\cos\frac{x}{2})^2$d$x$=\_\_\_

 解析:$\int_0^{\frac{\pi}{2}}(\sin\frac{x}{2}+\cos\frac{x}{2})^2$d$x=\int_0^{\frac{\pi}{2}}(1+\sin x)$d$x$

$=(x-\cos x)|_0^{\frac{\pi}{2}}=\frac{\pi}{2}+1$

 答案: $\frac{\pi}{2}+1$



 知识:微积分定理

 难度:1

 题目:已知\textit{f}(\textit{x})=3\textit{x}${}^{2}$+2\textit{x}+1,若$\int_{-1}^{1}$\textit{f}(\textit{x})d\textit{x}=2\textit{f}(\textit{a})成立,则\textit{a}=\_\_\_\_\_\_\_\_.

 解析: 由已知\textit{F}(\textit{x})=\textit{x}${}^{3}$+\textit{x}${}^{2}$+\textit{x},\textit{F}(1)=3,\textit{F}(-1)=-1,

$\mathrm{\therefore}\int_{-1}^{1}$\textit{f}(\textit{x})d\textit{x}=\textit{F}(1)-\textit{F}(-1)=4,

$\mathrm{\therefore}$2\textit{f}(\textit{a})=4,$\mathrm{\therefore}$\textit{f}(\textit{a})=2.

即3\textit{a}${}^{2}$+2\textit{a}+1=2.解得\textit{a}=-1或$\frac{1}{3}$.

 答案: -1或$\frac{1}{3}$



 知识:微积分定理

 难度:1

 题目:计算下列定积分:

(1)$\int_0^5$2\textit{x}d\textit{x};(2)$\int_0^1$(\textit{x}${}^{2}$-2\textit{x})d\textit{x};

(3)$\int_0^2$(4-2\textit{x})(4-\textit{x}${}^{2}$)d\textit{x};(4)$\int_1^2\frac{x^2+2x-3}{x}$d\textit{x}.

 解析:

 解: (1)$\int_0^5$2\textit{x}d\textit{x}=\textit{x}${}^{2}|_0^5$=25-0=25.

(2)$\int_0^1$(\textit{x}${}^{2}$-2\textit{x})d\textit{x}=$\int_0^1$\textit{x}${}^{2}$d\textit{x}-$\int_0^1$2\textit{x}d\textit{x}

=$\frac{1}{3}$\textit{x}${}^{3}|_0^1$-\textit{x}${}^{2}|_0^1$=$\frac{1}{3}$- 1=$-\frac{2}{3}$

(3)$\int_0^2$(4-2\textit{x})(4-\textit{x}${}^{2}$)d\textit{x}=$\int_0^2$(16-8\textit{x}-4\textit{x}${}^{2}$+2\textit{x}${}^{3}$)d\textit{x}

=$(16x-4x^2-\frac{4}{3}x^3+\frac{1}{2}x^4)|_0^2$

$32-16-\frac{32}{3}+8=\frac{40}{3}$.

(4)$\int_1^2\frac{x^2+2x-3}{x}$d\textit{x}=$\int_1^2(x+2-\frac{3}{x})$d\textit{x}

$=(\frac{1}{2}x^2+2x-3\ln x)|_1^2=\frac{7}{2}-3\ln 2$



 知识:微积分定理

 难度:1

 题目:计算下列定积分:

(1)$\int_{\frac{\pi}{6}}^{\frac{\pi}{4}}\cos 2x$d$x$

(2)$\int_2^3(\sqrt{x}+\frac{1}{\sqrt{x}})^2$d$x$

(3)$\int_0^{\frac{\pi}{2}}(3x+\sin x)$d$x$

(4)$\int_a^b e^x$d$x$

 解析:

 解: (1)取\textit{F}(\textit{x})=sin2\textit{x},则\textit{F}$'$(\textit{x})=cos2\textit{x}

$\therefore\int_{\frac{\pi}{6}}^{\frac{\pi}{4}}\cos 2x$d$x=F(\frac{\pi}{4})-F(\frac{\pi}{6})$

$=\frac{1}{2}(1-\frac{\sqrt{3}}{2})=\frac{1}{4}(2-\sqrt{3})$

(2)取\textit{F}(\textit{x})=+$\frac{x^2}{2}+\ln x+2x$,则

\textit{F}$'$(\textit{x})=\textit{x}+$\frac{1}{x}$+2.

$\mathrm{\therefore}\int_2^3(\sqrt{x}+\frac{1}{\sqrt{x}})^2$d\textit{x}=$\int_2^3(x+\frac{1}{x}+2)$d\textit{x}

=\textit{F}(3)-\textit{F}(2)

$=(\frac{9}{2}+\ln 3+6)-(\frac{1}{2}\times 4+\ln 2+4)$

=$\frac{9}{2}+\ln\frac{3}{2}$

(3)取\textit{F}(\textit{x})=$\frac{3}{2}$\textit{x}${}^{2}$-cos\textit{x},则\textit{F}$'$(\textit{x})=3\textit{x}+sin\textit{x}

$\therefore\int_0^{\frac{\pi}{2}}(3x+\sin x)$d$x=F(\frac{\pi}{2})-F(0)=\frac{3}{8}\pi^2+1$

(4)取$F(x)=e^x$, 则$F'(x)=e^x$, $\therefore\int_a^b e^x$d$x=e^x|_a^b=e^b-e^a$

 知识:微积分定理

 难度:1

 题目:计算下列定积分:

(1)$\int_{-4}^{0}$|\textit{x}+2|d\textit{x};

(2)已知\textit{f}(\textit{x})=$\left\{\begin{array}{l}|x-1|, (0\le x\le 2)\\0, (x<0\text{或}x>2)\end{array}\right.$,求$\int_{-1}^{3}$\textit{f}(\textit{x})d\textit{x}的值.

 解析:

 解: (1)$\mathrm{\because}$\textit{f}(\textit{x})=|\textit{x}+2|=$\left\{\begin{array}{l}x+2, x\ge -2\\-x-2, x<-2\end{array}\right.$

$\mathrm{\therefore}$$\int_{-4}^{0}$|\textit{x}+2|d\textit{x}=-$\int_{-4}^{-2}$(\textit{x}+2)d\textit{x}+$\int_{-2}^{0}$(\textit{x}+2)d\textit{x}

$=-(\frac{1}{2}x^2+2x)|_{-4}^{-2}+(\frac{1}{2}x^2+2x)|_{-2}^0=2+2=4$


(2)$\mathrm{\because}$\textit{f}(\textit{x})=$\left\{\begin{array}{l}|x-1|, (0\le x\le 2)\\0, (x<0\text{或}x>2)\end{array}\right.$=$\left\{\begin{array}{l}
0, (x<0)\\1-x, (0\le x<1)\\x-1, (1\le x \le 2)\\0, (x>2)\end{array}\right.$

$\mathrm{\therefore}$$\int_{-1}^{3}$\textit{f}(\textit{x})d\textit{x}=$\int_{-1}^{0}$\textit{f}(\textit{x})d\textit{x}+$\int_{0}^{1}$\textit{f}(\textit{x})d\textit{x}+$\int_{1}^{2}$\textit{f}(\textit{x})d\textit{x}+$\int_{2}^{3}$\textit{f}(\textit{x})d\textit{x}=$\int_{0}^{1}$(1-\textit{x})d\textit{x}+$\int_{1}^{2}$(\textit{x}-1)d\textit{x}

$=(x-\frac{x^2}{2})|_0^1+(\frac{x^2}{2}-x)|_1^2$

$=\frac{1}{2}+\frac{1}{2}=1$



 知识:微积分定理

 难度:1

 题目:(1)已知\textit{f}(\textit{a})=$\int_0^1$(2\textit{ax}${}^{2}$-\textit{a}${}^{2}$\textit{x})d\textit{x},求\textit{f}(\textit{a})的最大值;

(2)已知\textit{f}(\textit{x})=\textit{ax}${}^{2}$+\textit{bx}+\textit{c}(\textit{a}$\mathrm{\neq}$0),且\textit{f}(-1)=2,\textit{f}$'$(0)=0,$\int_{0}^{1}$\textit{f}(\textit{x})d\textit{x}=-2,求\textit{a},\textit{b},\textit{c}的值.

 解析:

 解: (1)取\textit{F}(\textit{x})=$\frac{2}{3}$\textit{ax}${}^{3}$-$\frac{1}{2}$\textit{a}${}^{2}$\textit{x}${}^{2}$

则\textit{F}$'$(\textit{x})=2\textit{ax}${}^{2}$-\textit{a}${}^{2}$\textit{x}

$\mathrm{\therefore}$\textit{f}(\textit{a})=$\int_0^1$(2\textit{ax}${}^{2}$-\textit{a}${}^{2}$\textit{x})d\textit{x}

=\textit{F}(1)-\textit{F}(0)=$\frac{2}{3}$\textit{a}-$\frac{1}{2}$\textit{a}${}^{2}$

$=-\frac{1}{2}(a-\frac{2}{3})^2+\frac{2}{9}$

$\mathrm{\therefore}$当\textit{a}=$\frac{2}{3}$时,\textit{f}(\textit{a})有最大值$\frac{2}{9}$.

(2)$\mathrm{\because}$\textit{f}(-1)=2,$\mathrm{\therefore}$\textit{a}-\textit{b}+\textit{c}=2①

又$\mathrm{\because}$\textit{f}$'$(\textit{x})=2\textit{ax}+\textit{b},$\mathrm{\therefore}$\textit{f}$'$(0)=\textit{b}=0②

而$\int_0^1$\textit{f}(\textit{x})d\textit{x}=$\int_0^1$(\textit{ax}${}^{2}$+\textit{bx}+\textit{c})d\textit{x}

取\textit{F}(\textit{x})=$\frac{1}{3}$\textit{ax}${}^{3}$+$\frac{1}{2}$\textit{bx}${}^{2}$+\textit{cx}

则\textit{F}$'$(\textit{x})=\textit{ax}${}^{2}$+\textit{bx}+\textit{c}

$\mathrm{\therefore}\int_0^1$\textit{f}(\textit{x})d\textit{x}=\textit{F}(1)-\textit{F}(0)=$\frac{1}{3}$\textit{a}+$\frac{1}{2}$\textit{b}+\textit{c}=-2③

解①②③得\textit{a}=6,\textit{b}=0,\textit{c}=-4.

 

 

 

 知识:定积分在几何上的应用,定积分在物理上的应用

 难度:1

 题目:如图所示,阴影部分的面积为(  )

\includegraphics*[width=1.17in, height=0.90in, keepaspectratio=false]{image33}

A.$\int_a^b$\textit{f}(\textit{x})d\textit{x}         B.$\int_a^b$\textit{g}(\textit{x})d\textit{x}

C.$\int_a^b$[\textit{f}(\textit{x})-\textit{g}(\textit{x})]d\textit{x}     D.$\int_a^b$[\textit{g}(\textit{x})-\textit{f}(\textit{x})]d\textit{x}

 解析: 由题图易知,当\textit{x}$\mathrm{\in}$[\textit{a},\textit{b}]时,\textit{f}(\textit{x})$\mathrm{>}$\textit{g}(\textit{x}),所以阴影部分的面积为[\textit{f}(\textit{x})-\textit{g}(\textit{x})]d\textit{x}.

 答案: C



 知识:定积分在几何上的应用,定积分在物理上的应用

 难度:1

 题目:如图所示,阴影部分的面积是(  )

\includegraphics*[width=1.18in, height=1.17in, keepaspectratio=false]{image34}

A.2$\sqrt{3}$       B.2-$\sqrt{3}$

C.$\frac{32}{3}$        D.$\frac{35}{3}$

 解析: \textit{S}=$\int_{-3}^{1}$(3-\textit{x}${}^{2}$-2\textit{x})d\textit{x}

即\textit{F}(\textit{x})=3\textit{x}-$\frac{1}{3}$\textit{x}${}^{3}$-\textit{x}${}^{2}$,

则\textit{F}(1)=$3-1-\frac{1}{3}=\frac{5}{3}$

\textit{F}(-3)=-9-9+9=-9.

$\mathrm{\therefore}$\textit{S}=\textit{F}(1)-\textit{F}(-3)$=\frac{5}{3}+9=\frac{32}{3}$故应选C.

 答案: C



 知识:定积分在几何上的应用,定积分在物理上的应用

 难度:1

 题目:由曲线\textit{y}=\textit{x}${}^{2}$-1、直线\textit{x}=0、\textit{x}=2和\textit{x}轴围成的封闭图形的面积(如图)是(  )

A.$\int_0^2$(\textit{x}${}^{2}$-1)d\textit{x}

\includegraphics*[width=1.03in, height=1.07in, keepaspectratio=false]{image35}B.|$\int_0^2$(\textit{x}${}^{2}$-1)d\textit{x}|

C.$\int_0^2$|\textit{x}${}^{2}$-1|d\textit{x}

D.$\int_0^1$(\textit{x}${}^{2}$-1)d\textit{x}+$\int_1^2$(\textit{x}${}^{2}$-1)d\textit{x}

 解析:\textit{y}=|\textit{x}${}^{2}$-1|将\textit{x}轴下方阴影反折到\textit{x}轴上方,其定积分为正,故应选C.

 答案: C



 知识:定积分在几何上的应用,定积分在物理上的应用

 难度:1

 题目:设\textit{f}(\textit{x})在[\textit{a},\textit{b}]上连续,则曲线\textit{f}(\textit{x})与直线\textit{x}=\textit{a},\textit{x}=\textit{b},\textit{y}=0围成图形的面积为(  )

A.$\int_a^b$\textit{f}(\textit{x})d\textit{x}       B.|$\int_a^b$\textit{f}(\textit{x})d\textit{x}|

C.$\int_a^b$|\textit{f}(\textit{x})|d\textit{x}       D.以上都不对

 解析: 当\textit{f}(\textit{x})在[\textit{a},\textit{b}]上满足\textit{f}(\textit{x})$\mathrm{<}$0时,\textit{f}(\textit{x})d\textit{x}$\mathrm{<}$0,排除A;当阴影有在\textit{x}轴上方也有在\textit{x}轴下方时,\textit{f}(\textit{x})d\textit{x}是两面积之差,排除B;无论什么情况C对,故应选C.

 答案:C

 

 知识:定积分在几何上的应用,定积分在物理上的应用

 难度:1

 题目:曲线\textit{y}=1-$\frac{16}{81}$\textit{x}${}^{2}$与\textit{x}轴所围图形的面积是(  )

A.4        B.3    

C.2        D.$\frac{5}{2}$

 解析:曲线与\textit{x}轴的交点为$(-\frac{9}{4},0)$,$(\frac{9}{4},0)$

$\therefore$ 所求面积$S=\int_{-\frac{9}{4}}^{\frac{9}{4}}(1-\frac{16}{81}x^2)dx$
=$(x-\frac{16}{243}x^3)|_{-\frac{9}{4}}^{\frac{9}{4}}$=$[\frac{9}{4}-\frac{16}{243}\times(\frac{9}{4})^3]\times2$=3
故应选B.

 答案: B





 知识:定积分在几何上的应用,定积分在物理上的应用

 难度:1

 题目:一物体以速度\textit{v}=(3\textit{t}${}^{2}$+2\textit{t})m/s做直线运动,则它在\textit{t}=0s到\textit{t}=3s时间段内的位移是(  )

A.31m       B.36m   

C.38m       D.40m

 解析: \textit{S}=$\int_{0}^{2}$(3\textit{t}${}^{2}$+2\textit{t})d\textit{t}=(\textit{t}${}^{3}$+\textit{t}${}^{2}$)$|_0^3$=3${}^{3}$+3${}^{2}$=36(m),故应选B.

 答案: B



 知识:定积分在几何上的应用,定积分在物理上的应用

 难度:1

 题目: (2010·山东理,7)由曲线\textit{y}=\textit{x}${}^{2}$,\textit{y}=\textit{x}${}^{3}$围成的封闭图形面积为(  )

A. $\frac{1}{12}$        B.  $\frac{1}{4}$  

C. $\frac{1}{3}$          D.  $\frac{7}{12}$

 解析: 由$\left\{\begin{array}{r}y=x^2\\y=x^3\end{array} \right.$得交点为(0,0),(1,1).

$\mathrm{\therefore}$\textit{S}=$\int_{0}^{1}$(\textit{x}${}^{2}$-\textit{x}${}^{3}$)d\textit{x}=$(\frac{1}{3}x^3-\frac{1}{4}x^4)|_0^1$=$\frac{1}{12}$.

 答案: A



 知识:定积分在几何上的应用,定积分在物理上的应用

 难度:1

 题目:一物体在力\textit{F}(\textit{x})=4\textit{x}-1(单位:N)的作用下,沿着与力\textit{F}相同的方向,从\textit{x}=1运动到\textit{x}=3处(单位:m),则力\textit{F}(\textit{x})所做的功为(  )

A.8J         B.10J   

C.12J       D.14J

 解析: 由变力做功公式有:\textit{W}=$\int_{1}^{3}$(4\textit{x}-1)d\textit{x}=(2\textit{x}${}^{2}$-\textit{x})$|_1^3$=14(J),故应选D.

 答案:D



 知识:定积分在几何上的应用,定积分在物理上的应用

 难度:1

 题目:若某产品一天内的产量(单位:百件)是时间\textit{t}的函数,若已知产量的变化率为\textit{a}=$\frac{3}{\sqrt{6}t}$,那么从3小时到6小时期间内的产量为(  )

A.$\frac{1}{2}$       B.$3-\frac{3}{2}\sqrt{2}$

C.$6+3\sqrt{2}$     D.$6-3\sqrt{2}$

 解析: $\int_{3}^{6}\frac{3}{\sqrt{6}t}$dt=$\frac{6}{\sqrt{6}}\sqrt{t}|_3^6$=$6-3\sqrt{2}$,故应选D.

 答案: D



 知识:定积分在几何上的应用,定积分在物理上的应用

 难度:1

 题目:过原点的直线\textit{l}与抛物线\textit{y}=\textit{x}${}^{2}$-2\textit{ax}(\textit{a}$\mathrm{>}$0)所围成的图形面积为$\frac{9}{2}$\textit{a}${}^{3}$,则直线\textit{l}的方程为(  )

A.\textit{y}=$\mathrm{\pm}$\textit{ax}       B.\textit{y}=\textit{ax}

C.\textit{y}=-\textit{ax}      D.\textit{y}=-5\textit{ax}

 解析: 设直线\textit{l}的方程为\textit{y}=\textit{kx},

由$\left\{\begin{array}{r}y=kx\\y=x^2-2ax\end{array} \right.$得交点坐标为(0,0),(2\textit{a}+\textit{k,}2\textit{ak}+\textit{k}${}^{2}$)

图形面积\textit{S}=$\int_{0}^{2a+k}$[\textit{kx}-(\textit{x}${}^{2}$-2\textit{ax})]d\textit{x}

=$(\frac{k+2a}{2}x^2-\frac{x^3}{3})|_0^{2a+k}$

=$\frac{(k+2a)^3}{2}$-$\frac{(2a+k)^3}{3}$=$\frac{(2a+k)^3}{6}$=$\frac{9}{2}$\textit{a}${}^{3}$

$\mathrm{\therefore}$\textit{k}=\textit{a},$\mathrm{\therefore}$\textit{l}的方程为\textit{y}=\textit{ax},故应选B.

 答案: B



 知识:定积分在几何上的应用,定积分在物理上的应用

 难度:1

 题目:由曲线\textit{y}${}^{2}$=2\textit{x},\textit{y}=\textit{x}-4所围图形的面积是\_\_\_\_\_\_\_\_.

 解析: 如图,为了确定图形的范围,先求出这两条曲线交点的坐标,解方程组$\left\{\begin{array}{r}y^2=2x\\y=x-4\end{array} \right.$得交点坐标为(2,-2),(8,4).

因此所求图形的面积\textit{S}=$\int_{-2}^{4}$(\textit{y}+4-$\frac{y^2}{2}$)d\textit{y}

\includegraphics*[width=1.38in, height=1.35in, keepaspectratio=false]{image37}

取\textit{F}(\textit{y})=$\frac{1}{2}$\textit{y}${}^{2}$+4\textit{y}-$\frac{y^2}{6}$,则\textit{F}$\mathrm{\prime}$(\textit{y})=\textit{y}+4-$\frac{y^2}{2}$,从而\textit{S}=\textit{F}(4)-\textit{F}(-2)=18.

 答案: 18



 知识:定积分在几何上的应用,定积分在物理上的应用

 难度:1

 题目:一物体沿直线以\textit{v}=$\sqrt{1+t}$m/s的速度运动,该物体运动开始后10s内所经过的路程是\_\_\_\_\_\_\_\_.

解析:$S=\int_{0}^{10}\sqrt{1+t}dt=\frac{2}{3}(1+t)^{\frac{3}{2}}|_0^{10}=\frac{2}{3}(11^{\frac{3}{2}}-1)$

答案:$\frac{2}{3}(11^{\frac{3}{2}}-1)$

 知识:定积分在几何上的应用,定积分在物理上的应用

 难度:1

 题目:由两条曲线\textit{y}=\textit{x}${}^{2}$,\textit{y}=$\frac{1}{4}$\textit{x}${}^{2}$与直线\textit{y}=1围成平面区域的面积是\_\_\_\_\_\_\_\_.

 解析:如图,\textit{y}=1与\textit{y}=\textit{x}${}^{2}$交点\textit{A}(1,1),\textit{y}=1与\textit{y}=$\frac{x^2}{4}$交点\textit{B}(2,1),由对称性可知面积\textit{S}=2($\int_{0}^{1}$\textit{x}${}^{2}$d\textit{x}+$\int_{1}^{2}$d\textit{x}-$\int_{0}^{2}\frac{1}{4}$\textit{x}${}^{2}$d\textit{x})=$\frac{4}{3}$.

\includegraphics*[width=2.10in, height=1.44in, keepaspectratio=false]{image39}

 答案: $\frac{4}{3}$

 知识:定积分在几何上的应用,定积分在物理上的应用

 难度:1

 题目:一变速运动物体的运动速度\textit{v}(\textit{t})=$\left\{\begin{array}{r}2t(0\le t\le1)\\a^t(1\le t\le2)\\\frac{b}{t}(2\le t\le e)\end{array} \right.$
则该物体在0$\mathrm{\le}$\textit{t}$\mathrm{\le}$\textit{e}时间段内运动的路程为(速度单位:m/s,时间单位:s)\_\_\_\_\_\_\_\_\_\_\_\_\_\_\_\_\_\_\_\_\_\_.

 解析: $\mathrm{\because}$0$\mathrm{\le}$\textit{t}$\mathrm{\le}$1时,\textit{v}(\textit{t})=2\textit{t},$\mathrm{\therefore}$\textit{v}(1)=2;

又1$\mathrm{\le}$\textit{t}$\mathrm{\le}$2时,\textit{v}(\textit{t})=\textit{a${}^{t}$},

$\mathrm{\therefore}$\textit{v}(1)=\textit{a}=2,\textit{v}(2)=\textit{a}${}^{2}$=2${}^{2}$=4;

又2$\mathrm{\le}$\textit{t}$\mathrm{\le}$\textit{e}时,\textit{v}(\textit{t})=$\frac{b}{t}$,

$\mathrm{\therefore}$\textit{v}(2)=$\frac{b}{2}$=4,$\mathrm{\therefore}$\textit{b}=8.

$\mathrm{\therefore}$路程为\textit{S}=$\int_{0}^{1}$2\textit{t}d\textit{t}+$\int_{1}^{2}$2\textit{${}^{t}$}d\textit{t}+$\int_{2}^{e}\frac{8}{t}$d\textit{t}=9-8ln2+$\frac{2}{ln2}$ .

 答案: 9-8ln2+$\frac{2}{ln2}$ 



 知识:定积分在几何上的应用,定积分在物理上的应用

 难度:1

 题目:计算曲线\textit{y}=\textit{x}${}^{2}$-2\textit{x}+3与直线\textit{y}=\textit{x}+3所围图形的面积.

 解析:

 解: 由$\left\{\begin{array}{r}y=x+3\\y=x^2-2x+3\end{array} \right.$解得\textit{x}=0及\textit{x}=3.

\includegraphics*[width=1.52in, height=1.43in, keepaspectratio=false]{image40}

从而所求图形的面积

\textit{S}=$\int_{0}^{3}$(\textit{x}+3)d\textit{x}-$\int_{0}^{3}$(\textit{x}${}^{2}$-2\textit{x}+3)d\textit{x}

=$\int_{0}^{3}$[(\textit{x}+3)-(\textit{x}${}^{2}$-2\textit{x}+3)]d\textit{x}

=$\int_{0}^{3}$(-\textit{x}${}^{2}$+3\textit{x})d\textit{x}

=$(-\frac{1}{3}x^3+\frac{3}{2}x^2)|_0^3$=$\frac{9}{2}$.



 知识:定积分在几何上的应用,定积分在物理上的应用

 难度:1

 题目:设\textit{y}=\textit{f}(\textit{x})是二次函数,方程\textit{f}(\textit{x})=0有两个相等的实根,且\textit{f}$\mathrm{\prime}$(\textit{x})=2\textit{x}+2.

(1)求\textit{y}=\textit{f}(\textit{x})的表达式;

(2)若直线\textit{x}=-\textit{t}(0<\textit{t}<1)把\textit{y}=\textit{f}(\textit{x})的图象与两坐标轴所围成图形的面积二等分,求\textit{t}的值.

 解析:

 解: (1)设\textit{f}(\textit{x})=\textit{ax}${}^{2}$+\textit{bx}+\textit{c}(\textit{a}$\mathrm{\neq}$0),则\textit{f}$\mathrm{\prime}$(\textit{x})=2\textit{ax}+\textit{b},

又已知\textit{f}$\mathrm{\prime}$(\textit{x})=2\textit{x}+2,$\mathrm{\therefore}$\textit{a}=1,\textit{b}=2,

$\mathrm{\therefore}$\textit{f}(\textit{x})=\textit{x}${}^{2}$+2\textit{x}+\textit{c}.

又方程\textit{f}(\textit{x})=0有两个相等实根.

$\mathrm{\therefore}$判别式$\Delta$=4-4\textit{c}=0,即\textit{c}=1.

故\textit{f}(\textit{x})=\textit{x}${}^{2}$+2\textit{x}+1.

(2)依题意有$\int_{-1}^{-t}$(\textit{x}${}^{2}$+2\textit{x}+1)d\textit{x}=$\int_{-t}^{0}$(\textit{x}${}^{2}$+2\textit{x}+1)d\textit{x},

$\mathrm{\therefore}$$(\frac{1}{3}x^3+x^2+x)|_{-1}^{-t}$=$(\frac{1}{3}x^3+x^2+x)|_{-t}^0$

即$-\frac{1}{3}$\textit{t}${}^{3}$+\textit{t}${}^{2}$-\textit{t}+$\frac{1}{3}$=$\frac{1}{3}$\textit{t}${}^{3}$-\textit{t}${}^{2}$+\textit{t}.

$\mathrm{\therefore}$2\textit{t}${}^{3}$-6\textit{t}${}^{2}$+6\textit{t}-1=0,

$\mathrm{\therefore}$2(\textit{t}-1)${}^{3}$=-1,$\mathrm{\therefore}$\textit{t}=$1-\frac{1}{\sqrt[3]{2}}$ .



 知识:定积分在几何上的应用,定积分在物理上的应用

 难度:1

 题目:\textit{A}、\textit{B}两站相距7.2km,一辆电车从\textit{A}站开往\textit{B}站,电车开出\textit{t}s后到达途中\textit{C}点,这一段速度为1.2\textit{t}(m/s),到\textit{C}点的速度达24m/s,从\textit{C}点到\textit{B}站前的\textit{D}点以等速行驶,从\textit{D}点开始刹车,经\textit{t}s后,速度为(24-1.2\textit{t})m/s,在\textit{B}点恰好停车,试求:

(1)\textit{A}、\textit{C}间的距离;

(2)\textit{B}、\textit{D}间的距离;

(3)电车从\textit{A}站到\textit{B}站所需的时间.

 解析:

 解: (1)设\textit{A}到\textit{C}经过\textit{t}${}_{1}$s,

由1.2\textit{t}=24得\textit{t}${}_{1}$=20(s),

所以\textit{AC}=$\int_{0}^{20}$1.2\textit{t}d\textit{t}=0.6\textit{t}${}^{2}$=240(m).

(2)设从\textit{D}$\mathrm{\to}$\textit{B}经过\textit{t}${}_{2}$s,

由24-1.2\textit{t}${}_{2}$=0得\textit{t}${}_{2}$=20(s),

所以\textit{DB}=$\int_{0}^{20}$(24-1.2\textit{t})d\textit{t}=240(m).

(3)\textit{CD}=7200-2$\mathrm{\times}$240=6720(m).

从\textit{C}到\textit{D}的时间为\textit{t}${}_{3}$=$\frac{6720}{24}$=280(s).

于是所求时间为20+280+20=320(s).



 知识:定积分在几何上的应用,定积分在物理上的应用

 难度:1

 题目:在曲线\textit{y}=\textit{x}${}^{2}$(\textit{x}$\mathrm{\ge}$0)上某一点\textit{A}处作一切线使之与曲线以及\textit{x}轴所围成的面积为$\frac{1}{12}$,试求:

(1)切点\textit{A}的坐标;

(2)过切点\textit{A}的切线方程.

 解析: 

 解:如图所示,设切点\textit{A}(\textit{x}${}_{0}$,\textit{y}${}_{0}$),由\textit{y}$\mathrm{\prime}$=2\textit{x},过\textit{A}点的切线方程为\textit{y}-\textit{y}${}_{0}$=2\textit{x}${}_{0}$(\textit{x}-\textit{x}${}_{0}$),

即\textit{y}=2\textit{x}${}_{0}$\textit{x}-$x_0^2$.

令\textit{y}=0得\textit{x}=$\frac{x_0}{2}$,即\textit{C}$(\frac{x_0}{2},0)$.

设由曲线和过\textit{A}点的切线及\textit{x}轴所围成图形的面积为\textit{S},

\includegraphics*[width=1.04in, height=1.05in, keepaspectratio=false]{image41}\textit{S}=\textit{S}${}_{\textrm{曲}\textrm{边}}$${}_{\mathrm{\vartriangle }}$\textit{${}_{AOB}$}-\textit{S}${}_{\mathrm{\vartriangle }}$\textit{${}_{ABC}$}.

\textit{S}${}_{\textrm{曲}\textrm{边}}$${}_{\mathrm{\vartriangle }}$\textit{${}_{AOB}$}=$\int$\textit{x}${}_{00}$\textit{x}${}^{2}$d\textit{x}=$\frac{1}{3}$\textit{x},

\textit{S}${}_{\mathrm{\vartriangle }}$\textit{${}_{ABC}$}=$\frac{1}{2}$|\textit{BC}|·|\textit{AB}|

=$\frac{1}{2}(x_0-\frac{x_0}{2})\cdot x_0^2$=$\frac{1}{4}x_0^3$,

即\textit{S}=$\frac{1}{3}x_0^3-\frac{1}{4}x_0^3$=$\frac{1}{12}x_0^3$=$\frac{1}{12}$.

所以\textit{x}${}_{0}$=1,从而切点\textit{A}(1,1),切线方程为\textit{y}=2\textit{x}-1.


\end{document}

