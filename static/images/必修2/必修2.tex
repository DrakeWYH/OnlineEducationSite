% Generated by GrindEQ Word-to-LaTeX 
\documentclass{article} %%% use \documentstyle for old LaTeX compilers

\usepackage[english]{babel} %%% 'french', 'german', 'spanish', 'danish', etc.
\usepackage{amssymb}
\usepackage{amsmath}
\usepackage{txfonts}
\usepackage{mathdots}
\usepackage[classicReIm]{kpfonts}
\usepackage[dvips]{graphicx} %%% use 'pdftex' instead of 'dvips' for PDF output
\usepackage{ctex}
\usepackage{comment}
% You can include more LaTeX packages here 


\begin{document}
%\selectlanguage{english} %%% remove comment delimiter ('%') and select language if required


知识:柱、锥、台、球的结构特征

难度:1

题目:下面多面体中,是棱柱的有( )

\includegraphics*[width=2.65in, height=0.67in, keepaspectratio=false]{image3}

A.1个   B.2个   C.3个   D.4个

解析:根据棱柱的定义进行判定知,这4个图都满足.

答案:D

知识:柱、锥、台、球的结构特征

难度:1

题目:下列说法正确的是( )

A.有2个面平行,其余各面都是梯形的几何体是棱台

B.多面体至少有3个面

C.各侧面都是正方形的四棱柱一定是正方体

D.九棱柱有9条侧棱,9个侧面,侧面为平行四边形

解析:选项A错误,反例如图1;一个多面体至少有4个面,如三棱锥有4个面,不存在有3个面的多面体,所以选项B错误;选项C错误,反例如图2,上、下底面是全等的菱形,各侧面是全等的正方形,它不是正方体;根据棱柱的定义,知选项D正确.

\includegraphics*[width=2.37in, height=1.27in, keepaspectratio=false]{image4}

答案:D

知识:棱柱、棱锥、棱台的结构特征

难度:1

题目:下列说法中正确的是( )

A.所有的棱柱都有一个底面   B.棱柱的顶点至少有6个

C.棱柱的侧棱至少有4条   D.棱柱的棱至少有4条

解析:棱柱有两个底面,所以A项不正确;棱柱底面的边数至少是3,则在棱柱中,三棱柱的顶点数至少是6,三棱柱的侧棱数至少是3,三棱柱的棱数至少是9,所以C、D项不正确,B项正确.

答案:B

知识:棱柱、棱锥、棱台的结构特征

难度:1

题目:下列图形经过折叠可以围成一个棱柱的是( )

\includegraphics*[width=2.45in, height=1.64in, keepaspectratio=false]{image5}

解析:A、B、C中底面图形的边数与侧面的个数不一致,故不能围成棱柱.故选D.

答案:D


知识:棱柱、棱锥、棱台的结构特征

难度:1

题目:观察如图所示的四个几何体,其中判断不正确的是( )

\includegraphics*[width=2.25in, height=2.26in, keepaspectratio=false]{image6}

A.①是棱柱   B.②不是棱锥 C.③不是棱锥   D.④是棱台

解析:①是棱柱,②是棱锥,③不是棱锥,④是棱台,故选B.

答案:B

知识:棱柱、棱锥、棱台的结构特征

难度:1

题目:用一个平面去截一个三棱锥,截面形状是( )

A.四边形    B.三角形

C.三角形或四边形   D.不可能为四边形

解析:按如图①所示用一个平面去截三棱锥,截面是三角形;按如图②所示用一个平面去截三棱锥,截面是四边形.

\includegraphics*[width=1.87in, height=1.05in, keepaspectratio=false]{image7}

答案:C

知识:棱柱、棱锥、棱台的结构特征

难度:1

题目:八棱锥的侧面个数是\_\_\_\_.

解析:八棱锥有8个侧面.

答案:8

知识:棱柱、棱锥、棱台的结构特征

难度:1

题目:下列说法正确的是\_\_\_\_.

①一个棱锥至少有四个面;

②如果四棱锥的底面是正方形,那么这个四棱锥的四条侧棱都相等;

③五棱锥只有五条棱;

④用与底面平行的平面去截三棱锥,得到的截面三角形和底面三角形相似.

解析:①正确.②不正确.四棱锥的底面是正方形,它的侧棱可以相等.也可以不等.③不正确.五棱锥除了五条侧棱外,还有五条底边,故共10条棱.④正确.

答案:①④


知识:棱柱、棱锥、棱台的结构特征

难度:1

题目:判断如图所示的几何体是不是棱台?为什么?

\includegraphics*[width=2.76in, height=1.15in, keepaspectratio=false]{image8}

解析:见答案。

答案:①②③都不是棱台,因为①和③都不是由棱锥所截得的,故①③都不是棱台,虽然②是由棱锥所截得的,但截面不和底面平行,故不是棱台,只有用平行于棱锥底面的平面去截棱锥,底面与截面之间的部分才是棱台.

知识:棱柱、棱锥、棱台的结构特征

难度:2

题目:(2016嘉峪关一中高一检测)下面说法正确的是( )

A.棱锥的侧面不一定是三角形

B.棱柱的各侧棱长不一定相等

C.棱台的各侧棱延长必交于一点

D.用一个平面截棱锥,得到两个几何体,一个是棱锥,另一个是棱台

解析:棱台的各侧棱延长后必交于一点,故选C.

答案:C

知识:棱柱、棱锥、棱台的结构特征

难度:2

题目:以三棱台的顶点为三棱锥的顶点,这样可以把一个三棱台分成三棱锥的个数为(  )

A.1   B.2   C.3   D.4

解析: 如图所示,在三棱台\textit{ABC}-\textit{A}${}_{1}$\textit{B}${}_{1}$\textit{C}${}_{1}$中,分别连接\textit{A}${}_{1}$\textit{B},\textit{A}${}_{1}$\textit{C},\textit{BC}${}_{1}$,则将三棱台分成3个三棱锥,即三棱锥\textit{A}-\textit{A}${}_{1}$\textit{BC},\textit{B}${}_{1}$-\textit{A}${}_{1}$\textit{BC}${}_{1}$,\textit{C}-\textit{A}${}_{1}$\textit{BC}${}_{1}$.

\includegraphics*[width=1.29in, height=0.96in, keepaspectratio=false]{image9}

答案:C

知识:棱柱、棱锥、棱台的结构特征

难度:2

题目:(2016·日照高一检测)如图,将装有水的长方体水槽固定底面一边后倾斜一个小角度,则倾斜后水槽中的水形成的几何体是( )

\includegraphics*[width=0.90in, height=1.22in, keepaspectratio=false]{image10}

A.棱柱  B.棱台

C.棱柱与棱锥的组合体 D.不能确定

解析:倾斜后水槽中的水形成的几何体是棱柱.

答案:A

知识:棱柱、棱锥、棱台的结构特征

难度:2

题目:五棱柱中,不同在任何侧面且不同在任何底面的两顶点的连线称为它的对角线,那么一个五棱柱的对角线共有\_\_\_\_\_条.

解析:在上底面选一个顶点,同时在下底选一个顶点,且这两个顶点不在同一侧面上,这样上底面每个顶点对应两条对角线,所以共有10条.

答案:10


知识:棱柱、棱锥、棱台的结构特征

难度:2

题目:在正方体上任意选择4个顶点,它们可能是如下各种几何形体的4个顶点,这些几何形体是\_\_\_\_(写出所有正确结论的编号).

①矩形;②不是矩形的平行四边形;③有三个面为等腰直角三角形,有一个面为等边三角形的四面体;④每个面都是等边三角形的四面体;⑤每个面都是直角三角形的四面体.

解析:在如图正方体\textit{ABCD}-\textit{A}${}_{1}$\textit{B}${}_{1}$\textit{C}${}_{1}$\textit{D}${}_{1}$中,若所取四点共面,则只能是正方体的表面或对角面.

\includegraphics*[width=1.19in, height=1.13in, keepaspectratio=false]{image11}

即正方形或长方形,$\mathrm{\therefore}$①正确,②错误.

棱锥\textit{A}-\textit{BDA}${}_{1}$符合③,$\mathrm{\therefore}$③正确;

棱锥\textit{A}${}_{1}$-\textit{BDC}${}_{1}$符合④,$\mathrm{\therefore}$④正确;

棱锥\textit{A}-\textit{A}${}_{1}$\textit{B}${}_{1}$\textit{C}${}_{1}$符合⑤,$\mathrm{\therefore}$⑤正确.

答案:①③④⑤

知识:棱柱、棱锥、棱台的结构特征

难度:2

题目:如图所示的几何体中,所有棱长都相等,分析此几何体的构成?有几个面、几个顶点、几条棱?

\includegraphics*[width=1.35in, height=1.31in, keepaspectratio=false]{image12}

解析:解析见答案

答案:这个几何体是由两个同底面的四棱锥组合而成的八面体,有8个面,都是全等的正三角形;有6个顶点;有12条棱.


知识:棱柱、棱锥、棱台的结构特征

难度:3

题目:一个几何体的表面展开平面图如图.

\includegraphics*[width=1.31in, height=1.62in, keepaspectratio=false]{image13}

(1)该几何体是哪种几何体;

(2)该几何体中与``祝''字面相对的是哪个面?与``你''字面相对的是哪个面?

解析:

答案:(1)该几何体是四棱台;

(2)与``祝''相对的面是``前'',与``你''相对的面是``程''.

知识:棱柱、棱锥、棱台的结构特征

难度:3

题目:根据如图所示的几何体的表面展开图,画出立体图形.

\includegraphics*[width=2.52in, height=1.28in, keepaspectratio=false]{image14}

解析:

答案:图1是以\textit{ABCD}为底面,\textit{P}为顶点的四棱锥.

图2是以\textit{ABCD}和\textit{A}${}_{1}$\textit{B}${}_{1}$\textit{C}${}_{1}$\textit{D}${}_{1}$为底面的棱柱.

其图形如图所示.

\includegraphics*[width=1.56in, height=0.88in, keepaspectratio=false]{image15}

知识:圆柱、圆锥、圆台、球的结构特征

难度:1

题目:下列几何体中不是旋转体的是( )

\includegraphics*[width=2.91in, height=0.82in, keepaspectratio=false]{image16}

解析:由旋转体的概念可知,选项D不是旋转体.

答案:D


知识:圆柱、圆锥、圆台、球的结构特征

难度:1

题目:用一个平面去截一个几何体,得到的截面是圆面,这个几何体不可能是( )

A.圆锥   B.圆柱   C.球   D.棱柱

解析:棱柱的任何截面都不可能是圆面.

答案:D

知识:圆柱、圆锥、圆台、球的结构特征

难度:1

题目:用任意一个平面截一个几何体,各个截面都是圆面,则这个几何体一定是( )

A.圆柱   B.圆锥   C.球体   D.圆台

解析:用任意一个平面截球体所得的截面都是圆面,故选C.

答案:C

知识:圆柱、圆锥、圆台、球的结构特征

难度:1

题目:下列命题中正确的是( )

①过球面上任意两点只能作球的一个大圆;

②球的任意两个大圆的交点的连线是球的直径;

③用不过球心的截面截球,球心和截面圆心的连线垂直于截面.

A.①②③     B.①②  C.②③       D.②

解析:过直径的两个端点可作无数个大圆,故①错;两个大圆的交点是两个大圆的公共点,也一定是直径的端点,故②正确;球心与截面圆心的连线一定垂直于截面,故③正确.

答案:C

知识:圆柱、圆锥、圆台、球的结构特征

难度:1

题目:如图(1)所示的几何体是由下图中的哪个平面图形旋转后得到的?(  )

\includegraphics*[width=3.06in, height=1.54in, keepaspectratio=false]{image17}

解析:因为简单组合体为一个圆台和一个圆锥所组成的,因此平面图形应为一个直角三角形和一个直角梯形构成,可排除B、D,再由圆台上、下底的大小比例关系可排除C,故选A.

答案:A

知识:圆柱、圆锥、圆台、球的结构特征

难度:1

题目:图中最左边的几何体由一个圆柱挖去一个以圆柱的上底面为底面,下底面圆心为顶点的圆锥而得.现用一个竖直的平面去截这个几何体,则截面图形可能是( D )

\includegraphics*[width=3.46in, height=2.68in, keepaspectratio=false]{image18}

A.(1)(2)   B.(1)(3) C.(1)(4)   D.(1)(5)

解析:圆锥除过轴的截面外,其它截面截圆锥得到的都不是三角形.

答案:D

知识:圆柱、圆锥、圆台、球的结构特征

难度:1

题目:一个与球心距离为1的平面截球所得的圆面面积为$\pi$,则球的直径为\_\_\_\_.

解析:设球心到平面的距离为\textit{d},截面圆的半径为\textit{r},则$\pi$\textit{r}${}^{2}$=$\pi$,$\mathrm{\therefore}$\textit{r}=1.设球的半径为\textit{R},则\textit{R}=$\sqrt{a^2+r^2}$=$\sqrt{2}$,故球的直径为2$\sqrt{2}$.

答案:2

知识:圆柱、圆锥、圆台、球的结构特征

难度:1

题目:已知一个正方体内接于一个球,过球心作一截面,则下图中,截面不可能是\_\_\_\_(填序号).

\includegraphics*[width=2.62in, height=0.83in, keepaspectratio=false]{image19}

解析:过球心的任何截面都不可能是圆的内接正方形.

答案:④

知识:圆柱、圆锥、圆台、球的结构特征

难度:1

题目:说出下图是由什么几何体组合而成的?

\includegraphics*[width=2.17in, height=1.45in, keepaspectratio=false]{image20}

解析:

答案:①三棱柱挖去一个圆柱

②球、圆柱和圆台

知识:圆柱、圆锥、圆台、球的结构特征

难度:2

题目:下列几何体中( )

\includegraphics*[width=2.76in, height=1.92in, keepaspectratio=false]{image21}

A.旋转体3个,台体(棱台和圆台)2个

B.旋转体3个,柱体(棱柱和圆柱)5个

C.柱体3个,锥体(棱锥或圆锥)4个

D.旋转体3个,多面体4个

解析: (6)(7)(8)为旋转体,(5)(7)为台体.

答案:A

知识:圆柱、圆锥、圆台、球的结构特征

难度:2

题目:下列命题,其中正确命题的个数是( )

①圆柱的轴截面是过母线的截面中面积最大的一个(注:轴截面是指过旋转轴的截面);②用任意一个平面去截球体得到的截面一定是一个圆面;③用任意一个平面去截圆锥得到的截面一定是一个圆.

A.0   B.1   C.2   D.3

解析:由圆锥与球的结构特征可知①②正确,故选择C.

答案:C

知识:圆柱、圆锥、圆台、球的结构特征

难度:2

题目:下列说法正确的是( )

①圆台可以由任意一个梯形绕其一边旋转形成;

②用任意一个与底面平行的平面截圆台,截面是圆面;

③在圆台上、下底面圆周上各取一点,则这两点的连线是圆台的母线;

④圆柱的任意两条母线平行,圆锥的任意两条母线相交,圆台的任意两条母线延长后相交.

A.①②  B.②③  C.①③  D.②④

解析:①错,圆台是直角梯形绕其直角边或等腰梯形绕其底边的中点连线旋转形成的;②正确;由母线的定义知③错;④正确.

答案:D

知识:圆柱、圆锥、圆台、球的结构特征

难度:2

题目:将一个等腰梯形绕着它的较长的底边所在的直线旋转一周,所得的几何体是由( )

A.一个圆台、两个圆锥构成 B.两个圆台、一个圆锥构成

C.两个圆柱、一个圆锥构成 D.一个圆柱、两个圆锥构成

解析:旋转体如图,可知选D.

\includegraphics*[width=0.86in, height=1.02in, keepaspectratio=false]{image22}

答案:D

知识:圆柱、圆锥、圆台、球的结构特征

难度:2

题目:一个圆锥的母线长为20 cm,母线与轴的夹角为30$\mathrm{{}^\circ}$,则圆锥的高为\_\_\_\_cm.

解析: \textit{h}=20cos30$\mathrm{{}^\circ}$=20$\mathrm{\times}\frac{\sqrt{3}}{2}$=10$\sqrt{3}$(cm).

答案:10$\sqrt{3}$

知识:圆柱、圆锥、圆台、球的结构特征

难度:2

题目:如图所示的几何体,关于其结构特征,下列说法不正确的是\_\_\_\_.

\includegraphics*[width=0.91in, height=1.22in, keepaspectratio=false]{image23}

①该几何体是由两个同底的四棱锥组成的几何体;

②该几何体有12条棱、6个顶点;

③该几何体有8个面,并且各面均为三角形;

④该几何体有9个面,其中一个面是四边形,其余均为三角形.

解析:平面\textit{ABCD}可将该几何体分割成两个四棱锥,因此该几何体是这两个四棱锥的组合体,因而四边形\textit{ABCD}是它的一个截面,而不是一个面,故填④.

答案:④

知识:圆柱、圆锥、圆台、球的结构特征

难度:3

题目:如图所示,几何体可看作由什么图形旋转360$\mathrm{{}^\circ}$得到?画出平面图形和旋转轴.

\includegraphics*[width=2.57in, height=1.46in, keepaspectratio=false]{image24}

解析:

答案:先出画几何体的轴,然后再观察寻找平面图形.旋转前的平面图形如下:

\includegraphics*[width=1.92in, height=1.35in, keepaspectratio=false]{image25}

知识:圆柱、圆锥、圆台、球的结构特征

难度:3

题目:如图(1)所示的图形绕虚线旋转一周后形成的几何体是由哪些简单几何体组成的.

\includegraphics*[width=0.93in, height=1.72in, keepaspectratio=false]{image26}

解析:

答案:如图(2)所示,①是矩形,旋转后形成圆柱,②、③是梯形,旋转后形成圆台.所以旋转后形成的几何体如图(3)所示,通过观察可知,该组合体是由一个圆柱、两个圆台拼接而成的.

\includegraphics*[width=1.98in, height=1.44in, keepaspectratio=false]{image27}



知识:中心投影与平行投影

难度:1

题目:下列投影是平行投影的是(  )

A.俯视图

B.路灯底下一个变长的身影

C.将书法家的真迹用电灯光投影到墙壁上

D.以一只白炽灯为光源的皮影

解析:三视图是由平行投影形成的,而B、C、D中由电灯发出的光得到的投影是中心投影.

答案:A


知识:中心投影与平行投影

难度:1

题目:已知$\mathrm{\vartriangle}$\textit{ABC},选定的投影面与$\mathrm{\vartriangle}$\textit{ABC}所在平面平行,则经过中心投影后得到的$\mathrm{\vartriangle}$\textit{A}$'$\textit{B}$'$\textit{C}$'$与$\mathrm{\vartriangle}$\textit{ABC}( )

A.全等    B.相似

C.不相似    D.以上都不对

解析:根据题意画出图形,如图所示.

\includegraphics*[width=1.49in, height=1.21in, keepaspectratio=false]{image28}

由图易得$\frac{AB}{A'B'}$=$\frac{BC}{B'C'}$=$\frac{AC}{A'C'}\mathrm{\neq}$1,所以$\mathrm{\vartriangle}$\textit{ABC}$\mathrm{\backsim}$$\mathrm{\vartriangle}$\textit{A}$'$\textit{B}$'$\textit{C}$'$.故选B.

答案:B

知识:空间几何体的三视图

难度:1

题目:如图所示的三视图表示的几何体可能是(  )

\includegraphics*[width=1.79in, height=1.41in, keepaspectratio=false]{image29}

A.圆台   B.四棱台   C.四棱锥   D.三棱台

解析:由三视图可知,该几何体是四棱台.

答案:B

知识:空间几何体的三视图

难度:1

题目:某空间几何体的正视图是三角形,则该几何体不可能是(  )

A.圆柱   B.圆锥   C.四面体   D.三棱柱

解析:圆柱的正视图不可能是三角形.

答案:A

知识:空间几何体的三视图

难度:1

题目:小周过生日,公司为她预订的生日蛋糕(示意图)如下图所示,则它的正视图应该是( )

\includegraphics*[width=3.15in, height=1.64in, keepaspectratio=false]{image30}

解析:A为俯视图,注意到封闭的线段情形,正视图应该是B.

答案:B

知识:空间几何体的三视图

难度:1

题目:若一个几何体的三视图如下图所示,则这个几何体是(  )

\includegraphics*[width=1.95in, height=1.34in, keepaspectratio=false]{image31}

A.三棱锥   B.四棱锥   C.三棱柱   D.四棱柱

解析:由俯视图可知底面为四边形,由正视图和侧视图知侧面为三角形,故几何体为四棱锥.

答案:B

知识:中心投影与平行投影

难度:1

题目:太阳光线与地面成60$\mathrm{{}^\circ}$的角,照射在地面上的一个皮球上,皮球在地面上的投影长是10$\sqrt{3}$,则皮球的直径是\_\_\_\_.

\includegraphics*[width=0.90in, height=0.75in, keepaspectratio=false]{image32}

解析:皮球的直径\textit{d}=10$\sqrt{3}$sin60$\mathrm{{}^\circ}$=10$\sqrt{3}\mathrm{\times}\frac{\sqrt{3}}{2}$=15.

答案:15

知识:空间几何体的三视图

难度:1

题目:如图所示,在正方体\textit{ABCD}-\textit{A}${}_{1}$\textit{B}${}_{1}$\textit{C}${}_{1}$\textit{D}${}_{1}$中,点\textit{P}是上底面\textit{A}${}_{1}$\textit{B}${}_{1}$\textit{C}${}_{1}$\textit{D}${}_{1}$内一动点,则三棱锥\textit{P}-\textit{ABC}的正视图与侧视图的面积的比值为\_\_\_\_.

\includegraphics*[width=1.63in, height=1.44in, keepaspectratio=false]{image33}

解析:三棱锥\textit{P}-\textit{ABC}的正视图与侧视图为等底等高的三
角形,故它们的面积相等,面积比值为1.

答案:1


知识:空间几何体的三视图

难度:1

题目:如图,四棱锥的底面是正方形,顶点在底面上的射影是底面正方形的中心,试画出其三视图.

\includegraphics*[width=1.06in, height=0.85in, keepaspectratio=false]{image34}

解析:

答案:所给四棱锥的三视图如下图.

\includegraphics*[width=1.32in, height=1.29in, keepaspectratio=false]{image35}

知识:中心投影与平行投影

难度:2

题目:下列说法不正确的是( )

A.光由一点向外散射形成的投影,叫作中心投影

B.在一束平行光线照射下形成的投影,叫作平行投影

C.空间几何体的三视图是用中心投影的方法得到的

D.在平行投影之下,与投影面平行的平面图形留下的影子,与这个平面图形的形状和大小是完全相同的

解析:空间几何体的三视图是在平行投影下得到的,故C中说法不正确.

答案:C


知识:空间几何体的三视图

难度:2

题目:某几何体的正视图和侧视图均如左图所示,则该几何体的俯视图不可能是( )

\includegraphics*[width=2.45in, height=1.54in, keepaspectratio=false]{image36}

解析:A、B、D都可能是该几何体的俯视图,C不可能是该几何体的俯视图,因为它的正视图上面应为如右图所示的矩形.

\includegraphics*[width=0.56in, height=0.69in, keepaspectratio=false]{image37}

答案:C

知识:空间几何体的三视图

难度:2

题目:如图所示,画出四面体\textit{AB}${}_{1}$\textit{CD}${}_{1}$三视图中的正视图,以面\textit{AA}${}_{1}$\textit{D}${}_{1}$\textit{D}为投影面,则得到的正视图可以为(  )

\includegraphics*[width=1.12in, height=1.06in, keepaspectratio=false]{image38}

\includegraphics*[width=3.14in, height=0.90in, keepaspectratio=false]{image39}

解析:显然\textit{AB}${}_{1}$、\textit{AC}、\textit{B}${}_{1}$\textit{D}${}_{1}$、\textit{CD}${}_{1}$分别投影得到正视图的外轮廓,\textit{B}${}_{1}$\textit{C}为可见实线,\textit{AD}${}_{1}$为不可见虚线.故A正确.

答案:A


知识:空间几何体的三视图

难度:2

题目:已知三棱柱\textit{ABC}-\textit{A}${}_{1}$\textit{B}${}_{1}$\textit{C}${}_{1}$,如图所示,则其三视图为(  )

\includegraphics*[width=1.23in, height=0.91in, keepaspectratio=false]{image40}\includegraphics*[width=2.50in, height=2.34in, keepaspectratio=false]{image41}

解析:其正视图为矩形,侧视图为三角形,俯视图中棱\textit{CC}${}_{1}$可见,为实线,只有A符合.

答案:A

知识:中心投影与平行投影

难度:2

题目:下列图形:①三角形;②直线;③平行四边形;④四面体;⑤球.其中投影不可能是线段的是\_\_\_\_.

解析:三角形的投影是线段或三角形;直线的投影是点或直线;平行四边形的投影是线段或平行四边形;四面体的投影是三角形或四边形;球的投影是圆.

答案:②④⑤

知识:空间几何体的三视图

难度:2

题目:已知某一几何体的正视图与侧视图如图所示,则下列图形中,可以是该几何体的俯视图的图形有\_\_\_\_.

\includegraphics*[width=2.77in, height=1.36in, keepaspectratio=false]{image42}

解析:

答案:①②③④

知识:空间几何体的三视图

难度:3

题目:由几何体的三视图如图所示,试分析该几何体的结构特征.

\includegraphics*[width=1.59in, height=1.64in, keepaspectratio=false]{image43}

解析:

答案:由正视图和侧视图可知,该物体的下半部分为柱体,上半部分为锥体,又因俯视图为一个正六边形,故该几何体是由一个正六棱柱和一个正六棱锥组合而成的,如图所示.

\includegraphics*[width=0.90in, height=1.27in, keepaspectratio=false]{image44}

知识:空间几何体的三视图

难度:3

题目:如图所示的螺栓是由棱柱和圆柱构成的组合体,试画出它的三视图.

\includegraphics*[width=1.34in, height=1.44in, keepaspectratio=false]{image45}

解析:

答案:三视图如图所示.

\includegraphics*[width=1.72in, height=1.63in, keepaspectratio=false]{image46}



知识:空间几何体的直观图

难度:1

题目:\textit{AB}=2\textit{CD},\textit{AB}//\textit{x}轴,\textit{CD}//\textit{y}轴,已知在直观图中,\textit{AB}的直观图是\textit{A}$'$\textit{B}$'$,\textit{CD}的直观图是\textit{C}$'$\textit{D}$'$,则(  )

A.\textit{A}$'$\textit{B}$'$=2\textit{C}$'$\textit{D}$'$   B.\textit{A}$'$\textit{B}$'$=\textit{C}$'$\textit{D}$'$

C.\textit{A}$'$\textit{B}$'$=4\textit{C}$'$\textit{D}$'$   D.\textit{A}$'$\textit{B}$'$=$\frac{1}{2}$\textit{C}$'$\textit{D}$'$

解析:$\mathrm{\because}$\textit{AB}//\textit{x}轴,\textit{CD}//\textit{y}轴,$\mathrm{\therefore}$\textit{AB}=\textit{A}$'$\textit{B}$'$,\textit{CD}=2\textit{C}$'$\textit{D}$'$,

$\mathrm{\therefore}$\textit{A}$'$\textit{B}$'$=\textit{AB}=2\textit{CD}=2(2\textit{C}$'$\textit{D}$'$)=4\textit{C}$'$\textit{D}$'$.

答案:C

知识:空间几何体的直观图

难度:1

题目:下列关于用斜二测画法画直观图的说法中,正确的是(  )

A.水平放置的正方形的直观图不可能是平行四边形

B.平行四边形的直观图仍是平行四边形

C.两条相交直线的直观图可能是平行直线

D.两条垂直的直线的直观图仍互相垂直

解析:平行四边形的边平行,则在直观图中仍然平行,故选项B正确.

答案:B

知识:空间几何体的直观图

难度:1

题目:给出以下关于斜二测直观图的结论,其中正确的个数是(  )

①角的水平放置的直观图一定是角;

②相等的角在直观图中仍相等;

③相等的线段在直观图中仍然相等;

④若两条线段平行,则在直观图中对应的两条线段仍然平行.

A.0   B.1   C.2   D.3

解析:由斜二测画法规则可知,直观图保持线段的平行性,$\mathrm{\therefore}$④对,①对;而线段的长度,角的大小在直观图中都会发生改变,$\mathrm{\therefore}$②③错.

答案:C

知识:空间几何体的直观图

难度:1

题目:如图,用斜二测画法画一个水平放置的平面图形的直观图为一个正方形,则原来图形的形状是(  )

\includegraphics*[width=3.15in, height=0.83in, keepaspectratio=false]{image47}

解析:由斜二测画法可知,与\textit{y}$'$轴平行的线段在原图中为在直观图中的2倍.故可判断A正确.

答案:A

知识:空间几何体的直观图

难度:1

题目:一个建筑物上部为四棱锥,下部为长方体,且四棱锥的底面与长方体的上底面尺寸一样,已知长方体的长、宽、高分别为20 m、5 m、10 m,四棱锥的高为8 m,若按1︰500的比例画出它的直观图,那么直观图中,长方体的长、宽、高和棱锥的高应分别为(  )

A.4 cm,1 cm, 2 cm,1.6 cm

B.4 cm,0.5 cm,2 cm,0.8 cm

C.4 cm,0.5 cm,2 cm,1.6 cm

D.2 cm,0.5 cm,1 cm,0.8 cm

解析:由比例尺可知长方体的长、宽、高和四棱锥的高分别为4 cm,1 cm,2 cm和1.6 cm,再结合斜二测画法,可知直观图的相应尺寸应分别为4 cm,0.5 cm,2 cm,1.6 cm.

答案:C


知识:空间几何体的直观图

难度:1

题目:如图Rt$\mathrm{\vartriangle}$\textit{O}$'$\textit{A}$'$\textit{B}$'$是一个平面图形的直观图,若\textit{O}$'$\textit{B}$'$=$\sqrt{2}$,则这个平面图形的面积是(  )

\includegraphics*[width=1.21in, height=0.85in, keepaspectratio=false]{image48}

A.1   B.$\sqrt{2}$ C.2$\sqrt{2}$   D.4$\sqrt{2}$

解析:由直观图可知,原平面图形是Rt$\mathrm{\vartriangle}$\textit{OAB},其中\textit{OA}$\mathrm{\bot}$\textit{OB},则\textit{OB}=\textit{O}$'$\textit{B}$'$=$\sqrt{2}$,\textit{OA}=2\textit{O}$'$\textit{A}$'$=4,$\mathrm{\therefore}$\textit{S}${}_{\vartriangle }$\textit{${}_{OAB}$}=\textit{OB}·\textit{OA}=2,故选C.

答案:C

知识:空间几何体的直观图

难度:1

题目:斜二测画法中,位于平面直角坐标系中的点\textit{M}(4,4)在直观图中的对应点是\textit{M}$'$,则点\textit{M}$'$的坐标为\_\_\_\_,点\textit{M}$'$的找法是\_\_\_\_.

解析:在\textit{x}$'$轴的正方向上取点\textit{M}${}_{1}$,使\textit{O}${}_{1}$\textit{M}${}_{1}$=4,在\textit{y}$'$轴上取点\textit{M}${}_{2}$,使\textit{O}$'$\textit{M}${}_{2}$=2,过\textit{M}${}_{1}$和\textit{M}${}_{2}$分别作平行于\textit{y}$'$轴和\textit{x}$'$轴的直线,则交点就是\textit{M}$'$.

答案:\textit{M}$'$(4,2);

在坐标系\textit{x}$'$\textit{O}$'$\textit{y}$'$中,过点(4,0)和\textit{y}$'$轴平行的直线与过点(0,2)和\textit{x}$'$轴平行的直线的交点即是点\textit{M}$'$

知识:空间几何体的直观图

难度:1

题目:如右图,水平放置的$\mathrm{\vartriangle}$\textit{ABC}的斜二测直观图是图中的$\mathrm{\vartriangle}$\textit{A}$'$\textit{B}$'$\textit{C}$'$,已知\textit{A}$'$\textit{C}$'$=6,\textit{B}$'$\textit{C}$'$=4,则\textit{AB}边的实际长度是\_\_\_\_.

\includegraphics*[width=1.13in, height=0.92in, keepaspectratio=false]{image49}

解析:由斜二测画法,可知$\mathrm{\vartriangle}$\textit{ABC}是直角三角形,且$\mathrm{\angle}$\textit{BCA}=90$\mathrm{{}^\circ}$,\textit{AC}=6,\textit{BC}=4$\mathrm{\times}$2=8,则\textit{AB}=$\sqrt{\text{AB}^2+\text{BC}^2}=$10.

答案:10

知识:空间几何体的直观图

难度:1

题目:如图所示,四边形\textit{ABCD}是一个梯形,\textit{CD}//\textit{AB},\textit{CD}=\textit{AO}=1,三角形\textit{AOD}为等腰直角三角形,\textit{O}为\textit{AB}的中点,试求梯形\textit{ABCD}水平放置的直观图的面积.

\includegraphics*[width=1.15in, height=0.70in, keepaspectratio=false]{image50}

解析:在梯形\textit{ABCD}中,\textit{AB}=2,高\textit{OD}=1,由于梯形\textit{ABCD}水平放置的直观图仍为梯形,且上底\textit{CD}和下底\textit{AB}的长度都不变,在直观图中,\textit{O}$'$\textit{D}$'$=$\frac{1}{2}$\textit{OD},梯形的高\textit{D}$'$\textit{E}$'$=$\frac{\sqrt{2}}{4}$,于是梯形\textit{A}$'$\textit{B}$'$\textit{C}$'$\textit{D}的面积为$\frac{1}{2}\times(1+2)\frac{\sqrt{2}}{4}= \frac{3\sqrt{2}}{8}$.

答案:$\frac{3\sqrt{2}}{8}$。

知识:空间几何体的直观图

难度:1
题目:一个几何体,它的下面是一个圆柱,上面是一个圆锥,并且圆锥的底面与圆柱的上底面重合,圆柱的底面直径为3 cm,高为4 cm,圆锥的高为3 cm,画出此几何体的直观图.

解析:

答案:(1)画轴.如图1所示,画\textit{x}轴、\textit{z}轴,使$\mathrm{\angle}$\textit{xOz}=90$\mathrm{{}^\circ}$.

(2)画圆柱的两底面,在\textit{x}轴上取\textit{A}、\textit{B}两点,使\textit{AB}的长度等于3 cm,且\textit{OA}=\textit{OB}.选择椭圆模板中适当的椭圆过\textit{A},\textit{B}两点,使它为圆柱的下底面.在\textit{Oz}上截取点\textit{O}$'$,使\textit{OO}$'$=4 cm,过\textit{O}$'$作\textit{Ox}的平行线\textit{O}$'$\textit{x}$'$,类似圆柱下底面的作法作出圆柱的上底面.

(3)画圆锥的顶点.在\textit{Oz}上截取点\textit{P},使\textit{PO}$'$等于圆锥的高3 cm.

(4)成图.连接\textit{A}$'$\textit{A}、\textit{B}$'$\textit{B}、\textit{PA}$'$、\textit{PB}$'$,整理得到此几何体的直观图.如图2所示.

\includegraphics*[width=2.08in, height=1.62in, keepaspectratio=false]{image51}

知识:空间几何体的直观图

难度:2

题目:如图,矩形\textit{O}$'$\textit{A}$'$\textit{B}$'$\textit{C}$'$是水平放置的一个平面图形用斜二测画法得到的直观图,其中\textit{O}$'$\textit{A}$'$=6 cm,\textit{O}$'$\textit{C}$'$=2 cm,则原图形是(  )

\includegraphics*[width=1.38in, height=0.83in, keepaspectratio=false]{image52}

A.正方形   B.矩形   C.菱形   D.梯形

解析:将直观图还原得到平行四边形\textit{OABC},如图所示.由题意知\textit{O}$'$\textit{D}$'$=\textit{O}$'$\textit{C}$'$=2 cm,

\textit{OD}=2\textit{O}$'$\textit{D}$'$=4 cm,\textit{C}$'$\textit{D}$'$=\textit{O}$'$\textit{C}$'$=2 cm,$\mathrm{\therefore}$\textit{CD}=2 cm,

\textit{OC}==6 cm,又\textit{OA}=\textit{O}$'$\textit{A}$'$=6 cm=\textit{OC},$\mathrm{\therefore}$原图形为菱形.

\includegraphics*[width=1.48in, height=1.17in, keepaspectratio=false]{image53}

答案:C

知识:空间几何体的直观图

难度:2

题目:如图,正方形\textit{O}$'$\textit{A}$'$\textit{B}$'$\textit{C}$'$的边长为1 cm,它是水平放置的一个平面图形的直观图,则原图形的周长是(  )

\includegraphics*[width=1.40in, height=0.75in, keepaspectratio=false]{image54}

A.8 cm   B.6 cm C.2(1+)cm   D.2(1+)cm

解析:根据直观图的画法可知,在原几何图形中,\textit{OABC}为平行四边形,且有\textit{OB}$\mathrm{\bot}$\textit{OA},\textit{OB}=2,\textit{OA}=1,所以\textit{AB}=3.从而原图的周长为8.

答案:A

知识:空间几何体的直观图

难度:2

题目:下图甲所示为一平面图形的直观图,则此平面图形可能是乙图中的(  )

\includegraphics*[width=3.15in, height=2.04in, keepaspectratio=false]{image55}

解析:按斜二测画法规则,平行于\textit{x}轴或\textit{x}轴上的线段的长度在新坐标系中不变,平行于\textit{y}轴或在\textit{y}轴上的线段在新坐标系中变为原来的,并注意到$\mathrm{\angle}$\textit{xOy}=90$\mathrm{{}^\circ}$,$\mathrm{\angle}$\textit{x}$'$\textit{O}$'$\textit{y}$'$=45$\mathrm{{}^\circ}$,将图形还原成原图形知选C.

答案:C

知识:空间几何体的直观图

难度:2

题目:已知正三角形\textit{ABC}的边长为\textit{a},那么用斜二测画法得到的$\mathrm{\vartriangle}$\textit{ABC}的平面直观图$\mathrm{\vartriangle}$\textit{A}$'$\textit{B}$'$\textit{C}$'$的面积为(  )

A.\textit{a}${}^{2}$   B.\textit{a}${}^{2}$   C.\textit{a}${}^{2}$   D.\textit{a}${}^{2}$

解析:根据题意,建立如图①所示的平面直角坐标系,再按照斜二测画法画出其直观图,如图②中$\mathrm{\vartriangle}$\textit{A}$'$\textit{B}$'$\textit{C}$'$所示.

\includegraphics*[width=2.47in, height=1.12in, keepaspectratio=false]{image56}

易知,\textit{A}$'$\textit{B}$'$=\textit{AB}=\textit{a},\textit{O}$'$\textit{C}$'$=\textit{OC}=\textit{a}.过点\textit{C}$'$作\textit{C}$'$\textit{D}$'$$\mathrm{\bot}$\textit{A}$'$\textit{B}$'$于点\textit{D}$'$,则\textit{C}$'$\textit{D}$'$=\textit{O}$'$\textit{C}$'$=\textit{a}.所以\textit{S}${}_{\vartriangle }$\textit{${}_{A}$}${}_{\mathrm{\prime }}$\textit{${}_{B}$}${}_{\mathrm{\prime }}$\textit{${}_{C}$}${}_{\mathrm{\prime }}$=\textit{A}$'$\textit{B}$'$·\textit{C}$'$\textit{D}$'$=\textit{a}$\mathrm{\times}$\textit{a}=\textit{a}${}^{2}$.

答案:D

知识:空间几何体的直观图

难度:2

题目:如图,是$\mathrm{\vartriangle}$\textit{AOB}用斜二测画法画出的直观图,则$\mathrm{\vartriangle}$\textit{AOB}的面积是\_\_\_\_.



\includegraphics*[width=1.10in, height=0.81in, keepaspectratio=false]{image57}

解析:由图易知$\mathrm{\vartriangle}$\textit{AOB}中,底边\textit{OB}=4,

又$\mathrm{\because}$底边\textit{OB}的高为8,

$\mathrm{\therefore}$面积\textit{S}=$\mathrm{\times}$4$\mathrm{\times}$8=16.

答案:16

知识:空间几何体的直观图

难度:2

题目:如右图所示,四边形\textit{OABC}是上底为2,下底为6,底角为45$\mathrm{{}^\circ}$的等腰梯形,用斜二侧画法,画出这个梯形的直观图\textit{O}$'$\textit{A}$'$\textit{B}$'$\textit{C}$'$,在直观图中梯形的高为\_\_\_\_.



\includegraphics*[width=1.10in, height=0.65in, keepaspectratio=false]{image58}

解析:因为\textit{OA}=6,\textit{CB}=2,所以\textit{OD}=2.又因为$\mathrm{\angle}$\textit{COD}=45$\mathrm{{}^\circ}$,所以\textit{CD}=2.梯形的直观图如右图,则\textit{C}$'$\textit{D}$'$=1.所以梯形的高\textit{C}$'$\textit{E}$'$=$\frac{\sqrt{2}}{2}$.

\includegraphics*[width=1.42in, height=0.75in, keepaspectratio=false]{image59}

答案:$\frac{\sqrt{2}}{2}$

知识:空间几何体的直观图

难度:3

题目:如图所示,已知几何体的三视图,用斜二测画法画出它的直观图.

\includegraphics*[width=1.60in, height=1.61in, keepaspectratio=false]{image60}

解析:

答案:由几何体的三视图可知,这个几何体是一个圆台,画法:①画轴.画\textit{x}轴、\textit{y}轴、\textit{z}轴,使$\mathrm{\angle}$\textit{xOy}=45$\mathrm{{}^\circ}$,$\mathrm{\angle}$\textit{xOz}=90$\mathrm{{}^\circ}$.②画圆台的两底面,取底面$\mathrm{\odot}$\textit{O}和上底面$\mathrm{\odot}$\textit{O}$'$的长为俯视图中的大圆和小圆的直径,画出$\mathrm{\odot}$\textit{O}与$\mathrm{\odot}$\textit{O}$'$.③取\textit{OO}$'$为正视图的高度.④成图.如图,整理得到三视图表示的几何体的直观图.

\includegraphics*[width=1.84in, height=1.02in, keepaspectratio=false]{image61}


知识:空间几何体的直观图

难度:3

题目:由如图所示几何体的三视图画出直观图.

\includegraphics*[width=2.07in, height=1.84in, keepaspectratio=false]{image62}

解析:

答案:(1)画轴.如图,画出\textit{x}轴、\textit{y}轴、\textit{z}轴,三轴相交于点\textit{O},使$\mathrm{\angle}$\textit{xOy}=45$\mathrm{{}^\circ}$,$\mathrm{\angle}$\textit{xOz}=90$\mathrm{{}^\circ}$.

(2)画底面.作水平放置的三角形(俯视图)的直观图$\mathrm{\vartriangle}$\textit{ABC}.

(3)画侧棱.过\textit{A}、\textit{B}、\textit{C}各点分别作\textit{z}轴的平行线,并在这些平行线上分别截取线段\textit{AA}$'$、\textit{BB}$'$、\textit{CC}$'$,且\textit{AA}$'$=\textit{BB}$'$=\textit{CC}$'$.

(4)成图,顺次连接\textit{A}$'$、\textit{B}$'$、\textit{C}$'$,并加以整理(擦去辅助线,将遮挡部分用虚线表示),得到的图形就是所求的几何体的直观图.

\includegraphics*[width=2.11in, height=1.21in, keepaspectratio=false]{image63}




知识:柱体、锥体、台体的表面积与体积

难度:1

题目:若圆锥的正视图是正三角形,则它的侧面积是底面积的(  )

A.倍   B.3倍   C.2倍   D.5倍

解析:设圆锥的底面半径为\textit{r},母线长为\textit{l},则由题意知,\textit{l}=2\textit{r},于是\textit{S}${}_{\textrm{侧}}$=\textit{$\pi$r}·2\textit{r}=2\textit{$\pi$r}${}^{2}$,\textit{S}${}_{\textrm{底}}$=\textit{$\pi$r}${}^{2}$.故选C.

答案:C

知识:柱体、锥体、台体的表面积与体积

难度:1

题目:长方体的高为1,底面积为2,垂直于底的对角面的面积是,则长方体的侧面积等于(  )

A.2$\sqrt{7}$   B.4$s\sqrt{3}$   C.6   D.3

解析:设长方体的长、宽、高分别为\textit{a}、\textit{b}、\textit{c},

则\textit{c}=1,\textit{ab}=2,$\sqrt{a^2+b^2}\cdot$\textit{c}=$\sqrt{5}$,

$\mathrm{\therefore}$\textit{a}=2,\textit{b}=1,故\textit{S}${}_{\textrm{侧}}$=2(\textit{ac}+\textit{bc})=6.

答案:C

知识:柱体、锥体、台体的表面积与体积

难度:1

题目:圆柱的侧面展开图是长12 cm,宽8 cm的矩形,则这个圆柱的体积为(  )

A. $\frac{288}{\pi}$cm$^3$    B. $\frac{192}{\pi}$cm$^3$

C. $\frac{288}{\pi}$cm$^3$或 $\frac{192}{\pi}$cm$^3$   D.192$\pi$ cm$^3$

解析:圆柱的高为8 cm时,\textit{V}=$\pi$$\mathrm{\times}$($\frac{12}{2\pi}$)${}^{2}$$\mathrm{\times}$8= $\frac{288}{\pi}$cm${}^{3}$,当圆柱的高为12cm时,\textit{V}=$\pi$$\mathrm{\times}$($\frac{8}{2\pi}$)${}^{2}$$\mathrm{\times}$12=$\frac{192}{\pi}$cm${}^{3}$.

答案:C

知识:柱体、锥体、台体的表面积与体积

难度:1

题目:圆台的体积为7$\pi$,上、下底面的半径分别为1和2,则圆台的高为(  )

A.3   B.4   C.5   D.6

解析:由题意,\textit{V}=($\pi$+2$\pi$+4$\pi$)\textit{h}=7$\pi$,$\mathrm{\therefore}$\textit{h}=3.

答案:A

知识:柱体、锥体、台体的表面积与体积

难度:1

题目:若一圆柱与圆锥的高相等,且轴截面面积也相等,那么圆柱与圆锥的体积之比为(  )

A.1   B. C.   D.

解析:设圆柱底面半径为\textit{R},圆锥底面半径\textit{r},高都为\textit{h},由已知得2\textit{Rh}=\textit{rh},$\mathrm{\therefore}$\textit{r}=2\textit{R},

\textit{V}${}_{\textrm{柱}}$︰\textit{V}${}_{\textrm{锥}}$=$\pi$\textit{R}${}^{2}$\textit{h}︰$\pi$\textit{r}${}^{2}$\textit{h}=3︰4,故选D.

答案:D

知识:柱体、锥体、台体的表面积与体积

难度:1

题目:(2015·山东文)已知等腰直角三角形的直角边的长为2,将该三角形绕其斜边所在的直线旋转一周而形成的曲面所围成的几何体的体积为(  )

A.$\frac{2\sqrt{2}}{3}\pi$   B.$\frac{4\sqrt{2}}{3}\pi$ C.2$\sqrt{2}\pi$   D.4$\sqrt{2}\pi$

解析:绕等腰直角三角形的斜边所在的直线旋转一周形成的曲面围成的几何体为两个底面重合,等体积的圆锥,如图所示.每一个圆锥的底面半径和高都为,故所求几何体的体积\textit{V}=2$\mathrm{\times}$$\mathrm{\times}$2$\pi$$\mathrm{\times}\sqrt{2}$=$\frac{4\sqrt{2}}{3}\pi$.

\includegraphics*[width=0.85in, height=1.27in, keepaspectratio=false]{image64}

答案:B

知识:柱体、锥体、台体的表面积与体积

难度:1

题目:一个几何体的三视图如图所示,其中俯视图为正三角形,则该几何体的表面积为\_\_\_\_.

\includegraphics*[width=1.25in, height=1.40in, keepaspectratio=false]{image65}

解析:该几何体是三棱柱,且两个底面是边长为2的正三角形,侧面是全等的矩形,且矩形的长是4,宽是2,所以该几何体的表面积为$2\times(\frac{1}{2}\times 2\times\sqrt{3})+3\times (4\times 2)=24+\sqrt{3}$.

答案:24+2$\sqrt{3}$

知识:柱体、锥体、台体的表面积与体积

难度:1

题目:设甲、乙两个圆柱的底面积分别为\textit{S}${}_{1}$、\textit{S}${}_{2}$,体积分别为\textit{V}${}_{1}$、\textit{V}${}_{2}$,若它们的的侧面积相等且\textit{S}${}_{1}$︰\textit{S}${}_{2}$=9︰4,则\textit{V}${}_{1}$︰\textit{V}${}_{2}$=\_\_\_\_.

解析:设甲圆柱底面半径\textit{r}${}_{1}$,高\textit{h}${}_{1}$,乙圆柱底面半径\textit{r}${}_{2}$,高\textit{h}${}_{2}$,$\frac{S_1}{S_2}=\frac{\pi r_1^2}{\pi r_2^2}=\frac{9}{4}$,$\mathrm{\therefore}\frac{r_1}{r_2}=\frac{3}{2}$,又侧面积相等得2\textit{$\pi$r}${}_{1}$\textit{h}${}_{1}$=2\textit{$\pi$r}${}_{2}$\textit{h}${}_{2}$,$\mathrm{\therefore}\frac{h_1}{h_2}=\frac{2}{3}$.因此$\frac{V_1}{V_2}=\frac{\pi r_1^2h_1}{\pi r_2^2h_2}=\frac{3}{2}$.

答案:3︰2

知识:柱体、锥体、台体的表面积与体积

难度:1

题目:如图所示的几何体是一棱长为4 cm的正方体,若在其中一个面的中心位置上,挖一个直径为2 cm、深为1 cm的圆柱形的洞,求挖洞后几何体的表面积是多少?($\pi$取3.14)

\includegraphics*[width=1.06in, height=0.97in, keepaspectratio=false]{image66}

解析:

答案:正方体的表面积为4$\mathrm{\times}$4$\mathrm{\times}$6=96(cm${}^{2}$),

圆柱的侧面积为2$\pi$$\mathrm{\times}$1$\mathrm{\times}$1$\mathrm{\approx}$6.28(cm${}^{2}$),

则挖洞后几何体的表面积约为96+6.28=102.28(cm${}^{2}$).

知识:柱体、锥体、台体的表面积与体积

难度:2

题目:(2017·浙江,3)某几何体的三视图如图所示(单位:cm),则该几何体的体积(单位:cm${}^{3}$)是(  )

\includegraphics*[width=1.34in, height=1.61in, keepaspectratio=false]{image67}

A.$\frac{\pi}{2}$+1       B.$\frac{\pi}{2}$+3 C.$\frac{3\pi}{2}$+1   D.$\frac{3\pi}{2}$+3

解析:由几何体的三视图可知,该几何体是一个底面半径为1,高为3的圆锥的一半与一个底面为直角边长是的等腰直角三角形,高为3的三棱锥的组合体,

$\mathrm{\therefore}$该几何体的体积

$V=\frac{1}{3}\times\frac{1}{2}\pi\times1^2\times 3+\frac{1}{3}\times\frac{1}{2}\times\sqrt{2}\times\sqrt{2}\times 3=\frac{\pi}{2}+1$

故选A.

答案:A

知识:柱体、锥体、台体的表面积与体积

难度:2

题目:某几何体的三视图如图所示,则该几何体的表面积为(  )

\includegraphics*[width=2.42in, height=1.81in, keepaspectratio=false]{image68}

A.180   B.200   C.220   D.240

解析:几何体为直四棱柱,其高为10,底面是上底为2,下底为8,高为4,腰为5的等腰梯形,故两个底面面积的和为$\mathrm{\times}$(2+8)$\mathrm{\times}$4$\mathrm{\times}$2=40,四个侧面面积的和为(2+8+5$\mathrm{\times}$2)$\mathrm{\times}$10=200,所以直四棱柱的表面积为\textit{S}=40+200=240.

答案:D

知识:柱体、锥体、台体的表面积与体积

难度:2

题目:(2015·全国卷Ⅱ)一个正方体被一个平面截去一部分后,剩余部分的三视图如图所示,则截去部分体积与剩余部分体积的比值为(  )

\includegraphics*[width=1.09in, height=1.02in, keepaspectratio=false]{image69}

A.$\frac{1}{8}$   B.$\frac{1}{7}$ C.$\frac{1}{6}$   D.$\frac{1}{5}$

解析:由三视图得,在正方体\textit{ABCD}-\textit{A}${}_{1}$\textit{B}${}_{1}$\textit{C}${}_{1}$\textit{D}${}_{1}$中,截去四面体\textit{A}-\textit{A}${}_{1}$\textit{B}${}_{1}$\textit{D}${}_{1}$,如图所示,设正方体棱长为\textit{a},则\textit{VA}-\textit{A}${}_{1}$\textit{B}${}_{1}$\textit{D}${}_{1}$=$\frac{1}{3}\times\frac{1}{2}a^3=\frac{1}{6}a^3$,故剩余几何体体积为$a^3-\frac{1}{6}a^3=\frac{5}{6}a^3$,所以截去部分体积与剩余部分体积的比值为$\frac{1}{5}$.

知识:柱体、锥体、台体的表面积与体积

难度:2

题目:(2017·全国卷Ⅰ理,7)某多面体的三视图如图所示,其中正视图和左视图都由正方形和等腰直角三角形组成,正方形的边长为2,俯视图为等腰直角三角形.该多面体的各个面中有若干个是梯形,这些梯形的面积之和为(  )

\includegraphics*[width=1.18in, height=1.20in, keepaspectratio=false]{image70}

A.10   B.12 C.14   D.16

解析:观察三视图可知该多面体是由直三棱柱和三棱锥组合而成的,且直三棱柱的底面是直角边长为2的等腰直角三角形,侧棱长为2.三棱锥的底面是直角边长为2的等腰直角三角形,高为2,如图所示.因此该多面体各个面中有2个梯形,且这两个梯形全等,梯形的上底长为2,下底长为4,高为2,故这些梯形的面积之和为$2\times\frac{1}{2}\times(2+4)\times2=12$.故选B.

\includegraphics*[width=0.67in, height=1.22in, keepaspectratio=false]{image71}

答案:B


知识:柱体、锥体、台体的表面积与体积

难度:2

题目:已知圆柱\textit{OO}$'$的母线\textit{l}=4 cm,全面积为42$\pi$ cm${}^{2}$,则圆柱\textit{OO}$'$的底面半径\textit{r}= \_\_\_\_cm.

解析:圆柱\textit{OO}$'$的侧面积为2$\pi$\textit{rl}=8$\pi$\textit{r}(cm${}^{2}$),两底面积为2$\mathrm{\times}$$\pi$\textit{r}${}^{2}$=2$\pi$\textit{r}${}^{2}$(cm${}^{2}$),

$\mathrm{\therefore}$2$\pi$\textit{r}${}^{2}$+8$\pi$\textit{r}=42$\pi$,

解得\textit{r}=3或\textit{r}=-7(舍去),

$\mathrm{\therefore}$圆柱的底面半径为3 cm.

答案:3

知识:柱体、锥体、台体的表面积与体积

题目:已知斜三棱柱的三视图如图所示,该斜三棱柱的体积为\_\_\_\_.

\includegraphics*[width=1.57in, height=1.86in, keepaspectratio=false]{image72}

解析:由三视图可知,斜三棱柱的底面三角形的底边长为2,高为1,斜三棱柱的高为2,故斜三棱柱的体积为\textit{V}=$\mathrm{\times}$2$\mathrm{\times}$1$\mathrm{\times}$2=2.

答案:2

知识:柱体、锥体、台体的表面积与体积

难度:3

题目:如图在底面半径为2,母线长为4的圆锥中内接一个高为的圆柱,求圆柱的表面积.



\includegraphics*[width=0.89in, height=1.10in, keepaspectratio=false]{image73}

解析:设圆锥的底面半径为\textit{R},圆柱的底面半径为\textit{r},表面积为\textit{S}.

\includegraphics*[width=0.87in, height=1.09in, keepaspectratio=false]{image74}

则\textit{R}=\textit{OC}=2,\textit{AC}=4,

\textit{AO}$=\sqrt{4^2-2^2}=2\sqrt{3}$.

如图所示易知$\mathrm{\vartriangle}$\textit{AEB}$\mathrm{\backsim}$$\mathrm{\vartriangle}$\textit{AOC},

$\mathrm{\therefore}\frac{AE}{AO}=\frac{EB}{OC}$,即$\frac{\sqrt{3}}{2\sqrt{3}}=\frac{r}{2}$,$\mathrm{\therefore}$\textit{r}=1,

\textit{S}${}_{\textrm{底}}$=2$\pi$\textit{r}${}^{2}$=2$\pi$,\textit{S}${}_{\textrm{侧}}$=2$\pi$\textit{r}·\textit{h}=2 $\sqrt{3}\pi$.

$\mathrm{\therefore}$\textit{S}=\textit{S}${}_{\textrm{底}}$+\textit{S}${}_{\textrm{侧}}$=2$\pi$+2$\sqrt{3}\pi$=(2+2)$\sqrt{3}\pi$.

答案:$(2+2\sqrt{3})\pi$

知识:柱体、锥体、台体的表面积与体积

难度:3

题目:在长方体\textit{ABCD}­\textit{A}${}_{1}$\textit{B}${}_{1}$\textit{C}${}_{1}$\textit{D}${}_{1}$中,截下一个棱锥\textit{C}­\textit{A}${}_{1}$\textit{DD}${}_{1}$,求棱锥\textit{C}­\textit{A}${}_{1}$\textit{DD}${}_{1}$的体积与剩余部分的体积之比.

\includegraphics*[width=1.00in, height=0.81in, keepaspectratio=false]{image75}

解析:设矩形\textit{ADD}${}_{1}$\textit{A}${}_{1}$的面积为\textit{S},\textit{AB}=\textit{h},

$\mathrm{\therefore}$\textit{VABCD}-\textit{A}${}_{1}$\textit{B}${}_{1}$\textit{C}${}_{1}$\textit{D}${}_{1}$=\textit{VADD}${}_{1}$\textit{A}${}_{1}$-\textit{BCC}${}_{1}$\textit{B}${}_{1}$=\textit{Sh}.

而棱锥\textit{C}­\textit{A}${}_{1}$\textit{DD}${}_{1}$的底面积为$\frac{1}{2}$\textit{S},高为\textit{h},故三棱锥\textit{C}­\textit{A}${}_{1}$\textit{DD}${}_{1}$的体积为:\textit{VC}­\textit{A}${}_{1}$\textit{DD}${}_{1}$=$\frac{1}{3}\times\frac{1}{2}S\times h=\frac{1}{6}S\times h$,

余下部分体积为:\textit{Sh}-$\frac{1}{6}$\textit{Sh}=$\frac{5}{6}$\textit{Sh}.所以棱锥\textit{C}­\textit{A}${}_{1}$\textit{DD}${}_{1}$的体积与剩余部分的体积之比为1$\mathrm{:}$5.

答案:1:5



知识:球的体积和表面积

难度:1

题目:如果三个球的半径之比是1︰2︰3,那么最大球的表面积是其余两个球的表面积之和的(  )

A.$\frac{5}{9}$倍   B.$\frac{9}{5}$倍   C.2倍   D.3倍

解析:设小球半径为1,则大球的表面积\textit{S}${}_{\textrm{大}}$=36$\pi$,\textit{S}${}_{\textrm{小}}$+\textit{S}${}_{\textrm{中}}$=20$\pi$,$\frac{36\pi}{20\pi}=\frac{9}{5}$.

答案:B

知识:球的体积和表面积

难度:1

题目:若两球的体积之和是12\textit{$\pi$},经过两球球心的截面圆周长之和为6\textit{$\pi$},则两球的半径之差为(  )

A.1   B.2   C.3   D.4

解析:设两球的半径分别为\textit{R}、\textit{r}(\textit{R}$\mathrm{>}$\textit{r}),则由题意得$\left \{
\begin{array}{l} 
\frac{4\pi}{3}R^3+\frac{4\pi}{3}r^3=12\pi\\  
2\pi R+2\pi r=6\pi
\end{array} 
\right.$,解得$\left \{
\begin{array}{l} 
R=2\\ 
r=1
\end{array} 
\right.$.故$R-r=1$.

答案:A

知识:球的体积和表面积

难度:1

题目:一个正方体表面积与一个球表面积相等,那么它们的体积比是(  )

A.$\frac{\sqrt{6\pi}}{6}$   B.$\frac{\sqrt{pi}}{2}$  C.$\frac{\sqrt{2\pi}}{2}$   D.$\frac{3\sqrt{pi}}{2\pi}$

解析:由6\textit{a}${}^{2}$=4$\pi$\textit{R}${}^{2}$得$\frac{a}{R}=\sqrt{\frac{2\pi}{3}}$,$\frac{V_1}{V_2}=\frac{a^3}{\frac{4}{3}\pi R^3}=\frac{3}{4\pi}(\sqrt{\frac{2\pi}{3}})^3=\frac{\sqrt{6\pi}}{6}$.

答案:A

知识:球的体积和表面积

难度:1

题目:球的表面积与它的内接正方体的表面积之比是(  )

A.$\frac{\pi}{3}$   B.$\frac{\pi}{4}$   C.$\frac{\pi}{2}$   D.$\pi$

解析:设正方体的棱长为\textit{a},球半径为\textit{R},则3\textit{a}${}^{2}$=4\textit{R}${}^{2}$,$\mathrm{\therefore}$\textit{a}${}^{2}$=$\frac{4}{3}$\textit{R}${}^{2}$,

球的表面积\textit{S}${}_{1}$=4$\pi$\textit{R}${}^{2}$,正方体的表面积 \textit{S}${}_{2}$=6\textit{a}${}^{2}$=6$\mathrm{\times}\frac{4}{3}$\textit{R}${}^{2}$=8\textit{R}${}^{2}$,$\mathrm{\therefore}$\textit{S}${}_{1}$︰\textit{S}${}_{2}$=$\frac{\pi}{2}$.

答案:C

知识:球的体积和表面积

难度:1

题目:正方体的内切球与其外接球的体积之比为(  )

A.1︰$\sqrt{3}$   B.1︰3 C.1︰3$\sqrt{3}$   D.1︰9

解析:设正方体的棱长为\textit{a},则它的内切球的半径为$\frac{1}{2}$\textit{a},它的外接球的半径为$\frac{\sqrt{3}}{2}$\textit{a},故所求体积之比为1︰3$\sqrt{3}$.

答案:C

知识:球的体积和表面积

难度:1

题目:若与球外切的圆台的上、下底面半径分别为\textit{r}、\textit{R},则球的表面积为(  )

A.4$\pi$(\textit{r}+\textit{R})${}^{2}$   B.4$\pi$\textit{r}${}^{2}$\textit{R}${}^{2}$${}^{ }$C.4$\pi$\textit{Rr}   D.$\pi$(\textit{R}+\textit{r})${}^{2}$

解析:解法一:如图,设球的半径为\textit{r}${}_{1}$,则在Rt$\mathrm{\vartriangle}$\textit{CDE}中,\textit{DE}=2\textit{r}${}_{1}$,\textit{CE}=\textit{R}-\textit{r},\textit{DC}=\textit{R}+\textit{r}.由勾股定理得4\textit{r}=(\textit{R}+\textit{r})${}^{2}$-(\textit{R}-\textit{r})${}^{2}$,解得\textit{r}${}_{1}$=$\sqrt{Rr}$.故球的表面积为\textit{D}${}_{\textrm{球}}$=4$\pi$\textit{r}=4$\pi$\textit{Rr}.

\includegraphics*[width=1.13in, height=1.02in, keepaspectratio=false]{image76}

解法二:如图,设球心为\textit{O},球的半径为\textit{r}${}_{1}$,连接\textit{OA}、\textit{OB},则在Rt$\mathrm{\vartriangle}$\textit{AOB}中,\textit{OF}是斜边\textit{AB}上的高.由相似三角形的性质得\textit{OF}${}^{2}$=\textit{BF}·\textit{AF}=\textit{Rr},即\textit{r}=\textit{Rr},故\textit{r}${}_{1}$=$\sqrt{Rr}$,故球的表面积为\textit{S}${}_{\textrm{球}}$=4$\pi$\textit{Rr}.

答案:C

知识:球的体积和表面积

难度:1

题目:(2017·天津理,10)已知一个正方体的所有顶点在一个球面上,若这个正方体的表面积为18,则这个球的体积为\_\_\_\_.

解析:设正方体的棱长为\textit{a},则6\textit{a}${}^{2}$=18,

$\mathrm{\therefore}$\textit{a}=$\sqrt{3}$.

设球的半径为\textit{R},则由题意知2\textit{R}=$\sqrt{a^2+a^2+a^2}$=3,

$\mathrm{\therefore}$\textit{R}=$\frac{3}{2}$.

故球的体积\textit{V}=$\frac{4}{3}\pi R^3=\frac{4}{3}\pi\times(\frac{3}{2})^3\frac{9\pi}{2}$.

答案:$\frac{9\pi}{2}$

知识:球的体积和表面积

难度:1

题目:已知棱长为2的正方体的体积与球\textit{O}的体积相等,则球\textit{O}的半径为\_\_\_\_.



解析:设球\textit{O}的半径为\textit{r},则$\frac{4}{3}\pi r^3=2^3$,

解得\textit{r}=$\sqrt[3]{\frac{6}{\pi}}$.

答案:$\sqrt[3]{\frac{6}{\pi}}$

知识:球的体积和表面积

难度:1

题目:体积相等的正方体、球、等边圆柱(轴截面为正方形)的全面积分别是\textit{S}${}_{1}$、\textit{S}${}_{2}$、\textit{S}${}_{3}$,试比较它们的大小.

解析:设正方体的棱长为\textit{a},球的半径为\textit{R},等边圆柱的底面半径为\textit{r},则\textit{S}${}_{1}$=6\textit{a}${}^{2}$,\textit{S}${}_{2}$=4$\pi$\textit{R}${}^{2}$,\textit{S}${}_{3}$=6$\pi$\textit{r}${}^{2}$.

由题意知,$\frac{4}{3}\pi$\textit{R}${}^{3}$=\textit{a}${}^{3}$=$\pi$\textit{r}${}^{2}$·2\textit{r},

$\mathrm{\therefore}$\textit{R}=$\sqrt[3]{\frac{3}{4\pi}a}$,\textit{r}=$\sqrt[3]{\frac{1}{2\pi}a}$,

$\mathrm{\therefore}$\textit{S}${}_{2}$=4$\pi(\sqrt[3]{\frac{3}{4\pi}}a)$=4$\pi$·$\sqrt[3]{\frac{9}{16\pi^2}}a^2$=$\sqrt[3]{36\pi}$\textit{a}${}^{2}$,

\textit{S}${}_{3}$=6$\pi(\sqrt[3]{\frac{1}{2\pi}}a)^2$=6$\pi$·$\sqrt[3]{\frac{1}{4\pi^2}}$\textit{a}${}^{2}$=$\sqrt[3]{54\pi}$\textit{a}${}^{2}$,

$\mathrm{\therefore}$\textit{S}${}_{2}$$\mathrm{<}$\textit{S}${}_{3}$.

又6\textit{a}${}^{2}$$\mathrm{>}$3$\sqrt[3]{2\pi}$\textit{a}${}^{2}$=$\sqrt[3]{54\pi}$\textit{a}${}^{2}$,即\textit{S}${}_{1}$$\mathrm{>}$\textit{S}${}_{3}$.

$\mathrm{\therefore}$\textit{S}${}_{1}$、\textit{S}${}_{2}$、\textit{S}${}_{3}$的大小关系是\textit{S}${}_{2}$$\mathrm{<}$\textit{S}${}_{3}$$\mathrm{<}$\textit{S}${}_{1}$.

答案:$S_2<S_3<S_1$。

知识:球的体积和表面积

难度:1

题目:某组合体的直观图如图所示,它的中间为圆柱形,左右两端均为半球形,若图中\textit{r}=1,\textit{l}=3,试求该组合体的表面积和体积.

\includegraphics*[width=1.80in, height=0.88in, keepaspectratio=false]{image77}

解析:该组合体的表面积\textit{S}=4$\pi$\textit{r}${}^{2}$+2$\pi$\textit{rl}=4$\pi$$\mathrm{\times}$1${}^{2}$+2$\pi$$\mathrm{\times}$1$\mathrm{\times}$3=10$\pi$.

该组合体的体积\textit{V}=$\pi$\textit{r}${}^{3}$+$\pi$\textit{r}${}^{2}$\textit{l}=$\frac{4}{3}\pi$$\mathrm{\times}$1${}^{3}$+$\pi$$\mathrm{\times}$1${}^{2}$$\mathrm{\times}$3=$\frac{13\pi}{3}$.

答案:$\frac{13\pi}{3}$

知识:球的体积和表面积

难度:2

题目:用一个平行于水平面的平面去截球,得到如图所示的几何体,则它的俯视图是(  )

\includegraphics*[width=0.97in, height=0.93in, keepaspectratio=false]{image78}\includegraphics*[width=2.88in, height=0.81in, keepaspectratio=false]{image79}

解析:选项D为主视图或者侧视图,俯视图中显然应有一个被遮挡的圆,所以内圆是虚线,故选B.

答案:B

知识:球的体积和表面积

难度:2

题目:若一个球的外切正方体的表面积等于6 cm${}^{2}$,则此球的体积为(  )

A. $\frac{\pi}{6}$cm${}^{3}$   B. $\frac{\sqrt{6}\pi}{8}$cm${}^{3}$${}^{ }$  C. $\frac{4\pi}{3}$cm${}^{3}$   D. $\frac{\sqrt{6}\pi}{6}$cm${}^{3}$

解析:设球的半径为\textit{R},正方体的棱长为\textit{a},



$\mathrm{\therefore}$6\textit{a}${}^{2}$=6,$\mathrm{\therefore}$\textit{a}=1.$\mathrm{\therefore}$2\textit{R}=1,$\mathrm{\therefore}$\textit{R}=$\frac{1}{2}$.

$\mathrm{\therefore}$球的体积\textit{V}=$\frac{4}{3}\pi R^3=\frac{4}{3}\pi \times(\frac{1}{2})^3=\frac{\pi}{6}$.

答案:A

知识:球的体积和表面积

难度:2

题目:一个球与一个上、下底面为正三角形,侧面为矩形的棱柱的三个侧面和两个底面都相切,已知这个球的体积为$\frac{32\pi}{3}$,那么这个正三棱柱的体积是(  )

A.96$\sqrt{3}$   B.16$\sqrt{3}$   C.24$\sqrt{3}$   D.48$\sqrt{3}$

解析:由题意可知正三棱柱的高等于球的直径,从棱柱中间截得球的大圆内切于正三角形,正三角形与棱柱底的三角形全等,设三角形边长为\textit{a},球半径为\textit{r},由\textit{V}${}_{\textrm{球}}$=$\frac{4}{3}\mathrm{\times}$$\pi$\textit{r}${}^{3}$=$\frac{32\pi}{3}$解\textit{r}=2.\textit{S}${}_{\textrm{底}}$=$\frac{1}{2}\mathrm{\times}$\textit{a}$\mathrm{\times}\sqrt{a^2-\frac{a^2}{4}}$=$\frac{1}{2}$\textit{a}·\textit{r}$\mathrm{\times}$3,得\textit{a}=2$\sqrt{3}$\textit{r}=4$\sqrt{3}$,所以\textit{V}${}_{\textrm{柱}}$=\textit{S}${}_{\textrm{底}}$·2\textit{r}=48$\sqrt{3}$.

答案:D

知识:球的体积和表面积

难度:2

题目:已知某几何体的三视图如图所示,其中正视图、侧视图均是由三角形与半圆构成,俯视图由圆与内接三角形构成,根据图中的数据可得此几何体的体积为( C )

\includegraphics*[width=1.80in, height=2.28in, keepaspectratio=false]{image80}

A.$\frac{\sqrt{2}\pi}{3}
+\frac{1}{2}$   B.$\frac{4\pi}{3}
+\frac{1}{6}$ C.$\frac{\sqrt{2}\pi}{6}
+\frac{1}{6}$   D.$\frac{2\pi}{3}
+\frac{1}{2}$

解析:由已知的三视图可知原几何体的上方是三棱锥,下方是半球,$\mathrm{\therefore}$\textit{V}=$\frac{1}{3}\times(\frac{1}{2}\times 1\times 1)\times 1+[\frac{4\pi}{3}\times(\frac{\sqrt{2}}{2})^3]\times\frac{1}{2}=\frac{1}{6}+\frac{\sqrt{2}\pi}{6}$,故选C.

答案:C

知识:球的体积和表面积

难度:2

题目:一个半径为2的球体经过切割后,剩余部分几何体的三视图如图所示,则该几何体的表面积为\_\_\_\_.

\includegraphics*[width=1.50in, height=1.66in, keepaspectratio=false]{image81}

解析:该几何体是从一个球体中挖去个$\frac{1}{4}$球体后剩余的部分,所以该几何体的表面积为$\frac{3}{4}\mathrm{\times}$(4$\pi$$\mathrm{\times}$2${}^{2}$)+2$\mathrm{\times}\frac{\pi\times 2^2}{2}$=16$\pi$.

答案:$16\pi$

知识:球的体积和表面积

难度:2

题目:圆柱形容器内盛有高度为8 cm的水,若放入三个相同的球(球的半径与圆柱的底面半径相同)后,水恰好淹没最上面的球(如图所示),则球的半径是\_\_\_\_cm.

\includegraphics*[width=0.65in, height=1.07in, keepaspectratio=false]{image82}

解析:设球的半径为\textit{r},则圆柱形容器的高为6\textit{r},容积为$\pi$\textit{r}${}^{2}$$\mathrm{\times}$6\textit{r}=6$\pi$\textit{r}${}^{3}$,高度为8 cm的水的体积为8$\pi$\textit{r}${}^{2,}$3个球的体积和为3$\mathrm{\times}\frac{4}{3}$$\pi$\textit{r}${}^{3}$=4$\pi$\textit{r}${}^{3}$,由题意得6$\pi$\textit{r}${}^{3}$-8$\pi$\textit{r}${}^{2}$=4$\pi$\textit{r}${}^{3}$,解得\textit{r}=4(cm).

答案:4

知识:球的体积和表面积

难度:3

题目:盛有水的圆柱形容器的内壁底面半径为5 cm,两个直径为5 cm的玻璃小球都浸没于水中,若取出这两个小球,则水面将下降多少?

解析:设取出小球后,容器中水面下降\textit{h} cm,

两个小球的体积为\textit{V}${}_{\textrm{球}}$=2$[\frac{4\pi}{3}\times (\frac{5}{2})^3]=\frac{125}{3}(cm^3)$,

此体积即等于它们的容器中排开水的体积

\textit{V}=$\pi$$\mathrm{\times}$5${}^{2}$$\mathrm{\times}$\textit{h},

所以$\frac{125\pi}{3}$=$\pi$$\mathrm{\times}$5${}^{2}$$\mathrm{\times}$\textit{h},

所以\textit{h}=$\frac{5}{3}$,即若取出这两个小球,则水面将下降$\frac{5}{3}$ cm.

知识:球的体积和表面积

难度:3

题目:已知四面体的各面都是棱长为\textit{a}的正三角形,求它外接球的体积及内切球的半径.



解析:如图,设\textit{SO}${}_{1}$是四面体\textit{S}-\textit{ABC}的高,则外接球的球心\textit{O}在\textit{SO}${}_{1}$上.

\includegraphics*[width=0.85in, height=0.97in, keepaspectratio=false]{image83}

设外接球半径为\textit{R}.

$\mathrm{\because}$四面体的棱长为\textit{a},\textit{O}${}_{1}$为正$\mathrm{\vartriangle}$\textit{ABC}中心,

$\mathrm{\therefore}$\textit{AO}${}_{1}$=$\frac{2}{3}\times\frac{\sqrt{3}}{2}a=\frac{\sqrt{3}}{3}a$,

\textit{SO}${}_{1}$=$\sqrt{SA^2-AO^2}$

$=\sqrt{a^2-\frac{1}{3}a^2}=\frac{\sqrt{6}}{3}a$,

在Rt$\mathrm{\vartriangle}$\textit{OO}${}_{1}$\textit{A}中,\textit{R}${}^{2}$=\textit{AO}+\textit{OO}=\textit{AO}+(\textit{SO}${}_{1}$-\textit{R})${}^{2}$,

即\textit{R}${}^{2}$=($\frac{\sqrt{3}}{3}$\textit{a})${}^{2}$+($\frac{\sqrt{6}}{3}$\textit{a}-\textit{R})${}^{2}$,解得\textit{R}=$\frac{\sqrt{6}}{4}$\textit{a},

$\mathrm{\therefore}$所求外接球体积\textit{V}${}_{\textrm{球}}$=$\frac{4}{3}\pi$\textit{R}${}^{3}$=$\frac{\sqrt{6}}{8}\pi$\textit{a}${}^{3}$.

$\mathrm{\therefore}$\textit{OO}${}_{1}$即为内切球的半径,\textit{OO}${}_{1}$=$\frac{\sqrt{6}}{3}$\textit{a}-$\frac{\sqrt{6}}{12}$\textit{a}=\textit{a},

$\mathrm{\therefore}$内切球的半径为$\frac{\sqrt{6}}{12}$\textit{a}.

答案:$\frac{\sqrt{6}}{12}$\textit{a}


知识:平面

难度:1

题目:若一直线\textit{a}在平面\textit{$\alpha$}内,则正确的图形是(  )

\includegraphics*[width=2.92in, height=0.83in, keepaspectratio=false]{image84}

解析:选项B、C、D中直线\textit{a}在平面\textit{$\alpha$}外,选项A中直线\textit{a}在平面\textit{$\alpha$}内.

答案:A

知识:平面

难度:1

题目:如图所示,下列符号表示错误的是(  )

\includegraphics*[width=1.00in, height=0.31in, keepaspectratio=false]{image85}

A.\textit{l}$\mathrm{\in}$\textit{$\alpha$}   B.\textit{P}$\mathrm{\notin}$\textit{l}   C.\textit{l}$\mathrm{\subset }$\textit{$\alpha$}   D.\textit{P}$\mathrm{\in}$\textit{$\alpha$}

解析:观察图知:\textit{P}$\mathrm{\notin}$\textit{l},\textit{P}$\mathrm{\in}$\textit{$\alpha$},\textit{l}$\mathrm{\subset }$\textit{$\alpha$},则\textit{l}$\mathrm{\in}$\textit{$\alpha$}是错误的.

答案:A

知识:平面

难度:1

题目:下面四个说法(其中\textit{A}、\textit{B}表示点,\textit{a}表示直线,\textit{$\alpha$}表示平面):

\textcircled{1}$\mathrm{\because}$\textit{A}$\mathrm{\subset }$\textit{$\alpha$},\textit{B}$\mathrm{\subset }$\textit{$\alpha$},$\mathrm{\therefore}$\textit{AB}$\mathrm{\subset }$\textit{$\alpha$};

\textcircled{2}$\mathrm{\because}$\textit{A}$\mathrm{\in}$\textit{$\alpha$},\textit{B}$\mathrm{\notin}$\textit{$\alpha$},$\mathrm{\therefore}$\textit{AB}$\mathrm{\notin}$\textit{$\alpha$};

\textcircled{3}$\mathrm{\because}$\textit{A}$\mathrm{\notin}$\textit{a},\textit{a}$\mathrm{\subset }$\textit{$\alpha$},$\mathrm{\therefore}$\textit{A}$\mathrm{\notin}$\textit{$\alpha$};

\textcircled{4}$\mathrm{\because}$\textit{A}$\mathrm{\in}$\textit{a},\textit{a}$\mathrm{\subset }$\textit{$\alpha$},$\mathrm{\therefore}$\textit{A}$\mathrm{\in}$\textit{$\alpha$}.

知识:平面

难度:1

题目:其中表述方式和推理都正确的命题的序号是(  )

A.\textcircled{1}\textcircled{4}   B.\textcircled{2}\textcircled{3}   C.\textcircled{4}  D.\textcircled{3}

解析:①错,应写为\textit{A}$\mathrm{\in}$\textit{$\alpha$},\textit{B}$\mathrm{\in}$\textit{$\alpha$};②错,应写为\textit{AB}$\mathrm{\nsubset}$\textit{$\alpha$};③错,推理错误,有可能\textit{A}$\mathrm{\in}$\textit{$\alpha$};④推理与表述都正确.

答案:C

知识:平面

难度:1

题目:(2016~2017安徽蚌埠高二期中)三条两两平行的直线可以确定平面的个数为(  )

A.0   B.1

C.0或1   D.1或3

解析:当三条直线是同一平面内的平行直线时,确定一个平面,当三条直线是三棱柱侧棱所在的直线时,确定三个平面.

答案:D

知识:平面

难度:1

题目:下列命题中,正确的是(  )

A.经过正方体任意两条面对角线,有且只有一个平面

B.经过正方体任意两条体对角线,有且只有一个平面

C.经过正方体任意两条棱,有且只有一个平面

D.经过正方体任意一条体对角线与任意一条面对角线,有且只有一个平面

解析:因为正方体的四条体对角线相交于同一点(正方体的中心),因此经过正方体任意两条体对角线,有且只有一个平面,故选B.

答案:B

知识:平面

难度:1

题目:如图所示,平面\textit{$\alpha$}$\mathrm{\cap}$\textit{$\beta$}=\textit{l},\textit{A}、\textit{B}$\mathrm{\in}$\textit{$\alpha$},\textit{C}$\mathrm{\in}$\textit{$\beta$}且\textit{C}$\mathrm{\notin}$\textit{l},\textit{AB}$\mathrm{\cap}$\textit{l}=\textit{R},设过\textit{A}、\textit{B}、\textit{C}三点的平面为\textit{$\gamma$},则\textit{$\beta$}$\mathrm{\cap}$\textit{$\gamma$}等于(  )

\includegraphics*[width=0.77in, height=0.82in, keepaspectratio=false]{image86}

A.直线\textit{AC}   B.直线\textit{BC }C.直线\textit{CR}   D.以上都不对

解析:
由\textit{C},\textit{R}是平面\textit{$\beta$}和\textit{$\gamma$}的两个公共点,可知\textit{$\beta$}$\mathrm{\cap}$\textit{$\gamma$}=\textit{CR}.

答案:C

知识:平面

难度:1

题目:在长方体\textit{ABCD}-\textit{A}${}_{1}$\textit{B}${}_{1}$\textit{C}${}_{1}$\textit{D}${}_{1}$的所有棱中,既与\textit{AB}共面,又与\textit{CC}${}_{1}$共面的棱有\_\_\underbar{}\_\_条.

解析:如图,

\includegraphics*[width=1.08in, height=1.15in, keepaspectratio=false]{image87}

由图可知,既与\textit{AB}共面又与\textit{CC}${}_{1}$共面的棱有\textit{CD}、\textit{BC}、\textit{BB}${}_{1}$、\textit{AA}${}_{1}$、\textit{C}${}_{1}$\textit{D}${}_{1}$共5条.

答案:5

知识:平面

难度:1

题目:在正方体\textit{ABCD}-\textit{A}${}_{1}$\textit{B}${}_{1}$\textit{C}${}_{1}$\textit{D}${}_{1}$中,下列说法正确的是\_\_\underbar{}\_\_ (填序号).



(1)直线\textit{AC}${}_{1}$在平面\textit{CC}${}_{1}$\textit{B}${}_{1}$\textit{B}内.

(2)设正方形\textit{ABCD}与\textit{A}${}_{1}$\textit{B}${}_{1}$\textit{C}${}_{1}$\textit{D}${}_{1}$的中心分别为\textit{O}、\textit{O}${}_{1}$,则平面\textit{AA}${}_{1}$\textit{C}${}_{1}$\textit{C}与平面\textit{BB}${}_{1}$\textit{D}${}_{1}$\textit{D}的交线为\textit{OO}${}_{1}$.

(3)由\textit{A}、\textit{C}${}_{1}$、\textit{B}${}_{1}$确定的平面是\textit{ADC}${}_{1}$\textit{B}${}_{1}$.

(4)由\textit{A}、\textit{C}${}_{1}$、\textit{B}${}_{1}$确定的平面与由\textit{A}、\textit{C}${}_{1}$、\textit{D}确定的平面是同一个平面.

解析:
(1)错误.如图所示,点\textit{A}$\mathrm{\notin}$平面\textit{CC}${}_{1}$\textit{B}${}_{1}$\textit{B},所以直线\textit{AC}${}_{1}$$\mathrm{\nsubset}$平面\textit{CC}${}_{1}$\textit{B}${}_{1}$\textit{B}.

\includegraphics*[width=1.27in, height=1.19in, keepaspectratio=false]{image88}

(2)正确.如图所示.

因为\textit{O}$\mathrm{\in}$直线\textit{AC}$\mathrm{\subset }$平面\textit{AA}${}_{1}$\textit{C}${}_{1}$\textit{C},\textit{O}$\mathrm{\in}$直线\textit{BD}$\mathrm{\subset }$平面\textit{BB}${}_{1}$\textit{D}${}_{1}$\textit{D},\textit{O}${}_{1}$$\mathrm{\in}$直线\textit{A}${}_{1}$\textit{C}${}_{1}$$\mathrm{\subset }$平面\textit{AA}${}_{1}$\textit{C}${}_{1}$\textit{C},\textit{O}${}_{1}$$\mathrm{\in}$直线\textit{B}${}_{1}$\textit{D}${}_{1}$$\mathrm{\subset }$平面\textit{BB}${}_{1}$\textit{D}${}_{1}$\textit{D},所以平面\textit{AA}${}_{1}$\textit{C}${}_{1}$\textit{C}与平面\textit{BB}${}_{1}$\textit{D}${}_{1}$\textit{D}的交线为\textit{OO}${}_{1}$.

\includegraphics*[width=1.21in, height=1.15in, keepaspectratio=false]{image89}

(3)(4)都正确,因为\textit{AD}//\textit{B}${}_{1}$\textit{C}${}_{1}$且\textit{AD}=\textit{B}${}_{1}$\textit{C}${}_{1}$,

所以四边形\textit{AB}${}_{1}$\textit{C}${}_{1}$\textit{D}是平行四边形,

所以\textit{A},\textit{B}${}_{1}$,\textit{C}${}_{1}$,\textit{D}共面.

\includegraphics*[width=1.30in, height=1.14in, keepaspectratio=false]{image90}

答案:(2)(3)(4)

知识:平面

难度:1

题目:在正方体\textit{ABCD}-\textit{A}${}_{1}$\textit{B}${}_{1}$\textit{C}${}_{1}$\textit{D}${}_{1}$中,\textit{E}为\textit{AB}的中点,\textit{F}为\textit{AA}${}_{1}$的中点,求证:



\includegraphics*[width=1.17in, height=1.04in, keepaspectratio=false]{image91}

(1)\textit{E}、\textit{C}、\textit{D}${}_{1}$、\textit{F}、四点共面;

(2)\textit{CE}、\textit{D}${}_{1}$\textit{F}、\textit{DA}三线共点.

解析:
(1)分别连接\textit{EF}、\textit{A}${}_{1}$\textit{B}、\textit{D}${}_{1}$\textit{C},

$\mathrm{\because}$\textit{E}、\textit{F}分别是\textit{AB}和\textit{AA}${}_{1}$的中点,

\includegraphics*[width=1.33in, height=1.19in, keepaspectratio=false]{image92}

$\mathrm{\therefore}$\textit{EF}//\textit{A}${}_{1}$\textit{B}且\textit{EF}=\textit{A}${}_{1}$\textit{B}.

又$\mathrm{\because}$\textit{A}${}_{1}$\textit{D}${}_{1}$=\textit{B}${}_{1}$\textit{C}${}_{1}$=\textit{BC},

$\mathrm{\therefore}$四边形\textit{A}${}_{1}$\textit{D}${}_{1}$\textit{CB}是平行四边形,

$\mathrm{\therefore}$\textit{A}${}_{1}$\textit{B}//\textit{CD}${}_{1}$,从而\textit{EF}//\textit{CD}${}_{1}$.

\textit{EF}与\textit{CD}${}_{1}$确定一个平面.

$\mathrm{\therefore}$\textit{E}、\textit{F}、\textit{D}${}_{1}$、\textit{C}四点共面.

(2)$\mathrm{\because}$\textit{EF}=\textit{CD}${}_{1}$,

$\mathrm{\therefore}$直线\textit{D}${}_{1}$\textit{F}和\textit{CE}必相交.设\textit{D}${}_{1}$\textit{F}$\mathrm{\cap}$\textit{CE}=\textit{P},

$\mathrm{\because}$\textit{D}${}_{1}$\textit{F}$\mathrm{\subset }$平面\textit{AA}${}_{1}$\textit{D}${}_{1}$\textit{D},\textit{P}$\mathrm{\in}$\textit{D}${}_{1}$\textit{F},$\mathrm{\therefore}$\textit{P}$\mathrm{\in}$平面\textit{AA}${}_{1}$\textit{D}${}_{1}$\textit{D}.

又\textit{CE}$\mathrm{\subset }$平面\textit{ABCD},\textit{P}$\mathrm{\in}$\textit{EC},$\mathrm{\therefore}$\textit{P}$\mathrm{\in}$平面\textit{ABCD},

即\textit{P}是平面\textit{ABCD}与平面\textit{AA}${}_{1}$\textit{D}${}_{1}$\textit{D}的公共点.

而平面\textit{ABCD}$\mathrm{\cap}$平面\textit{AA}${}_{1}$\textit{D}${}_{1}$\textit{D}=直线\textit{AD},

$\mathrm{\therefore}$\textit{P}$\mathrm{\in}$直线\textit{AD}(公理3),$\mathrm{\therefore}$直线\textit{CE}、\textit{D}${}_{1}$\textit{F}、\textit{DA}三线共点.

答案:答案见解析。

知识:平面

难度:2

题目:

空间中四点可确定的平面有(  )

A.1个    B.3个

C.4个    D.1个或4个或无数个

解析:
当四个点在同一条直线上时,经过这四个点的平面有无数个;当这四个点为三棱锥的四个顶点时,可确定四个平面;当这四个点为平面四边形的四个顶点时,确定一个平面;当其中三点共线于\textit{l},另一点不在直线\textit{l}上时,也确定一个平面,故选D.

答案:D

知识:平面

难度:2

题目:
设\textit{P}表示一个点,\textit{a}、\textit{b}表示两条直线,\textit{$\alpha$}、\textit{$\beta$}表示两个平面,给出下列四个命题,其中正确的命题是(  )

①\textit{P}$\mathrm{\in}$\textit{a},\textit{P}$\mathrm{\in}$\textit{$\alpha$}$\mathrm{\Rightarrow }$\textit{a}$\mathrm{\subset }$\textit{$\alpha$}

②\textit{a}$\mathrm{\cap}$\textit{b}=\textit{P},\textit{b}$\mathrm{\subset }$\textit{$\beta$}$\mathrm{\Rightarrow }$\textit{a}$\mathrm{\subset }$\textit{$\beta$}

③\textit{a}//\textit{b},\textit{a}$\mathrm{\subset }$\textit{$\alpha$},\textit{P}$\mathrm{\in}$\textit{b},\textit{P}$\mathrm{\in}$\textit{$\alpha$}$\mathrm{\Rightarrow }$\textit{b}$\mathrm{\subset }$\textit{$\alpha$}

④\textit{$\alpha$}$\mathrm{\cap}$\textit{$\beta$}=\textit{b},\textit{P}$\mathrm{\in}$\textit{$\alpha$},\textit{P}$\mathrm{\in}$\textit{$\beta$}$\mathrm{\Rightarrow }$\textit{P}$\mathrm{\in}$\textit{b}

A.①②   B.②③   C.①④   D.③④

解析:
当\textit{a}$\mathrm{\cap}$\textit{$\alpha$}=\textit{P}时,\textit{P}$\mathrm{\in}$\textit{a},\textit{P}$\mathrm{\in}$\textit{$\alpha$},但\textit{a}$\mathrm{\nsubset}$\textit{$\alpha$},$\mathrm{\therefore}$①错;

\textit{a}$\mathrm{\cap}$\textit{$\beta$}=\textit{P}时,②错;如图$\mathrm{\because}$\textit{a}//\textit{b},\textit{P}$\mathrm{\in}$\textit{b},$\mathrm{\therefore}$\textit{P}$\mathrm{\notin}$\textit{a},$\mathrm{\therefore}$由直线\textit{a}与点\textit{P}确定唯一平面\textit{$\alpha$},又\textit{a}//\textit{b},由\textit{a}与\textit{b}确定唯一平面\textit{$\beta$},但\textit{$\beta$}经过直线\textit{a}与点\textit{P},$\mathrm{\therefore}$\textit{$\beta$}与\textit{$\alpha$}重合,$\mathrm{\therefore}$\textit{b}$\mathrm{\subset }$\textit{$\alpha$},故③正确;

\includegraphics*[width=1.34in, height=0.63in, keepaspectratio=false]{image93}

两个平面的公共点必在其交线上,故④正确,选D.

答案:D

知识:平面

难度:2

题目:如图,\textit{$\alpha$}$\mathrm{\cap}$\textit{$\beta$}=\textit{l},\textit{A}$\mathrm{\in}$\textit{$\alpha$},\textit{C}$\mathrm{\in}$\textit{$\beta$},\textit{C}$\mathrm{\notin}$\textit{l},直线\textit{AD}$\mathrm{\cap}$\textit{l}=\textit{D},过\textit{A}、\textit{B}、\textit{C}三点确定的平面为\textit{$\gamma$},则平面\textit{$\gamma$}、\textit{$\beta$}的交线必过(  )

\includegraphics*[width=1.35in, height=0.94in, keepaspectratio=false]{image94}

A.点\textit{A}    B.点\textit{B}

C.点\textit{C},但不过点\textit{D}   D.点\textit{C}和点\textit{D}

解析:
\textit{A}、\textit{B}、\textit{C}确定的平面\textit{$\gamma$}与直线\textit{BD}和点\textit{C}确定的平面重合,故\textit{C}、\textit{D}$\mathrm{\in}$\textit{$\gamma$},且\textit{C}、\textit{D}$\mathrm{\in}$\textit{$\beta$},故\textit{C},\textit{D}在\textit{$\gamma$}和\textit{$\beta$}的交线上.

答案:D

知识:平面

难度:2

题目:下列各图均是正六棱柱,\textit{P}、\textit{O}、\textit{R}、\textit{S}分别是所在棱的中点,这四个点不共面的图形是(  )

\includegraphics*[width=2.29in, height=1.66in, keepaspectratio=false]{image95}

解析:
在选项\textit{A}、\textit{B}、\textit{C}中,由棱柱、正六边形、中位线的性质,知均有\textit{PS}//\textit{OR},即在此三个图形中\textit{P}、\textit{O}、\textit{R}、\textit{S}共面,故选D.

答案:D

知识:平面

难度:2

题目:若直线\textit{l}与平面\textit{$\alpha$}相交于点\textit{O}、\textit{A}、\textit{B}$\mathrm{\in}$\textit{l}、\textit{C}、\textit{D}$\mathrm{\in}$\textit{$\alpha$},且\textit{AC}////\textit{BD},则\textit{O}、\textit{C}、\textit{D}三点的位置关系是\_\_\underbar{}\_\_.

\includegraphics*[width=1.07in, height=1.06in, keepaspectratio=false]{image96}

解析:
$\mathrm{\because}$\textit{AC}//\textit{BD},

$\mathrm{\therefore}$\textit{AC}与\textit{BD}确定一个平面,记作平面\textit{$\beta$},则\textit{$\alpha$}$\mathrm{\cap}$\textit{$\beta$}=直线\textit{CD}.

$\mathrm{\because}$\textit{l}$\mathrm{\cap}$\textit{$\alpha$}=\textit{O},$\mathrm{\therefore}$\textit{O}$\mathrm{\in}$\textit{$\alpha$}.

又$\mathrm{\because}$\textit{O}$\mathrm{\in}$\textit{AB}$\mathrm{\subset }$\textit{$\beta$},

$\mathrm{\therefore}$\textit{O}$\mathrm{\in}$直线\textit{CD},$\mathrm{\therefore}$\textit{O}、\textit{C}、\textit{D}三点共线.

答案:共线

知识:平面

难度:2

题目:已知\textit{$\alpha$}、\textit{$\beta$}是不同的平面,\textit{l}、\textit{m}、\textit{n}是不同的直线,\textit{P}为空间中一点.若\textit{$\alpha$}$\mathrm{\cap}$\textit{$\beta$}=\textit{l},\textit{m}$\mathrm{\subset }$\textit{$\alpha$}、\textit{n}$\mathrm{\subset }$\textit{$\beta$}、\textit{m}$\mathrm{\cap}$\textit{n}=\textit{P},则点\textit{P}与直线\textit{l}的位置关系用符号表示为\_\_\textit{}\_\_.

解析:
因为\textit{m}$\mathrm{\subset }$\textit{$\alpha$},\textit{n}$\mathrm{\subset }$\textit{$\beta$},\textit{m}$\mathrm{\cap}$\textit{n}=\textit{P},所以\textit{P}$\mathrm{\in}$\textit{$\alpha$}且\textit{P}$\mathrm{\in}$\textit{$\beta$}.又\textit{$\alpha$}$\mathrm{\in}$\textit{$\beta$}=\textit{l},所以点\textit{P}在直线\textit{l}上,所以\textit{P}$\mathrm{\in}$\textit{l}.

答案:\underbar{P}\underbar{$\mathrm{\in}$}\textit{\underbar{l}}

知识:平面

难度:3

题目:如图,在四面体\textit{A}-\textit{BCD}中作截面\textit{PQR},若\textit{PQ}、\textit{CB}的延长线交于点\textit{M},\textit{RQ}、\textit{DB}的延长线交于点\textit{N},\textit{RP}、\textit{DC}的延长线交于点\textit{K}.

\includegraphics*[width=1.65in, height=1.31in, keepaspectratio=false]{image97}

求证:\textit{M}、\textit{N}、\textit{K}三点共线.

解析:

答案:
$\mathrm{\because}$\textit{M}$\mathrm{\in}$\textit{PQ},直线\textit{PQ}$\mathrm{\subset }$平面\textit{PQR},

\textit{M}$\mathrm{\in}$\textit{BC},直线\textit{BC}$\mathrm{\subset }$平面\textit{BCD},

$\mathrm{\therefore}$\textit{M}是平面\textit{PQR}与平面\textit{BCD}的一个公共点,

$\mathrm{\therefore}$\textit{M}在平面\textit{PQR}与平面\textit{BCD}的交线上.

同理可证,\textit{N}、\textit{K}也在平面\textit{PQR}与平面\textit{BCD}的交线上.

$\mathrm{\therefore}$\textit{M}、\textit{N}、\textit{K}三点共线.

知识:平面

难度:3

题目:如图所示,在棱长为\textit{a}的正方体\textit{ABCD}-\textit{A}${}_{1}$\textit{B}${}_{1}$\textit{C}${}_{1}$\textit{D}${}_{1}$中,\textit{M},\textit{N}分别是\textit{AA}${}_{1}$,\textit{D}${}_{1}$\textit{C}${}_{1}$的中点,过\textit{D},\textit{M},\textit{N}三点的平面与正方体的下底面相交于直线\textit{l}.

\includegraphics*[width=0.87in, height=0.87in, keepaspectratio=false]{image98}

(1)画出直线\textit{l}的位置;

(2)设\textit{l}$\mathrm{\cap}$\textit{A}${}_{1}$\textit{B}${}_{1}$=\textit{P},求线段\textit{PB}${}_{1}$的长.

解析:

答案:
(1)延长\textit{DM}交\textit{D}${}_{1}$\textit{A}${}_{1}$的延长线于\textit{E},连接\textit{NE},则\textit{NE}即为直线\textit{l}的位置.

\includegraphics*[width=1.00in, height=1.02in, keepaspectratio=false]{image99}

(2)$\mathrm{\because}$\textit{M}为\textit{AA}${}_{1}$的中点,\textit{AD}//\textit{ED}${}_{1}$,

$\mathrm{\therefore}$\textit{AD}=\textit{A}${}_{1}$\textit{E}=\textit{A}${}_{1}$\textit{D}${}_{1}$=\textit{a}.

$\mathrm{\because}$\textit{A}${}_{1}$\textit{P}//\textit{D}${}_{1}$\textit{N},且\textit{D}${}_{1}$\textit{N}=\textit{a},

$\mathrm{\therefore}$\textit{A}${}_{1}$\textit{P}=\textit{D}${}_{1}$\textit{N}=\textit{a},

于是\textit{PB}${}_{1}$=\textit{A}${}_{1}$\textit{B}${}_{1}$-\textit{A}${}_{1}$\textit{P}=\textit{a}-\textit{a}=\textit{a}.





知识:空间中直线与直线之间的位置关系

难度:1

题目:异面直线是指(  )

A.空间中两条不相交的直线

B.分别位于两个不同平面内的两条直线

C.平面内的一条直线与平面外的一条直线

D.不同在任何一个平面内的两条直线

解析:
对于A,空间两条不相交的直线有两种可能,一是平行(共面),另一个是异面.$\mathrm{\therefore}$A应排除.

对于B,分别位于两个平面内的直线,既可能平行也可能相交也可异面,如右图,就是相交的情况,$\mathrm{\therefore}$B应排除.

对于C,如右图的\textit{a},\textit{b}可看作是平面\textit{$\alpha$}内的一条直线\textit{a}与平面\textit{$\alpha$}外的一条直线\textit{b},显然它们是相交直线,$\mathrm{\therefore}$C应排除.只有D符合定义.$\mathrm{\therefore}$应选D.

\includegraphics*[width=1.14in, height=1.00in, keepaspectratio=false]{image101}

答案:D

知识:空间中直线与直线之间的位置关系

难度:1

题目:正方体\textit{ABCD}-\textit{A}${}_{1}$\textit{B}${}_{1}$\textit{C}${}_{1}$\textit{D}${}_{1}$中,与对角线\textit{AC}${}_{1}$异面的棱有(  )

A.3条   B.4条   C.6条   D.8条

解析:与\textit{AC}${}_{1}$异面的棱有:\textit{A}${}_{1}$\textit{D}${}_{1}$,\textit{A}${}_{1}$\textit{B}${}_{1}$,\textit{DD}${}_{1}$,\textit{CD},\textit{BC},\textit{BB}${}_{1}$共6条.

答案:C

知识:空间中直线与直线之间的位置关系

难度:1

题目:若\textit{a}、\textit{b}是异面直线,\textit{b}、\textit{c}是异面直线,则(  )

A.\textit{a}//\textit{c}    B.\textit{a}、\textit{c}是异面直线

C.\textit{a}、\textit{c}相交    D.\textit{a}、\textit{c}平行或相交或异面

解析:例如在正方体\textit{ABCD}-\textit{A}${}_{1}$\textit{B}${}_{1}$\textit{C}${}_{1}$\textit{D}${}_{1}$中,取\textit{AB},\textit{CD}所在直线分别为\textit{a},\textit{c},\textit{B}${}_{1}$\textit{C}${}_{1}$所在直线为\textit{b},满足条件要求,此时\textit{a}//\textit{c};又取\textit{AB},\textit{BC}所在直线分别为\textit{a},\textit{c},\textit{DD}${}_{1}$,所在直线为\textit{b},也满足题设要求,此时\textit{a}与\textit{c}相交;又取\textit{AB},\textit{CC}${}_{1}$所在直线分别为\textit{a},\textit{c},\textit{A}${}_{1}$\textit{D}${}_{1}$所在直线为\textit{b},则此时,\textit{a}与\textit{c}异面.故选D.

答案:D

知识:空间中直线与直线之间的位置关系

难度:1

题目:过直线\textit{l}外两点可以作\textit{l}的平行线条数为(  )

A.1条   B.2条   C.3条   D.0条或1条

解析:以如图所示的正方体\textit{ABCD}-\textit{A}${}_{1}$\textit{B}${}_{1}$\textit{C}${}_{1}$\textit{D}${}_{1}$为例.

\includegraphics*[width=1.67in, height=1.70in, keepaspectratio=false]{image102}

令\textit{A}${}_{1}$\textit{B}${}_{1}$所在直线为直线\textit{l},过\textit{l}外的两点\textit{A}、\textit{B}可以作一条直线与\textit{l}平行,过\textit{l}外的两点\textit{B}、\textit{C}不能作直线与\textit{l}平行,故选D.

答案:D

知识:空间中直线与直线之间的位置关系

难度:1

题目:空间四边形\textit{ABCD}中,\textit{E}、\textit{F}分别为\textit{AC}、\textit{BD}中点,若\textit{CD}=2\textit{AB},\textit{EF}$\mathrm{\bot}$\textit{AB},则\textit{EF}与\textit{CD}所成的角为(  )

A.30$\mathrm{{}^\circ}$   B.45$\mathrm{{}^\circ}$   C.60$\mathrm{{}^\circ}$   D.90$\mathrm{{}^\circ}$

解析:取\textit{AD}的中点\textit{H},连\textit{FH}、\textit{EH},在$\mathrm{\vartriangle}$\textit{EFH}中 $\mathrm{\angle}$\textit{EFH}=90$\mathrm{{}^\circ}$,

\includegraphics*[width=1.09in, height=1.00in, keepaspectratio=false]{image103}

\textit{HE}=2\textit{HF},从而$\mathrm{\angle}$\textit{FEH}=30$\mathrm{{}^\circ}$,故选A.

答案:A

知识:空间中直线与直线之间的位置关系

难度:1

题目:下列命题中,正确的结论有(  )

①如果一个角的两边与另一个角的两边分别平行,那么这两个角相等;

②如果两条相交直线和另两条相交直线分别平行,那么这两组直线所成的锐角(或直角)相等;

③如果一个角的两边和另一个角的两边分别垂直,那么这两个角相等或互补;

④如果两条直线同时平行于第三条直线,那么这两条直线互相平行.

A.1个   B.2个   C.3个   D.4个

解析:②④是正确的.

答案:B

知识:空间中直线与直线之间的位置关系

难度:1

题目:已知\textit{E}、\textit{F}、\textit{G}、\textit{H}为空间四边形\textit{ABCD}的边\textit{AB}、\textit{BC}、\textit{CD}、\textit{DA}上的点,若==,==,则四边形\textit{EFGH}形状为\_\_\_\_.

解析:如右图

\includegraphics*[width=1.25in, height=1.37in, keepaspectratio=false]{image104}

在$\mathrm{\vartriangle}$\textit{ABD}中,$\mathrm{\because}$==,

$\mathrm{\therefore}$\textit{EH}//\textit{BD}且\textit{EH}=\textit{BD}.

在$\mathrm{\vartriangle}$\textit{BCD}中,$\mathrm{\because}$==,

$\mathrm{\therefore}$\textit{FG}//\textit{BD}且\textit{FG}=\textit{BD},$\mathrm{\therefore}$\textit{EH}//\textit{FG}且\textit{EH}$\mathrm{>}$\textit{FG},

$\mathrm{\therefore}$四边形\textit{EFGH}为梯形.

答案:梯形

知识:空间中直线与直线之间的位置关系

难度:1

题目:已知棱长为\textit{a}的正方体\textit{ABCD}-\textit{A}$'$\textit{B}$'$\textit{C}$'$\textit{D}$'$中,\textit{M}、\textit{N}分别为\textit{CD}、\textit{AD}的中点,则\textit{MN}与\textit{A}$'$\textit{C}$'$的位置关系是\_\_\_\_.

解析:如图所示,\textit{MN}=\textit{AC},

\includegraphics*[width=1.04in, height=1.04in, keepaspectratio=false]{image105}

又$\mathrm{\because}$\textit{AC}=\textit{A}$'$\textit{C}$'$,

$\mathrm{\therefore}$\textit{MN}=\textit{A}$'$\textit{C}$'$.

答案:平行

知识:空间中直线与直线之间的位置关系

难度:1

题目:在平行六面体\textit{ABCD}-\textit{A}${}_{1}$\textit{B}${}_{1}$\textit{C}${}_{1}$\textit{D}${}_{1}$中,\textit{M}、\textit{N}、\textit{P}分别是\textit{CC}${}_{1}$、\textit{B}${}_{1}$\textit{C}${}_{1}$、\textit{C}${}_{1}$\textit{D}${}_{1}$的中点.

\includegraphics*[width=1.53in, height=0.98in, keepaspectratio=false]{image106}

求证:$\mathrm{\angle}$\textit{NMP}=$\mathrm{\angle}$\textit{BA}${}_{1}$\textit{D}.

解析:

答案:如图,连接\textit{CB}${}_{1}$、\textit{CD}${}_{1}$,$\mathrm{\because}$\textit{CD}=\textit{A}${}_{1}$\textit{B}${}_{1}$,

\includegraphics*[width=1.53in, height=0.98in, keepaspectratio=false]{image107}

$\mathrm{\therefore}$四边形\textit{A}${}_{1}$\textit{B}${}_{1}$\textit{CD}是平行四边形,

$\mathrm{\therefore}$\textit{A}${}_{1}$\textit{D}//\textit{B}${}_{1}$\textit{C}.

$\mathrm{\because}$\textit{M}、\textit{N}分别是\textit{CC}${}_{1}$、\textit{B}${}_{1}$\textit{C}${}_{1}$的中点,

$\mathrm{\therefore}$\textit{MN}//\textit{B}${}_{1}$\textit{C},$\mathrm{\therefore}$\textit{MN}//\textit{A}${}_{1}$\textit{D}.

$\mathrm{\because}$\textit{BC}=\textit{A}${}_{1}$\textit{D}${}_{1}$,$\mathrm{\therefore}$四边形\textit{A}${}_{1}$\textit{BCD}${}_{1}$是平行四边形,

$\mathrm{\therefore}$\textit{A}${}_{1}$\textit{B}//\textit{CD}${}_{1}$.

$\mathrm{\because}$\textit{M}、\textit{P}分别是\textit{CC}${}_{1}$、\textit{C}${}_{1}$\textit{D}${}_{1}$的中点,$\mathrm{\therefore}$\textit{MP}//\textit{CD}${}_{1}$,

$\mathrm{\therefore}$\textit{MP}//\textit{A}${}_{1}$\textit{B},

$\mathrm{\therefore}$$\mathrm{\angle}$\textit{NMP}和$\mathrm{\angle}$\textit{BA}${}_{1}$\textit{D}的两边分别平行且方向都相反,

$\mathrm{\therefore}$$\mathrm{\angle}$\textit{NMP}=$\mathrm{\angle}$\textit{BA}${}_{1}$\textit{D}.

知识:空间中直线与直线之间的位置关系

难度:2

题目:若直线\textit{a}、\textit{b}分别与直线\textit{l}相交且所成的角相等,则\textit{a}、\textit{b}的位置关系是( )

A.异面    B.平行

C.相交    D.三种关系都有可能

解析:以正方体\textit{ABCD}-\textit{A}${}_{1}$\textit{B}${}_{1}$\textit{C}${}_{1}$\textit{D}${}_{1}$为例.

\includegraphics*[width=1.67in, height=1.70in, keepaspectratio=false]{image108}

\textit{A}${}_{1}$\textit{B}${}_{1}$、\textit{AB}所在直线与\textit{BB}${}_{1}$所在直线相交且所成的角相等,\textit{A}${}_{1}$\textit{B}${}_{1}$//\textit{AB};\textit{A}${}_{1}$\textit{B}${}_{1}$、\textit{BC}所在直线与\textit{BB}${}_{1}$所在直线相交且所成的角相等,\textit{A}${}_{1}$\textit{B}${}_{1}$与\textit{BC}是异面直线;\textit{AB}、\textit{BC}所在直线与\textit{AC}所在直线相交且所成的角相等,\textit{AB}与\textit{BC}相交,故选D.

答案:D

知识:空间中直线与直线之间的位置关系

难度:2

题目:空间四边形的对角线互相垂直且相等,顺次连接这个四边形各边中点,所组成的四边形是(  )

A.梯形   B.矩形 C.平行四边形   D.正方形

解析:$\mathrm{\because}$\textit{E}、\textit{F}、\textit{G}、\textit{H}分别为中点,如图.

\includegraphics*[width=1.31in, height=1.02in, keepaspectratio=false]{image109}

$\mathrm{\therefore}$\textit{FG}=\textit{EH}=\textit{BD},

\textit{HG}=\textit{EF}=\textit{AC},

又$\mathrm{\because}$\textit{BD}$\mathrm{\bot}$\textit{AC}且\textit{BD}=\textit{AC},

$\mathrm{\therefore}$\textit{FG}$\mathrm{\bot}$\textit{HG}且\textit{FG}=\textit{HG},$\mathrm{\therefore}$四边形\textit{EFGH}为正方形.

答案:D

知识:空间中直线与直线之间的位置关系

难度:2

题目:点\textit{E}、\textit{F}分别是三棱锥\textit{P}-\textit{ABC}的棱\textit{AP}、\textit{BC}的中点,\textit{AB}=6,\textit{PC}=8,\textit{EF}=5,则异面直线\textit{AB}与\textit{PC}所成的角为(  )

A.60$\mathrm{{}^\circ}$   B.45$\mathrm{{}^\circ}$   C.30$\mathrm{{}^\circ}$   D.90$\mathrm{{}^\circ}$

解析:如图,取\textit{PB}的中点\textit{G},连结\textit{EG}、\textit{FG},则

\includegraphics*[width=1.27in, height=1.08in, keepaspectratio=false]{image110}

\textit{EG}=\textit{AB},\textit{GF}=\textit{PC},则$\mathrm{\angle}$\textit{EGF}(或其补角)即为\textit{AB}与\textit{PC}所成的角,在$\mathrm{\vartriangle}$\textit{EFG}中,\textit{EG}=\textit{AB}=3,\textit{FG}=\textit{PC}=4,\textit{EF}=5,所以$\mathrm{\angle}$\textit{EGF}=90$\mathrm{{}^\circ}$.

答案:D

知识:空间中直线与直线之间的位置关系

难度:2

题目:如图所示,空间四边形\textit{ABCD}的对角线\textit{AC}=8,\textit{BD}=6,\textit{M}、\textit{N}分别为\textit{AB}、\textit{CD}的中点,并且异面直线\textit{AC}与\textit{BD}所成的角为90$\mathrm{{}^\circ}$,则\textit{MN}等于(  )

\includegraphics*[width=1.40in, height=1.36in, keepaspectratio=false]{image111}

A.5   B.6 C.8   D.10

解析:如图,取\textit{AD}的中点\textit{P},连接\textit{PM}、\textit{PN},则\textit{BD}//\textit{PM},\textit{AC}//\textit{PN},$\mathrm{\therefore}$$\mathrm{\angle}$\textit{MPN}即异面直线\textit{AC}与\textit{BD}所成的角,$\mathrm{\therefore}$$\mathrm{\angle}$\textit{MPN}=90$\mathrm{{}^\circ}$,\textit{PN}=\textit{AC}=4,\textit{PM}=\textit{BD}=3,$\mathrm{\therefore}$\textit{MN}=5.

\includegraphics*[width=1.08in, height=0.81in, keepaspectratio=false]{image112}

答案:A

知识:空间中直线与直线之间的位置关系

难度:2

题目:如图正方体\textit{ABCD}-\textit{A}${}_{1}$\textit{B}${}_{1}$\textit{C}${}_{1}$\textit{D}${}_{1}$中,与\textit{AD}${}_{1}$异面且与\textit{AD}${}_{1}$所成的角为90$\mathrm{{}^\circ}$的面对角线(面对角线是指正方体各个面上的对角线)共有\_\_\_\_条.

\includegraphics*[width=1.08in, height=1.01in, keepaspectratio=false]{image113}

解析:与\textit{AD}${}_{1}$异面的面对角线分别为:\textit{A}${}_{1}$\textit{C}${}_{1}$、\textit{B}${}_{1}$\textit{C}、\textit{BD}、\textit{BA}${}_{1}$、\textit{C}${}_{1}$\textit{D},其中只有\textit{B}${}_{1}$\textit{C}和\textit{AD}${}_{1}$所成的角为90$\mathrm{{}^\circ}$.

答案:1

知识:空间中直线与直线之间的位置关系

难度:2

题目:一个正方体纸盒展开后如图所示,在原正方体纸盒中有如下结论:

\includegraphics*[width=1.31in, height=1.02in, keepaspectratio=false]{image114}

①\textit{AB}//\textit{CM};

②\textit{EF}与\textit{MN}是异面直线;

③\textit{MN}//\textit{CD}.

以上结论中正确结论的序号为\_\_\_\_.

解析:把正方体平面展开图还原到原来的正方体,如图所示,\textit{EF}与\textit{MN}是异面直线,\textit{AB}//\textit{CM},\textit{MN}$\mathrm{\bot}$\textit{CD},只有①②正确.

\includegraphics*[width=1.32in, height=1.17in, keepaspectratio=false]{image115}

答案:①②

知识:空间中直线与直线之间的位置关系

难度:3

题目:已知空间四边形\textit{ABCD}中,\textit{AB}$\mathrm{\neq}$\textit{AC},\textit{BD}=\textit{BC},\textit{AE}是$\mathrm{\vartriangle}$\textit{ABC}的边\textit{BC}上的高,\textit{DF}是$\mathrm{\vartriangle}$\textit{BCD}的边\textit{BC}上的中线,求证:\textit{AE}与\textit{DF}是异面直线.

解析:

答案:由已知,得\textit{E}、\textit{F}不重合.

设$\mathrm{\vartriangle}$\textit{BCD}所在平面为\textit{$\alpha$},

则\textit{DF}$\mathrm{\subset }$\textit{$\alpha$},\textit{A}$\mathrm{\notin}$\textit{$\alpha$},\textit{E}$\mathrm{\in}$\textit{$\alpha$},\textit{E}$\mathrm{\notin}$\textit{DF},

$\mathrm{\therefore}$\textit{AE}与\textit{DF}异面.

知识:空间中直线与直线之间的位置关系

难度:3

题目:梯形\textit{ABCD}中,\textit{AB}//\textit{CD},\textit{E}、\textit{F}分别为\textit{BC}和\textit{AD}的中点,将平面\textit{DCEF}沿\textit{EF}翻折起来,使\textit{CD}到\textit{C}$'$\textit{D}$'$的位置,\textit{G}、\textit{H}分别为\textit{AD}$'$和\textit{BC}$'$的中点,求证:四边形\textit{EFGH}为平行四边形.


解析:

答案:$\mathrm{\because}$梯形\textit{ABCD}中,\textit{AB}//\textit{CD},

\textit{E}、\textit{F}分别为\textit{BC}、\textit{AD}的中点,

$\mathrm{\therefore}$\textit{EF}//\textit{AB}且\textit{EF}=(\textit{AB}+\textit{CD}),

又\textit{C}$'$\textit{D}$'$//\textit{EF},\textit{EF}//\textit{AB},$\mathrm{\therefore}$\textit{C}$'$\textit{D}$'$//\textit{AB}.

$\mathrm{\because}$\textit{G}、\textit{H}分别为\textit{AD}$'$、\textit{BC}$'$的中点,

\includegraphics*[width=2.83in, height=1.04in, keepaspectratio=false]{image116}

$\mathrm{\therefore}$\textit{GH}//\textit{AB}且\textit{GH}=(\textit{AB}+\textit{C}$'$\textit{D}$'$)=(\textit{AB}+\textit{CD}),

$\mathrm{\therefore}$\textit{GH}=\textit{EF},$\mathrm{\therefore}$四边形\textit{EFGH}为平行四边形.

知识:空间中直线与平面之间的位置关系、平面与平面之间的位置关系

难度:1

题目:正方体的六个面中相互平行的平面有(  )

A.2对   B.3对   C.4对   D.5对

解析:正方体的六个面中有3对相互平行的平面.

\includegraphics*[width=1.14in, height=1.04in, keepaspectratio=false]{image118}

答案:B

知识:空间中直线与平面之间的位置关系、平面与平面之间的位置关系

难度:1

题目:三棱台\textit{ABC}-\textit{A}$'$\textit{B}$'$\textit{C}$'$的一条侧棱\textit{AA}$'$所在直线与平面\textit{BCC}$'$\textit{B}$'$之间的关系是(  )

A.相交  B.平行

C.直线在平面内  D.平行或直线在平面内

解析:由棱台的定义知,棱台的所有侧棱所在的直线都交于同一点,而任一侧面所在的平面由两条侧棱所在直线所确定,故这条侧棱与不含这条侧棱的任意一个侧面所在的平面都相交.

答案:A

知识:空间中直线与平面之间的位置关系、平面与平面之间的位置关系

难度:1

题目:若直线\textit{a}//平面\textit{$\alpha$},直线\textit{b}//平面\textit{$\alpha$},则\textit{a}与\textit{b}的位置关系是(  )

A.平行    B.相交

C.异面    D.以上都有可能

解析:如图所示,长方体\textit{ABCD}-\textit{A}${}_{1}$\textit{B}${}_{1}$\textit{C}${}_{1}$\textit{D}${}_{1}$中,\textit{A}${}_{1}$\textit{B}${}_{1}$//平面\textit{AC},\textit{A}${}_{1}$\textit{D}${}_{1}$//平面\textit{AC},有\textit{A}${}_{1}$\textit{B}${}_{1}$$\mathrm{\cap}$\textit{A}${}_{1}$\textit{D}${}_{1}$=\textit{A}${}_{1}$;又\textit{D}${}_{1}$\textit{C}${}_{1}$//平面\textit{AC},有\textit{A}${}_{1}$\textit{B}${}_{1}$//\textit{D}${}_{1}$\textit{C}${}_{1}$;取\textit{BB}${}_{1}$和\textit{CC}${}_{1}$的中点\textit{M}、\textit{N},则\textit{MN}//\textit{B}${}_{1}$\textit{C}${}_{1}$,则\textit{MN}//平面\textit{AC},有\textit{A}${}_{1}$\textit{B}${}_{1}$与\textit{MN}异面,故选D.

\includegraphics*[width=1.05in, height=1.08in, keepaspectratio=false]{image119}

答案:D

知识:空间中直线与平面之间的位置关系、平面与平面之间的位置关系

难度:1

题目:如果直线\textit{a}//平面\textit{$\alpha$},那么直线\textit{a}与平面\textit{$\alpha$}内的(  )

A.唯一一条直线不相交 B.仅两条相交直线不相交

C.仅与一组平行直线不相交 D.任意一条直线都不相交

解析:根据直线和平面平行定义,易知排除A、B.对于C,仅有一组平行线不相交,不正确,应排除C.与平面\textit{$\alpha$}内任意一条直线都不相交,才能保证直线\textit{a}与平面\textit{$\alpha$}平行,$\mathrm{\therefore}$D正确.

答案:D

知识:空间中直线与平面之间的位置关系、平面与平面之间的位置关系

难度:1

题目:平面\textit{$\alpha$}//平面\textit{$\beta$},直线\textit{a}//\textit{$\alpha$},则(  )

A.\textit{a}//\textit{$\beta$}   B.\textit{a}在面\textit{$\beta$}上 C.\textit{a}与\textit{$\beta$}相交   D.\textit{a}//\textit{$\beta$}或\textit{a}$\mathrm{\subset }$\textit{$\beta$}

解析:如图(1)满足\textit{a}//\textit{$\alpha$},\textit{$\alpha$}//\textit{$\beta$},此时\textit{a}//\textit{$\beta$};

如图(2)满足\textit{a}//\textit{$\alpha$},\textit{$\alpha$}//\textit{$\beta$},此时\textit{a}$\mathrm{\subset }$\textit{$\beta$},故选D.

\includegraphics*[width=1.86in, height=1.16in, keepaspectratio=false]{image120}

答案:D

知识:空间中直线与平面之间的位置关系、平面与平面之间的位置关系

难度:1

题目:设\textit{P}是异面直线\textit{a},\textit{b}外一点,则过\textit{P}与\textit{a},\textit{b}都平行的直线有(  )条

A.1   B.2   C.0   D.0或1

解析:反证法.若存在直线\textit{c}//\textit{a},且\textit{c}//\textit{b},则\textit{a}//\textit{b}与\textit{a},\textit{b}异面矛盾.故选C.

答案:C

知识:空间中直线与平面之间的位置关系、平面与平面之间的位置关系

难度:1

题目:如图,在正方体\textit{ABCD}-\textit{A}${}_{1}$\textit{B}${}_{1}$\textit{C}${}_{1}$\textit{D}${}_{1}$中判断下列位置关系:

\includegraphics*[width=1.17in, height=1.17in, keepaspectratio=false]{image121}

(1)\textit{AD}${}_{1}$所在的直线与平面\textit{BCC}${}_{1}$的位置关系是\_\_\_\_;

(2)平面\textit{A}${}_{1}$\textit{BC}${}_{1}$与平面\textit{ABCD}的位置关系是\_\_\_\_.

答案:(1)平行;(2)相交

知识:空间中直线与平面之间的位置关系、平面与平面之间的位置关系

难度:1

题目:两个不重合的平面可以把空间分成\_\_\_\_部分.

解析:两平面平行时,把空间分成三部分.两平面相交时,把空间分成四部分.

答案:三或四

知识:空间中直线与平面之间的位置关系、平面与平面之间的位置关系

难度:1

题目:如图所示,直线\textit{A}$'$\textit{B}与长方体\textit{ABCD}-\textit{A}$'$\textit{B}$'$\textit{C}$'$\textit{D}$'$的六个面所在的平面有什么位置关系?平面\textit{A}$'$\textit{B}与长方体\textit{ABCD}-\textit{A}$'$\textit{B}$'$\textit{C}$'$\textit{D}$'$的其余五个面的位置关系如何?



\includegraphics*[width=1.21in, height=1.04in, keepaspectratio=false]{image122}

解析:

答案:$\mathrm{\because}$直线\textit{A}$'$\textit{B}与平面\textit{ABB}$'$\textit{A}$'$有无数个公共点,

$\mathrm{\therefore}$直线\textit{A}$'$\textit{B}在平面\textit{ABB}$'$\textit{A}$'$内.

$\mathrm{\because}$直线\textit{A}$'$\textit{B}与平面\textit{ABCD},平面\textit{BCC}$'$\textit{B}$'$都有且只有一个公共点\textit{B},

$\mathrm{\therefore}$直线\textit{A}$'$\textit{B}与平面\textit{ABCD},平面\textit{BCC}$'$\textit{B}$'$相交.

$\mathrm{\because}$直线\textit{A}$'$\textit{B}与平面\textit{ADD}$'$\textit{A}$'$,平面\textit{A}$'$\textit{B}$'$\textit{C}$'$\textit{D}$'$都有且只有一个公共点\textit{A}$'$,

$\mathrm{\therefore}$直线\textit{A}$'$\textit{B}与平面\textit{ADD}$'$\textit{A}$'$,平面\textit{A}$'$\textit{B}$'$\textit{C}$'$\textit{D}$'$相交.

$\mathrm{\because}$直线\textit{A}$'$\textit{B}与平面\textit{DCC}$'$\textit{D}$'$没有公共点,

$\mathrm{\therefore}$直线\textit{A}$'$\textit{B}与平面\textit{DCC}$'$\textit{D}$'$平行.

平面\textit{A}$'$\textit{B}//平面\textit{CD}$'$,

平面\textit{A}$'$\textit{B}与平面\textit{AD}$'$、平面\textit{BC}$'$、平面\textit{AC}平面\textit{A}$'$\textit{C}$'$都相交.

知识:空间中直线与平面之间的位置关系、平面与平面之间的位置关系

难度:1

题目:如图所示,已知平面\textit{$\alpha$}$\mathrm{\cap}$\textit{$\beta$}=\textit{l},点\textit{A}$\mathrm{\in}$\textit{$\alpha$},点\textit{B}$\mathrm{\in}$\textit{$\alpha$},点\textit{C}$\mathrm{\in}$\textit{$\beta$},且\textit{A}$\mathrm{\notin}$\textit{l},\textit{B}$\mathrm{\notin}$\textit{l},直线\textit{AB}与\textit{l}不平行,那么平面\textit{ABC}与平面\textit{$\beta$}的交线与\textit{l}有什么关系?证明你的结论.

\includegraphics*[width=0.89in, height=0.67in, keepaspectratio=false]{image123}

解析:平面\textit{ABC}与平面\textit{$\beta$}的交线与\textit{l}相交.

答案:$\mathrm{\because}$\textit{AB}与\textit{l}不平行,且\textit{AB}$\mathrm{\subset }$\textit{$\alpha$},\textit{l}$\mathrm{\subset }$\textit{$\alpha$},

$\mathrm{\therefore}$\textit{AB}与\textit{l}一定相交.设\textit{AB}$\mathrm{\cap}$\textit{l}=\textit{P},

则\textit{P}$\mathrm{\in}$\textit{AB},\textit{P}$\mathrm{\in}$\textit{l}.

又$\mathrm{\because}$\textit{AB}$\mathrm{\subset }$平面\textit{ABC},\textit{l}$\mathrm{\subset }$\textit{$\beta$},

$\mathrm{\therefore}$\textit{P}$\mathrm{\in}$平面\textit{ABC},\textit{P}$\mathrm{\in}$\textit{$\beta$}.

$\mathrm{\therefore}$点\textit{P}是平面\textit{ABC}与平面\textit{$\beta$}的一个公共点,而点\textit{C}也是平面\textit{ABC}与平面\textit{$\beta$}的一个公共点,且\textit{P},\textit{C}是不同的两点,

$\mathrm{\therefore}$直线\textit{PC}就是平面\textit{ABC}与平面\textit{$\beta$}的交线.

即平面\textit{ABC}$\mathrm{\cap}$平面\textit{$\beta$}=\textit{PC},而\textit{PC}$\mathrm{\cap}$\textit{l}=\textit{P},

$\mathrm{\therefore}$平面\textit{ABC}与平面\textit{$\beta$}的交线与\textit{l}相交.

知识:空间中直线与平面之间的位置关系、平面与平面之间的位置关系

难度:2

题目:直线\textit{a}在平面\textit{$\gamma$}外,则(  )

A.\textit{a}//\textit{$\gamma$}    B.\textit{a}与\textit{$\gamma$}至少有一个公共点

C.\textit{a}$\mathrm{\cap}$\textit{$\gamma$}=\textit{A}    D.\textit{a}与\textit{$\gamma$}至多有一个公共点

解析:直线\textit{$\alpha$}在平面\textit{$\gamma$}外,包括两种情况,一种是平行,另一种相交,故选D.

答案:D

知识:空间中直线与平面之间的位置关系、平面与平面之间的位置关系

难度:2

题目:若平面\textit{$\alpha$}//平面\textit{$\beta$},则(  )

A.平面\textit{$\alpha$}内任一条直线与平面\textit{$\beta$}平行

B.平面\textit{$\alpha$}内任一条直线与平面\textit{$\beta$}内任一条直线平行

C.平面\textit{$\alpha$}内存在一条直线与平面\textit{$\beta$}不平行

D.平面\textit{$\alpha$}内一条直线与平面\textit{$\beta$}内一条直线有可能相交

解析:

答案:A

知识:空间中直线与平面之间的位置关系、平面与平面之间的位置关系

难度:2

题目:若三个平面两两相交,且三条交线互相平行,则这三个平面把空间分成(  )

A.5部分   B.6部分   C.7部分   D.8部分

解析:垂直于交线的截面如图,把空间分成7部分,故选C.

\includegraphics*[width=1.00in, height=0.71in, keepaspectratio=false]{image124}

答案:C

知识:空间中直线与平面之间的位置关系、平面与平面之间的位置关系

难度:2

题目:如图所示,用符号语言可表示为(  )

\includegraphics*[width=1.29in, height=0.79in, keepaspectratio=false]{image125}

A.\textit{$\alpha$}$\mathrm{\cap}$\textit{$\beta$}=\textit{l} B.\textit{$\alpha$}//\textit{$\beta$},\textit{l}$\mathrm{\in}$\textit{$\alpha$} C.\textit{l}//\textit{$\beta$},\textit{l}$\mathrm{\nsubset}$\textit{$\alpha$}   D.\textit{$\alpha$}//\textit{$\beta$},\textit{l}$\mathrm{\subset }$\textit{$\alpha$}

解析:由图可知,\textit{$\alpha$}//\textit{$\beta$},\textit{l}$\mathrm{\subset }$\textit{$\alpha$}.

答案:D

知识:空间中直线与平面之间的位置关系、平面与平面之间的位置关系

难度:2

题目:下列命题正确的有\_\_\_\_.

①若直线与平面有两个公共点,则直线在平面内;

②若直线\textit{l}上有无数个点不在平面\textit{$\alpha$}内,则\textit{l}//\textit{$\alpha$};

③若直线\textit{l}与平面\textit{$\alpha$}相交,则\textit{l}与平面\textit{$\alpha$}内的任意直线都是异面直线;

④如果两条异面直线中的一条与一个平面平行,则另一条直线一定与该平面相交;

⑤若直线\textit{l}与平面\textit{$\alpha$}平行,则\textit{l}与平面\textit{$\alpha$}内的直线平行或异面;

⑥若平面\textit{$\alpha$}//平面\textit{$\beta$},直线\textit{a}$\mathrm{\subset }$\textit{$\alpha$},直线\textit{b}$\mathrm{\subset }$\textit{$\beta$},则直线\textit{a}//\textit{b}.

解析:①显然是正确的;②中,直线\textit{l}还可能与\textit{$\alpha$}相交,所以②是错误的;③中,直线\textit{l}和平面\textit{$\alpha$}内过\textit{l}与\textit{$\alpha$}交点的直线都相交而不是异面,所以③是错误的;④中,异面直线中的另一条直线和该平面的关系不能具体确定,它们可以相交,可以平行,还可以在该平面内,所以④是错误的;⑤中,直线\textit{l}与平面\textit{$\alpha$}没有公共点,所以直线\textit{l}与平面\textit{$\alpha$}内的直线没有公共点,即它们平行或异面,所以⑤是正确的;⑥中,分别在两个平行平面内的直线可以平行,也可以异面,所以⑥是错误的.

答案:①⑤

知识:空间中直线与平面之间的位置关系、平面与平面之间的位置关系

难度:2

题目:将一个长方体的四个侧面和两个底面延展成平面后,可将空间分成\_\_\_\_部分.

解析:

答案:27

知识:空间中直线与平面之间的位置关系、平面与平面之间的位置关系

难度:3

题目:已知三个平面\textit{$\alpha$}、\textit{$\beta$}、\textit{$\gamma$},如果\textit{$\alpha$}//\textit{$\beta$},\textit{$\gamma$}$\mathrm{\cap}$\textit{$\alpha$}=\textit{a},\textit{$\gamma$}$\mathrm{\cap}$\textit{$\beta$}=\textit{b},且直线\textit{c}$\mathrm{\subset }$\textit{$\beta$},\textit{c}//\textit{b}.

(1)判断\textit{c}与\textit{$\alpha$}的位置关系,并说明理由;

(2)判断\textit{c}与\textit{a}的位置关系,并说明理由.

解析:

答案:(1)\textit{c}//\textit{$\alpha$},因为\textit{$\alpha$}//\textit{$\beta$},所以\textit{$\alpha$}与\textit{$\beta$}没有公共点.又\textit{c}$\mathrm{\subset }$\textit{$\beta$},所以\textit{c}与\textit{$\alpha$}无公共点,所以\textit{c}//\textit{$\alpha$}.

(2)\textit{c}//\textit{a},因为\textit{$\alpha$}//\textit{$\beta$},所以\textit{$\alpha$}与\textit{$\beta$}没有公共点.又\textit{$\gamma$}$\mathrm{\cap}$\textit{$\alpha$}=\textit{a},\textit{$\gamma$}

$\mathrm{\cap}$\textit{$\beta$}=\textit{b},则\textit{a}$\mathrm{\subset }$\textit{$\alpha$},\textit{b}$\mathrm{\subset }$\textit{$\beta$},且\textit{a}、\textit{b}$\mathrm{\subset }$\textit{$\gamma$},所以\textit{a}、\textit{b}没有公共点.由于\textit{a},\textit{b}都在平面\textit{$\gamma$}内,因此\textit{a}//\textit{b}.又\textit{c}//\textit{b},所以\textit{c}//\textit{a}.

如图,在正方体\textit{ABCD}-\textit{A}${}_{1}$\textit{B}${}_{1}$\textit{C}${}_{1}$\textit{D}${}_{1}$中,\textit{E}是\textit{AA}${}_{1}$的中点,画出过\textit{D}${}_{1}$、\textit{C}、\textit{E}的平面与平面\textit{ABB}${}_{1}$\textit{A}${}_{1}$的交线,并说明理由.

\includegraphics*[width=1.07in, height=1.03in, keepaspectratio=false]{image126}

解析:

答案:如图,取\textit{AB}的中点\textit{F},连接\textit{EF}、\textit{A}${}_{1}$\textit{B}、\textit{CF}.

\includegraphics*[width=1.11in, height=1.07in, keepaspectratio=false]{image127}

$\mathrm{\because}$\textit{E}是\textit{AA}${}_{1}$的中点,$\mathrm{\therefore}$\textit{EF}//\textit{A}${}_{1}$\textit{B}.

知识:空间中直线与平面之间的位置关系、平面与平面之间的位置关系

难度:3

题目:在正方体\textit{ABCD}-\textit{A}${}_{1}$\textit{B}${}_{1}$\textit{C}${}_{1}$\textit{D}${}_{1}$中,\textit{A}${}_{1}$\textit{D}${}_{1}$//\textit{BC},\textit{A}${}_{1}$\textit{D}${}_{1}$=\textit{BC},

$\mathrm{\therefore}$四边形\textit{A}${}_{1}$\textit{BCD}${}_{1}$是平行四边形.

$\mathrm{\therefore}$\textit{A}${}_{1}$\textit{B}//\textit{CD}${}_{1}$,$\mathrm{\therefore}$\textit{EF}//\textit{CD}${}_{1}$.

$\mathrm{\therefore}$\textit{E}、\textit{F}、\textit{C}、\textit{D}${}_{1}$四点共面.

$\mathrm{\because}$\textit{E}$\mathrm{\in}$平面\textit{ABB}${}_{1}$\textit{A}${}_{1}$,\textit{E}$\mathrm{\in}$平面\textit{D}${}_{1}$\textit{CE},

\textit{F}$\mathrm{\in}$平面\textit{ABB}${}_{1}$\textit{A}${}_{1}$,\textit{F}$\mathrm{\in}$平面\textit{D}${}_{1}$\textit{CE},

$\mathrm{\therefore}$平面\textit{ABB}${}_{1}$\textit{A}${}_{1}$$\mathrm{\cap}$平面\textit{D}${}_{1}$\textit{CE}=\textit{EF}.

$\mathrm{\therefore}$过\textit{D}${}_{1}$、\textit{C}、\textit{E}的平面与平面\textit{ABB}${}_{1}$\textit{A}${}_{1}$的交线为\textit{EF}.


知识:直线与平面平行的判定

难度:1

题目:圆台的底面内的任意一条直径与另一个底面的位置关系是(  )

A.平行    B.相交 C.在平面内   D.不确定

解析:圆台底面内的任意一条直径与另一个底面无公共点,则它们平行.

答案:A

知识:直线与平面平行的判定

难度:1

题目:若\textit{l}//\textit{$\alpha$},\textit{m}$\mathrm{\subset }$\textit{$\alpha$},则\textit{l}与\textit{m}的关系是(  )

A.\textit{l}//\textit{m}    B.\textit{l}与\textit{m}异面 

C.\textit{l}与\textit{m}相交    D.\textit{l}与\textit{m}无公共点

解析:\textit{l}与\textit{$\alpha$}无公共点,$\mathrm{\therefore}$\textit{l}与\textit{m}无公共点.

答案:D

知识:直线与平面平行的判定

难度:1

题目:在三棱锥\textit{A}-\textit{BCD}中,\textit{E}、\textit{F}分别是\textit{AB}和\textit{BC}上的点,若\textit{AE}︰\textit{EB}=\textit{CF}︰\textit{FB}=2︰5,则直线\textit{AC}与平面\textit{DEF}的位置关系是(  )

A.平行    B.相交

C.直线\textit{AC}在平面\textit{DEF}内   D.不能确定

解析:如图所示,

\includegraphics*[width=1.25in, height=0.94in, keepaspectratio=false]{image129}

$\mathrm{\because}$\textit{AE}︰\textit{EB}=\textit{CF}︰\textit{FB}=2︰5,$\mathrm{\therefore}$\textit{EF}//\textit{AC}.又\textit{EF}$\mathrm{\subset }$平面\textit{DEF},\textit{AC}$\mathrm{\nsubset}$平面\textit{DEF},$\mathrm{\therefore}$\textit{AC}//平面\textit{DEF}.

答案:A

知识:直线与平面平行的判定

难度:1

题目:\textit{a}//\textit{b},且\textit{a}与平面\textit{$\alpha$}相交,那么直线\textit{b}与平面\textit{$\alpha$}的位置关系是(  )

A.必相交    B.有可能平行

C.相交或平行    D.相交或在平面内

解析:如图所示:

\includegraphics*[width=1.11in, height=0.64in, keepaspectratio=false]{image130}

答案:A

知识:直线与平面平行的判定

难度:1

题目:下列命题:

(1)如果一条直线不在平面内,则这条直线就与这个平面平行;

(2)过直线外一点,可以作无数个平面与这条直线平行;

(3)如果一条直线与平面平行,则它与平面内的任何直线平行.

其中正确命题的个数为(  )

A.0个   B.1个   C.2个   D.3个

解析:(1)中,直线可能与平面相交,故(1)错;(2)是正确的;(3)中,一条直线与平面平行,则它与平面内的直线平行或异面,故(3)错.

答案:B

知识:直线与平面平行的判定

难度:1

题目:下列四个正方体图形中,\textit{A}、\textit{B}为正方体的两个顶点,\textit{M}、\textit{N}、\textit{P}分别为其所在棱的中点,能得出\textit{AB}//平面\textit{MNP}的图形的序号是(  )

\includegraphics*[width=3.12in, height=0.92in, keepaspectratio=false]{image131}

A.①③   B.①④   C.①③   D.②④

解析:对于选项①,取\textit{NP}中点\textit{G},由三角形中位线性质易证:\textit{MG}//\textit{AB},故①正确;对于选项④,易证\textit{NP}//\textit{AB},故选B.

答案:B

知识:直线与平面平行的判定

难度:1

题目:已知\textit{l}、\textit{m}是两条直线,\textit{$\alpha$}是平面,若要得到``\textit{l}//\textit{$\alpha$}'',则需要在条件``\textit{m}$\mathrm{\subset }$\textit{$\alpha$},\textit{l}//\textit{m}''中另外添加的一个条件是\_\_\_\_.

解析:根据直线与平面平行的判定定理,知需要添加的一个条件是``\textit{l}$\mathrm{\nsubset}$\textit{$\alpha$}''.

答案:\textit{l}$\mathrm{\nsubset}$\textit{$\alpha$}

知识:直线与平面平行的判定

难度:1

题目:如图,在正方体\textit{ABCD}-\textit{A}${}_{1}$\textit{B}${}_{1}$\textit{C}${}_{1}$\textit{D}${}_{1}$中,\textit{M}是\textit{A}${}_{1}$\textit{D}${}_{1}$的中点,则直线\textit{MD}与平面\textit{A}${}_{1}$\textit{ACC}${}_{1}$的位置关系是\_\_\_\_.直线\textit{MD}与平面\textit{BCC}${}_{1}$\textit{B}${}_{1}$的位置关系是\_\_\_\_.

\includegraphics*[width=1.12in, height=1.10in, keepaspectratio=false]{image132}

解析:因为\textit{M}是\textit{A}${}_{1}$\textit{D}${}_{1}$的中点,所以直线\textit{DM}与直线\textit{AA}${}_{1}$相交,所以\textit{DM}与平面\textit{A}${}_{1}$\textit{ACC}${}_{1}$有一个公共点,所以\textit{DM}与平面\textit{A}${}_{1}$\textit{ACC}${}_{1}$相交.

取\textit{B}${}_{1}$\textit{C}${}_{1}$中点\textit{M}${}_{1}$,\textit{MM}${}_{1}$=\textit{C}${}_{1}$\textit{D}${}_{1}$,\textit{C}${}_{1}$\textit{D}${}_{1}$=\textit{CD},

$\mathrm{\therefore}$四边形\textit{DMM}${}_{1}$\textit{C}为平行四边形,

$\mathrm{\therefore}$\textit{DM}=\textit{CM}${}_{1}$,

$\mathrm{\therefore}$\textit{DM}//平面\textit{BCC}${}_{1}$\textit{B}${}_{1}$.

答案:相交;平行

知识:直线与平面平行的判定

难度:1

题目:如图所示,在正方体\textit{ABCD}-\textit{A}${}_{1}$\textit{B}${}_{1}$\textit{C}${}_{1}$\textit{D}${}_{1}$中,\textit{S},\textit{E},\textit{G}分别是\textit{B}${}_{1}$\textit{D}${}_{1}$,\textit{BC},\textit{SC}的中点.求证:直线\textit{EG}//平面\textit{BDD}${}_{1}$\textit{B}${}_{1}$.

\includegraphics*[width=1.17in, height=1.00in, keepaspectratio=false]{image133}

解析:

答案:如图所示,连接\textit{SB}.

\includegraphics*[width=1.06in, height=0.96in, keepaspectratio=false]{image134}

$\mathrm{\because}$\textit{E}、\textit{G}分别是\textit{BC}、\textit{SC}的中点,

$\mathrm{\therefore}$\textit{EG}//\textit{SB}.

又$\mathrm{\because}$\textit{SB}$\mathrm{\subset }$平面\textit{BDD}${}_{1}$\textit{B}${}_{1}$,

\textit{EG}$\mathrm{\nsubset}$平面\textit{BDD}${}_{1}$\textit{B}${}_{1}$,

$\mathrm{\therefore}$直线\textit{EG}//平面\textit{BDD}${}_{1}$\textit{B}${}_{1}$.

知识:直线与平面平行的判定

难度:1

题目:如图所示,在三棱柱\textit{ABC}-\textit{A}${}_{1}$\textit{B}${}_{1}$\textit{C}${}_{1}$中,\textit{AC}=\textit{BC},点\textit{D}是\textit{AB}的中点,求证:\textit{BC}${}_{1}$//平面\textit{CA}${}_{1}$\textit{D}.

\includegraphics*[width=1.15in, height=1.25in, keepaspectratio=false]{image135}

解析:要证\textit{BC}${}_{1}$//平面\textit{CA}${}_{1}$\textit{D},观察图形,可以发现\textit{AB}与平面相交于点\textit{D},且与\textit{BC}${}_{1}$相交,\textit{D}为\textit{AB}的中点,于是构造$\mathrm{\vartriangle}$\textit{ABC}的中位线,与\textit{BC}${}_{1}$平行,这只要连接\textit{AC}${}_{1}$交\textit{A}${}_{1}$\textit{C}于\textit{E}即可.

答案:连接\textit{AC}${}_{1}$,设\textit{AC}${}_{1}$$\mathrm{\cap}$\textit{A}${}_{1}$\textit{C}=\textit{E},

则\textit{E}为\textit{AC}${}_{1}$的中点,又\textit{D}为\textit{AB}的中点,

$\mathrm{\therefore}$\textit{DE}//\textit{BC}${}_{1}$.

$\mathrm{\because}$\textit{DE}$\mathrm{\subset }$平面\textit{A}${}_{1}$\textit{DC},\textit{BC}${}_{1}$$\mathrm{\nsubset}$平面\textit{A}${}_{1}$\textit{DC},

$\mathrm{\therefore}$\textit{BC}${}_{1}$//平面\textit{A}${}_{1}$\textit{DC}.

知识:直线与平面平行的判定

难度:2

题目:与两个相交平面的交线平行的直线和这两个平面的位置关系是(  )

A.平行    B.都相交

C.在这两个平面内   D.至少和其中一个平行

解析:与两个相交平面的交线平行的直线与这两个平面的位置关系只有两种:一是在这两个平面的某一个平面内;二是与这两个平面都平行.

答案:D

知识:直线与平面平行的判定

难度:2

题目:直线\textit{a}、\textit{b}是异面直线,直线\textit{a}和平面\textit{$\alpha$}平行,则直线\textit{b}和平面\textit{$\alpha$}的位置关系是(  )

A.\textit{b}$\mathrm{\subset }$\textit{$\alpha$}    B.\textit{b}//\textit{$\alpha$ }

C.\textit{b}与\textit{$\alpha$}相交    D.以上都有可能

解析:如图,

\includegraphics*[width=1.30in, height=1.25in, keepaspectratio=false]{image136}

在长方体\textit{ABCD}-\textit{A}${}_{1}$\textit{B}${}_{1}$\textit{C}${}_{1}$\textit{D}${}_{1}$中,\textit{A}${}_{1}$\textit{A}与\textit{BC}是异面直线,\textit{A}${}_{1}$\textit{A}//平面\textit{BCC}${}_{1}$\textit{B}${}_{1}$,而\textit{BC}$\mathrm{\subset }$平面\textit{BCC}${}_{1}$\textit{B}${}_{1}$;\textit{A}${}_{1}$\textit{A}与\textit{CD}是异面直线,\textit{A}${}_{1}$\textit{A}//平面\textit{BCC}${}_{1}$\textit{B}${}_{1}$,而\textit{CD}与平面\textit{BCC}${}_{1}$\textit{B}${}_{1}$相交;\textit{M}、\textit{N}、\textit{P}、\textit{Q}分别为\textit{AB}、\textit{CD}、\textit{C}${}_{1}$\textit{D}${}_{1}$、\textit{A}${}_{1}$\textit{B}${}_{1}$的中点,\textit{A}${}_{1}$\textit{A}与\textit{BC}是异面直线,\textit{A}${}_{1}$\textit{A}//平面\textit{MNPQ},\textit{BC}//平面\textit{MNPQ},故选D.

答案:D

知识:直线与平面平行的判定

难度:2

题目:如图所示,\textit{P}为矩形\textit{ABCD}所在平面外一点,矩形对角线交点为\textit{O},\textit{M}为\textit{PB}的中点,给出五个结论:

\includegraphics*[width=1.16in, height=0.77in, keepaspectratio=false]{image137}

①\textit{OM}//\textit{PD};②\textit{OM}//平面\textit{PCD};③\textit{OM}//平面\textit{PDA};④\textit{OM}//平面\textit{PBA};⑤\textit{OM}//平面\textit{PBC}.其中正确的个数有(  )

A.1   B.2   C.3   D.4

解析:矩形\textit{ABCD}的对角线\textit{AC}与\textit{BD}交于\textit{O}点,所以\textit{O}为\textit{BD}的中点.在$\mathrm{\vartriangle}$\textit{PBD}中,\textit{M}是\textit{PB}的中点,所以\textit{OM}是中位线,\textit{OM}//\textit{PD},则\textit{OM}//平面\textit{PCD},且\textit{OM}//平面\textit{PDA}.因为\textit{M}$\mathrm{\in}$\textit{PB},所以\textit{OM}与平面\textit{PBA}、平面\textit{PBC}相交.

答案:C

知识:直线与平面平行的判定

难度:2

题目:如图,在四面体\textit{ABCD}中,若截面\textit{PQMN}是正方形,则在下列结论中错误的为(  )

\includegraphics*[width=1.19in, height=1.05in, keepaspectratio=false]{image138}

A.\textit{AC}$\mathrm{\bot}$\textit{BD  }B.\textit{AC}//截面\textit{PQMN}

C.\textit{AC}=\textit{BD  }D.异面直线\textit{PM}与\textit{BD}所成的角为45$\mathrm{{}^\circ}$

解析:依题意得\textit{MN}//\textit{PQ},\textit{MN}//平面\textit{ABC},又\textit{MN}、\textit{AC}$\mathrm{\subset }$平面\textit{ACD},且\textit{MN}与\textit{AC}无公共点,因此有\textit{MN}//\textit{AC},\textit{AC}//平面\textit{MNPQ}.同理,\textit{BD}//\textit{PN}.又截面\textit{MNPQ}是正方形,因此有\textit{AC}$\mathrm{\bot}$\textit{BD},直线\textit{PM}与\textit{BD}所成的角是45$\mathrm{{}^\circ}$.综上所述,其中错误的是C,选C.

答案:C

知识:直线与平面平行的判定

难度:2

题目:如图,在五面体\textit{FE}-\textit{ABCD}中,四边形\textit{CDEF}为矩形,\textit{M}、\textit{N}分别是\textit{BF}、\textit{BC}的中点,则\textit{MN}与平面\textit{ADE}的位置关系是\_\_\_\_.

\includegraphics*[width=1.35in, height=1.06in, keepaspectratio=false]{image139}

解析:$\mathrm{\because}$\textit{M}、\textit{N}分别是\textit{BF}、\textit{BC}的中点,$\mathrm{\therefore}$\textit{MN}//\textit{CF}.又四边形\textit{CDEF}为矩形,$\mathrm{\therefore}$\textit{CF}//\textit{DE},$\mathrm{\therefore}$\textit{MN}//\textit{DE}.又\textit{MN}$\mathrm{\nsubset}$平面\textit{ADE},\textit{DE}$\mathrm{\subset }$平面\textit{ADE},$\mathrm{\therefore}$\textit{MN}//平面\textit{ADE}.

答案:平行

知识:直线与平面平行的判定

难度:2

题目:已知直线\textit{b},平面\textit{$\alpha$},有以下条件:

①\textit{b}与\textit{$\alpha$}内一条直线平行;

②\textit{b}与\textit{$\alpha$}内所有直线都没有公共点;

③\textit{b}与\textit{$\alpha$}无公共点;

④\textit{b}不在\textit{$\alpha$}内,且与\textit{$\alpha$}内的一条直线平行.

其中能推出\textit{b}//\textit{$\alpha$}的条件有\_\_\_\_.(把你认为正确的序号都填上)

解析:①中\textit{b}可能在\textit{$\alpha$}内,不符合;②和③是直线与平面平行的定义,④是直线与平面平行的判定定理,都能推出\textit{b}//\textit{$\alpha$}.

答案:②③④

知识:直线与平面平行的判定

难度:3

题目:在五面体\textit{ABCDEF}中,点\textit{O}是矩形\textit{ABCD}的对角线的交点,$\mathrm{\vartriangle}$\textit{CDE}是等边三角形,棱\textit{EF}=$\frac{1}{2}$\textit{BC},证明:\textit{FO}//平面\textit{CDE}.

\includegraphics*[width=1.65in, height=0.79in, keepaspectratio=false]{image140}

解析:

答案:如图所示,取\textit{CD}中点\textit{M},连接\textit{OM}.

\includegraphics*[width=1.65in, height=0.79in, keepaspectratio=false]{image141}

在矩形\textit{ABCD}中,\textit{OM}=$\frac{1}{2}$\textit{BC},又\textit{EF}=$\frac{1}{2}$\textit{BC}.

则\textit{EF}=\textit{OM},连接\textit{EM}.

$\mathrm{\therefore}$四边形\textit{EFOM}为平行四边形,$\mathrm{\therefore}$\textit{FO}//\textit{EM}.

又$\mathrm{\because}$\textit{FO}$\mathrm{\nsubset}$平面\textit{CDE},且\textit{EM}$\mathrm{\subset }$平面\textit{CDE},

$\mathrm{\therefore}$\textit{FO}//平面\textit{CDE}.

知识:直线与平面平行的判定

难度:3

题目:如图,正方体\textit{ABCD}-\textit{A}${}_{1}$\textit{B}${}_{1}$\textit{C}${}_{1}$\textit{D}${}_{1}$中,点\textit{N}在\textit{BD}上,点\textit{M}在\textit{B}${}_{1}$\textit{C}上,且\textit{CM}=\textit{DN},求证:\textit{MN}//平面\textit{AA}${}_{1}$\textit{B}${}_{1}$\textit{B}.

\includegraphics*[width=1.21in, height=1.17in, keepaspectratio=false]{image142}

解析:

答案:
解法一:如图,作\textit{ME}//\textit{BC}交\textit{B}${}_{1}$\textit{B}于\textit{E},作\textit{NF}//\textit{AD}交\textit{AB}于\textit{F},连接\textit{EF},则\textit{EF}$\mathrm{\subset }$平面\textit{AA}${}_{1}$\textit{B}${}_{1}$\textit{B}.

$\mathrm{\therefore}\frac{ME}{BC}=\frac{B_1M}{B_1C}, \frac{NF}{AD}=\frac{BN}{BD}$.

\includegraphics*[width=1.20in, height=1.16in, keepaspectratio=false]{image143}

$\mathrm{\because}$在正方体\textit{ABCD}-\textit{A}${}_{1}$\textit{B}${}_{1}$\textit{C}${}_{1}$\textit{D}${}_{1}$中,\textit{CM}=\textit{DN},

$\mathrm{\therefore}$\textit{B}${}_{1}$\textit{M}=\textit{BN}.

又$\mathrm{\because}$\textit{B}${}_{1}$\textit{C}=\textit{BD},$\mathrm{\therefore}\frac{ME}{BC}=\frac{BN}{BD}=\frac{NF}{AD}$.

$\mathrm{\therefore}$\textit{ME}=\textit{NF}.

又\textit{ME}//\textit{BC}//\textit{AD}//\textit{NF},

\includegraphics*[width=1.40in, height=1.12in, keepaspectratio=false]{image144}

$\mathrm{\therefore}$四边形\textit{MEFN}为平行四边形.

$\mathrm{\therefore}$\textit{MN}//\textit{EF},

$\mathrm{\therefore}$\textit{MN}//平面\textit{AA}${}_{1}$\textit{B}${}_{1}$\textit{B}.

解法二:如图,连接\textit{CN}并延长交\textit{BA}所在直线于点\textit{P},连接\textit{B}${}_{1}$\textit{P}.

则\textit{B}${}_{1}$\textit{P}$\mathrm{\subset }$平面\textit{AA}${}_{1}$\textit{B}${}_{1}$\textit{B}.

$\mathrm{\because}$$\mathrm{\vartriangle}$\textit{NDC}$\mathrm{\backsim}$$\mathrm{\vartriangle}$\textit{NBP},$\mathrm{\therefore}\frac{DN}{NB}=\frac{CN}{NP}$.

又\textit{CM}=\textit{DN},\textit{B}${}_{1}$\textit{C}=\textit{BD},

$\mathrm{\therefore}\frac{CM}{MB_1}=\frac{DN}{NB}=\frac{CN}{NP}$.

$\mathrm{\therefore}$\textit{MN}//\textit{B}${}_{1}$\textit{P}.

$\mathrm{\because}$\textit{B}${}_{1}$\textit{P}$\mathrm{\subset }$平面\textit{AA}${}_{1}$\textit{B}${}_{1}$\textit{B},

$\mathrm{\therefore}$\textit{MN}//平面\textit{AA}${}_{1}$\textit{B}${}_{1}$\textit{B}.


知识:平面与平面平行的判定

难度:1

题目:在长方体\textit{ABCD}-\textit{A}$'$\textit{B}$'$\textit{C}$'$\textit{D}$'$中,下列结论正确的是(  )

A.平面\textit{ABCD}//平面\textit{ABB}$'$\textit{A}$'$

B.平面\textit{ABCD}//平面\textit{ADD}$'$\textit{A}$'$

C.平面\textit{ABCD}//平面\textit{CDD}$'$\textit{C}$'$

D.平面\textit{ABCD}//平面\textit{A}$'$\textit{B}$'$\textit{C}$'$\textit{D}$'$

解析:长方体\textit{ABCD}-\textit{A}$'$\textit{B}$'$\textit{C}$'$\textit{D}$'$中,上底面\textit{ABCD}与下底面\textit{A}$'$\textit{B}$'$\textit{C}$'$\textit{D}$'$平行,故选D.

答案:D

知识:平面与平面平行的判定

难度:1

题目:下列命题正确的是( )

①一个平面内有两条直线都与另外一个平面平行,则这两个平面平行;

②一个平面内有无数条直线都与另外一个平面平行,则这两个平面平行;

③一个平面内任何直线都与另外一个平面平行,则这两个平面平行;

④一个平面内有两条相交直线都与另外一个平面平行,则这两个平面平行.

A.①③   B.②④   C.②③④   D.③④

解析:如果两个平面没有任何一个公共点,那么我们就说这两个平面平行,也即是两个平面没有任何公共直线.

对于①:一个平面内有两条直线都与另外一个平面平行,如果这两条直线不相交,而是平行,那么这两个平面相交也能够找得到这样的直线存在.

对于②:一个平面内有无数条直线都与另外一个平面平行,同①.

对于③:一个平面内任何直线都与另外一个平面平行,则这两个平面平行.这是两个平面平行的定义.

对于④:一个平面内有两条相交直线都与另外一个平面平行,则这两个平面平行.这是两个平面平行的判定定理.

所以只有③④正确,选择D.

答案:D 

知识:平面与平面平行的判定

难度:1

题目:已知一条直线与两个平行平面中的一个相交,则它必与另一个平面(  )

A.平行    B.相交

C.平行或相交    D.平行或在平面内

解析:如图所示.

\includegraphics*[width=1.32in, height=1.55in, keepaspectratio=false]{image146}

答案:B

知识:平面与平面平行的判定

难度:1

题目:经过平面\textit{$\alpha$}外两点,作与\textit{$\alpha$}平行的平面,则这样的平面可以作(  )

A.1个或2个    B.0个或1个

C.1个    D.0个

解析:若平面\textit{$\alpha$}外的两点所确定的直线与平面\textit{$\alpha$}平行,则过该直线与平面\textit{$\alpha$}平行的平面有且只有一个;若平面\textit{$\alpha$}外的两点所确定的直线与平面\textit{$\alpha$}相交,则过该直线的平面与平面\textit{$\alpha$}平行的平面不存在.

答案:B

知识:平面与平面平行的判定

难度:1

题目:如右图所示,设\textit{E}、\textit{F}、\textit{E}${}_{1}$、\textit{F}${}_{1}$分别是长方体\textit{ABCD}-\textit{A}${}_{1}$\textit{B}${}_{1}$\textit{C}${}_{1}$\textit{D}${}_{1}$的棱\textit{AB}、\textit{CD}、\textit{A}${}_{1}$\textit{B}${}_{1}$、\textit{C}${}_{1}$\textit{D}${}_{1}$的中点,则平面\textit{EFD}${}_{1}$\textit{A}${}_{1}$与平面\textit{BCF}${}_{1}$\textit{E}${}_{1}$的位置关系是(  )

\includegraphics*[width=1.23in, height=0.87in, keepaspectratio=false]{image147}

A.平行   B.相交   C.异面   D.不确定

解析:$\mathrm{\because}$\textit{E}${}_{1}$和\textit{F}${}_{1}$分别是\textit{A}${}_{1}$\textit{B}${}_{1}$和\textit{D}${}_{1}$\textit{C}${}_{1}$的中点,

$\mathrm{\therefore}$\textit{A}${}_{1}$\textit{D}${}_{1}$//\textit{E}${}_{1}$\textit{F}${}_{1}$,又\textit{A}${}_{1}$\textit{D}${}_{1}$$\mathrm{\nsubset}$平面\textit{BCF}${}_{1}$\textit{E}${}_{1}$,\textit{E}${}_{1}$\textit{F}${}_{1}$$\mathrm{\subset }$平面\textit{BCF}${}_{1}$\textit{E}${}_{1}$,

$\mathrm{\therefore}$\textit{A}${}_{1}$\textit{D}${}_{1}$//平面\textit{BCF}${}_{1}$\textit{E}${}_{1}$.

又\textit{E}${}_{1}$和\textit{E}分别是\textit{A}${}_{1}$\textit{B}${}_{1}$和\textit{AB}的中点,

$\mathrm{\therefore}$\textit{A}${}_{1}$\textit{E}${}_{1}$=\textit{BE},$\mathrm{\therefore}$四边形\textit{A}${}_{1}$\textit{EBE}${}_{1}$是平行四边形,

$\mathrm{\therefore}$\textit{A}${}_{1}$\textit{E}//\textit{BE}${}_{1}$,

又\textit{A}${}_{1}$\textit{E}$\mathrm{\nsubset}$平面\textit{BCF}${}_{1}$\textit{E}${}_{1}$,\textit{BE}${}_{1}$$\mathrm{\subset }$平面\textit{BCF}${}_{1}$\textit{E}${}_{1}$,

$\mathrm{\therefore}$\textit{A}${}_{1}$\textit{E}//平面\textit{BCF}${}_{1}$\textit{E}${}_{1}$,

又\textit{A}${}_{1}$\textit{E}$\mathrm{\subset }$平面\textit{EFD}${}_{1}$\textit{A}${}_{1}$,\textit{A}${}_{1}$\textit{D}${}_{1}$$\mathrm{\subset }$平面\textit{EFD}${}_{1}$\textit{A}${}_{1}$,\textit{A}${}_{1}$\textit{E}$\mathrm{\cap}$\textit{A}${}_{1}$\textit{D}${}_{1}$=\textit{A}${}_{1}$,

$\mathrm{\therefore}$平面\textit{EFD}${}_{1}$\textit{A}${}_{1}$//平面\textit{BCF}${}_{1}$\textit{E}${}_{1}$.

答案:A

知识:平面与平面平行的判定

难度:1

题目:已知直线\textit{l}、\textit{m},平面\textit{$\alpha$}、\textit{$\beta$},下列命题正确的是(  )

A.\textit{l}//\textit{$\beta$},\textit{l}$\mathrm{\subset }$\textit{$\alpha$}$\mathrm{\Rightarrow }$\textit{$\alpha$}//\textit{$\beta$}

B.\textit{l}//\textit{$\beta$},\textit{m}//\textit{$\beta$},\textit{l}$\mathrm{\subset }$\textit{$\alpha$},\textit{m}$\mathrm{\subset }$\textit{$\alpha$}$\mathrm{\Rightarrow }$\textit{$\alpha$}//\textit{$\beta$}

C.\textit{l}//\textit{m},\textit{l}$\mathrm{\subset }$\textit{$\alpha$},\textit{m}$\mathrm{\subset }$\textit{$\beta$}$\mathrm{\Rightarrow }$\textit{$\alpha$}//\textit{$\beta$}

D.\textit{l}//\textit{$\beta$},\textit{m}//\textit{$\beta$},\textit{l}$\mathrm{\subset }$\textit{$\alpha$},\textit{m}$\mathrm{\subset }$\textit{$\alpha$},\textit{l}$\mathrm{\cap}$\textit{m}=\textit{M}$\mathrm{\Rightarrow }$\textit{$\alpha$}//\textit{$\beta$}

解析:如右图所示,在长方体\textit{ABCD}-\textit{A}${}_{1}$\textit{B}${}_{1}$\textit{C}${}_{1}$\textit{D}${}_{1}$中,直线\textit{AB}//\textit{CD},则直线\textit{AB}//平面\textit{DC}${}_{1}$,直线\textit{AB}$\mathrm{\subset }$平面\textit{AC},但是平面\textit{AC}与平面\textit{DC}${}_{1}$不平行,所以选项A错误;取\textit{BB}${}_{1}$的中点\textit{E},\textit{CC}${}_{1}$的中点\textit{F},则可证\textit{EF}//平面\textit{AC},\textit{B}${}_{1}$\textit{C}${}_{1}$//平面\textit{AC}.又\textit{EF}$\mathrm{\subset }$平面\textit{BC}${}_{1}$,\textit{B}${}_{1}$\textit{C}${}_{1}$$\mathrm{\subset }$平面\textit{BC}${}_{1}$,但是平面\textit{AC}与平面\textit{BC}${}_{1}$不平行,所以选项B错误;直线\textit{AD}//\textit{B}${}_{1}$\textit{C}${}_{1}$,\textit{AD}$\mathrm{\subset }$平面\textit{AC},\textit{B}${}_{1}$\textit{C}${}_{1}$$\mathrm{\subset }$平面\textit{BC}${}_{1}$,但平面\textit{AC}与平面\textit{BC}${}_{1}$不平行,所以选项C错误;很明显选项D是两个平面平行的判定定理,所以选项D正确.

\includegraphics*[width=1.22in, height=0.84in, keepaspectratio=false]{image148}

答案:D

知识:平面与平面平行的判定

难度:1

题目:若夹在两个平面间的三条平行线段相等,那么这两个平面的位置关系为\_\_\_\_.

解析:三条平行线段共面时,两平面可能相交也可能平行,当三条平行线段不共面时,两平面一定平行.

答案:平行或相交

已知平面\textit{$\alpha$}和\textit{$\beta$},在平面\textit{$\alpha$}内任取一条直线\textit{a},在\textit{$\beta$}内总存在直线\textit{b}//\textit{a},则\textit{$\alpha$}与\textit{$\beta$}的位置关系是\_\_\_\_(填``平行''或``相交'').

解析:假若\textit{$\alpha$}$\mathrm{\cap}$\textit{$\beta$}=\textit{l},则在平面\textit{$\alpha$}内,与\textit{l}相交的直线\textit{a},设\textit{a}$\mathrm{\cap}$\textit{l}=\textit{A},对于\textit{$\beta$}内的任意直线\textit{b},若\textit{b}过点\textit{A},则\textit{a}与\textit{b}相交,若\textit{b}不过点\textit{A},则\textit{a}与\textit{b}异面,即\textit{$\beta$}内不存在直线\textit{b}//\textit{a}.故\textit{$\alpha$}//\textit{$\beta$}.

答案:平行

知识:平面与平面平行的判定

难度:1

题目:如图所示,四棱锥\textit{P}-\textit{ABCD}的底面\textit{ABCD}为矩形,\textit{E}、\textit{F}、\textit{H}分别为\textit{AB}、\textit{CD}、\textit{PD}的中点.求证:平面\textit{AFH}//平面\textit{PCE}.

\includegraphics*[width=1.43in, height=1.21in, keepaspectratio=false]{image149}

解析:

答案:因为\textit{F}为\textit{CD}的中点,\textit{H}为\textit{PD}的中点,

所以\textit{FH}//\textit{PC},所以\textit{FH}//平面\textit{PCE}.

又\textit{AE}//\textit{CF}且\textit{AE}=\textit{CF},

所以四边形\textit{AECF}为平行四边形,

所以\textit{AF}//\textit{CE},所以\textit{AF}//平面\textit{PCE}.

由\textit{FH}$\mathrm{\subset }$平面\textit{AFH},\textit{AF}$\mathrm{\subset }$平面\textit{AFH},\textit{FH}$\mathrm{\cap}$\textit{AF}=\textit{F},

所以平面\textit{AFH}//平面\textit{PCE}.

\includegraphics*[width=1.25in, height=1.37in, keepaspectratio=false]{image150}

知识:平面与平面平行的判定

难度:1

题目:(2016·南平高二检测)在正方体\textit{ABCD}-\textit{A}${}_{1}$\textit{B}${}_{1}$\textit{C}${}_{1}$\textit{D}${}_{1}$中,\textit{M},\textit{N},\textit{P}分别是\textit{AD}${}_{1}$,\textit{BD}和\textit{B}${}_{1}$\textit{C}的中点.

求证:
(1)\textit{MN}//平面\textit{CC}${}_{1}$\textit{D}${}_{1}$\textit{D};

(2)平面\textit{MNP}//平面\textit{CC}${}_{1}$\textit{D}${}_{1}$\textit{D}.

解析:

答案:
(1)连接\textit{AC},\textit{CD}${}_{1}$.

\includegraphics*[width=1.05in, height=1.07in, keepaspectratio=false]{image151}

因为\textit{ABCD}为正方形,\textit{N}为\textit{BD}中点,所以\textit{N}为\textit{AC}中点.

又因为\textit{M}为\textit{AD}${}_{1}$中点,

所以\textit{MN}//\textit{CD}${}_{1}$.

因为\textit{MN}$\mathrm{\nsubset}$平面\textit{CC}${}_{1}$\textit{D}${}_{1}$\textit{D},\textit{CD}${}_{1}$$\mathrm{\subset }$平面\textit{CC}${}_{1}$\textit{D}${}_{1}$\textit{D},

所以\textit{MN}//平面\textit{CC}${}_{1}$\textit{D}${}_{1}$\textit{D}.

(2)连接\textit{BC}${}_{1}$,\textit{C}${}_{1}$\textit{D},因为\textit{B}${}_{1}$\textit{BCC}${}_{1}$为正方形,\textit{P}为\textit{BC}${}_{1}$的中点,

所以\textit{P}为\textit{BC}${}_{1}$中点,又因为\textit{N}为\textit{BD}中点,所以\textit{PN}//\textit{C}${}_{1}$\textit{D}.

因为\textit{PN}$\mathrm{\nsubset}$平面\textit{CC}${}_{1}$\textit{D}${}_{1}$\textit{D},\textit{C}${}_{1}$\textit{D}$\mathrm{\subset }$平面\textit{CC}${}_{1}$\textit{D}${}_{1}$\textit{D},所以\textit{PN}//平面\textit{CC}${}_{1}$\textit{D}${}_{1}$\textit{D},

由(1)知,\textit{MN}//平面\textit{CC}${}_{1}$\textit{D}${}_{1}$\textit{D}且\textit{MN}$\mathrm{\cap}$\textit{PN}=\textit{N},

所以平面\textit{MNP}//平面\textit{CC}${}_{1}$\textit{D}${}_{1}$\textit{D}.

知识:平面与平面平行的判定

难度:2

题目:\textit{a}、\textit{b}、\textit{c}为三条不重合的直线,\textit{$\alpha$}、\textit{$\beta$}、\textit{$\gamma$}为三个不重合平面,现给出六个命题.

①\textit{a}//\textit{c},\textit{b}//\textit{c}$\mathrm{\Rightarrow }$\textit{a}//\textit{b};

②\textit{a}//\textit{$\gamma$},\textit{b}//\textit{$\gamma$}$\mathrm{\Rightarrow }$\textit{a}//\textit{b};

③\textit{$\alpha$}//\textit{c},\textit{$\beta$}//\textit{c}$\mathrm{\Rightarrow }$\textit{$\alpha$}//\textit{$\beta$};

④\textit{$\alpha$}//\textit{$\gamma$},\textit{$\beta$}//\textit{$\gamma$}$\mathrm{\Rightarrow }$\textit{$\alpha$}//\textit{$\beta$};

⑤\textit{$\alpha$}//\textit{c},\textit{a}//\textit{c}$\mathrm{\Rightarrow }$\textit{$\alpha$}//\textit{a};

⑥\textit{a}//\textit{$\gamma$},\textit{$\alpha$}//\textit{$\gamma$}$\mathrm{\Rightarrow }$\textit{$\alpha$}//\textit{a}.

其中正确的命题是(  )

A.①②③   B.①④⑤  C.①④   D.①③④

解析:

①平行公理.

②两直线同时平行于一平面,这两条直线可相交、平行或异面.

③两平面同时平行于一直线,这两个平面相交或平行.

④面面平行传递性.

⑤一直线和一平面同时平行于另一直线,这条直线和平面或平行或直线在平面内.

⑥一直线和一平面同时平行于另一平面,这直线和平面可平行也可能直线在平面内.故①④正确.

答案:C

下列结论中:

(1)过不在平面内的一点,有且只有一个平面与这个平面平行;

(2)过不在平面内的一条直线,有且只有一个平面与这个平面平行;

(3)过不在直线上的一点,有且只有一条直线与这条直线平行;

(4)过不在直线上的一点,有且仅有一个平面与这条直线平行.

正确的序号为(  )

A.(1)(2)   B.(3)(4)   C.(1)(3)   D.(2)(4)

解析:

答案:C

知识:平面与平面平行的判定

难度:2

题目:若\textit{a}、\textit{b}、\textit{c}、\textit{d}是直线,\textit{$\alpha$}、\textit{$\beta$}是平面,且\textit{a}、\textit{b}$\mathrm{\subset }$\textit{$\alpha$},\textit{c}、\textit{d}$\mathrm{\subset }$\textit{$\beta$},且\textit{a}//\textit{c},\textit{b}//\textit{d},则平面\textit{$\alpha$}与平面\textit{$\beta$}(  )

A.平行   B.相交 C.异面   D.不能确定

解析:

答案:D

知识:平面与平面平行的判定

难度:2

题目:若平面\textit{$\alpha$}//平面\textit{$\beta$},直线\textit{a}//\textit{$\alpha$},点\textit{B}$\mathrm{\in}$\textit{$\beta$},则在平面\textit{$\beta$}内过点\textit{B}的所有直线中(  )

A.不一定存在与\textit{a}平行的直线

B.只有两条与\textit{a}平行的直线

C.存在无数条与\textit{a}平行的直线

D.存在唯一一条与\textit{a}平行的直线

解析:当直线\textit{a}$\mathrm{\subset }$\textit{$\beta$},\textit{B}$\mathrm{\in}$\textit{a}上时满足条件,此时过\textit{B}不存在与\textit{a}平行的直线,故选A.

答案:A

知识:平面与平面平行的判定

难度:2

题目:如图是四棱锥的平面展开图,其中四边形\textit{ABCD}为正方形,\textit{E}、\textit{F}、\textit{G}、\textit{H}分别为\textit{PA}、\textit{PD}、\textit{PC}、\textit{PB}的中点,在此几何体中,给出下面四个结论:

\includegraphics*[width=1.23in, height=1.22in, keepaspectratio=false]{image152}

①平面\textit{EFGH}//平面\textit{ABCD};

②平面\textit{PAD}//\textit{BC};

③平面\textit{PCD}//\textit{AB};

④平面\textit{PAD}//平面\textit{PAB}.

其中正确的有\_\_\_\_.(填序号)

解析:把平面展开图还原为四棱锥如图所示,则\textit{EH}//\textit{AB},所以\textit{EH}//平面\textit{ABCD}.同理可证\textit{EF}//平面\textit{ABCD},所以平面\textit{EFGH}//平面\textit{ABCD};平面\textit{PAD},平面\textit{PBC},平面\textit{PAB},平面\textit{PDC}均是四棱锥的四个侧面,则它们两两相交.$\mathrm{\because}$\textit{AB}//\textit{CD},$\mathrm{\therefore}$平面\textit{PCD}//\textit{AB}.同理平面\textit{PAD}//\textit{BC}.

\includegraphics*[width=1.04in, height=0.83in, keepaspectratio=false]{image153}

答案:①②③

知识:平面与平面平行的判定

难度:2

题目:如右图所示,在正方体\textit{ABCD}-\textit{A}${}_{1}$\textit{B}${}_{1}$\textit{C}${}_{1}$\textit{D}${}_{1}$中,\textit{E}、\textit{F}、\textit{G}、\textit{H}分别为棱\textit{CC}${}_{1}$、\textit{C}${}_{1}$\textit{D}${}_{1}$、\textit{D}${}_{1}$\textit{D}、\textit{CD}的中点,\textit{N}是\textit{BC}的中点,点\textit{M}在四边形\textit{EFGH}及其内部运动,则\textit{M}满足\_\_\_\_时,有\textit{MN}//平面\textit{B}${}_{1}$\textit{BDD}${}_{1}$.

\includegraphics*[width=1.17in, height=1.10in, keepaspectratio=false]{image154}

解析:$\mathrm{\because}$\textit{FH}//\textit{BB}${}_{1}$,\textit{HN}//\textit{BD},\textit{FH}$\mathrm{\cap}$\textit{HN}=\textit{H},

$\mathrm{\therefore}$平面\textit{FHN}//平面\textit{B}${}_{1}$\textit{BDD}${}_{1}$,

又平面\textit{FHN}$\mathrm{\cap}$平面\textit{EFGH}=\textit{FH},

$\mathrm{\therefore}$当\textit{M}$\mathrm{\in}$\textit{FH}时,\textit{MN}$\mathrm{\subset }$平面\textit{FHN},

$\mathrm{\therefore}$\textit{MN}//平面\textit{B}${}_{1}$\textit{BDD}${}_{1}$.

答案:点\textit{M}在\textit{FH}上

知识:平面与平面平行的判定

难度:3

题目:已知点\textit{S}是正三角形\textit{ABC}所在平面外的一点,且\textit{SA}=\textit{SB}=\textit{SC},\textit{SG}为$\mathrm{\vartriangle}$\textit{SAB}边\textit{AB}上的高,\textit{D}、\textit{E}、\textit{F}分别是\textit{AC}、\textit{BC}、\textit{SC}的中点,试判断\textit{SG}与平面\textit{DEF}的位置关系,并给予证明.

\includegraphics*[width=0.96in, height=1.04in, keepaspectratio=false]{image155}

解析:

答案:
解法一:连接\textit{CG}交\textit{DE}于点\textit{H},

$\mathrm{\because}$\textit{DE}是$\mathrm{\vartriangle}$\textit{ABC}的中位线,

$\mathrm{\therefore}$\textit{DE}//\textit{AB}.

在$\mathrm{\vartriangle}$\textit{ACG}中,\textit{D}是\textit{AC}的中点,且\textit{DH}//\textit{AG},$\mathrm{\therefore}$\textit{H}是\textit{CG}的中点.

$\mathrm{\therefore}$\textit{FH}是$\mathrm{\vartriangle}$\textit{SCG}的中位线,

$\mathrm{\therefore}$\textit{FH}//\textit{SG}.

又\textit{SG}$\mathrm{\nsubset}$平面\textit{DEF},\textit{FH}$\mathrm{\subset }$平面\textit{DEF},

$\mathrm{\therefore}$\textit{SG}//平面\textit{DEF}.

解法二:$\mathrm{\because}$\textit{EF}为$\mathrm{\vartriangle}$\textit{SBC}的中位线,

$\mathrm{\therefore}$\textit{EF}//\textit{SB}.

$\mathrm{\because}$\textit{EF}$\mathrm{\nsubset}$平面\textit{SAB},\textit{SB}$\mathrm{\subset }$平面\textit{SAB},

$\mathrm{\therefore}$\textit{EF}//平面\textit{SAB}.

同理:\textit{DF}//平面\textit{SAB},\textit{EF}$\mathrm{\cap}$\textit{DF}=\textit{F},

$\mathrm{\therefore}$平面\textit{SAB}//平面\textit{DEF},

又$\mathrm{\because}$\textit{SG}$\mathrm{\subset }$平面\textit{SAB},$\mathrm{\therefore}$\textit{SG}//平面\textit{DEF}.

知识:平面与平面平行的判定

难度:3

题目:如图,在正方形\textit{ABCD}-\textit{A}${}_{1}$\textit{B}${}_{1}$\textit{C}${}_{1}$\textit{D}${}_{1}$中,\textit{E},\textit{F},\textit{M}分别是棱\textit{B}${}_{1}$\textit{C}${}_{1}$,\textit{BB}${}_{1}$,\textit{C}${}_{1}$\textit{D}${}_{1}$的中点,是否存在过点\textit{E},\textit{M}且与平面\textit{A}${}_{1}$\textit{FC}平行的平面?若存在,请作出并证明;若不存在,请说明理由.

\includegraphics*[width=1.12in, height=1.00in, keepaspectratio=false]{image156}

解析:由正方体的特征及\textit{N}为\textit{BB}${}_{1}$的中点,可知平面\textit{A}${}_{1}$\textit{FC}与直线\textit{DD}${}_{1}$相交,且交点为\textit{DD}${}_{1}$的中点\textit{G}.

若过\textit{M},\textit{E}的平面\textit{$\alpha$}与平面\textit{A}${}_{1}$\textit{FCG}平行,注意到\textit{EM}//\textit{B}${}_{1}$\textit{D}${}_{1}$//\textit{FG},则平面\textit{$\alpha$}必与\textit{CC}${}_{1}$相交于点\textit{N},结合\textit{M},\textit{E}为棱\textit{C}${}_{1}$\textit{D}${}_{1}$,\textit{B}${}_{1}$\textit{C}${}_{1}$的中点,易知\textit{C}${}_{1}$\textit{N}$\mathrm{:}$\textit{C}${}_{1}$\textit{C}=$\frac{1}{4}$.于是平面\textit{EMN}满足要求.
如图,设\textit{N}是棱\textit{C}${}_{1}$\textit{C}上的一点,且\textit{C}${}_{1}$\textit{N}=$\frac{1}{4}$\textit{C}${}_{1}$\textit{C}时,平面\textit{EMN}过点\textit{E},\textit{M}且与平面\textit{A}${}_{1}$\textit{FC}平行.

\includegraphics*[width=1.12in, height=1.00in, keepaspectratio=false]{image157}

答案:设\textit{H}为棱\textit{C}${}_{1}$\textit{C}的中点,连接\textit{B}${}_{1}$\textit{H},\textit{D}${}_{1}$\textit{H}.

$\mathrm{\because}$\textit{C}${}_{1}$\textit{N}=$\frac{1}{4}$\textit{C}${}_{1}$\textit{C},

$\mathrm{\therefore}$\textit{C}${}_{1}$\textit{N}=$\frac{1}{2}$\textit{C}${}_{1}$\textit{H}.

又\textit{E}为\textit{B}${}_{1}$\textit{C}${}_{1}$的中点,

$\mathrm{\therefore}$\textit{EN}//\textit{B}${}_{1}$\textit{H}.

又\textit{CF}//\textit{B}${}_{1}$\textit{H},

$\mathrm{\therefore}$\textit{EN}//\textit{CF}.

又\textit{EN}$\mathrm{\nsubset}$平面\textit{A}${}_{1}$\textit{FC},\textit{CF}$\mathrm{\subset }$平面\textit{A}${}_{1}$\textit{FC},

$\mathrm{\therefore}$\textit{EN}//平面\textit{A}${}_{1}$\textit{FC}.

同理\textit{MN}//\textit{D}${}_{1}$\textit{H},\textit{D}${}_{1}$\textit{H}//\textit{A}${}_{1}$\textit{F},

$\mathrm{\therefore}$\textit{MN}//\textit{A}${}_{1}$\textit{F}.

又\textit{MN}$\mathrm{\nsubset}$平面\textit{A}${}_{1}$\textit{FC},\textit{A}${}_{1}$\textit{F}$\mathrm{\subset }$平面\textit{A}${}_{1}$\textit{FC},

$\mathrm{\therefore}$\textit{MN}//平面\textit{A}${}_{1}$\textit{FC}.

又\textit{EN}$\mathrm{\cap}$\textit{MN}=\textit{N},

$\mathrm{\therefore}$平面\textit{EMN}//平面\textit{A}${}_{1}$\textit{FC}.

知识:直线与平面平行的性质

难度:1

题目:正方体\textit{ABCD}-\textit{A}${}_{1}$\textit{B}${}_{1}$\textit{C}${}_{1}$\textit{D}${}_{1}$中,截面\textit{BA}${}_{1}$\textit{C}${}_{1}$与直线\textit{AC}的位置关系是(  )

\includegraphics*[width=1.38in, height=1.37in, keepaspectratio=false]{image159}

A.\textit{AC}//截面\textit{BA}${}_{1}$\textit{C}${}_{1}$   B.\textit{AC}与截面\textit{BA}${}_{1}$\textit{C}${}_{1}$相交

C.\textit{AC}在截面\textit{BA}${}_{1}$\textit{C}${}_{1}$内   D.以上答案都错误

解析:$\mathrm{\because}$\textit{AC}//\textit{A}${}_{1}$\textit{C}${}_{1}$,又$\mathrm{\because}$\textit{AC}$\mathrm{\nsubset}$面\textit{BA}${}_{1}$\textit{C}${}_{1}$,

$\mathrm{\therefore}$\textit{AC}//面\textit{BA}${}_{1}$\textit{C}${}_{1}$.

答案:A

知识:直线与平面平行的性质

难度:1

题目:如右图所示的三棱柱\textit{ABC}-\textit{A}${}_{1}$\textit{B}${}_{1}$\textit{C}${}_{1}$中,过\textit{A}${}_{1}$\textit{B}${}_{1}$的平面与平面\textit{ABC}交于直线\textit{DE},则\textit{DE}与\textit{AB}的位置关系是(  )

\includegraphics*[width=0.84in, height=1.04in, keepaspectratio=false]{image160}

A.异面  B.平行

C.相交  D.以上均有可能

解析:$\mathrm{\because}$\textit{A}${}_{1}$\textit{B}${}_{1}$//\textit{AB},\textit{AB}$\mathrm{\subset }$平面\textit{ABC},\textit{A}${}_{1}$\textit{B}${}_{1}$$\mathrm{\nsubset}$平面\textit{ABC},

$\mathrm{\therefore}$\textit{A}${}_{1}$\textit{B}${}_{1}$//平面\textit{ABC}.

又\textit{A}${}_{1}$\textit{B}${}_{1}$$\mathrm{\subset }$平面\textit{A}${}_{1}$\textit{B}${}_{1}$\textit{ED},平面\textit{A}${}_{1}$\textit{B}${}_{1}$\textit{ED}$\mathrm{\cap}$平面\textit{ABC}=\textit{DE},$\mathrm{\therefore}$\textit{DE}//\textit{A}${}_{1}$\textit{B}${}_{1}$.

又\textit{AB}//\textit{A}${}_{1}$\textit{B}${}_{1}$,$\mathrm{\therefore}$\textit{DE}//\textit{AB}.

答案:B

知识:直线与平面平行的性质

难度:1

题目:下列命题正确的是(  )

A.若直线\textit{a}//平面\textit{$\alpha$},直线\textit{b}//平面\textit{$\alpha$},则直线\textit{a}//直线\textit{b}

B.若直线\textit{a}//平面\textit{$\alpha$},直线\textit{a}与直线\textit{b}相交,则直线\textit{b}与平面\textit{$\alpha$}相交

C.若直线\textit{a}//平面\textit{$\alpha$},直线\textit{a}//直线\textit{b},则直线\textit{b}//平面\textit{$\alpha$}

D.若直线\textit{a}//平面\textit{$\alpha$},则直线\textit{a}与平面\textit{$\alpha$}内任意一条直线都无公共点

解析:A中,直线\textit{a}与直线\textit{b}也可能异面、相交,所以不正确;B中,直线\textit{b}也可能与平面\textit{$\alpha$}平行,所以不正确;C中,直线\textit{b}也可能在平面\textit{$\alpha$}内,所以不正确;根据直线与平面平行的定义知D正确,故选D.

答案:D

知识:直线与平面平行的性质

难度:1

题目:如图,在三棱柱\textit{ABC}-\textit{A}${}_{1}$\textit{B}${}_{1}$\textit{C}${}_{1}$中,\textit{AM}=2\textit{MA}${}_{1}$,\textit{BN}=2\textit{NB}${}_{1}$,过\textit{MN}作一平面交底面三角形\textit{ABC}的边\textit{BC}、\textit{AC}于点\textit{E}、\textit{F},则(  )

\includegraphics*[width=1.42in, height=1.23in, keepaspectratio=false]{image161}

A.\textit{MF}//\textit{NE}

B.四边形\textit{MNEF}为梯形

C.四边形\textit{MNEF}为平行四边形

D.\textit{A}${}_{1}$\textit{B}${}_{1}$//\textit{NE}

解析:$\mathrm{\because}$在▱\textit{AA}${}_{1}$\textit{B}${}_{1}$\textit{B}中,\textit{AM}=2\textit{MA}${}_{1}$,\textit{BN}=2\textit{NB}${}_{1}$,$\mathrm{\therefore}$\textit{AM}=\textit{BN},$\mathrm{\therefore}$\textit{MN}=\textit{AB}.又\textit{MN}$\mathrm{\nsubset}$平面\textit{ABC},\textit{AB}$\mathrm{\subset }$平面\textit{ABC},$\mathrm{\therefore}$\textit{MN}//平面\textit{ABC}.又\textit{MN}$\mathrm{\subset }$平面\textit{MNEF},平面\textit{MNEF}$\mathrm{\cap}$平面\textit{ABC}=\textit{EF},$\mathrm{\therefore}$\textit{MN}//\textit{EF},$\mathrm{\therefore}$\textit{EF}//\textit{AB},显然在$\mathrm{\vartriangle}$\textit{ABC}中\textit{EF}$\mathrm{\neq}$\textit{AB},$\mathrm{\therefore}$\textit{EF}$\mathrm{\neq}$\textit{MN},$\mathrm{\therefore}$四边形\textit{MNEF}为梯形.故选B.

答案:B

知识:直线与平面平行的性质

难度:1

题目:如右图所示,在空间四边形\textit{ABCD}中,\textit{E}、\textit{F}、\textit{G}、\textit{H}分别是\textit{AB}、\textit{BC}、\textit{CD}、\textit{DA}上的点,\textit{EH}//\textit{FG},则\textit{EH}与\textit{BD}的位置关系是(  )

\includegraphics*[width=1.08in, height=0.92in, keepaspectratio=false]{image162}

A.平行 B.相交 C.异面 D.不确定

解析:$\mathrm{\because}$\textit{EH}//\textit{FG},\textit{FG}$\mathrm{\subset }$平面\textit{BCD},\textit{EH}$\mathrm{\nsubset}$平面\textit{BCD},

$\mathrm{\therefore}$\textit{EH}//平面\textit{BCD}.

$\mathrm{\because}$\textit{EH}$\mathrm{\subset }$平面\textit{ABD},平面\textit{ABD}$\mathrm{\cap}$平面\textit{BCD}=\textit{BD},

$\mathrm{\therefore}$\textit{EH}//\textit{BD}.

答案:A

知识:直线与平面平行的性质

难度:1

题目:已知正方体\textit{AC}${}_{1}$的棱长为1,点\textit{P}是面\textit{AA}${}_{1}$\textit{D}${}_{1}$\textit{D}的中心,点\textit{Q}是面\textit{A}${}_{1}$\textit{B}${}_{1}$\textit{C}${}_{1}$\textit{D}${}_{1}$的对角线\textit{B}${}_{1}$\textit{D}${}_{1}$上一点,且\textit{PQ}//平面\textit{AA}${}_{1}$\textit{B}${}_{1}$\textit{B},则线段\textit{PQ}的长为(  )

A.1   B.$\sqrt{2}$   C.$\frac{\sqrt{2}}{2}$   D.$\frac{\sqrt{3}}{2}$

解析:由\textit{PQ}//平面\textit{AA}${}_{1}$\textit{BB}知\textit{PQ}//\textit{AB}${}_{1}$,又\textit{P}为\textit{AO}${}_{1}$的中点,$\mathrm{\therefore}$\textit{PQ}=$\frac{1}{2}$\textit{AB}${}_{1}$=$\frac{\sqrt{2}}{2}$.

答案:C

知识:直线与平面平行的性质

难度:1

题目:(2016·扬州高二检测)在正方体\textit{ABCD}-\textit{A}${}_{1}$\textit{B}${}_{1}$\textit{C}${}_{1}$\textit{D}${}_{1}$中,若过\textit{A},\textit{C},\textit{B}${}_{1}$三点的平面与底面\textit{A}${}_{1}$\textit{B}${}_{1}$\textit{C}${}_{1}$\textit{D}${}_{1}$的交线为\textit{l},则\textit{l}与\textit{A}${}_{1}$\textit{C}${}_{1}$的位置关系是\_\_\_\_.

解析:$\mathrm{\because}$平面\textit{ABCD}//平面\textit{A}${}_{1}$\textit{B}${}_{1}$\textit{C}${}_{1}$\textit{D}${}_{1}$,\textit{AC}$\mathrm{\subset }$平面\textit{ABCD},

\includegraphics*[width=1.02in, height=1.04in, keepaspectratio=false]{image164}

$\mathrm{\therefore}$\textit{AC}//平面\textit{A}${}_{1}$\textit{B}${}_{1}$\textit{C}${}_{1}$\textit{D}${}_{1}$.

又平面\textit{ACB}${}_{1}$经过直线\textit{AC}与平面\textit{A}${}_{1}$\textit{B}${}_{1}$\textit{C}${}_{1}$\textit{D}${}_{1}$相交于直线\textit{l},

$\mathrm{\therefore}$\textit{AC}//\textit{l}.

答案:\textit{l}//\textit{A}${}_{1}$\textit{C}${}_{1}$

知识:直线与平面平行的性质

难度:1

题目:如图,在正方体\textit{ABCD}-\textit{A}${}_{1}$\textit{B}${}_{1}$\textit{C}${}_{1}$\textit{D}${}_{1}$中,\textit{E}、\textit{F}分别是棱\textit{AA}${}_{1}$和\textit{BB}${}_{1}$的中点,过\textit{EF}的平面\textit{EFGH}分别交\textit{BC}和\textit{AD}于点\textit{G}、\textit{H},求证:\textit{AB}//\textit{GH}.

\includegraphics*[width=1.21in, height=1.16in, keepaspectratio=false]{image165}

解析:

答案:$\mathrm{\because}$\textit{E}、\textit{F}分别是\textit{AA}${}_{1}$和\textit{BB}${}_{1}$的中点,$\mathrm{\therefore}$\textit{EF}//\textit{AB}.

又\textit{AB}$\mathrm{\nsubset}$平面\textit{EFGH},\textit{EF}$\mathrm{\subset }$平面\textit{EFGH},

$\mathrm{\therefore}$\textit{AB}//平面\textit{EFGH}.

又\textit{AB}$\mathrm{\subset }$平面\textit{ABCD},

平面\textit{ABCD}$\mathrm{\cap}$平面\textit{EFGH}=\textit{GH},$\mathrm{\therefore}$\textit{AB}//\textit{GH}.

知识:直线与平面平行的性质

难度:1

题目:四棱锥\textit{P}-\textit{ABCD}的底面\textit{ABCD}是梯形,\textit{AB}//\textit{CD},且\textit{AB}=$\frac{2}{3}$\textit{CD}.试问在\textit{PC}上能否找到一点\textit{E},使得\textit{BE}//平面\textit{PAD}?若能,请确定\textit{E}点的位置,并给出证明;若不能,请说明理由.

解析:在\textit{PC}上取点\textit{E},使=,

\includegraphics*[width=1.08in, height=1.39in, keepaspectratio=false]{image166}

则\textit{BE}//平面\textit{PAD}.

答案:延长\textit{DA}和\textit{CB}交于点\textit{F},连接\textit{PF}.

梯形\textit{ABCD}中,\textit{AB}//\textit{CD},

\textit{AB}=$\frac{2}{3}$\textit{CD}.

$\mathrm{\therefore}\frac{AB}{CD}=\frac{BF}{FC}\frac{2}{3}$,

$\mathrm{\therefore}\frac{1}{2}$.

又$\frac{CE}{PE}=\frac{1}{2}$,$\mathrm{\therefore}$$\mathrm{\vartriangle}$\textit{PFC}中,$\frac{CE}{PE}=\frac{BC}{BF}$,

$\mathrm{\therefore}$\textit{BE}//\textit{PF},

而\textit{BE}$\mathrm{\nsubset}$平面\textit{PAD},\textit{PF}$\mathrm{\subset }$平面\textit{PAD}.

$\mathrm{\therefore}$\textit{BE}//平面\textit{PAD}.

知识:直线与平面平行的性质

难度:2

题目:\textit{a}、\textit{b}是两条异面直线,下列结论正确的是( )

A.过不在\textit{a}、\textit{b}上的任一点,可作一个平面与\textit{a}、\textit{b}平行

B.过不在\textit{a}、\textit{b}上的任一点,可作一条直线与\textit{a}、\textit{b}相交

C.过不在\textit{a}、\textit{b}上的任一点,可作一条直线与\textit{a}、\textit{b}都平行

D.过\textit{a}可以并且只可以作一个平面与\textit{b}平行

解析:A错,若点与\textit{a}所确定的平面与\textit{b}平行时,就不能使这个平面与\textit{a}平行了.

B错,若点与\textit{a}所确定的平面与\textit{b}平行时,就不能作一条直线与\textit{a},\textit{b}相交.

C错,假如这样的直线存在,根据公理4就可有\textit{a}//\textit{b},这与\textit{a},\textit{b}异面矛盾.

D正确,在\textit{a}上任取一点\textit{A},过\textit{A}点作直线\textit{c}//\textit{b},则\textit{c}与\textit{a}确定一个平面与\textit{b}平行,这个平面是唯一的.

答案:D 

知识:直线与平面平行的性质

难度:2

题目:过平面\textit{$\alpha$}外的直线\textit{l},作一组平面与\textit{$\alpha$}相交,如果所得的交线为\textit{a}、\textit{b}、\textit{c}、$\dots$,那么这些交线的位置关系为(  )

A.都平行

B.都相交且一定交于同一点

C.都相交但不一定交于同一点

D.都平行或交于同一点

解析:若\textit{l}//平面\textit{$\alpha$},则交线都平行;

若\textit{l}$\mathrm{\cap}$平面\textit{$\alpha$}=\textit{A},则交线都交于同一点\textit{A}.

答案:D

知识:直线与平面平行的性质

难度:2

题目:如图,在三棱锥\textit{S}-\textit{ABC}中,\textit{E}、\textit{F}分别是\textit{SB}、\textit{SC}上的点,且\textit{EF}//平面\textit{ABC},则(  )

\includegraphics*[width=1.10in, height=1.12in, keepaspectratio=false]{image167}

A.\textit{EF}与\textit{BC}相交  B.\textit{EF}//\textit{BC}

C.\textit{EF}与\textit{BC}异面  D.以上均有可能

解析:$\mathrm{\because}$\textit{EF}$\mathrm{\subset }$平面\textit{SBC},\textit{EF}//平面\textit{ABC},平面\textit{SBC}$\mathrm{\cap}$平面\textit{ABC}=\textit{BC},$\mathrm{\therefore}$\textit{EF}//\textit{BC}.

答案:B

知识:直线与平面平行的性质

难度:2

题目:不同直线\textit{m}、\textit{n}和不同平面\textit{$\alpha$}、\textit{$\beta$},给出下列命题:

①$\left\{\begin{array}{l}\alpha // \beta\\ m\subset \alpha \end{array}\right.\mathrm{\Rightarrow }$\textit{m}//\textit{$\beta$};
②$\left\{\begin{array}{l} m//n\\ m//\beta m\subset \alpha \end{array}\right.\mathrm{\Rightarrow }$\textit{n}//\textit{$\beta$};
③$\left\{\begin{array}{l} m\subset \alpha\\ n\subset \beta \end{array}\right.\mathrm{\Rightarrow }$\textit{m}、\textit{n}异面.

其中假命题有(  )

A.0个  B.1个  C.2个  D.3个

解析:$\mathrm{\because}$\textit{$\alpha$}//\textit{$\beta$},$\mathrm{\therefore}$\textit{$\alpha$}与\textit{$\beta$}没有公共点.

又$\mathrm{\because}$\textit{m}$\mathrm{\subset }$\textit{$\alpha$},$\mathrm{\therefore}$\textit{m}与\textit{$\beta$}没有公共点,

$\mathrm{\therefore}$\textit{m}//\textit{$\beta$},故①正确,②③错误.

答案:C

知识:直线与平面平行的性质

难度:2

题目:已知\textit{A}、\textit{B}、\textit{C}、\textit{D}四点不共面,且\textit{AB}//平面\textit{$\alpha$},\textit{CD}//\textit{$\alpha$},\textit{AC}$\mathrm{\cap}$\textit{$\alpha$}=\textit{E},\textit{AD}$\mathrm{\cap}$\textit{$\alpha$}=\textit{F},\textit{BD}$\mathrm{\cap}$\textit{$\alpha$}=\textit{H},\textit{BC}$\mathrm{\cap}$\textit{$\alpha$}=\textit{G},则四边形\textit{EFHG}是\_\_\_\_四边形.

\includegraphics*[width=1.02in, height=1.27in, keepaspectratio=false]{image168}

解析:$\mathrm{\because}$\textit{AB}//\textit{$\alpha$},平面\textit{ABD}$\mathrm{\cap}$\textit{$\alpha$}=\textit{FH},平面\textit{ABC}$\mathrm{\cap}$\textit{$\alpha$}=\textit{EG},

$\mathrm{\therefore}$\textit{AB}//\textit{FH},\textit{AB}//\textit{EG},$\mathrm{\therefore}$\textit{FH}//\textit{EG},同理\textit{EF}//\textit{GH},$\mathrm{\therefore}$四边形\textit{EFHG}是平行四边形.

答案:平行

知识:直线与平面平行的性质

难度:2

题目:(2016·成都高二检测)长方体\textit{ABCD}-\textit{A}${}_{1}$\textit{B}${}_{1}$\textit{C}${}_{1}$\textit{D}${}_{1}$的底面\textit{ABCD}是正方形,其侧面展开图是边长为8的正方形.\textit{E},\textit{F}分别是侧棱\textit{AA}${}_{1}$、\textit{CC}${}_{1}$上的动点,\textit{AE}+\textit{CF}=8.\textit{P}在棱\textit{AA}${}_{1}$上,且\textit{AP}=2,若\textit{EF}//平面\textit{PBD},则\textit{CF}=\_\_\_\_.

\includegraphics*[width=0.83in, height=1.85in, keepaspectratio=false]{image169}

解析:连接\textit{AC}交\textit{BD}于\textit{O},连接\textit{PO}.

因为\textit{EF}//平面\textit{PBD},\textit{EF}$\mathrm{\subset }$平面\textit{EACF},平面\textit{EACF}$\mathrm{\cap}$平面\textit{PBD}=\textit{PO},所以\textit{EF}//\textit{PO},在\textit{PA}${}_{1}$上截取\textit{PQ}=\textit{AP}=2,连接\textit{QC},则\textit{QC}//\textit{PO},所以\textit{EF}//\textit{QC},所以\textit{EFCQ}为平行四边形,则\textit{CF}=\textit{EQ},又因为\textit{AE}+\textit{CF}=8,\textit{AE}+\textit{A}${}_{1}$\textit{E}=8,所以\textit{A}${}_{1}$\textit{E}=\textit{CF}=\textit{EQ}=$\frac{1}{2}$\textit{A}${}_{1}$\textit{Q}=2,从而\textit{CF}=2.

\includegraphics*[width=0.83in, height=1.85in, keepaspectratio=false]{image170}

答案:2

知识:直线与平面平行的性质

难度:3

题目:如图所示,一平面与空间四边形对角线\textit{AC}、\textit{BD}都平行,且交空间四边形边\textit{AB}、\textit{BC}、\textit{CD}、\textit{DA}分别于\textit{E}、\textit{F}、\textit{G}、\textit{H}.

\includegraphics*[width=1.85in, height=1.59in, keepaspectratio=false]{image171}

(1)求证:\textit{EFGH}为平行四边形;

(2)若\textit{AC}=\textit{BD},\textit{EFGH}能否为菱形?

(3)若\textit{AC}=\textit{BD}=\textit{a},求证:平行四边形\textit{EFGH}周长为定值.

解析:

答案:(2)$\mathrm{\because}$\textit{AC}//平面\textit{EFGH},平面\textit{ACD}$\mathrm{\cap}$平面\textit{EFGH}=\textit{GH},且\textit{AC}$\mathrm{\subset }$面\textit{ACD},

$\mathrm{\therefore}$\textit{AC}//\textit{GH},同理可证,\textit{AC}//\textit{EF},\textit{BD}//\textit{EH},\textit{BD}//\textit{FG}.

$\mathrm{\therefore}$\textit{EF}//\textit{GH},\textit{EH}//\textit{FG}.$\mathrm{\therefore}$四边形\textit{EFGH}为平行四边形.

(2)设\textit{AC}=\textit{BD}=\textit{a},\textit{EH}=\textit{x},\textit{GH}=\textit{y},$\frac{AH}{HD}=\frac{m}{n}$.

$\mathrm{\because}$\textit{GH}//\textit{AC},$\mathrm{\therefore}$\textit{GH}︰\textit{AC}=\textit{DH}︰\textit{DA}=\textit{DH}︰(\textit{DH}+\textit{HA}).

即:\textit{y}︰\textit{a}=\textit{n}︰(\textit{m}+\textit{n}),$\mathrm{\therefore}$\textit{y}=$\frac{n}{m+n}$\textit{a}.

同理可得:\textit{x}=\textit{EH}=$\frac{m}{m+n}$\textit{a}.

$\mathrm{\therefore}$当\textit{AC}=\textit{BD}时,若\textit{m}=\textit{n}即\textit{AH}=\textit{HD}时,则\textit{EH}=\textit{GH},四边形\textit{EFGH}为菱形.

(3)设\textit{EH}=\textit{x},\textit{GH}=\textit{y},

\textit{H}为\textit{AD}上一点且\textit{AH}︰\textit{HD}=\textit{m}︰\textit{n}.

$\mathrm{\because}$\textit{EH}//\textit{BD},$\mathrm{\therefore}\frac{EH}{BD}=\frac{AH}{AD}$.

即$\frac{x}{a}=\frac{m}{m+n}$,$\mathrm{\therefore}$\textit{x}=$\frac{m}{m+n}$\textit{a}.

同理:\textit{y}=$\frac{n}{m+n}$\textit{a},

$\mathrm{\therefore}$周长=2(\textit{x}+\textit{y})=2\textit{a}(定值).

知识:直线与平面平行的性质

难度:3

题目:如图,在三棱柱\textit{ABC}-\textit{A}${}_{1}$\textit{B}${}_{1}$\textit{C}${}_{1}$中,点\textit{E}、\textit{F}分别是棱\textit{CC}${}_{1}$、\textit{BB}${}_{1}$上的点,点\textit{M}是线段\textit{AC}上的动点,\textit{EC}=2\textit{FB}=2,若\textit{MB}//平面\textit{AEF},试判断点\textit{M}在何位置.

\includegraphics*[width=0.98in, height=1.04in, keepaspectratio=false]{image172}

解析:

答案:若\textit{MB}//平面\textit{AEF},过\textit{F}、\textit{B}、\textit{M}作平面\textit{FBMN}交\textit{AE}于\textit{N},连接\textit{MN}、\textit{NF}.因为\textit{BF}//平面\textit{AA}${}_{1}$\textit{C}${}_{1}$\textit{C},\textit{BF}$\mathrm{\subset }$平面\textit{FBMN},平面\textit{FBMN}$\mathrm{\cap}$平面\textit{AA}${}_{1}$\textit{C}${}_{1}$\textit{C}=\textit{MN},所以\textit{BF}//\textit{MN}.

\includegraphics*[width=0.96in, height=1.02in, keepaspectratio=false]{image173}

又\textit{MB}//平面\textit{AEF},\textit{MB}$\mathrm{\subset }$平面\textit{FBMN},平面\textit{FBMN}$\mathrm{\cap}$平面\textit{AEF}=\textit{FN},所以\textit{MB}//\textit{FN},所以\textit{BFNM}是平行四边形,

所以\textit{MN}//\textit{BF},\textit{MN}=\textit{BF}=1.

而\textit{EC}//\textit{FB},\textit{EC}=2\textit{FB}=2,

所以\textit{MN}//\textit{EC},\textit{MN}=$\frac{1}{2}$\textit{EC}=1,

故\textit{MN}是$\mathrm{\vartriangle}$\textit{ACE}的中位线.

所以\textit{M}是\textit{AC}的中点时,\textit{MB}//平面\textit{AEF}.


知识:平面与平面平行的性质

难度:1

题目:若\textit{AB}、\textit{BC}、\textit{CD}是不在同一平面内的三条线段,则过它们中点的平面和直线\textit{AC}的位置关系是(  )

A.平行    B.相交

C.\textit{AC}在此平面内   D.平行或相交

解析:利用中位线性质定理得线线平行,进而得直线与平面平行.

答案:A

知识:平面与平面平行的性质

难度:1

题目:已知平面\textit{$\alpha$}//平面\textit{$\beta$},\textit{P}$\mathrm{\notin}$\textit{$\alpha$},\textit{P}$\mathrm{\notin}$\textit{$\beta$},过点\textit{P}的两直线分别交\textit{$\alpha$}、\textit{$\beta$}于\textit{A}、\textit{B}和\textit{C}、\textit{D}四点,\textit{A}、\textit{C}$\mathrm{\in}$\textit{$\alpha$},\textit{B}、\textit{D}$\mathrm{\in}$\textit{$\beta$},且\textit{PA}=6,\textit{AB}=2,\textit{BD}=12,则\textit{AC}之长为(  )

A.10或18  B.9   C.18或9  D.6

解析:由\textit{PA}=6,\textit{AB}=2知,\textit{P}点不可能在\textit{$\alpha$}与\textit{$\beta$}之间,$\mathrm{\therefore}$点\textit{P}在两平行平面所夹空间外面,$\mathrm{\therefore}\frac{PA}{PB}=\frac{AC}{BD}$或$\frac{PB}{PA}=\frac{BD}{AC}$,$\mathrm{\therefore}$\textit{AC}=9或\textit{AC}=18,$\mathrm{\therefore}$选C.

答案:C

知识:平面与平面平行的性质

难度:1

题目:若平面\textit{$\alpha$}//平面\textit{$\beta$},直线\textit{a}$\mathrm{\subset }$\textit{$\alpha$},点\textit{B}$\mathrm{\in}$\textit{$\beta$},过点\textit{B}的所有直线中(  )

A.不一定存在与\textit{a}平行的直线

B.只有两条与\textit{a}平行的直线

C.存在无数条与\textit{a}平行的直线

D.有且只有一条与\textit{a}平行的直线

解析:$\mathrm{\because}$\textit{$\alpha$}//\textit{$\beta$},\textit{B}$\mathrm{\in}$\textit{$\beta$},\textit{a}$\mathrm{\subset }$\textit{$\alpha$},$\mathrm{\therefore}$\textit{B}$\mathrm{\notin}$\textit{a},

\includegraphics*[width=1.38in, height=1.37in, keepaspectratio=false]{image175}

$\mathrm{\therefore}$点\textit{B}与直线\textit{a}确定一个平面\textit{$\gamma$},

$\mathrm{\because}$\textit{$\gamma$}与\textit{$\beta$}有一个公共点\textit{B},

$\mathrm{\therefore}$\textit{$\gamma$}与\textit{$\beta$}有且仅有一条经过点\textit{B}的直线\textit{b},

$\mathrm{\because}$\textit{$\alpha$}//\textit{$\beta$},$\mathrm{\therefore}$\textit{a}//\textit{b}.

故选D.

答案:D

知识:平面与平面平行的性质

难度:1

题目:已知\textit{a}、\textit{b}表示直线,\textit{$\alpha$}、\textit{$\beta$}、\textit{$\gamma$}表示平面,则下列推理正确的是(  )

A.\textit{$\alpha$}$\mathrm{\cap}$\textit{$\beta$}=\textit{a},\textit{b}$\mathrm{\subset }$\textit{$\alpha$}$\mathrm{\Rightarrow }$\textit{a}//\textit{b}

B.\textit{$\alpha$}$\mathrm{\cap}$\textit{$\beta$}=\textit{a},\textit{a}//\textit{b}$\mathrm{\Rightarrow }$\textit{b}//\textit{$\alpha$}且\textit{b}//\textit{$\beta$}

C.\textit{a}//\textit{$\beta$},\textit{b}//\textit{$\beta$},\textit{a}$\mathrm{\subset }$\textit{$\alpha$},\textit{b}$\mathrm{\subset }$\textit{$\alpha$}$\mathrm{\Rightarrow }$\textit{$\alpha$}//\textit{$\beta$}

D.\textit{$\alpha$}//\textit{$\beta$},\textit{$\alpha$}$\mathrm{\cap}$\textit{$\gamma$}=\textit{a},\textit{$\beta$}$\mathrm{\cap}$\textit{$\gamma$}=\textit{b}$\mathrm{\Rightarrow }$\textit{a}//\textit{b}

解析:选项A中,\textit{$\alpha$}$\mathrm{\cap}$\textit{$\beta$}=\textit{a},\textit{b}$\mathrm{\subset }$\textit{$\alpha$},则\textit{a},\textit{b}可能平行也可能相交,故A不正确;

选项B中,\textit{$\alpha$}$\mathrm{\cap}$\textit{$\beta$}=\textit{a},\textit{a}//\textit{b},则可能\textit{b}//\textit{$\alpha$}且\textit{b}//\textit{$\beta$},也可能\textit{b}在平面\textit{$\alpha$}或\textit{$\beta$}内,故B不正确;

选项C中,\textit{a}//\textit{$\beta$},\textit{b}//\textit{$\beta$},\textit{a}$\mathrm{\subset }$\textit{$\alpha$},\textit{b}$\mathrm{\subset }$\textit{$\alpha$},根据面面平行的判定定理,再加上条件\textit{a}$\mathrm{\cap}$\textit{b}=\textit{A},才能得出\textit{$\alpha$}//\textit{$\beta$},故C不正确;

选项D为面面平行性质定理的符号语言,故选D.

答案:D

知识:平面与平面平行的性质

难度:1

题目:已知两条直线\textit{m}、\textit{n}两个平面\textit{$\alpha$}、\textit{$\beta$},给出下面四个命题:

①\textit{$\alpha$}$\mathrm{\cap}$\textit{$\beta$}=\textit{m},\textit{n}$\mathrm{\subset }$\textit{$\alpha$}$\mathrm{\Rightarrow }$\textit{m}//\textit{n}或者\textit{m},\textit{n}相交;

②\textit{$\alpha$}//\textit{$\beta$},\textit{m}$\mathrm{\subset }$\textit{$\alpha$},\textit{n}$\mathrm{\subset }$\textit{$\beta$}$\mathrm{\Rightarrow }$\textit{m}//\textit{n};

③\textit{m}//\textit{n},\textit{m}//\textit{$\alpha$}$\mathrm{\Rightarrow }$\textit{n}//\textit{$\alpha$};

④\textit{$\alpha$}$\mathrm{\cap}$\textit{$\beta$}=\textit{m},\textit{m}//\textit{n}$\mathrm{\Rightarrow }$\textit{n}//\textit{$\beta$}且\textit{n}//\textit{$\alpha$}.

其中正确命题的序号是(  )

A.①   B.①④ C.④   D.③④

解析:

答案:A

知识:平面与平面平行的性质

难度:1

题目:平面\textit{$\alpha$}//平面\textit{$\beta$},$\mathrm{\vartriangle}$\textit{ABC}、$\mathrm{\vartriangle}$\textit{A}$'$\textit{B}$'$\textit{C}$'$分别在\textit{$\alpha$}、\textit{$\beta$}内,线段\textit{AA}$'$、\textit{BB}$'$、\textit{CC}$'$共点于\textit{O},\textit{O}在\textit{$\alpha$}、\textit{$\beta$}之间.若\textit{AB}=2,\textit{AC}=1,$\mathrm{\angle}$\textit{BAC}=60$\mathrm{{}^\circ}$,\textit{OA}:\textit{OA}$'$=3:2,则$\mathrm{\vartriangle}$\textit{A}$'$\textit{B}$'$\textit{C}$'$的面积为(  )

A.$\frac{\sqrt3}{9}$   B.$\frac{\sqrt3}{3}$ C.$\frac{2\sqrt3}{9}$   D.$\frac{2\sqrt3}{3}$

解析:如图$\mathrm{\because}$\textit{$\alpha$}//\textit{$\beta$},

\includegraphics*[width=1.16in, height=1.34in, keepaspectratio=false]{image176}

$\mathrm{\therefore}$\textit{BC}//\textit{B}$'$\textit{C}$'$,\textit{AB}//\textit{A}$'$\textit{B}$'$,\textit{AC}//\textit{A}$'$\textit{C}$'$,$\mathrm{\therefore}$$\mathrm{\vartriangle}$\textit{ABC}$\mathrm{\backsim}$$\mathrm{\vartriangle}$\textit{A}$'$\textit{B}$'$\textit{C}$'$,

且由$\frac{AB}{A\prime B\prime}=\frac{OA}{OA\prime}=\frac{2}{3}$知相似比为$\frac{2}{3}$,

又由\textit{AB}=2,\textit{AC}=1,$\mathrm{\angle}$\textit{BAC}=60$\mathrm{{}^\circ}$,

\includegraphics*[width=0.97in, height=0.65in, keepaspectratio=false]{image177}

知\textit{S}${}_{\vartriangle }$\textit{${}_{ABC}$}=$\frac{1}{2}$\textit{AB}·\textit{CD}=$\frac{1}{2}$\textit{AB}·(\textit{AC}·sin60$\mathrm{{}^\circ}$)=$\frac{\sqrt3}{2}$,

$\mathrm{\therefore}$\textit{S}${}_{\vartriangle }$\textit{${}_{A}$}${}_{\mathrm{\prime }}$\textit{${}_{B}$}${}_{\mathrm{\prime }}$\textit{${}_{C}$}${}_{\mathrm{\prime }}$=$\frac{2\sqrt3}{9}$.

答案:C

知识:平面与平面平行的性质

难度:1

题目:如右图是长方体被一平面所截得的几何体,四边形\textit{EFGH}为截面,则四边形\textit{EFGH}的形状为\_\_\_\_.

\includegraphics*[width=1.38in, height=1.42in, keepaspectratio=false]{image178}

解析:$\mathrm{\because}$平面\textit{ABFE}//平面\textit{CDHG},

又平面\textit{EFGH}$\mathrm{\cap}$平面\textit{ABFE}=\textit{EF},

平面\textit{EFGH}$\mathrm{\cap}$平面\textit{CDHG}=\textit{HG},

$\mathrm{\therefore}$\textit{EF}//\textit{HG}.

同理\textit{EH}//\textit{FG},

$\mathrm{\therefore}$四边形\textit{EFGH}的形状是平行四边形.

答案:平行四边形

知识:平面与平面平行的性质

难度:1

题目:如图,四棱锥\textit{P}-\textit{ABCD}中,\textit{AB}//\textit{CD},\textit{AB}=2\textit{CD},\textit{E}为\textit{PB}的中点.

\includegraphics*[width=1.16in, height=0.93in, keepaspectratio=false]{image179}

求证:\textit{CE}//平面\textit{PAD}.

解析:

答案:
解法一:如图所示,取\textit{PA}的中点\textit{H},连接\textit{EH}、\textit{DH}.

\includegraphics*[width=1.17in, height=0.94in, keepaspectratio=false]{image180}

因为\textit{E}为\textit{PB}的中点,

所以\textit{EH}//\textit{AB},\textit{EH}=$\frac{1}{2}$\textit{AB}.

又\textit{AB}//\textit{CD},\textit{CD}=$\frac{1}{2}$\textit{AB},

所以\textit{EH}//\textit{CD},\textit{EH}=\textit{CD}.

因此四边形\textit{DCEH}是平行四边形,

所以\textit{CE}//\textit{DH}.

又\textit{DH}$\mathrm{\subset }$平面\textit{PAD},\textit{CE}$\mathrm{\nsubset}$平面\textit{PAD},

因此\textit{CE}//平面\textit{PAD}.

解法二:如图所示,取\textit{AB}的中点\textit{F},连接\textit{CF}、\textit{EF},

\includegraphics*[width=1.17in, height=0.94in, keepaspectratio=false]{image181}

所以\textit{AF}=$\frac{1}{2}$\textit{AB}.

又\textit{CD}=$\frac{1}{2}$\textit{AB},所以\textit{AF}=\textit{CD}.

又\textit{AF}//\textit{CD},所以四边形\textit{AFCD}为平行四边形,

因此\textit{CF}//\textit{AD}.

又\textit{CF}$\mathrm{\nsubset}$平面\textit{PAD},所以\textit{CF}//平面\textit{PAD}.

因为\textit{E},\textit{F}分别为\textit{PB},\textit{AB}的中点,所以\textit{EF}//\textit{PA}.

又\textit{EF}$\mathrm{\nsubset}$平面\textit{PAD},所以\textit{EF}//平面\textit{PAD}.

因为\textit{CF}$\mathrm{\cap}$\textit{EF}=\textit{F},故平面\textit{CEF}//平面\textit{PAD}.

又\textit{CE}$\mathrm{\subset }$平面\textit{CEF},所以\textit{CE}//平面\textit{PAD}.

知识:平面与平面平行的性质

难度:1

题目:如图,在正方体\textit{ABCD}-\textit{A}${}_{1}$\textit{B}${}_{1}$\textit{C}${}_{1}$\textit{D}${}_{1}$中,\textit{O}为底面\textit{ABCD}的中心,\textit{P}是\textit{DD}${}_{1}$的中点,设\textit{Q}是\textit{CC}${}_{1}$上的点,问:当点\textit{Q}在什么位置时,平面\textit{D}${}_{1}$\textit{BQ}与平面\textit{PAO}平行?

\includegraphics*[width=1.12in, height=1.06in, keepaspectratio=false]{image182}

解析:

答案:当\textit{Q}为\textit{CC}${}_{1}$的中点时,平面\textit{D}${}_{1}$\textit{BQ}//平面\textit{PAO}.

连接\textit{BD},由题意可知,\textit{BD}$\mathrm{\cap}$\textit{AC}=0,

\includegraphics*[width=1.12in, height=1.06in, keepaspectratio=false]{image183}

\textit{O}为\textit{BD}的中点,又\textit{P}为\textit{DD}${}_{1}$的中点,

$\mathrm{\therefore}$\textit{OP}//\textit{BD}${}_{1}$,又\textit{BD}${}_{1}$$\mathrm{\nsubset}$平面\textit{PAO},

\textit{PO}$\mathrm{\subset }$平面\textit{PAO},

$\mathrm{\therefore}$\textit{BD}${}_{1}$//平面\textit{PAO},连接\textit{PC}.

$\mathrm{\because}$\textit{PD}${}_{1}$=\textit{CQ},$\mathrm{\therefore}$\textit{D}${}_{1}$\textit{Q}//\textit{PC}.又\textit{PC}$\mathrm{\subset }$平面\textit{PAO},\textit{D}${}_{1}$\textit{Q}$\mathrm{\nsubset}$平面\textit{PAO},$\mathrm{\therefore}$\textit{D}${}_{1}$\textit{Q}//平面\textit{PAO}.

又\textit{D}${}_{1}$\textit{Q}$\mathrm{\cap}$\textit{BD}${}_{1}$=\textit{D}${}_{1}$,$\mathrm{\therefore}$平面\textit{D}${}_{1}$\textit{BQ}//平面\textit{PAC}.

知识:平面与平面平行的性质

难度:2

题目:已知直线\textit{a}//平面\textit{$\alpha$},\textit{a}//平面\textit{$\beta$},\textit{$\alpha$}$\mathrm{\cap}$\textit{$\beta$}=\textit{b},则\textit{a}与\textit{b}(  )

A.相交    B.平行

C.异面    D.共面或异面

解析: $\mathrm{\because}$直线\textit{a}//\textit{$\alpha$},\textit{a}//\textit{$\beta$},$\mathrm{\therefore}$在平面\textit{$\alpha$}、\textit{$\beta$}中必分别有一直线平行于\textit{a},不妨设为\textit{m}、\textit{n},$\mathrm{\therefore}$\textit{a}//\textit{m},\textit{a}//\textit{n},$\mathrm{\therefore}$\textit{m}//\textit{n}.又\textit{$\alpha$}、\textit{$\beta$}相交,\textit{m}在平面\textit{$\alpha$}内,\textit{n}在平面\textit{$\beta$}内,$\mathrm{\therefore}$\textit{m}//\textit{$\beta$},$\mathrm{\therefore}$\textit{m}//\textit{b},$\mathrm{\therefore}$\textit{a}//\textit{b}.故选B.

答案:B

知识:平面与平面平行的性质

难度:2

题目:如图,在多面体\textit{ABC}-\textit{DEFG}中,平面\textit{ABC}//平面\textit{DEFG},\textit{EF}//\textit{DG},且\textit{AB}=\textit{DE},\textit{DG}=2\textit{EF},则(  )

\includegraphics*[width=1.35in, height=1.29in, keepaspectratio=false]{image184}

A.\textit{BF}//平面\textit{ACGD}   B.\textit{CF}//平面\textit{ABED}

C.\textit{BC}//\textit{FG}    D.平面\textit{ABED}//平面\textit{CGF}

解析:取\textit{DG}的中点为\textit{M},连接\textit{AM}、\textit{FM},如图所示.

\includegraphics*[width=1.35in, height=1.29in, keepaspectratio=false]{image185}

则由已知条件易证四边形\textit{DEFM}是平行四边形$\mathrm{\therefore}$\textit{DE}=\textit{FM}.

$\mathrm{\because}$平面\textit{ABC}//平面\textit{DEFG},平面\textit{ABC}$\mathrm{\cap}$平面\textit{ADEB}=\textit{AB},

平面\textit{DEFG}$\mathrm{\cap}$平面\textit{ADEB}=\textit{DE},$\mathrm{\therefore}$\textit{AB}//\textit{DE},$\mathrm{\therefore}$\textit{AB}//\textit{FM}.

又\textit{AB}=\textit{DE},$\mathrm{\therefore}$\textit{AB}=\textit{FM},

$\mathrm{\therefore}$四边形\textit{ABFM}是平行四边形,即\textit{BF}//\textit{AM}.

又\textit{BF}$\mathrm{\nsubset}$平面\textit{ACGD},$\mathrm{\therefore}$\textit{BF}//平面\textit{ACGD}.故选A.

答案:A

知识:平面与平面平行的性质

难度:2

题目:设平面\textit{$\alpha$}//平面\textit{$\beta$},点\textit{A}$\mathrm{\in}$\textit{$\alpha$},点\textit{B}$\mathrm{\in}$\textit{$\beta$},\textit{C}是\textit{AB}的中点,当点\textit{A}、\textit{B}分别在平面\textit{$\alpha$}、\textit{$\beta$}内运动时,那么所有的动点\textit{C}(  )

A.不共面

B.不论\textit{A}、\textit{B}如何移动,都共面

C.当且仅当\textit{A}、\textit{B}分别在两直线上移动时时才共面

D.当且仅当\textit{A}、\textit{B}分别在两条给定的异面直线上移动时才共面

解析:如图,不论点\textit{A}、\textit{B}如何移动,点\textit{C}都共面,且所在平面与平面\textit{$\alpha$}、平面\textit{$\beta$}平行.

\includegraphics*[width=1.43in, height=1.09in, keepaspectratio=false]{image186}

答案:B

知识:平面与平面平行的性质

难度:2

题目:\textit{a}//\textit{$\alpha$},\textit{b}//\textit{$\beta$},\textit{$\alpha$}//\textit{$\beta$},则\textit{a}与\textit{b}位置关系是(  )

A.平行      B.异面

C.相交      D.平行或异面或相交

解析:如图(1),(2),(3)所示,\textit{a}与\textit{b}的关系分别是平行、异面或相交.

\includegraphics*[width=2.95in, height=1.12in, keepaspectratio=false]{image187}

答案:D


知识:平面与平面平行的性质

难度:2

题目:如图所示,平面四边形\textit{ABCD}所在的平面与平面\textit{$\alpha$}平行,且四边形\textit{ABCD}在平面\textit{$\alpha$}内的平行投影\textit{A}${}_{1}$\textit{B}${}_{1}$\textit{C}${}_{1}$\textit{D}${}_{1}$是一个平行四边形,则四边形\textit{ABCD}的形状一定是\_\_\_\_.

\includegraphics*[width=1.04in, height=0.92in, keepaspectratio=false]{image188}

解析:$\mathrm{\because}$平面\textit{AC}//\textit{$\alpha$},平面\textit{AA}${}_{1}$\textit{B}${}_{1}$\textit{B}$\mathrm{\cap}$\textit{$\alpha$}=\textit{A}${}_{1}$\textit{B}${}_{1}$,平面\textit{AA}${}_{1}$\textit{B}${}_{1}$\textit{B}$\mathrm{\cap}$平面\textit{ABCD}=\textit{AB},$\mathrm{\therefore}$\textit{AB}//\textit{A}${}_{1}$\textit{B}${}_{1}$,同理可证\textit{CD}//\textit{C}${}_{1}$\textit{D}${}_{1}$,又\textit{A}${}_{1}$\textit{B}${}_{1}$//\textit{C}${}_{1}$\textit{D}${}_{1}$,$\mathrm{\therefore}$\textit{AB}//\textit{CD},同理可证\textit{AD}//\textit{BC},$\mathrm{\therefore}$四边形\textit{ABCD}是平行四边形.

答案:平行四边形

知识:平面与平面平行的性质

难度:2

题目:如图,在四面体\textit{ABCD}中,若截面\textit{PQMN}是正方形,则在下列结论中正确的为\_\_\_\_.

\includegraphics*[width=1.19in, height=1.05in, keepaspectratio=false]{image189}

①\textit{AC}$\mathrm{\bot}$\textit{BD};

②\textit{AC}//截面\textit{PQMN};

③\textit{AC}=\textit{BD};

④异面直线\textit{PM}与\textit{BD}所成的角为45$\mathrm{{}^\circ}$.

解析:$\mathrm{\because}$\textit{MN}//\textit{PQ},$\mathrm{\therefore}$\textit{PQ}//平面\textit{ACD},又平面\textit{ACD}$\mathrm{\cap}$平面\textit{ABC}=\textit{AC},$\mathrm{\therefore}$\textit{PQ}//\textit{AC},从而\textit{AC}//截面\textit{PQMN},②正确;同理可得\textit{MQ}//\textit{BD},故\textit{AC}$\mathrm{\bot}$\textit{BD},①正确;又\textit{MQ}//\textit{BD},$\mathrm{\angle}$\textit{PMQ}=45$\mathrm{{}^\circ}$,$\mathrm{\therefore}$异面直线\textit{PM}与\textit{BD}所成的角为45$\mathrm{{}^\circ}$,故④正确.根据已知条件无法得到\textit{AC},\textit{BD}长度之间的关系.故填①②④.

答案:①②④

知识:平面与平面平行的性质

难度:3

题目:如图,在四棱锥\textit{O}-\textit{ABCD}中,底面\textit{ABCD}是菱形,\textit{M}为\textit{OA}的中点,\textit{N}为\textit{BC}的中点.证明:直线\textit{MN}//平面\textit{OCD}.

\includegraphics*[width=1.27in, height=1.34in, keepaspectratio=false]{image190}

解析:

答案:
解法一:如图(1),取\textit{OB}的中点\textit{G},连接\textit{GN}、\textit{GM}.

$\mathrm{\because}$\textit{M}为\textit{OA}的中点,$\mathrm{\therefore}$\textit{MG}//\textit{AB}.

$\mathrm{\because}$\textit{AB}//\textit{CD},$\mathrm{\therefore}$\textit{MG}//\textit{CD}.

$\mathrm{\because}$\textit{MG}$\mathrm{\nsubset}$平面\textit{OCD},\textit{CD}$\mathrm{\subset }$平面\textit{OCD},

$\mathrm{\therefore}$\textit{MG}//平面\textit{OCD}.

又$\mathrm{\because}$\textit{G}、\textit{N}分别为\textit{OB}、\textit{BC}的中点,

$\mathrm{\therefore}$\textit{GN}//\textit{OC}.

$\mathrm{\because}$\textit{GN}$\mathrm{\nsubset}$平面\textit{OCD},\textit{OC}$\mathrm{\subset }$平面\textit{OCD},

$\mathrm{\therefore}$\textit{GN}//平面\textit{OCD}.

又$\mathrm{\because}$\textit{MG}$\mathrm{\subset }$平面\textit{MNG},\textit{GN}$\mathrm{\subset }$平面\textit{MNG},\textit{MG}$\mathrm{\cap}$\textit{GN}=\textit{G},

$\mathrm{\therefore}$平面\textit{MNG}//平面\textit{OCD}.

$\mathrm{\because}$\textit{MN}$\mathrm{\subset }$平面\textit{MNG},

$\mathrm{\therefore}$\textit{MN}//平面\textit{OCD}.

\includegraphics*[width=2.58in, height=1.53in, keepaspectratio=false]{image191}

解法二:如图(2),取\textit{OD}的中点\textit{P},连接\textit{MP}、\textit{CP}.

$\mathrm{\because}$\textit{M}为\textit{OA}的中点,$\mathrm{\therefore}$\textit{MP}=$\frac{1}{2}$\textit{AD}.

$\mathrm{\because}$\textit{N}为\textit{BC}的中点,$\mathrm{\therefore}$\textit{CN}$=\frac{1}{2}$\textit{AD},

$\mathrm{\therefore}$\textit{MP}=\textit{CN},$\mathrm{\therefore}$四边形\textit{MNCP}为平行四边形,

$\mathrm{\therefore}$\textit{MN}//\textit{PC}.

又$\mathrm{\because}$\textit{MN}$\mathrm{\nsubset}$平面\textit{OCD},\textit{PC}$\mathrm{\subset }$平面\textit{OCD},

$\mathrm{\therefore}$\textit{MN}//平面\textit{OCD}.

知识:平面与平面平行的性质

难度:3

题目:如图,在四棱柱\textit{ABCD}-\textit{A}${}_{1}$\textit{B}${}_{1}$\textit{C}${}_{1}$\textit{D}${}_{1}$中,底面\textit{ABCD}为梯形,\textit{AD}//\textit{BC},平面\textit{A}${}_{1}$\textit{DCE}与\textit{B}${}_{1}$\textit{B}交于点\textit{E}.

\includegraphics*[width=1.15in, height=1.31in, keepaspectratio=false]{image192}

求证:\textit{EC}//\textit{A}${}_{1}$\textit{D}.

解析:

答案:
因为\textit{BE}//\textit{AA}${}_{1}$,\textit{AA}${}_{1}$$\mathrm{\subset }$平面\textit{AA}${}_{1}$\textit{D},\textit{BE}$\mathrm{\nsubset}$平面\textit{AA}${}_{1}$\textit{D},

所以\textit{BE}//平面\textit{AA}${}_{1}$\textit{D}.

因为\textit{BC}//\textit{AD},\textit{AD}$\mathrm{\subset }$平面\textit{AA}${}_{1}$\textit{D},\textit{BC}$\mathrm{\nsubset}$平面\textit{AA}${}_{1}$\textit{D},

所以\textit{BC}//平面\textit{AA}${}_{1}$\textit{D}.

又\textit{BE}$\mathrm{\cap}$\textit{BC}=\textit{B},\textit{BE}$\mathrm{\subset }$平面\textit{BCE},\textit{BC}$\mathrm{\subset }$平面\textit{BCE},

所以平面\textit{BCE}//平面\textit{AA}${}_{1}$\textit{D}.

又平面\textit{A}${}_{1}$\textit{DCE}$\mathrm{\cap}$平面\textit{BCE}=\textit{EC},平面\textit{A}${}_{1}$\textit{DCE}$\mathrm{\cap}$平面\textit{AA}${}_{1}$\textit{D}=\textit{A}${}_{1}$\textit{D},

所以\textit{EC}//\textit{A}${}_{1}$\textit{D}.

知识:直线与平面垂直的判定

难度:1

题目:一条直线和平面所成角为\textit{$\theta$},那么\textit{$\theta$}的取值范围是(  )

A.(0$\mathrm{{}^\circ}$,90$\mathrm{{}^\circ}$)   B.[0$\mathrm{{}^\circ}$,90$\mathrm{{}^\circ}$] C.(0$\mathrm{{}^\circ}$,90$\mathrm{{}^\circ}$]   D.[0$\mathrm{{}^\circ}$,180$\mathrm{{}^\circ}$]

解析:由线面角的定义知B正确.

答案:B

知识:直线与平面垂直的判定

难度:1

题目:在正方体\textit{ABCD}-\textit{A}${}_{1}$\textit{B}${}_{1}$\textit{C}${}_{1}$\textit{D}${}_{1}$的六个面中,与\textit{AA}${}_{1}$垂直的平面的个数是(  )

A.1   B.2   C.3   D.6

解析:仅有平面\textit{AC}和平面\textit{A}${}_{1}$\textit{C}${}_{1}$与直线\textit{AA}${}_{1}$垂直.

答案:B

知识:直线与平面垂直的判定

难度:1

题目:如图,在四棱锥\textit{P}-\textit{ABCD}中,底面\textit{ABCD}为矩形,\textit{PA}$\mathrm{\bot}$平面\textit{ABCD},则图中共有直角三角形的个数为(  )

\includegraphics*[width=1.18in, height=0.98in, keepaspectratio=false]{image194}

A.1   B.2   C.3   D.4

解析:$\mathrm{\because}$\textit{PA}$\mathrm{\bot}$平面\textit{ABCD},

$\mathrm{\therefore}$\textit{PA}$\mathrm{\bot}$\textit{AB},\textit{PA}$\mathrm{\bot}$\textit{AD},\textit{PA}$\mathrm{\bot}$\textit{BC},\textit{PA}$\mathrm{\bot}$\textit{CD}.

$\left. \begin{array}{l}
AB\bot BC\\
PA\bot BC\\
PA \cap AB = A
\end{array}\right\} \mathrm{\Rightarrow }$\textit{BC}$\mathrm{\bot}$平面\textit{PAB}$\mathrm{\Rightarrow }$\textit{BC}$\mathrm{\bot}$\textit{PB}

由$\left.\begin{array}{r}
CD\bot AD\\
CD\bot PA\\
PA \cap AD = A
\end{array}\right\}\mathrm{\Rightarrow }$\textit{CD}$\mathrm{\bot}$平面\textit{PAD}$\mathrm{\Rightarrow }$\textit{CD}$\mathrm{\bot}$\textit{PD}.

$\mathrm{\therefore}$$\mathrm{\vartriangle}$\textit{PAB},$\mathrm{\vartriangle}$\textit{PAD},$\mathrm{\vartriangle}$\textit{PBC},$\mathrm{\vartriangle}$\textit{PCD}都是直角三角形.

答案:D

知识:直线与平面垂直的判定

难度:1

题目:直线\textit{a}与平面\textit{$\alpha$}所成的角为50$\mathrm{{}^\circ}$,直线\textit{b}//\textit{a},则直线\textit{b}与平面\textit{$\alpha$}所成的角等于(  )

A.40$\mathrm{{}^\circ}$   B.50$\mathrm{{}^\circ}$   C.90$\mathrm{{}^\circ}$   D.150$\mathrm{{}^\circ}$

解析:根据两条平行直线和同一平面所成的角相等,知\textit{b}与\textit{$\alpha$}所成的角也是50$\mathrm{{}^\circ}$.

答案:B

知识:直线与平面垂直的判定

难度:1

题目:给出下列三个命题:

①一条直线垂直于一个平面内的三条直线,则这条直线和这个平面垂直;

②一条直线与一个平面内的任何直线所成的角相等,则这条直线和这个平面垂直;

③一条直线在平面内的射影是一点,则这条直线和这个平面垂直.

其中正确的个数是(  )

A.0   B.1   C.2   D.3

解析:①中三条直线不一定存在两条直线相交,因此直线不一定与平面垂直;②中直线与平面所成角必为直角,因此直线与平面垂直;③根据射影定义知正确.故选C.

答案:C

知识:直线与平面垂直的判定

难度:1

题目:如图,已知六棱锥\textit{P}-\textit{ABCDEF}的底面是正六边形,\textit{PA}$\mathrm{\bot}$平面\textit{ABC},\textit{PA}=2\textit{AB},则下列结论正确的是(  )

\includegraphics*[width=1.15in, height=1.07in, keepaspectratio=false]{image195}

A.\textit{PB}$\mathrm{\bot}$\textit{AD}

B.平面\textit{PAB}$\mathrm{\bot}$平面\textit{PBC}

C.直线\textit{BC}//平面\textit{PAE}

D.直线\textit{PD}与平面\textit{ABC}所成的角为45$\mathrm{{}^\circ}$

解析:设\textit{AB}长为1,由\textit{PA}=2\textit{AB}得\textit{PA}=2,

又\textit{ABCDEF}是正六边形,所以\textit{AD}长也为2,

又\textit{PA}$\mathrm{\bot}$平面\textit{ABC},所以\textit{PA}$\mathrm{\bot}$\textit{AD},

所以$\mathrm{\vartriangle}$\textit{PAD}为直角三角形.

$\mathrm{\because}$\textit{PA}=\textit{AD},$\mathrm{\therefore}$$\mathrm{\angle}$\textit{PDA}=45$\mathrm{{}^\circ}$,

$\mathrm{\therefore}$\textit{PD}与平面\textit{ABC}所成的角为45$\mathrm{{}^\circ}$,故选D.

答案:D


知识:直线与平面垂直的判定

难度:1

题目:已知$\mathrm{\vartriangle}$\textit{ABC}所在平面外一点\textit{P}到$\mathrm{\vartriangle}$\textit{ABC}三顶点的距离都相等,则点\textit{P}在平面\textit{ABC}内的射影是$\mathrm{\vartriangle}$\textit{ABC}的\_\_\_\_.(填``重心''、``外心''、``内心''、``垂心'')

解析:\textit{P}到$\mathrm{\vartriangle}$\textit{ABC}三顶点的距离都相等,则点\textit{P}在平面\textit{ABC}内的射影到$\mathrm{\vartriangle}$\textit{ABC}三顶点的距离都相等,所以是外心.

答案:外心

知识:直线与平面垂直的判定

难度:1

题目:等腰直角三角形\textit{ABC}的斜边\textit{AB}在平面\textit{$\alpha$}内,若\textit{AC}与\textit{$\alpha$}所成的角为30$\mathrm{{}^\circ}$,则斜边上的中线\textit{CM}与\textit{$\alpha$}所成的角为\_\_\_\_.

\includegraphics*[width=1.08in, height=0.79in, keepaspectratio=false]{image196}

解析:如图,设\textit{C}在平面\textit{$\alpha$}内的射影为\textit{O}点,

连结\textit{AO},\textit{MO},

则$\mathrm{\angle}$\textit{CAO}=30$\mathrm{{}^\circ}$,$\mathrm{\angle}$\textit{CMO}就是\textit{CM}与\textit{$\alpha$}所成的角.

设\textit{AC}=\textit{BC}=1,则\textit{AB}=$\sqrt{2}$,

$\mathrm{\therefore}$\textit{CM}=$\frac{\sqrt{2}}{2}$,\textit{CO}=$\frac{1}{2}$.

$\mathrm{\therefore}$sin\textit{CMO}=$\frac{CO}{CM}$=$\frac{\sqrt{2}}{2}$,

$\mathrm{\therefore}$$\mathrm{\angle}$\textit{CMO}=45$\mathrm{{}^\circ}$.

答案:45$\mathrm{{}^\circ}$

知识:直线与平面垂直的判定

难度:1

题目:如图,在三棱锥\textit{A}-\textit{BCD}中,\textit{CA}=\textit{CB},\textit{DA}=\textit{DB}.作\textit{BE}$\mathrm{\bot}$\textit{CD},\textit{E}为垂足,作\textit{AH}$\mathrm{\bot}$\textit{BE}于\textit{H}.求证:\textit{AH}$\mathrm{\bot}$平面\textit{BCD}.

\includegraphics*[width=1.02in, height=0.94in, keepaspectratio=false]{image197}

解析:

答案:取\textit{AB}的中点\textit{F},连接\textit{CF}、\textit{DF}.

$\mathrm{\because}$\textit{CA}=\textit{CB},\textit{DA}=\textit{DB},$\mathrm{\therefore}$\textit{CF}$\mathrm{\bot}$\textit{AB},\textit{DF}$\mathrm{\bot}$\textit{AB}.

\includegraphics*[width=1.02in, height=0.94in, keepaspectratio=false]{image198}

$\mathrm{\because}$\textit{CF}$\mathrm{\cap}$\textit{DF}=\textit{F},$\mathrm{\therefore}$\textit{AB}$\mathrm{\bot}$平面\textit{CDF}.

$\mathrm{\because}$\textit{CD}$\mathrm{\subset }$平面\textit{CDF},$\mathrm{\therefore}$\textit{AB}$\mathrm{\bot}$\textit{CD}.

又\textit{CD}$\mathrm{\bot}$\textit{BE},\textit{AB}$\mathrm{\cap}$\textit{BE}=\textit{B},$\mathrm{\therefore}$\textit{CD}$\mathrm{\bot}$平面\textit{ABE}.

$\mathrm{\because}$\textit{AH}$\mathrm{\subset }$平面\textit{ABE},$\mathrm{\therefore}$\textit{CD}$\mathrm{\bot}$\textit{AH}.

$\mathrm{\because}$\textit{AH}$\mathrm{\bot}$\textit{BE},\textit{BE}$\mathrm{\cap}$\textit{CD}=\textit{E},$\mathrm{\therefore}$\textit{AH}$\mathrm{\bot}$平面\textit{BCD}.

知识:直线与平面垂直的判定

难度:1

题目:如图在三棱锥\textit{P}-\textit{ABC}中,\textit{PA}=\textit{PB}=\textit{PC}=13,$\mathrm{\angle}$\textit{ABC}=90$\mathrm{{}^\circ}$,\textit{AB}=8,\textit{BC}=6,\textit{M}为\textit{AC}的中点.

\includegraphics*[width=1.25in, height=1.20in, keepaspectratio=false]{image199}

(2)求证:\textit{PM}$\mathrm{\bot}$平面\textit{ABC};

(2)求直线\textit{BP}与平面\textit{ABC}所成的角的正切值.

解析:

答案:
(1)$\mathrm{\because}$\textit{PA}=\textit{PC},\textit{M}为\textit{AC}的中点,

$\mathrm{\therefore}$\textit{PM}$\mathrm{\bot}$\textit{AC}.①

又$\mathrm{\angle}$\textit{ABC}=90$\mathrm{{}^\circ}$,\textit{AB}=8,\textit{BC}=6,

$\mathrm{\therefore}$\textit{AM}=\textit{MC}=\textit{MB}=$\frac{1}{2}$\textit{AC}=5.

在$\mathrm{\vartriangle}$\textit{PMB}中,\textit{PB}=13,\textit{MB}=5.

\textit{PM}=$\sqrt{PC^2-MC^2}=\sqrt{13^2-5^2}=12$.

$\mathrm{\therefore}$\textit{PB}${}^{2}$=\textit{MB}${}^{2}$+\textit{PM}${}^{2}$,

$\mathrm{\therefore}$\textit{PM}$\mathrm{\bot}$\textit{MB}.②

由①②可知\textit{PM}$\mathrm{\bot}$平面\textit{ABC}.

(2)$\mathrm{\because}$\textit{PM}$\mathrm{\bot}$平面\textit{ABC},

$\mathrm{\therefore}$\textit{MB}为\textit{BP}在平面\textit{ABC}内的射影,

$\mathrm{\therefore}$$\mathrm{\angle}$\textit{PBM}为\textit{BP}与底面\textit{ABC}所成的角.

在Rt$\mathrm{\vartriangle}$\textit{PMB}中tan$\mathrm{\angle}$\textit{PBM}=$\frac{PM}{MB}$=$\frac{12}{5}$.

知识:直线与平面垂直的判定

难度:2

题目:如图,在正方体\textit{ABCD}-\textit{A}${}_{1}$\textit{B}${}_{1}$\textit{C}${}_{1}$\textit{D}${}_{1}$中,\textit{E}为\textit{A}${}_{1}$\textit{C}${}_{1}$上的点,则下列直线中一定与\textit{CE}垂直的是(  )

\includegraphics*[width=1.25in, height=1.25in, keepaspectratio=false]{image200}

A.\textit{AC }B.\textit{BD }C.\textit{A}${}_{1}$\textit{D}${}_{1}$${}_{ }$D.\textit{A}${}_{1}$\textit{A}

解析:$\mathrm{\because}$\textit{BD}$\mathrm{\bot}$\textit{AC},\textit{BD}$\mathrm{\bot}$\textit{A}${}_{1}$\textit{A},\textit{AC}$\mathrm{\cap}$\textit{A}${}_{1}$\textit{A}=\textit{A},$\mathrm{\therefore}$\textit{BD}$\mathrm{\bot}$平面\textit{ACC}${}_{1}$\textit{A}${}_{1}$.

又$\mathrm{\because}$\textit{CE}$\mathrm{\subset }$平面\textit{ACC}${}_{1}$\textit{A}${}_{1}$,$\mathrm{\therefore}$\textit{BD}$\mathrm{\bot}$\textit{CE}.

答案:B

知识:直线与平面垂直的判定

难度:2

题目:空间四边形\textit{ABCD}的四边相等,则它的两对角线\textit{AC}、\textit{BD}的关系是(  )

A.垂直且相交    B.相交但不一定垂直

C.垂直但不相交    D.不垂直也不相交

解析:
取\textit{BD}中点\textit{O},

连接\textit{AO}、\textit{CO},

\includegraphics*[width=0.94in, height=1.08in, keepaspectratio=false]{image201}

则\textit{BD}$\mathrm{\bot}$\textit{AO},\textit{BD}$\mathrm{\bot}$\textit{CO},

$\mathrm{\therefore}$\textit{BD}$\mathrm{\bot}$面\textit{AOC},\textit{BD}$\mathrm{\bot}$\textit{AC},

又\textit{BD}、\textit{AC}异面,$\mathrm{\therefore}$选C.

答案:C

知识:直线与平面垂直的判定

难度:2

题目:如图,三条相交于点P的线段PA,PB,PC两两垂直,P在平面ABC外,PH$\bot$平面ABC于H,则垂足H是△ABC的(  )

A.外心  	B.内心  	C.垂心  	D.重心

解析:$\mathrm{\because}$\textit{PC}$\mathrm{\bot}$\textit{PA},\textit{PC}$\mathrm{\bot}$\textit{PB},

\textit{PA}$\mathrm{\cap}$\textit{PB}=\textit{P},$\mathrm{\therefore}$\textit{PC}$\mathrm{\bot}$平面\textit{PAB}.

又$\mathrm{\because}$\textit{AB}$\mathrm{\subset }$平面\textit{PAB},$\mathrm{\therefore}$\textit{AB}$\mathrm{\bot}$\textit{PC}.

又$\mathrm{\because}$\textit{AB}$\mathrm{\bot}$\textit{PH},\textit{PH}$\mathrm{\cap}$\textit{PC}=\textit{P},$\mathrm{\therefore}$\textit{AB}$\mathrm{\bot}$平面\textit{PCH}.

又$\mathrm{\because}$\textit{CH}$\mathrm{\subset }$平面\textit{PCH},$\mathrm{\therefore}$\textit{AB}$\mathrm{\bot}$\textit{CH}.

同理\textit{BC}$\mathrm{\bot}$\textit{AH},\textit{AC}$\mathrm{\bot}$\textit{BH}.

$\mathrm{\therefore}$\textit{H}为$\mathrm{\vartriangle}$\textit{ABC}的垂心.

答案:C

知识:直线与平面垂直的判定

难度:2

题目:如图,\textit{ABCD}-\textit{A}${}_{1}$\textit{B}${}_{1}$\textit{C}${}_{1}$\textit{D}${}_{1}$为正方体,下面结论错误的是(  )

\includegraphics*[width=1.07in, height=1.02in, keepaspectratio=false]{image203}

A.\textit{BD}//平面\textit{CB}${}_{1}$\textit{D}${}_{1}$${}_{  }$B.\textit{AC}${}_{1}$$\mathrm{\bot}$\textit{BD}

C.\textit{AC}${}_{1}$$\mathrm{\bot}$平面\textit{CB}${}_{1}$\textit{D}${}_{1}$${}_{  }$D.异面直线\textit{AD}与\textit{CB}${}_{1}$所成的角为60$\mathrm{{}^\circ}$

解析:
$\mathrm{\because}$\textit{AD}//\textit{BC},$\mathrm{\therefore}$$\mathrm{\angle}$\textit{BCB}${}_{1}$为异面直线\textit{AD}与\textit{CB}${}_{1}$所成的角.又$\mathrm{\vartriangle}$\textit{B}${}_{1}$\textit{BC}为等腰直角三角形,故$\mathrm{\angle}$\textit{BCB}${}_{1}$=45$\mathrm{{}^\circ}$.即异面直线\textit{AD}与\textit{CB}${}_{1}$所成的角为45$\mathrm{{}^\circ}$.

答案:D

知识:直线与平面垂直的判定

难度:2

题目:已知\textit{PA}垂直于平行四边形\textit{ABCD}所在的平面,若\textit{PC}$\mathrm{\bot}$\textit{BD},则平行四边形\textit{ABCD}一定是\_\_\_\_.

解析:由于\textit{PA}$\mathrm{\bot}$平面\textit{ABCD},\textit{BD}$\mathrm{\subset }$平面\textit{ABCD},

所以\textit{PA}$\mathrm{\bot}$\textit{BD}.

又\textit{PC}$\mathrm{\bot}$\textit{BD},且\textit{PC}$\mathrm{\subset }$平面\textit{PAC},\textit{PA}$\mathrm{\subset }$平面\textit{PAC},\textit{PC}$\mathrm{\cap}$\textit{PA}=\textit{P},所以\textit{BD}$\mathrm{\bot}$平面\textit{PAC}.

又\textit{AC}$\mathrm{\subset }$平面\textit{PAC},所以\textit{BD}$\mathrm{\bot}$\textit{AC}.

又四边形\textit{ABCD}是平行四边形,所以四边形\textit{ABCD}是菱形.

答案:菱形

知识:直线与平面垂直的判定

难度:2

题目:如图所示,已知在矩形\textit{ABCD}中,\textit{AB}=1,\textit{BC}=\textit{a}(\textit{a}$\mathrm{>}$0),\textit{PA}$\mathrm{\bot}$平面\textit{AC},且\textit{PA}=1,若\textit{BC}边上存在点\textit{Q},使得\textit{PQ}$\mathrm{\bot}$\textit{QD},则\textit{a}的取值范围是\_\_\_\_.

\includegraphics*[width=1.22in, height=0.90in, keepaspectratio=false]{image204}

解析:因为\textit{PA}$\mathrm{\bot}$平面\textit{AC},\textit{QD}$\mathrm{\subset }$平面\textit{AC},$\mathrm{\therefore}$\textit{PA}$\mathrm{\bot}$\textit{QD}.

又$\mathrm{\because}$\textit{PQ}$\mathrm{\bot}$\textit{QD},\textit{PA}$\mathrm{\cap}$\textit{PQ}=\textit{P},

$\mathrm{\therefore}$\textit{QD}$\mathrm{\bot}$平面\textit{PAQ},所以\textit{AQ}$\mathrm{\bot}$\textit{QD}.

①当0$\mathrm{<}$\textit{a}$\mathrm{<}$2时,由四边形\textit{ABCD}是矩形且\textit{AB}=1知,以\textit{AD}为直径的圆与\textit{BC}无交点,即对\textit{BC}上任一点\textit{Q},都有$\mathrm{\angle}$\textit{AQD}$\mathrm{<}$90$\mathrm{{}^\circ}$,此时\textit{BC}边上不存在点\textit{Q},使\textit{PQ}$\mathrm{\bot}$\textit{QD};

②当\textit{a}=2时,以\textit{AD}为直径的圆与\textit{BC}相切于\textit{BC}的中点\textit{Q},此时$\mathrm{\angle}$\textit{AQD}=90$\mathrm{{}^\circ}$,所以\textit{BC}边上存在一点\textit{Q},使\textit{PQ}$\mathrm{\bot}$\textit{QD};

③当\textit{a}$\mathrm{>}$2时,以\textit{AD}为直径的圆与\textit{BC}相交于点\textit{Q}${}_{1}$、\textit{Q}${}_{2}$,此时$\mathrm{\angle}$\textit{AQ}${}_{1}$\textit{D}=$\mathrm{\angle}$\textit{AQ}${}_{2}$\textit{D}=90$\mathrm{{}^\circ}$,故\textit{BC}边上存在两点\textit{Q}(即\textit{Q}${}_{1}$与\textit{Q}${}_{2}$),使\textit{PQ}$\mathrm{\bot}$\textit{QD}.

答案:[2,+$\mathrm{\infty}$)

知识:直线与平面垂直的判定

难度:3

题目:如图所示,在棱长为1的正方体\textit{ABCD}-\textit{A}${}_{1}$\textit{B}${}_{1}$\textit{C}${}_{1}$\textit{D}${}_{1}$中,点\textit{E}是棱\textit{BC}的中点,点\textit{F}是棱\textit{CD}上的动点.试确定点\textit{F}的位置,使得\textit{D}${}_{1}$\textit{E}$\mathrm{\bot}$平面\textit{AB}${}_{1}$\textit{F}.

\includegraphics*[width=1.34in, height=1.23in, keepaspectratio=false]{image205}

解析:

答案:当\textit{F}为\textit{CD}的中点时,\textit{D}${}_{1}$\textit{E}$\mathrm{\bot}$平面\textit{AB}${}_{1}$\textit{F}.

\includegraphics*[width=1.34in, height=1.23in, keepaspectratio=false]{image206}

连接\textit{A}${}_{1}$\textit{B}、\textit{CD}${}_{1}$,则\textit{A}${}_{1}$\textit{B}$\mathrm{\bot}$\textit{AB}${}_{1}$,\textit{A}${}_{1}$\textit{D}${}_{1}$$\mathrm{\bot}$\textit{AB}${}_{1}$,又\textit{A}${}_{1}$\textit{D}${}_{1}$$\mathrm{\cap}$\textit{A}${}_{1}$\textit{B}=\textit{A}${}_{1}$,$\mathrm{\therefore}$\textit{AB}${}_{1}$$\mathrm{\bot}$面\textit{A}${}_{1}$\textit{BCD}${}_{1}$,

又\textit{D}${}_{1}$\textit{E}$\mathrm{\subset }$面\textit{A}${}_{1}$\textit{BCD}${}_{1}$,

$\mathrm{\therefore}$\textit{AB}${}_{1}$$\mathrm{\bot}$\textit{D}${}_{1}$\textit{E}.

又\textit{DD}${}_{1}$$\mathrm{\bot}$平面\textit{BD},

$\mathrm{\therefore}$\textit{AF}$\mathrm{\bot}$\textit{DD}${}_{1}$.

又\textit{AF}$\mathrm{\bot}$\textit{DE},

$\mathrm{\therefore}$\textit{AF}$\mathrm{\bot}$平面\textit{D}${}_{1}$\textit{DE},

$\mathrm{\therefore}$\textit{AF}$\mathrm{\bot}$\textit{D}${}_{1}$\textit{E}.

$\mathrm{\therefore}$\textit{D}${}_{1}$\textit{E}$\mathrm{\bot}$平面\textit{AB}${}_{1}$\textit{E}.

即当点\textit{F}是\textit{CD}的中点时,\textit{D}${}_{1}$\textit{E}$\mathrm{\bot}$平面\textit{AB}${}_{1}$\textit{F}. 

知识:直线与平面垂直的判定

难度:3

题目:如图,在锥体\textit{P}-\textit{ABCD}中,底面\textit{ABCD}是菱形,且$\mathrm{\angle}$\textit{DAB}=60$\mathrm{{}^\circ}$,\textit{PA}=\textit{PD},\textit{E},\textit{F}分别是\textit{BC},\textit{PC}的中点.

求证:\textit{AD}$\mathrm{\bot}$平面\textit{DEF}.

\includegraphics*[width=1.38in, height=0.88in, keepaspectratio=false]{image207}

解析:

答案:取\textit{AD}的中点\textit{G},连接\textit{PG},\textit{BG}.因为\textit{PA}=\textit{PD},

\includegraphics*[width=1.38in, height=0.88in, keepaspectratio=false]{image208}

所以\textit{AD}$\mathrm{\bot}$\textit{PG}.

设菱形\textit{ABCD}边长为1.

在$\mathrm{\vartriangle}$\textit{ABG}中,因为$\mathrm{\angle}$\textit{GAB}=60$\mathrm{{}^\circ}$,\textit{AG}=$\frac{1}{2}$,\textit{AB}=1,

所以$\mathrm{\angle}$\textit{AGB}=90$\mathrm{{}^\circ}$,即\textit{AD}$\mathrm{\bot}$\textit{GB}.

又\textit{PG}$\mathrm{\cap}$\textit{GB}=\textit{G},所以\textit{AD}$\mathrm{\bot}$平面\textit{PGB},

从而\textit{AD}$\mathrm{\bot}$\textit{PB}.

因为\textit{E},\textit{F}分别是\textit{BC},\textit{PC}的中点,所以\textit{EF}//\textit{PB},从而\textit{AD}$\mathrm{\bot}$\textit{EF}.

易证\textit{DE}//\textit{GB},且\textit{AD}$\mathrm{\bot}$\textit{GB},

所以\textit{AD}$\mathrm{\bot}$\textit{DE},因为\textit{DE}$\mathrm{\cap}$\textit{EF}=\textit{E},

所以\textit{AD}$\mathrm{\bot}$平面\textit{DEF}.

知识:平面与平面垂直的判定

难度:1

题目:已知直线\textit{l}$\mathrm{\bot}$平面\textit{$\alpha$},则经过\textit{l}且和\textit{$\alpha$}垂直的平面(  )

A.有1个   B.有2个 C.有无数个   D.不存在

解析:经过\textit{l}的平面都与\textit{$\alpha$}垂直,而经过\textit{l}的平面有无数个,故选C.

答案:C

知识:平面与平面垂直的判定

难度:1

题目:已知\textit{$\alpha$}、\textit{$\beta$}是平面,\textit{m}、\textit{n}是直线,给出下列表述:

①若\textit{m}$\mathrm{\bot}$\textit{$\alpha$},\textit{m}$\mathrm{\subset }$\textit{$\beta$},则\textit{$\alpha$}$\mathrm{\bot}$\textit{$\beta$};

②若\textit{m}$\mathrm{\subset }$\textit{$\alpha$},\textit{n}$\mathrm{\subset }$\textit{$\alpha$},\textit{m}//\textit{$\beta$},\textit{n}//\textit{$\beta$},则\textit{$\alpha$}//\textit{$\beta$};

③如果\textit{m}$\mathrm{\subset }$\textit{$\alpha$},\textit{n}$\mathrm{\nsubset}$\textit{$\alpha$},\textit{m},\textit{n}是异面直线,那么\textit{n}与\textit{$\alpha$}相交;

④若\textit{$\alpha$}$\mathrm{\cap}$\textit{$\beta$}=\textit{m},\textit{n}//\textit{m},且\textit{n}$\mathrm{\nsubset}$\textit{$\alpha$},\textit{n}$\mathrm{\nsubset}$\textit{$\beta$},则\textit{n}//\textit{$\alpha$}且\textit{n}//\textit{$\beta$}.

其中表述正确的个数是(  )

A.1   B.2   C.3   D.4

解析:①是平面与平面垂直的判定定理,所以①正确;②中,\textit{m},\textit{n}不一定是相交直线,不符合两个平面平行的判定定理,所以②不正确;③中,还可能\textit{n}//\textit{$\alpha$},所以③不正确;④中,由于\textit{n}//\textit{m},\textit{n}$\mathrm{\nsubset}$\textit{$\alpha$},\textit{m}$\mathrm{\subset }$\textit{$\alpha$},则\textit{n}//\textit{$\alpha$},同理\textit{n}//\textit{$\beta$},所以④正确.

答案:B

知识:平面与平面垂直的判定

难度:1

题目:如图,\textit{AB}是圆的直径,\textit{PA}垂直于圆所在的平面,\textit{C}是圆上一点(不同于\textit{A}、\textit{B})且\textit{PA}=\textit{AC},则二面角\textit{P}-\textit{BC}-\textit{A}的大小为(  )

\includegraphics*[width=1.29in, height=1.02in, keepaspectratio=false]{image210}

A.60$\mathrm{{}^\circ}$   B.30$\mathrm{{}^\circ}$  C.45$\mathrm{{}^\circ}$   D.15$\mathrm{{}^\circ}$

解析:由条件得:\textit{PA}$\mathrm{\bot}$\textit{BC},\textit{AC}$\mathrm{\bot}$\textit{BC}又\textit{PA}$\mathrm{\cap}$\textit{AC}=\textit{C},

$\mathrm{\therefore}$\textit{BC}$\mathrm{\bot}$平面\textit{PAC},$\mathrm{\therefore}$$\mathrm{\angle}$\textit{PCA}为二面角\textit{P}-\textit{BC}-\textit{A}的平面角.在Rt$\mathrm{\vartriangle}$\textit{PAC}中,由\textit{PA}=\textit{AC}得$\mathrm{\angle}$\textit{PCA}=45$\mathrm{{}^\circ}$,故选C.

答案:C

知识:平面与平面垂直的判定

难度:1

题目:在棱长都相等的四面体\textit{P}-\textit{ABC}中,\textit{D}、\textit{E}、\textit{F}分别是\textit{AB}、\textit{BC}、\textit{CA}的中点,则下面四个结论中不成立的是(  )

A.\textit{BC}//平面\textit{PDF}

B.\textit{DF}$\mathrm{\bot}$平面\textit{PAE}

C.平面\textit{PDF}$\mathrm{\bot}$平面\textit{ABC}

D.平面\textit{PAE}$\mathrm{\bot}$平面\textit{ABC}

解析:可画出对应图形,如图所示,则\textit{BC}//\textit{DF},又\textit{DF}$\mathrm{\subset }$平面\textit{PDF},\textit{BC}$\mathrm{\nsubset}$平面\textit{PDF},$\mathrm{\therefore}$\textit{BC}//平面\textit{PDF},故A成立;由\textit{AE}$\mathrm{\bot}$\textit{BC},\textit{PE}$\mathrm{\bot}$\textit{BC},\textit{BC}//\textit{DF},知\textit{DF}$\mathrm{\bot}$\textit{AE},\textit{DF}$\mathrm{\bot}$\textit{PE},$\mathrm{\therefore}$\textit{DF}$\mathrm{\bot}$平面\textit{PAE},故B成立;又\textit{DF}$\mathrm{\subset }$平面\textit{ABC},$\mathrm{\therefore}$平面\textit{ABC}$\mathrm{\bot}$平面\textit{PAE},故D成立.

\includegraphics*[width=1.00in, height=0.96in, keepaspectratio=false]{image211}

答案:C

知识:平面与平面垂直的判定

难度:1

题目:如图,在正方体\textit{ABCD}-\textit{A}${}_{1}$\textit{B}${}_{1}$\textit{C}${}_{1}$\textit{D}${}_{1}$中,\textit{E}为\textit{A}${}_{1}$\textit{C}${}_{1}$上的点,则下列直线中一定与\textit{CE}垂直的是(  )

\includegraphics*[width=1.12in, height=1.07in, keepaspectratio=false]{image212}

A.\textit{AC}   B.\textit{BD}   C.\textit{A}${}_{1}$\textit{D}${}_{1}$   D.\textit{A}${}_{1}$\textit{A}

解析:在正方体中,\textit{AA}${}_{1}$$\mathrm{\bot}$平面\textit{ABCD},

$\mathrm{\therefore}$\textit{AA}${}_{1}$$\mathrm{\bot}$\textit{BD}.

又正方形\textit{ABCD}中,\textit{BD}$\mathrm{\bot}$\textit{AC},且\textit{AA}${}_{1}$$\mathrm{\cap}$\textit{AC}=\textit{A},

$\mathrm{\therefore}$\textit{BD}$\mathrm{\bot}$平面\textit{AA}${}_{1}$\textit{C}${}_{1}$\textit{C}.

$\mathrm{\because}$\textit{E}$\mathrm{\in}$\textit{A}${}_{1}$\textit{C}${}_{1}$,$\mathrm{\therefore}$\textit{E}$\mathrm{\in}$平面\textit{AA}${}_{1}$\textit{C}${}_{1}$\textit{C},

$\mathrm{\therefore}$\textit{CE}$\mathrm{\subset }$平面\textit{AA}${}_{1}$\textit{C}${}_{1}$\textit{C},

$\mathrm{\therefore}$\textit{BD}$\mathrm{\bot}$\textit{CE}.

答案:B


知识:平面与平面垂直的判定

难度:1

题目:在三棱锥\textit{P}-\textit{ABC}中,已知\textit{PA}$\mathrm{\bot}$\textit{PB},\textit{PB}$\mathrm{\bot}$\textit{PC},\textit{PC}$\mathrm{\bot}$\textit{PA},如右图所示,则在三棱锥\textit{P}-\textit{ABC}的四个面中,互相垂直的面有\_\_\_\_对.

\includegraphics*[width=1.29in, height=0.90in, keepaspectratio=false]{image213}

解析:$\mathrm{\because}$\textit{PA}$\mathrm{\bot}$\textit{PB},\textit{PA}$\mathrm{\bot}$\textit{PC},\textit{PB}$\mathrm{\cap}$\textit{PC}=\textit{P},

$\mathrm{\therefore}$\textit{PA}$\mathrm{\bot}$平面\textit{PBC},

$\mathrm{\because}$\textit{PA}$\mathrm{\subset }$平面\textit{PAB},\textit{PA}$\mathrm{\subset }$平面\textit{PAC},

$\mathrm{\therefore}$平面\textit{PAB}$\mathrm{\bot}$平面\textit{PBC},平面\textit{PAC}$\mathrm{\bot}$平面\textit{PBC}.同理可证:平面\textit{PAB}$\mathrm{\bot}$平面\textit{PAC}.

答案:3

知识:平面与平面垂直的判定

难度:1

题目:已知三棱锥\textit{D}-\textit{ABC}的三个侧面与底面全等,且\textit{AB}=\textit{AC}=,\textit{BC}=2,则二面角\textit{D}-\textit{BC}-\textit{A}的大小为\_\_\_\_.

\includegraphics*[width=1.35in, height=1.21in, keepaspectratio=false]{image214}

解析:如图,由题意知\textit{AB}=\textit{AC}=\textit{BD}=\textit{CD}=$\sqrt{3}$,\textit{BC}=\textit{AD}=2.

取\textit{BC}的中点\textit{E},连接\textit{DE}、\textit{AE},则\textit{AE}$\mathrm{\bot}$\textit{BC},\textit{DE}$\mathrm{\bot}$\textit{BC},所以$\mathrm{\angle}$\textit{DEA}为所求二面角的平面角.

易得\textit{AE}=\textit{DE}=$\sqrt{2}$,又\textit{AD}=2,

所以$\mathrm{\angle}$\textit{DEA}=90$\mathrm{{}^\circ}$.

答案:90$\mathrm{{}^\circ}$

知识:平面与平面垂直的判定

难度:1

题目:如图所示,$\mathrm{\vartriangle}$\textit{ABC}为正三角形,\textit{CE}$\mathrm{\bot}$平面\textit{ABC},\textit{BD}//\textit{CE},且\textit{CE}=\textit{AC}=2\textit{BD},\textit{M}是\textit{AE}的中点.

\includegraphics*[width=0.93in, height=1.14in, keepaspectratio=false]{image215}

(1)求证:\textit{DE}=\textit{DA};

(2)求证:平面\textit{BDM}$\mathrm{\bot}$平面\textit{ECA};

解析:

答案:(1)取\textit{EC}的中点\textit{F},连接\textit{DF}.

$\mathrm{\because}$\textit{CE}$\mathrm{\bot}$平面\textit{ABC},

$\mathrm{\therefore}$\textit{CE}$\mathrm{\bot}$\textit{BC}.易知\textit{DF}//\textit{BC},$\mathrm{\therefore}$\textit{CE}$\mathrm{\bot}$\textit{DF}.

$\mathrm{\because}$\textit{BD}//\textit{CE},$\mathrm{\therefore}$\textit{BD}$\mathrm{\bot}$平面\textit{ABC}.

在Rt$\mathrm{\vartriangle}$\textit{EFD}和Rt$\mathrm{\vartriangle}$\textit{DBA}中,

\textit{EF}=$\frac{1}{2}$\textit{CE}=\textit{DB},\textit{DF}=\textit{BC}=\textit{AB},

$\mathrm{\therefore}$Rt$\mathrm{\vartriangle}$\textit{EFD}≌Rt$\mathrm{\vartriangle}$\textit{DBA}.故\textit{DE}=\textit{DA}.

(2)取\textit{AC}的中点\textit{N},连接\textit{MN}、\textit{BN},则\textit{MN}//\textit{CF}.

\includegraphics*[width=0.93in, height=1.14in, keepaspectratio=false]{image216}

$\mathrm{\because}$\textit{BD}//\textit{CF},$\mathrm{\therefore}$\textit{MN}//\textit{BD},

$\mathrm{\therefore}$\textit{N}$\mathrm{\in}$平面\textit{BDM}.

$\mathrm{\because}$\textit{EC}$\mathrm{\bot}$平面\textit{ABC},$\mathrm{\therefore}$\textit{EC}$\mathrm{\bot}$\textit{BN}.

又$\mathrm{\because}$\textit{AC}$\mathrm{\bot}$\textit{BN},\textit{EC}$\mathrm{\cap}$\textit{AC}=\textit{C},$\mathrm{\therefore}$\textit{BN}$\mathrm{\bot}$平面\textit{ECA}.

又$\mathrm{\because}$\textit{BN}$\mathrm{\subset }$平面\textit{BDM},$\mathrm{\therefore}$平面\textit{BDM}$\mathrm{\bot}$平面\textit{ECA}.

知识:平面与平面垂直的判定

难度:1

题目:如图所示,在四棱锥\textit{P}-\textit{ABCD}中,底面是边长为\textit{a}的正方形,侧棱\textit{PD}=\textit{a},\textit{PA}=\textit{PC}=$\sqrt{2}$\textit{a},

\includegraphics*[width=1.02in, height=1.06in, keepaspectratio=false]{image217}

(1)求证:\textit{PD}$\mathrm{\bot}$平面\textit{ABCD};

(2)求证:平面\textit{PAC}$\mathrm{\bot}$平面\textit{PBD};

(3)求二面角\textit{P}-\textit{AC}-\textit{D}的正切值.

解析:

答案:(1)$\mathrm{\because}$\textit{PD}=\textit{a},\textit{DC}=\textit{a},\textit{PC}=$\sqrt{2}$\textit{a},

$\mathrm{\therefore}$\textit{PC}${}^{2}$=\textit{PD}${}^{2}$+\textit{DC}${}^{2}$,

$\mathrm{\therefore}$\textit{PD}$\mathrm{\bot}$\textit{DC}.

同理可证\textit{PD}$\mathrm{\bot}$\textit{AD},又\textit{AD}$\mathrm{\cap}$\textit{DC}=\textit{D},

$\mathrm{\therefore}$\textit{PD}$\mathrm{\bot}$平面\textit{ABCD}.

(2)由(1)知\textit{PD}$\mathrm{\bot}$平面\textit{ABCD},

$\mathrm{\therefore}$\textit{PD}$\mathrm{\bot}$\textit{AC},而四边形\textit{ABCD}是正方形,

$\mathrm{\therefore}$\textit{AC}$\mathrm{\bot}$\textit{BD},又\textit{BD}$\mathrm{\cap}$\textit{PD}=\textit{D},

$\mathrm{\therefore}$\textit{AC}$\mathrm{\bot}$平面\textit{PDB}.

同时,\textit{AC}$\mathrm{\subset }$平面\textit{PAC},

$\mathrm{\therefore}$平面\textit{PAC}$\mathrm{\bot}$平面\textit{PBD}.

(3)设\textit{AC}$\mathrm{\cap}$\textit{BD}=\textit{O},连接\textit{PO}.

\includegraphics*[width=1.04in, height=1.04in, keepaspectratio=false]{image218}

由\textit{PA}=\textit{PC},知\textit{PO}$\mathrm{\bot}$\textit{AC}.

又由\textit{DO}$\mathrm{\bot}$\textit{AC},故$\mathrm{\angle}$\textit{POD}为二面角\textit{P}-\textit{AC}-\textit{D}的平面角.易知\textit{OD}=$\frac{\sqrt{2}}{2}$\textit{a}.

在Rt$\mathrm{\vartriangle}$\textit{PDO}中,tan$\mathrm{\angle}$\textit{POD}=$\frac{PD}{OD}$=$\frac{a}{\frac{\sqrt{2}}{2}a}$=$\sqrt{2}$.

知识:平面与平面垂直的判定

难度:2

题目:设直线\textit{m}与平面\textit{$\alpha$}相交但不垂直,则下列说法中,正确的是(  )

A.在平面\textit{$\alpha$}内有且只有一条直线与直线\textit{m}垂直

B.过直线\textit{m}有且只有一个平面与平面\textit{$\alpha$}垂直

C.与直线\textit{m}垂直的直线不可能与平面\textit{$\alpha$}平行

D.与直线\textit{m}平行的平面不可能与平面\textit{$\alpha$}垂直

解析:由题意,\textit{m}与\textit{$\alpha$}斜交,令其在\textit{$\alpha$}内的射影为\textit{m}$'$,则在\textit{$\alpha$}内可作无数条与\textit{m}$'$垂直的直线,它们都与\textit{m}垂直,A错;如图示(1),在\textit{$\alpha$}外,可作与\textit{$\alpha$}内直线\textit{l}平行的直线,C错;如图(2),\textit{m}$\mathrm{\subset }$\textit{$\beta$},\textit{$\alpha$}$\mathrm{\bot}$\textit{$\beta$}.可作\textit{$\beta$}的平行平面\textit{$\gamma$},则\textit{m}//\textit{$\gamma$}且\textit{$\gamma$}$\mathrm{\bot}$\textit{$\alpha$},D错.

\includegraphics*[width=1.92in, height=0.87in, keepaspectratio=false]{image219}

答案:B

知识:平面与平面垂直的判定

难度:2

题目:把正方形\textit{ABCD}沿对角线\textit{BD}折成直二面角,则$\mathrm{\vartriangle}$\textit{ABC}是(  )

A.正三角形   B.直角三角形 C.锐角三角形   D.钝角三角形

解析:设正方形边长为1,\textit{AC}与\textit{BD}相交于\textit{O},则折成直二面角后,\textit{AB}=\textit{BC}=1,\textit{AC}=$\sqrt{CO^2+AO^2}$=$\sqrt{\frac{\sqrt{2}}{2}^2+\frac{\sqrt{2}}{2}^2}$=1,则$\mathrm{\vartriangle}$\textit{ABC}是正三角形.

\includegraphics*[width=2.52in, height=1.05in, keepaspectratio=false]{image220}

答案:A



知识:平面与平面垂直的判定

难度:2

题目:在二面角\textit{$\alpha$}-\textit{l}-\textit{$\beta$}中,\textit{A}$\mathrm{\in}$\textit{$\alpha$},\textit{AB}$\mathrm{\bot}$平面\textit{$\beta$}于\textit{B},\textit{BC}$\mathrm{\bot}$平面\textit{$\alpha$}于\textit{C},若\textit{AB}=6,\textit{BC}=3,则二面角\textit{$\alpha$}-\textit{l}-\textit{$\beta$}的平面角的大小为(  )

A.30$\mathrm{{}^\circ}$   B.60$\mathrm{{}^\circ}$ C.30$\mathrm{{}^\circ}$或150$\mathrm{{}^\circ}$   D.60$\mathrm{{}^\circ}$或120$\mathrm{{}^\circ}$

解析:如图,$\mathrm{\because}$\textit{AB}$\mathrm{\bot}$\textit{$\beta$},$\mathrm{\therefore}$\textit{AB}$\mathrm{\bot}$\textit{l},$\mathrm{\because}$\textit{BC}$\mathrm{\bot}$\textit{$\alpha$},

\includegraphics*[width=1.62in, height=1.54in, keepaspectratio=false]{image221}

$\mathrm{\therefore}$\textit{BC}$\mathrm{\bot}$\textit{l},$\mathrm{\therefore}$\textit{l}$\mathrm{\bot}$平面\textit{ABC},

设平面\textit{ABC}$\mathrm{\cap}$\textit{l}=\textit{D},

则$\mathrm{\angle}$\textit{ADB}为二面角\textit{$\alpha$}-\textit{l}-\textit{$\beta$}的平面角或补角,

$\mathrm{\because}$\textit{AB}=6,\textit{BC}=3,

$\mathrm{\therefore}$$\mathrm{\angle}$\textit{BAC}=30$\mathrm{{}^\circ}$,

$\mathrm{\therefore}$$\mathrm{\angle}$\textit{ADB}=60$\mathrm{{}^\circ}$,

$\mathrm{\therefore}$二面角大小为60$\mathrm{{}^\circ}$或120$\mathrm{{}^\circ}$.

答案:D

知识:平面与平面垂直的判定

难度:2

题目:如图,在正方形\textit{ABCD}中,\textit{E}、\textit{F}分别是\textit{BC}、\textit{CD}的中点,\textit{G}是\textit{EF}的中点,现在沿\textit{AE}、\textit{AF}及\textit{EF}把这个正方形折成一个空间图形,使\textit{B}、\textit{C}、\textit{D}三点重合,重合后的点记为\textit{H},那么,在这个空间图形中必有(  )

\includegraphics*[width=1.87in, height=1.08in, keepaspectratio=false]{image222}

A.\textit{AH}$\mathrm{\bot}$$\mathrm{\vartriangle}$\textit{EFH}所在平面  B.\textit{AG}$\mathrm{\bot}$$\mathrm{\vartriangle}$\textit{EFH}所在平面

C.\textit{HF}$\mathrm{\bot}$$\mathrm{\vartriangle}$\textit{AEF}所在平面  D.\textit{HG}$\mathrm{\bot}$$\mathrm{\vartriangle}$\textit{AEF}所在平面

解析: 由平面图得:\textit{AH}$\mathrm{\bot}$\textit{HE},\textit{AH}$\mathrm{\bot}$\textit{HF},$\mathrm{\therefore}$\textit{AH}$\mathrm{\bot}$平面\textit{HEF},$\mathrm{\therefore}$选A.

答案:A

知识:平面与平面垂直的判定

难度:2

题目:在三棱锥\textit{P}-\textit{ABC}中,\textit{PA}=\textit{PB}=\textit{AC}=\textit{BC}=2,\textit{PC}=1,\textit{AB}=2,则二面角\textit{P}-\textit{AB}-\textit{C}的大小为\_\_\_\_.

\includegraphics*[width=1.00in, height=1.09in, keepaspectratio=false]{image223}

解析: 取\textit{AB}中点\textit{M},连接\textit{PM},\textit{MC},则\textit{PM}$\mathrm{\bot}$\textit{AB},\textit{CM}$\mathrm{\bot}$\textit{AB},$\mathrm{\therefore}$$\mathrm{\angle}$\textit{PMC}就是二面角\textit{P}-\textit{AB}-\textit{C}的平面角.在$\mathrm{\vartriangle}$\textit{PAB}中,\textit{PM}=$\sqrt{2^2-\sqrt{3}^2}$=1,

\includegraphics*[width=1.11in, height=1.21in, keepaspectratio=false]{image224}

同理\textit{MC}=1,则$\mathrm{\vartriangle}$\textit{PMC}是等边三角形,$\mathrm{\therefore}$$\mathrm{\angle}$\textit{PMC}=60$\mathrm{{}^\circ}$.

答案:60$\mathrm{{}^\circ}$

知识:平面与平面垂直的判定

难度:2

题目:如图所示,在四棱锥\textit{P}-\textit{ABCD}中,\textit{PA}$\mathrm{\bot}$底面\textit{ABCD}.底面各边都相等,\textit{M}是\textit{PC}上的一动点,当点\textit{M}满足\_\_\_\_时,平面\textit{MBD}$\mathrm{\bot}$平面\textit{PCD}.(注:只要填写一个你认为正确的即可)

\includegraphics*[width=1.02in, height=0.86in, keepaspectratio=false]{image225}

解析:$\mathrm{\because}$四边形\textit{ABCD}的边长相等,

$\mathrm{\therefore}$四边形为菱形.$\mathrm{\because}$\textit{AC}$\mathrm{\bot}$\textit{BD},

又$\mathrm{\because}$\textit{PA}$\mathrm{\bot}$平面\textit{ABCD},$\mathrm{\therefore}$\textit{PA}$\mathrm{\bot}$\textit{BD},

$\mathrm{\therefore}$\textit{BD}$\mathrm{\bot}$平面\textit{PAC},$\mathrm{\therefore}$\textit{BD}$\mathrm{\bot}$\textit{PC}.

若\textit{PC}$\mathrm{\bot}$平面\textit{BMD},则\textit{PC}垂直于平面\textit{BMD}中两条相交直线.

$\mathrm{\therefore}$当\textit{BM}$\mathrm{\bot}$\textit{PC}时,\textit{PC}$\mathrm{\bot}$平面\textit{BDM}.

$\mathrm{\therefore}$平面\textit{PCD}$\mathrm{\bot}$平面\textit{BDM}.

答案:\textit{BM}$\mathrm{\bot}$\textit{PC}(其他合理即可)

知识:平面与平面垂直的判定

难度:3

题目:(2015·湖南)如下图,直三棱柱\textit{ABC}-\textit{A}${}_{1}$\textit{B}${}_{1}$\textit{C}${}_{1}$的底面是边长为2的正三角形,\textit{E}、\textit{F}分别是\textit{BC}、\textit{CC}${}_{1}$的中点.

(1)证明:平面\textit{AEF}$\mathrm{\bot}$平面\textit{B}${}_{1}$\textit{BCC}${}_{1}$;

(2)若直线\textit{A}${}_{1}$\textit{C}与平面\textit{A}${}_{1}$\textit{ABB}${}_{1}$所成的角为45$\mathrm{{}^\circ}$,求三棱锥\textit{F}-\textit{AEC}的体积.

\includegraphics*[width=1.65in, height=1.62in, keepaspectratio=false]{image226}

解析:

答案:(1)如图,因为三棱柱\textit{ABC}-\textit{A}${}_{1}$\textit{B}${}_{1}$\textit{C}${}_{1}$是直三棱柱,

所以\textit{AE}$\mathrm{\bot}$\textit{BB}${}_{1}$,又\textit{E}是正三角形\textit{ABC}的边\textit{BC}的中点,

所以\textit{AE}$\mathrm{\bot}$\textit{BC},因此\textit{AE}$\mathrm{\bot}$平面\textit{B}${}_{1}$\textit{BCC}${}_{1}$,而\textit{AE}$\mathrm{\subset }$平面\textit{AEF},

所以平面\textit{AEF}$\mathrm{\bot}$平面\textit{B}${}_{1}$\textit{BCC}${}_{1}$.

\includegraphics*[width=1.68in, height=1.67in, keepaspectratio=false]{image227}

(2)设\textit{AB}的中点为\textit{D},连接\textit{A}${}_{1}$\textit{D},\textit{CD},因为$\mathrm{\vartriangle}$\textit{ABC}是正三角形,所以\textit{CD}$\mathrm{\bot}$\textit{AB},又三棱柱\textit{ABC}-\textit{A}${}_{1}$\textit{B}${}_{1}$\textit{C}${}_{1}$是直三棱柱,所以\textit{CD}$\mathrm{\bot}$\textit{AA}${}_{1}$,因此\textit{CD}$\mathrm{\bot}$平面\textit{A}${}_{1}$\textit{AB}${}_{1}$\textit{B},于是$\mathrm{\angle}$\textit{CA}${}_{1}$\textit{D}为直线\textit{A}${}_{1}$\textit{C}与平面\textit{A}${}_{1}$\textit{ABB}${}_{1}$所成的角,由题设知$\mathrm{\angle}$\textit{CA}${}_{1}$\textit{D}=45$\mathrm{{}^\circ}$,

所以\textit{A}${}_{1}$\textit{D}=\textit{CD}=$\frac{\sqrt{3}}{2}$\textit{AB}=$\sqrt{3}$,

在Rt$\mathrm{\vartriangle}$\textit{AA}${}_{1}$\textit{D}中,\textit{AA}${}_{1}$=$\sqrt{A_1D^2+AD^2}$=$\sqrt{3-1}$=$\sqrt{2}$,

所以\textit{FC}=$\frac{1}{2}$\textit{AA}${}_{1}$=$\frac{\sqrt{2}}{2}$,

故三棱锥\textit{F}-\textit{AEC}的体积\textit{V}=$\frac{1}{3}$\textit{S${}_{AEC}$}$\mathrm{\times}$\textit{FC}=$\frac{1}{3}$$\mathrm{\times}$$\frac{\sqrt{3}}{2}$$\mathrm{\times}$$\frac{\sqrt{2}}{2}$=$\frac{\sqrt{6}}{12}$.

知识:平面与平面垂直的判定

难度:3

题目:如图所示,四棱锥\textit{P}-\textit{ABCD}的底面\textit{ABCD}是边长为1的菱

形,$\mathrm{\angle}$\textit{BCD}=60$\mathrm{{}^\circ}$,\textit{E}是\textit{CD}的中点,\textit{PA}$\mathrm{\bot}$底面\textit{ABCD},\textit{PA}=$\sqrt{3}$.

\includegraphics*[width=1.29in, height=1.25in, keepaspectratio=false]{image228}

(1)证明:平面\textit{PBE}$\mathrm{\bot}$平面\textit{PAB};

(2)求二面角\textit{A}-\textit{BE}-\textit{P}的大小.

解析:

答案: (1)如图所示,连接\textit{BD},由\textit{ABCD}是菱形且$\mathrm{\angle}$\textit{BCD}=60$\mathrm{{}^\circ}$知,$\mathrm{\vartriangle}$\textit{BCD}是等边三角形.

\includegraphics*[width=1.29in, height=1.25in, keepaspectratio=false]{image229}

因为\textit{E}是\textit{CD}的中点,所以\textit{BE}$\mathrm{\bot}$\textit{CD},

又\textit{AB}//\textit{CD},所以\textit{BE}$\mathrm{\bot}$\textit{AB},

又因为\textit{PA}$\mathrm{\bot}$平面\textit{ABCD},\textit{BE}$\mathrm{\subset }$平面\textit{ABCD},所以\textit{PA}$\mathrm{\bot}$\textit{BE}.

而\textit{PA}$\mathrm{\cap}$\textit{AB}=\textit{A},因此\textit{BE}$\mathrm{\bot}$平面\textit{PAB}.

又\textit{BE}$\mathrm{\subset }$平面\textit{PBE},所以平面\textit{PBE}$\mathrm{\bot}$平面\textit{PAB}.

(2)由(1)知,\textit{BE}$\mathrm{\bot}$平面\textit{PAB},\textit{PB}$\mathrm{\subset }$平面\textit{PAB},所以\textit{PB}$\mathrm{\bot}$\textit{BE}.又\textit{AB}$\mathrm{\bot}$\textit{BE},所以$\mathrm{\angle}$\textit{PBA}是二面角\textit{A}-\textit{BE}-\textit{P}的平面角.

在Rt$\mathrm{\vartriangle}$\textit{PAB}中,tan$\mathrm{\angle}$\textit{PBA}=$\frac{PA}{AB}$=$\sqrt{3}$,$\mathrm{\angle}$\textit{PBA}=60$\mathrm{{}^\circ}$.

故二面角\textit{A}-\textit{BE}-\textit{P}的大小是60$\mathrm{{}^\circ}$.

知识:直线与平面垂直的性质

难度:1

题目:平面\textit{$\alpha$}//平面\textit{$\beta$},直线\textit{a}//\textit{$\alpha$},直线\textit{b}$\mathrm{\bot}$\textit{$\beta$},那么直线\textit{a}与直线\textit{b}的位置关系一定是

(  )

A.平行   B.异面   C.垂直   D.不相交

解析: $\mathrm{\because}$\textit{$\alpha$}//\textit{$\beta$},\textit{b}$\mathrm{\bot}$\textit{$\beta$},$\mathrm{\therefore}$\textit{b}$\mathrm{\bot}$\textit{$\alpha$}.

又$\mathrm{\because}$\textit{a}//\textit{$\alpha$},$\mathrm{\therefore}$\textit{b}$\mathrm{\bot}$\textit{a}.

答案:C

知识:直线与平面垂直的性质

难度:1

题目:设\textit{m}、\textit{n}是两条不同的直线,\textit{$\alpha$}、\textit{$\beta$}是两个不同的平面.(  )

A.若\textit{m}//\textit{$\alpha$},\textit{n}//\textit{$\alpha$},则\textit{m}//\textit{n}

B.若\textit{m}//\textit{$\alpha$},\textit{m}//\textit{$\beta$},则\textit{$\alpha$}//\textit{$\beta$}

C.若\textit{m}//\textit{n},\textit{m}$\mathrm{\bot}$\textit{$\alpha$},则\textit{n}$\mathrm{\bot}$\textit{$\alpha$}

D.若\textit{m}//\textit{$\alpha$},\textit{$\alpha$}$\mathrm{\bot}$\textit{$\beta$},则\textit{m}$\mathrm{\bot}$\textit{$\beta$}

解析: $\mathrm{\because}$\textit{m}//\textit{n},\textit{m}$\mathrm{\bot}$\textit{$\alpha$},则\textit{n}$\mathrm{\bot}$\textit{$\alpha$},故选C.

答案:C

知识:直线与平面垂直的性质

难度:1

题目:如图,已知$\mathrm{\vartriangle}$\textit{ABC}为直角三角形,其中$\mathrm{\angle}$\textit{ACB}=90$\mathrm{{}^\circ}$,\textit{M}为\textit{AB}的中点,\textit{PM}垂直于$\mathrm{\vartriangle}$\textit{ABC}所在平面,那么(  )

\includegraphics*[width=1.08in, height=1.11in, keepaspectratio=false]{image231}

A.\textit{PA}=\textit{PB}>\textit{PC  }B.\textit{PA}=\textit{PB}<\textit{PC}

C.\textit{PA}=\textit{PB}=\textit{PC  }D.\textit{PA}$\mathrm{\neq}$\textit{PA}$\mathrm{\neq}$\textit{PC}

解析: $\mathrm{\because}$\textit{PM}$\mathrm{\bot}$平面\textit{ABC},\textit{MC}$\mathrm{\subset }$平面\textit{ABC},

$\mathrm{\therefore}$\textit{PM}$\mathrm{\bot}$\textit{MC},\textit{PM}$\mathrm{\bot}$\textit{AB}.

又$\mathrm{\because}$\textit{M}为\textit{AB}中点,$\mathrm{\angle}$\textit{ACB}=90$\mathrm{{}^\circ}$,

$\mathrm{\therefore}$\textit{MA}=\textit{MB}=\textit{MC}.$\mathrm{\therefore}$\textit{PA}=\textit{PA}=\textit{PC}.

答案:C

知识:直线与平面垂直的性质

难度:1

题目:如图,设平面\textit{$\alpha$}$\mathrm{\cap}$平面\textit{$\beta$}=\textit{PQ},\textit{EG}$\mathrm{\bot}$平面\textit{$\alpha$},\textit{FH}$\mathrm{\bot}$平面\textit{$\alpha$},垂足分别为\textit{G}、\textit{H}.为使\textit{PQ}$\mathrm{\bot}$\textit{GH},则需增加的一个条件是(  )

\includegraphics*[width=1.47in, height=1.27in, keepaspectratio=false]{image232}

A.\textit{EF}$\mathrm{\bot}$平面\textit{$\alpha$}    B.\textit{EF}$\mathrm{\bot}$平面\textit{$\beta$}

C.\textit{PQ}$\mathrm{\bot}$\textit{GE}    D.\textit{PQ}$\mathrm{\bot}$\textit{FH}

解析: 因为\textit{EG}$\mathrm{\bot}$平面\textit{$\alpha$},\textit{PQ}$\mathrm{\subset }$平面\textit{$\alpha$},所以\textit{EG}$\mathrm{\bot}$\textit{PQ}.若\textit{EF}$\mathrm{\bot}$平面\textit{$\beta$},则由\textit{PQ}$\mathrm{\subset }$平面\textit{$\beta$},得\textit{EF}$\mathrm{\bot}$\textit{PQ}.又\textit{EG}与\textit{EF}为相交直线,所以\textit{PQ}$\mathrm{\bot}$平面\textit{EFHG},所以\textit{PQ}$\mathrm{\bot}$\textit{GH},故选B.

答案:B

知识:直线与平面垂直的性质

难度:1

题目:下列命题正确的是(  )

①$\left. \begin{array}{r}
a// b\\
a\bot \sigma
\end{array} \right\}\mathrm{\Rightarrow }$\textit{b}$\mathrm{\bot}$\textit{$\sigma$};②$\left. \begin{array}{r}
a\bot \sigma\\
b\bot \sigma
\end{array} \right\}\mathrm{\Rightarrow }$\textit{a}//\textit{b};③$\left. \begin{array}{r}
a\bot \sigma\\
a\bot b
\end{array} \right\}\mathrm{\Rightarrow }$\textit{b}//\textit{$\sigma$};④$\left. \begin{array}{r}
a// \sigma\\
a\bot b
\end{array} \right\}\mathrm{\Rightarrow }$\textit{b}$\mathrm{\bot}$\textit{$\sigma$}.

A.①②   B.①②③   C.②③④   D.①②④

解析:由性质定理可得(1)(2)正确.

答案:A

知识:直线与平面垂直的性质

难度:1

题目:如图,正方体\textit{ABCD}-\textit{A}${}_{1}$\textit{B}${}_{1}$\textit{C}${}_{1}$\textit{D}${}_{1}$中,点\textit{P}在侧面\textit{BCC}${}_{1}$\textit{B}${}_{1}$及其边界上运动,并且总是保持\textit{AP}$\mathrm{\bot}$\textit{BD}${}_{1}$,则动点\textit{P}的轨迹是(  )

\includegraphics*[width=1.05in, height=0.98in, keepaspectratio=false]{image233}

A.线段\textit{B}${}_{1}$\textit{C}

B.线段\textit{BC}${}_{1}$

C.\textit{BB}${}_{1}$中点与\textit{CC}${}_{1}$中点连成的线段

D.\textit{BC}中点与\textit{B}${}_{1}$\textit{C}${}_{1}$中点连成的线段

解析: $\mathrm{\because}$\textit{DD}${}_{1}$$\mathrm{\bot}$平面\textit{ABCD},

$\mathrm{\therefore}$\textit{D}${}_{1}$\textit{D}$\mathrm{\bot}$\textit{AC},

又\textit{AC}$\mathrm{\bot}$\textit{BD},$\mathrm{\therefore}$\textit{AC}$\mathrm{\bot}$平面\textit{BDD}${}_{1}$,

$\mathrm{\therefore}$\textit{AC}$\mathrm{\bot}$\textit{BD}${}_{1}$.同理\textit{BD}${}_{1}$$\mathrm{\bot}$\textit{B}${}_{1}$\textit{C}.

\includegraphics*[width=1.25in, height=1.10in, keepaspectratio=false]{image234}

又$\mathrm{\because}$\textit{B}${}_{1}$\textit{C}$\mathrm{\cap}$\textit{AC}=\textit{C},

$\mathrm{\therefore}$\textit{BD}${}_{1}$$\mathrm{\bot}$平面\textit{AB}${}_{1}$\textit{C}.

而\textit{AP}$\mathrm{\bot}$\textit{BD}${}_{1}$,$\mathrm{\therefore}$\textit{AP}$\mathrm{\subset }$平面\textit{AB}${}_{1}$\textit{C}.

又\textit{P}$\mathrm{\in}$平面\textit{BB}${}_{1}$\textit{C}${}_{1}$\textit{C},$\mathrm{\therefore}$\textit{P}点轨迹为平面\textit{AB}${}_{1}$\textit{C}与平面\textit{BB}${}_{1}$\textit{C}${}_{1}$\textit{C}的交线\textit{B}${}_{1}$\textit{C}.故选A.

答案:A

知识:直线与平面垂直的性质

难度:1

题目:线段\textit{AB}在平面\textit{$\alpha$}的同侧,\textit{A}、\textit{B}到\textit{$\alpha$}的距离分别为3和5,则\textit{AB}的中点到\textit{$\alpha$}的距离为\_\_\_\_.

解析: 如图,设\textit{AB}的中点为\textit{M},分别过\textit{A}、\textit{M}、\textit{B}向\textit{$\alpha$}作垂线,垂足分别为\textit{A}${}_{1}$、\textit{M}${}_{1}$、\textit{B}${}_{1}$,则由线面垂直的性质可知,\textit{AA}${}_{1}$//\textit{MM}${}_{1}$//\textit{BB}${}_{1}$,

\includegraphics*[width=1.17in, height=1.07in, keepaspectratio=false]{image235}

四边形\textit{AA}${}_{1}$\textit{B}${}_{1}$\textit{B}为直角梯形,

\textit{AA}${}_{1}$=3,\textit{BB}${}_{1}$=5,\textit{MM}${}_{1}$为其中位线,

$\mathrm{\therefore}$\textit{MM}${}_{1}$=4.

答案:4

知识:直线与平面垂直的性质

难度:1

题目:正三棱锥的底面边长为2,侧面均为直角三角形,则此三棱锥的体积是\_\_\_\_.

解析:如图,

\includegraphics*[width=1.60in, height=1.33in, keepaspectratio=false]{image236}

由已知得\textit{PA}$\mathrm{\bot}$\textit{PB},\textit{PA}$\mathrm{\bot}$\textit{PC},\textit{PB}$\mathrm{\cap}$\textit{PC}=\textit{P},

$\mathrm{\therefore}$\textit{PA}$\mathrm{\bot}$平面\textit{PBC}.

又\textit{PB}$\mathrm{\bot}$\textit{PC},\textit{PB}=\textit{PC},\textit{BC}=2,

$\mathrm{\therefore}$\textit{PB}=\textit{PC}=$\sqrt{2}$.

$\mathrm{\therefore}$\textit{V${}_{P}$}${}_{\textrm{-}}$\textit{${}_{ABC}$}=\textit{V${}_{A}$}${}_{\textrm{-}}$\textit{${}_{PBC}$}=$\frac{1}{3}$\textit{PA}·\textit{S}${}_{\vartriangle }$\textit{${}_{PBC}$}=$\frac{1}{3}\mathrm{\times}$$\sqrt{2}\mathrm{\times}$$\frac{1}{2}\mathrm{\times}$$\sqrt{2}\mathrm{\times}\sqrt{2}$=$\frac{\sqrt{2}}{3}$.

答案:$\frac{\sqrt{2}}{3}$


知识:直线与平面垂直的性质

难度:1

题目:如图,四棱柱\textit{ABCD}-\textit{A}${}_{1}$\textit{B}${}_{1}$\textit{C}${}_{1}$\textit{D}${}_{1}$的底面\textit{ABCD}是正方形,\textit{O}为底面中心,\textit{A}${}_{1}$\textit{O}$\mathrm{\bot}$平面\textit{ABCD},\textit{AB}=\textit{AA}${}_{1}$=$\sqrt{2}$.

\includegraphics*[width=1.31in, height=0.81in, keepaspectratio=false]{image237}

证明:\textit{A}${}_{1}$\textit{C}$\mathrm{\bot}$平面\textit{BB}${}_{1}$\textit{D}${}_{1}$\textit{D}.

解析:

答案: $\mathrm{\because}$\textit{A}${}_{1}$\textit{O}$\mathrm{\bot}$平面\textit{ABCD},$\mathrm{\therefore}$\textit{A}${}_{1}$\textit{O}$\mathrm{\bot}$\textit{BD}.

又底面\textit{ABCD}是正方形,

$\mathrm{\therefore}$\textit{BD}$\mathrm{\bot}$\textit{AC},$\mathrm{\therefore}$\textit{BD}$\mathrm{\bot}$平面\textit{A}${}_{1}$\textit{OC},$\mathrm{\therefore}$\textit{BD}$\mathrm{\bot}$\textit{A}${}_{1}$\textit{C}.

又\textit{OA}${}_{1}$是\textit{AC}的中垂线,

$\mathrm{\therefore}$\textit{A}${}_{1}$\textit{A}=\textit{A}${}_{1}$\textit{C}=$\sqrt{2}$,且\textit{AC}=2,$\mathrm{\therefore}$\textit{AC}${}^{2}$=\textit{AA}+\textit{A}${}_{1}$\textit{C}${}^{2}$,

$\mathrm{\therefore}$$\mathrm{\vartriangle}$\textit{AA}${}_{1}$\textit{C}是直角三角形,$\mathrm{\therefore}$\textit{AA}${}_{1}$$\mathrm{\bot}$\textit{A}${}_{1}$\textit{C}.

又\textit{BB}${}_{1}$//\textit{AA}${}_{1}$,$\mathrm{\therefore}$\textit{A}${}_{1}$\textit{C}$\mathrm{\bot}$\textit{BB}${}_{1}$,$\mathrm{\therefore}$\textit{A}${}_{1}$\textit{C}$\mathrm{\bot}$平面\textit{BB}${}_{1}$\textit{D}${}_{1}$\textit{D}.

10.如右图所示,在直四棱柱\textit{ABCD}-\textit{A}${}_{1}$\textit{B}${}_{1}$\textit{C}${}_{1}$\textit{D}${}_{1}$中,已知\textit{DC}=\textit{DD}${}_{1}$=2\textit{AD}=2\textit{AB},\textit{AD}$\mathrm{\bot}$\textit{DC},\textit{AB}//\textit{DC}.

\includegraphics*[width=1.15in, height=1.15in, keepaspectratio=false]{image238}

(1)求证:\textit{D}${}_{1}$\textit{C}$\mathrm{\bot}$\textit{AC}${}_{1}$;

(2)设\textit{E}是\textit{DC}上一点,试确定\textit{E}的位置,使\textit{D}${}_{1}$\textit{E}//平面\textit{A}${}_{1}$\textit{BD},并说明理由.

解析:

答案:
(1)连接\textit{C}${}_{1}$\textit{D}.

$\mathrm{\because}$\textit{DC}=\textit{DD}${}_{1}$,$\mathrm{\therefore}$四边形\textit{DCC}${}_{1}$\textit{D}${}_{1}$是正方形,$\mathrm{\therefore}$\textit{DC}${}_{1}$$\mathrm{\bot}$\textit{D}${}_{1}$\textit{C}.

$\mathrm{\because}$\textit{AD}$\mathrm{\bot}$\textit{DC},\textit{AD}$\mathrm{\bot}$\textit{DD}${}_{1}$,\textit{DC}$\mathrm{\cap}$\textit{DD}${}_{1}$=\textit{D},

$\mathrm{\therefore}$\textit{AD}$\mathrm{\bot}$平面\textit{DCC}${}_{1}$\textit{D}${}_{1}$,\textit{D}${}_{1}$\textit{C}$\mathrm{\subset }$平面\textit{DCC}${}_{1}$\textit{D}${}_{1}$,$\mathrm{\therefore}$\textit{AD}$\mathrm{\bot}$\textit{D}${}_{1}$\textit{C}.又\textit{AD}$\mathrm{\cap}$\textit{DC}${}_{1}$=\textit{D},$\mathrm{\therefore}$\textit{D}${}_{1}$\textit{C}$\mathrm{\bot}$平面\textit{ADC}${}_{1}$.

又\textit{AC}${}_{1}$$\mathrm{\subset }$平面\textit{ADC}${}_{1}$,$\mathrm{\therefore}$\textit{D}${}_{1}$\textit{C}$\mathrm{\bot}$\textit{AC}${}_{1}$.

(2)如图,连接\textit{AD}${}_{1}$、\textit{AE}、\textit{D}${}_{1}$\textit{E},

\includegraphics*[width=1.29in, height=1.21in, keepaspectratio=false]{image239}

设\textit{AD}${}_{1}$$\mathrm{\cap}$\textit{A}${}_{1}$\textit{D}=\textit{M},\textit{BD}$\mathrm{\cap}$\textit{AE}=\textit{N},连接\textit{MN}.

$\mathrm{\because}$平面\textit{AD}${}_{1}$\textit{E}$\mathrm{\cap}$平面\textit{A}${}_{1}$\textit{BD}=\textit{MN},

要使\textit{D}${}_{1}$\textit{E}//平面\textit{A}${}_{1}$\textit{BD},

须使\textit{MN}//\textit{D}${}_{1}$\textit{E},又\textit{M}是\textit{AD}${}_{1}$的中点,

$\mathrm{\therefore}$\textit{N}是\textit{AE}的中点.

又易知$\mathrm{\vartriangle}$\textit{ABN}≌$\mathrm{\vartriangle}$\textit{EDN},$\mathrm{\therefore}$\textit{AB}=\textit{DE}.

即\textit{E}是\textit{DC}的中点.

综上所述,当\textit{E}是\textit{DC}的中点时,可使\textit{D}${}_{1}$\textit{E}//平面\textit{A}${}_{1}$\textit{BD}.

知识:直线与平面垂直的性质

难度:2

题目:已知平面\textit{$\alpha$}与平面\textit{$\beta$}相交,直线\textit{m}$\mathrm{\bot}$\textit{$\alpha$},则(  )

A.\textit{$\beta$}内必存在直线与\textit{m}平行,且存在直线与\textit{m}垂直

B.\textit{$\beta$}内不一定存在直线与\textit{m}平行,不一定存在直线与\textit{m}垂直

C.\textit{$\beta$}内不一定存在直线与\textit{m}平行,必存在直线与\textit{m}垂直

D.\textit{$\beta$}内必存在直线与\textit{m}平行,不一定存在直线与\textit{m}垂直

解析:

答案:C

知识:直线与平面垂直的性质

难度:2

题目:如图,正方体\textit{AC}${}_{1}$的棱长为1,过点\textit{A}作平面\textit{A}${}_{1}$\textit{BD}的垂线,垂足为\textit{H},则以下命题中,错误的命题是(  )

\includegraphics*[width=1.02in, height=1.00in, keepaspectratio=false]{image240}

A.点\textit{H}是$\mathrm{\vartriangle}$\textit{A}${}_{1}$\textit{BD}的垂心

B.\textit{AH}垂直于平面\textit{CB}${}_{1}$\textit{D}${}_{1}$

C.\textit{AH}的延长线经过点\textit{C}${}_{1}$

D.直线\textit{AH}和\textit{BB}${}_{1}$所成角为45$\mathrm{{}^\circ}$

解析: A中,$\mathrm{\vartriangle}$\textit{A}${}_{1}$\textit{BD}为等边三角形,$\mathrm{\therefore}$四心合一,$\mathrm{\because}$\textit{AB}=\textit{AA}${}_{1}$=\textit{AD},$\mathrm{\therefore}$\textit{H}到$\mathrm{\vartriangle}$\textit{A}${}_{1}$\textit{BD}各顶点的距离相等,$\mathrm{\therefore}$A正确;

易知\textit{CD}${}_{1}$//\textit{BA}${}_{1}$,\textit{CB}${}_{1}$//\textit{DA}${}_{1}$,又\textit{CD}${}_{1}$$\mathrm{\cap}$\textit{CB}${}_{1}$=\textit{C},\textit{BA}${}_{1}$$\mathrm{\cap}$\textit{DA}${}_{1}$=\textit{A}${}_{1}$,$\mathrm{\therefore}$平面\textit{CB}${}_{1}$\textit{D}${}_{1}$//平面\textit{A}${}_{1}$\textit{BD},$\mathrm{\therefore}$\textit{AH}$\mathrm{\bot}$平面\textit{CB}${}_{1}$\textit{D}${}_{1}$,$\mathrm{\therefore}$B正确;

连接\textit{AC}${}_{1}$,则\textit{AC}${}_{1}$$\mathrm{\bot}$\textit{B}${}_{1}$\textit{D}${}_{1}$,$\mathrm{\because}$\textit{B}${}_{1}$\textit{D}${}_{1}$//\textit{BD},

$\mathrm{\therefore}$\textit{AC}${}_{1}$$\mathrm{\bot}$\textit{BD},同理,\textit{AC}${}_{1}$$\mathrm{\bot}$\textit{BA}${}_{1}$,又\textit{BA}${}_{1}$$\mathrm{\cap}$\textit{BD}=\textit{B},$\mathrm{\therefore}$\textit{AC}${}_{1}$$\mathrm{\bot}$平面\textit{A}${}_{1}$\textit{BD},

$\mathrm{\therefore}$\textit{A}、\textit{H}、\textit{C}${}_{1}$三点共线,$\mathrm{\therefore}$C正确,利用排除法选D.

答案:D

知识:直线与平面垂直的性质

难度:2

题目:如图所示,\textit{PA}垂直于$\mathrm{\odot}$\textit{O}所在平面,\textit{AB}是$\mathrm{\odot}$\textit{O}的直径,\textit{C}是$\mathrm{\odot}$\textit{O}上的一点,\textit{E}、\textit{F}分别是点\textit{A}在\textit{PB}、\textit{PC}上的射影,给出下列结论:①\textit{AF}$\mathrm{\bot}$\textit{PB};②\textit{EF}$\mathrm{\bot}$\textit{PB};③\textit{AF}$\mathrm{\bot}$\textit{BC};④\textit{AE}$\mathrm{\bot}$\textit{BC}.其中正确的个数为(  )

\includegraphics*[width=1.22in, height=1.47in, keepaspectratio=false]{image241}

A.1   B.2 C.3   D.4

解析: $\mathrm{\because}$\textit{AB}是$\mathrm{\odot}$\textit{O}的直径,$\mathrm{\therefore}$\textit{AC}$\mathrm{\bot}$\textit{BC}.$\mathrm{\because}$\textit{PA}垂直于$\mathrm{\odot}$\textit{O}所在的平面,$\mathrm{\therefore}$\textit{PA}$\mathrm{\bot}$\textit{AB},\textit{PA}$\mathrm{\bot}$\textit{AC},\textit{PA}$\mathrm{\bot}$\textit{BC},\textit{BC}$\mathrm{\bot}$平面\textit{PAC},$\mathrm{\therefore}$\textit{BC}$\mathrm{\bot}$\textit{AF},$\mathrm{\therefore}$③正确.又\textit{AF}$\mathrm{\bot}$\textit{PC},$\mathrm{\therefore}$\textit{AF}$\mathrm{\bot}$平面\textit{PBC},$\mathrm{\therefore}$\textit{AF}$\mathrm{\bot}$\textit{PB},$\mathrm{\therefore}$①正确.又\textit{AE}$\mathrm{\bot}$\textit{PB},$\mathrm{\therefore}$\textit{PB}$\mathrm{\bot}$平面\textit{AEF},$\mathrm{\therefore}$\textit{EF}$\mathrm{\bot}$\textit{PB},$\mathrm{\therefore}$②正确.若\textit{AE}$\mathrm{\bot}$\textit{BC},则由\textit{AE}$\mathrm{\bot}$\textit{PB},得\textit{AE}$\mathrm{\bot}$平面\textit{PBC},此时\textit{E}、\textit{F}重合,与已知矛盾,$\mathrm{\therefore}$④错误.故选C.

答案:C



知识:直线与平面垂直的性质

难度:2

题目:已知三棱锥\textit{P}-\textit{ABC},\textit{PA}$\mathrm{\bot}$平面\textit{ABC},\textit{AC}$\mathrm{\bot}$\textit{BC},\textit{PA}=2,\textit{AC}=\textit{BC}=1,则三棱锥\textit{P}-\textit{ABC}外接球的体积为\_\_\_\_.

\includegraphics*[width=1.52in, height=1.55in, keepaspectratio=false]{image242}

解析:如图所示

\includegraphics*[width=1.56in, height=1.44in, keepaspectratio=false]{image243}

取\textit{PB}的中点\textit{O},$\mathrm{\because}$\textit{PA}$\mathrm{\bot}$平面\textit{ABC},

$\mathrm{\therefore}$\textit{PA}$\mathrm{\bot}$\textit{AB},\textit{PA}$\mathrm{\bot}$\textit{BC},又\textit{BC}$\mathrm{\bot}$\textit{AC},\textit{PA}$\mathrm{\cap}$\textit{AC}=\textit{A},$\mathrm{\therefore}$\textit{BC}$\mathrm{\bot}$平面\textit{PAC},

$\mathrm{\therefore}$\textit{BC}$\mathrm{\bot}$\textit{PC}.$\mathrm{\therefore}$\textit{OA}=$\frac{1}{2}$\textit{PB},\textit{OC}=$\frac{1}{2}$\textit{PB},$\mathrm{\therefore}$\textit{OA}=\textit{OB}=\textit{OC}=\textit{OP},故\textit{O}为外接球的球心.

又\textit{PA}=2,\textit{AC}=\textit{BC}=1,

$\mathrm{\therefore}$\textit{AB}=$\sqrt{2}$,\textit{PB}=$\sqrt{6}$,

$\mathrm{\therefore}$外接球的半径\textit{R}=$\frac{\sqrt{6}}{2}$.

$\mathrm{\therefore}$\textit{V}${}_{\textrm{球}}$=$\frac{4}{3}\pi$\textit{R}${}^{3}$=$\frac{4\pi}{3}\mathrm{\times}$$(\frac{\sqrt{6}}{2})^{3}$=$\sqrt{6}\pi$.

答案:$\sqrt{6}\pi$

知识:直线与平面垂直的性质

难度:2

题目:$\mathrm{\vartriangle}$\textit{ABC}的三个顶点\textit{A}、\textit{B}、\textit{C}到平面\textit{$\alpha$}的距离分别为2 cm、3 cm、4 cm,且它们在\textit{$\alpha$}的同侧,则$\mathrm{\vartriangle}$\textit{ABC}的重心到平面\textit{$\alpha$}的距离为\_\_\_\_.

解析:如图,设\textit{A}、\textit{B}、\textit{C}在平面\textit{$\alpha$}上的射影分别为\textit{A}$'$、\textit{B}$'$、\textit{C}$'$,

\includegraphics*[width=1.70in, height=1.63in, keepaspectratio=false]{image244}

$\mathrm{\vartriangle}$\textit{ABC}的重心为\textit{G},连接\textit{CG}并延长交\textit{AB}于中点\textit{E},

又设\textit{E}、\textit{G}在平面\textit{$\alpha$}上的射影分别为\textit{E}$'$、\textit{G}$'$,

则\textit{E}$'$$\mathrm{\in}$\textit{A}$'$\textit{B}$'$,\textit{G}$'$$\mathrm{\in}$\textit{C}$'$\textit{E}$'$,\textit{EE}$'$=$\frac{1}{2}$(\textit{A}$'$\textit{A}+\textit{B}$'$\textit{B})=$\frac{5}{2}$,\textit{CC}$'$=4,\textit{CG}︰\textit{GE}=2︰1,在直角梯形\textit{EE}$'$\textit{C}$'$\textit{C}中,可求得\textit{GG}$'$=3.

答案:3 cm

知识:直线与平面垂直的性质

难度:3

题目:如图,在直三棱柱\textit{ABC}-\textit{A}${}_{1}$\textit{B}${}_{1}$\textit{C}${}_{1}$中,已知\textit{AC}$\mathrm{\bot}$\textit{BC},\textit{BC}=\textit{CC}${}_{1}$,设\textit{AB}${}_{1}$的中点为\textit{D},\textit{B}${}_{1}$\textit{C}$\mathrm{\cap}$\textit{BC}${}_{1}$=\textit{E}.

\includegraphics*[width=1.10in, height=1.45in, keepaspectratio=false]{image245}

求证:(1)\textit{DE}//平面\textit{AA}${}_{1}$\textit{C}${}_{1}$\textit{C};

(2)\textit{BC}${}_{1}$$\mathrm{\bot}$\textit{AB}${}_{1}$.

解析:

答案:(1)由题意知,\textit{E}为\textit{B}${}_{1}$\textit{C}的中点,

\includegraphics*[width=1.10in, height=1.45in, keepaspectratio=false]{image246}

又\textit{D}为\textit{AB}${}_{1}$的中点,因此\textit{DE}//\textit{AC}.

又因为\textit{DE}$\mathrm{\nsubset}$平面\textit{AA}${}_{1}$\textit{C}${}_{1}$\textit{C},\textit{AC}$\mathrm{\subset }$平面\textit{AA}${}_{1}$\textit{C}${}_{1}$\textit{C},

所以\textit{DE}//平面\textit{AA}${}_{1}$\textit{C}${}_{1}$\textit{C}.

(2)因为棱柱\textit{ABC}-\textit{A}${}_{1}$\textit{B}${}_{1}$\textit{C}${}_{1}$是直三棱柱,

所以\textit{CC}${}_{1}$$\mathrm{\bot}$平面\textit{ABC}.

因为\textit{AC}$\mathrm{\subset }$平面\textit{ABC},所以\textit{AC}$\mathrm{\bot}$\textit{CC}${}_{1}$.

又因为\textit{AC}$\mathrm{\bot}$\textit{BC},\textit{CC}${}_{1}$$\mathrm{\subset }$平面\textit{BCC}${}_{1}$\textit{B}${}_{1}$,

\textit{BC}$\mathrm{\subset }$平面\textit{BCC}${}_{1}$\textit{B}${}_{1}$,\textit{BC}$\mathrm{\cap}$\textit{CC}${}_{1}$=\textit{C},

所以\textit{AC}$\mathrm{\bot}$平面\textit{BCC}${}_{1}$\textit{B}${}_{1}$,

又因为\textit{BC}${}_{1}$$\mathrm{\subset }$平面\textit{BCC}${}_{1}$\textit{B}${}_{1}$,所以\textit{B}${}_{1}$\textit{C}$\mathrm{\bot}$\textit{AC}.

因为\textit{BC}=\textit{CC}${}_{1}$,所以矩形\textit{BCC}${}_{1}$\textit{B}${}_{1}$是正方形,因此\textit{BC}${}_{1}$$\mathrm{\bot}$\textit{B}${}_{1}$\textit{C}.

因为\textit{AC},\textit{B}${}_{1}$\textit{C}$\mathrm{\subset }$平面\textit{B}${}_{1}$\textit{AC},\textit{AC}$\mathrm{\cap}$\textit{B}${}_{1}$\textit{C}=\textit{C},所以\textit{BC}${}_{1}$$\mathrm{\bot}$平面\textit{B}${}_{1}$\textit{AC}.

又因为\textit{AB}${}_{1}$$\mathrm{\subset }$平面\textit{B}${}_{1}$\textit{AC},所以\textit{BC}${}_{1}$$\mathrm{\bot}$\textit{AB}${}_{1}$.

知识:直线与平面垂直的性质

难度:3

题目:如图,在四棱锥\textit{P}-\textit{ABCD}中,\textit{PA}$\mathrm{\bot}$平面\textit{ABCD},\textit{AB}=\textit{BC}=2,\textit{AD}=\textit{CD}=$\sqrt{7}$,\textit{PA}=$\sqrt{3}$,$\mathrm{\angle}$\textit{ABC}=120$\mathrm{{}^\circ}$.\textit{G}为线段\textit{PC}上的点.

\includegraphics*[width=1.12in, height=1.27in, keepaspectratio=false]{image247}

(1)证明:\textit{BD}$\mathrm{\bot}$平面\textit{APC};

(2)若\textit{G}为\textit{PC}的中点,求\textit{DG}与平面\textit{APC}所成角的正切值;

(3)若\textit{G}满足\textit{PC}$\mathrm{\bot}$平面\textit{BGD},求$\frac{PG}{GC}$的值.

解析:

答案:(1)设点\textit{O}为\textit{AC}、\textit{BD}的交点.

由\textit{AB}=\textit{BC},\textit{AD}=\textit{CD},得\textit{BD}垂直平分线段\textit{AC}.

所以\textit{O}为\textit{AC}的中点,\textit{BD}$\mathrm{\bot}$\textit{AC}.

又因为\textit{PA}$\mathrm{\bot}$平面\textit{ABCD},\textit{BD}$\mathrm{\subset }$平面\textit{ABCD},

所以\textit{PA}$\mathrm{\bot}$\textit{BD}.

又\textit{PA}$\mathrm{\cap}$\textit{AC}=\textit{A},

所以\textit{BD}$\mathrm{\bot}$平面\textit{APC}.

(2)连接\textit{OG}.由(1)可知\textit{OD}$\mathrm{\bot}$平面\textit{APC},则\textit{DG}在平面\textit{APC}内的射影为\textit{OG},所以$\mathrm{\angle}$\textit{OGD}是\textit{DG}与平面\textit{PAC}所成的角.

由题意得\textit{OG}=$\frac{1}{2}$\textit{PA}=$\frac{\sqrt{3}}{2}$.

在$\mathrm{\vartriangle}$\textit{ABC}中,

因为\textit{AB}=\textit{BC},$\mathrm{\angle}$\textit{ABC}=120$\mathrm{{}^\circ}$,\textit{AO}=\textit{CO},

\includegraphics*[width=1.18in, height=1.27in, keepaspectratio=false]{image248}

所以$\mathrm{\angle}$\textit{ABO}=$\frac{1}{2}\mathrm{\angle}$\textit{ABC}=60$\mathrm{{}^\circ}$,

所以\textit{AO}=\textit{OC}=\textit{AB}·sin60$\mathrm{{}^\circ}$=$\sqrt{3}$.

在Rt$\mathrm{\vartriangle}$\textit{OCD}中,\textit{OD}=$\sqrt{CD^2-OC^2}$=2.

在Rt$\mathrm{\vartriangle}$\textit{OGD}中,tan$\mathrm{\angle}$\textit{OGD}=$\frac{OD}{OG}$=$\frac{4\sqrt{3}}{3}$.

所以\textit{DG}与平面\textit{APC}所成角的正切值为$\frac{4\sqrt{3}}{3}$.

(3)因为\textit{PC}$\mathrm{\bot}$平面\textit{BGD},\textit{OG}$\mathrm{\subset }$平面\textit{BGD},所以\textit{PC}$\mathrm{\bot}$\textit{OG}.

在Rt$\mathrm{\vartriangle}$\textit{PAC}中,\textit{PC}=$\sqrt{\sqrt{3}^2+(2\sqrt{3})^2}$=$\sqrt{15}$.

所以\textit{GC}=$\frac{AC\cdot OC}{PC母}$=$\frac{2\sqrt{15}}{5}$.

从而\textit{PG}=$\frac{3\sqrt{15}}{5}$,

所以$\frac{PG}{GC}$=$\frac{3}{2}$.

知识:平面与平面垂直的性质

难度:1

题目:平面\textit{$\alpha$}$\mathrm{\bot}$平面\textit{$\beta$},\textit{$\alpha$}$\mathrm{\cap}$\textit{$\beta$}=\textit{l},\textit{m}$\mathrm{\subset }$\textit{$\alpha$},\textit{m}$\mathrm{\bot}$\textit{l},则(  )

A.\textit{m}//\textit{$\beta$  }B.\textit{m}$\mathrm{\subset }$\textit{$\beta$}

C.\textit{m}$\mathrm{\bot}$\textit{$\beta$  }D.\textit{m}与\textit{$\beta$}相交但不一定垂直

解析:
如图,

\includegraphics*[width=1.18in, height=0.73in, keepaspectratio=false]{image250}

$\mathrm{\because}$\textit{$\alpha$}$\mathrm{\bot}$\textit{$\beta$},\textit{$\alpha$}$\mathrm{\cap}$\textit{$\beta$}=\textit{l},\textit{m}$\mathrm{\subset }$\textit{$\alpha$},\textit{m}$\mathrm{\bot}$\textit{l},$\mathrm{\therefore}$\textit{m}$\mathrm{\bot}$\textit{$\beta$}.

答案:C

知识:平面与平面垂直的性质

难度:1

题目:设有直线\textit{m}、\textit{n}和平面\textit{$\alpha$}、\textit{$\beta$},则下列命题中正确的是(  )

A.若\textit{m}$\mathrm{\bot}$\textit{n},\textit{m}$\mathrm{\subset }$\textit{$\alpha$},\textit{n}$\mathrm{\subset }$\textit{$\beta$},则\textit{$\alpha$}$\mathrm{\bot}$\textit{$\beta$}

B.若\textit{m}//\textit{n},\textit{n}$\mathrm{\bot}$\textit{$\beta$},\textit{m}$\mathrm{\subset }$\textit{$\alpha$},则\textit{$\alpha$}$\mathrm{\bot}$\textit{$\beta$}

C.若\textit{m}//\textit{n},\textit{m}$\mathrm{\bot}$\textit{$\alpha$},\textit{n}$\mathrm{\bot}$\textit{$\beta$},则\textit{$\alpha$}$\mathrm{\bot}$\textit{$\beta$}

D.若\textit{m}$\mathrm{\bot}$\textit{n},\textit{$\alpha$}$\mathrm{\cap}$\textit{$\beta$}=\textit{m},\textit{n}$\mathrm{\subset }$\textit{$\alpha$},则\textit{$\alpha$}$\mathrm{\bot}$\textit{$\beta$}

解析:$\left. \begin{array}{r}
\left. \begin{array}{r}
n\bot \beta\\
m\\ n
\end{array} \right\}\mathrm{\Rightarrow }m\mathrm{\bot}\beta\\
m\subset \alpha
\end{array} \right\}\mathrm{\Rightarrow }$\textit{$\alpha$}$\mathrm{\bot}$\textit{$\beta$},

$\mathrm{\therefore}$B正确.

答案:B

知识:平面与平面垂直的性质

难度:1

题目:若平面\textit{$\alpha$}$\mathrm{\bot}$平面\textit{$\beta$},且平面\textit{$\alpha$}内的一条直线\textit{a}垂直于平面\textit{$\beta$}内的一条直线\textit{b},则(  )

A.直线\textit{a}必垂直于平面\textit{$\beta$}

B.直线\textit{b}必垂直于平面\textit{$\alpha$}

C.直线\textit{a}不一定垂直于平面\textit{$\beta$}

D.过\textit{a}的平面与过\textit{b}的平面垂直

解析: \textit{$\alpha$}$\mathrm{\bot}$\textit{$\beta$},\textit{a}$\mathrm{\subset }$\textit{$\alpha$},\textit{b}$\mathrm{\subset }$\textit{$\beta$},\textit{a}$\mathrm{\bot}$\textit{b},当\textit{$\alpha$}$\mathrm{\cap}$\textit{$\beta$}=\textit{a}时,\textit{b}$\mathrm{\bot}$\textit{$\alpha$};当\textit{$\alpha$}$\mathrm{\cap}$\textit{$\beta$}=\textit{b}时,\textit{a}$\mathrm{\bot}$\textit{$\beta$},其他情形则未必有\textit{b}$\mathrm{\bot}$\textit{$\alpha$}或\textit{a}$\mathrm{\bot}$\textit{$\beta$},所以选项A、B、D都错误,故选C.

答案:C

知识:平面与平面垂直的性质

难度:1

题目:如右图所示,三棱锥\textit{P}-\textit{ABC}的底面在平面\textit{$\alpha$}内,且\textit{AC}$\mathrm{\bot}$\textit{PC},平面\textit{PAC}$\mathrm{\bot}$平面\textit{PBC},点\textit{P},\textit{A},\textit{B}是定点,则动点\textit{C}的轨迹是(  )

\includegraphics*[width=1.17in, height=0.96in, keepaspectratio=false]{image251}

A.一条线段  B.一条直线 

C.一个圆  D.一个圆,但要去掉两个点

解析: $\mathrm{\because}$平面\textit{PAC}$\mathrm{\bot}$平面\textit{PBC},\textit{AC}$\mathrm{\bot}$\textit{PC},平面\textit{PAC}$\mathrm{\cap}$平面\textit{PBC}=\textit{PC},\textit{AC}$\mathrm{\subset }$平面\textit{PAC},$\mathrm{\therefore}$\textit{AC}$\mathrm{\bot}$平面\textit{PBC}.

又$\mathrm{\because}$\textit{BC}$\mathrm{\subset }$平面\textit{PBC},$\mathrm{\therefore}$\textit{AC}$\mathrm{\bot}$\textit{BC}.$\mathrm{\therefore}$$\mathrm{\angle}$\textit{ACB}=90$\mathrm{{}^\circ}$.

$\mathrm{\therefore}$动点\textit{C}的轨迹是以\textit{AB}为直径的圆,除去\textit{A}和\textit{B}两点.

答案:D

知识:平面与平面垂直的性质

难度:1

题目:已知直线\textit{m},\textit{n}和平面\textit{$\alpha$},\textit{$\beta$},若\textit{$\alpha$}$\mathrm{\bot}$\textit{$\beta$},\textit{$\alpha$}$\mathrm{\cap}$\textit{$\beta$}=\textit{m},\textit{n}$\mathrm{\subset }$\textit{a},要使\textit{n}$\mathrm{\bot}$\textit{$\beta$},则应增加的条件是(  )

A.\textit{m}//\textit{n}   B.\textit{n}$\mathrm{\bot}$\textit{m }C.\textit{n}//\textit{$\alpha$}   D.\textit{n}$\mathrm{\bot}$\textit{$\alpha$}

解析: 由面面垂直的性质定理知,要使\textit{n}$\mathrm{\bot}$\textit{$\beta$},应有\textit{n}与交线\textit{m}垂直,$\mathrm{\therefore}$应增加条件\textit{n}$\mathrm{\bot}$\textit{m}.

答案:B

知识:平面与平面垂直的性质

难度:1

题目:如图,平面\textit{$\alpha$}$\mathrm{\bot}$平面\textit{$\beta$},\textit{A}$\mathrm{\in}$\textit{$\alpha$},\textit{B}$\mathrm{\in}$\textit{$\beta$},\textit{AB}与两平面\textit{$\alpha$}、\textit{$\beta$}所成的角分别为$\frac{\pi}{4}$和$\frac{\pi}{6}$.过\textit{A}、\textit{B}分别作两平面交线的垂线,垂足为\textit{A}$'$、\textit{B}$'$,则\textit{AB}︰\textit{A}$'$\textit{B}$'$等于(  )

\includegraphics*[width=1.19in, height=1.11in, keepaspectratio=false]{image252}

A.2︰1   B.3︰1 C.3︰2   D.4︰3

解析:由已知条件可知$\mathrm{\angle}$\textit{BAB}$'$=$\frac{\pi}{4}$,

$\mathrm{\angle}$\textit{ABA}$'$=$\frac{\pi}{6}$,设\textit{AB}=2\textit{a},

则\textit{BB}$'$=2\textit{a}sin$\frac{\pi}{4}$=$\sqrt{2}$\textit{a},\textit{A}$'$\textit{B}=2\textit{a}cos$\frac{\pi}{6}$=$\sqrt{3}$\textit{a},

$\mathrm{\therefore}$在Rt$\mathrm{\vartriangle}$\textit{BB}$'$\textit{A}$'$中,得\textit{A}$'$\textit{B}$'$=\textit{a},$\mathrm{\therefore}$\textit{AB}︰\textit{A}$'$\textit{B}$'$=2︰1.

答案:A

知识:平面与平面垂直的性质

难度:1

题目:已知直线\textit{l}$\mathrm{\bot}$平面\textit{$\alpha$},直线\textit{m}$\mathrm{\subset }$平面\textit{$\beta$},给出下列四个命题:

①\textit{$\alpha$}//\textit{$\beta$},\textit{l}$\mathrm{\nsubset}$\textit{$\beta$}$\mathrm{\Rightarrow }$\textit{l}$\mathrm{\bot}$\textit{m};    ②\textit{$\alpha$}$\mathrm{\bot}$\textit{$\beta$}$\mathrm{\Rightarrow }$\textit{l}//\textit{m};

③\textit{l}//\textit{m}$\mathrm{\Rightarrow }$\textit{$\alpha$}$\mathrm{\bot}$\textit{$\beta$};    ④\textit{l}$\mathrm{\bot}$\textit{m}$\mathrm{\Rightarrow }$\textit{$\alpha$}//\textit{$\beta$}.

其中正确的两个命题是\_\_\_\_.

解析: $\left. \begin{array}{r}
\left. \begin{array}{r}
l\bot \alpha\\
\alpha \\ \beta
\end{array} \right\}\mathrm{\Rightarrow }l\mathrm{\bot}\beta\\
m\subset \beta
\end{array} \right\}\mathrm{\Rightarrow }$\textit{l}$\mathrm{\bot}$\textit{m},故①对;

$\left. \begin{array}{r}
\alpha \bot \beta\\
l \bot \alpha
\end{array} \right\}\mathrm{\Rightarrow }$\textit{l}//\textit{$\beta$}或\textit{l}$\mathrm{\subset }$\textit{$\beta$},又\textit{m}是\textit{$\beta$}内的一条直线,故\textit{l}//\textit{m}不对;

$\left. \begin{array}{r}
\left. \begin{array}{r}
l// m\\
m \subset \beta
\end{array} \right\}\mathrm{\Rightarrow }l//\beta $或$ l\subset\beta\\
l\bot \alpha
\end{array} \right\}\mathrm{\Rightarrow }$\textit{$\alpha$}$\mathrm{\bot}$\textit{$\beta$},$\mathrm{\therefore}$③对;

$\left. \begin{array}{r}
l\bot \alpha\\
l \bot m
\end{array} \right\}\mathrm{\Rightarrow }$\textit{m}$\mathrm{\subset }$\textit{$\alpha$}或\textit{m}//\textit{$\alpha$},无论哪种情况与\textit{m}$\mathrm{\subset }$\textit{$\beta$}结合都不能得出\textit{$\alpha$}//\textit{$\beta$},$\mathrm{\therefore}$选D.

答案:①③

知识:平面与平面垂直的性质

难度:1

题目:三棱锥\textit{P}-\textit{ABC}的高为\textit{PH},若三个侧面两两垂直,则\textit{H}为$\mathrm{\vartriangle}$\textit{ABC}的\_\_\_\_.

解析: 由三个侧面两两垂直知三条侧棱两两垂直,则有\textit{BC}$\mathrm{\bot}$\textit{PA},\textit{AB}$\mathrm{\bot}$\textit{PC},\textit{CA}$\mathrm{\bot}$\textit{PB},又由\textit{BC}$\mathrm{\bot}$\textit{PA},\textit{PH}$\mathrm{\bot}$\textit{BC},得\textit{BC}$\mathrm{\bot}$平面\textit{PAH},则\textit{BC}$\mathrm{\bot}$\textit{AH},同理有\textit{AB}$\mathrm{\bot}$\textit{CH},\textit{CA}$\mathrm{\bot}$\textit{BH},所以\textit{H}为$\mathrm{\vartriangle}$\textit{ABC}高线的交点,即垂心.

答案:垂心

知识:平面与平面垂直的性质

难度:1

题目:把一副三角板如图拼接,设\textit{BC}=6,$\mathrm{\angle}$\textit{A}=90$\mathrm{{}^\circ}$,\textit{AB}=\textit{AC},$\mathrm{\angle}$\textit{BCD}=90$\mathrm{{}^\circ}$,$\mathrm{\angle}$\textit{D}=60$\mathrm{{}^\circ}$,使两块三角板所在的平面互相垂直.求证:平面\textit{ABD}$\mathrm{\bot}$平面\textit{ACD}.

\includegraphics*[width=1.20in, height=1.14in, keepaspectratio=false]{image253}

解析:

答案:$\left. \begin{array}{r}
\left. \begin{array}{r}
$平面ABC$\bot $平面BCD$\\
CD \bot BC
\end{array} \right\}\mathrm{\Rightarrow }CD\mathrm{\bot}$平面ABC$ \\
AB\subset $平面ABC$
\end{array} \right\}\mathrm{\Rightarrow }$

$\left. \begin{array}{r}
\left. \begin{array}{r}
CD\bot AB\\
AB \bot AC
\end{array} \right\}\mathrm{\Rightarrow }AB\mathrm{\bot}$平面ACD$ \\
AB\subset $平面ABD$
\end{array} \right\}\mathrm{\Rightarrow }$平面\textit{ABD}$\mathrm{\bot}$平面\textit{ACD}.

知识:平面与平面垂直的性质

难度:1

题目:如图所示,在四棱锥\textit{P}-\textit{ABCD}中,侧面\textit{PAD}$\mathrm{\bot}$底面\textit{ABCD},侧棱\textit{PA}$\mathrm{\bot}$\textit{PD},底面\textit{ABCD}是直角梯形,其中\textit{BC}//\textit{AD},$\mathrm{\angle}$\textit{BAD}=90$\mathrm{{}^\circ}$,\textit{AD}=3\textit{BC},\textit{O}是\textit{AD}上一点.

\includegraphics*[width=1.29in, height=0.81in, keepaspectratio=false]{image254}

(1)若\textit{CD}//平面\textit{PBO},试指出点\textit{O}的位置;

(2)求证:平面\textit{PAB}$\mathrm{\bot}$平面\textit{PCD}.

解析:

答案: (1)$\mathrm{\because}$\textit{CD}//平面\textit{PBO},\textit{CD}$\mathrm{\subset }$平面\textit{ABCD},

且平面\textit{ABCD}$\mathrm{\cap}$平面\textit{PBO}=\textit{BO},

$\mathrm{\therefore}$\textit{BO}//\textit{CD}.

又\textit{BC}//\textit{AD},$\mathrm{\therefore}$四边形\textit{BCDO}为平行四边形.

则\textit{BC}=\textit{DO},而\textit{AD}=3\textit{BC},

$\mathrm{\therefore}$\textit{AD}=3\textit{OD},即点\textit{O}是靠近点\textit{D}的线段\textit{AD}的一个三等分点.

(2)证明:$\mathrm{\because}$侧面\textit{PAD}$\mathrm{\bot}$底面\textit{ABCD},侧面\textit{PAD}$\mathrm{\cap}$底面\textit{ABCD}=\textit{AD},\textit{AB}$\mathrm{\subset }$底面\textit{ABCD},且\textit{AB}$\mathrm{\bot}$\textit{AD},

$\mathrm{\therefore}$\textit{AB}$\mathrm{\bot}$平面\textit{PAD}.

又\textit{PD}$\mathrm{\subset }$平面\textit{PAD},$\mathrm{\therefore}$\textit{AB}$\mathrm{\bot}$\textit{PD}.

又\textit{PA}$\mathrm{\bot}$\textit{PD},且\textit{PA}$\mathrm{\subset }$平面\textit{PAB},\textit{AB}$\mathrm{\subset }$平面\textit{PAB},\textit{AB}$\mathrm{\cap}$\textit{PA}=\textit{A},$\mathrm{\therefore}$\textit{PD}$\mathrm{\bot}$平面\textit{PAB}.

又\textit{PD}$\mathrm{\subset }$平面\textit{PCD},$\mathrm{\therefore}$平面\textit{PAB}$\mathrm{\bot}$平面\textit{PCD}.

知识:平面与平面垂直的性质

难度:2

题目:\textit{m}、\textit{n}是两条不同的直线,\textit{$\alpha$}、\textit{$\beta$}、\textit{$\gamma$}是三个不同的平面,给出如下命题:

①若\textit{$\alpha$}$\mathrm{\bot}$\textit{$\beta$},\textit{$\alpha$}$\mathrm{\cap}$\textit{$\beta$}=\textit{m},\textit{n}$\mathrm{\subset }$\textit{$\alpha$},\textit{n}$\mathrm{\bot}$\textit{m},则\textit{n}$\mathrm{\bot}$\textit{$\beta$};

②若\textit{$\alpha$}$\mathrm{\bot}$\textit{$\gamma$},\textit{$\beta$}$\mathrm{\bot}$\textit{$\gamma$},则\textit{$\alpha$}//\textit{$\beta$};

③若\textit{$\alpha$}$\mathrm{\bot}$\textit{$\beta$},且\textit{n}$\mathrm{\bot}$\textit{$\beta$},\textit{n}$\mathrm{\bot}$\textit{m},则\textit{m}$\mathrm{\bot}$\textit{$\alpha$};

④\textit{$\alpha$}$\mathrm{\bot}$\textit{$\beta$},\textit{m}$\mathrm{\bot}$\textit{$\beta$},\textit{m}$\mathrm{\nsubset}$\textit{$\alpha$},则\textit{m}//\textit{$\alpha$};

⑤若\textit{$\alpha$}$\mathrm{\bot}$\textit{$\beta$},\textit{m}//\textit{$\alpha$},则\textit{m}$\mathrm{\bot}$\textit{$\beta$}.

其中正确命题的个数为(  )

A.1   B.2   C.3   D.4

解析:根据平面与平面垂直的性质知①正确;②中,\textit{$\alpha$}、\textit{$\beta$}可能平行,也可能相交,不正确;③中,\textit{m}还可能在\textit{$\alpha$}内或\textit{m}//\textit{$\alpha$},或\textit{m}与\textit{$\alpha$}斜交,不正确;④中,\textit{$\alpha$}$\mathrm{\bot}$\textit{$\beta$},\textit{m}$\mathrm{\bot}$\textit{$\beta$},\textit{m}$\mathrm{\nsubset}$\textit{$\alpha$}时,呆可能有\textit{m}//\textit{$\alpha$},正确;⑤中,\textit{m}与\textit{$\beta$}的位置关系可能是\textit{m}//\textit{$\beta$}或\textit{m}$\mathrm{\subset }$\textit{$\beta$}或\textit{m}与\textit{$\beta$}相交,不正确.综上,可知正确命题的个数为2,故选B.

答案:B

知识:平面与平面垂直的性质

难度:2

题目:在空间中,下列命题正确的是(  )

A.若三条直线两两相交,则这三条直线确定一个平面

B.若直线\textit{m}与平面\textit{$\alpha$}内的一条直线平行,则\textit{m}//\textit{$\alpha$}

C.若平面\textit{$\alpha$}$\mathrm{\bot}$\textit{$\beta$},且\textit{$\alpha$}$\mathrm{\cap}$\textit{$\beta$}=\textit{l},则过\textit{$\alpha$}内一点\textit{P}与\textit{l}垂直的直线垂直于平面\textit{$\beta$}

D.若直线\textit{a}//\textit{b},且直线\textit{l}$\mathrm{\bot}$\textit{a},则\textit{l}$\mathrm{\bot}$\textit{b}

解析:选项A中,若有3个交点,则确定一个平面,若三条直线交于一点,则不一定能确定一个平面,如正方体\textit{ABCD}-\textit{A}${}_{1}$\textit{B}${}_{1}$\textit{C}${}_{1}$\textit{D}${}_{1}$中,\textit{AA}${}_{1}$、\textit{AB}、\textit{AD}两两相交,但由\textit{AA}${}_{1}$、\textit{AB}、\textit{AD}不能确定一个平面,所以A不正确;选项B中,缺少条件\textit{m}是平面\textit{$\alpha$}外的一条直线,所以B不正确;选项C中,不满足面面垂直的性质定理的条件,必须是\textit{$\alpha$}内垂直于\textit{l}的直线,所以C不正确;由于两条平行直线中的一条与第三条直线垂直,那么另一条也与第三条直线垂直,所以D正确.

答案:D

知识:平面与平面垂直的性质

难度:2

题目:如图,点\textit{P}为四边形\textit{ABCD}外一点,平面\textit{PAD}$\mathrm{\bot}$平面\textit{ABCD},\textit{PA}=\textit{PD},\textit{E}为\textit{AD}的中点,则下列结论不一定成立的是(  )

\includegraphics*[width=1.36in, height=1.32in, keepaspectratio=false]{image255}

A.\textit{PE}$\mathrm{\bot}$\textit{AC  }B.\textit{PE}$\mathrm{\bot}$\textit{BC}

C.平面\textit{PBE}$\mathrm{\bot}$平面\textit{ABCD }D.平面\textit{PBE}$\mathrm{\bot}$平面\textit{PAD}

解析: 因为\textit{PA}=\textit{PD},\textit{E}为\textit{AD}的中点,所以\textit{PE}$\mathrm{\bot}$\textit{AD}.又平面\textit{PAD}$\mathrm{\bot}$平面\textit{ABCD},平面\textit{PAD}$\mathrm{\cap}$平面\textit{ABCD}=\textit{AD},所以\textit{PE}$\mathrm{\bot}$平面\textit{ABCD},所以\textit{PE}$\mathrm{\bot}$\textit{AC},\textit{PE}$\mathrm{\bot}$\textit{BC},所以A、B成立.又\textit{PE}$\mathrm{\subset }$平面\textit{PBE},所以平面\textit{PBE}$\mathrm{\bot}$平面\textit{ABCD},所以C成立.若平面\textit{PBE}$\mathrm{\bot}$平面\textit{PAD},则\textit{AD}$\mathrm{\bot}$平面\textit{PBE},必有\textit{AD}$\mathrm{\bot}$\textit{BE},此关系不一定成立,故选D.

答案:D

知识:平面与平面垂直的性质

难度:2

题目:如图所示,\textit{P}是菱形\textit{ABCD}所在平面外的一点,且$\mathrm{\angle}$\textit{DAB}=60$\mathrm{{}^\circ}$,边长为\textit{a}.侧面\textit{PAD}为正三角形,其所在平面垂直于底面\textit{ABCD},\textit{PB}与平面\textit{AC}所成的角为\textit{$\theta$},则\textit{$\theta$}=\_\_\_\_.

\includegraphics*[width=1.12in, height=1.06in, keepaspectratio=false]{image256}

解析: 如图所示,取\textit{AD}的中点\textit{G},连接\textit{PG},\textit{BG},\textit{BD}.

\includegraphics*[width=1.12in, height=1.08in, keepaspectratio=false]{image257}

$\mathrm{\because}$$\mathrm{\vartriangle}$\textit{PAD}是等边三角形,

$\mathrm{\therefore}$\textit{PG}$\mathrm{\bot}$\textit{AD},又平面\textit{PAD}$\mathrm{\bot}$平面\textit{AC},平面\textit{PAD}$\mathrm{\cap}$平面\textit{AC}=\textit{AD},\textit{PG}$\mathrm{\subset }$平面\textit{PAD},

$\mathrm{\therefore}$\textit{PG}$\mathrm{\bot}$平面\textit{AC},$\mathrm{\therefore}$$\mathrm{\angle}$\textit{PBG}是\textit{PB}与平面\textit{AC}所成的角\textit{$\theta$}.

在$\mathrm{\vartriangle}$\textit{PBG}中,\textit{PG}$\mathrm{\bot}$\textit{BG},\textit{BG}=\textit{PG},

$\mathrm{\therefore}$$\mathrm{\angle}$\textit{PBG}=45$\mathrm{{}^\circ}$,即\textit{$\theta$}=45$\mathrm{{}^\circ}$.

答案:45$\mathrm{{}^\circ}$

知识:平面与平面垂直的性质

难度:2

题目:(2016·四川文)如图,在四棱锥\textit{P}-\textit{ABCD}中,\textit{PA}$\mathrm{\bot}$\textit{CD},\textit{AD}//\textit{BC},$\mathrm{\angle}$\textit{ADC}=$\mathrm{\angle}$\textit{PAB}=90$\mathrm{{}^\circ}$,\textit{BC}=\textit{CD}=$\frac{1}{2}$\textit{AD}.

\includegraphics*[width=1.59in, height=1.21in, keepaspectratio=false]{image258}

(1)在平面\textit{PAD}内找一点\textit{M},使得直线\textit{CM}//平面\textit{PAB},并说明理由;

(2)证明:平面\textit{PAB}$\mathrm{\bot}$平面\textit{PBD}.

解析:

答案:(1)取棱\textit{AD}的中点\textit{M}(\textit{M}$\mathrm{\in}$平面\textit{PAD}),点\textit{M}即为所求的一个点.

\includegraphics*[width=1.59in, height=1.21in, keepaspectratio=false]{image259}

理由如下:

因为\textit{AD}//\textit{BC},\textit{BC}=$\frac{1}{2}$\textit{AD},所以\textit{BC}//\textit{AM},且\textit{BC}=\textit{AM},

所以四边形\textit{AMCB}是平行四边形,

从而\textit{CM}//\textit{AB}.

又\textit{AB}$\mathrm{\subset }$平面\textit{PAB},\textit{CM}$\mathrm{\nsubset}$平面\textit{PAB},

所以\textit{CM}//平面\textit{PAB}.

(说明:取棱\textit{PD}的中点\textit{N},则所找的点可以是直线\textit{MN}上任意一点)

(2)由已知,\textit{PA}$\mathrm{\bot}$\textit{AB},\textit{PA}$\mathrm{\bot}$\textit{CD},

因为\textit{AD}//\textit{BC},\textit{BC}=$\frac{1}{2}$\textit{AD},所以直线\textit{AB}与\textit{CD}相交.

所以\textit{PA}$\mathrm{\bot}$平面\textit{ABCD}.

从而\textit{PA}$\mathrm{\bot}$\textit{BD}.

连接\textit{BM},

因为\textit{AD}//\textit{BC},\textit{BC}=$\frac{1}{2}$\textit{AD},

所以\textit{BC}//\textit{MD},且\textit{BC}=\textit{MD}.

所以四边形\textit{BCDM}是平行四边形.

所以\textit{BM}=\textit{CD}=$\frac{1}{2}$\textit{AD},所以\textit{BD}$\mathrm{\bot}$\textit{AB}.

又\textit{AB}$\mathrm{\cap}$\textit{AP}=\textit{A},所以\textit{BD}$\mathrm{\bot}$平面\textit{PAB}.

又\textit{BD}$\mathrm{\subset }$平面\textit{PBD}.

所以平面\textit{PAB}$\mathrm{\bot}$平面\textit{PBD}.

知识:平面与平面垂直的性质

难度:3

题目:如图所示,在四棱锥\textit{P}-\textit{ABCD}中,底面\textit{ABCD}是$\mathrm{\angle}$\textit{DAB}=60$\mathrm{{}^\circ}$且边长为\textit{a}的菱形,侧面\textit{PAD}为正三角形,其所在平面垂直于底面\textit{ABCD}.

(1)求证\textit{AD}$\mathrm{\bot}$\textit{PB};

(2)若\textit{E}为\textit{BC}边的中点,能否在棱\textit{PC}上找到一点\textit{F},使平面\textit{DEF}$\mathrm{\bot}$平面\textit{ABCD}?并证明你的结论.

\includegraphics*[width=1.20in, height=0.94in, keepaspectratio=false]{image260}

解析:

答案:(1)设\textit{G}为\textit{AD}的中点,连接\textit{BG}、\textit{PG},

\includegraphics*[width=1.20in, height=0.94in, keepaspectratio=false]{image261}

$\mathrm{\because}$$\mathrm{\vartriangle}$\textit{PAD}为正三角形,$\mathrm{\therefore}$\textit{PG}$\mathrm{\bot}$\textit{AD}.

在菱形\textit{ABCD}中,$\mathrm{\angle}$\textit{DAB}=60$\mathrm{{}^\circ}$,\textit{G}为\textit{AD}的中点,

$\mathrm{\therefore}$\textit{BG}$\mathrm{\bot}$\textit{AD}.

又\textit{BG}$\mathrm{\cap}$\textit{PG}=\textit{G},$\mathrm{\therefore}$\textit{AD}$\mathrm{\bot}$平面\textit{PGB}.

$\mathrm{\because}$\textit{PB}$\mathrm{\subset }$平面\textit{PGB},$\mathrm{\therefore}$\textit{AD}$\mathrm{\bot}$\textit{PB}.

(2)当\textit{F}为\textit{PC}的中点时,平面\textit{DEF}$\mathrm{\bot}$平面\textit{ABCD}.

证明如下:

在$\mathrm{\vartriangle}$\textit{PBC}中,$\mathrm{\because}$\textit{F}是\textit{PC}的中点,$\mathrm{\therefore}$\textit{EF}//\textit{PB}.

在菱形\textit{ABCD}中,\textit{GB}//\textit{DE},而\textit{FE}$\mathrm{\subset }$平面\textit{DEF},\textit{DE}$\mathrm{\subset }$平面\textit{DEF},\textit{EF}$\mathrm{\cap}$\textit{DE}=\textit{E},

$\mathrm{\therefore}$平面\textit{DEF}//平面\textit{PGB},

由(1)得\textit{PG}$\mathrm{\bot}$平面\textit{ABCD},而\textit{PG}$\mathrm{\subset }$平面\textit{PGB},

$\mathrm{\therefore}$平面\textit{PGB}$\mathrm{\bot}$平面\textit{ABCD},$\mathrm{\therefore}$平面\textit{DEF}$\mathrm{\bot}$平面\textit{ABCD}.

知识:平面与平面垂直的性质

难度:3

题目:(2016·泰安二中高一检测)如图所示,边长为2的等边$\mathrm{\vartriangle}$\textit{PCD}所在的平面垂直于矩形\textit{ABCD}所在的平面,\textit{BC}=$2\sqrt{2}$,\textit{M}为\textit{BC}的中点.

\includegraphics*[width=1.52in, height=1.32in, keepaspectratio=false]{image262}

(1)证明:\textit{AM}$\mathrm{\bot}$\textit{PM};

(2)求二面角\textit{P}-\textit{AM}-\textit{D}的大小.

解析:

答案:(1)如图所示,取\textit{CD}的中点\textit{E},连接\textit{PE}、\textit{EM}、\textit{EA}.

$\mathrm{\because}$$\mathrm{\vartriangle}$\textit{PCD}为正三角形,

$\mathrm{\therefore}$\textit{PE}$\mathrm{\bot}$\textit{CD},\textit{PE}=\textit{PD}sin$\mathrm{\angle}$\textit{PDE}=2sin60$\mathrm{{}^\circ}$=$\sqrt{3}$.

$\mathrm{\because}$平面\textit{PCD}$\mathrm{\bot}$平面\textit{ABCD},

$\mathrm{\therefore}$\textit{PE}$\mathrm{\bot}$平面\textit{ABCD},而\textit{AM}$\mathrm{\subset }$平面\textit{ABCD},$\mathrm{\because}$\textit{PE}$\mathrm{\bot}$\textit{AM}.

$\mathrm{\therefore}$四边形\textit{ABCD}是矩形,

$\mathrm{\therefore}$$\mathrm{\vartriangle}$\textit{ADE}、$\mathrm{\vartriangle}$\textit{ECM}、$\mathrm{\vartriangle}$\textit{ABM}均为直角三角形,

由勾股定理可求得\textit{EM}=$\sqrt{3}$,\textit{AM}=$\sqrt{6}$,\textit{AE}=3,

\includegraphics*[width=1.52in, height=1.32in, keepaspectratio=false]{image263}

$\mathrm{\therefore}$\textit{EM}${}^{2}$+\textit{AM}${}^{2}$=\textit{AE}${}^{2}$.$\mathrm{\therefore}$\textit{AM}$\mathrm{\bot}$\textit{EM}.

又\textit{PE}$\mathrm{\cap}$\textit{EM}=\textit{E},$\mathrm{\therefore}$\textit{AM}$\mathrm{\bot}$平面\textit{PEM},$\mathrm{\therefore}$\textit{AM}$\mathrm{\bot}$\textit{PM}.

(2)由(1)可知,\textit{EM}$\mathrm{\bot}$\textit{AM},\textit{PM}$\mathrm{\bot}$\textit{AM},

$\mathrm{\therefore}$$\mathrm{\angle}$\textit{PME}是二面角\textit{P}-\textit{AM}-\textit{D}的平面角.

在Rt$\mathrm{\vartriangle}$\textit{PEM}中,tan$\mathrm{\angle}$\textit{PME}=$\frac{PE}{EM}$=$\frac{\sqrt{3}}{\sqrt{3}}$=1,$\mathrm{\therefore}$$\mathrm{\angle}$\textit{PME}=45$\mathrm{{}^\circ}$.

$\mathrm{\therefore}$二面角\textit{P}-\textit{AM}-\textit{D}的大小为45$\mathrm{{}^\circ}$. 


知识:倾斜角与斜率

难度:1

题目:(2016~2017·烟台高一检测)若直线的倾斜角为60$\mathrm{{}^\circ}$,则直线的斜率为(  )

A.$\sqrt{3}$   B.-$\sqrt{3}$   C.$\frac{\sqrt{3}}{3}$   D.-$\frac{\sqrt{3}}{3}$

解析:直线的斜率\textit{k}=tan60$\mathrm{{}^\circ}$=$\sqrt{3}$.故选A.

答案:A

知识:倾斜角与斜率

难度:1

题目:若过两点\textit{A}(4,\textit{y})、\textit{B}(2,-3)的直线的倾斜角为45$\mathrm{{}^\circ}$,则\textit{y}等于(  )

A.-$\frac{\sqrt{3}}{2}$   B.$\frac{\sqrt{3}}{2}$   C.-1   D.1

解析:$\mathrm{\because}$直线的倾斜角为45$\mathrm{{}^\circ}$,

$\mathrm{\therefore}$直线的斜率\textit{k}=tan45$\mathrm{{}^\circ}$=1,

$-\frac{-3-y}{2-4}=1$,$\mathrm{\therefore}$\textit{y}=-1.

答案:C

知识:倾斜角与斜率

难度:1

题目:(2016·肥城高一检测)若\textit{A}(-2,3)、\textit{B}(3,-2)、\textit{C}($\frac{1}{2}$,\textit{m})三点共线,则\textit{m}的值为(  )

A.$\frac{1}{2}$   B.-$\frac{1}{2}$   C.-2   D.2

解析:由已知得,\textit{k${}_{AB}$}=\textit{k${}_{AC}$},

$\mathrm{\therefore}\frac{-2-3}{3-(-2)}=\frac{m-3}{\frac{1}{2}-(-2)}$,解得\textit{m}=$\frac{1}{2}$.

答案:A

知识:倾斜角与斜率

难度:1

题目:直线\textit{l}的倾斜角是斜率为的直线的倾斜角的2倍,则\textit{l}的斜率为(  )

A.1   B.$\sqrt{3}$   C.$\frac{2\sqrt{3}}{3}$   D.$-\sqrt{3}$

解析:$\mathrm{\because}$tan\textit{$\alpha$}=$\frac{\sqrt{3}}{3}$,0$\mathrm{{}^\circ}$$\mathrm{\le}$\textit{$\alpha$}$\mathrm{<}$180$\mathrm{{}^\circ}$,$\mathrm{\therefore}$\textit{$\alpha$}=30$\mathrm{{}^\circ}$,

$\mathrm{\therefore}$2\textit{$\alpha$}=60$\mathrm{{}^\circ}$,$\mathrm{\therefore}$\textit{k}=tan2\textit{$\alpha$}=$\sqrt{3}$.故选B.

答案:B

知识:倾斜角与斜率

难度:1

题目:如下图,已知直线\textit{l}${}_{1}$、\textit{l}${}_{2}$、\textit{l}${}_{3}$的斜率分别为\textit{k}${}_{1}$、\textit{k}${}_{2}$、\textit{k}${}_{3}$,则(  )

\includegraphics*[width=1.50in, height=1.19in, keepaspectratio=false]{image265}

A.\textit{k}${}_{1}$$\mathrm{<}$\textit{k}${}_{2}$$\mathrm{<}$\textit{k}${}_{3}$   B.\textit{k}${}_{3}$$\mathrm{<}$\textit{k}${}_{1}$$\mathrm{<}$\textit{k}${}_{2}$${}_{ }$C.\textit{k}${}_{3}$$\mathrm{<}$\textit{k}${}_{2}$$\mathrm{<}$\textit{k}${}_{1}$   D.\textit{k}${}_{1}$$\mathrm{<}$\textit{k}${}_{3}$$\mathrm{<}$\textit{k}${}_{2}$

解析:可由直线的倾斜程度,结合倾斜角与斜率的关系求解.设直线\textit{l}${}_{1}$、\textit{l}${}_{2}$、\textit{l}${}_{3}$的倾斜角分别是\textit{$\alpha$}${}_{1}$、\textit{$\alpha$}${}_{2}$、\textit{$\alpha$}${}_{3}$,由图可知\textit{$\alpha$}${}_{1}$$\mathrm{>}$90$\mathrm{{}^\circ}$$\mathrm{>}$\textit{$\alpha$}${}_{2}$$\mathrm{>}$\textit{$\alpha$}${}_{3}$$\mathrm{>}$0$\mathrm{{}^\circ}$,

所以\textit{k}${}_{1}$$\mathrm{<}$0$\mathrm{<}$\textit{k}${}_{3}$$\mathrm{<}$\textit{k}${}_{2}$.

答案:D

知识:倾斜角与斜率

难度:1

题目:设直线\textit{l}过坐标原点,它的倾斜角为\textit{$\alpha$},如果将\textit{l}绕坐标原点按逆时针方向旋转45$\mathrm{{}^\circ}$,得到直线\textit{l}${}_{1}$,那么\textit{l}${}_{1}$的倾斜角为(  )

A.\textit{$\alpha$}+45$\mathrm{{}^\circ}$

B.\textit{$\alpha$}-135$\mathrm{{}^\circ}$

C.135$\mathrm{{}^\circ}$-\textit{$\alpha$}

D.当0$\mathrm{{}^\circ}$$\mathrm{\le}$\textit{$\alpha$}$\mathrm{<}$135$\mathrm{{}^\circ}$时,倾斜角为\textit{$\alpha$}+45$\mathrm{{}^\circ}$;当135$\mathrm{{}^\circ}$$\mathrm{\le}$\textit{$\alpha$}$\mathrm{<}$180$\mathrm{{}^\circ}$时,倾斜角为\textit{$\alpha$}-135$\mathrm{{}^\circ}$

解析:根据题意,画出图形,如图所示:

\includegraphics*[width=1.91in, height=0.87in, keepaspectratio=false]{image266}

因为0$\mathrm{{}^\circ}$$\mathrm{\le}$\textit{$\alpha$}$\mathrm{<}$180$\mathrm{{}^\circ}$,显然A,B,C未分类讨论,均不全面,不合题意.通过画图(如图所示)可知:

当0$\mathrm{{}^\circ}$$\mathrm{\le}$\textit{$\alpha$}$\mathrm{<}$135$\mathrm{{}^\circ}$,\textit{l}${}_{1}$的倾斜角为\textit{$\alpha$}+45$\mathrm{{}^\circ}$;

当135$\mathrm{{}^\circ}$$\mathrm{\le}$\textit{$\alpha$}$\mathrm{<}$180$\mathrm{{}^\circ}$时,\textit{l}${}_{1}$的倾斜角为45$\mathrm{{}^\circ}$+\textit{$\alpha$}-180$\mathrm{{}^\circ}$=\textit{$\alpha$}-135$\mathrm{{}^\circ}$.

故选D.

答案:D

知识:倾斜角与斜率

难度:1

题目:经过两点\textit{A}(2,1)、\textit{B}(1,\textit{m}${}^{2}$)的直线\textit{l}的倾斜角为锐角,则\textit{m}的取值范围是(  )

A.\textit{m}<1    B.\textit{m}>-1

C.-1<\textit{m}<1    D.\textit{m}>1或\textit{m}<-1

解析:设直线\textit{l}的倾斜角为\textit{$\alpha$},则

\textit{k${}_{AB}$}=$\frac{m^2-1}{1-2}$=tan\textit{$\alpha$}>0.

$\mathrm{\therefore}$1-\textit{m}${}^{2}$>0,解得-1<\textit{m}<1.

答案:C

知识:倾斜角与斜率

难度:2

题目:已知点\textit{A}(1,3)、\textit{B}(-2,-1).若过点\textit{P}(2,1)的直线\textit{l}与线段\textit{AB}相交,则直线\textit{l}的斜率\textit{k}的取值范围是(  )

A.\textit{k}$\mathrm{\ge}\frac{1}{2}$    B.\textit{k}$\mathrm{\le}$-2

C.\textit{k}$\mathrm{\ge}\frac{1}{2}$或\textit{k}$\mathrm{\le}$-2    D.-2$\mathrm{\le}$\textit{k}$\mathrm{\le}\frac{1}{2}$

解析:过点\textit{P}(2,1)的直线可以看作绕\textit{P}(2,1)进行旋转运动,通过画图可求得\textit{k}的取值范围.由已知直线\textit{l}恒过定点\textit{P}(2,1),如图.

\includegraphics*[width=1.21in, height=1.06in, keepaspectratio=false]{image267}

若\textit{l}与线段\textit{AB}相交,则\textit{k${}_{PA}$}$\mathrm{\le}$\textit{k}$\mathrm{\le}$\textit{k${}_{PB}$},

$\mathrm{\because}$\textit{k${}_{PA}$}=-2,\textit{k${}_{PB}$}=$\frac{1}{2}$,$\mathrm{\therefore}$-2$\mathrm{\le}$\textit{k}$\mathrm{\le}\frac{1}{2}$.

答案:D

知识:倾斜角与斜率

难度:1

题目:设\textit{P}为\textit{x}轴上的一点,\textit{A}(-3,8)、\textit{B}(2,14),若\textit{PA}的斜率是\textit{PB}的斜率的两倍,则点\textit{P}的坐标为\_\_\_\_.

解析:设\textit{P}(\textit{x,}0)为满足题意的点,则\textit{k${}_{PA}$}=$\frac{8}{-3-x}$,\textit{k${}_{PB}$}=$\frac{14}{2-x}$,于是$\frac{8}{-3-x}$=2$\frac{14}{2-x}$,解得\textit{x}=-5.

答案:(-5,0)

知识:倾斜角与斜率

难度:1

题目:直线\textit{l}过点\textit{A}(1,2),且不过第四象限,则直线\textit{l}的斜率的取值范围是\_\_\_\_.

解析:如图,当直线\textit{l}在\textit{l}${}_{1}$位置时,\textit{k}=tan0$\mathrm{{}^\circ}$=0;当直线\textit{l}在\textit{l}${}_{2}$位置时,\textit{k}=$\frac{2-0}{1-0}$=2.故直线\textit{l}的斜率的取值范围是[0,2].

\includegraphics*[width=1.21in, height=0.98in, keepaspectratio=false]{image269}

答案:[0,2]

知识:倾斜角与斜率

难度:1

题目:在同一坐标平面内,画出满足下列条件的直线:

(1)直线\textit{l}${}_{1}$过原点,斜率为1;

(2)直线\textit{l}${}_{2}$过点(3,0),斜率为-$\frac{2}{3}$;

(3)直线\textit{l}${}_{3}$过点(-3,0),斜率为$\frac{2}{3}$;

(4)直线\textit{l}${}_{4}$过点(3,1)斜率不存在.

解析:

答案:如图所示.

\includegraphics*[width=2.82in, height=2.46in, keepaspectratio=false]{image270}

知识:倾斜角与斜率

难度:1

题目:已知两点\textit{A}(-3,4),\textit{B}(3,2),过点\textit{P}(1,0)的直线\textit{l}与线段\textit{AB}有公共点.

(1)求直线\textit{l}的斜率\textit{k}的取值范围;

(2)求直线\textit{l}的倾斜角\textit{$\alpha$}的取值范围.

解析:

答案:如图,由题意可知,直线\textit{PA}的斜率\textit{k${}_{PA}$}=$\frac{4-0}{-3-1}$=-1,直线\textit{PB}的斜率\textit{k${}_{PB}$}=$\frac{2-0}{3-1}$=1,

\includegraphics*[width=1.29in, height=1.16in, keepaspectratio=false]{image271}

(1)要使\textit{l}与线段\textit{AB}有公共点,则直线\textit{l}的斜率\textit{k}的取值范围是\textit{k}$\mathrm{\le}$-1,或\textit{k}$\mathrm{\ge}$1.

(2)由题意可知直线\textit{l}的倾斜角介于直线\textit{PB}与\textit{PA}的倾斜角之间,又直线\textit{PB}的倾斜角是45$\mathrm{{}^\circ}$,直线\textit{PA}的倾斜角是135$\mathrm{{}^\circ}$,

故\textit{$\alpha$}的取值范围是45$\mathrm{{}^\circ}$$\mathrm{\le}$\textit{$\alpha$}$\mathrm{\le}$135$\mathrm{{}^\circ}$.






知识:两条直线平行与垂直的判定

难度:1

题目:(2016·临沧高一检测)直线\textit{l}${}_{1}$、\textit{l}${}_{2}$的斜率是方程\textit{x}${}^{2}$-3\textit{x}-1=0的两根,则\textit{l}${}_{1}$与\textit{l}${}_{2}$的位置关系是(  )

A.平行   B.重合 C.相交但不垂直   D.垂直

解析:设方程\textit{x}${}^{2}$-3\textit{x}-1=0的两根为\textit{x}${}_{1}$、\textit{x}${}_{2}$,则\textit{x}${}_{1}$\textit{x}${}_{2}$=-1.

$\mathrm{\therefore}$直线\textit{l}${}_{1}$、\textit{l}${}_{2}$的斜率\textit{k}${}_{1}$\textit{k}${}_{2}$=-1,

故\textit{l}${}_{1}$与\textit{l}${}_{2}$垂直.

答案:D

知识:两条直线平行与垂直的判定

难度:1

题目:(2016·盐城高一检测)已知直线\textit{l}的倾斜角为20$\mathrm{{}^\circ}$,直线\textit{l}${}_{1}$$\mathrm{\//}$\textit{l},直线\textit{l}${}_{2}$$\mathrm{\bot}$\textit{l},则直线\textit{l}${}_{1}$与\textit{l}${}_{2}$的倾斜角分别是(  )

A.20$\mathrm{{}^\circ}$,20$\mathrm{{}^\circ}$   B.70$\mathrm{{}^\circ}$,70$\mathrm{{}^\circ}$ C.20$\mathrm{{}^\circ}$,110$\mathrm{{}^\circ}$   D.110$\mathrm{{}^\circ}$,20$\mathrm{{}^\circ}$

解析:$\mathrm{\because}$\textit{l}${}_{1}$$\mathrm{\//}$\textit{l},$\mathrm{\therefore}$直线\textit{l}${}_{1}$与\textit{l}的倾斜角相等,

$\mathrm{\therefore}$直线\textit{l}${}_{1}$的倾斜角为20$\mathrm{{}^\circ}$,

又$\mathrm{\because}$\textit{l}${}_{2}$$\mathrm{\bot}$\textit{l},

$\mathrm{\therefore}$直线\textit{l}${}_{2}$的倾斜角为110$\mathrm{{}^\circ}$.

答案:C

知识:两条直线平行与垂直的判定

难度:1

题目:满足下列条件的直线\textit{l}${}_{1}$与\textit{l}${}_{2}$,其中\textit{l}${}_{1}$$\mathrm{\//}$\textit{l}${}_{2}$的是( B )

①\textit{l}${}_{1}$的斜率为2,\textit{l}${}_{2}$过点\textit{A}(1,2)、\textit{B}(4,8);

②\textit{l}${}_{1}$经过点\textit{P}(3,3)、\textit{Q}(-5,3),\textit{l}${}_{2}$平行于\textit{x}轴,但不经过\textit{P}点;

③\textit{l}${}_{1}$经过点\textit{M}(-1,0)、\textit{N}(-5,-2),\textit{l}${}_{2}$经过点\textit{R}(-4,3)、\textit{S}(0,5).

A.①②   B.②③   C.①③   D.①②③

解析:\textit{k${}_{AB}$}=$\frac{8-2}{4-1}$=2,

$\mathrm{\therefore}$\textit{l}${}_{1}$与\textit{l}${}_{2}$平行或重合,故①不正确,排除A、C、D,故选B.

知识:两条直线平行与垂直的判定

难度:1

题目:若过点\textit{A}(2,-2)、\textit{B}(5,0)的直线与过点\textit{P}(2\textit{m,}1)、\textit{Q}(-1,\textit{m})的直线平行,则\textit{m}的值为(  )

A.-1   B.$\frac{1}{7}$   C.2   D.$\frac{1}{2}$

解析:\textit{k${}_{AB}$}=$\frac{0-(-2)}{5-2}$=$\frac{2}{3}$,

$\mathrm{\therefore}$\textit{k${}_{PQ}$}=$\frac{m-1}{-1-2m}$=$\frac{2}{3}$,解得\textit{m}=$\frac{1}{7}$.

答案:B

知识:两条直线平行与垂直的判定

难度:1

题目:已知,过\textit{A}(1,1)、\textit{B}(1,-3)两点的直线与过\textit{C}(-3,\textit{m})、\textit{D}(\textit{n,}2)两点的直线互相垂直,则点(\textit{m},\textit{n})有(  )

A.1个   B.2个 C.3个   D.无数个

解析: $\mathrm{\because}$由条件知过\textit{A}(1,1),\textit{B}(1,-3)两点的直线的斜率不存在,而\textit{AB}$\mathrm{\bot}$\textit{CD},$\mathrm{\therefore}$\textit{k${}_{CD}$}=0,即$\frac{2-m}{n+3}$=0,得\textit{m}=2,\textit{n}$\mathrm{\neq}$-3,$\mathrm{\therefore}$点(\textit{m},\textit{n})有无数个.

答案:D

知识:两条直线平行与垂直的判定

难度:1

题目:以\textit{A}(-1,1)、\textit{B}(2,-1)、\textit{C}(1,4)为顶点的三角形是(  )

A.锐角三角形

B.钝角三角形

C.以\textit{A}点为直角顶点的直角三角形

D.以\textit{B}点为直角顶点的直角三角形

解析:\textit{k${}_{AB}$}=$\frac{-1-1}{2-(-1)}$=-$\frac{2}{3}$,

\textit{k${}_{AC}$}=$\frac{4-1}{1-(-1)}$=$\frac{3}{2}$.

$\mathrm{\therefore}$\textit{k${}_{AB}$}·\textit{k${}_{AC}$}=-$\frac{2}{3}\mathrm{\times}\frac{3}{2}$=-1,

$\mathrm{\therefore}$\textit{AB}$\mathrm{\bot}$\textit{AC},故选C.

答案:C

知识:两条直线平行与垂直的判定

难度:1

题目:已知直线\textit{l}${}_{1}$经过两点(-1,-2),(-1,4),直线\textit{l}${}_{2}$经过两点(2,1)、(6,\textit{y}),且\textit{l}${}_{1}$$\mathrm{\bot}$\textit{l}${}_{2}$,则\textit{y}=( )

A.2   B.-2   C.4   D.1

解析:$\mathrm{\because}$\textit{l}${}_{1}$$\mathrm{\bot}$\textit{l}${}_{2}$且\textit{k}${}_{1}$不存在,$\mathrm{\therefore}$\textit{k}${}_{2}$=0,

$\mathrm{\therefore}$\textit{y}=1.故选D.

答案:D

知识:两条直线平行与垂直的判定

难度:1

题目:已知两点\textit{A}(2,0)、\textit{B}(3,4),直线\textit{l}过点\textit{B},且交\textit{y}轴于点\textit{C}(0,\textit{y}),\textit{O}是坐标原点,且\textit{O}、\textit{A}、\textit{B}、\textit{C}四点共圆,那么\textit{y}的值是(  )

A.19   B.$\frac{19}{4}$   C.5   D.4

解析:由于\textit{A}、\textit{B}、\textit{C}、\textit{O}四点共圆,



所以\textit{AB}$\mathrm{\bot}$\textit{BC},$\mathrm{\therefore}\frac{4-0}{3-2}\cdot\frac{4-y}{3-0}$=-1,$\mathrm{\therefore}$\textit{y}=$\frac{19}{4}$.

故选B.

答案:B


知识:两条直线平行与垂直的判定

难度:1

题目:直线\textit{l}${}_{1}$、\textit{l}${}_{2}$的斜率\textit{k}${}_{1}$、\textit{k}${}_{2}$是关于\textit{k}的方程2\textit{k}${}^{2}$-3\textit{k}-\textit{b}=0的两根,若\textit{l}${}_{1}$$\mathrm{\bot}$\textit{l}${}_{2}$,则\textit{b}=\_\_\_\_;若\textit{l}${}_{1}$$\mathrm{\//}$\textit{l}${}_{2}$,则\textit{b}=\_\_\_\_.

解析:当\textit{l}${}_{1}$$\mathrm{\bot}$\textit{l}${}_{2}$时,\textit{k}${}_{1}$\textit{k}${}_{2}$=-1,

$\mathrm{\therefore}$-$\frac{b}{2}$=-1.$\mathrm{\therefore}$\textit{b}=2.

当\textit{l}${}_{1}$$\mathrm{\//}$\textit{l}${}_{2}$时,\textit{k}${}_{1}$=\textit{k}${}_{2}$,

$\mathrm{\therefore}$$\Delta$=(-3)${}^{2}$+4$\mathrm{\times}$2\textit{b}=0.$\mathrm{\therefore}$\textit{b}=-$\frac{9}{8}$.

答案:2, $-\frac{9}{8}$

知识:两条直线平行与垂直的判定

难度:1

题目:经过点\textit{P}(-2,-1)和点\textit{Q}(3,\textit{a})的直线与倾斜角是45$\mathrm{{}^\circ}$的直线平行,则\textit{a}=\_\_\_\_.

解析:由题意,得tan45$\mathrm{{}^\circ}$=$\frac{a+1}{3+2}$,解得\textit{a}=4.

答案:4

知识:两条直线平行与垂直的判定

难度:1

题目:已知在▱\textit{ABCD}中,\textit{A}(1,2)、\textit{B}(5,0)、\textit{C}(3,4).

(1)求点\textit{D}的坐标;

(2)试判定▱\textit{ABCD}是否为菱形?

解析:

答案:(1)设\textit{D}(\textit{a},\textit{b}),$\mathrm{\because}$四边形\textit{ABCD}为平行四边形,

$\mathrm{\therefore}$\textit{k${}_{AB}$}=\textit{k${}_{CD}$},\textit{k${}_{AD}$}=\textit{k${}_{BC}$},

$\mathrm{\therefore}\left\{ \begin{array}{l} \frac{0-2}{5-1}=\frac{b-4}{a-3}\\ \frac{b-2}{a-1}=\frac{4-0}{3-5} \end{array}\right.$,解得$\left\{ \begin{array}{l} a=-1\\	b=6 \end{array}\right.$.

$\mathrm{\therefore}$\textit{D}(-1,6).

(2)$\mathrm{\because}$\textit{k${}_{AC}$}=$\frac{4-2}{3-1}$=1,\textit{k${}_{BD}$}=$\frac{6-0}{-1-5}$=-1,

$\mathrm{\therefore}$\textit{k${}_{AC}$}·\textit{k${}_{BD}$}=-1.$\mathrm{\therefore}$\textit{AC}$\mathrm{\bot}$\textit{BD}.$\mathrm{\therefore}$▱\textit{ABCD}为菱形.



知识:两条直线平行与垂直的判定

难度:1

题目:$\mathrm{\vartriangle}$\textit{ABC}的顶点\textit{A}(5,-1)、\textit{B}(1,1)、\textit{C}(2,\textit{m}),若$\mathrm{\vartriangle}$\textit{ABC}为直角三角形,求\textit{m}的值.

解析:

答案:(1)若$\mathrm{\angle}$\textit{A}=90$\mathrm{{}^\circ}$,则\textit{AB}$\mathrm{\bot}$\textit{AC},\textit{k${}_{AB}$}·\textit{k${}_{AC}$}=-1,

\textit{k${}_{AB}$}=$\frac{1+1}{1-5}$=-$\frac{1}{2}$,\textit{k${}_{AC}$}=$\frac{m+1}{2-5}$=-$\frac{m+1}{3}$.

$\mathrm{\therefore}$-$\frac{1}{2}\mathrm{\times}$(-$\frac{m+1}{3}$)=-1,$\mathrm{\therefore}$\textit{m}=-7.



(2)若$\mathrm{\angle}$\textit{B}=90$\mathrm{{}^\circ}$,则\textit{BA}$\mathrm{\bot}$\textit{BC},\textit{k${}_{BA}$}·\textit{k${}_{BC}$}=-1,

\textit{k${}_{BC}$}=$\frac{m-1}{2-1}$=\textit{m}-1,\textit{k${}_{BA}$}=-$\frac{1}{2}$,

$\mathrm{\therefore}$(\textit{m}-1)$\mathrm{\times}$(-$\frac{1}{2}$)=1,$\mathrm{\therefore}$\textit{m}=3.

(3)若$\mathrm{\angle}$\textit{C}=90$\mathrm{{}^\circ}$,则\textit{CA}$\mathrm{\bot}$\textit{CB},\textit{k${}_{CA}$}·\textit{k${}_{CB}$}=-1,

\textit{k${}_{CA}$}=$\frac{m+1}{2-5}$=-$\frac{m+1}{3}$,\textit{k${}_{CB}$}=$\frac{m-1}{2-1}$=\textit{m}-1,

\textit{k${}_{CA}$}·\textit{k${}_{CB}$}=-1,$\mathrm{\therefore}$(-$\frac{m+1}{3}$)$\mathrm{\times}$(\textit{m}-1)=-1,

$\mathrm{\therefore}$\textit{m}${}^{2}$=4,$\mathrm{\therefore}$\textit{m}=$\mathrm{\pm}$2.

综上所述,\textit{m}=-2,2,-7,3.


知识:两条直线平行与垂直的判定

难度:2

题目:已知四边形\textit{ABCD}的顶点\textit{A}(\textit{m},\textit{n})、\textit{B}(5,-1)、\textit{C}(4,2)、\textit{D}(2,2),求\textit{m}和\textit{n}的值,使四边形\textit{ABCD}为直角梯形.

解析:

答案:(1)如图,当$\mathrm{\angle}$\textit{A}=$\mathrm{\angle}$\textit{D}=90$\mathrm{{}^\circ}$时,

\includegraphics*[width=1.25in, height=1.10in, keepaspectratio=false]{image273}

$\mathrm{\because}$四边形\textit{ABCD}为直角梯形,

$\mathrm{\therefore}$\textit{AB}$\mathrm{\//}$\textit{DC}且\textit{AD}$\mathrm{\bot}$\textit{AB}.

$\mathrm{\because}$\textit{k${}_{DC}$}=0,$\mathrm{\therefore}$\textit{m}=2,\textit{n}=-1.

(2)如图,当$\mathrm{\angle}$\textit{A}=$\mathrm{\angle}$\textit{B}=90$\mathrm{{}^\circ}$时,

\includegraphics*[width=1.27in, height=1.13in, keepaspectratio=false]{image274}

$\mathrm{\because}$四边形\textit{ABCD}为直角梯形,

$\mathrm{\therefore}$\textit{AD}$\mathrm{\//}$\textit{BC},且\textit{AB}$\mathrm{\bot}$\textit{BC},$\mathrm{\therefore}$\textit{k${}_{AD}$}=\textit{k${}_{BC}$},\textit{k${}_{AB}$k${}_{BC}$}=-1.

$\mathrm{\therefore}\left\{\begin{array}{l} \frac{n-2}{m-2}=\frac{2-(-1)}{4-5},\\ \frac{n+1}{m-5}\cdot\frac{2-(-1)}{4-5}=-1, \end{array}\right.$

解得\textit{m}=$\frac{16}{5}$、\textit{n}=-$\frac{8}{5}$.

综上所述,\textit{m}=2、\textit{n}=-1或\textit{m}=$\frac{16}{5}$、\textit{n}=-$\frac{8}{5}$.


知识:直线的点斜式方程

难度:1

题目:直线\textit{y}=-2\textit{x}-7在\textit{x}轴上的截距为\textit{a},在\textit{y}轴上的截距为\textit{b},则\textit{a}、\textit{b}的值是(  )

A.\textit{a}=-7,\textit{b}=-7   B.\textit{a}=-7,\textit{b}=-$\frac{7}{2}$

C.\textit{a}=-$\frac{7}{2}$,\textit{b}=7    D.\textit{a}=-$\frac{7}{2}$,\textit{b}=-7

解析:令\textit{x}=0,得\textit{y}=-7,即\textit{b}=-7,

令\textit{y}=0,得\textit{x}=-$\frac{7}{2}$,即\textit{a}=-$\frac{7}{2}$.

答案:D

知识:直线的点斜式方程

难度:1

题目:若直线\textit{y}=-\textit{ax}-与直线\textit{y}=3\textit{x}-2垂直,则\textit{a}的值为(  )

A.-3   B.3   C.-$\frac{2}{3}$   D.$\frac{2}{3}$

解析:由题意,得-$\frac{1}{2}$\textit{a}$\mathrm{\times}$3=-1,$\mathrm{\therefore}$\textit{a}=$\frac{2}{3}$.

答案:D

知识:直线的点斜式方程

难度:1

题目:(2016大同高一检测)与直线\textit{y}=2\textit{x}+1垂直,且在\textit{y}轴上的截距为4的直线的斜截式方程为(  )

A.\textit{y}=$\frac{1}{2}$\textit{x}+4   B.\textit{y}=2\textit{x}+4 C.\textit{y}=-2\textit{x}+4   D.\textit{y}=-$\frac{1}{2}$\textit{x}+4

解析:

答案:D

知识:直线的点斜式方程

难度:1

题目:已知两条直线\textit{y}=\textit{ax}-2和\textit{y}=(2-\textit{a})\textit{x}+1互相平行,则\textit{a}等于(  )

A.2   B.1   C.0   D.-1

解析:根据两条直线的方程可以看出它们的斜率分别是\textit{k}${}_{1}$=\textit{a},\textit{k}${}_{2}$=2-\textit{a}.两直线平行,则有\textit{k}${}_{1}$=\textit{k}${}_{2}$.

所以\textit{a}=2-\textit{a},解得\textit{a}=1.

答案:B

知识:直线的点斜式方程

难度:1

题目:\textit{y}=\textit{a}|\textit{x}|(\textit{a}$\mathrm{<}$0)的图象可能是(  )

\includegraphics*[width=2.81in, height=0.93in, keepaspectratio=false]{image276}

解析:$\mathrm{\because}$\textit{a}$\mathrm{<}$0,$\mathrm{\therefore}$\textit{y}$\mathrm{\le}$0,其图象在\textit{x}轴下方,故选D.

答案:D

知识:直线的点斜式方程

难度:1

题目:(2016·天水高一检测)直线\textit{y}=\textit{kx}+\textit{b}通过第一、三、四象限,则有(  )

A.\textit{k}$\mathrm{>}$0,\textit{b}$\mathrm{>}$0   B.\textit{k}$\mathrm{>}$0,\textit{b}$\mathrm{<}$0 C.\textit{k}$\mathrm{<}$0,\textit{b}$\mathrm{>}$0   D.\textit{k}$\mathrm{<}$0,\textit{b}$\mathrm{<}$0

解析:如图,

\includegraphics*[width=1.10in, height=0.71in, keepaspectratio=false]{image277}

由图可知,\textit{k}$\mathrm{>}$0,\textit{b}$\mathrm{<}$0.

答案:B

知识:直线的点斜式方程

难度:1

题目:方程\textit{y}=\textit{ax}+$\frac{1}{a}$表示的直线可能是( B )

\includegraphics*[width=2.22in, height=2.13in, keepaspectratio=false]{image278}

解析:直线\textit{y}=\textit{ax}+$\frac{1}{a}$的斜率是\textit{a},在\textit{y}轴上的截距是$\frac{1}{a}$.当\textit{a}$\mathrm{>}$0时,斜率\textit{a}$\mathrm{>}$0,在\textit{y}轴上的截距是$\frac{1}{a}\mathrm{>}$0,则直线\textit{y}=\textit{ax}+$\frac{1}{a}$过第一、二、三象限,四个选项都不符合;当\textit{a}$\mathrm{<}$0时,斜率\textit{a}$\mathrm{<}$0,在\textit{y}轴上的截距是$\mathrm{<}$0,则直线\textit{y}=\textit{ax}+$\frac{1}{a}$过第二、三、四象限,仅有选项B符合.

答案:B

知识:直线的点斜式方程

难度:1

题目:(2016~2017合肥高一检测)下列四个结论:

①方程\textit{k}=与方程\textit{y}-2=\textit{k}(\textit{x}+1)可表示同一直线;

②直线\textit{l}过点\textit{P}(\textit{x}${}_{1}$,\textit{y}${}_{1}$),倾斜角为,则其方程为\textit{x}=\textit{x}${}_{1}$;

③直线\textit{l}过点\textit{P}(\textit{x}${}_{1}$,\textit{y}${}_{1}$),斜率为0,则其方程为\textit{y}=\textit{y}${}_{1}$;

④所有直线都有点斜式和斜截式方程.

其中正确的个数为( B )

A.1   B.2   C.3   D.4

解析:①④不正确,②③正确,故选B.

答案:B 

知识:直线的点斜式方程

难度:1

题目:已知点(1,-4)和(-1,0)是直线\textit{y}=\textit{kx}+\textit{b}上的两点,则\textit{k}=\_\_\_\_,\textit{b}=\_\_\_\_.

解析:由题意,得$\left\{\begin{array}{l} -4=k+b\\ 0=-k+b \end{array}\right.$,解得\textit{k}=-2,\textit{b}=-2.

答案:\textit{k}=-2,\textit{b}=-2.

知识:直线的点斜式方程

难度:1

题目:(2016·杭州高一检测)直线\textit{l}${}_{1}$与直线\textit{l}${}_{2}$:\textit{y}=3\textit{x}+1平行,又直线\textit{l}${}_{1}$过点(3,5),则直线\textit{l}${}_{1}$的方程为\_\_\_\_.

解析:$\mathrm{\because}$直线\textit{l}${}_{2}$的斜率\textit{k}${}_{2}$=3,\textit{l}${}_{1}$与\textit{l}${}_{2}$平行.

$\mathrm{\therefore}$直线\textit{l}${}_{1}$的斜率\textit{k}${}_{1}$=3.

又直线\textit{l}${}_{1}$过点(3,5),

$\mathrm{\therefore}$\textit{l}${}_{1}$的方程为\textit{y}-5=3(\textit{x}-3),即\textit{y}=3\textit{x}-4.

答案:$y=3x-4$

知识:直线的点斜式方程

难度:1

题目:(2016~2017·福州高一检测)直线\textit{l}过点\textit{P}(2,-3)且与过点\textit{M}(-1,2),\textit{N}(5,2)的直线垂直,求直线\textit{l}的方程.

解析:

答案:过\textit{M},\textit{N}两点的直线斜率\textit{k}=0,

$\mathrm{\therefore}$直线\textit{l}与直线\textit{MN}垂直,

$\mathrm{\therefore}$直线\textit{l}的斜率不存在.

又直线\textit{l}过点\textit{P}(2,-3),

$\mathrm{\therefore}$直线\textit{l}的方程为\textit{x}=2.


知识:直线的点斜式方程

难度:1

题目:已知直线\textit{y}=-$\frac{\sqrt{3}}{3}$\textit{x}+5的倾斜角是直线\textit{l}的倾斜角的大小的5倍,分别求满足下列条件的直线\textit{l}的方程.

(1)过点\textit{P}(3,-4);

(2)在\textit{x}轴上截距为-2;

(3)在\textit{y}轴上截距为3.

解析:

答案:直线\textit{y}=-$\frac{\sqrt{3}}{3}$\textit{x}+5的斜率\textit{k}=tan\textit{$\alpha$}=-$\frac{\sqrt{3}}{3}$,

$\mathrm{\therefore}$\textit{$\alpha$}=150$\mathrm{{}^\circ}$,

故所求直线\textit{l}的倾斜角为30$\mathrm{{}^\circ}$,斜率\textit{k}$'$=$\frac{\sqrt{3}}{3}$.

(1)过点\textit{P}(3,-4),由点斜式方程得:

\textit{y}+4=$\frac{\sqrt{3}}{3}$(\textit{x}-3),

$\mathrm{\therefore}$\textit{y}=$\frac{\sqrt{3}}{3}$\textit{x}-$\sqrt{3}$-4.

(2)在\textit{x}轴截距为-2,即直线\textit{l}过点(-2,0),

由点斜式方程得:\textit{y}-0=$\frac{\sqrt{3}}{3}$(\textit{x}+2),$\mathrm{\therefore}$\textit{y}=$\frac{\sqrt{3}}{3}$\textit{x}+$\frac{2\sqrt{3}}{3}$.

(3)在\textit{y}轴上截距为3,由斜截式方程得\textit{y}=$\frac{\sqrt{3}}{3}$\textit{x}+3.


知识:直线的点斜式方程

难度:3

题目:求与直线\textit{y}=$\frac{4}{3}$\textit{x}+$\frac{5}{3}$垂直,并且与两坐标轴围成的三角形面积为24的直线\textit{l}的方程.

解析:

答案:由直线\textit{l}与直线\textit{y}=$\frac{4}{3}$\textit{x}+$\frac{5}{3}$垂直,可设直线\textit{l}的方程为\textit{y}=-$\frac{3}{4}$\textit{x}+\textit{b},

则直线\textit{l}在\textit{x}轴,\textit{y}轴上的截距分别为\textit{x}${}_{0}$=$\frac{4}{3}$\textit{b},\textit{y}${}_{0}$=\textit{b}.

又因为直线\textit{l}与两坐标轴围成的三角形的面积为24,

所以\textit{S}=$\frac{1}{2}$|\textit{x}${}_{0}$||\textit{y}${}_{0}$|=24,

即$\frac{1}{2}$|$\frac{4}{3}$\textit{b}||\textit{b}|=24,\textit{b}${}^{2}$=36,

解得\textit{b}=6,或\textit{b}=-6.

故所求的直线方程为\textit{y}=-$\frac{3}{4}$\textit{x}+6,或\textit{y}=-$\frac{3}{4}$\textit{x}-6.

答案:直线方程为\textit{y}=-$\frac{3}{4}$\textit{x}+6,或\textit{y}=-$\frac{3}{4}$\textit{x}-6

知识:直线的两点式方程

难度:1

题目:直线$\frac{x}{2}-\frac{y}{5}$=1在\textit{x}轴、\textit{y}轴上的截距分别为(  )

A.2,5   B.2,-5 C.-2,-5   D.-2,5

解析:将$\frac{x}{2}-\frac{y}{5}$=1化成直线截距式的标准形式为$\frac{x}{2}+\frac{y}{-5}$=1,故直线$\frac{x}{2}-\frac{y}{5}$=1在\textit{x}轴、\textit{y}轴上的截距分别为2、-5.

答案:B

知识:直线的两点式方程

难度:1

题目:已知点\textit{M}(1,-2)、\textit{N}(\textit{m,}2),若线段\textit{MN}的垂直平分线的方程是$\frac{x}{2}$+\textit{y}=1,则实数\textit{m}的值是(  )

A.-2   B.-7 C.3   D.1

解析:由中点坐标公式,得线段\textit{MN}的中点是($\frac{1+m}{2}$,0).又点($\frac{1+m}{2}$,0)在线段\textit{MN}的垂直平分线上,所以$\frac{1+m}{4}$+0=1,所以\textit{m}=3,选C.

答案:C

知识:直线的两点式方程

难度:1

题目:如右图所示,直线\textit{l}的截距式方程是$\frac{x}{a}$+$\frac{y}{b}$=1,则有(  )

\includegraphics*[width=0.83in, height=0.90in, keepaspectratio=false]{image280}

A.\textit{a}$\mathrm{>}$0,\textit{b}$\mathrm{>}$0 B.\textit{a}$\mathrm{>}$0,\textit{b}$\mathrm{<}$0 C.\textit{a}$\mathrm{<}$0,\textit{b}$\mathrm{>}$0 D.\textit{a}$\mathrm{<}$0,\textit{b}$\mathrm{<}$0

解析:很明显\textit{M}(\textit{a,}0)、\textit{N}(0,\textit{b}),由图知\textit{M}在\textit{x}轴正半轴上,\textit{N}在\textit{y}轴负半轴上,则\textit{a}$\mathrm{>}$0,\textit{b}$\mathrm{<}$0.

答案:B

知识:直线的两点式方程

难度:1

题目:已知$\mathrm{\vartriangle}$\textit{ABC}三顶点\textit{A}(1,2)、\textit{B}(3,6)、\textit{C}(5,2),\textit{M}为\textit{AB}中点,\textit{N}为\textit{AC}中点,则中位线\textit{MN}所在直线方程为( A )

A.2\textit{x}+\textit{y}-8=0    B.2\textit{x}-\textit{y}+8=0 

C.2\textit{x}+\textit{y}-12=0    D.2\textit{x}-\textit{y}-12=0

解析:点\textit{M}的坐标为(2,4),点\textit{N}的坐标为(3,2),由两点式方程得$\frac{y-2}{4-2}=\frac{x-3}{2-3}$,即2\textit{x}+\textit{y}-8=0.

答案:A

知识:直线的两点式方程

难度:1

题目:如果直线\textit{l}过(-1,-1)、(2,5)两点,点(1 008,\textit{b})在直线\textit{l}上,那么\textit{b}的值为(  )

A.2 014   B.2 015 C.2 016   D.2 017

解析:根据三点共线,得$\frac{5-(-1)}{2-(-1)}=\frac{b-5}{1008-2}$,得\textit{b}=2 017.

答案:D

知识:直线的两点式方程

难度:1

题目:两直线$\frac{x}{m}-\frac{y}{n}$=1与$\frac{x}{n}-\frac{y}{m}$=1的图象可能是图中的哪一个(  )

\includegraphics*[width=2.81in, height=0.85in, keepaspectratio=false]{image281}

解析:直线$\frac{x}{m}-\frac{y}{n}$=1化为

\textit{y}=$\frac{n}{m}$\textit{x}-\textit{n},直线$\frac{x}{n}-\frac{y}{m}$=1化为

\textit{y}=$\frac{m}{n}$\textit{x}-\textit{m},故两直线的斜率同号,故选B.

答案:B

知识:直线的两点式方程

难度:2

题目:已知\textit{A}、\textit{B}两点分别在两条互相垂直的直线\textit{y}=2\textit{x}和\textit{x}+\textit{ay}=0上,且线段\textit{AB}的中点为\textit{P}(0,$\frac{10}{a}$),则直线\textit{AB}的方程为(  )

A.\textit{y}=-$\frac{3}{4}$\textit{x}+5    B.\textit{y}=$\frac{3}{4}$\textit{x}-5

C.\textit{y}=$\frac{3}{4}$\textit{x}+5    D.\textit{y}=-$\frac{3}{4}$\textit{x}-5

解析:依题意,\textit{a}=2,\textit{P}(0,5).设\textit{A}(\textit{x}${}_{0,}$2\textit{x}${}_{0}$)、\textit{B}(-2\textit{y}${}_{0}$,\textit{y}${}_{0}$),则由中点坐标公式,得$\left\{\begin{array}{l} x_0-2y_0=0\\ 2x_0+y_0=10 \end{array}\right.$,解得$\left\{\begin{array}{l} x_0=4\\ y_0=2 \end{array}\right.$,所以\textit{A}(4,8)、\textit{B}(-4,2).由直线的两点式方程,得直线\textit{AB}的方程是$\frac{y-8}{2-8}=\frac{x-4}{-4-4}$,即\textit{y}=$\frac{3}{4}$\textit{x}+5,选C.

答案:C

知识:直线的两点式方程

难度:2

题目:过\textit{P}(4,-3)且在坐标轴上截距相等的直线有(  )

A.1条   B.2条 C.3条   D.4条

解析:解法一:设直线方程为\textit{y}+3=\textit{k}(\textit{x}-4)(\textit{k}$\mathrm{\neq}$0).

令\textit{y}=0得\textit{x}=$\frac{3+4k}{k}$,令\textit{x}=0得\textit{y}=-4\textit{k}-3.

由题意,$\frac{3+4k}{k}$=-4\textit{k}-3,解得\textit{k}=-$\frac{3}{4}$或\textit{k}=-1.

因而所求直线有两条,$\mathrm{\therefore}$应选B.

解法二:当直线过原点时显然符合条件,当直线不过原点时,设直线在坐标轴上截距为(\textit{a,}0),(0,\textit{a}),\textit{a}$\mathrm{\neq}$0,则直线方程为$\frac{x}{a}+\frac{y}{a}$=1,把点\textit{P}(4,-3)的坐标代入方程得\textit{a}=1.

$\mathrm{\therefore}$所求直线有两条,$\mathrm{\therefore}$应选B.

答案:B

知识:直线的两点式方程

难度:1

题目:已知点\textit{P}(-1,2\textit{m}-1)在经过\textit{M}(2,-1)、\textit{N}(-3,4)两点的直线上,则\textit{m}=\_\_\_\_.

[解析] 解法一:\textit{MN}的直线方程为:$\frac{y+1}{4+1}=\frac{x-2}{-3-2}$,即\textit{x}+\textit{y}-1=0,

代入\textit{P}(-1,2\textit{m}-1)得\textit{m}=$\frac{3}{2}$.

解法二:\textit{M}、\textit{N}、\textit{P}三点共线,

$\mathrm{\therefore}\frac{4-2m-1}{-3+1}=\frac{4-(-1)}{-3-2}$,解得\textit{m}=$\frac{3}{2}$.

答案:$\frac{3}{2}$

知识:直线的两点式方程

难度:1

题目:(2016~2017·衡水高一检测)已知直线\textit{l}的斜率为6,且在两坐标轴上的截距之和为10,则此直线\textit{l}的方程为\_\_\_\_.

解析:设\textit{l}:\textit{y}=6\textit{x}+\textit{b},令\textit{y}=0得\textit{x}=-$\frac{b}{6}$.

由条件知\textit{b}+$(-\frac{b}{6})$=10,$\mathrm{\therefore}$\textit{b}=12.

$\mathrm{\therefore}$直线\textit{l}方程为\textit{y}=6\textit{x}+12.

解法2:设直线\textit{l}:$\frac{x}{a}+\frac{y}{b}$=1,变形为\textit{y}=-$\frac{b}{a}$\textit{x}+\textit{b}.

由条件知$\left\{\begin{array}{l} -\frac{b}{a}=6,\\ a+b=10, \end{array}\right.$解得$\left\{\begin{array}{l} b=12,\\ a=-2 \end{array}\right.$.

$\mathrm{\therefore}$直线\textit{l}方程为$\frac{x}{-2}+\frac{y}{12}$=1.即6\textit{x}-\textit{y}+12=0.

答案:$6x-y+12=0$

知识:直线的两点式方程

难度:2

题目:求分别满足下列条件的直线\textit{l}的方程:

(1)斜率是,且与两坐标轴围成的三角形的面积是6;

(2)经过两点\textit{A}(1,0)、\textit{B}(\textit{m,}1);

(3)经过点(4,-3),且在两坐标轴上的截距的绝对值相等.

解析:

答案:(1)设直线\textit{l}的方程为\textit{y}=$\frac{3}{4}$\textit{x}+\textit{b}.

令\textit{y}=0,得\textit{x}=-$\frac{4}{3}$\textit{b},

$\mathrm{\therefore}\frac{1}{2}$|\textit{b}·(-$\frac{4}{3}$\textit{b})|=6,\textit{b}=$\mathrm{\pm}$3.

$\mathrm{\therefore}$直线\textit{l}的方程为\textit{y}=$\frac{4}{3}$\textit{x}$\mathrm{\pm}$3.

(2)当\textit{m}$\mathrm{\neq}$1时,直线\textit{l}的方程是

$\frac{y-0}{1-0}=\frac{x-1}{m-1}$,即\textit{y}=$\frac{1}{m-1}$(\textit{x}-1)

当\textit{m}=1时,直线\textit{l}的方程是\textit{x}=1.

(3)设\textit{l}在\textit{x}轴、\textit{y}轴上的截距分别为\textit{a}、\textit{b}.

当\textit{a}$\mathrm{\neq}$0,\textit{b}$\mathrm{\neq}$0时,\textit{l}的方程为$\frac{x}{a}+\frac{y}{b}=1$;

$\mathrm{\because}$直线过\textit{P}(4,-3),$\mathrm{\therefore}\frac{4}{a}-\frac{3}{b}$1.

又$\mathrm{\because}$|\textit{a}|=|\textit{b}|,

$\mathrm{\therefore}\left\{\begin{array}{l} \frac{4}{a}-\frac{3}{b}=1\\ a=\pm b  \end{array}\right.$,解得$\left\{\begin{array}{l} a=1\\ b=1 \end{array}\right.$,或$\left\{\begin{array}{l} a=7\\ b=-7 \end{array}\right.$.

当\textit{a}=\textit{b}=0时,直线过原点且过(4,-3),

$\mathrm{\therefore}$\textit{l}的方程为\textit{y}=-$\frac{3}{4}$\textit{x}.

综上所述,直线\textit{l}的方程为\textit{x}+\textit{y}=1或$\frac{x}{7}+\frac{y}{-7}$=1或\textit{y}=-$\frac{3}{4}$\textit{x}.

知识:直线的两点式方程

难度:2

题目:$\mathrm{\vartriangle}$\textit{ABC}的三个顶点分别为\textit{A}(0,4)、\textit{B}(-2,6)、\textit{C}(-8,0).

(1)分别求边\textit{AC}和\textit{AB}所在直线的方程;

(2)求\textit{AC}边上的中线\textit{BD}所在直线的方程;

(3)求\textit{AC}边的中垂线所在直线的方程;

(4)求\textit{AC}边上的高所在直线的方程;

(5)求经过两边\textit{AB}和\textit{AC}的中点的直线方程.

解析:

答案:(1)由\textit{A}(0,4),\textit{C}(-8,0)可得直线\textit{AC}的截距式方程为$\frac{x}{-8}+\frac{y}{4}$=1,

即\textit{x}-2\textit{y}+8=0.

由\textit{A}(0,4),\textit{B}(-2,6)可得直线\textit{AB}的两点式方程为$\frac{y-4}{6-4}=\frac{x-0}{-2-0}$,即\textit{x}+\textit{y}-4=0.

(2)设\textit{AC}边的中点为\textit{D}(\textit{x},\textit{y}),由中点坐标公式可得\textit{x}=-4,\textit{y}=2,所以直线\textit{BD}的两点式方程为$\frac{y-6}{2-6}=\frac{x+2}{-4+2}$,即2\textit{x}-\textit{y}+10=0.

(3)由直线\textit{AC}的斜率为\textit{k${}_{AC}$}=$\frac{4-0}{0+8}$=$\frac{1}{2}$,故\textit{AC}边的中垂线的斜率为\textit{k}=-2.又\textit{AC}的中点\textit{D}(-4,2),

所以\textit{AC}边的中垂线方程为\textit{y}-2=-2(\textit{x}+4),

即2\textit{x}+\textit{y}+6=0.

(4)\textit{AC}边上的高线的斜率为-2,且过点\textit{B}(-2,6),所以其点斜式方程为\textit{y}-6=-2(\textit{x}+2),即2\textit{x}+\textit{y}-2=0.

(5)\textit{AB}的中点\textit{M}(-1,5),\textit{AC}的中点\textit{D}(-4,2),

$\mathrm{\therefore}$直线\textit{DM}方程为$\frac{y-2}{5-2}\frac{x-(-4)}{-1-(-4)}$,

即\textit{x}-\textit{y}+6=0.

知识:直线的两点式方程

难度:2

题目:已知抛物线\textit{y}=-\textit{x}${}^{2}$-2\textit{x}+3与\textit{x}轴交于\textit{A}、\textit{B}两点,点\textit{M}在此抛物线上,点\textit{N}在\textit{y}轴上,以\textit{A}、\textit{B}、\textit{M}、\textit{N}为顶点的四边形为平行四边形,求点\textit{M}的坐标.

解析:

答案:容易求得抛物线与\textit{x}轴的交点分别为(-3,0)、(1,0)不妨设\textit{A}(-3,0)、\textit{B}(1,0),由已知,设\textit{M}(\textit{a},\textit{b})、\textit{N}(0,\textit{n}),

根据平行四边形两条对角线互相平分的性质,可得两条对角线的中点重合.

按\textit{A}、\textit{B}、\textit{M}、\textit{N}两两连接的线段分别作为平行四边形的对角线进行分类,有以下三种情况:

①若以\textit{AB}为对角线,可得\textit{a}+0=-3+1,解得\textit{a}=-2;

②若以\textit{AN}为对角线,可得\textit{a}+1=-3+0,解得\textit{a}=-4;

③若以\textit{BN}为对角线,可得\textit{a}+(-3)=1+0,解得\textit{a}=4.

因为点\textit{M}在抛物线上,将其横坐标的值分别代入抛物线的解析式,可得\textit{M}(-2,3)或\textit{M}(-4,-5)或\textit{M}(4,-21).

知识:直线方程的一般式

难度:1

题目:(2016·南安一中高一检测)直线\textit{x}-\textit{y}+2=0的倾斜角是(  )

A.30$\mathrm{{}^\circ}$   B.45$\mathrm{{}^\circ}$   C.60$\mathrm{{}^\circ}$   D.90

解析:由\textit{x}-\textit{y}+2=0,得\textit{y}=\textit{x}+2.

其斜率为1,倾斜角为45$\mathrm{{}^\circ}$.

答案:B

知识:直线方程的一般式

难度:1

题目:(2016·葫芦岛高一检测)已知直线\textit{l}${}_{1}$:\textit{x}+2\textit{y}-1=0与直线\textit{l}${}_{2}$:\textit{mx}-\textit{y}=0平行,则实数\textit{m}的值为(  )

A.-$\frac{1}{2}$   B.$\frac{1}{2}$   C.2   D.-2

解析:$\mathrm{\because}$\textit{l}${}_{1}$$\mathrm{\//}$\textit{l}${}_{2}$,$\mathrm{\therefore}$1$\mathrm{\times}$(-1)-2\textit{m}=0,

$\mathrm{\therefore}$\textit{m}=-$\frac{1}{2}$.

答案:A

3.直线3\textit{x}-2\textit{y}-4=0在\textit{x}轴、\textit{y}轴上的截距分别是(  )

A.$\frac{3}{4}$,-$\frac{1}{2}$    B.$\frac{1}{3}$,$\frac{1}{2}$

C.$\frac{3}{4}$,-2    D.$\frac{4}{3}$,-2

解析:将3\textit{x}-2\textit{y}-4=0化成截距式为$\frac{x}{\frac{4}{3}}+\frac{y}{-2}=1$,故该直线在\textit{x}轴、\textit{y}轴上的截距分别是$\frac{4}{3}$,-2.

答案:D

知识:直线方程的一般式

难度:1

题目:若直线\textit{ax}+2\textit{y}+1=0与直线\textit{x}+\textit{y}-2=0互相垂直,则\textit{a}的值为(  )

A.1   B.-$\frac{1}{3}$   C.-$\frac{2}{3}$   D.-2

解析:由题意,得(-$\frac{a}{2}$)$\mathrm{\times}$(-1)=-1,\textit{a}=-2.

答案:D

知识:直线方程的一般式

难度:1

题目:直线\textit{l}垂直于直线\textit{y}=\textit{x}+1,且\textit{l}在\textit{y}轴上的截距为,则直线\textit{l}的方程是(  )

A.\textit{x}+\textit{y}-$\sqrt{2}$=0    B.\textit{x}+\textit{y}+1=0

C.\textit{x}+\textit{y}-1=0    D.\textit{x}+\textit{y}+$\sqrt{2}$=0

解析: 解法一:因为直线\textit{l}与直线\textit{y}=\textit{x}+1垂直,所以设直线\textit{l}的方程为\textit{y}=-\textit{x}+\textit{b},又\textit{l}在\textit{y}轴上截距为$\sqrt{2}$,所以所求直线\textit{l}的方程为\textit{y}=-\textit{x}+$\sqrt{2}$,即\textit{x}+\textit{y}-$\sqrt{2}$=0.

解法二:将直线\textit{y}=\textit{x}+1化为一般式\textit{x}-\textit{y}+1=0,因为直线\textit{l}垂直于直线\textit{y}=\textit{x}+1,可以设直线\textit{l}的方程为\textit{x}+\textit{y}+\textit{c}=0,令\textit{x}=0,得\textit{y}=-\textit{c},又直线\textit{l}在\textit{y}轴上截距为$\sqrt{2}$,所以-\textit{c}=$\sqrt{2}$,即\textit{c}=-$\sqrt{2}$,所以直线\textit{l}的方程为\textit{x}+\textit{y}-$\sqrt{2}$=0.

答案:A

知识:直线方程的一般式

难度:1

题目:直线\textit{l}:(\textit{k}+1)\textit{x}-(\textit{k}-1)\textit{y}-2\textit{k}=0恒过定点(  )

A.(-1,1)    B.(1,-1)

C.(-1,-1)    D.(1,1)

解析:由(\textit{k}+1)\textit{x}-(\textit{k}-1)\textit{y}-2\textit{k}=0,得\textit{k}(\textit{x}-\textit{y}-2)+\textit{x}+\textit{y}=0,

由$\left\{\begin{array}{l} x-y-2=0\\ x+y=0 \end{array}\right.$,得$\left\{\begin{array}{l} x=1\\ y=-1 \end{array}\right.$.

$\mathrm{\therefore}$直线\textit{l}过定点(1,-1).

答案:B

知识:直线方程的一般式

难度:1

题目:若直线\textit{l}${}_{1}$:2\textit{x}+(\textit{m}+1)\textit{y}+4=0与直线\textit{l}${}_{2}$:\textit{mx}+3\textit{y}-2=0平行,则\textit{m}的值为\_\_\_\_.

解析:若\textit{m}=-1,则\textit{l}${}_{1}$的斜率不存在,\textit{l}${}_{2}$的斜率为$\frac{1}{3}$,此时\textit{l}${}_{1}$与\textit{l}${}_{2}$不平行;若\textit{m}$\mathrm{\neq}$-1,则\textit{l}${}_{1}$的斜率为\textit{k}${}_{1}$=-$\frac{2}{m+1}$,\textit{l}${}_{2}$的斜率为\textit{k}${}_{2}$=-$\frac{m}{3}$.因为\textit{l}${}_{1}$$\mathrm{\//}$\textit{l}${}_{2}$,所以\textit{k}${}_{1}$=\textit{k}${}_{2}$,即-$\frac{2}{m+1}$=-$\frac{m}{3}$,解得\textit{m}=2或-3.经检验均符合题意.

答案:2或-3

知识:直线方程的一般式

难度:1

题目:若直线(2\textit{t}-3)\textit{x}+\textit{y}+6=0不经过第一象限,则\textit{t}的取值范围是\_\_\_\_.

解析:直线方程可化为\textit{y}=(3-2\textit{t})\textit{x}-6,$\mathrm{\therefore}$3-2\textit{t}$\mathrm{\le}$0,$\mathrm{\therefore}$\textit{t}$\mathrm{\ge}\frac{3}{2}$.

答案:$[\frac{3}{2}, +\infty)$

知识:直线方程的一般式

难度:1

题目:求与直线3\textit{x}-4\textit{y}+7=0平行,且在两坐标轴上截距之和为1的直线\textit{l}的方程.

解析:

答案:解法一:由题意知:可设\textit{l}的方程为3\textit{x}-4\textit{y}+\textit{m}=0,

则\textit{l}在\textit{x}轴、\textit{y}轴上的截距分别为$-\frac{m}{3}$,$\frac{m}{4}$.

由$-\frac{m}{3}+\frac{m}{4}=$1知,\textit{m}=-12.

$\mathrm{\therefore}$直线\textit{l}的方程为:3\textit{x}-4\textit{y}-12=0.

解法二:设直线方程为$\frac{x}{a}+\frac{y}{b}=$1,

由题意得$\left\{\begin{array}{l} a+b=1\\ -\frac{b}{a}=\frac{4}{3} \end{array}\right.$ 解得$\left\{\begin{array}{l} a=4\\ b=-3 \end{array}\right.$.

$\mathrm{\therefore}$直线\textit{l}的方程为:$\frac{x}{4}+\frac{y}{-3}$=1.

即3\textit{x}-4\textit{y}-12=0.

知识:直线方程的一般式

难度:1

题目:设直线\textit{l}的方程为(\textit{m}${}^{2}$-2\textit{m}-3)\textit{x}+(2\textit{m}${}^{2}$+\textit{m}-1)\textit{y}=2\textit{m}-6,根据下列条件分别确定实数\textit{m}的值.

(1)\textit{l}在\textit{x}轴上的截距为-3;

(2)斜率为1.

解析:

答案:(1)令\textit{y}=0,依题意得

$\left\{\begin{array}{l} m^2-2m-3\neq 0 \text{①}\\ \frac{2m-6}{m^2-2m-3}=-3\text{②} \end{array}\right.$

由①得\textit{m}$\mathrm{\neq}$3且\textit{m}$\mathrm{\neq}$-1;

由②得3\textit{m}${}^{2}$-4\textit{m}-15=0,

解得\textit{m}=3或\textit{m}=-$\frac{5}{3}$.

综上所述,\textit{m}=-$\frac{5}{3}$

(2)由题意得$\left\{\begin{array}{l} 2m^2+m-1\neq 0 \text{③}\\ \frac{-m^2-2m-3}{2m^2+m-1}=1\text{④} \end{array}\right.$,

由③得\textit{m}$\mathrm{\neq}$-1且\textit{m}$\mathrm{\neq}\frac{1}{2}$,

解④得\textit{m}=-1或$\frac{4}{3}$,

$\mathrm{\therefore}$\textit{m}=$\frac{4}{3}$.

知识:直线方程的一般式

难度:2

题目:(2016~2017·西宁高一检测)若直线\textit{l}:\textit{ax}+\textit{y}-2-\textit{a}=0在\textit{x}轴和\textit{y}轴上的截距相等,则直线\textit{l}的斜率为(  )

A.1   B.-1 C.-2或1   D.-1或2

解析:在方程\textit{ax}+\textit{y}-2-\textit{a}=0中,令\textit{x}=0得\textit{y}=2+\textit{a},令\textit{y}=0得,\textit{x}=$\frac{a+2}{a}$(\textit{a}$\mathrm{\neq}$0).

$\mathrm{\therefore}$2+\textit{a}=$\frac{a+2}{a}$,$\mathrm{\therefore}$\textit{a}=-2或1.

当\textit{a}=-2时,\textit{l}的斜率\textit{k}=2;

当\textit{a}=1时,\textit{l}的斜率\textit{k}=-1.

故选D.

答案:D

知识:直线方程的一般式

难度:2

题目:直线\textit{ax}+\textit{by}-1=0(\textit{ab}$\mathrm{\neq}$0)与两坐标轴围成的三角形的面积是(  )

A.$\frac{1}{2}$\textit{ab}   B.$\frac{1}{2}$|\textit{ab}| C.$\frac{1}{2ab}$   D.$\frac{1}{2|ab|}$

解析:$\mathrm{\because}$\textit{ab}$\mathrm{\neq}$0,$\mathrm{\therefore}$令\textit{y}=0,得\textit{x}=$\frac{1}{a}$,

令\textit{x}=0,得\textit{y}=$\frac{1}{b}$,

$\mathrm{\therefore}$三角形的面积\textit{S}=$\frac{1}{2}\cdot\frac{1}{|a|}\cdot\frac{1}{|b|}=\frac{1}{2|ab|}$.

答案:D

知识:直线方程的一般式

难度:2

题目:方程\textit{y}=\textit{k}(\textit{x}+4)表示(  )

A.过点(-4,0)的一切直线

B.过点(4,0)的一切直线

C.过点(-4,0)且不垂直于\textit{x}轴的一切直线

D.过点(-4,0)且不平行于\textit{x}轴的一切直线

解析:方程\textit{y}=\textit{k}(\textit{x}+4)表示过点(-4,0)且斜率存在的直线,故选C.

答案:C

知识:直线方程的一般式

难度:2

题目:两直线\textit{mx}+\textit{y}-\textit{n}=0与\textit{x}+\textit{my}+1=0互相平行的条件是(  )

A.\textit{m}=1    B.\textit{m}=$\mathrm{\pm}$1

C.$\left\{\begin{array}{l} m=1\\ n\neq -1 \end{array}\right.$    D.$\left\{\begin{array}{l} m=1\\ n\neq -1 \end{array}\right.$或$\left\{\begin{array}{l} m=-1\\ n\neq 1 \end{array}\right.$

解析:根据两直线平行可得$\frac{m}{1}=\frac{1}{m}$,所以\textit{m}=$\mathrm{\pm}$1,又两直线不可重合,所以\textit{m}=1时,\textit{n}$\mathrm{\neq}$-1;\textit{m}=-1时,\textit{n}$\mathrm{\neq}$1.

答案:D

知识:直线方程的一般式

难度:2

题目:(2016~2017·合肥高一检测)已知直线\textit{l}与直线3\textit{x}+4\textit{y}-7=0平行,并且与两坐标轴围成的三角形的面积为24,则直线\textit{l}的方程为\_\_\_\_.

解析:设直线\textit{l}方程为3\textit{x}+4\textit{y}+\textit{b}=0,

令\textit{x}=0得\textit{y}=-$\frac{b}{4}$;

令\textit{y}=0得\textit{x}=-$\frac{b}{3}$.

由条件知$\frac{1}{2}\cdot|-\frac{b}{4}|\cdot|-\frac{b}{3}|$=24.

解之得\textit{b}=$\mathrm{\pm}$24.

$\mathrm{\therefore}$直线\textit{l}方程为3\textit{x}+\textit{y}$\mathrm{\pm}$24=0.

答案:$3x+4y\pm 24=0$

知识:直线方程的一般式

难度:2

题目:若直线(\textit{m}+1)\textit{x}+(\textit{m}${}^{2}$-\textit{m}-2)\textit{y}=\textit{m}+1在\textit{y}轴上截距等于1,则实数\textit{m}的值\_\_\_\_.

解析:直线(\textit{m}+1)\textit{x}+(\textit{m}${}^{2}$-\textit{m}-2)\textit{y}=\textit{m}+1的方程可化为(\textit{m}+1)\textit{x}+(\textit{m}+1)(\textit{m}-2)\textit{y}=\textit{m}+1,

由题意知\textit{m}+1$\mathrm{\neq}$0,(\textit{m}-2)\textit{y}=1,由题意得$\frac{1}{m-2}$=1,

$\mathrm{\therefore}$\textit{m}=3.

答案:3

知识:直线方程的一般式

难度:3

题目:已知直线\textit{l}:5\textit{ax}-5\textit{y}-\textit{a}+3=0.

(1)求证:不论\textit{a}为何值,直线\textit{l}总经过第一象限;

(2)为使直线\textit{l}不经过第一、三、四象限,求\textit{a}的取值范围.

解析:

答案: (1)将直线\textit{l}的方程整理为\textit{y}-=\textit{a}$(x-\frac{1}{5})$,所以\textit{l}的斜率为\textit{a},且过定点\textit{A}$(\frac{1}{5},\frac{3}{5})$,而点\textit{A}$(\frac{1}{5}, \frac{3}{5})$在第一象限,故不论\textit{a}为何值,直线\textit{l}恒过第一象限.

(2)将方程化为斜截式方程:\textit{y}=\textit{ax}-$\frac{a-3}{5}$.要使\textit{l}经过第一、三、四象限,则$\left\{\begin{array}{l} a>0\\ -\frac{a-3}{5}<0 \end{array}\right.$,解得\textit{a}$\mathrm{>}$3.

知识:直线方程的一般式

难度:3

题目:求满足下列条件的直线方程.

(1)经过点\textit{A}(-1,-3),且斜率等于直线3\textit{x}+8\textit{y}-1=0斜率的2倍;

(2)过点\textit{M}(0,4),且与两坐标轴围成三角形的周长为12.

解析:

答案:(1)因为3\textit{x}+8\textit{y}-1=0可化为\textit{y}=-$\frac{3}{8}$\textit{x}+$\frac{1}{8}$,

所以直线3\textit{x}+8\textit{y}-1=0的斜率为-$\frac{3}{8}$,

则所求直线的斜率\textit{k}=2$\mathrm{\times}$(-$\frac{3}{8}$)=-$\frac{3}{4}$.

又直线经过点(-1,-3),

因此所求直线的方程为\textit{y}+3=-$\frac{3}{4}$(\textit{x}+1),

即3\textit{x}+4\textit{y}+15=0.

(2)设直线与\textit{x}轴的交点为(\textit{a,}0),

因为点\textit{M}(0,4)在\textit{y}轴上,所以由题意有4+$\sqrt{a^2+4^2}$+|\textit{a}|=12,

解得\textit{a}=$\mathrm{\pm}$3,

所以所求直线的方程为$\frac{x}{3}$+$\frac{y}{4}$=1或$\frac{x}{-3}$+$\frac{y}{4}$=1,

即4\textit{x}+3\textit{y}-12=0或4\textit{x}-3\textit{y}+12=0.

知识:两点间的距离公式

难度:1

题目:点\textit{M}(1,2)关于\textit{y}轴的对称点\textit{N}到原点的距离为(  )

A.2   B.1   C.   D.5

解析:\textit{N}(-1,2),|\textit{ON}|=$\sqrt{(-1)^2}+2^2$=$\sqrt{5}$.故选C.

答案:C

知识:两点间的距离公式

难度:1

题目:已知\textit{A}(2,1)、\textit{B}(-1,\textit{b}),|\textit{AB}|=5,则\textit{b}等于(  )

A.-3   B.5   C.-3或5   D.-1或-3

解析:由两点间的距离公式知

|\textit{AB}|$\sqrt{(-1-2)^2+(b-1)^2}=\sqrt{b^2-2b+10}$,

由5=$\sqrt{b^2-2b+10}$,

解得\textit{b}=-3或\textit{b}=5.

答案:C

知识:两条直线的交点坐标

难度:1

题目:经过两点\textit{A}(-2,5)、\textit{B}(1,-4)的直线\textit{l}与\textit{x}轴的交点的坐标是(  )

A.(-$\frac{1}{2}$,0)   B.(-3,0)   C.($\frac{1}{3}$,0)   D.(3,0)

解析:过点\textit{A}(-2,5)和\textit{B}(1,-4)的直线方程为3\textit{x}+\textit{y}+1=0,故它与\textit{x}轴的交点的坐标为(-$\frac{1}{2}$,0).

答案:A

知识:两条直线的交点坐标

难度:1

题目:若三条直线2\textit{x}+3\textit{y}+8=0,\textit{x}-\textit{y}=1,和\textit{x}+\textit{ky}=0相交于一点,则\textit{k}的值等于(  )

A.-2   B.-$\frac{1}{2}$   C.2   D.$\frac{1}{2}$

解析:由$\left\{\begin{array}{l} x-y=1\\ 2x+3y+8=0 \end{array}\right.$,得交点(-1,-2),

代入\textit{x}+\textit{ky}=0得\textit{k}=-$\frac{1}{2}$,故选B.

答案:B

知识:两条直线的交点坐标

难度:1

题目:一条平行于\textit{x}轴的线段长是5个单位,它的一个端点是\textit{A}(2,1),则它的另一个端点\textit{B}的坐标为(  )

A.(-3,1)或(7,1)    B.(2,-2)或(2,7)

C.(-3,1)或(5,1)    D.(2,-3)或(2,5)

解析: $\mathrm{\because}$\textit{AB}$\mathrm{\//}$\textit{x}轴,$\mathrm{\therefore}$设\textit{B}(\textit{a,}1),又|\textit{AB}|=5,$\mathrm{\therefore}$\textit{a}=-3或7.

答案:A

知识:两点间的距离公式

难度:1

题目:设点\textit{A}在\textit{x}轴上,点\textit{B}在\textit{y}轴上,\textit{AB}的中点是\textit{P}(2,-1),则|\textit{AB}|等于(  )

A.5   B.4$\sqrt{2}$   C.2$\sqrt{5}$   D.2$\sqrt{10}$

解析:设\textit{A}(\textit{x,}0)、\textit{B}(0,\textit{y}),由中点公式得\textit{x}=4,\textit{y}=-2,则由两点间的距离公式得|\textit{AB}|=$\sqrt{(0-4)^2+(-2-0)^2}$=$\sqrt{20}$=2$\sqrt{5}$.

答案:C

知识:两点间的距离公式

难度:1

题目:已知\textit{A}(1,-1)、\textit{B}(\textit{a,}3)、\textit{C}(4,5),且|\textit{AB}|=|\textit{BC}|,则\textit{a}=\_\_\_\_.

解析:$\sqrt{(a-1)^2+(a+1)^2}=\sqrt{(4-a)^2+(5-3)^2}$,

解得\textit{a}=$\frac{1}{2}$.

答案:$\frac{1}{2}$

知识:两条直线的交点坐标

难度:1

题目:直线(\textit{a}+2)\textit{x}+(1-\textit{a})\textit{y}-3=0与直线(\textit{a}+2)\textit{x}+(2\textit{a}+3)\textit{y}+2=0不相交,则实数\textit{a}=\_\_\_\_.

解析:由题意,得(\textit{a}+2)(2\textit{a}+3)-(1-\textit{a})(\textit{a}+2)=0,解得\textit{a}=-2或-$\frac{2}{3}$.

答案:-2或$-\frac{2}{3}$

知识:两条直线的交点坐标

难度:1

题目:(2016~2017·哈尔滨高一检测)求平行于直线2\textit{x}-\textit{y}+3=0,且与两坐标轴围成的直角三角形面积为9的直线方程.

解析:设所求的直线方程为2\textit{x}-\textit{y}+\textit{c}=0,令\textit{y}=0,\textit{x}=-$\frac{c}{2}$,令\textit{x}=0,\textit{y}=\textit{c},所以$|c\cdot(-\frac{c}{2})|$=9,解得\textit{c}=$\mathrm{\pm}$6,故所求直线方程为2\textit{x}-\textit{y}$\mathrm{\pm}$6=0.

解法2:设所求直线方程为$\frac{x}{a}+\frac{y}{b}$1.

变形得\textit{bx}+\textit{ay}-\textit{ab}=0.

由条件知$\left\{\begin{array}{l} \frac{b}{2}=\frac{a}{-1}\text{①}\\ \frac{1}{2}|ab|=9 \text{②} \end{array}\right.$

由①得\textit{b}=-2\textit{a}代入②得\textit{a}${}^{2}$=9,

$\mathrm{\therefore}$\textit{a}=$\mathrm{\pm}$3.

当\textit{a}=3时,\textit{b}=-6,当\textit{a}=-3时,\textit{b}=6,

$\mathrm{\therefore}$所求直线方程为2\textit{x}-\textit{y}$\mathrm{\pm}$6=0.

答案:2\textit{x}-\textit{y}$\mathrm{\pm}$6=0

知识:两直线的交点坐标

难度:1

题目:已知直线\textit{x}+\textit{y}-3\textit{m}=0和2\textit{x}-\textit{y}+2\textit{m}-1=0的交点\textit{M}在第四象限,求实数\textit{m}的取值范围.

解析:

答案:由$\left\{\begin{array}{l} x+y-3m=0\\ 2x-y+2m-1=0 \end{array}\right.$,得$\left\{\begin{array}{l} x=\frac{m+1}{3}\\ y=\frac{8m-1}{3} \end{array}\right.$.

$\mathrm{\therefore}$交点\textit{M}的坐标为($\frac{m+1}{3}$,$\frac{8m-1}{1}$).

$\mathrm{\because}$交点\textit{M}在第四象限,

$\mathrm{\therefore}\left\{\begin{array}{l} \frac{m+1}{3}>0\\ \frac{8m-1}{3}<0 \end{array}\right.$,解得-1$\mathrm{<}$\textit{m}$\mathrm{<}\frac{1}{8}$.

$\mathrm{\therefore}$\textit{m}的取值范围是(-1,$\frac{1}{8}$).

知识:两点间的距离公式

难度:2

题目:已知点\textit{A}(2,3)和\textit{B}(-4,1),则线段\textit{AB}的长及中点坐标分别是(  )

A.2$\sqrt{10}$,(1,2)    B.2$\sqrt{10}$,(-1,-2)

C.2$\sqrt{10}$,(-1,2)    D.2$\sqrt{10}$,(1,-2)

解析:|\textit{AB}|=$\sqrt{(-4-2)^2+(1-3)^2}$=2$\sqrt{10}$,中点坐标为($\frac{2-4}{2}$,$\frac{3+1}{2}$),即(-1,2),故选C.

答案:C

知识:两点间的距离公式

难度:2

题目:已知两点\textit{P}(\textit{m,}1)和\textit{Q}(1,2\textit{m})之间的距离大于$\sqrt{10}$,则实数\textit{m}的范围是(  )

A.-$\frac{4}{5}$<\textit{m}<2    B.\textit{m}<-$\frac{4}{5}$或\textit{m}>2

C.\textit{m}<-2或\textit{m}> $\frac{4}{5}$  D.-2<\textit{m}<$\frac{4}{5}$

解析:根据两点间的距离公式

|\textit{PQ}|=$\sqrt{(m-1)^2+(1-2m)^2}=\sqrt{5m^2-6m-8}$>0,$\mathrm{\therefore}$5\textit{m}${}^{2}$-6\textit{m}-8>0,$\mathrm{\therefore}$\textit{m}<-$\frac{4}{5}$或\textit{m}>2.

答案:B

知识:两条直线的交点坐标

难度:2

题目:(2016~2017·宿州高一检测)在同一平面直角坐标系中,直线\textit{l}${}_{1}$:\textit{ax}+\textit{y}+\textit{b}=0和直线\textit{l}${}_{2}$:\textit{bx}+\textit{y}+\textit{a}=0有可能是(  )

\includegraphics*[width=2.76in, height=0.69in, keepaspectratio=false]{image284}

解析:\textit{l}${}_{1}$:\textit{y}=-\textit{ax}-\textit{b},\textit{l}${}_{2}$:\textit{y}=-\textit{bx}-\textit{a},

由图A中\textit{l}${}_{1}$知,-\textit{b}$\mathrm{>}$0,与\textit{l}${}_{2}$中-\textit{b}$\mathrm{<}$0矛盾,排除A;同理排除D.在图C中,由\textit{l}${}_{1}$知-\textit{b}$\mathrm{<}$0,与\textit{l}${}_{2}$中,-\textit{b}$\mathrm{>}$0矛盾,排除C.选B.

答案:B



知识:两条直线的交点坐标

难度:2

题目:已知直线\textit{mx}+4\textit{y}-2=0与2\textit{x}-5\textit{y}+\textit{n}=0互相垂直,垂足为(1,\textit{p}),则\textit{m}-\textit{n}+\textit{p}为(  )

A.24   B.20   C.0   D.-4

解析:$\mathrm{\because}$两直线互相垂直,$\mathrm{\therefore}$\textit{k}${}_{1}$·\textit{k}${}_{2}$=-1,

$\mathrm{\therefore}-\frac{m}{4}\cdot\frac{2}{5}=-1$,$\mathrm{\therefore}$\textit{m}=10.又$\mathrm{\because}$垂足为(1,\textit{p}),

$\mathrm{\therefore}$代入直线10\textit{x}+4\textit{y}-2=0得\textit{p}=-2,

将(1,-2)代入直线2\textit{x}-5\textit{y}+\textit{n}=0得\textit{n}=-12,

$\mathrm{\therefore}$\textit{m}-\textit{n}+\textit{p}=20.

答案:B

知识:两条直线的交点坐标

难度:2

题目:已知直线5\textit{x}+4\textit{y}=2\textit{a}+1与直线2\textit{x}+3\textit{y}=\textit{a}的交点位于第四象限,则\textit{a}的取值范围是\_\_\underbar{-$\mathrm{<}$}\textit{\underbar{a}}\underbar{$\mathrm{<}$2}\_\_.

解析:解方程组$\left\{\begin{array}{l} 5x+4y=2a+1\\ 2x+3y=a \end{array}\right.$,得$\left\{\begin{array}{l} x=\frac{2a+3}{7}\\ y=\frac{a-2}{7} \end{array}\right.$.

交点在第四象限,所以$\left\{\begin{array}{l} \frac{2a+3}{7}>0\\ \frac{a-2}{7}<0 \end{array}\right.$,解得-$\frac{3}{2}\mathrm{<}$\textit{a}$\mathrm{<}$2.

答案:$-\frac{3}{2}<a<2$



知识:两点间的距离公式

难度:2

题目:已知点\textit{A}(5,2\textit{a}-1)、\textit{B}(\textit{a}+1,\textit{a}-4),若|\textit{AB}|取得最小值,则实数\textit{a}的值是\_\_\_\_.

解析:由题意得|\textit{AB}|=

$\sqrt{(5-a-1)^2+(2a-1-a+4)^2}=\sqrt{2a^2-2a+25}=\sqrt{2\times(a-\frac{1}{2})^2+\frac{49}{2}}$,所以当\textit{a}=$\frac{1}{2}$时,|\textit{AB}|取得最小值.

答案:$\frac{1}{2}$


知识:两条直线的交点坐标

难度:3

题目:直线\textit{l}过定点\textit{P}(0,1),且与直线\textit{l}${}_{1}$:\textit{x}-3\textit{y}+10=0,\textit{l}${}_{2}$:2\textit{x}+\textit{y}-8=0分别交于\textit{A}、\textit{B}两点.若线段\textit{AB}的中点为\textit{P},求直线\textit{l}的方程.

解析:

答案: 解法一:设\textit{A}(\textit{x}${}_{0}$,\textit{y}${}_{0}$),由中点公式,有\textit{B}(-\textit{x}${}_{0}$,2-\textit{y}${}_{0}$),$\mathrm{\because}$\textit{A}在\textit{l}${}_{1}$上,\textit{B}在\textit{l}${}_{2}$上,

$\mathrm{\therefore}\left\{\begin{array}{l} x_0-3y_0+10=0\\	-2x_0+2-y_0-8=0 \end{array}\right.\Rightarrow \left\{\begin{array}{l} x_0=-4\\ y_0=2 \end{array}\right.$,

$\mathrm{\therefore}$\textit{k}${}_{AP}=\frac{1-2}{0+4}=-\frac{1}{4}$,

故所求直线\textit{l}的方程为:\textit{y}=-$\frac{1}{4}$\textit{x}+1,

即\textit{x}+4\textit{y}-4=0.

解法二:设所求直线\textit{l}方程为:

\textit{y}=\textit{kx}+1,\textit{l}与\textit{l}${}_{1}$、\textit{l}${}_{2}$分别交于\textit{M}、\textit{N}.

解方程组$\left\{\begin{array}{l} y=kx+1\\ x-3y+10=0 \end{array}\right.$,得\textit{N}($\frac{7}{3k-1}$,$\frac{10k-1}{3k-1}$).

解方程组$\left\{\begin{array}{l} 	y=kx+1\\ 	2x+y-8=0 \end{array}\right.$,得\textit{M}($\frac{7}{k+2}$,$\frac{8k+2}{k+2}$).

$\mathrm{\because}$\textit{M}、\textit{N}的中点为\textit{P}(0,1)则有:

$\frac{1}{2}$($\frac{7}{3k-1}$+$\frac{7}{k+2}$)=0,解得$\mathrm{\therefore}$\textit{k}=-$\frac{1}{4}$.

故所求直线\textit{l}的方程为\textit{x}+4\textit{y}-4=0.



知识:两点间的距离公式

难度:3

题目:如下图所示,一个矩形花园里需要铺设两条笔直的小路,已知矩形花园的长\textit{AD}=5 m,宽\textit{AB}=3 m,其中一条小路定为\textit{AC},另一条小路过点\textit{D},问是否在\textit{BC}上存在一点\textit{M},使得两条小路\textit{AC}与\textit{DM}相互垂直?若存在,则求出小路\textit{DM}的长.

\includegraphics*[width=1.31in, height=0.79in, keepaspectratio=false]{image285}

解析:

答案:以\textit{B}为坐标原点,\textit{BC}、\textit{BA}所在直线为\textit{x}、\textit{y}轴建立如图所示的平面直角坐标系.

\includegraphics*[width=1.37in, height=1.00in, keepaspectratio=false]{image286}

因为\textit{AD}=5 m,\textit{AB}=3 m,

所以\textit{C}(5,0)、\textit{D}(5,3)、\textit{A}(0,3).

设点\textit{M}的坐标为(\textit{x,}0),因为\textit{AC}$\mathrm{\bot}$\textit{DM},

所以\textit{k${}_{AC}$}·\textit{k${}_{DM}$}=-1,

即$\frac{3-0}{0-5}\cdot\frac{3-0}{5-x}$=-1.

所以\textit{x}=3.2,即|\textit{BM}|=3.2,

即点\textit{M}的坐标为(3.2,0)时,两条小路\textit{AC}与\textit{DM}相互垂直.

故在\textit{BC}上存在一点\textit{M}(3.2,0)满足题意.

由两点间距离公式得|\textit{DM}|=$\sqrt{(5-3.2)^2+(3-0)^2}=\frac{3\sqrt{24}}{5}$.


知识:两条平行直线间的距离

难度:1

题目:两直线3\textit{x}+4\textit{y}-2=0与6\textit{x}+8\textit{y}-5=0的距离等于(  )

A.3   B.7   C.$\frac{1}{10}$   D.$\frac{1}{2}$

解析:在3\textit{x}+4\textit{y}-2=0上取一点(0,$\frac{1}{2}$),其到6\textit{x}+8\textit{y}-5=0的距离即为两平行线间的距离,\textit{d}=$\frac{|0+8\times\frac{1}{2}-5|}{\sqrt{6^2+8^2}}$=$\frac{1}{10}$.

答案:C

知识:点到直线的距离

难度:1

题目:已知$\mathrm{\vartriangle}$\textit{ABC}的三个顶点坐标分别为\textit{A}(2,6)、\textit{B}(-4,3)、\textit{C}(2,-3),则点\textit{A}到\textit{BC}边的距离为(  )

A.$\frac{9}{2}$   B.$\frac{9\sqrt{2}}{2}$   C.$\frac{2\sqrt{5}}{5}$   D.4$\sqrt{3}$

解析:\textit{BC}边所在直线的方程为$\frac{y-3}{-3-3}=\frac{x+4}{2+4}$,即\textit{x}+\textit{y}+1=0;则\textit{d}=$\frac{|2\times 1+6\times 1+1|}{\sqrt{2}}$=$\frac{9\sqrt{2}}{2}$.

答案:B

知识:点到直线的距离

难度:1

题目:若点\textit{A}(-3,-4)、\textit{B}(6,3)到直线\textit{l}:\textit{ax}+\textit{y}+1=0的距离相等,则实数\textit{a}的值为(  )

A.$\frac{7}{9}$   B.$-\frac{1}{3}$ C.-$\frac{7}{9}$或-$\frac{1}{3}$   D.$\frac{7}{9}$或$\frac{1}{3}$

解析:由题意及点到直线的距离公式得$\frac{|-3a-4+1|}{\sqrt{a^2+1}}=\frac{|6a+3+1|}{\sqrt{a^2+1}}$,解得\textit{a}=-$\frac{1}{3}$或-$\frac{7}{9}$.

答案:C

知识:点到直线的距离

难度:1

题目:若点\textit{P}在直线3\textit{x}+\textit{y}-5=0上,且点\textit{P}到直线\textit{x}-\textit{y}-1=0的距离为,则点\textit{P}的坐标为(  )

A.(1,2)    B.(2,1)

C.(1,2)或(2,-1)   D.(2,1)或(-1,2)

解析:设点\textit{P}的坐标为(\textit{x}${}_{0}$,\textit{y}${}_{0}$),则有

$\left\{\begin{array}{l} 3x_0+y_0-5=0\\ \frac{|x_0-y_0-1|}{\sqrt{2}}=\sqrt{2} \end{array}\right.$,解得$\left\{\begin{array}{l} x_0=1\\ y_0=2 \end{array}\right.$或$\left\{\begin{array}{l} x_0=2\\ y_0=-1 \end{array}\right.$.

答案:C

知识:点到直线的距离

难度:1

题目:已知点\textit{A}(1,3)、\textit{B}(3,1)、\textit{C}(-1,0),则$\mathrm{\vartriangle}$\textit{ABC}的面积等于(  )

A.3   B.4   C.5   D.6

解析:设\textit{AB}边上的高为\textit{h},则\textit{S}${}_{\vartriangle }$\textit{${}_{ABC}$}=$\frac{1}{2}$|\textit{AB}|·\textit{h}.|\textit{AB}|=$\sqrt{(3-1)^2+(1-3)^2}$=2$\sqrt{2}$,\textit{AB}边上的高\textit{h}就是点\textit{C}到直线\textit{AB}的距离.\textit{AB}边所在的直线方程为$\frac{y-3}{1-3}=\frac{x-1}{3-1}$,即\textit{x}+\textit{y}-4=0.点\textit{C}到直线\textit{x}+\textit{y}-4=0的距离为$\frac{|-1+0-4|}{\sqrt{2}}=\frac{5}{\sqrt{2}}$,因此,\textit{S}${}_{\vartriangle }$\textit{${}_{ABC}$}=$\frac{1}{2}\mathrm{\times}2\sqrt{2}\mathrm{\times}\frac{5}{\sqrt{2}}$=5.

答案:C

知识:两条平行直线间的距离

难度:1

题目:直线\textit{l}垂直于直线\textit{y}=\textit{x}+1,且\textit{l}在\textit{y}轴上的截距为$\sqrt{2}$,则直线\textit{l}的方程是(  )

A.\textit{x}+\textit{y}-$\sqrt{2}$=0    B.\textit{x}+\textit{y}+1=0

C.\textit{x}+\textit{y}-1=0    D.\textit{x}+\textit{y}+$\sqrt{2}$=0

解析: 方法1:因为直线\textit{l}与直线\textit{y}=\textit{x}+1垂直,所以设直线\textit{l}的方程为\textit{y}=-\textit{x}+\textit{b},又\textit{l}在\textit{y}轴上截距为$\sqrt{2}$,所以所求直线\textit{l}的方程为\textit{y}=-\textit{x}+$\sqrt{2}$,即\textit{x}+\textit{y}-$\sqrt{2}$=0.

方法2:将直线\textit{y}=\textit{x}+1化为一般式\textit{x}-\textit{y}+1=0,因为直线\textit{l}垂直于直线\textit{y}=\textit{x}+1,可以设直线\textit{l}的方程为\textit{x}+\textit{y}+\textit{c}=0,令\textit{x}=0,得\textit{y}=-\textit{c},又直线\textit{l}在\textit{y}轴上截距为$\sqrt{2}$,所以-\textit{c}=$\sqrt{2}$,即\textit{c}=-$\sqrt{2}$,所以直线\textit{l}的方程为\textit{x}+\textit{y}-$\sqrt{2}$=0.

答案:A

知识:两条平行直线间的距离

难度:1

题目:已知直线\textit{l}${}_{1}$:(\textit{k}-3)\textit{x}+(4-\textit{k})\textit{y}+1=0与直线\textit{l}${}_{2}$:2(\textit{k}-3)\textit{x}-2\textit{y}+3=0平行,则\textit{l}${}_{1}$与\textit{l}${}_{2}$间的距离为\_\_\_\_.

解析:$\mathrm{\because}$\textit{l}${}_{1}$$\mathrm{\//}$\textit{l}${}_{2}$,

$\mathrm{\therefore}\left\{\begin{array}{l} 2(k-3)+2(k-3)(4-k)=0\\ -\frac{2}{3}\neq 4-k \end{array}\right.$,

解得\textit{k}=3或\textit{k}=5.

当\textit{k}=3时,\textit{l}${}_{1}$:\textit{y}=-1,\textit{l}${}_{2}$:\textit{y}=$\frac{3}{2}$,此时\textit{l}${}_{1}$与\textit{l}${}_{2}$间的距离为$\frac{5}{2}$;

当\textit{k}=5时,\textit{l}${}_{1}$:2\textit{x}-\textit{y}+1=0,\textit{l}${}_{2}$:4\textit{x}-2\textit{y}+3=0,

此时\textit{l}${}_{1}$与\textit{l}${}_{2}$间的距离为$\frac{|3-2|}{\sqrt{4^2+(-2)^2}}=\frac{\sqrt{5}}{10}$.

答案:$\frac{\sqrt{5}}{10}$

知识:两条平行直线间的距离

难度:1

题目:过点\textit{A}(-3,1)的所有直线中,与原点距离最远的直线方程是\_\_\_\_.

解析:当原点与点\textit{A}的连线与过点\textit{A}的直线垂直时,距离最大.$\mathrm{\because}$\textit{k${}_{OA}$}=-$\frac{1}{3}$,$\mathrm{\therefore}$所求直线的方程为\textit{y}-1=3(\textit{x}+3),即3\textit{x}-\textit{y}+10=0.

答案:3\textit{x}-\textit{y}+10=0

知识:点到直线的距离

难度:1

题目:已知正方形的中心为直线2\textit{x}-\textit{y}+2=0和\textit{x}+\textit{y}+1=0的交点,其一边所在直线的方程为\textit{x}+3\textit{y}-5=0,求其它三边的方程.

解析:

答案:由$\left\{\begin{array}{l} 2x-y+2=0\\ x+y+1=0 \end{array}\right.$,解得$\left\{\begin{array}{l} x=-1\\ y=0 \end{array}\right.$.

即该正方形的中心为(-1,0).

所求正方形相邻两边方程3\textit{x}-\textit{y}+\textit{p}=0和\textit{x}+3\textit{y}+\textit{q}=0.

$\mathrm{\because}$中心(-1,0)到四边距离相等,

$\mathrm{\therefore}\frac{|-3+p|}{\sqrt{10}}=\frac{6}{\sqrt{10}}, \frac{|-1+q|}{\sqrt{10}}=\frac{6}{\sqrt{10}}$

解得\textit{p}${}_{1}$=-3,\textit{p}${}_{2}$=9和\textit{q}${}_{1}$=-5,\textit{q}${}_{2}$=7,

$\mathrm{\therefore}$所求方程为3\textit{x}-\textit{y}-3=0,3\textit{x}-\textit{y}+9=0,\textit{x}+3\textit{y}+7=0.

知识:点到直线的距离

难度:1

题目:已知三条直线\textit{l}${}_{1}$:4\textit{x}+\textit{y}-4=0,\textit{l}${}_{2}$:\textit{mx}+\textit{y}=0,\textit{l}${}_{3}$:2\textit{x}-3\textit{my}-4=0.求\textit{m}的值,使它分别满足以下条件:(1)\textit{l}${}_{1}$,\textit{l}${}_{2}$,\textit{l}${}_{3}$交于同一点;(2)\textit{l}${}_{1}$,\textit{l}${}_{2}$,\textit{l}${}_{3}$不能围成三角形.

解析:

答案:
(1)由4\textit{x}+\textit{y}-4=0得\textit{y}=-4\textit{x}+4代入\textit{l}${}_{2}$,\textit{l}${}_{3}$的方程中分别得

\textit{x}${}_{1}$=$\frac{-4}{m-4}$,\textit{x}${}_{2}$=$\frac{6m+3}{1+6ms}$,

由$\frac{-4}{m-4}=\frac{6m+3}{1+6ms}$,解得\textit{m}=-1或$\frac{2}{3}$,经检验都符合题意.

(2)首先由(1)知,当\textit{m}=-1或时,不能围成三角形;

又\textit{kl}${}_{1}$=-4,\textit{kl}${}_{2}$=-\textit{m},\textit{kl}${}_{3}$=$\frac{2}{3m}$,

若\textit{l}${}_{1}$$\mathrm{\//}$\textit{l}${}_{2}$,则\textit{m}=4;若\textit{l}${}_{1}$$\mathrm{\//}$\textit{l}${}_{3}$,则\textit{m}=-$\frac{1}{6}$;

由于\textit{kl}${}_{2}$与\textit{kl}${}_{3}$异号,显然\textit{l}${}_{2}$与\textit{l}${}_{3}$不平行.

综上知,\textit{m}=-1,-$\frac{1}{6}$,$\frac{2}{3}$或4.

知识:两条平行直线间的距离

难度:2

题目:\textit{P}、\textit{Q}分别为3\textit{x}+4\textit{y}-12=0与6\textit{x}+8\textit{y}+6=0上任一点,则|\textit{PQ}|的最小值为(  )

A.$\frac{9}{5}$   B.$\frac{18}{5}$   C.3   D.6

解析:|\textit{PQ}|的最小值是这两条平行线间的距离.在直线3\textit{x}+4\textit{y}-12=0上取点(4,0),然后利用点到直线的距离公式得|\textit{PQ}|的最小值为3.

答案:C

知识:两条平行直线间的距离

难度:2

题目:(2016·潍坊高一检测)与直线\textit{l}:3\textit{x}-4\textit{y}-1=0平行且到直线\textit{l}的距离为2的直线方程是(  )

A.3\textit{x}-4\textit{y}-11=0或3\textit{x}-4\textit{y}+9=0

B.3\textit{x}-4\textit{y}-11=0

C.3\textit{x}-4\textit{y}+11=0或3\textit{x}-4\textit{y}-9=0

D.3\textit{x}-4\textit{y}+9=0

解析:设所求直线方程为3\textit{x}-4\textit{y}+\textit{m}=0,由题意得$\frac{||m-(-1)}{3^2+(-4)^2}$=2,

解得\textit{m}=9或-11.

答案:A

知识:点到直线的距离

难度:2

题目:到两条直线\textit{l}${}_{1}$:3\textit{x}-4\textit{y}+5=0与\textit{l}${}_{2}$:5\textit{x}-12\textit{y}+13=0的距离相等的点\textit{P}(\textit{x},\textit{y})必定满足方程(  )

A.\textit{x}-4\textit{y}+4=0

B.7\textit{x}+4\textit{y}=0

C.\textit{x}-4\textit{y}+4=0或4\textit{x}-8\textit{y}+9=0

D.7\textit{x}+4\textit{y}=0或32\textit{x}-56\textit{y}+65=0

解析:结合图形可知,这样的直线应该有两条,恰好是两条相交直线所成角的平分线.由公式可得$\frac{|3x-4y+5|}{3^2+(-4)^2}=\frac{|5x-12y+13|}{\sqrt{5^2+(-12)^2}}$,即$\frac{3x-4y+5}{5}=\mathrm{\pm}\frac{5x-12y+13}{13}$,化简得7\textit{x}+4\textit{y}=0或32\textit{x}-56\textit{y}+65=0.

答案:D

知识:两条平行直线间的距离

难度:2

题目:(2016~2017山西吕梁汾阳四中期中)已知两直线3\textit{x}+\textit{y}-3=0与6\textit{x}+\textit{my}+1=0平行,则它们之间的距离为(  )

A.4   B.$\frac{2\sqrt{13}}{13}$   C.$\frac{5\sqrt{13}}{26}$   D.$\frac{7\sqrt{10}}{20}$

解析:$\mathrm{\because}$两直线平行,

$\mathrm{\therefore}\frac{6}{3}=\frac{m}{1}$.

$\mathrm{\therefore}$\textit{m}=2.

$\mathrm{\therefore}$两直线方程为6\textit{x}+2\textit{y}-6=0和6\textit{x}+2\textit{y}+1=0,其距离\textit{d}=$\frac{|-6-1|}{\sqrt{6^2+2^2}}$=$\frac{7\sqrt{10}}{20}$.故选D.

答案:D

知识:点到直线的距离

难度:2

题目:点\textit{P}(\textit{x},\textit{y})在直线\textit{x}+\textit{y}-4=0上,则\textit{x}${}^{2}$+\textit{y}${}^{2}$的最小值是\_\_\_\_.

解析:\textit{x}${}^{2}$+\textit{y}${}^{2}$表示直线上的点\textit{P}(\textit{x},\textit{y})到原点距离的平方,

$\mathrm{\because}$原点到直线\textit{x}+\textit{y}-4=0的距离为$\frac{|-4|}{\sqrt{2}}$=$2\sqrt{2}$,

$\mathrm{\therefore}$\textit{x}${}^{2}$+\textit{y}${}^{2}$最小值为8.

答案:8

知识:点到直线的距离

难度:2

题目:已知点\textit{A}(1,1)、\textit{B}(2,2),点\textit{P}在直线\textit{y}=$\frac{1}{2}$\textit{x}上,则当|\textit{PA}|${}^{2}$+|\textit{PB}|${}^{2}$取得最小值时点\textit{P}的坐标为\_\_\_\_.

解析:设\textit{P}(2\textit{t},\textit{t}),则|\textit{PA}|${}^{2}$+|\textit{PB}|${}^{2}$=(2\textit{t}-1)${}^{2}$+(\textit{t}-1)${}^{2}$+(2\textit{t}-2)${}^{2}$+(\textit{t}-2)${}^{2}$=10\textit{t}${}^{2}$-18\textit{t}+10=10(\textit{t}${}^{2}$-$\frac{9}{5}$\textit{t}+1)=10(\textit{t}-$\frac{9}{10}$)${}^{2}$+$\frac{19}{10}$,当\textit{t}=$\frac{9}{10}$时,|\textit{PA}|${}^{2}$+|\textit{PB}${}^{2}$|取得最小值,即\textit{P}($\frac{9}{5}$,$\frac{9}{10}$).

答案:$(\frac{9}{5}, \frac{9}{10})$

知识:点到直线的距离

难度:3

题目:(2016~2017·嘉兴高一检测)在$\mathrm{\vartriangle}$\textit{ABC}中,已知\textit{BC}边上的高所在直线的方程为\textit{x}-2\textit{y}+1=0,$\mathrm{\angle}$\textit{A}的平分线所在直线的方程为\textit{y}=0,若点\textit{B}的坐标为(1,2).

(1)求直线\textit{BC}的方程.

(2)求直线\textit{AB}的方程.

解析:

答案:(1)设\textit{AD}$\mathrm{\bot}$\textit{BC},垂足为\textit{D},

则\textit{k${}_{AD}$}=$\frac{1}{2}$,

$\mathrm{\therefore}$\textit{k${}_{BC}$}=-2.

$\mathrm{\therefore}$\textit{BC}边所在直线方程为\textit{y}-2=-2(\textit{x}-1).

即2\textit{x}+\textit{y}-4=0.

(2)$\mathrm{\because}$$\mathrm{\angle}$\textit{A}的平分线所在直线方程为\textit{y}=0,

$\mathrm{\therefore}$设\textit{A}(\textit{a,}0).

又点\textit{A}在直线\textit{AD}上,$\mathrm{\therefore}$\textit{a}-0+1=0,

$\mathrm{\therefore}$\textit{a}=-1.

$\mathrm{\therefore}$\textit{A}(-1,0),

$\mathrm{\therefore}$直线\textit{AB}方程为:\textit{y}=\textit{x}+1.即\textit{x}-\textit{y}+1=0.

知识:两条平行直线间的距离

难度:3

题目:已知直线\textit{l}经过点\textit{A}(2,4),且被平行直线\textit{l}${}_{1}$:\textit{x}-\textit{y}+1=0与\textit{l}${}_{2}$:\textit{x}-\textit{y}-1=0所截得的线段的中点\textit{M}在直线\textit{x}+\textit{y}-3=0上.求直线\textit{l}的方程.

解析:

答案:解法一:$\mathrm{\because}$点\textit{M}在直线\textit{x}+\textit{y}-3=0上,

$\mathrm{\therefore}$设点\textit{M}坐标为(\textit{t,}3-\textit{t}),则点\textit{M}到\textit{l}${}_{1}$、\textit{l}${}_{2}$的距离相等,

即$\frac{|t-(3-t)+1|}{\sqrt{2}}=\frac{|t-(3-t)-1|}{\sqrt{2}}$,

解得\textit{t}=$\frac{3}{2}$,$\mathrm{\therefore}$\textit{M}$(\frac{3}{2}, \frac{3}{2})$.

又\textit{l}过点\textit{A}(2,4),

由两点式得$\frac{y-\frac{3}{2}}{4-\frac{3}{2}}=\frac{x-\frac{3}{2}}{2-\frac{3}{2}}$,

即5\textit{x}-\textit{y}-6=0,

故直线\textit{l}的方程为5\textit{x}-\textit{y}-6=0.

解法二:设与\textit{l}${}_{1}$、\textit{l}${}_{2}$平行且距离相等的直线\textit{l}${}_{3}$:\textit{x}-\textit{y}+\textit{c}=0,由两平行直线间的距离公式得$\frac{|c-1|}{\sqrt{2}}=\frac{|c+1|}{\sqrt{2}}$,解得\textit{c}=0,即\textit{l}${}_{3}$:\textit{x}-\textit{y}=0.由题意得中点\textit{M}在\textit{l}${}_{3}$上,又点\textit{M}在\textit{x}+\textit{y}-3=0上.

解方程组$\left\{\begin{array}{l} x-y=0\\ x+y-3=0 \end{array}\right.$,得$\left\{\begin{array}{l} x=\frac{3}{2}\\ y=\frac{3}{2} \end{array}\right.$.

$\mathrm{\therefore}$\textit{M}$(\frac{3}{2}, \frac{3}{2})$.又\textit{l}过点\textit{A}(2,4),

故由两点式得直线\textit{l}的方程为5\textit{x}-\textit{y}-6=0.

解法三:由题意知直线\textit{l}的斜率必存在,

设\textit{l}:\textit{y}-4=\textit{k}(\textit{x}-2),

由$\left\{\begin{array}{l} y-4=k(x-2)\\ x-y-1=0 \end{array}\right.$,得$\left\{\begin{array}{l} x=\frac{2k-5}{k-1}\\ y=\frac{k-4}{k-1} \end{array}\right.$.

$\mathrm{\therefore}$直线\textit{l}与\textit{l}${}_{1}$、\textit{l}${}_{2}$的交点分别为$(\frac{2k-3}{k-1}, \frac{3k-4}{k-1})$,$(\frac{2k-5}{k-1}, \frac{k-4}{k-1})$

.

$\mathrm{\because}$\textit{M}为中点,$\mathrm{\therefore}$\textit{M}$(\frac{2k-4}{k-1}, \frac{2k-4}{k-1})$.

又点\textit{M}在直线\textit{x}+\textit{y}-3=0上,

$\mathrm{\therefore}\frac{2k-4}{k-1}+\frac{2k-4}{k-1}$-3=0,解得\textit{k}=5.

故所求直线\textit{l}的方程为\textit{y}-4=5(\textit{x}-2),

即5\textit{x}-\textit{y}-6=0.





知识:圆的标准方程

难度:1

题目:圆心是(4,-1),且过点(5,2)的圆的标准方程是(  )

A.(\textit{x}-4)${}^{2}$+(\textit{y}+1)${}^{2}$=10

B.(\textit{x}+4)${}^{2}$+(\textit{y}-1)${}^{2}$=10

C.(\textit{x}-4)${}^{2}$+(\textit{y}+1)${}^{2}$=100

D.(\textit{x}-4)${}^{2}$+(\textit{y}+1)${}^{2}$=$\sqrt{10}$

解析: 设圆的标准方程为(\textit{x}-4)${}^{2}$+(\textit{y}+1)${}^{2}$=\textit{r}${}^{2}$,把点(5,2)代入可得\textit{r}${}^{2}$=10,即得选A.

答案:A

知识:圆的标准方程

难度:1

题目:已知圆的方程是(\textit{x}-2)${}^{2}$+(\textit{y}-3)${}^{2}$=4,则点\textit{P}(3,2)满足(  )

A.是圆心   B.在圆上 C.在圆内   D.在圆外

解析:因为(3-2)${}^{2}$+(2-3)${}^{2}$=2$\mathrm{<}$4,

故点\textit{P}(3,2)在圆内.

答案:C

知识:圆的标准方程

难度:1

题目:圆(\textit{x}+1)${}^{2}$+(\textit{y}-2)${}^{2}$=4的圆心坐标和半径分别为(  )

A.(-1,2),2   B.(1,-2),2  C.(-1,2),4   D.(1,-2),4

解析: 圆(\textit{x}+1)${}^{2}$+(\textit{y}-2)${}^{2}$=4的圆心坐标为(-1,2),半径\textit{r}=2.

答案:A

知识:圆的标准方程

难度:1

题目:(2016·锦州高一检测)若圆\textit{C}与圆(\textit{x}+2)${}^{2}$+(\textit{y}-1)${}^{2}$=1关于原点对称,则圆\textit{C}的方程是(  )

A.(\textit{x}-2)${}^{2}$+(\textit{y}+1)${}^{2}$=1 B.(\textit{x}-2)${}^{2}$+(\textit{y}-1)${}^{2}$=1

C.(\textit{x}-1)${}^{2}$+(\textit{y}+2)${}^{2}$=1 D.(\textit{x}+1)${}^{2}$+(\textit{y}+2)${}^{2}$=1

解析: $\mathrm{\because}$点\textit{P}(\textit{x},\textit{y})关于原点的对称点为\textit{P}$'$(-\textit{x},-\textit{y}),

$\mathrm{\therefore}$将-\textit{x},-\textit{y}代入$\mathrm{\odot}$\textit{C}的方程得(-\textit{x}+2)${}^{2}$+(-\textit{y}-1)${}^{2}$=1.

即(\textit{x}-2)${}^{2}$+(\textit{y}+1)${}^{2}$=1.

故选A.

答案:A

知识:圆的标准方程

难度:1

题目:(2016·全国卷Ⅱ)圆\textit{x}${}^{2}$+\textit{y}${}^{2}$-2\textit{x}-8\textit{y}+13=0的圆心到直线\textit{ax}+\textit{y}-1=0的距离为1,则\textit{a}=(  )

A.$-\frac{4}{3}$   B.$-\frac{3}{4}$  C.$\sqrt{3}$   D.2

解析:配方得(\textit{x}-1)${}^{2}$+(\textit{y}-4)${}^{2}$=4,

$\mathrm{\therefore}$圆心为\textit{C}(1,4).

由条件知$\frac{|a+4-1|}{\sqrt{a^2+1}}$=1.解之得\textit{a}=$-\frac{4}{3}$.

故选A.

答案:A

知识:圆的标准方程

难度:1

题目:若\textit{P}(2,-1)为圆(\textit{x}-1)${}^{2}$+\textit{y}${}^{2}$=25的弦\textit{AB}的中点,则直线\textit{AB}的方程是(  )

A.\textit{x}-\textit{y}-3=0   B.2\textit{x}+\textit{y}-3=0 C.\textit{x}+\textit{y}-1=0   D.2\textit{x}-\textit{y}-5=0

解析: $\mathrm{\because}$点\textit{P}(2,-1)为弦\textit{AB}的中点,又弦\textit{AB}的垂直平分线过圆心(1,0),

$\mathrm{\therefore}$弦\textit{AB}的垂直平分线的斜率\textit{k}=$\frac{0-(-1)}{1-2}$=-1,

$\mathrm{\therefore}$直线\textit{AB}的斜率\textit{k}$'$=1,

故直线\textit{AB}的方程为\textit{y}-(-1)=\textit{x}-2,即\textit{x}-\textit{y}-3=0.

答案:A


知识:圆的标准方程

难度:1

题目:以点(2,-1)为圆心且与直线\textit{x}+\textit{y}=6相切的圆的方程是\_\_\_\_.

解析: 将直线\textit{x}+\textit{y}=6化为\textit{x}+\textit{y}-6=0,圆的半径\textit{r}=$\frac{|2-1-6|}{\sqrt{1+1}}$=$\frac{5}{\sqrt{2}}$,所以圆的方程为(\textit{x}-2)${}^{2}$+(\textit{y}+1)${}^{2}$=$\frac{25}{2}$.

答案:(\textit{x}-2)${}^{2}$+(\textit{y}+1)${}^{2}$=$\frac{25}{2}$

知识:圆的标准方程

难度:1

题目:圆心既在直线\textit{x}-\textit{y}=0上,又在直线\textit{x}+\textit{y}-4=0上,且经过原点的圆的方程是\_\_\_\_.

解析: 由$\left\{\begin{array}{r} x-y=0\\ x+y-4=0 \end{array} \right.$,得$\left\{\begin{array}{r} x=2\\ y=2 \end{array} \right.$.

$\mathrm{\therefore}$圆心坐标为(2,2),半径\textit{r}=$\sqrt{2^2+2^2}$=$2\sqrt{2}$,

故所求圆的方程为(\textit{x}-2)${}^{2}$+(\textit{y}-2)${}^{2}$=8.

答案:(\textit{x}-2)${}^{2}$+(\textit{y}-2)${}^{2}$=8


知识:圆的标准方程

难度:1

题目:圆过点\textit{A}(1,-2)、\textit{B}(-1,4),求

(1)周长最小的圆的方程;

(2)圆心在直线2\textit{x}-\textit{y}-4=0上的圆的方程.

解析:

答案:
(1)当\textit{AB}为直径时,过\textit{A}、\textit{B}的圆的半径最小,从而周长最小.即\textit{AB}中点(0,1)为圆心,半径\textit{r}=$\frac{1}{2}$|\textit{AB}|=$\sqrt{10}$.则圆的方程为:\textit{x}${}^{2}$+(\textit{y}-1)${}^{2}$=10.

(2)解法一:\textit{AB}的斜率为\textit{k}=-3,则\textit{AB}的垂直平分线的方程是\textit{y}-1=$\frac{1}{3}$\textit{x}.即\textit{x}-3\textit{y}+3=0

由$\left\{\begin{array}{r} x-3y+3=0\\ 2x-y-4=0 \end{array} \right.$,得$\left\{\begin{array}{r} x=3\\ y=2 \end{array} \right.$.

即圆心坐标是\textit{C}(3,2).

\textit{r}=|\textit{AC}|=$\sqrt{(3-1)^2+(2+2)^2}$=$2\sqrt{5}$.

$\mathrm{\therefore}$圆的方程是(\textit{x}-3)${}^{2}$+(\textit{y}-2)${}^{2}$=20.

解法二:待定系数法

设圆的方程为:(\textit{x}-\textit{a})${}^{2}$+(\textit{y}-\textit{b})${}^{2}$=\textit{r}${}^{2}$.

则$\left\{\begin{array}{r} (1-a)^2+(-2-b)^2=x^2\\ (-1-a)^2+(4-b)^2=x^2\\ 2a-b-4=0 \end{array} \right.$,$\left\{\begin{array}{r} a=3\\ b=2\\ x^2=20 \end{array} \right.$.

$\mathrm{\therefore}$圆的方程为:(\textit{x}-3)${}^{2}$+(\textit{y}-2)${}^{2}$=20.

知识:圆的标准方程

难度:1

题目:已知圆\textit{N}的标准方程为(\textit{x}-5)${}^{2}$+(\textit{y}-6)${}^{2}$=\textit{a}${}^{2}$(\textit{a}$\mathrm{>}$0).

(1)若点\textit{M}(6,9)在圆上,求\textit{a}的值;

(2)已知点\textit{P}(3,3)和点\textit{Q}(5,3),线段\textit{PQ}(不含端点)与圆\textit{N}有且只有一个公共点,求\textit{a}的取值范围.

解析:

答案:
(1)因为点\textit{M}在圆上,

所以(6-5)${}^{2}$+(9-6)${}^{2}$=\textit{a}${}^{2}$,

又由\textit{a}$\mathrm{>}$0,可得\textit{a}=$\sqrt{10}$.

(2)由两点间距离公式可得

|\textit{PN}|=$\sqrt{(3-5)^2+(3-6)^2}$=$\sqrt{13}$,

|\textit{QN}|=$\sqrt{(5-5)^2+(3-6)^2}$=3,

因为线段\textit{PQ}与圆有且只有一个公共点,即\textit{P}、\textit{Q}两点一个在圆内、另一个在圆外,由于3$\mathrm{<}\sqrt{13}$,所以3$\mathrm{<}$\textit{a}$\mathrm{<}\sqrt{13}$.即\textit{a}的取值范围是$(3,\sqrt{13})$.

知识:圆的标准方程

难度:2

题目:(2016~2017·宁波高一检测)点$(\frac{1}{2},\frac{\sqrt{3}}{2})$与圆\textit{x}${}^{2}$+\textit{y}${}^{2}$=$\frac{1}{2}$的位置关系是(  )

A.在圆上   B.在圆内 C.在圆外   D.不能确定

解析:将点$(\frac{1}{2},\frac{\sqrt{3}}{2})$的坐标代入圆的方程可知$(\frac{1}{2}){}^{2}$+$(\frac{\sqrt{3}}{2}){}^{2}$=1$\mathrm{>}\frac{1}{2}$.

$\mathrm{\therefore}$点在圆外.

答案:C

知识:圆的标准方程

难度:2

题目:若点(2\textit{a},\textit{a}-1)在圆\textit{x}${}^{2}$+(\textit{y}+1)${}^{2}$=5的内部,则\textit{a}的取值范围是(  )

A.(-$\mathrm{\infty}$,1]   B.(-1,1) C.(2,5)   D.(1,+$\mathrm{\infty}$)

解析: 点(2\textit{a},\textit{a}-1)在圆\textit{x}${}^{2}$+(\textit{y}+1)${}^{2}$=5的内部,则(2\textit{a})${}^{2}$+\textit{a}${}^{2}$<5,解得-1<\textit{a}<1.

答案:B

知识:圆的标准方程

难度:2

题目:若点\textit{P}(1,1)为圆(\textit{x}-3)${}^{2}$+\textit{y}${}^{2}$=9的弦\textit{MN}的中点,则弦\textit{MN}所在直线方程为(  )

A.2\textit{x}+\textit{y}-3=0   B.\textit{x}-2\textit{y}+1=0 C.\textit{x}+2\textit{y}-3=0   D.2\textit{x}-\textit{y}-1=0

解析: 圆心\textit{C}(3,0),\textit{k${}_{PC}$}=$-\frac{1}{2}$,又点\textit{P}是弦\textit{MN}的中点,$\mathrm{\therefore}$\textit{PC}$\mathrm{\bot}$\textit{MN},$\mathrm{\therefore}$\textit{k${}_{MN}$k${}_{PC}$}=-1,

$\mathrm{\therefore}$\textit{k${}_{MN}$}=2,$\mathrm{\therefore}$弦\textit{MN}所在直线方程为\textit{y}-1=2(\textit{x}-1),即2\textit{x}-\textit{y}-1=0.

答案:D

知识:圆的标准方程

难度:2

题目:点\textit{M}在圆(\textit{x}-5)${}^{2}$+(\textit{y}-3)${}^{2}$=9上,则点\textit{M}到直线3\textit{x}+4\textit{y}-2=0的最短距离为(  )

A.9   B.8 C.5   D.2

解析: 圆心(5,3)到直线3\textit{x}+4\textit{y}-2=0的距离为\textit{d}=$\frac{|3\times5+4\times3-2|}{\sqrt{3^2+4^2}}$=5.又\textit{r}=3,则\textit{M}到直线的最短距离为5-3=2.

答案:D


知识:圆的标准方程

难度:2

题目:已知圆\textit{C}经过\textit{A}(5,1)、\textit{B}(1,3)两点,圆心在\textit{x}轴上,则\textit{C}的方程为\_\_\_\_.

解析: 设所求圆\textit{C}的方程为(\textit{x}-\textit{a})${}^{2}$+\textit{y}${}^{2}$=\textit{r}${}^{2}$,

把所给两点坐标代入方程得$\left\{\begin{array}{r} (5-a)^2+1^2=x^2\\ (1-a)^2+3^2=x^2 \end{array} \right.$

,解得$\left\{\begin{array}{r} a=2\\ x^2=10\\ \end{array} \right.$,

所以所求圆\textit{C}的方程为(\textit{x}-2)${}^{2}$+\textit{y}${}^{2}$=10.

答案:(\textit{x}-2)${}^{2}$+\textit{y}${}^{2}$=10

知识:圆的标准方程

难度:2

题目:以直线2\textit{x}+\textit{y}-4=0与两坐标轴的一个交点为圆心,过另一个交点的圆的方程为\_\_\_\_.

解析:令\textit{x}=0得\textit{y}=4,令\textit{y}=0得\textit{x}=2,
$\mathrm{\therefore}$直线与两轴交点坐标为\textit{A}(0,4)和\textit{B}(2,0),以\textit{A}为圆心过\textit{B}的圆方程为\textit{x}${}^{2}$+(\textit{y}-4)${}^{2}$=20,

以\textit{B}为圆心过\textit{A}的圆方程为(\textit{x}-2)${}^{2}$+\textit{y}${}^{2}$=20.

答案:\textit{x}${}^{2}$+(\textit{y}-4)${}^{2}$=20或(\textit{x}-2)${}^{2}$+\textit{y}${}^{2}$=20


知识:圆的标准方程

难度:3

题目:如图,矩形\textit{ABCD}的两条对角线相交于点\textit{M}(2,0),\textit{AB}边所在直线的方程为\textit{x}-3\textit{y}-6=0,点\textit{T}(-1,1)在\textit{AD}边所在的直线上.求\textit{AD}边所在直线的方程.

\includegraphics*[width=1.14in, height=0.98in, keepaspectratio=false]{image289}

解析:

答案: 因为\textit{AB}边所在直线的方程为\textit{x}-3\textit{y}-6=0,且\textit{AD}与\textit{AB}垂直,所以直线\textit{AD}的斜率为-3.

又因为点\textit{T}(-1,1)在直线\textit{AD}上,所以\textit{AD}边所在直线的方程为\textit{y}-1=-3(\textit{x}+1),即3\textit{x}+\textit{y}+2=0.

知识:圆的标准方程

难度:3

题目:求圆心在直线4\textit{x}+\textit{y}=0上,且与直线\textit{l}:\textit{x}+\textit{y}-1=0切于点\textit{P}(3,-2)的圆的方程,并找出圆的圆心及半径.

解析:

答案: 设圆的标准方程为(\textit{x}-\textit{a})${}^{2}$+(\textit{y}-\textit{b})${}^{2}$=\textit{r}${}^{2}$,由题意有$\left\{\begin{array}{r} 4a+b=0\\ \frac{b+2}{a-3}=1\\ (3-a)^2+(-2-b)^2=x^2 \end{array} \right.$,化简得$\left\{\begin{array}{r} 4a+b=2\\ b=a-5\\ (3-a)^2+(-2-b)^2=x^2 \end{array} \right.$,

解得$\left\{\begin{array}{r} a=1\\ b=-4\\ x^2=8 \end{array} \right.$.所求圆的方程为(\textit{x}-1)${}^{2}$+(\textit{y}+4)${}^{2}$=8,它是以(1,-4)为圆心,以$2\sqrt{2}$为半径的圆.

知识:圆的一般方程

难度:1

题目:圆\textit{x}${}^{2}$+\textit{y}${}^{2}$-4\textit{x}+6\textit{y}=0的圆心坐标是(  )

A.(2,3)   B.(-2,3)  C.(-2,-3)   D.(2,-3)

解析:圆的一般程化成标准方程为(\textit{x}-2)${}^{2}$+(\textit{y}+3)${}^{2}$=13,可知圆心坐标为(2,-3).

答案:D

知识:圆的一般方程

难度:1

题目:(2016~2017·曲靖高一检测)方程\textit{x}${}^{2}$+\textit{y}${}^{2}$+2\textit{ax}-\textit{by}+\textit{c}=0表示圆心为\textit{C}(2,2),半径为2的圆,则\textit{a},\textit{b},\textit{c}的值依次为(  )

A.-2,4,4   B.-2,-4,4  C.2,-4,4   D.2,-4,-4

解析: 配方得(\textit{x}+\textit{a})${}^{2}$+(\textit{y}-)${}^{2}$=\textit{a}${}^{2}$+-\textit{c},

由条件知$\left\{\begin{array}{r} -a=2\\ \frac{b}{2}=2\\ \sqrt{a^2+\frac{b^2}{4}-c}=2 \end{array} \right.$$\mathrm{\therefore}$$\left\{\begin{array}{r} a=-2\\ b=4\\ c=4 \end{array} \right.$

答案:A

知识:圆的一般方程

难度:1

题目:(2016~2017·长沙高一检测)已知圆\textit{C}过点\textit{M}(1,1),\textit{N}(5,1),且圆心在直线\textit{y}=\textit{x}-2上,则圆\textit{C}的方程为(  )

A.\textit{x}${}^{2}$+\textit{y}${}^{2}$-6\textit{x}-2\textit{y}+6=0 B.\textit{x}${}^{2}$+\textit{y}${}^{2}$+6\textit{x}-2\textit{y}+6=0

C.\textit{x}${}^{2}$+\textit{y}${}^{2}$+6\textit{x}+2\textit{y}+6=0 D.\textit{x}${}^{2}$+\textit{y}${}^{2}$-2\textit{x}-6\textit{y}+6=0

解析: 由条件知,圆心\textit{C}在线段\textit{MN}的中垂线\textit{x}=3上,又在直线\textit{y}=\textit{x}-2上,$\mathrm{\therefore}$圆心\textit{C}(3,1),半径\textit{r}=|\textit{MC}|=2.

方程为(\textit{x}-3)${}^{2}$+(\textit{y}-1)${}^{2}$=4,即\textit{x}${}^{2}$+\textit{y}${}^{2}$-6\textit{x}-2\textit{y}+6=0.

故选A.

答案:A

知识:圆的一般方程

难度:1

题目:设圆的方程是\textit{x}${}^{2}$+\textit{y}${}^{2}$+2\textit{ax}+2\textit{y}+(\textit{a}-1)${}^{2}$=0,若0$\mathrm{<}$\textit{a}$\mathrm{<}$1,则原点与圆的位置关系是(  )

A.在圆上   B.在圆外 C.在圆内   D.不确定

解析:将原点坐标(0,0)代入圆的方程得(\textit{a}-1)${}^{2}$,

$\mathrm{\because}$0$\mathrm{<}$\textit{a}$\mathrm{<}$1,$\mathrm{\therefore}$(\textit{a}-1)${}^{2}$$\mathrm{>}$0,$\mathrm{\therefore}$原点在圆外.

答案:B

知识:圆的一般方程

难度:1

题目:若圆\textit{x}${}^{2}$+\textit{y}${}^{2}$-2\textit{x}-4\textit{y}=0的圆心到直线\textit{x}-\textit{y}+\textit{a}=0的距离为$\frac{\sqrt{2}}{2}$,则\textit{a}的值为(  )

A.-2或2   B.$\frac{1}{2}$或$\frac{3}{2}$  C.2或0   D.-2或0

解析:化圆的标准方程为(\textit{x}-1)${}^{2}$+(\textit{y}-2)${}^{2}$=5,则由圆心(1,2)到直线\textit{x}-\textit{y}+\textit{a}=0距离为$\frac{\sqrt{2}}{2}$,得$\frac{|1-2+a|}{\sqrt{2}}$=$\frac{\sqrt{2}}{2}$,$\mathrm{\therefore}$\textit{a}=2或0.

答案:C

知识:圆的一般方程

难度:1

题目:圆\textit{x}${}^{2}$+\textit{y}${}^{2}$-2\textit{y}-1=0关于直线\textit{y}=\textit{x}对称的圆的方程是(  )

A.(\textit{x}-1)${}^{2}$+\textit{y}${}^{2}$=2    B.(\textit{x}+1)${}^{2}$+\textit{y}${}^{2}$=2 

C.(\textit{x}-1)${}^{2}$+\textit{y}${}^{2}$=4    D.(\textit{x}+1)${}^{2}$+\textit{y}${}^{2}$=4

解析: 圆\textit{x}${}^{2}$+\textit{y}${}^{2}$-2\textit{y}-1=0的圆心坐标为(0,1),半径\textit{r}=$\sqrt{2}$,圆心(0,1)关于直线\textit{y}=\textit{x}对称的点的坐标为(1,0),故所求圆的方程为(\textit{x}-1)${}^{2}$+\textit{y}${}^{2}$=2.

答案:A

知识:圆的一般方程

难度:1

题目:圆心是(-3,4),经过点\textit{M}(5,1)的圆的一般方程为\_\_\_\_.

解析:只要求出圆的半径即得圆的标准方程,再展开化为一般式方程.

答案:\textit{x}${}^{2}$+\textit{y}${}^{2}$+6\textit{x}-8\textit{y}-48=0

知识:圆的一般方程

难度:1

题目:设圆\textit{x}${}^{2}$+\textit{y}${}^{2}$-4\textit{x}+2\textit{y}-11=0的圆心为\textit{A},点\textit{P}在圆上,则\textit{PA}的中点\textit{M}的轨迹方程是\_\_\_\_.

解析: 设\textit{M}(\textit{x},\textit{y}),\textit{A}(2,-1),则\textit{P}(2\textit{x}-2,2\textit{y}+1),将\textit{P}代入圆方程得:(2\textit{x}-2)${}^{2}$+(2\textit{y}+1)${}^{2}$-4(2\textit{x}-2)+2(2\textit{y}+1)-11=0,即为:\textit{x}${}^{2}$+\textit{y}${}^{2}$-4\textit{x}+2\textit{y}+1=0.

答案:\textit{x}${}^{2}$+\textit{y}${}^{2}$-4\textit{x}+2\textit{y}+1=0

知识:圆的一般方程

难度:1

题目:判断方程\textit{x}${}^{2}$+\textit{y}${}^{2}$-4\textit{mx}+2\textit{my}+20\textit{m}-20=0能否表示圆,若能表示圆,求出圆心和半径.

解析:

答案:解法一:由方程\textit{x}${}^{2}$+\textit{y}${}^{2}$-4\textit{mx}+2\textit{my}+20\textit{m}-20=0,

可知\textit{D}=-4\textit{m},\textit{E}=2\textit{m},\textit{F}=20\textit{m}-20,

$\mathrm{\therefore}$\textit{D}${}^{2}$+\textit{E}${}^{2}$-4\textit{F}=16\textit{m}${}^{2}$+4\textit{m}${}^{2}$-80\textit{m}+80=20(\textit{m}-2)${}^{2}$,因此,当\textit{m}=2时,\textit{D}${}^{2}$+\textit{E}${}^{2}$-4\textit{F}=0,它表示一个点,当\textit{m}$\mathrm{\neq}$2时,\textit{D}${}^{2}$+\textit{E}${}^{2}$-4\textit{F}$\mathrm{>}$0,原方程表示圆的方程,此时,圆的圆心为(2\textit{m},-\textit{m}),半径为\textit{r}=$\frac{1}{2}\sqrt{D^2+E^2-4F}$=$\sqrt{5}$|\textit{m}-2|.

解法二:原方程可化为(\textit{x}-2\textit{m})${}^{2}$+(\textit{y}+\textit{m})${}^{2}$=5(\textit{m}-2)${}^{2}$,因此,当\textit{m}=2时,它表示一个点,

当\textit{m}$\mathrm{\neq}$2时,原方程表示圆的方程.

此时,圆的圆心为(2\textit{m},-\textit{m}),半径为\textit{r}=$\sqrt{5}$|\textit{m}-2|.

知识:圆的一般方程

难度:1

题目:求过点\textit{A}(-1,0)、\textit{B}(3,0)和\textit{C}(0,1)的圆的方程.

解析:

答案:解法一:设圆的方程为

\textit{x}${}^{2}$+\textit{y}${}^{2}$+\textit{Dx}+\textit{Ey}+\textit{F}=0(*)

把\textit{A}、\textit{B}、\textit{C}三点坐标代入方程(*)得

$\left\{\begin{array}{r} 1-D+F=0\\ 9+3D+F=0\\ 1+E+F=0 \end{array} \right.$,$\mathrm{\therefore}$$\left\{\begin{array}{r} D=-2\\ E=2\\ F=-3 \end{array} \right.$.

故所求圆的方程为\textit{x}${}^{2}$+\textit{y}${}^{2}$-2\textit{x}+2\textit{y}-3=0

解法二:线段\textit{AB}的中垂线方程为\textit{x}=1,线段\textit{AC}的中垂线方程为\textit{x}+\textit{y}=0,

由$\left\{\begin{array}{r} x=1\\ x+y=0 \end{array} \right.$,得圆心坐标为\textit{M}(1,-1),

半径\textit{r}=|\textit{MA}|=$\sqrt{5}$,

$\mathrm{\therefore}$圆的方程为(\textit{x}-1)${}^{2}$+(\textit{y}+1)${}^{2}$=5.

知识:圆的一般方程

难度:2

题目:若圆\textit{x}${}^{2}$+\textit{y}${}^{2}$-2\textit{ax}+3\textit{by}=0的圆心位于第三象限,那么直线\textit{x}+\textit{ay}+\textit{b}=0一定不经过(  )

A.第一象限   B.第二象限  C.第三象限   D.第四象限

解析: 圆\textit{x}${}^{2}$+\textit{y}${}^{2}$-2\textit{ax}+3\textit{by}=0的圆心为(\textit{a},$-\frac{3}{2}$\textit{b}),

则\textit{a}$\mathrm{<}$0,\textit{b}$\mathrm{>}$0.直线\textit{y}=$-\frac{1}{a}$\textit{x}$-\frac{b}{a}$,其斜率\textit{k}=$-\frac{1}{a}$$\mathrm{>}$0,在\textit{y}轴上的截距为$-\frac{b}{a}$$\mathrm{>}$0,所以直线不经过第四象限,故选D.

答案:D

知识:圆的一般方程

难度:2

题目:在圆\textit{x}${}^{2}$+\textit{y}${}^{2}$-2\textit{x}-6\textit{y}=0内,过点\textit{E}(0,1)的最长弦和最短弦分别为\textit{AC}和\textit{BD},则四边形\textit{ABCD}的面只为(  )

A.$5\sqrt{2}$   B.$10\sqrt{2}$  C.$15\sqrt{2}$   D.$20\sqrt{2}$

解析: 圆\textit{x}${}^{2}$+\textit{y}${}^{2}$-2\textit{x}-6\textit{y}=0化成标准方程为(\textit{x}-1)${}^{2}$+(\textit{y}-3)${}^{2}$=10,则圆心坐标为\textit{M}(1,3),半径长为$\sqrt{10}$.由圆的几何性质可知:过点\textit{E}的最长弦\textit{AC}为点\textit{E}所在的直径,则|\textit{AC}|=$2\sqrt{10}$.\textit{BD}是过点\textit{E}的最短弦,则点\textit{E}为线段\textit{BD}的中点,且\textit{AC}$\mathrm{\bot}$\textit{BD},\textit{E}为\textit{AC}与\textit{BD}的交点,则由垂径定理可是|\textit{BD}|=$2\sqrt{|BM|^2-|ME|^2}$=$2\sqrt{10-[(1-0)^2+(3-1)^2]}$=$2\sqrt{5}$.从而四边形\textit{ABCD}的面积为$\frac{1}{2}$|\textit{AC}||\textit{BD}|=$\frac{1}{2}$$\mathrm{\times}$$2\sqrt{10}$$\mathrm{\times}$$2\sqrt{5}$=$10\sqrt{2}$.

答案:B

知识:圆的一般方程

难度:2

题目:若点(2\textit{a},\textit{a}-1)在圆\textit{x}${}^{2}$+\textit{y}${}^{2}$-2\textit{y}-5\textit{a}${}^{2}$=0的内部,则\textit{a}的取值范围是(  )

A.(-$\mathrm{\infty}$,$\frac{4}{5}$]   B.($-\frac{4}{3}$,$\frac{4}{3}$) C.($-\frac{4}{3}$,+$\mathrm{\infty}$)   D.($\frac{4}{3}$,+$\mathrm{\infty}$)

解析: 化圆的标准方程为\textit{x}${}^{2}$+(\textit{y}-1)${}^{2}$=5\textit{a}${}^{2}$+1,点(2\textit{a},\textit{a}-1)的圆的内部,则(2\textit{a})${}^{2}$+(\textit{a}-1-1)${}^{2}$<5\textit{a}${}^{2}$+1,解得\textit{a}>$\frac{3}{4}$.

答案:D

知识:圆的一般方程

难度:2

题目:若直线\textit{l}:\textit{ax}+\textit{by}+1=0始终平分圆\textit{M}:\textit{x}${}^{2}$+\textit{y}${}^{2}$+4\textit{x}+2\textit{y}+1=0的周长,则(\textit{a}-2)${}^{2}$+(\textit{b}-2)${}^{2}$的最小值为(  )

A.$\sqrt{5}$   B.5   C.$2\sqrt{5}$   D.10

解析:由题意,得直线\textit{l}过圆心\textit{M}(-2,-1),

则-2\textit{a}-\textit{b}+1=0,则\textit{b}=-2\textit{a}+1,

所以(\textit{a}-2)${}^{2}$+(\textit{b}-2)${}^{2}$=(\textit{a}-2)${}^{2}$+(-2\textit{a}+1-2)${}^{2}$=5\textit{a}${}^{2}$+5$\mathrm{\ge}$5,

所以(\textit{a}-2)${}^{2}$+(\textit{b}-2)${}^{2}$的最小值为5.

答案:B

知识:圆的一般方程

难度:2

题目:已知圆\textit{C}:\textit{x}${}^{2}$+\textit{y}${}^{2}$+2\textit{x}+\textit{ay}-3=0(\textit{a}为实数)上任意一点关于直线\textit{l}:\textit{x}-\textit{y}+2=0的对称点都在圆\textit{C}上,则\textit{a}=\_\_\_\_.

解析:由题意可知直线\textit{l}:\textit{x}-\textit{y}+2=0过圆心,

$\mathrm{\therefore}$-1+$\frac{a}{2}$+2=0,$\mathrm{\therefore}$\textit{a}=-2.

答案:-2

知识:圆的一般方程

难度:2

题目:若实数\textit{x}、\textit{y}满足\textit{x}${}^{2}$+\textit{y}${}^{2}$+4\textit{x}-2\textit{y}-4=0,则$\sqrt{x^2+y^2}$的最大值是\_\_\_\_.

解析: 关键是搞清式子$\sqrt{x^2+y^2}$的意义.实数\textit{x},\textit{y}满足方程\textit{x}${}^{2}$+\textit{y}${}^{2}$+4\textit{x}-2\textit{y}-4=0,所以(\textit{x},\textit{y})为方程所表示的曲线上的动点$\sqrt{x^2+y^2}$=$\sqrt{(x-0)^2+(y-0)^2}$,表示动点(\textit{x},\textit{y})到原点(0,0)的距离.对方程进行配方,得(\textit{x}+2)${}^{2}$+(\textit{y}-1)${}^{2}$=9,它表示以\textit{C}(-2,1)为圆心,3为半径的圆,而原点的圆内.连接\textit{CO}交圆于点\textit{M},\textit{N},由圆的几何性质可知,\textit{MO}的长即为所求的最大值.

\includegraphics*[width=1.13in, height=1.06in, keepaspectratio=false]{image291}

答案:$\sqrt{5}+3$

知识:圆的一般方程

难度:3

题目:设圆的方程为\textit{x}${}^{2}$+\textit{y}${}^{2}$=4,过点\textit{M}(0,1)的直线\textit{l}交圆于点\textit{A}、\textit{B},\textit{O}是坐标原点,点\textit{P}为\textit{AB}的中点,当\textit{l}绕点\textit{M}旋转时,求动点\textit{P}的轨迹方程.

解析:

答案:设点\textit{P}的坐标为(\textit{x},\textit{y})、\textit{A}(\textit{x}${}_{1}$,\textit{y}${}_{1}$)、\textit{B}(\textit{x}${}_{2}$,\textit{y}${}_{2}$).

因为\textit{A}、\textit{B}在圆上,所以\textit{x}+\textit{y}=4,\textit{x}+\textit{y}=4,

两式相减得\textit{x}-\textit{x}+\textit{y}-\textit{y}=0,

所以(\textit{x}${}_{1}$-\textit{x}${}_{2}$)(\textit{x}${}_{1}$+\textit{x}${}_{2}$)+(\textit{y}${}_{1}$-\textit{y}${}_{2}$)(\textit{y}${}_{1}$+\textit{y}${}_{2}$)=0.

当\textit{x}${}_{1}$$\mathrm{\neq}$\textit{x}${}_{2}$时,有\textit{x}${}_{1}$+\textit{x}${}_{2}$+(\textit{y}${}_{1}$+\textit{y}${}_{2}$)·$\frac{y_1-y_2}{x_1-x_2}$=0,①

并且$\left\{\begin{array}{r} x=\frac{x_1+x_2}{2}\\ y=\frac{y_1+y_2}{2}\\ \frac{y-1}{x}=\frac{y_1-y_2}{x_1-x_2} \end{array} \right.$②

将②代入①并整理得\textit{x}${}^{2}$+(\textit{y}-$\frac{1}{2}$)${}^{2}$=$\frac{1}{4}$.③

当\textit{x}${}_{1}$=\textit{x}${}_{2}$时,点\textit{A}、\textit{B}的坐标为(0,2)、(0,-2),这时点\textit{P}的坐标为(0,0)也满足③.

所以点\textit{P}的轨迹方程为\textit{x}${}^{2}$+(\textit{y}-$\frac{1}{2}$)${}^{2}$=$\frac{1}{4}$.

知识:圆的一般方程

难度:3

题目:已知方程\textit{x}${}^{2}$+\textit{y}${}^{2}$-2(\textit{m}+3)\textit{x}+2(1-4\textit{m}${}^{2}$)\textit{y}+16\textit{m}${}^{4}$+9=0表示一个圆.

(1)求实数\textit{m}的取值范围;

(2)求该圆的半径\textit{r}的取值范围;

(3)求圆心\textit{C}的轨迹方程.

解析:

答案:(1)要使方程表示圆,则

4(\textit{m}+3)${}^{2}$+4(1-4\textit{m}${}^{2}$)${}^{2}$-4(16\textit{m}${}^{4}$+9)>0,

即4\textit{m}${}^{2}$+24\textit{m}+36+4-32\textit{m}${}^{2}$+64\textit{m}${}^{4}$-64\textit{m}${}^{4}$-36>0,

整理得7\textit{m}${}^{2}$-6\textit{m}-1<0,解得$-\frac{1}{7}$<\textit{m}<1.

(2)\textit{r}=$\frac{1}{2}\sqrt{4(m+3)^2+4(1-4m^2)^2-4(16m^2+9)}$=$\sqrt{-7m^2+6m+1}$=$\sqrt{-7(m-\frac{3}{7})^2+\frac{16}{7}}$.

$\mathrm{\therefore}$0<\textit{r}$\mathrm{\le}\frac{4\sqrt{7}}{7}$.

(3)设圆心坐标为(\textit{x},\textit{y}),则$\left\{\begin{array}{r} x=m+3\\ y=4m^2-1\\ \end{array} \right.$.

消去\textit{m}可得(\textit{x}-3)${}^{2}$=$\frac{1}{4}$(\textit{y}+1).

$\mathrm{\because}$$-\frac{1}{7}$<\textit{m}<1,$\mathrm{\therefore}$$\frac{20}{7}$<\textit{x}<4.

故圆心\textit{C}的轨迹方程为(\textit{x}-3)${}^{2}$=$\frac{1}{4}$(\textit{y}+1)($\frac{20}{7}$<\textit{x}<4).


知识:直线与圆的位置关系

难度:1

题目:若直线3\textit{x}+\textit{y}+\textit{a}=0平分圆\textit{x}${}^{2}$+\textit{y}${}^{2}$+2\textit{x}-4\textit{y}=0,则\textit{a}的值为(  )

A.-1   B.1   C.3   D.-3

解析:$\mathrm{\because}$圆心(-1,2)在直线3\textit{x}+\textit{y}+\textit{a}=0上,

$\mathrm{\therefore}$-3+2+\textit{a}=0,$\mathrm{\therefore}$\textit{a}=1.

答案:B

知识:直线与圆的位置关系

难度:1

题目:(2016·高台高一检测)已知直线\textit{ax}+\textit{by}+\textit{c}=0(\textit{a}、\textit{b}、\textit{c}都是正数)与圆\textit{x}${}^{2}$+\textit{y}${}^{2}$=1相切,则以\textit{a}、\textit{b}、\textit{c}为三边长的三角形是(  )

A.锐角三角形   B.直角三角形  C.钝角三角形   D.不存在

解析:由题意,得$\frac{|c|}{\sqrt{a^2+b^2}}$=1,

$\mathrm{\therefore}$\textit{a}${}^{2}$+\textit{b}${}^{2}$=\textit{c}${}^{2}$,故选B.

答案:B

知识:直线与圆的位置关系

难度:1

题目:(2016·北京文)圆(\textit{x}+1)${}^{2}$+\textit{y}${}^{2}$=2的圆心到直线\textit{y}=\textit{x}+3的距离为(  )

A.1   B.2   C.$\sqrt{2}$   D.$2\sqrt{2}$

解析:由圆的标准方程(\textit{x}+1)${}^{2}$+\textit{y}${}^{2}$=2,知圆心为(-1,0),故圆心到直线\textit{y}=\textit{x}+3,即\textit{x}-\textit{y}+3=0的距离\textit{d}=$\frac{|-1-0+3|}{\sqrt{2}}$=$\sqrt{2}$.

答案:C

知识:直线与圆的位置关系

难度:1

题目:(2016·铜仁高一检测)直线\textit{x}+\textit{y}=\textit{m}与圆\textit{x}${}^{2}$+\textit{y}${}^{2}$=\textit{m}(\textit{m}$\mathrm{>}$0)相切,则\textit{m}=(  )

A.$\frac{1}{2}$   B.$\frac{\sqrt{2}}{2}$   C.$\sqrt{2}$   D.2

解析:圆心到直线距离$\frac{|m|}{\sqrt{2}}$=$\sqrt{m}$,解得\textit{m}=2.

答案:D

知识:直线与圆的位置关系

难度:1

题目:圆心坐标为(2,-1)的圆在直线\textit{x}-\textit{y}-1=0上截得的弦长为$2\sqrt{2}$,那么这个圆的方程为(  )

A.(\textit{x}-2)${}^{2}$+(\textit{y}+1)${}^{2}$=4 B.(\textit{x}-2)${}^{2}$+(\textit{y}+1)${}^{2}$=2

C.(\textit{x}-2)${}^{2}$+(\textit{y}+1)${}^{2}$=8 D.(\textit{x}-2)${}^{2}$+(\textit{y}+1)${}^{2}$=16

解析:\textit{d}=$\frac{|2+1-1|}{\sqrt{1+1}}$=$\sqrt{2}$,\textit{r}=$\sqrt{2+2}$=2,
$\mathrm{\therefore}$圆的方程为(\textit{x}-2)${}^{2}$+(\textit{y}+1)${}^{2}$=4.

答案:A

知识:直线与圆的位置关系

难度:1

题目:圆(\textit{x}-3)${}^{2}$+(\textit{y}-3)${}^{2}$=9上到直线3\textit{x}+4\textit{y}-11=0的距离等于1的点有(  )

A.1个   B.2个   C.3个   D.4个

解析: 圆心(3,3)到直线3\textit{x}+4\textit{y}-11=0的距离,\textit{d}=$\frac{|3x3+4x3-11|}{5}$=2,又\textit{r}=3,

故有三个点到直线3\textit{x}+4\textit{y}-11=0的距离等于1.

答案:C

知识:直线与圆的位置关系

难度:1

题目:(2016·天津文)已知圆\textit{C}的圆心在\textit{x}轴的正半轴上,点\textit{M}(0,$\sqrt{5}$)在圆\textit{C}上,且圆心到直线2\textit{x}-\textit{y}=0的距离为$\frac{4\sqrt{5}}{5}$,则圆\textit{C}的方程为\_\_\_\_.

解析: 设圆心为(\textit{a,}0)(\textit{a}$\mathrm{>}$0),则圆心到直线2\textit{x}-\textit{y}=0的距离\textit{d}=$\frac{|2a-0|}{\sqrt{4+1}}$=$\frac{4\sqrt{5}}{5}$,解得\textit{a}=2,半径\textit{r}=$\sqrt{(2-0)^2+(0-\sqrt{5})^2}$=3,所以圆\textit{C}的方程为(\textit{x}-2)${}^{2}$+\textit{y}${}^{2}$=9.

答案:(\textit{x}-2)${}^{2}$+\textit{y}${}^{2}$=9

知识:直线与圆的位置关系

难度:1

题目:过点(3,1)作圆(\textit{x}-2)${}^{2}$+(\textit{y}-2)${}^{2}$=4的弦,其中最短弦的长为\_\_\_\_.

解析: 最短弦为过点(3,1),且垂直于点(3,1)与圆心的连线的弦,易知弦心距\textit{d}=$\sqrt{(3-2)^2+(1-2)^2}$,所以最短弦长为$2\sqrt{r^2-d^2}$=$2\sqrt{2^2-\sqrt{2}^2}$=$2\sqrt{2}$.

答案:2


知识:直线与圆的位置关系

难度:1

题目:当\textit{m}为何值时,直线\textit{x}-\textit{y}-\textit{m}=0与圆\textit{x}${}^{2}$+\textit{y}${}^{2}$-4\textit{x}-2\textit{y}+1=0有两个公共点?有一个公共点?无公共点?

解析:

答案:由$\left\{\begin{array}{r} x-y-m=0\\ x^2+y^2-4x-2y+1=0 \end{array} \right.$,

得2\textit{x}${}^{2}$-2(\textit{m}+3)\textit{x}+\textit{m}${}^{2}$+2\textit{m}+1=0,

$\Delta$=4(\textit{m}+3)${}^{2}$-8(\textit{m}${}^{2}$+2\textit{m}+1)

=-4\textit{m}${}^{2}$+8\textit{m}+28,

当$\Delta$$\mathrm{>}$0,即$-2\sqrt{2}$+1$\mathrm{<}$\textit{m}$\mathrm{<}$$2\sqrt{2}$+1时,直线与圆相交,有两个公共点;

当$\Delta$=0,即\textit{m}=$-2\sqrt{2}$+1或\textit{m}=$2\sqrt{2}$+1时,直线与圆相切,有一个公共点;

当$\Delta$$\mathrm{<}$0,即\textit{m}$\mathrm{<}-2\sqrt{2}$+1或\textit{m}$\mathrm{>}$$2\sqrt{2}$+1时,直线与圆相离,无公共点.

知识:直线与圆的位置关系

难度:1

题目:(2016·潍坊高一检测)已知圆\textit{C}:\textit{x}${}^{2}$+(\textit{y}-1)${}^{2}$=5,直线\textit{l}:\textit{mx}-\textit{y}+1-\textit{m}=0.

(1)求证:对\textit{m}$\mathrm{\in}$R,直线\textit{l}与圆\textit{C}总有两个不同的交点;

(2)若直线\textit{l}与圆\textit{C}交于\textit{A}、\textit{B}两点,当|\textit{AB}|=$\sqrt{17}$时,求\textit{m}的值.

解析:
(1)解法一:由$\left\{\begin{array}{r} x^2+(y-1)^2=5\\ mx-y+1-m=0 \end{array} \right.$,消去\textit{y}整理,得(\textit{m}${}^{2}$+1)\textit{x}${}^{2}$-2\textit{m}${}^{2}$\textit{x}+\textit{m}${}^{2}$-5=0.

$\mathrm{\because}$$\Delta$=(-2\textit{m}${}^{2}$)${}^{2}$-4(\textit{m}${}^{2}$+1)(\textit{m}${}^{2}$-5)=16\textit{m}${}^{2}$+20$\mathrm{>}$0,对一切\textit{m}$\mathrm{\in}$R成立,$\mathrm{\therefore}$直线\textit{l}与圆\textit{C}总有两个不同交点.

解法二:由已知\textit{l}:\textit{y}-1=\textit{m}(\textit{x}-1),

故直线恒过定点\textit{P}(1,1).

$\mathrm{\because}$1${}^{2}$+(1-1)${}^{2}$$\mathrm{<}$5,$\mathrm{\therefore}$\textit{P}(1,1)在圆\textit{C}内.

$\mathrm{\therefore}$直线\textit{l}与圆\textit{C}总有两个不同的交点.

(2)解法一:圆半径\textit{r}=,

圆心(0,1)到直线\textit{l}的距离为\textit{d},

\textit{d}=$\sqrt{x^2-(\frac{|AB|}{2})^2}$=$\frac{\sqrt{3}}{2}$.

由点到直线的距离公式,得$\frac{|-m|}{\sqrt{m^2+(-1)^2}}$=$\frac{\sqrt{3}}{2}$,

解得\textit{m}=$\mathrm{\pm}\sqrt{3}$.

解法二:设\textit{A}(\textit{x}${}_{1}$,\textit{y}${}_{1}$),\textit{B}(\textit{x}${}_{2}$,\textit{y}${}_{2}$),

$\mathrm{\therefore}$|\textit{AB}|=$\sqrt{(x_2-x_1)^2+(y_2-y_1)^2}$

=$\sqrt{(x_2-x_1)^2+[(m-mx_2-1)-(m-mx_1-1)]^2}$

=$\sqrt{m^2+1[(x_1+x_2)^2-4x_1x_2]}$

=$\sqrt{m^2+1[(\frac{2m^2}{m^2+1})^2-4\cdot\frac{m^2-5}{m^2+1}]}$

=$\sqrt{17}$.

$\mathrm{\therefore}$\textit{m}=$\mathrm{\pm}\sqrt{3}$.

知识:直线与圆的位置关系

难度:2

题目:过点(2,1)的直线中,被圆\textit{x}${}^{2}$+\textit{y}${}^{2}$-2\textit{x}+4\textit{y}=0截得的弦最长的直线的方程是(  )

A.3\textit{x}-\textit{y}-5=0   B.3\textit{x}+\textit{y}-7=0 C.3\textit{x}-\textit{y}-1=0   D.3\textit{x}+\textit{y}-5=0

解析:\textit{x}${}^{2}$+\textit{y}${}^{2}$-2\textit{x}+4\textit{y}=0的圆心为(1,-2),截得弦最长的直线必过点(2,1)和圆心(1,-2)

$\mathrm{\therefore}$直线方程为3\textit{x}-\textit{y}-5=0,故选A.

答案:A

知识:直线与圆的位置关系

难度:2

题目:(2016·泰安二中高一检测)已知2\textit{a}${}^{2}$+2\textit{b}${}^{2}$=\textit{c}${}^{2}$,则直线\textit{ax}+\textit{by}+\textit{c}=0与圆\textit{x}${}^{2}$+\textit{y}${}^{2}$=4的位置关系是(  )

A.相交但不过圆心   B.相交且过圆心

C.相切                       D.相离

解析:$\mathrm{\because}$2\textit{a}${}^{2}$+2\textit{b}${}^{2}$=\textit{c}${}^{2}$,

$\mathrm{\therefore}$\textit{a}${}^{2}$+\textit{b}${}^{2}$=$\frac{c^2}{2}$.

$\mathrm{\therefore}$圆心(0,0)到直线\textit{ax}+\textit{by}+\textit{c}=0的距离\textit{d}=$\frac{|c|}{\sqrt{a^2+b^2}}$=$\frac{|c|}{\sqrt{\frac{c^2}{2}}}$=$\sqrt{2}\mathrm{<}$2,

$\mathrm{\therefore}$直线\textit{ax}+\textit{by}+\textit{c}=0与圆\textit{x}${}^{2}$+\textit{y}${}^{2}$=4相交,

又$\mathrm{\because}$点(0,0)不在直线\textit{ax}+\textit{by}+\textit{c}=0上,故选A.

答案:A

知识:直线与圆的位置关系

难度:2

题目:若过点\textit{A}(4,0)的直线\textit{l}与曲线(\textit{x}-2)${}^{2}$+\textit{y}${}^{2}$=1有公共点,则直线\textit{l}的斜率的取值范围为(  )

A.(-$\sqrt{3}$,$\sqrt{3}$)   B.[-$\sqrt{3}$,$\sqrt{3}$] C.(-$\frac{\sqrt{3}}{3}$,$\frac{\sqrt{3}}{3}$)   D.[-$\frac{\sqrt{3}}{3}$,$\frac{\sqrt{3}}{3}$]

解析:

答案:解法一:如图,\textit{BC}=1,\textit{AC}=2,

\includegraphics*[width=1.67in, height=1.21in, keepaspectratio=false]{image293}

$\mathrm{\therefore}$$\mathrm{\angle}$\textit{BAC}=30$\mathrm{{}^\circ}$,

$\mathrm{\therefore}$-$\frac{\sqrt{3}}{3}$$\mathrm{\le}$\textit{k}$\mathrm{\le}$$\frac{\sqrt{3}}{3}$.

解法二:设直线\textit{l}方程为\textit{y}=\textit{k}(\textit{x}-4),则由题意知$\frac{|2k-0-4k|}{\sqrt{1+k^2}}$$\mathrm{\le}$1,$\mathrm{\therefore}$-$\frac{\sqrt{3}}{3}$$\mathrm{\le}$\textit{k}$\mathrm{\le}$$\frac{\sqrt{3}}{3}$.

解法三:过\textit{A}(4,0)的直线\textit{l}可设为\textit{x}=\textit{my}+4,代入(\textit{x}-2)${}^{2}$+\textit{y}${}^{2}$=1中得:

(\textit{m}${}^{2}$+1)\textit{y}${}^{2}$+4\textit{my}+3=0,

由$\Delta$=16\textit{m}${}^{2}$-12(\textit{m}${}^{2}$+1)=4\textit{m}${}^{2}$-12$\mathrm{\ge}$0得

\textit{m}$\mathrm{\le}$-$\sqrt{3}$或\textit{m}$\mathrm{\ge}\sqrt{3}$.

$\mathrm{\therefore}$\textit{l}的斜率\textit{k}=$\mathrm{\in}$[-$\frac{\sqrt{3}}{3}$,0)$\mathrm{\cup}$(0,$\frac{\sqrt{3}}{3}$],特别地,当\textit{k}=0时,显然有公共点,

$\mathrm{\therefore}$\textit{k}$\mathrm{\in}$[-$\frac{\sqrt{3}}{3}$,$\frac{\sqrt{3}}{3}$].

知识:直线与圆的位置关系

难度:2

题目:设圆(\textit{x}-3)${}^{2}$+(\textit{y}+5)${}^{2}$=\textit{r}${}^{2}$(\textit{r}$\mathrm{>}$0)上有且仅有两个点到直线4\textit{x}-3\textit{y}-2=0的距离等于1,则圆半径\textit{r}的取值范围是(  )

A.3$\mathrm{<}$\textit{r}$\mathrm{<}$5   B.4$\mathrm{<}$\textit{r}$\mathrm{<}$6  
C.\textit{r}$\mathrm{>}$4   D.\textit{r}$\mathrm{>}$5

解析:圆心\textit{C}(3,-5),半径为\textit{r},圆心\textit{C}到直线4\textit{x}-3\textit{y}-2=0的距离\textit{d}=$\frac{|12+15-2|}{\sqrt{4^2+(-3)^2}}$=5,由于圆\textit{C}上有且仅有两个点到直线4\textit{x}-3\textit{y}-2=0的距离等于1,则\textit{d}-1$\mathrm{<}$\textit{r}$\mathrm{<}$\textit{d}+1,所以4$\mathrm{<}$\textit{r}$\mathrm{<}$6.

答案:B

知识:直线与圆的位置关系

难度:2

题目:(2016~2017·宜昌高一检测)过点\textit{P}($\frac{1}{2}$,1)的直线\textit{l}与圆\textit{C}:(\textit{x}-1)${}^{2}$+\textit{y}${}^{2}$=4交于\textit{A},\textit{B}两点,\textit{C}为圆心,当$\mathrm{\angle}$\textit{ACB}最小时,直线\textit{l}的方程为\_\_\_\_.

解析:当$\mathrm{\angle}$\textit{ACB}最小时,弦长\textit{AB}最短,此时\textit{CP}$\mathrm{\bot}$\textit{AB}.

由于\textit{C}(1,0),\textit{P}($\frac{1}{2}$,1),$\mathrm{\therefore}$\textit{k${}_{CP}$}=-2,$\mathrm{\therefore}$\textit{k${}_{AB}$}=$\frac{1}{2}$,$\mathrm{\therefore}$直线\textit{l}方程为\textit{y}-1=$\frac{1}{2}$(\textit{x}-$\frac{1}{2}$),即2\textit{x}-4\textit{y}+3=0.

答案:2\textit{x}-4\textit{y}+3=0

知识:直线与圆的位置关系

难度:2

题目:(2016~2017·福州高一检测)过点(-1,-2)的直线\textit{l}被圆\textit{x}${}^{2}$+\textit{y}${}^{2}$-2\textit{x}-2\textit{y}+1=0截得的弦长为$\sqrt{2}$,则直线\textit{l}的斜率为\_\_\_\_.

解析: 圆心\textit{C}(1,1),半径\textit{r}=1,弦长为$\sqrt{2}$.$\mathrm{\therefore}$\textit{C}到\textit{l}的距离\textit{d}=$\sqrt{1^2-(\frac{\sqrt{2}}{2})^2}$=$\frac{\sqrt{2}}{2}$.

设\textit{l}:\textit{y}+2=\textit{k}(\textit{x}+1),即\textit{kx}-\textit{y}+\textit{k}-2=0.

则$\frac{|k-1+k-2|}{\sqrt{1+k^2}}$=$\frac{\sqrt{2}}{2}$.

解之得\textit{k}=1或$\frac{17}{7}$.

答案:1或

知识:直线与圆的位置关系

难度:3

题目:求满足下列条件的圆\textit{x}${}^{2}$+\textit{y}${}^{2}$=4的切线方程:

(1)经过点\textit{P}($\sqrt{3}$,1);

(2)斜率为-1;

(3)过点\textit{Q}(3,0).

解析:

答案:
(1)$\mathrm{\because}$点\textit{P}($\sqrt{3}$,1)在圆上.

$\mathrm{\therefore}$所求切线方程为$\sqrt{3}$\textit{x}+\textit{y}-4=0.

(2)设圆的切线方程为\textit{y}=-\textit{x}+\textit{b},

代入圆的方程,整理得

2\textit{x}${}^{2}$-2\textit{bx}+\textit{b}${}^{2}$-4=0,$\mathrm{\because}$直线与圆相切,

$\mathrm{\therefore}$$\Delta$=(-2\textit{b})${}^{2}$-4$\mathrm{\times}$2(\textit{b}${}^{2}$-4)=0.

解得\textit{b}=$\mathrm{\pm}$$2\sqrt{2}$.

$\mathrm{\therefore}$所求切线方程为\textit{x}+\textit{y}$\mathrm{\pm}$$2\sqrt{2}$=0.

也可用几何法\textit{d}=\textit{r}求解.

(3)解法一:$\mathrm{\because}$3${}^{2}$+0${}^{2}$$\mathrm{>}$4,$\mathrm{\therefore}$点\textit{Q}在圆外.

设切线方程为\textit{y}=\textit{k}(\textit{x}-3),即\textit{kx}-\textit{y}-3\textit{k}=0.

$\mathrm{\because}$直线与圆相切,$\mathrm{\therefore}$圆心到直线的距离等于半径,

$\mathrm{\therefore}$$\frac{|-3k|}{\sqrt{1+k^2}}$=2,$\mathrm{\therefore}$\textit{k}=$\mathrm{\pm}\frac{2}{5}\sqrt{5}$,

$\mathrm{\therefore}$所求切线方程为2\textit{x}$\mathrm{\pm}\sqrt{5}$\textit{y}-6=0.

解法二:设切点为\textit{M}(\textit{x}${}_{0}$,\textit{y}${}_{0}$),则过点\textit{M}的切线方程为

\textit{x}${}_{0}$\textit{x}+\textit{y}${}_{0}$\textit{y}=4,$\mathrm{\because}$点\textit{Q}(3,0)在切线上,$\mathrm{\therefore}$\textit{x}${}_{0}$=$\frac{4}{3}$①

又\textit{M}(\textit{x}${}_{0}$,\textit{y}${}_{0}$)在圆\textit{x}${}^{2}$+\textit{y}${}^{2}$=4上,$\mathrm{\therefore}$\textit{x}+\textit{y}=4②

由①②构成的方程组可解得

$\left\{\begin{array}{r} x_0=\frac{4}{3}\\ y_0=\frac{2\sqrt{5}}{3} \end{array} \right.$,或$\left\{\begin{array}{r} x_0=\frac{4}{3}\\ y_0=-\frac{2\sqrt{5}}{3} \end{array} \right.$.

$\mathrm{\therefore}$所求切线方程为

$\frac{4}{3}$\textit{x}+$\frac{2\sqrt{5}}{3}$\textit{y}=4或$\frac{4}{3}$\textit{x}-$\frac{2\sqrt{5}}{3}$\textit{y}=4,

即2\textit{x}+$\sqrt{5}$\textit{y}-6=0或2\textit{x}-$\sqrt{5}$\textit{y}-6=0.

知识:直线与圆的位置关系

难度:3

题目:设圆上的点\textit{A}(2,3)关于直线\textit{x}+2\textit{y}=0的对称点仍在圆上,且与直线\textit{x}-\textit{y}+1=0相交的弦长为$2\sqrt{2}$,求圆的方程.

解析:

答案: 设圆的方程为(\textit{x}-\textit{a})${}^{2}$+(\textit{y}-\textit{b})${}^{2}$=\textit{r}${}^{2}$.

由已知可知,直线\textit{x}+2\textit{y}=0过圆心,则\textit{a}+2\textit{b}=0, ①

又点\textit{A}在圆上,则(2-\textit{a})${}^{2}$+(3-\textit{b})${}^{2}$=\textit{r}${}^{2}$, ②

因为直线\textit{x}-\textit{y}+1=0与圆相交的弦长为2.

所以($\sqrt{2}$)${}^{2}$+($\frac{a-b+1}{\sqrt{1^2+(-1)^2}}$)${}^{2}$=\textit{r}${}^{2}$ ③

解由①②③所组成的方程组得

$\left\{\begin{array}{r} a=6\\ b=-3\\ r^2=52 \end{array} \right.$,或$\left\{\begin{array}{r} a=14\\ b=-7\\ r^2=244 \end{array} \right.$.

故所求方程为(\textit{x}-6)${}^{2}$+(\textit{y}+3)${}^{2}$=52或(\textit{x}-14)${}^{2}$+(\textit{y}+7)${}^{2}$=244.

知识:圆与圆的位置关系

难度:1

题目:已知圆\textit{C}${}_{1}$:(\textit{x}+1)${}^{2}$+(\textit{y}-3)${}^{2}$=25,圆\textit{C}${}_{2}$与圆\textit{C}${}_{1}$关于点(2,1)对称,则圆\textit{C}${}_{2}$的方程是(  )

A.(\textit{x}-3)${}^{2}$+(\textit{y}-5)${}^{2}$=25 B.(\textit{x}-5)${}^{2}$+(\textit{y}+1)${}^{2}$=25

C.(\textit{x}-1)${}^{2}$+(\textit{y}-4)${}^{2}$=25 D.(\textit{x}-3)${}^{2}$+(\textit{y}+2)${}^{2}$=25

解析:设$\mathrm{\odot}$\textit{C}${}_{2}$上任一点\textit{P}(\textit{x},\textit{y}),它关于(2,1)的对称点(4-\textit{x,}2-\textit{y})在$\mathrm{\odot}$\textit{C}${}_{1}$上,$\mathrm{\therefore}$(\textit{x}-5)${}^{2}$+(\textit{y}+1)${}^{2}$=25.

答案:B

知识:圆与圆的位置关系

难度:1

题目:圆\textit{x}${}^{2}$+\textit{y}${}^{2}$-2\textit{x}-5=0和圆\textit{x}${}^{2}$+\textit{y}${}^{2}$+2\textit{x}-4\textit{y}-4=0的交点为\textit{A}、\textit{B},则线段\textit{AB}的垂直平分线方程为(  )

A.\textit{x}+\textit{y}-1=0    B.2\textit{x}-\textit{y}+1=0

C.\textit{x}-2\textit{y}+1=0    D.\textit{x}-\textit{y}+1=0

解析:解法一:线段\textit{AB}的中垂线即两圆的连心线所在直线\textit{l},由圆心\textit{C}${}_{1}$(1,0),\textit{C}${}_{2}$(-1,2),得\textit{l}方程为\textit{x}+\textit{y}-1=0.

解法二:直线\textit{AB}的方程为:4\textit{x}-4\textit{y}+1=0,因此线段\textit{AB}的垂直平分线斜率为-1,过圆心(1,0),方程为\textit{y}=-(\textit{x}-1),故选A.

答案:A

知识:圆与圆的位置关系

难度:1

题目:若圆(\textit{x}-\textit{a})${}^{2}$+(\textit{y}-\textit{b})${}^{2}$=\textit{b}${}^{2}$+1始终平分圆(\textit{x}+1)${}^{2}$+(\textit{y}+1)${}^{2}$=4的周长,则\textit{a}、\textit{b}应满足的关系式是(  )

A.\textit{a}${}^{2}$-2\textit{a}-2\textit{b}-3=0  B.\textit{a}${}^{2}$+2\textit{a}+2\textit{b}+5=0
C.\textit{a}${}^{2}$+2\textit{b}${}^{2}$+2\textit{a}+2\textit{b}+1=0 D.3\textit{a}${}^{2}$+2\textit{b}${}^{2}$+2\textit{a}+2\textit{b}+1=0

解析:利用公共弦始终经过圆(\textit{x}+1)${}^{2}$+(\textit{y}+1)${}^{2}$=4的圆心即可求得.两圆的公共弦所在直线方程为:(2\textit{a}+2)\textit{x}+(2\textit{b}+2)\textit{y}-\textit{a}${}^{2}$-1=0,它过圆心(-1,-1),代入得\textit{a}${}^{2}$+2\textit{a}+2\textit{b}+5=0.

答案:B

知识:圆与圆的位置关系

难度:1

题目:(2016~2017·太原高一检测)已知半径为1的动圆与圆(\textit{x}-5)${}^{2}$+(\textit{y}+7)${}^{2}$=16相外切,则动圆圆心的轨迹方程是(  )

A.(\textit{x}-5)${}^{2}$+(\textit{y}+7)${}^{2}$=25 B.(\textit{x}-5)${}^{2}$+(\textit{y}+7)${}^{2}$=9

C.(\textit{x}-5)${}^{2}$+(\textit{y}+7)${}^{2}$=15 D.(\textit{x}+5)${}^{2}$+(\textit{y}-7)${}^{2}$=25

解析:设动圆圆心为\textit{P}(\textit{x},\textit{y}),则$\sqrt{(x-5)^2+(y+7)^2}$=4+1,

$\mathrm{\therefore}$(\textit{x}-5)${}^{2}$+(\textit{y}+7)${}^{2}$=25.

故选A.

答案:A

知识:圆与圆的位置关系

难度:1

题目:两圆\textit{x}${}^{2}$+\textit{y}${}^{2}$=16与(\textit{x}-4)${}^{2}$+(\textit{y}+3)${}^{2}$=\textit{r}${}^{2}$(\textit{r}$\mathrm{>}$0)在交点处的切线互相垂直,则\textit{r}
(  )

A.5   B.4   C.3   D.$2\sqrt{2}$

解析:设一个交点\textit{P}(\textit{x}${}_{0}$,\textit{y}${}_{0}$),则\textit{x}+\textit{y}=16,(\textit{x}${}_{0}$-4)${}^{2}$+(\textit{y}${}_{0}$+3)${}^{2}$=\textit{r}${}^{2}$,

$\mathrm{\therefore}$\textit{r}${}^{2}$=41-8\textit{x}${}_{0}$+6\textit{y}${}_{0}$,

$\mathrm{\because}$两切线互相垂直,

$\mathrm{\therefore}$$\frac{y_0}{x_0}\cdot\frac{y_0+3}{x_0-4}$=-1,$\mathrm{\therefore}$3\textit{y}${}_{0}$-4\textit{x}${}_{0}$=-16.

$\mathrm{\therefore}$\textit{r}${}^{2}$=41+2(3\textit{y}${}_{0}$-4\textit{x}${}_{0}$)=9,$\mathrm{\therefore}$\textit{r}=3.

答案:C

知识:圆与圆的位置关系

难度:1

题目:半径长为6的圆与\textit{y}轴相切,且与圆(\textit{x}-3)${}^{2}$+\textit{y}${}^{2}$=1内切,则此圆的方程为(  )

A.(\textit{x}-6)${}^{2}$+(\textit{y}-4)${}^{2}$=6 B.(\textit{x}-6)${}^{2}$+(\textit{y}$\mathrm{\pm}$4)${}^{2}$=6

C.(\textit{x}-6)${}^{2}$+(\textit{y}-4)${}^{2}$=36 D.(\textit{x}-6)${}^{2}$+(\textit{y}$\mathrm{\pm}$4)${}^{2}$=36

解析:半径长为6的圆与\textit{x}轴相切,设圆心坐标为(\textit{a},\textit{b}),则\textit{a}=6,再由$\sqrt{b^2+3^2}$=5可以解得\textit{b}=$\mathrm{\pm}$4,故所求圆的方程为(\textit{x}-6)${}^{2}$+(\textit{y}$\mathrm{\pm}$4)${}^{2}$=36.

答案:D

知识:圆与圆的位置关系

难度:1

题目:圆\textit{x}${}^{2}$+\textit{y}${}^{2}$+6\textit{x}-7=0和圆\textit{x}${}^{2}$+\textit{y}${}^{2}$+6\textit{y}-27=0的位置关系是\_\_\_\_.


解析:圆\textit{x}${}^{2}$+\textit{y}${}^{2}$+6\textit{x}-7=0的圆心为\textit{O}${}_{1}$(-3,0),半径\textit{r}${}_{1}$=4,圆\textit{x}${}^{2}$+\textit{y}${}^{2}$+6\textit{y}-27=0的圆心为\textit{O}${}_{2}$(0,-3),半径为\textit{r}${}_{2}$=6,$\mathrm{\therefore}$|\textit{O}${}_{1}$\textit{O}${}_{2}$|=$\sqrt{(-3-0)^2+(0+3)^2}$=$3\sqrt{2}$,$\mathrm{\therefore}$\textit{r}${}_{2}$-\textit{r}${}_{1}$$\mathrm{<}$|\textit{O}${}_{1}$\textit{O}${}_{2}$|$\mathrm{<}$\textit{r}${}_{1}$+\textit{r}${}_{2}$,故两圆相交.

答案:相交

知识:圆与圆的位置关系

难度:1

题目:若圆\textit{x}${}^{2}$+\textit{y}${}^{2}$=4与圆\textit{x}${}^{2}$+\textit{y}${}^{2}$+2\textit{ay}-6=0(\textit{a}>0)的公共弦长为$2\sqrt{3}$,则\textit{a}=\_\_\_\_.

解析:两个圆的方程作差,可以得到公共弦的直线方程为\textit{y}=$\frac{1}{a}$,圆心(0,0)到直线\textit{y}=$\frac{1}{a}$的距离\textit{d}=|$\frac{1}{a}$|,于是由($\frac{2\sqrt{3}}{2}$)${}^{2}$+|$\frac{1}{a}$|${}^{2}$=2${}^{2}$,解得\textit{a}=1.

答案:1

知识:圆与圆的位置关系

难度:1

题目:求以圆\textit{C}${}_{1}$:\textit{x}${}^{2}$+\textit{y}${}^{2}$-12\textit{x}-2\textit{y}-13=0和圆\textit{C}${}_{2}$:\textit{x}${}^{2}$+\textit{y}${}^{2}$+12\textit{x}+16\textit{y}-25=0的公共弦为直径的圆\textit{C}的方程.

解析:

答案:解法一:联立两圆方程

$\left\{\begin{array}{r} x^2+y^2-12x-2y-13=0\\ x^2+y^2+12x+16y-25=0 \end{array} \right.$,

相减得公共弦所在直线方程为4\textit{x}+3\textit{y}-2=0.

再由$\left\{\begin{array}{r} 4x+3y-2=0\\ x^2+y^2-12x-2y-13=0 \end{array} \right.$,

联立得两圆交点坐标(-1,2)、(5,-6).

$\mathrm{\because}$所求圆以公共弦为直径,

$\mathrm{\therefore}$圆心\textit{C}是公共弦的中点(2,-2),半径为

$\frac{1}{2}\sqrt{(5+1)^2+(-6-2)^2}$=5.

$\mathrm{\therefore}$圆\textit{C}的方程为(\textit{x}-2)${}^{2}$+(\textit{y}+2)${}^{2}$=25.

解法二:由解法一可知公共弦所在直线方程为4\textit{x}+3\textit{y}-2=0.设所求圆的方程为\textit{x}${}^{2}$+\textit{y}${}^{2}$-12\textit{x}-2\textit{y}-13+\textit{$\lambda$}(\textit{x}${}^{2}$+\textit{y}${}^{2}$+12\textit{x}+16\textit{y}-25)=0(\textit{$\lambda$}为参数).

可求得圆心\textit{C}(-$\frac{12\lambda-12}{2(1+\lambda)}$,-$\frac{16\lambda-2}{2(1+\lambda)}$).

$\mathrm{\because}$圆心\textit{C}在公共弦所在直线上,

$\mathrm{\therefore}4\cdot\frac{-(12\lambda-12)}{2(1+\lambda)}+3\cdot\frac{-16(\lambda-2)}{2(1+\lambda)}-$2=0,

解得$\lambda=\frac{1}{2}$.

$\mathrm{\therefore}$圆\textit{C}的方程为\textit{x}${}^{2}$+\textit{y}${}^{2}$-4\textit{x}+4\textit{y}-17=0.

知识:圆与圆的位置关系

难度:1

题目:判断下列两圆的位置关系.

(1)\textit{C}${}_{1}$:\textit{x}${}^{2}$+\textit{y}${}^{2}$-2\textit{x}-3=0,\textit{C}${}_{2}$:\textit{x}${}^{2}$+\textit{y}${}^{2}$-4\textit{x}+2\textit{y}+3=0;

(2)\textit{C}${}_{1}$:\textit{x}${}^{2}$+\textit{y}${}^{2}$-2\textit{y}=0,\textit{C}${}_{2}$:\textit{x}${}^{2}$+\textit{y}${}^{2}$-2\textit{x}-6=0;

(3)\textit{C}${}_{1}$:\textit{x}${}^{2}$+\textit{y}${}^{2}$-4\textit{x}-6\textit{y}+9=0,\textit{C}${}_{2}$:\textit{x}${}^{2}$+\textit{y}${}^{2}$+12\textit{x}+6\textit{y}-19=0;

(4)\textit{C}${}_{1}$:\textit{x}${}^{2}$+\textit{y}${}^{2}$+2\textit{x}-2\textit{y}-2=0,\textit{C}${}_{2}$:\textit{x}${}^{2}$+\textit{y}${}^{2}$-4\textit{x}-6\textit{y}-3=0.

解析:

答案:(1)$\mathrm{\because}$\textit{C}${}_{1}$:(\textit{x}-1)${}^{2}$+\textit{y}${}^{2}$=4,\textit{C}${}_{2}$:(\textit{x}-2)${}^{2}$+(\textit{y}+1)${}^{2}$=2.

$\mathrm{\therefore}$圆\textit{C}${}_{1}$的圆心坐标为(1,0),半径\textit{r}${}_{1}$=2,

圆\textit{C}${}_{2}$的圆心坐标为(2,-1),半径\textit{r}${}_{2}$=$\sqrt{2}$,

\textit{d}=|\textit{C}${}_{1}$\textit{C}${}_{2}$|=$\sqrt{(2-1)^2+(-1)^2}$=$\sqrt{2}$.

$\mathrm{\because}$\textit{r}${}_{1}$+\textit{r}${}_{2}$=2+$\sqrt{2}$,\textit{r}${}_{1}$-\textit{r}${}_{2}$=2-$\sqrt{2}$,

$\mathrm{\therefore}$\textit{r}${}_{1}$-\textit{r}${}_{2}$$\mathrm{<}$\textit{d}$\mathrm{<}$\textit{r}${}_{1}$+\textit{r}${}_{2}$,两圆相交.

(2)$\mathrm{\because}$\textit{C}${}_{1}$:\textit{x}${}^{2}$+(\textit{y}-1)${}^{2}$=1,\textit{C}${}_{2}$:(\textit{x}-$\sqrt{3}$)${}^{2}$+\textit{y}${}^{2}$=9,

$\mathrm{\therefore}$圆\textit{C}${}_{1}$的圆心坐标为(0,1),\textit{r}${}_{1}$=1,

圆\textit{C}${}_{2}$的圆心坐标为($\sqrt{3}$,0),\textit{r}${}_{2}$=3,

\textit{d}=|\textit{C}${}_{1}$\textit{C}${}_{2}$|=$\sqrt{3+1}$=2.

$\mathrm{\because}$\textit{r}${}_{2}$-\textit{r}${}_{1}$=2,$\mathrm{\therefore}$\textit{d}=\textit{r}${}_{2}$-\textit{r}${}_{1}$,两圆内切.

(3)$\mathrm{\because}$\textit{C}${}_{1}$:(\textit{x}-2)${}^{2}$+(\textit{y}-3)${}^{2}$=4,

\textit{C}${}_{2}$:(\textit{x}+6)${}^{2}$+(\textit{y}+3)${}^{2}$=64.

$\mathrm{\therefore}$圆\textit{C}${}_{1}$的圆心坐标为(2,3),半径\textit{r}${}_{1}$=2,

圆\textit{C}${}_{2}$的圆心坐标为(-6,-3),半径\textit{r}${}_{2}$=8,

$\mathrm{\therefore}$|\textit{C}${}_{1}$\textit{C}${}_{2}$|=$\sqrt{(2+6)^2+(3+3)^2}$=10=\textit{r}${}_{1}$+\textit{r}${}_{2}$,

$\mathrm{\therefore}$两圆外切.

(4)\textit{C}${}_{1}$:(\textit{x}+1)${}^{2}$+(\textit{y}-1)${}^{2}$=4,

\textit{C}${}_{2}$:(\textit{x}-2)${}^{2}$+(\textit{y}-3)${}^{2}$=16,

$\mathrm{\therefore}$圆\textit{C}${}_{1}$的圆心坐标为(-1,1),半径\textit{r}${}_{1}$=2,

圆\textit{C}${}_{2}$的圆心坐标为(2,3),半径\textit{r}${}_{2}$=4,

$\mathrm{\therefore}$|\textit{C}${}_{1}$\textit{C}${}_{2}$|=$\sqrt{(2+1)^2+(3-1)^2}$=$\sqrt{13}$.

$\mathrm{\because}$|\textit{r}${}_{1}$-\textit{r}${}_{2}$|$\mathrm{<}$|\textit{C}${}_{1}$\textit{C}${}_{2}$|$\mathrm{<}$\textit{r}${}_{1}$+\textit{r}${}_{2}$,

$\mathrm{\therefore}$两圆相交.

知识:圆与圆的位置关系

难度:2

题目:已知\textit{M}是圆\textit{C}:(\textit{x}-1)${}^{2}$+\textit{y}${}^{2}$=1上的点,\textit{N}是圆\textit{C}$'$:(\textit{x}-4)${}^{2}$+(\textit{y}-4)${}^{2}$=8${}^{2}$上的点,则|\textit{MN}|的最小值为(  )

A.4   B.$4\sqrt{2}-1$  C.$2\sqrt{2}-2$   D.2

解析:$\mathrm{\because}$|\textit{CC}$'$|=5<\textit{R}-\textit{r}=7,

$\mathrm{\therefore}$圆\textit{C}内含于圆\textit{C}$'$,则|\textit{MN}|的最小值为\textit{R}-|\textit{CC}$'$|-\textit{r}=2.

答案:D

知识:圆与圆的位置关系

难度:2

题目:过圆\textit{x}${}^{2}$+\textit{y}${}^{2}$=4外一点\textit{M}(4,-1)引圆的两条切线,则经过两切点的直线方程为(  )

A.4\textit{x}-\textit{y}-4=0   B.4\textit{x}+\textit{y}-4=0 C.4\textit{x}+\textit{y}+4=0   D.4\textit{x}-\textit{y}+4=0

解析:以线段\textit{OM}为直径的圆的方程为\textit{x}${}^{2}$+\textit{y}${}^{2}$-4\textit{x}+\textit{y}=0,经过两切点的直线就是两圆的公共弦所在的直线,将两圆的方程相减得4\textit{x}-\textit{y}-4=0,这就是经过两切点的直线方程.

答案:A

知识:圆与圆的位置关系

难度:2

题目:已知两圆相交于两点\textit{A}(1,3),\textit{B}(\textit{m},-1),两圆圆心都在直线\textit{x}-\textit{y}+\textit{c}=0上,则\textit{m}+\textit{c}的值是(  )

A.-1   B.2 C.3   D.0

解析:两点\textit{A},\textit{B}关于直线\textit{x}-\textit{y}+\textit{c}=0对称,\textit{k${}_{AB}$}=$\frac{-4}{m-1}$=-1.

$\mathrm{\therefore}$\textit{m}=5,线段\textit{AB}的中点(3,1)在直线\textit{x}-\textit{y}+\textit{c}=0上,

$\mathrm{\therefore}$\textit{c}=-2,$\mathrm{\therefore}$\textit{m}+\textit{c}=3.

答案:C

知识:圆与圆的位置关系

难度:2

题目:(2016·山东文)已知圆\textit{M}:\textit{x}${}^{2}$+\textit{y}${}^{2}$-2\textit{ay}=0(\textit{a}$\mathrm{>}$0)截直线\textit{x}+\textit{y}=0所得线段的长度是$2\sqrt{2}$,则圆\textit{M}与圆\textit{N}:(\textit{x}-1)${}^{2}$+(\textit{y}-1)${}^{2}$=1的位置关系是(  )

A.内切   B.相交 C.外切   D.相离

解析:由题知圆\textit{M}:\textit{x}${}^{2}$+(\textit{y}-\textit{a})${}^{2}$=\textit{a}${}^{2}$,圆心(0,\textit{a})到直线\textit{x}+\textit{y}=0的距离\textit{d}=$\frac{a}{\sqrt{2}}$,所以$2\sqrt{a^2-\frac{a^2}{2}}$=$2\sqrt{2}$,解得\textit{a}=2.圆\textit{M}、圆\textit{N}的圆心距|\textit{MN}|=$\sqrt{2}$,两圆半径之差为1、半径之和为3,故两圆相交.

答案:B


知识:圆与圆的位置关系

难度:2

题目:若点\textit{A}(\textit{a},\textit{b})在圆\textit{x}${}^{2}$+\textit{y}${}^{2}$=4上,则圆(\textit{x}-\textit{a})${}^{2}$+\textit{y}${}^{2}$=1与圆\textit{x}${}^{2}$+(\textit{y}-\textit{b})${}^{2}$=1的位置关系是\_\_\_\_.

解析:$\mathrm{\because}$点\textit{A}(\textit{a},\textit{b})在圆\textit{x}${}^{2}$+\textit{y}${}^{2}$=4上,$\mathrm{\therefore}$\textit{a}${}^{2}$+\textit{b}${}^{2}$=4.

又圆\textit{x}${}^{2}$+(\textit{y}-\textit{b})${}^{2}$=1的圆心\textit{C}${}_{1}$(0,\textit{b}),半径\textit{r}${}_{1}$=1,

圆(\textit{x}-\textit{a})${}^{2}$+\textit{y}${}^{2}$=1的圆心\textit{C}${}_{2}$(\textit{a,}0),半径\textit{r}${}_{2}$=1,

则\textit{d}=|\textit{C}${}_{1}$\textit{C}${}_{2}$|=$\sqrt{a^2+b^2}$=$\sqrt{4}$=2,

$\mathrm{\therefore}$\textit{d}=\textit{r}${}_{1}$+\textit{r}${}_{2}$.$\mathrm{\therefore}$两圆外切.

答案:外切

知识:圆与圆的位置关系

难度:2

题目:与直线\textit{x}+\textit{y}-2=0和圆\textit{x}${}^{2}$+\textit{y}${}^{2}$-12\textit{x}-12\textit{y}+54=0都相切的半径最小的圆的标准方程是\_\_\_\_.

解析:已知圆的标准方程为(\textit{x}-6)${}^{2}$+(\textit{y}-6)${}^{2}$=18,则过圆心(6,6)且与直线\textit{x}+\textit{y}-2=0垂直的方程为\textit{x}-\textit{y}=0.方程\textit{x}-\textit{y}=0分别与直线\textit{x}+\textit{y}-2=0和已知圆联立得交点坐标分别为(1,1)和(3,3)或(-3,-3).由题意知所求圆在已知直线和已知圆之间,故所求圆的圆心为(2,2),半径为$\sqrt{2}$,即圆的标准方程为(\textit{x}-2)${}^{2}$+(\textit{y}-2)${}^{2}$=2.

答案:(\textit{x}-2)${}^{2}$+(\textit{y}-2)${}^{2}$=2

知识:圆与圆的位置关系

难度:3

题目:已知圆\textit{M}:\textit{x}${}^{2}$+\textit{y}${}^{2}$-2\textit{mx}-2\textit{ny}+\textit{m}${}^{2}$-1=0与圆\textit{N}:\textit{x}${}^{2}$+\textit{y}${}^{2}$+2\textit{x}+2\textit{y}-2=0交于\textit{A}、\textit{B}两点,且这两点平分圆\textit{N}的圆周,求圆心\textit{M}的轨迹方程.

解析:

答案:两圆方程相减,得公共弦\textit{AB}所在的直线方程为2(\textit{m}+1)\textit{x}+2(\textit{n}+1)\textit{y}-\textit{m}${}^{2}$-1=0,由于\textit{A}、\textit{B}两点平分圆\textit{N}的圆周,所以\textit{A}、\textit{B}为圆\textit{N}直径的两个端点,即直线\textit{AB}过圆\textit{N}的圆心\textit{N},而\textit{N}(-1,-1),所以-2(\textit{m}+1)-2(\textit{n}+1)-\textit{m}${}^{2}$-1=0,即\textit{m}${}^{2}$+2\textit{m}+2\textit{n}+5=0,即(\textit{m}+1)${}^{2}$=-2(\textit{n}+2)(\textit{n}$\mathrm{\le}$-2),由于圆\textit{M}的圆心\textit{M}(\textit{m},\textit{n}),从而可知圆心\textit{M}的轨迹方程为

(\textit{x}+1)${}^{2}$=-2(\textit{y}+2)(\textit{y}$\mathrm{\le}$-2).

知识:圆与圆的位置关系

难度:3

题目:(2016~2017·金华高一检测)已知圆\textit{O}:\textit{x}${}^{2}$+\textit{y}${}^{2}$=1和定点\textit{A}(2,1),由圆\textit{O}外一点\textit{P}(\textit{a},\textit{b})向圆\textit{O}引切线\textit{PQ},切点为\textit{Q},|\textit{PQ}|=|\textit{PA}|成立,如图.

\includegraphics*[width=1.28in, height=1.14in, keepaspectratio=false]{image295}

(1)求\textit{a},\textit{b}间的关系;

(2)求|\textit{PQ}|的最小值.

解析:

答案:(1)连接\textit{OQ},\textit{OP},

则$\mathrm{\vartriangle}$\textit{OQP}为直角三角形,

又|\textit{PQ}|=|\textit{PA}|,

所以|\textit{OP}|${}^{2}$=|\textit{OQ}|${}^{2}$+|\textit{PQ}|${}^{2}$

=1+|\textit{PA}|${}^{2}$,

所以\textit{a}${}^{2}$+\textit{b}${}^{2}$=1+(\textit{a}-2)${}^{2}$+(\textit{b}-1)${}^{2}$,

故2\textit{a}+\textit{b}-3=0.

(2)由(1)知,\textit{P}在直线\textit{l}:2\textit{x}+\textit{y}-3=0上,所以|\textit{PQ}|${}_{min}$=|\textit{PA}|${}_{min}$,为\textit{A}到直线\textit{l}的距离,

所以|\textit{PQ}|${}_{min}$=$\frac{|2x2+1-3|}{\sqrt{2^2+1^2}}$=$\frac{2\sqrt{5}}{5}$.

知识:直线与圆的方程的应用

难度:1

题目:一辆卡车宽1.6 m,要经过一个半圆形隧道(半径为3.6 m),则这辆卡车的平顶车篷篷顶距地面高度不得超过(  )

A.1.4 m   B.3.5 m   C.3.6 m   D.2.0 m

解析:圆半径\textit{OA}=3.6,卡车宽1.6,所以\textit{AB}=0.8,

所以弦心距\textit{OB}=$\sqrt{3.6^2-0.8^2}$$\mathrm{\approx}$3.5(m).

\includegraphics*[width=1.13in, height=0.77in, keepaspectratio=false]{image297}

答案:B

知识:直线与圆的方程的应用

难度:1

题目:已知实数\textit{x}、\textit{y}满足\textit{x}${}^{2}$+\textit{y}${}^{2}$-2\textit{x}+4\textit{y}-20=0,则\textit{x}${}^{2}$+\textit{y}${}^{2}$的最小值是(  )

A.$30-10\sqrt{5}$   B.$5-\sqrt{5}$  C.5   D.25

解析:$\sqrt{x^2+y^2}$为圆上一点到原点的距离.圆心到原点的距离\textit{d}=$\sqrt{5}$,半径为5,所以最小值为(5-$\sqrt{5}$)${}^{2}$=$30-10\sqrt{5}$.

答案:A

知识:直线与圆的方程的应用

难度:1

题目:方程\textit{y}=-$\sqrt{4-x^2}$对应的曲线是(  )

\includegraphics*[width=3.14in, height=0.79in, keepaspectratio=false]{image298}

解析:由方程\textit{y}=-$\sqrt{4-x^2}$得\textit{x}${}^{2}$+\textit{y}${}^{2}$=4(\textit{y}$\mathrm{\le}$0),它表示的图形是圆\textit{x}${}^{2}$+\textit{y}${}^{2}$=4在\textit{x}轴上和以下的部分.

答案:A

知识:直线与圆的方程的应用

难度:1

题目:\textit{y}=|\textit{x}|的图象和圆\textit{x}${}^{2}$+\textit{y}${}^{2}$=4所围成的较小的面积是(  )

A.$\frac{\pi}{4}$   B.$\frac{3\pi}{4}$    C.$\frac{3\pi}{2}$   D.$\pi$

解析:数形结合,所求面积是圆\textit{x}${}^{2}$+\textit{y}${}^{2}$=4面积的$\frac{1}{4}$.

\includegraphics*[width=2.12in, height=1.10in, keepaspectratio=false]{image299}

答案:D

知识:直线与圆的方程的应用

难度:1

题目:方程$\sqrt{1-x^2}$=\textit{x}+\textit{k}有惟一解,则实数\textit{k}的范围是(  )

A.\textit{k}=-$\sqrt{2}$    B.\textit{k}$\mathrm{\in}$(-$\sqrt{2}$,$\sqrt{2}$)

C.\textit{k}$\mathrm{\in}$[-1,1)    D.\textit{k}=$\sqrt{2}$或-1$\mathrm{\le}$\textit{k}$\mathrm{<}$1

解析:由题意知,直线\textit{y}=\textit{x}+\textit{k}与半圆\textit{x}${}^{2}$+\textit{y}${}^{2}$=1(\textit{y}$\mathrm{\ge}$0只有一个交点.结合图形易得-1$\mathrm{\le}$\textit{k}$\mathrm{<}$1或\textit{k}=$\sqrt{2}$.

答案:D

知识:直线与圆的方程的应用

难度:1

题目:点\textit{P}是直线2\textit{x}+\textit{y}+10=0上的动点,直线\textit{PA}、\textit{PB}分别与圆\textit{x}${}^{2}$+\textit{y}${}^{2}$=4相切于\textit{A}、\textit{B}两点,则四边形\textit{PAOB}(\textit{O}为坐标原点)的面积的最小值等于(  )

A.24   B.16   C.8   D.4

解析:$\mathrm{\because}$四边形\textit{PAOB}的面积\textit{S}=2$\mathrm{\times}$$\frac{1}{2}$|\textit{PA}|$\mathrm{\times}$|\textit{OA}|=2$\sqrt{OP^2-OA^2}$=2$\sqrt{OP^2-4}$,$\mathrm{\therefore}$当直线\textit{OP}垂直直线2\textit{x}+\textit{y}+10=0时,其面积\textit{S}最小.

答案:C


知识:直线与圆的方程的应用

难度:1

题目:已知实数\textit{x}、\textit{y}满足\textit{x}${}^{2}$+\textit{y}${}^{2}$=1,则$\frac{y+2}{x+1}$的取值范围为\_\_\_\_.

解析:如右图所示,设\textit{P}(\textit{x},\textit{y})是圆\textit{x}${}^{2}$+\textit{y}${}^{2}$=1上的点,则$\frac{y+2}{x+1}$表示过\textit{P}(\textit{x},\textit{y})和\textit{Q}(-1,-2)两点的直线\textit{PQ}的斜率,过点\textit{Q}作圆的两条切线\textit{QA},\textit{QB},由图可知\textit{QB}$\mathrm{\bot}$\textit{x}轴,\textit{k${}_{QB}$}不存在,且\textit{k${}_{QP}$}$\mathrm{\ge}$\textit{k${}_{QA}$}.

\includegraphics*[width=1.23in, height=1.30in, keepaspectratio=false]{image300}

设切线\textit{QA}的斜率为\textit{k},则它的方程为\textit{y}+2=\textit{k}(\textit{x}+1),由圆心到\textit{QA}的距离为1,得$\frac{|k-2|}{\sqrt{k^2+1}}$=1,解得\textit{k}=$\frac{3}{4}$.所以的取值范围是[$\frac{3}{4}$,+$\mathrm{\infty}$).

答案: [$\frac{3}{4}$,+$\mathrm{\infty}$)

知识:直线与圆的方程的应用

难度:1

题目:已知\textit{M}=$\mathrm{\{}$(\textit{x},\textit{y})|\textit{y}=$\sqrt{9-x^2}$,\textit{y}$\mathrm{\neq}$0$\mathrm{\}}$,\textit{N}=$\mathrm{\{}$(\textit{x},\textit{y})|\textit{y}=\textit{x}+\textit{b}$\mathrm{\}}$,若\textit{M}$\mathrm{\cap}$\textit{N}$\mathrm{\neq}$$\mathrm{\varnothing}$,则实数\textit{b}的取值范围是\_\_ \_\_.

\includegraphics*[width=1.29in, height=1.13in, keepaspectratio=false]{image301}

解析: 数形结合法,注意\textit{y}=$\sqrt{9-x^2}$,\textit{y}$\mathrm{\neq}$0等价于\textit{x}${}^{2}$+\textit{y}${}^{2}$=9(\textit{y}>0),它表示的图形是圆\textit{x}${}^{2}$+\textit{y}${}^{2}$=9在\textit{x}轴之上的部分(如图所示).结合图形不难求得,当-3<\textit{b}$\mathrm{\le}$3$\sqrt{2}$时,直线\textit{y}=\textit{x}+\textit{b}与半圆\textit{x}${}^{2}$+\textit{y}${}^{2}$=9(\textit{y}>0)有公共点.

答案:(-3,3$\sqrt{2}$]


知识:直线与圆的方程的应用

难度:1

题目:为了适应市场需要,某地准备建一个圆形生猪储备基地(如右图),它的附近有一条公路,从基地中心\textit{O}处向东走1 km是储备基地的边界上的点\textit{A},接着向东再走7 km到达公路上的点\textit{B};从基地中心\textit{O}向正北走8 km到达公路的另一点\textit{C}.现准备在储备基地的边界上选一点\textit{D},修建一条由\textit{D}通往公路\textit{BC}的专用线\textit{DE},求\textit{DE}的最短距离.

\includegraphics*[width=0.93in, height=0.92in, keepaspectratio=false]{image302}

解析:

答案:以\textit{O}为坐标原点,过\textit{OB}、\textit{OC}的直线分别为\textit{x}轴和\textit{y}轴,建立平面直角坐标系,则圆\textit{O}的方程为\textit{x}${}^{2}$+\textit{y}${}^{2}$=1,因为点\textit{B}(8,0)、\textit{C}(0,8),所以直线\textit{BC}的方程为$\frac{x}{8}$+$\frac{y}{8}$=1,即\textit{x}+\textit{y}=8.

当点\textit{D}选在与直线\textit{BC}平行的直线(距\textit{BC}较近的一条)与圆相切所成切点处时,\textit{DE}为最短距离,此时\textit{DE}的最小值为$\frac{|0+0-8|}{\sqrt{2}}$-1=(4$\sqrt{2}$-1)km.

知识:直线与圆的方程的应用

难度:1

题目:某圆拱桥的示意图如图所示,该圆拱的跨度\textit{AB}是36 m,拱高\textit{OP}是6 m,在建造时,每隔3 m需用一个支柱支撑,求支柱\textit{A}${}_{2}$\textit{P}${}_{2}$的长.(精确到0.01 m)

\includegraphics*[width=1.78in, height=0.58in, keepaspectratio=false]{image303}

解析:

答案:如图,以线段\textit{AB}所在的直线为\textit{x}轴,线段\textit{AB}的中点\textit{O}为坐标原点建立平面直角坐标系,那么点\textit{A}、\textit{B}、\textit{P}的坐标分别为(-18,0)、(18,0)、(0,6).

\includegraphics*[width=1.79in, height=0.65in, keepaspectratio=false]{image304}

设圆拱所在的圆的方程是\textit{x}${}^{2}$+\textit{y}${}^{2}$+\textit{Dx}+\textit{Ey}+\textit{F}=0.

因为\textit{A}、\textit{B}、\textit{P}在此圆上,故有

$\left\{\begin{array}{r} 18^2-18D+F=0\\ 18^2+18D+F=0\\ 6^2+6E+F=0 \end{array} \right.$,解得$\left\{\begin{array}{r} D=0\\ E=48\\ F=-324 \end{array} \right.$.

故圆拱所在的圆的方程是\textit{x}${}^{2}$+\textit{y}${}^{2}$+48\textit{y}-324=0.

将点\textit{P}${}_{2}$的横坐标\textit{x}=6代入上式,解得\textit{y}=-24+12$\sqrt{6}$.

答:支柱\textit{A}${}_{2}$\textit{P}${}_{2}$的长约为12$\sqrt{6}$-24 m.

知识:直线与圆的方程的应用

难度:2

题目:(2016·葫芦岛高一检测)已知圆\textit{C}的方程是\textit{x}${}^{2}$+\textit{y}${}^{2}$+4\textit{x}-2\textit{y}-4=0,则\textit{x}${}^{2}$+\textit{y}${}^{2}$的最大值为(  )

A.9   B.14 C.14-6$\sqrt{5}$   D.14+6$\sqrt{5}$

解析:圆\textit{C}的标准方程为(\textit{x}+2)${}^{2}$+(\textit{y}-1)${}^{2}$=9,圆心为\textit{C}(-2,1),半径为3.|\textit{OC}|=$\sqrt{5}$,圆上一点(\textit{x},\textit{y})到原点的距离的最大值为3+$\sqrt{5}$,\textit{x}${}^{2}$+\textit{y}${}^{2}$表示圆上的一点(\textit{x},\textit{y})到原点的距离的平方,最大值为(3+$\sqrt{5}$)${}^{2}$=14+6$\sqrt{5}$.

答案:D

知识:直线与圆的方程的应用

难度:2

题目:对于两条平行直线和圆的位置关系定义如下:若两直线中至少有一条与圆相切,则称该位置关系为``平行相切'';若两直线都与圆相离,则称该位置关系为``平行相离'';否则称为``平行相交''.已知直线\textit{l}${}_{1}$:\textit{ax}+3\textit{y}+6=0,\textit{l}${}_{2}$:2\textit{x}+(\textit{a}+1)\textit{y}+6=0与圆\textit{C}:\textit{x}${}^{2}$+\textit{y}${}^{2}$+2\textit{x}=\textit{b}${}^{2}$-1(\textit{b}$\mathrm{>}$0)的位置关系是``平行相交'',则实数\textit{b}的取值范围为(  )

A.($\sqrt{2}$,$\frac{3\sqrt{2}}{2}$)  B.(0,$\frac{3\sqrt{2}}{2}$)

C.(0,$\sqrt{2}$)  D.($\sqrt{2}$,$\frac{3\sqrt{2}}{2}$)$\mathrm{\cup}$($\frac{3\sqrt{2}}{2}$,+$\mathrm{\infty}$)

解析:圆\textit{C}的标准方程为(\textit{x}+1)${}^{2}$+\textit{y}${}^{2}$=\textit{b}${}^{2}$.由两直线平行,可得\textit{a}(\textit{a}+1)-6=0,解得\textit{a}=2或\textit{a}=-3.当\textit{a}=2时,直线\textit{l}${}_{1}$与\textit{l}${}_{2}$重合,舍去;当\textit{a}=-3时,\textit{l}${}_{1}$:\textit{x}-\textit{y}-2=0,\textit{l}${}_{2}$:\textit{x}-\textit{y}+3=0.由\textit{l}${}_{1}$与圆\textit{C}相切,得\textit{b}=$\frac{|-1-2|}{\sqrt{2}}$=$\frac{3\sqrt{2}}{2}$,由\textit{l}${}_{2}$与圆\textit{C}相切,得\textit{b}=$\frac{|-1+3|}{\sqrt{2}}$=$\sqrt{2}$.当\textit{l}${}_{1}$、\textit{l}${}_{2}$与圆\textit{C}都外离时,\textit{b}$\mathrm{<}$$\sqrt{2}$.所以,当\textit{l}${}_{1}$、\textit{l}${}_{2}$与圆\textit{C}``平行相交''时,\textit{b}满足$\left\{\begin{array}{r} b\ge \sqrt{2}\\ b\ne \sqrt{2},b\ne \frac{3\sqrt{2}}{2} \end{array} \right.$,故实数\textit{b}的取值范围是($\sqrt{2}$,$\frac{3\sqrt{2}}{2}$)$\mathrm{\cup}$($\frac{3\sqrt{2}}{2}$,+$\mathrm{\infty}$).

答案:D

知识:直线与圆的方程的应用

难度:2

题目:已知圆的方程为\textit{x}${}^{2}$+\textit{y}${}^{2}$-6\textit{x}-8\textit{y}=0.设该圆过点(3,5)的最长弦和最短弦分别为\textit{AC}和\textit{BD},则四边形\textit{ABCD}的面积为(  )

A.10$\sqrt{6}$   B.20$\sqrt{6}$   C.30$\sqrt{6}$   D.40$\sqrt{6}$

解析:圆心坐标是(3,4),半径是5,圆心到点(3,5)的距离为1,根据题意最短弦\textit{BD}和最长弦(即圆的直径)\textit{AC}垂直,故最短弦的长为2$\sqrt{5^2-1^2}$=4$\sqrt{6}$,所以四边形\textit{ABCD}的面积为$\frac{1}{2}\mathrm{\times}$\textit{AC}$\mathrm{\times}$\textit{BD}=$\frac{1}{2}\mathrm{\times}$10$\mathrm{\times}$4$\sqrt{6}$=20$\sqrt{6}$.

答案:B

知识:直线与圆的方程的应用

难度:2

题目:在平面直角坐标系中,\textit{A},\textit{B}分别是\textit{x}轴和\textit{y}轴上的动点,若以\textit{AB}为直径的圆\textit{C}与直线2\textit{x}+\textit{y}-4=0相切,则圆\textit{C}面积的最小值为(  )

A.$\frac{4\pi}{5}$   B.$\frac{3\pi}{4}$  C.(6-2$\sqrt{5}$)$\pi$   D.$\frac{5\pi}{4}$

解析:原点\textit{O}到直线2\textit{x}+\textit{y}-4=0的距离为\textit{d},则\textit{d}=$\frac{4}{\sqrt{5}}$,点\textit{C}到直线2\textit{x}+\textit{y}-4=0的距离是圆的半径\textit{r},由题知\textit{C}是\textit{AB}的中点,又以斜边为直径的圆过直角顶点,则在直角$\mathrm{\vartriangle}$\textit{AOB}中,圆\textit{C}过原点\textit{O},即|\textit{OC}|=\textit{r},所以2\textit{r}$\mathrm{\ge}$\textit{d},所以\textit{r}最小为$\frac{2}{\sqrt{5}}$,面积最小为$\frac{4\pi}{5}$,故选A.

答案:A


知识:直线与圆的方程的应用

难度:2

题目:某公司有\textit{A}、\textit{B}两个景点,位于一条小路(直道)的同侧,分别距小路 $\sqrt{2}$km和2$\sqrt{2}$ km,且\textit{A}、\textit{B}景点间相距2 km,今欲在该小路上设一观景点,使两景点在同时进入视线时有最佳观赏和拍摄效果,则观景点应设于\_\_\_\_.

解析:所选观景点应使对两景点的视角最大.由平面几何知识,该点应是过\textit{A}、\textit{B}两点的圆与小路所在的直线相切时的切点,以小路所在直线为\textit{x}轴,过\textit{B}点与\textit{x}轴垂直的直线为\textit{y}轴上建立直角坐标系.由题意,得\textit{A}($\sqrt{2}$,$\sqrt{2}$)、\textit{B}(0,2$\sqrt{2}$),设圆的方程为(\textit{x}-\textit{a})${}^{2}$+(\textit{y}-\textit{b})${}^{2}$=\textit{b}${}^{2}$.由\textit{A}、\textit{B}在圆上,得$\left\{\begin{array}{r} a=0\\ b=\sqrt{2} \end{array} \right.$,或$\left\{\begin{array}{r} a=4\sqrt{2}\\ b=5\sqrt{2} \end{array} \right.$,由实际意义知$\left\{\begin{array}{r} a=0\\ b=\sqrt{2} \end{array} \right.$.$\mathrm{\therefore}$圆的方程为\textit{x}${}^{2}$+(\textit{y}-$\sqrt{2}$)${}^{2}$=2,切点为(0,0),$\mathrm{\therefore}$观景点应设在\textit{B}景点在小路的投影处.

答案:\textit{B}景点在小路的投影处

知识:直线与圆的方程的应用

难度:2

题目:设集合\textit{A}=$\mathrm{\{}$(\textit{x},\textit{y})|(\textit{x}-4)${}^{2}$+\textit{y}${}^{2}$=1$\mathrm{\}}$,\textit{B}=$\mathrm{\{}$(\textit{x},\textit{y})|(\textit{x}-\textit{t})${}^{2}$+(\textit{y}-\textit{at}+2)${}^{2}$=1$\mathrm{\}}$,若存在实数\textit{t},使得\textit{A}$\mathrm{\cap}$\textit{B}$\mathrm{\neq}$$\mathrm{\varnothing}$,则实数\textit{a}的取值范围是\_\_ \_\_.

解析:首先集合\textit{A}、\textit{B}实际上是圆上的点的集合,即\textit{A}、\textit{B}表示两个圆,\textit{A}$\mathrm{\cap}$\textit{B}$\mathrm{\neq}$$\mathrm{\varnothing}$说明这两个圆相交或相切(有公共点),由于两圆半径都是1,因此两圆圆心距不大于半径之和2,即$\sqrt{(t-4)^2+(at-2)^2}$$\mathrm{\le}$2,整理成关于\textit{t}的不等式:(\textit{a}${}^{2}$+1)\textit{t}${}^{2}$-4(\textit{a}+2)\textit{t}+16$\mathrm{\le}$0,据题意此不等式有实解,因此其判别式不小于零,即\textit{$\mathit{\Delta}$}=16(\textit{a}+2)${}^{2}$-4(\textit{a}${}^{2}$+1)$\mathrm{\times}$16$\mathrm{\ge}$0,解得0$\mathrm{\le}$\textit{a}$\mathrm{\le}\frac{4}{3}$.

答案:[0,$\frac{4}{3}$]

知识:直线与圆的方程的应用

难度:3

题目:如图,已知一艘海监船\textit{O}上配有雷达,其监测范围是半径为25 km的圆形区域,一艘外籍轮船从位于海监船正东40 km的\textit{A}处出发,径直驶向位于海监船正北30 km的\textit{B}处岛屿,速度为28 km/h.

\includegraphics*[width=1.17in, height=0.98in, keepaspectratio=false]{image305}

问:这艘外籍轮船能否被海监船监测到?若能,持续时间多长?(要求用坐标法)

解析:

答案:如图,以\textit{O}为原点,东西方向为\textit{x}轴建立直角坐标系,则\textit{A}(40,0),\textit{B}(0,30),圆\textit{O}方程\textit{x}${}^{2}$+\textit{y}${}^{2}$=25${}^{2}$.

\includegraphics*[width=1.04in, height=0.87in, keepaspectratio=false]{image306}

直线\textit{AB}方程:$\frac{x}{40}$+$\frac{y}{30}$=1,即3\textit{x}+4\textit{y}-120=0.

设\textit{O}到\textit{AB}距离为\textit{d},则\textit{d}=$\frac{|-120|}{5}$=24$\mathrm{<}$25,

所以外籍轮船能被海监船监测到.

设监测时间为\textit{t},则\textit{t}=$\frac{2\sqrt{25^2-24^2}}{28}$=$\frac{1}{2}$(h)

答:外籍轮船能被海监船监测到,时间是0.5 h.

知识:直线与圆的方程的应用

难度:3

题目:已知隧道的截面是半径为4.0 m的半圆,车辆只能在道路中心线一侧行驶,一辆宽为2.7 m、高为3 m的货车能不能驶入这个隧道?假设货车的最大宽度为\textit{a} m,那么要正常驶入该隧道,货车的限高为多少?

\includegraphics*[width=1.09in, height=0.88in, keepaspectratio=false]{image307}

解析:

答案:以某一截面半圆的圆心为坐标原点,半圆的直径\textit{AB}所在的直线为\textit{x}轴,建立如图所示的平面直角坐标系,那么半圆的方程为:\textit{x}${}^{2}$+\textit{y}${}^{2}$=16(\textit{y}$\mathrm{\ge}$0).

将\textit{x}=2.7代入,得

\textit{y}=$\sqrt{16-2.7^2}$=$\sqrt{8.71}$$\mathrm{<}$3,

所以,在离中心线2.7 m处,隧道的高度低于货车的高度,因此,货车不能驶入这个隧道.

将\textit{x}=\textit{a}代入\textit{x}${}^{2}$+\textit{y}${}^{2}$=16(\textit{y}$\mathrm{\ge}$0)得\textit{y}=$\sqrt{16-a^2}$.

所以,货车要正常驶入这个隧道,最大高度(即限高)为$\sqrt{16-a^2}$ m.

知识:空间直角坐标系

难度:1

题目:下列命题中错误的是(  )

A.在空间直角坐标系中,在\textit{x}轴上的点的坐标一定是(0,\textit{b},\textit{c})

B.在空间直角坐标系中,在\textit{yOz}平面上的点的坐标一定是(0,\textit{b},\textit{c})

C.在空间直角坐标系中,在\textit{z}轴上的点的坐标可记作(0,0,\textit{c})

D.在空间直角坐标系中,在\textit{xOz}平面上的点的坐标是(\textit{a,}0,\textit{c})

解析:空间直角坐标系中,在\textit{x}轴上的点的坐标是(\textit{a,}0,0).

答案:A

知识:空间直角坐标系

难度:1

题目:在空间直角坐标系中,点\textit{M}(3,0,2)位于(  )

A.\textit{y}轴上   B.\textit{x}轴上 C.\textit{xOz}平面内   D.\textit{yOz}平面内

解析:由\textit{x}=3,\textit{y}=0,\textit{z}=2可知点\textit{M}位于\textit{xOz}平面内.

答案:C

知识:空间直角坐标系

难度:1

题目:(2016~2017·襄阳高一检测)若已知点\textit{M}(3,4,1),点\textit{N}(0,0,1),则线段\textit{MN}的长为(  )

A.5   B.0   C.3   D.1

解析:|\textit{MN}|=$\sqrt{(3-0)^2+(4-0)^2+(1-1)^2}$=5.

答案:A

知识:空间直角坐标系

难度:1

题目:在空间直角坐标系中,已知\textit{A}(1,-2,1),\textit{B}(2,2,2),点\textit{P}在\textit{z}轴上,且满足|\textit{PA}|=|\textit{PB}|,则\textit{P}点坐标为(  )

A.(3,0,0)   B.(0,3,0) C.(0,0,3)   D.(0,0,-3)

解析:设\textit{P}(0,0,\textit{z}),则有$\sqrt{1^2+(-2)^2+(z-1)^2}$=$\sqrt{2^2+2^2+(z-2)^2}$,解得\textit{z}=3.

答案:C

知识:空间直角坐标系

难度:1

题目:点\textit{P}(-1,2,3)关于\textit{xOz}平面对称的点的坐标是(  )

A.(1,2,3)          B.(-1,-2,3)

C.(-1,2,-3)       D.(1,-2,-3)

解析:点\textit{P}(-1,2,3)关于\textit{xOz}平面对称的点的坐标是(-1,-2,3),故选B.

答案:B

知识:空间直角坐标系

难度:1

题目:已知点\textit{A}(-3,1,5)与点\textit{B}(4,3,1),则\textit{AB}的中点坐标是(  )

A.($\frac{7}{2}$,1,-2)   B.($\frac{1}{2}$,2,3) C.(-12,3,5)   D.($\frac{1}{3}$,$\frac{4}{3}$,2)

解析:

答案:B

知识:空间直角坐标系

难度:1

题目:.如图所示,在长方体\textit{OABC}-\textit{O}${}_{1}$\textit{A}${}_{1}$\textit{B}${}_{1}$\textit{C}${}_{1}$中,|\textit{OA}|=2,|\textit{AB}|=3,|\textit{AA}${}_{1}$|=2,\textit{M}是\textit{OB}${}_{1}$与\textit{BO}${}_{1}$的交点,则\textit{M}点的坐标是\_\_\_\_.

\includegraphics*[width=1.23in, height=0.96in, keepaspectratio=false]{image309}

解析:由长方体性质可知,\textit{M}为\textit{OB}${}_{1}$中点,而\textit{B}${}_{1}$(2,3,2),故\textit{M}(1,$\frac{3}{2}$,1).

答案:(1,$\frac{3}{2}$,1)

知识:空间直角坐标系

难度:1

题目:在$\mathrm{\vartriangle}$\textit{ABC}中,已知\textit{A}(-1,2,3)、\textit{B}(2,-2,3)、\textit{C}($\frac{1}{2}$,$\frac{5}{2}$,3),则\textit{AB}边上的中线\textit{CD}的长是\_\_\_\_.

解析: \textit{AB}中点\textit{D}坐标为($\frac{1}{2}$,0,3),

|\textit{CD}|=$\sqrt{(\frac{1}{2}-\frac{1}{2})^2+(\frac{5}{0}-0)^2+(3-3)^2}$=$\frac{5}{2}$.

答案:$\frac{5}{2}$

知识:空间直角坐标系

难度:1

题目:已知点\textit{A}(\textit{x,}5-\textit{x,}2\textit{x}-1)、\textit{B}(1,\textit{x}+2,2-\textit{x}),求|\textit{AB}|的最小值.

解析:

答案:$\mathrm{\because}$|\textit{AB}|=

$\sqrt{(x-1)^2+(3-2x)^2+(3x-3)^2}$=$\sqrt{14x^2-32x+19}$=$\sqrt{14(x-\frac{8}{7})^2+\frac{5}{7}}$$\mathrm{\ge}$$\frac{\sqrt{35}}{7}$,

当\textit{x}=$\frac{8}{7}$时,|\textit{AB}|取最小值$\frac{\sqrt{35}}{7}$.

知识:空间直角坐标系

难度:1

题目:长方体\textit{ABCD}-\textit{A}${}_{1}$\textit{B}${}_{1}$\textit{C}${}_{1}$\textit{D}${}_{1}$中,\textit{AB}=\textit{BC}=2,\textit{D}${}_{1}$\textit{D}=3,点\textit{M}是\textit{B}${}_{1}$\textit{C}${}_{1}$的中点,点\textit{N}是\textit{AB}的中点.建立如图所示的空间直角坐标系.

\includegraphics*[width=1.06in, height=1.41in, keepaspectratio=false]{image310}

(1)写出点\textit{D}、\textit{N}、\textit{M}的坐标;

(2)求线段\textit{MD}、\textit{MN}的长度.

解析:

答案:(1)因为\textit{D}是原点,则\textit{D}(0,0,0).

由\textit{AB}=\textit{BC}=2,\textit{D}${}_{1}$\textit{D}=3,

得\textit{A}(2,0,0)、\textit{B}(2,2,0)、\textit{C}(0,2,0)、\textit{B}${}_{1}$(2,2,3)、\textit{C}${}_{1}$(0,2,3).

$\mathrm{\because}$\textit{N}是\textit{AB}的中点,$\mathrm{\therefore}$\textit{N}(2,1,0).

同理可得\textit{M}(1,2,3).

(2)由两点间距离公式,得

|\textit{MD}|=$\sqrt{(1-0)^2+(2-0)^2}+(3-0)^2$=$\sqrt{14}$,

|\textit{MN}|=$\sqrt{(1-2)^2+(2-1)^2}+(3-0)^2$=$\sqrt{11}$.

知识:空间直角坐标系

难度:2

题目:(2016·大同高一检测)空间直角坐标系中,\textit{x}轴上到点\textit{P}(4,1,2)的距离为$\sqrt{30}$的点有(  )

A.2个   B.1个   C.0个   D.无数个

解析:设\textit{x}轴上满足条件的点为\textit{B}(\textit{x,}0,0),则由|\textit{PB}|=$\sqrt{30}$,

得$\sqrt{(x-4)^2+(0-1)^2}+(0-20)^2$=$\sqrt{30}$.

解之得\textit{x}=-1或9.

故选A.

答案:A

知识:空间直角坐标系

难度:2

题目:正方体不在同一面上的两顶点\textit{A}(-1,2,-1)、\textit{B}(3,-2,3),则正方体的体积是(  )

A.16   B.192   C.64   D.48

解析:|\textit{AB}|=$\sqrt{(3+1)^2+(-2-2)^2+(3+1)^2}$=$4\sqrt{3}$,

$\mathrm{\therefore}$正方体的棱长为$\frac{4\sqrt{3}}{\sqrt{3}}$=4.

$\mathrm{\therefore}$正方体的体积为4${}^{3}$=64.

答案:C

知识:空间直角坐标系

难度:2

题目:已知$\mathrm{\vartriangle}$\textit{ABC}的顶点坐标分别为\textit{A}(1,-2,11)、\textit{B}(4,2,3)、\textit{C}(6,-1,4),则$\mathrm{\vartriangle}$\textit{ABC}是(  )

A.直角三角形   B.钝角三角形

C.锐角三角形   D.等腰三角形

解析:由两点间距离公式得|\textit{AB}|=$\sqrt{89}$,|\textit{AC}|=$\sqrt{75}$,|\textit{BC}|=$\sqrt{14}$,满足|\textit{AB}|${}^{2}$=|\textit{AC}|${}^{2}$+|\textit{BC}|${}^{2}$.

答案:A

知识:空间直角坐标系

难度:2

题目:$\mathrm{\vartriangle}$\textit{ABC}的顶点坐标是\textit{A}(3,1,1),\textit{B}(-5,2,1),\textit{C}(-$\frac{8}{3}$,2,3),则它在\textit{yOz}平面上射影图形的面积是(  )

A.4   B.3   C.2   D.1

解析: $\mathrm{\vartriangle}$\textit{ABC}的顶点在\textit{yOz}平面上的射影点的坐标分别为\textit{A}$'$(0,1,1)、\textit{B}$'$(0,2,1)、\textit{C}$'$(0,2,3),$\mathrm{\vartriangle}$\textit{ABC}在\textit{yOz}平面上的射影是一个直角三角形\textit{A}$'$\textit{B}$'$\textit{C}$'$,容易求出它的面积为1.

答案:D


知识:空间直角坐标系

难度:2

题目:已知\textit{P}($\frac{3}{2}$,$\frac{5}{2}$,\textit{z})到直线\textit{AB}中点的距离为3,其中\textit{A}(3,5,-7)、\textit{B}(-2,4,3),则\textit{z}=\_\_\_\_.

解析:利用中点坐标公式可得\textit{AB}中点\textit{C}($\frac{1}{2}$,$\frac{9}{2}$,-2),因为|\textit{PC}|=3,所以

$\sqrt{(\frac{3}{2}-\frac{1}{2})^2+(\frac{5}{2}-\frac{9}{2})^2+[z-(-2)]^2}$=3,解得\textit{z}=0或\textit{z}=-4.

答案:0或-4

知识:空间直角坐标系

难度:2

题目:在空间直角坐标系中,正方体\textit{ABCD}---\textit{A}${}_{1}$\textit{B}${}_{1}$\textit{C}${}_{1}$\textit{D}${}_{1}$的顶点\textit{A}(3,-1,2),其中心\textit{M}的坐标为(0,1,2),则该正方体的棱长为\_\_\_\_.

解析:|\textit{AM}|=$\sqrt{(3-0)^2+(-1-1)^2+(2-2)^2}$

=$\sqrt{13}$,$\mathrm{\therefore}$对角线|\textit{AC}${}_{1}$|=2$\sqrt{13}$,

设棱长\textit{x},则3\textit{x}${}^{2}$=$(2\sqrt{13})^2$,$\mathrm{\therefore}$\textit{x}=$\frac{2\sqrt{39}}{3}$.

答案:$\frac{2\sqrt{39}}{3}$

知识:空间直角坐标系

难度:3

题目:如图所示,已知正方体\textit{ABCD}-\textit{A}${}_{1}$\textit{B}${}_{1}$\textit{C}${}_{1}$\textit{D}${}_{1}$的棱长为\textit{a},过点\textit{B}${}_{1}$作\textit{B}${}_{1}$\textit{E}$\mathrm{\bot}$\textit{BD}${}_{1}$于点\textit{E},求\textit{A}、\textit{E}两点之间的距离.

\includegraphics*[width=1.08in, height=1.00in, keepaspectratio=false]{image311}

解析:

答案: 根据题意,可得\textit{A}(\textit{a,}0,0)、\textit{B}(\textit{a},\textit{a,}0)、\textit{D}${}_{1}$(0,0,\textit{a})、\textit{B}${}_{1}$(\textit{a},\textit{a},\textit{a}).

过点\textit{E}作\textit{EF}$\mathrm{\bot}$\textit{BD}于\textit{F},如图所示,

则在Rt$\mathrm{\vartriangle}$\textit{BB}${}_{1}$\textit{D}${}_{1}$中,

|\textit{BB}${}_{1}$|=\textit{a},|\textit{BD}${}_{1}$|=$\sqrt{3}$\textit{a},|\textit{B}${}_{1}$\textit{D}${}_{1}$|=$\sqrt{2}$\textit{a},

所以|\textit{B}${}_{1}$\textit{E}|=$\frac{a\cdot \sqrt{2}a}{\sqrt{3}a}$=$\frac{\sqrt{6}a}{3}$,

所以Rt$\mathrm{\vartriangle}$\textit{BEB}${}_{1}$中,|\textit{BE}|=$\frac{\sqrt{3}}{3}$\textit{a}

由Rt$\mathrm{\vartriangle}$\textit{BEF}$\mathrm{\backsim}$Rt$\mathrm{\vartriangle}$\textit{BD}${}_{1}$\textit{D},得|\textit{BF}|=$\frac{\sqrt{2}}{3}$\textit{a},|\textit{EF}|=$\frac{a}{3}$,所以点\textit{F}的坐标为($\frac{2a}{3}$,$\frac{2a}{3}$,0),

则点\textit{E}的坐标为($\frac{2a}{3}$,$\frac{2a}{3}$,$\frac{a}{3}$).

由两点间的距离公式,得

|\textit{AE}|=$\sqrt{(a-\frac{2a}{3})^2+(0-\frac{2a}{3})^2+(0-\frac{a}{3})^2}$=$\frac{\sqrt{6}}{3}$\textit{a},

所以\textit{A}、\textit{E}两点之间的距离是$\frac{\sqrt{6}}{3}$\textit{a}.

知识:空间直角坐标系

难度:3

题目:如图所示,\textit{V}-\textit{ABCD}是正棱锥,\textit{O}为底面中心,\textit{E}、\textit{F}分别为\textit{BC}、\textit{CD}的中点.已知|\textit{AB}|=2,|\textit{VO}|=3,建立如右所示空间直角坐标系,试分别写出各个顶点的坐标.

\includegraphics*[width=1.25in, height=1.23in, keepaspectratio=false]{image312}

解析:

答案:$\mathrm{\because}$底面是边长为2的正方形,

$\mathrm{\therefore}$|\textit{CE}|=|\textit{CF}|=1.

$\mathrm{\because}$\textit{O}点是坐标原点,

$\mathrm{\therefore}$\textit{C}(1,1,0),同样的方法可以确定\textit{B}(1,-1,0)、\textit{A}(-1,-1,0)、\textit{D}(-1,1,0).

$\mathrm{\because}$\textit{V}在\textit{z}轴上,$\mathrm{\therefore}$\textit{V}(0,0,3).



知识:圆的标准方程,圆的一般方程

难度:1

题目:圆\textit{x}${}^{2}$+\textit{y}${}^{2}$+\textit{x}-3\textit{y}-$\frac{3}{2}$=0的半径是(  )

A.1    B.$\sqrt{2}$  C.2   D.$2\sqrt{2}$

解析: 圆\textit{x}${}^{2}$+\textit{y}${}^{2}$+\textit{x}-3\textit{y}-$\frac{3}{2}$=0化为标准方程为(\textit{x}+$\frac{1}{2}$)${}^{2}$+(\textit{y}-$\frac{3}{2}$)${}^{2}$=4,$\mathrm{\therefore}$\textit{r}=2.

答案:C

知识:直线与圆的位置关系

难度:1

题目:已知点\textit{A}(\textit{x,}1,2)和点\textit{B}(2,3,4),且|\textit{AB}|=$2\sqrt{6}$,则实数\textit{x}的值是(  )

A.-3或4   B.6或2  C.3或-4   D.6或-2

解析:由空间两点间的距离公式得

$\sqrt{(x-2)^2+(1-3)^2+(2-4)^2}$=$2\sqrt{6}$,解得\textit{x}=6或\textit{x}=-2.

答案:D

知识:圆与圆的位置关系

难度:1

题目:圆\textit{O}${}_{1}$:\textit{x}${}^{2}$+\textit{y}${}^{2}$-2\textit{x}=0与圆\textit{O}${}_{2}$:\textit{x}${}^{2}$+\textit{y}${}^{2}$-4\textit{y}=0的位置关系是(  )

A.外离   B.相交 C.外切   D.内切

解析: 圆\textit{O}${}_{1}$(1,0),\textit{r}${}_{1}$=1,圆\textit{O}${}_{2}$(0,2),\textit{r}${}_{2}$=2,|\textit{O}${}_{1}$\textit{O}${}_{2}$|=$\sqrt{(1-0)^2+(0-2)^2}$=$\sqrt{5}$<1+2,且$\sqrt{5}$>2-1,故两圆相交.

答案:B

知识:直线与圆的位置关系

难度:1

题目:数轴上三点\textit{A}、\textit{B}、\textit{C},已知\textit{AB}=2.5,\textit{BC}=-3,若\textit{A}点坐标为0,则\textit{C}点坐标为(  )

A.0.5   B.-0.5   C.5.5   D.-5.5

解析: 由已知得,\textit{x${}_{B}$}-\textit{x${}_{A}$}=2.5,\textit{x${}_{C}$}-\textit{x${}_{B}$}=-3,且\textit{x${}_{A}$}=0,$\mathrm{\therefore}$两式相加得,\textit{x${}_{C}$}-\textit{x${}_{A}$}=-0.5,即\textit{x${}_{C}$}=-0.5.

答案:B

知识:圆的一般方程

难度:1

题目:(2016·沧州高一检测)方程\textit{x}${}^{2}$+\textit{y}${}^{2}$+\textit{ax}+2\textit{ay}+$\frac{5}{4}$\textit{a}${}^{2}$+\textit{a}-1=0表示圆,则\textit{a}的取值范围是(  )

A.\textit{a}$\mathrm{<}$-2或\textit{a}$\mathrm{>}\frac{2}{3}$   B.-$\frac{2}{3}\mathrm{<}$\textit{a}$\mathrm{<}$2  C.\textit{a}$\mathrm{>}$1   D.\textit{a}$\mathrm{<}$1

解析: 由题意知,\textit{a}${}^{2}$+(2\textit{a})${}^{2}$-4$(\frac{5}{4}a^2+a-1)$=-4\textit{a}+4$\mathrm{>}$0.

$\mathrm{\therefore}$\textit{a}$\mathrm{<}$1.故选D.

答案:D

知识:直线与圆的位置关系

难度:1

题目:已知圆\textit{C}:\textit{x}${}^{2}$+\textit{y}${}^{2}$-4\textit{y}=0,直线\textit{l}过点\textit{P}(0,1),则(  )

A.\textit{l}与\textit{C}相交    B.\textit{l}与\textit{C}相切 

C.\textit{l}与\textit{C}相离    D.以上三个选项均有可能

解析:$\mathrm{\because}$圆\textit{C}的圆心坐标为(0,2),

半径\textit{r}=2,$\mathrm{\therefore}$|\textit{CP}|=1$\mathrm{<}$2,

$\mathrm{\therefore}$点\textit{P}(0,1)在内部,

$\mathrm{\therefore}$直线\textit{l}与\textit{C}相交.

答案:A

知识:直线与圆的位置关系

难度:2

题目:(2016~2017·南平高一检测)以(-2,1)为圆心且与直线\textit{x}+\textit{y}=3相切的圆的方程为(  )

A.(\textit{x}-2)${}^{2}$+(\textit{y}+1)${}^{2}$=2 B.(\textit{x}+2)${}^{2}$+(\textit{y}-1)${}^{2}$=4

C.(\textit{x}-2)${}^{2}$+(\textit{y}+1)${}^{2}$=8 D.(\textit{x}+2)${}^{2}$+(\textit{y}-1)${}^{2}$=8

解析: 由所求的圆与直线\textit{x}+\textit{y}-3=0相切,$\mathrm{\therefore}$圆心(-2,1)到直线\textit{x}+\textit{y}-3=0的距离\textit{d}=$\frac{|-2+1-3|}{\sqrt{2}}$=$2\sqrt{2}$,

$\mathrm{\therefore}$所求圆的方程为(\textit{x}+2)${}^{2}$+(\textit{y}-1)${}^{2}$=8.

答案:D

知识:直线与圆的位置关系

难度:1

题目:当\textit{a}为任意实数时,直线(\textit{a}-1)\textit{x}-\textit{y}+\textit{a}+1=0恒过定点\textit{C},则以\textit{C}为圆心,$\sqrt{5}$为半径的圆的方程为(  )

A.\textit{x}${}^{2}$+\textit{y}${}^{2}$-2\textit{x}+4\textit{y}=0   B.\textit{x}${}^{2}$+\textit{y}${}^{2}$+2\textit{x}+4\textit{y}=0

C.\textit{x}${}^{2}$+\textit{y}${}^{2}$+2\textit{x}-4\textit{y}=0   D.\textit{x}${}^{2}$+\textit{y}${}^{2}$-2\textit{x}-4\textit{y}=0

解析: 由(\textit{a}-1)\textit{x}-\textit{y}+\textit{a}+1=0得\textit{a}(\textit{x}+1)-(\textit{x}+\textit{y}-1)=0,

所以直线恒过定点(-1,2),

所以圆的方程为(\textit{x}+1)${}^{2}$+(\textit{y}-2)${}^{2}$=5,

即\textit{x}${}^{2}$+\textit{y}${}^{2}$+2\textit{x}-4\textit{y}=0.

答案:C

知识:直线与圆的位置关系

难度:2

题目:(2016·葫芦岛高一检测)已知圆\textit{C}方程为(\textit{x}-2)${}^{2}$+(\textit{y}-1)${}^{2}$=9,直线\textit{l}的方程为3\textit{x}-4\textit{y}-12=0,在圆\textit{C}上到直线\textit{l}的距离为1的点有几个(  )

A.4   B.3  C.2   D.1

解析:圆心\textit{C}(2,1),半径\textit{r}=3,

圆心\textit{C}到直线3\textit{x}-4\textit{y}-12=0的距离\textit{d}=$\frac{|6-4-12|}{\sqrt{3^2+(-4)^2}}$=2,

即\textit{r}-\textit{d}=1.

$\mathrm{\therefore}$在圆\textit{C}上到直线\textit{l}的距离为1的点有3个.

答案:B

知识:直线与圆的位置关系

难度:3

题目:直线\textit{l}${}_{1}$:\textit{y}=\textit{x}+\textit{a}和\textit{l}${}_{2}$:\textit{y}=\textit{x}+\textit{b}将单位圆\textit{C}:\textit{x}${}^{2}$+\textit{y}${}^{2}$=1分成长度相等的四段弧,则\textit{a}${}^{2}$+\textit{b}${}^{2}$=(  )

A. $\sqrt{2}$  B.2  C.1   D.3

解析: 依题意,圆心(0,0)到两条直线的距离相等,且每段弧的长度都是圆周的$\frac{1}{4}$,即$\frac{|a|}{\sqrt{2}}$=$\frac{|b|}{\sqrt{2}}$,$\frac{|a|}{\sqrt{2}}$=1$\mathrm{\times}$cos45$\mathrm{{}^\circ}$=$\frac{\sqrt{2}}{2}$,所以\textit{a}${}^{2}$=\textit{b}${}^{2}$=1,故\textit{a}${}^{2}$+\textit{b}${}^{2}$=2.

答案:B

知识:直线与圆的位置关系

难度:3

题目:设\textit{P}是圆(\textit{x}-3)${}^{2}$+(\textit{y}+1)${}^{2}$=4上的动点,\textit{Q}是直线\textit{x}=-3上的动点,则|\textit{PQ}|的最小值为(  )

A.6   B.4  C.3   D.2

解析:|\textit{PQ}|的最小值为圆心到直线的距离减去半径.因为圆的圆心为(3,-1),半径为2,所以|\textit{PQ}|的最小值\textit{d}=3-(-3)-2=4.

答案:B

知识:直线与圆的位置关系

难度:3

题目:在平面直角坐标系\textit{xOy}中,设直线\textit{l}:\textit{kx}-\textit{y}+1=0与圆\textit{C}:\textit{x}${}^{2}$+\textit{y}${}^{2}$=4相交于\textit{A}、\textit{B}两点,以\textit{OA}、\textit{OB}为邻边作平行四边形\textit{OAMB},若点\textit{M}在圆\textit{C}上,则实数\textit{k}等于(  )

A.1   B.2  C.0   D.-1

解析:如图,由题意可知平行四边形\textit{OAMB}为菱形,

\includegraphics*[width=1.53in, height=1.08in, keepaspectratio=false]{image313}

又$\mathrm{\because}$\textit{OA}=\textit{OM},$\mathrm{\therefore}$$\mathrm{\vartriangle}$\textit{AOM}为正三角形.

又\textit{OA}=2,$\mathrm{\therefore}$\textit{OC}=1,且\textit{OC}$\mathrm{\bot}$\textit{AB}.

$\mathrm{\therefore}$$\frac{1}{\sqrt{k^2+1}}$=1,$\mathrm{\therefore}$\textit{k}=0.

答案:C

知识:直线与圆的位置关系

难度:1

题目:已知点\textit{A}(1,2,3)、\textit{B}(2,-1,4),点\textit{P}在\textit{y}轴上,且|\textit{PA}|=|\textit{PB}|,则点\textit{P}的坐标是\_\_\_\_.

解析:设点\textit{P}(0,\textit{b,}0),则

$\sqrt{(1-0)^2+(2-b)^2+(3-0)^2}$=$\sqrt{(2-0)^2+(-1-b)^2+(4-0)^2}$

,解得\textit{b}=-$\frac{7}{6}$.

答案: (0,-$\frac{7}{6}$,0)

知识:直线与圆的位置关系

难度:2

题目:(2016·南安一中高一检测)设\textit{O}为原点,点\textit{M}在圆\textit{C}:(\textit{x}-3)${}^{2}$+(\textit{y}-4)${}^{2}$=1上运动,则|\textit{OM}|的最大值为\_\_\_\_.

解析:圆心\textit{C}的坐标为(3,4),

$\mathrm{\therefore}$|\textit{OC}|=$\sqrt{(3-0)^2+(4-0)^2}$=5,

$\mathrm{\therefore}$|\textit{OM}|${}_{max}$=5+1=6.

答案:6

知识:直线与圆的位置关系,弧度制与角度制

难度:2

题目:过点\textit{A}(1,$\sqrt{2}$)的直线\textit{l}将圆(\textit{x}-2)${}^{2}$+\textit{y}${}^{2}$=4分成两段弧,当劣弧所对的圆心角最小时,直线\textit{l}的斜率\textit{k}=\_\_\_\_.

解析:点\textit{A}(1,$\sqrt{2}$)在圆(\textit{x}-2)${}^{2}$+\textit{y}${}^{2}$=4内,当劣弧所对的圆心角最小时,\textit{l}垂直于过点\textit{A}(1,$\sqrt{2}$)和圆心\textit{M}(2,0)的直线.

$\mathrm{\therefore}$\textit{k}=-$\frac{1}{k_{AM}}$=-$\frac{2-1}{0-\sqrt{2}}$=$\frac{\sqrt{2}}{2}$.

答案:$\frac{\sqrt{2}}{2}$

知识:圆的标准方程,直线与圆的位置关系

难度:2

题目:(2015·江苏卷)在平面直角坐标系\textit{xOy}中,以点(1,0)为圆心且与直线\textit{mx}-\textit{y}-2\textit{m}-1=0(\textit{m}$\mathrm{\in}$R)相切的所有圆中,半径最大的圆的标准方程为\_\_\_\_.

解析:直\textit{mx}-\textit{y}-2\textit{m}-1=0可化为

\textit{m}(\textit{x}-2)+(-\textit{y}-1)=0,

由$\left\{\begin{array}{r} x-2=0\\ -y-1=0 \end{array} \right.$,得$\left\{\begin{array}{r} x=2\\ y=-1 \end{array} \right.$.

$\mathrm{\therefore}$直线过定点\textit{P}(2,-1).以点\textit{C}(1,0)为圆心且与直线\textit{mx}-\textit{y}-2\textit{m}-1=0相切的所有圆中,最大的半径为|\textit{PC}|=$\sqrt{(2-1)^2+(-1-0)^2}$=$\sqrt{2}$,

故圆的标准方程为(\textit{x}-1)${}^{2}$+\textit{y}${}^{2}$=2.

答案:(\textit{x}-1)${}^{2}$+\textit{y}${}^{2}$=2


知识:直线与圆的位置关系

难度:1

题目:已知三角形的三个顶点分别为\textit{A}(-3,1)、\textit{B}(3,-3)、\textit{C}(1,7).

证明:$\mathrm{\vartriangle}$\textit{ABC}为等腰直角三角形.

解析:

答案: |\textit{AB}|=$\sqrt{([3-(-3)]^2+(-3-1)^2}$=$2\sqrt{13}$,

|\textit{AC}|=$\sqrt{[1-(-3)]^2+(7-1)^2}$=$2\sqrt{13}$,

|\textit{BC}|=$\sqrt{(1-3)^2+[7-(-3)]^2}$=$2\sqrt{26}$.

$\mathrm{\therefore}$|\textit{AB}|=|\textit{AC}|,|\textit{AB}|${}^{2}$+|\textit{AC}|${}^{2}$=|\textit{BC}|${}^{2}$,

$\mathrm{\therefore}$$\mathrm{\vartriangle}$\textit{ABC}为等腰直角三角形.

知识:圆的标准方程

难度:2

题目:已知方程\textit{x}${}^{2}$+\textit{y}${}^{2}$-2(\textit{t}+3)\textit{x}+2(1-4\textit{t}${}^{2}$)\textit{y}+16\textit{t}${}^{4}$+9=0表示圆.

(1)求实数\textit{t}的取值范围;

(2)求该圆的半径\textit{r}的取值范围.

解析:

答案: (1)$\mathrm{\because}$方程\textit{x}${}^{2}$+\textit{y}${}^{2}$-2(\textit{t}+3)\textit{x}+2(1-4\textit{t}${}^{2}$)\textit{y}+16\textit{t}${}^{4}$+9=0表示圆,

$\mathrm{\therefore}$4(\textit{t}+3)${}^{2}$+4(1-4\textit{t}${}^{2}$)${}^{2}$-4(16\textit{t}${}^{4}$+9)$\mathrm{>}$0,

即7\textit{t}${}^{2}$-6\textit{t}-1$\mathrm{<}$0,解得-$\frac{1}{7}$$\mathrm{<}$\textit{t}$\mathrm{<}$1.

即实数\textit{t}的取值范围为(-$\frac{1}{7}$,1).

(2)\textit{r}${}^{2}$=(\textit{t}+3)${}^{2}$+(1-4\textit{t}${}^{2}$)${}^{2}$-(16\textit{t}${}^{4}$+9)

=-7\textit{t}${}^{2}$+6\textit{t}+1

=-7(\textit{t}-$\frac{3}{7}$)${}^{2}$+$\frac{16}{7}$,

$\mathrm{\therefore}$\textit{r}${}^{2}$$\mathrm{\in}$(0,$\frac{16}{7}$],$\mathrm{\therefore}$\textit{r}$\mathrm{\in}$(0,$\frac{4\sqrt{7}}{7}$].

即\textit{r}的取值范围为(0,$\frac{4\sqrt{7}}{7}$].

知识:直线与圆的位置关系

难度:1

题目:一圆与两平行直线\textit{x}+3\textit{y}-5=0和\textit{x}+3\textit{y}-3=0都相切,圆心在直线2\textit{x}+\textit{y}+1=0上,求圆的方程.

解析:

答案: 两平行直线之间的距离为$\frac{|-5+3|}{\sqrt{1+9}}$=$\frac{2}{\sqrt{10}}$,$\mathrm{\therefore}$圆的半径为$\frac{1}{\sqrt{10}}$,设圆的方程为(\textit{x}-\textit{a})${}^{2}$+(\textit{y}-\textit{b})${}^{2}$=$\frac{1}{10}$,则$\left\{\begin{array}{r} 2a+b+1=0\\ \frac{|a+3b-5|}{\sqrt{10}}=\frac{1}{\sqrt{10}}\\ \frac{|a+3b-3|}{\sqrt{10}}=\frac{1}{\sqrt{10}} \end{array} \right.$,

解得$\left\{\begin{array}{r} a=-\frac{7}{5}\\ b=\frac{9}{5} \end{array} \right.$.

故所求圆的方程为$(x+\frac{7}{5})^2+(x-\frac{9}{5})^2$=$\frac{1}{10}$.

知识:直线与圆的位置关系

难度:1

题目:(2016·泰安二中高一检测)直线\textit{l}经过两点(2,1)、(6,3).

(1)求直线\textit{l}的方程;

(2)圆\textit{C}的圆心在直线\textit{l}上,并且与\textit{x}轴相切于(2,0)点,求圆\textit{C}的方程.

解析:

答案:(1)直线\textit{l}的斜率\textit{k}=$\frac{3-1}{6-2}$=$\frac{1}{2}$,

$\mathrm{\therefore}$直线\textit{l}的方程为\textit{y}-1=$\frac{1}{2}$(\textit{x}-2),

即\textit{x}-2\textit{y}=0.

(2)由题意可设圆心坐标为(2\textit{a},\textit{a}),

$\mathrm{\because}$圆\textit{C}与\textit{x}轴相切于(2,0)点,

$\mathrm{\therefore}$圆心在直线\textit{x}=2上,

$\mathrm{\therefore}$\textit{a}=1.

$\mathrm{\therefore}$圆心坐标为(2,1),半径\textit{r}=1.

$\mathrm{\therefore}$圆\textit{C}的方程为(\textit{x}-2)${}^{2}$+(\textit{y}-1)${}^{2}$=1.

知识:直线与圆的方程的应用

难度:3

题目:某市气象台测得今年第三号台风中心在其正东300 km处,以40 km/h的速度向北偏西60$\mathrm{{}^\circ}$方向移动.据测定,距台风中心250 km的圆形区域内部都将受玻台风影响,请你推算该市受台风影响的持续时间.

解析:

答案: 以该市所在位置\textit{A}为原点,正东方向为\textit{x}轴的正方向,正北方向为\textit{y}轴的正方向建立直角坐标系.开始时台风中心在\textit{B}(300,0)处,台风中心沿倾斜角为150$\mathrm{{}^\circ}$方向直线移动,其轨迹方程为\textit{y}=-$\frac{\sqrt{3}}{3}$(\textit{x}-300)(\textit{x}$\mathrm{\le}$300).该市受台风影响时,台风中心在圆\textit{x}${}^{2}$+\textit{y}${}^{2}$=250${}^{2}$内,设直线与圆交于\textit{C},\textit{D}两点,则|\textit{CA}|=|\textit{AD}|=250,所以台风中心到达\textit{C}时,开始受影响该市,中心移至点\textit{D}时,影响结束,作\textit{AH}$\mathrm{\bot}$\textit{CD}于点\textit{H},则|\textit{AH}|=$\frac{100\sqrt{3}}{\sqrt{\frac{1}{3}+1}}$=150,|\textit{CD}|=2$\sqrt{|AC|^2-|AH|^2}$=400,$\mathrm{\therefore}$\textit{t}=$\frac{400}{4}$=10(h).即台风对该市的影响持续时间为10小时.

知识:直线与圆的位置关系

难度:3

题目:如下图,在平面直角坐标系\textit{xOy}中,点\textit{A}(0,3),直线\textit{l}:\textit{y}=2\textit{x}-4.设圆\textit{C}的半径为1,圆心在\textit{l}上.

\includegraphics*[width=1.25in, height=1.37in, keepaspectratio=false]{image314}

(1)若圆心\textit{C}也在直线\textit{y}=\textit{x}-1上,过点\textit{A}作圆\textit{C}的切线,求切线的方程;

(2)若圆\textit{C}上存在点\textit{M},使\textit{MA}=2\textit{MO},求圆心\textit{C}的横坐标\textit{a}的取值范围.

解析:

答案: (1)由题设,圆心\textit{C}是直线\textit{y}=2\textit{x}-4和\textit{y}=\textit{x}-1的交点,解得点\textit{C}(3,2),于是切线的斜率必存在.

设过\textit{A}(0,3)的圆\textit{C}的切线方程为\textit{y}=\textit{kx}+3,

由题意,得$\frac{|3k+1|}{\sqrt{k^2+1}}$=1,解得\textit{k}=0或\textit{k}=-$\frac{3}{4}$,

故所求切线方程为\textit{y}=3或3\textit{x}+4\textit{y}-12=0.

(2)因为圆心在直线\textit{y}=2\textit{x}-4上,所以圆\textit{C}的方程为(\textit{x}-\textit{a})${}^{2}$+[\textit{y}-2(\textit{a}-2)]${}^{2}$=1.

设点\textit{M}(\textit{x},\textit{y}),因为\textit{MA}=2\textit{MO},所以=2,化简得\textit{x}${}^{2}$+\textit{y}${}^{2}$+2\textit{y}-3 =0,即\textit{x}${}^{2}$+(\textit{y}+1)${}^{2}$=4,

所以点\textit{M}在以\textit{D}(0,-1)为圆心,2为半径的圆上.

由题意,点\textit{M}(\textit{x},\textit{y})在圆\textit{C}上,所以圆\textit{C}与圆\textit{D}有公共点,

则|2-1|$\mathrm{\le}$\textit{CD}$\mathrm{\le}$2+1,即1$\mathrm{\le}$$\mathrm{\le}$3.

由5\textit{a}${}^{2}$-12\textit{a}+8$\mathrm{\ge}$0,得\textit{a}$\mathrm{\in}$R;

由5\textit{a}${}^{2}$-12\textit{a}$\mathrm{\le}$0,得0$\mathrm{\le}$\textit{a}$\mathrm{\le}\frac{12}{5}$,

所以点\textit{C}的横坐标\textit{a}的取值范围为[0,$\frac{12}{5}$].



\end{document}

