% Generated by GrindEQ Word-to-LaTeX 
\documentclass{article} %%% use \documentstyle for old LaTeX compilers

\usepackage[english]{babel} %%% 'french', 'german', 'spanish', 'danish', etc.
\usepackage{amssymb}
\usepackage{amsmath}
\usepackage{txfonts}
\usepackage{mathdots}
\usepackage[classicReIm]{kpfonts}
\usepackage[dvips]{graphicx} %%% use 'pdftex' instead of 'dvips' for PDF output

\usepackage{ctex} 


\begin{document}

%\selectlanguage{english} %%% remove comment delimiter ('%') and select language if required


\noindent 

\noindent 

\noindent 

\noindent 

\noindent 

\noindent 

知识:命题的概念

难度:1

题目:``红豆生南国,春来发几枝?愿君多采撷,此物最相思.''这是唐代诗人王维的《相思》,在这4句诗中,可作为命题的是(  )

A.红豆生南国   

B.春来发几枝

C.愿君多采撷   

D.此物最相思

解析:``红豆生南国''是陈述句,意思是``红豆生长在南方'',故本句是命题;``春来发几枝''是疑问句,``愿君多采撷''是祈使句,``此物最相思''是感叹句,都不是命题.

答案:A.



知识:命题的概念

难度:1

题目:下列命题为真命题的是(  )

A.若$\frac{1}{x}$=$\frac{1}{y}$,则$x=y$

B.若$x^{2}$=1,则$x=1$

C.若$x=y$,则$\sqrt{x}=\sqrt{y}$

D.若$x<y$,则$x^2<y^2$

解析:很明显A正确;B中,由$x^{2}=1$,得$x={\pm}1$,所以B是假命题;C中,当$x=y<0$时,结论不成立,所以C是假命题;D中,当$x=-1,y=1$时,结论不成立,所以D是假命题.

答案:A.



知识:命题的概念

难度:1

题目:给出下列命题:

①若直线$l\bot$平面$\alpha$,直线$m{\bot}$平面$\alpha$,则$l{\bot}m$;

②若$a,b$都是正实数,则$a+b{\ge}$2;

③若$x^{2}>x$,则$x>1$

④ 函数$y=x^{3}$是指数函数.

其中假命题为(  )

A.①③   

B.①②③

C.①③④   

D.①④

解析:①显然错误,所以①是假命题;由基本不等式,知②是真命题;③中,由$x^{2}>x$,得$x<0$或$x>1$,所以③是假命题;④中函数$y=x^{3}$是幂函数,不是指数函数,④是假命题.

答案:C.



知识:命题的概念

难度:1

题目:命题``垂直于同一条直线的两个平面平行''的条件是(  )

A.两个平面

B.一条直线

C.垂直

D.两个平面垂直于同一条直线

解析:把命题改为``若p则q''的形式为若两个平面垂直于同一条直线,则这两个平面平行,则条件为``两个平面垂直于同一条直线''.

答案:D.



知识:命题的概念

难度:1

题目:下列语句中命题的个数为(  )

①若a,G,b成等比数列,则$G^{2}=ab$.

② $4-x^2\ge 0$

③梯形是中心对称图形.

④ $\pi > \sqrt{2}$吗?

⑤2016 年是我人生中最难忘的一年!

A.2    

B.3    

C.4    

D.5

解析:依据命题的概念知④和⑤不是陈述句,故④⑤不是命题;再从``能否判断真假''的角度分析:②不是命题.只有①③为命题,故选A.

答案:A.



知识:命题的概念

难度:1

题目:下列语句:①$\sqrt{2}$是无限循环小数;②$x^2-3x+2=0$;③当$x=4$时,$2x>0$;④把门关上!其中不是命题的是\_\_\_\_\_\_\_\_.

解析:①是命题;②不是命题,因为语句中含有变量x,在没给变量x赋值的情况下,无法判断语句的真假;③是命题;④是祈使句,不是命题.

答案:②④.



知识:命题的概念

难度:1

题目:已知命题``$f(x)=\cos^{2}\omega x-sin^{2}\omega x$的最小正周期是$\pi$''是真命题,则实数$\omega$的值为\_\_\_\_\_\_\_\_.

解析:$f(x)=\cos^{2}\omega x-sin^{2}\omega x=\cos 2\omega x$,所以$|\frac{2\pi}{2\omega}|=\pi$,解得$\omega =\pm 1$.

答案:$\pm 1$.



知识:命题的概念

难度:1

题目:下列命题:

①若$xy=1$,则$x,y$互为倒数;

②二次函数的图象与x轴有公共点;

③平行四边形是梯形;

④若$ac^2>bc^2$,则$a>b$.

其中真命题是\_\_\_\_\_\_\_\_(写出所有真命题的编号).

解析:对于②,二次函数图象与x轴不一定有公共点;对于③,平行四边形不是梯形.

答案:①④.



知识:命题的概念

难度:1

题目:把下列命题改写成``若p,则q''的形式,并判断其真假.

(1)末位数字是0的整数能被5整除;

(2)偶函数的图象关于y轴对称;

(3)菱形的对角线互相垂直.

解析:

解:(1)若一个整数的末位数字是0,则这个整数能被5整除,为真命题.

(2)若一个函数是偶函数,则这个函数的图象关于y轴对称,为真命题.

(3)若一个四边形是菱形,则它的对角线互相垂直,为真命题.



知识:命题的概念

难度:1

题目:已知:A:$5x-1>a$,B:$x>1$,请选择适当的实数a,使得利用A、B构造的命题``若p,则q''为真命题.

解:若视A为p,则命题``若p,则q''为``若$x>\frac{1+a}{5}$,则$x>1$''.由命题为真命题可知$\frac{1+a}{5}\ge 1$,解得$a\ge 4$;

若视B为p,则命题``若p,则q''为``若$x>1$,则$x>\frac{1+a}{5}$''.由命题为真命题可知$\frac{1+a}{5}\le 1$,解得$a\le 4$.

故a取任一实数均可利用A,B构造出一个真命题,比如这里取a=1,则有真命题``若$x>1$,则$x>\frac{2}{5}$''.



知识:命题的概念

难度:2

题目:给出命题``方程$x^2+ax+1=0$没有实数根'',则使该命题为真命题的a的一个值可以是(  )

A.4    

B.2    

C.1    

D.-3

解析:C中,当a=1时,$\Delta =1^2-4\times 1\times 1=-3<0$,方程无实根,其余3项中,a的值使方程均有实根.

答案:C.



知识:命题的概念

难度:1

题目:①若$a\cdot b=a\cdot c$,则$b=c$;

②若$a=(1,k),b=(-2,6)$,$a//b$,则$k=-3$;

③非零向量a和b满足$|a|=|b|=|a-b|$,则$a$与$a+b$的夹角为$60^{\circ}$.

其中真命题的序号为\_\_\_\_\_\_\_\_(写出所有真命题的序号).

解析:取$a=0$,满足$a\cdot b=a\cdot c$,但不一定有$b=c$,故①不正确;

当$a=(1,k),b=(-2,6),a//b$时,$6+2k=0$,

所以$k=-3$,则②正确;

非零向量a和b满足$|a|=|b|=|a-b|$时,$|a|,|b|,|a-b|$构成等边三角形,所以$a$与$a+b$的夹角为$30^{\circ}$,因此③错误.

答案:②.



知识:命题的概念

难度:1

题目:把下列命题改写成``若p,则q''的形式,并判断真假.

(1)乘积为1的两个实数互为倒数;

(2)奇函数的图象关于原点对称;

(3)与同一直线平行的两个平面平行.

解析:

解:(1)``若两个实数乘积为1,则这两个实数互为倒数'',它是真命题.

(2)``若一个函数为奇函数,则它的图象关于原点对称''.它是真命题.

(3)``若两个平面与同一条直线平行,则这两个平面平行''.它是假命题,这两个平面也可能相交.

知识:四种命题,四种命题的关系

难度:1

题目:命题``对角线相等的四边形是矩形''是命题``矩形的对角线相等''的(  )

A.逆命题   

B.否命题

C.逆否命题   

D.无关命题

解析:将命题``对角线相等的四边形是矩形''写成``若p,则q''的形式为:``若一个四边形的对角线相等,则这个四边形是矩形''.而将命题``矩形的对角线相等''写成``若p,则q''的形式为:``若一个四边形是矩形,则四边形的对角线相等''.则前一个命题为后一个命题的逆命题.

答案:A.



知识:四种命题,四种命题的关系

难度:1

题目:已知$a,b,c\in R$,命题``若$a+b+c=3$,则$a^{2}+b^{2}+c^{2}\ge 3$''的否命题是(  )

A.若$a+b+c\neq 3$,则$a^{2}+b^{2}+c^{2}<3$

B.若$a+b+c=3$,则$a^{2}+b^{2}+c^{2}<3$

C.若$a+b+c\neq 3$,则$a^{2}+b^{2}+c^{2}\ge 3$

D.若$a+b+c\ge 3$,则$a^{2}+b^{2}+c^{2}=3$

解析:否定条件,得$a+b+c\ne 3$,否定结论,得$a^{2}+b^{2}+c^{2}<3$.所以否命题是``若$a+b+c\neq3$,则$a^{2}+b^{2}+c^{2}<3$''.

答案:A



知识:四种命题,四种命题的关系

难度:1

题目:与命题``能被6整除的整数,一定能被3整除''等价的命题是(  )

A.能被3整除的整数,一定能被6整除

B.不能被3整除的整数,一定不能被6整除

C.不能被6整除的整数,一定不能被3整除

D.不能被6整除的整数,不一定能被3整除

解析:原命题与它的逆否命题是等价命题,原命题的逆否命题是:不能被3整除的整数,一定不能被6整除.

答案:B



知识:,四种命题,四种命题的关系

难度:1

题目:下列说法:

①原命题为真,它的否命题为假;

②原命题为真,它的逆命题不一定为真;

③一个命题的逆命题为真,它的否命题一定为真;

④一个命题的逆否命题为真,它的否命题一定为真.

其中正确的是(  )

A.①②      

B.②③

C.③④   

D.②③④

解析:互为逆否命题的两个命题同真假,互为否命题和逆命题的两个命题,它们的真假性没有关系.

答案:B



知识:,四种命题,四种命题的关系

难度:1

题目:有下列四种命题:

①``若$x+y=0$,则$x,y$互为相反数''的否命题;

②``若$x>y$,则$x^{2}>y^{2}$''的逆否命题;

③``若$x\le3$,则$x^{2}-x-6>0$''的否命题;

④``对顶角相等''的逆命题.

其中真命题的个数是(  )

A.0  

B.1 

C.2  

D.3

解析:(1)原命题的否命题与其逆命题有相同的真假性,其逆命题为``若$x,y$互为相反数,则$x+y=0$'',为真命题;(2)原命题与其逆否命题具有相同的真假性.而原命题为假命题(如$x=0,y=-1$),故其逆否命题为假命题;(3)该命题的否命题为``若$x>3$,则$x^{2}-x-6\le0$'',很明显为假命题;(4)该命题的逆命题是``相等的角是对顶角'',显然是假命题.

答案:B



知识:,四种命题,四种命题的关系

难度:1

题目:命题``若$x^{2}<4$,则$-2<x<2$''的逆否命题为\_\_\_\_\_\_\_\_\_\_\_\_\_\_\_,是\_\_\_\_\_\_\_\_\_\_\_\_\_\_(填``真''或``假'')命题.

解析:命题``若$x^2<4$,则$-2<x<2$''的逆否命题为``若$x\ge2$或$x\le-2$,则$x^{2}\ge4$'',因为原命题是真命题,所以其逆否命题也是真命题.

答案:若$x\ge2$或$x\le-2$,则$x^{2}\ge4$ 真



知识:四种命题,四种命题的关系

难度:1

题目:命题``当AB=AC时,${\vartriangle}$ABC是等腰三角形''与它的逆命题、否命题、逆否命题这四个命题中,真命题有\_\_\_\_\_\_\_\_个.

解析:原命题``当AB=AC时,${\vartriangle}$ABC是等腰三角形''是真命题,且互为逆否命题等价,故其逆否命题为真命题.其逆命题``若${\vartriangle}$ABC是等腰三角形,则AB=AC''是假命题,则否命题是假命题.则4个命题中有2个是真命题.

答案:2



知识:四种命题,四种命题的关系

难度:1

题目:设有两个命题:①不等式$mx^{2}+1>0$的解集是R;②函数$f(x)=\log_{m}x$是减函数.如果这两个命题中有且只有一个是真命题,则实数m的取值范围是\_\_\_\_\_\_\_\_.

解析:①当$m=0$时,$mx^{2}+1=1>0$恒成立,解集为R.当$m\neq0$时,若$mx^{2}+1>0$的解集为R,必有$m>0$. 综上知,不等式$mx^{2}+1>0$的解集为R,必有$m\ge0$.

②当$0<m<1$时,$f(x)=\log_{m}x$是减函数,当两个命题中有且只有一个真命题时$\left\{
\begin{array}{l}
m \ge 0, \\
m \le 0 或 m\ge 1
\end{array}
\right.$或$\left\{
\begin{array}{l}
m<0, \\
0<m<1,
\end{array}
\right.$

所以 $m=0$或$m\ge1$.

答案:$m=0$或$m\ge1$



知识:四种命题,四种命题的关系

难度:1

题目:写出命题``在${\vartriangle}$ABC中,若$a>b$,则$A>B$''的逆命题、否命题和逆否命题,并判断它们的真假.

解析:

解:(1)逆命题:在${\vartriangle}$ABC中,若$A>B$,则$a>b$为真命题.否命题:在${\vartriangle}$ABC中,若$a\le b$,则$A\le B$为真命题.逆否命题:在${\vartriangle}$ABC中,若$A\le B$,则$a\le b$为真命题.



知识:四种命题,四种命题的关系

难度:2

题目:判断命题``已知$a,x$为实数,若关于x的不等式$x^{2}+(2a+1)x+a^{2}+2>0$的解集是$\mathcal{R}$,则$a<\frac{7}{4}$''的逆否命题的真假.

解:先判断原命题的真假如下:因为$a,x$为实数,关于$x$的不等式$x^{2}+(2a+1)x+a^{2}+2>0$的解集为$\mathcal{R}$,且抛物线$y=x^{2}+(2a+1)x+a^{2}+2$的开口向上,所以$\Delta=(2a+1)^{2}-4(a^{2}+2)=4a-7<0$.

所以$a<\frac{7}{4}$.所以原命题是真命题.

因为互为逆否命题的两个命题同真同假,所以原命题的逆否命题为真命题.



知识:四种命题,四种命题的关系

难度:2

题目:若命题p的逆命题是q,命题q的否命题是m,则m是p的(  )

A.原命题   

B.逆命题

C.否命题   

D.逆否命题

解析:设命题p为``若k,则l'',则命题q为``若l,则k'',从而命题m为``若非l,则非k'',即命题m是命题p的逆否命题.

答案:D



知识:四种命题,四种命题的关系

难度:2

题目:给出命题:若函数$y=f(x)$是幂函数,则函数$y=f(x)$的图象不过第四象限,在它的逆命题、否命题、逆否命题三个命题中,为真命题的是\_\_\_\_\_\_\_\_.

解析:由于原命题为真命题,则其逆否命题也为真命题.其否命题:若函数$y=f(x)$不是幂函数,则$y=f(x)$的图象过第四象限,为假命题,从而原命题的逆命题也是假命题.

答案:逆否命题



知识:四种命题,四种命题的关系

难度:2

题目:已知p:$x^{2}+mx+1=0$有两个不等的负根,q:$4x^{2}+4(m-2)x+1=0$无实数根.若p,q一真一假,求m的取值范围.

解析:

解:当p为真时,即方程$x^{2}+mx+1=0$有两个不等的负根,设两个负根为$x_{1},x_{2}$,则有$\left\{
\begin{array}{l}
m^2-4>0, \\
x_1+x_2=-m<0
\end{array}
\right.$

解得$m{>}2$.

当q为真时,即方程$4x^{2}+4(m-2)x+1=0$无实数根,则有$16(m-2)^{2}-4\times4\times1<0$,解得$1<m<3$.

若p真,q假,则$\left\{
\begin{array}{l}
m>2, \\
m\le 1 或 m\ge 3,
\end{array}
\right.$得$m\in [3,+\infty)$;

若p假,q真,则$\left\{
\begin{array}{l}
m \le 2, \\
1<m<3,
\end{array}
\right.$得$m \in (1,2]$.

综上所述,$m$的取值范围是$(1,2]\cup[3,+\infty)$.



知识:充分条件与必要条件,充要条件

难度:1

题目:``$\alpha=\frac{\pi}{6}$''是``$\cos 2\alpha=\frac{1}{2}$''的(  )

A.充分而不必要条件

B.必要而不充分条件

C.充要条件

D.既不充分也不必要条件

解析:由$\cos 2\alpha=\frac{1}{2}$,可得$\alpha=k\pi \pm\frac{\pi}{6}(k\in Z)$,故选A.

答案:A



知识:充分条件与必要条件,充要条件

难度:1

题目:(2016·天津卷)设$x>0$,$y\in R$,则``$x>y$''是``$x>|y|$''的(  )

A.充要条件

B.充分而不必要条件

C.必要而不充分条件

D.既不充分也不必要条件

解析:当$x=1,y=-2$时,$x>y$,但$x>|y|$不成立;

若$x>|y|$,因为$|y|\ge y$,所以$x{>}y$.

所以$x{>}y$是$x>|y|$的必要而不充分条件.

答案:C



知识:充分条件与必要条件,充要条件

难度:1

题目:$x^{2}<4$的必要不充分条件是(  )

A.$0<x\le 2$   

B.$-2<x<0$

C.$-2\le x\le 2$   

D.$1<x<3$

解析:$x^{2}<4$即$-2<x<2$,因为$-2<x<2$能推出$-2\le x\le 2$,而$-2\le x\le2$不出$-2<x<2$,所以$x^{2}<4$的必要不充分条件是$-2\le x\le 2$.

答案:C



知识:充分条件与必要条件,充要条件

难度:1

题目:(2016·山东卷)已知直线$a,b$分别在两个不同的平面$\alpha$,$\beta$内,则``直线$a$和直线$b$相交''是``平面$\alpha$和平面$\beta$相交''的(  )

A.充分不必要条件

B.必要不充分条件

C.充要条件

D.既不充分也不必要条件

解析:由题意知a${\subset }$$\alpha$,b${\subset }$$\beta$,若$a,b$相交,则$a,b$有公共点,从而$\alpha$,$\beta$有公共点,可得出$\alpha$,$\beta$相交;反之,若$\alpha$,$\beta$相交,则a,b的位置关系可能为平行、相交或异面.因此``直线$a$和直线$b$相交''是``平面$\alpha$和平面$\beta$相交''的充分不必要条件.故选A.

答案:A



知识:充分条件与必要条件,充要条件

难度:1

题目:函数$f(x)=x^{2}+mx+1$的图象关于直线$x=1$对称的充要条件是(  )

A.$m=2$   

B.$m=-2$

C.$m=-1$   

D.$m=1$

解析:当$m=-2$时,$f(x)=x^{2}-2x+1$,

其图象关于直线$x=1$对称,反之也成立,

所以函数$f(x)=x^{2}+mx+1$的图象关于直线$x=1$对称的充要条件是$m=-2$.

答案:B



知识:充分条件与必要条件,充要条件

难度:1

题目:设$a,b$是实数,则``$a+b>0$''是``$ab>0$''的\_\_\_\_\_\_\_\_\_\_\_\_\_条件.

解析:若$a+b>0$,取$a=3,b=-2$,则$ab>0$不成立;

反之,若$a=-2,b=-3$,则$a+b>0$也不成立,

因此``$a+b>0$''是``$ab>0$''的既不充分也不必要条件.

答案:既不充分也不必要条件



知识:充分条件与必要条件,充要条件

难度:1

题目:关于$x$的不等式$|2x-3|>a$的解集为$\mathcal{R}$的充要条件是\_\_\_\_\_\_\_\_.

解析:由题意知$|2x-3|>a$恒成立.

因为$|2x-3|\ge 0$,所以$a<0$.

答案:$a<0$



知识:充分条件与必要条件,充要条件

难度:1

题目:对任意实数$a,b,c$,给出下列命题:

①`` $a=b$''是``$ac=bc$''的充要条件;

②`` $b-2$ 是无理数''是``$b$是无理数''的充要条件;

③`` $a>b$''是``$a^{2}>b^{2}$''的充分条件;

④`` $a<5$''是``$a<3$''的必要条件.

其中真命题的序号是\_\_\_\_\_\_\_\_.

解析:①中由``$a=b$''可得$ac=bc$,

但由``$ac=bc$''得不到``$a=b$'',所以不是充要条件;

②是真命题;

③中$a>b$时,$a^{2}>b^{2}$不一定成立,所以③是假命题;

④中由``$a<5$''得不到``$a<3$'',

但由``$a<3$''可以得出``$a<5$'',

所以``$a<5$''是``$a<3$''的必要条件,是真命题.

答案:②④



知识:充分条件与必要条件,充要条件

难度:1

题目:已知p:$-4<x-a<4$,q:$(x-2)(x-3)<0$,且q是p的充分而不必要条件,试求a的取值范围.

解析:

解:设q,p表示的范围为集合A,B,则$A=(2,3),B=(a-4,a+4)$.由于q是p的充分而不必要要件,则有$A\subsetneqq B$,即或解得$-1\le a\le 6$.



知识:充分条件与必要条件,充要条件

难度:1

题目:求证:关于$x$的方程$ax^{2}+bx+c=0$有一个根为1的充要条件是$a+b+c=0$.

解析:

证明:必要性:因为方程$ax^{2}+bx+c=0$有一个根为1,

所以$x=1$满足方程$ax^{2}+bx+c=0$,即$a+b+c=0$.

充分性:因为$a+b+c=0$,所以$c=-a-b$,

代入方程$ax^{2}+bx+c=0$中可得$ax^{2}+bx-a-b=0$,

即$(x-1)(ax+a+b)=0$.

故方程$ax^{2}+bx+c=0$有一个根为1.

所以关于x的方程$ax^{2}+bx+c=0$有一个根为1的充要条件是$a+b+c=0$.



知识:充分条件与必要条件,充要条件

难度:2

题目:$m=\frac{1}{2}$是直线$(m+2)x+3my+1=0$与直线$(m-2)x+(m+2)y-3=0$相互垂直的(  )

A.充要条件

B.充分不必要条件

C.必要不充分条件

D.既不充分也不必要条件

解析:当$m=\frac{1}{2}$时,两直线为$\frac{5}{2}x+\frac{3}{2}y+1=0$和$-\frac{3}{2}x+\frac{5}{2}y-3=0$,两直线斜率之积为-1,两直线垂直;而当两直线垂直时,$(m+2)(m-2)+3m(m+2)=0$,即$2(m+2)(2m-1)=0$,所以 $m=-2$或$m=\frac{1}{2}$.所以为充分不必要条件.

答案:B



知识:充分条件与必要条件,充要条件

难度:2

题目:已知p:不等式$x^{2}+2x+m>0$的解集为$\mathbb{R}$;q:指数函数$f(x)=(m+\frac{1}{4})^x$为增函数,则p是q成立的\_\_\_\_\_\_\_\_条件.

解析:p:不等式$x^{2}+2x+m>0$的解集为$\mathbb{R}$,

即$\Delta=4-4m<0,m>1$;q:指数函数$f(x)=(m+\frac{1}{4})^x$为增函数,即$m+\frac{1}{4}>1,m>\frac{3}{4}$,则p是q成立的充分不必要条件.

答案:充分不必要



知识:充分条件与必要条件,充要条件

难度:2

题目:已知p:$-2\le x\le 10$,q:$x^{2}-2x+1-m^{2}\le0(m>0)$,若$\neg$p是$\neg$q的充分不必要条件.求实数m的取值范围.

解析:

解:p:$-2\le x\le 10$.q:$x^{2}-2x+1-m^{2}\le0(m>0)\Leftrightarrow [x-(1-m)][x-(1+m)]\le 0(m>0)\Leftrightarrow 1-m\le x \le 1+m(m>0)$.

因为$\neg$p是$\neg$q的充分不必要条件,所以q是p的充分不必要条件,即$\{x|1-m\le x\le 1+m\}\subsetneqq \{x|-2\le x\le 10 \}$,故有$\left\{
\begin{array}{l}
1-m\ge -2, \\
1+m<-10
\end{array}
\right.$或 $\left\{
\begin{array}{l}
1-m>-2, \\
1+m\le 10
\end{array}
\right.$

解得$m\le3$.又$m>0$,所以实数m的取值范围为$\{m|0<m\le 3 \}$.

本题还可用以下方法求解.

因为p:$-2\le x\le 10$,所以$\neg$p:$x<-2$或$x>10$.

q:$x^{2}-2x+1-m^{2}\le 0(m>0)\Leftrightarrow [x-(1-m)][x-(1+m)] \le 0(m>0) \Leftrightarrow 1-m \le x \le 1+m(m>0)$,

$\neg$q:$x<1-m$或$x>1+m(m>0)$.因为$\neg$p是$\neg$q的充分不必要条件,所以

$\{x|x<-2或x>10 \} \subsetneqq \{x|x<1-m或x>1+m \}$,

故有$\left\{
\begin{array}{l}
1-m\ge -2, \\
1+m<10
\end{array}
\right.$或$\left\{
\begin{array}{l}
1-m>-2, \\
1+m\le 10
\end{array}
\right.$

解得$m\le 3$.又$m>0$,所以实数m的取值范围为$\{m|0<m\le 3 \}$.



知识:逻辑连接词

难度:1

题目:命题``2是3的约数或2是4的约数''中,使用的逻辑联结词的情况是(  )

A.没有使用逻辑联结词

B.使用了逻辑联结词``且''

C.使用了逻辑联结词``或''

D.使用了逻辑联结词``非''

解析:

答案:C



知识:逻辑连接词

难度:1

题目:若命题``p且q''为假,且$\neg$p为假,则(  )

A.p或q为假   

B.q假

C.q真   

D.p假

解析:$\neg$p为假,则p为真,而$p \wedge q$为假,得q为假.

答案:B



知识:逻辑连接词

难度:1

题目:下列命题中,既是``p或q''形式的命题,又是真命题的是(  )

A.方程$x^{2}-x+2=0$的两根是-2,1

B.方程$x^{2}+x+1=0$没有实根

C.$2n-1(n \in Z)$是奇数

D.$a^{2}+b^{2}\ge 0(a,b\in R)$

解析:选项A中,-2,1都不是方程的根;选项B不是``p或q''的形式;选项C也不是``p或q''的形式;选项D中,$a^{2}+b^{2}\ge 0 \Leftrightarrow a^{2}+b^{2}>0$或$a^{2}+b^{2}=0$,且是真命题,故选D.

答案:D



知识:逻辑连接词

难度:1

题目:已知p:$x\in A\cap B$,则$\neg$p是(  )

A.$x\in A$且$x\notin B$   

B.$x \notin A$或$x\notin B$

C.$x\notin A$且$x \notin B$   

D.$x\in A\cup B$

解析:p:$x\in A \cap B$,即$x\in A$且$x\in B$,故$\neg$p是$x\notin A$或$x\notin B$.

答案:B



知识:逻辑连接词

难度:1

题目:给出命题p:函数$y=x^{2}-x-1$有两个不同的零点;q:若$\frac{1}{x}{<}1$,则$x{>}1$.那么在下列四个命题中,真命题是(  )

A.($\neg$p)${\vee}$q   

B.p${\wedge}$q

C.($\neg$p)${\wedge}$($\neg$q)   

D.($\neg$p)${\vee}$($\neg$q)

解析:对于p,函数对应的方程$x^{2}-x-1=0$的判别式$\Delta=(-1)^{2}-4\times(-1)=5>0$,所以函数有两个不同的零点,故p为真.

对于q,当$x{<}0$时,不等式$\frac{1}{x}{<}1$恒成立;当$x{>}0$时,不等式的解集为$\{x|x>1 \}$.故不等式$\frac{1}{x}{<}1$的解集为$\{x|x<0或x>1 \}$.故q为假.结合各选项知,只有($\neg$p)${\vee}$($\neg$q)为真.故选D.

答案:D



知识:逻辑连接词

难度:1

题目:命题``若$a{<}b$,则$2^{a}{<}2^{b}$''的否命题是\_\_\_\_\_\_\_\_\_\_\_\_\_\_\_\_,命题的否定是\_\_\_\_\_\_\_\_\_\_\_\_\_\_.

解析:命题``若p,则q''的否命题是``若$\neg$p,则$\neg$q'',命题的否定是``若p,则$\neg$q''.

答案:若$a{\ge}b$,则$2^{a}{\ge}2^{b}$ 若$a{<}b$,则$2^{a}{\ge}2^{b}$



知识:逻辑连接词

难度:1

题目:已知命题p:对任意$x{\in}R$,总有$|x|\ge 0$.q:$x=1$是方程$x+2=0$的根,则p${\wedge}$($\neg$q)为\_\_\_\_\_\_\_\_命题(填``真''或``假'').

解析:命题p为真命题,命题q为假命题,所以命题$\neg$q为真命题,所以p${\wedge}$$\neg$q为真命题.

答案:真



知识:逻辑连接词

难度:1

题目:已知p:$x^{2}-x{\ge}6$,q:$x{\in}Z$.若``p${\wedge}$q''``$\neg$q''都是假命题,则x的值组成的集合为\_\_\_\_\_\_\_\_.

解析:因为``p${\wedge}$q''为假,``$\neg$q''为假,所以q为真,p为假.故即因此,x的值可以是-1,0,1,2.

答案:$\{-1,0,1,2\}$



知识:逻辑连接词

难度:1

题目:写出下列命题的p${\vee}$q,p${\wedge}$q,$\neg$p的形式,并判断其真假:

(1)p:$\sqrt{2}$是有理数;q:$\sqrt{2}$是实数.

(2)p:5不是15的约数;q:5是15的倍数.

(3)p:空集是任何集合的子集;q:空集是任何集合的真子集.

解析:

解:(1)p${\vee}$q:$\sqrt{2}$是有理数或$\sqrt{2}$是实数,真命题;

p${\wedge}$q:$\sqrt{2}$是有理数且$\sqrt{2}$是实数,假命题;$\neg$p:$\sqrt{2}$不是有理数,真命题.

(2)p${\vee}$q:5不是15的约数或5是15的倍数,假命题;

p${\wedge}$q:5不是15的约数且5是15的倍数,假命题;

$\neg$p:5是15的约数,真命题.

(3)p${\vee}$q:空集是任何集合的子集或空集是任何集合的真子集,真命题;

p${\wedge}$q:空集是任何集合的子集且空集是任何集合的真子集,假命题;

$\neg$p:空集不是任何集合的子集,假命题.



知识:逻辑连接词

难度:1

题目:已知命题p:方程$x^{2}+2x+a=0$有实数根;命题q:函数$f(x)=(a^{2}-a)x$在R上是增函数.若p${\wedge}$q为真命题,求实数a的取值范围.

解析:

解:当p是真命题时,$\Delta=4-4a\ge0$,解得$a\le 1$.当q是真命题时,$a^{2}-a>0$,解得$a<0$或$a>1$.

由题意,得p,q都是真命题,所以$\left\{
\begin{array}{l}
a\le 1, \\
a<0 或 a>1
\end{array}
\right.$

解得$a<0$,

所以实数a的取值范围是$(-\infty,0)$.



知识:逻辑连接词

难度:2

题目:给定命题p:若$x^{2}\ge0$,则$x\ge0$;命题q:已知非零向量a,b,则``$a\bot b$''是``$|a-b|=|a+b|$''的充要条件,则下列各命题中,假命题是(  )

A.p${\vee}$q   

B.($\neg$p)${\vee}$q

C.($\neg$p)${\wedge}$q   

D.($\neg$p)${\wedge}$($\neg$q)

解析:命题p为假命题,命题q为真命题,所以$\neg$p是真命题,$\neg$q为假命题,所以($\neg$p)${\wedge}$($\neg$q)为假命题.

答案:D



知识:逻辑连接词

难度:2

题目:给出下列结论:

(1)当p是真命题时,``p且q''一定是真命题;

(2)当p是假命题时,``p且q''一定是假命题;

(3)当``p且q''是假命题时,p一定是假命题;

(4)当``p且q''是真命题时,p一定是真命题.

其中正确结论的序号是\_\_\_\_\_\_\_\_.

解析:(1)错误,当q是假命题时,``p且q''是假命题,当q也是真命题时,``p且q''是真命题;(2)正确;(3)错误,p也可能是真命题;(4)正确.

答案:(2)(4)



知识:逻辑连接词

难度:2

题目:已知$a>0$,设p:函数$y=a^{x}$在R上单调递减;q:不等式$x+|x-2a|>1$的解集为R,如果``p${\vee}$q''为真,``p${\wedge}$q''为假,求实数a的取值范围.

解析:

解:对于命题p:函数$y=a^{x}$在R上单调递减,即$0<a<1$.对于命题q:不等式$x+|x-2a|>1$的解集为R,即函数$y=x+|x-2a|$在R上恒大于1,又$y=\left\{
\begin{array}{l}
2x-2a, x\ge 2a, \\
2a, x<2a
\end{array}
\right.$,所以 $y_{min}=2a>1$,即$a>\frac{1}{2}$.

由p${\vee}$q为真,p${\wedge}$q为假,根据复合命题真值表知p、q一真一假.如果p真q假,则$0<a{\le}\frac{1}{2}$;如果p假q真,则$a{\ge}1$.

综上所述,a的取值范围为$(0,\frac{1}{2}]{\cup}[1,+{\infty})$.

\noindent 

知识:全称量词与存在量词

难度:1

题目:以下四个命题既是特称命题又是真命题的是(  )

A.锐角三角形的内角是锐角或钝角

B.至少有一个实数x,使$x^{2}\le0$

C.两个无理数的和必是无理数

D.存在一个负数x,使$\frac{1}{x}>2$

解析:A中锐角三角形的内角是锐角或钝角是全称命题;B中$x=0$时,$x^{2}=0$,所以B既是特称命题又是真命题;C中因为$\sqrt{3}+(-\sqrt{3})=0$,所以C是假命题;D中对于任一个负数x,都有$\frac{1}{x}<0$,所以D是假命题.

答案:B



知识:全称量词与存在量词,含有一个量词的命题的否定

难度:1

题目:命题``$\forall x\in R,x^{2}\neq x$''的否定是(  )

A.$\forall x\notin R,x^{2} \neq x$   

B.${\forall }x{\in}R,x^{2}=x$

C.${\exists }x{\notin}R,x^{2}{\neq}x$   

D.${\exists }x\in R,x^{2}=x$

解析:全称命题的否定是特称命题,所以命题``${\forall }x{\in}R,x^{2}{\neq}x$''的否定是``${\exists }x{\in}R,x^{2}=x$''.

答案:D



知识:全称量词与存在量词

难度:1

题目:下列特称命题中假命题的个数是(  )

①有一条直线与两个平行平面垂直;

②有一条直线与两个相交平面平行;

③存在两条相交直线与同一个平面垂直.

A.0    

B.1    

C.2    

D.3

解析:①②都是真命题,③是假命题.

答案:B



知识:全称量词与存在量词

难度:1

题目:设函数$f(x)=x^{2}+mx(m{\in}R)$,则下列命题中的真命题是(  )

A.任意$m{\in}R$,使$y=f(x)$都是奇函数

B.存在$m{\in}R$,使$y=f(x)$是奇函数

C.任意$m{\in}R$,使$x=f(x)$都是偶函数

D.存在$m{\in}R$,使$y=f(x)$是偶函数

解析:当$m=0$时,$f(x)=x^{2}$为偶函数,故选D.

答案:D



知识:全称量词与存在量词

难度:1

题目:若$(\frac{1}{3})^{x^2-2ax}<33x+a^{2}$恒成立,则实数a的取值范围是(  )

A.$0<a<1$   

B.$a>\frac{3}{4}$

C.$0<a<\frac{3}{4}$   

D.$a<\frac{3}{4}$

解析:由题意,得$-x^{2}+2ax<3x+a^{2}$,即$x^{2}+(3-2a)x+a^{2}>0$恒成立,所以$\Delta=(3-2a)^{2}-4a^{2}<0$,解得$a>\frac{3}{4}$.
答案:B



知识:全称量词与存在量词,含有一个量词的命题的否定

难度:1

题目:命题``${\exists }x_{0},y_{0}{\in}Z,3x_{0}-2y_{0}=10$''的否定是\_\_\_\_\_\_\_\_\_\_\_\_\_\_.

解析:特称命题的否定是全称命题,则否定为${\forall }x,y{\in}Z,3x-2y{\neq}10$.

答案:${\forall }x,y{\in}Z,3x-2y{\neq}10$



知识:全称量词与存在量词

难度:1

题目:下列命题中,是全称命题的是\_\_\_\_\_\_\_\_;是特称命题的是\_\_\_\_\_\_\_\_.

①正方形的四条边相等;

②有两个角相等的三角形是等腰三角形;

③正数的平方根不等于0;

④至少有一个正整数是偶数.

解析:①可表述为``每一个正方形的四条边相等'',是全称命题;②是全称命题,即``凡是有两个角相等的三角形都是等腰三角形'';③可表述为``所有正数的平方根不等于0''是全称命题;④是特称命题.

答案:①②③ ④



知识:全称量词与存在量词

难度:1

题目:下面四个命题:

① ${\forall }x{\in}R,x^{2}-3x+2>0$恒成立;②${\exists }x_{0}{\in}Q,x=2$;③${\exists }x_{0}{\in}R,x+1=0$;④${\forall }x{\in}R,4x^{2}>2x-1+3x^{2}$.

其中真命题的个数为\_\_\_\_\_\_\_\_.

解析:$x^{2}-3x+2>0,\Delta=(-3)^{2}-4{\times}2>0$,所以当$x>2$或$x<1$时,$x^{2}-3x+2>0$才成立,所以①为假命题.当且仅当$x={\pm}$时,$x^{2}=2$,所以不存在$x{\in}Q$,使得$x^{2}=2$,所以②为假命题.对${\forall }x{\in}R,x^{2}+1{\neq}0$,所以③为假命题.$4x^{2}-(2x-1+3x^{2})=x^{2}-2x+1=(x-1)^{2}{\ge}0$,即当$x=1$时,$4x^{2}=2x-1+3x^{2}$成立,所以④为假命题.所以①②③④均为假命题.

答案:0



知识:全称量词与存在量词,含有一个量词的命题的否定

难度:1

题目:判断下列各命题的真假,并写出命题的否定.

(1)有一个实数a,使不等式$x^{2}-(a+1)x+a>0$恒成立;

(2)对任意实数$x$,不等式$|x+2|\le 0$恒成立;

(3)在实数范围内,有些一元二次方程无解.

解析:

解:(1)方程$x^{2}-(a+1)x+a=0$的判别式$\Delta=(a+1)^{2}-4a=(a-1)^{2}{\ge}0$,

则不存在实数$a$,使不等式$x^{2}-(a+1)x+a>0$恒成立,所以原命题为假命题.

它的否定:对任意实数$a$,不等式$x^{2}-(a+1)x+a>0$不恒成立.

(2)当$x=1$时,$|x+2|>0$,所以原命题是假命题.

它的否定:存在实数x,使不等式$|x+2|>0$成立.

(3)由一元二次方程解的情况,知该命题为真命题.

它的否定:在实数范围内,所有的一元二次方程都有解.



知识:全称量词与存在量词

难度:1

题目:对于任意实数$x$,不等式$\sin x+\cos x>m$恒成立,求实数m的取值范围.

解:令$y=\sin x+\cos x$,则$y=\sin x+\cos x=$

$=\sqrt{2}(\frac{\sqrt{2}}{2}\sin x+\frac{\sqrt{2}}{2}\cos x)=\sqrt{2}\sin(x+\frac{\pi}{4})$.

因为$-1\le\sin(x+\frac{\pi}{4})\le 1$,所以$\sqrt{2}\sin(x+\frac{\pi}{4})\ge-\sqrt{2}$.

因为${\forall }x{\in}R,\sin x+\cos x>m$恒成立,

所以只要$m<-\sqrt{2}$即可.

故实数$m$的取值范围是$(-{\infty},-\sqrt{2})$.



知识:全称量词与存在量词

难度:2

题目:若命题p:${\forall }x{\in}R,\log_{2}x{>}0$,命题q:${\exists }x_{0}{\in}R,2x_{0}{<}0$,则下列命题为真命题的是(  )

A.p${\vee}$q   

B.p${\wedge}$q

C.($\neg$p)${\wedge}$q   

D.p${\vee}$($\neg$q)

解析:命题p:${\forall }x{\in}R,\log_{2}x{>}0$为假命题,命题q:${\exists }x_{0}{\in}R,2x_{0}{<}0$为假命题,所以p${\vee}$($\neg$q)为真命题,故选D.

答案:D



知识:全称量词与存在量词

难度:2

题目:已知命题``${\exists } x_{0}{\in}R,2x_0^2+(a-1)x_{0}+\frac{1}{2}{\le}0$''是假命题,则实数a的取值范围是\_\_\_\_\_\_\_\_.

解析:由题意可得``对${\forall }x{\in}R,2x^{2}+(a-1)x+\frac{1}{2}>0$恒成立''是真命题,令$\Delta=(a-1)^{2}-4<0$,得$-1<a<3$.

答案:$(-1,3)$



知识:全称量词与存在量词

难度:1

题目:已知命题p:``${\forall }x{\in}[1,2],x^{2}-a{\ge}0$'',命题q:``${\exists }x_{0}{\in}R,x_0^2+2ax_{0}+a+2=0$'',若命题``p或q''是真命题,求实数a的取值范围.

解析:

解:$p{\Leftrightarrow }a{\le}(x^{2})_{min}=1$.$q{\Leftrightarrow }\Delta=4a^{2}-4(a+2){\ge}0{\Leftrightarrow }a{\le}-1或a{\ge}2$.

因为``p或q''为真命题,

所以 p、q中至少有一个真命题.

所以 a${\le}$1或a${\le}$-1或a${\ge}$2,所以 a${\le}$1或a${\ge}$2.

所以 ``p或q''是真命题时,实数a的取值范围是$(-{\infty},1]{\cup}[2,+{\infty})$.

\noindent 

知识:椭圆的定义

难度:1


题目:已知$F_{1},F_{2}$是定点,$|F_1F_2|=8$,动点M满足$|MF_1|+|MF_2|=8$,则动点M的轨迹是(  )

A.椭圆        

B.直线

C.圆   

D.线段

解析:因为$|MF_1|+|MF_2|=8=|F_1F_2|$,所以点M的轨迹是线段$F_{1}F_{2}$,故选D.

答案:D



知识:椭圆的定义

难度:1

题目:椭圆$\frac{x^2}{25}+\frac{y^2}{169}=1$的焦点坐标是(  )

A.(${\pm}$5,0)   

B.(0,${\pm}$5)

C.(0,${\pm}$12)   

D.(${\pm}$12,0)

解析:因为$c^{2}=a^{2}-b^{2}=169-25=12^{2}$,所以 $c=12$.又焦点在y轴上,故焦点坐标为(0,${\pm}$12),

答案:C



知识:椭圆的定义

难度:1

题目:已知椭圆$\frac{x^2}{m}+\frac{y^2}{16}=1$上一点P到椭圆的一个焦点的距离为3,到另一个焦点的距离为7,则m=(  )

A.10  

B.5  

C.15  

D.25

解析:设椭圆的焦点分别为$F_{1},F_{2}$,则由椭圆的定义,知$|PF_1|+|PF_2|=2a=10$,所以 a=5,所以 $a^{2}=25$,所以 椭圆的焦点在x轴上,m=25.

答案:D



知识:椭圆的定义

难度:1

题目:若椭圆焦点在x轴上且经过点$(-4,0),c=3$,则该椭圆的标准方程为(  )

A.$\frac{x^2}{16}+\frac{y^2}{8}=1$   

B.$\frac{x_2}{16}+\frac{y^2}{7}=1$

C.$\frac{x^2}{9}+\frac{y^2}{16}=1$   

D.$\frac{x^2}{7}+\frac{y^2}{16}=1$

解析:因为椭圆过点(-4,0),所以a=4,又因为c=3,所以$b=\sqrt{7}$,所以椭圆的标准方程为$\frac{x_2}{16}+\frac{y^2}{7}=1$.

答案:B



知识:椭圆的定义

难度:1

题目:若方程$\frac{x_2}{m+9}+\frac{y^2}{25-m}=1$表示焦点在x轴上的椭圆,则实数m的取值范围是(  )

A.-9${<}$m${<}$25   

B.8${<}$m${<}$25

C.16${<}$m${<}$25   

D.m${>}$8

解析:依题意有$\left\{
\begin{array}{l}
25-m > 0, \\
m+9>0, \\
m+9>25-m
\end{array}
\right.$解得8${<}$m${<}$25.



知识:椭圆的定义

难度:1

题目:已知椭圆$5x^{2}-ky^{2}=5$的一个焦点是(0,2),则k=\_\_\_\_\_\_\_\_.

解析:易知k${\neq}$0,椭圆方程可化为$x^{2}+\frac{y^2}{-\frac{5}{k}}=1$,

所以 $a^{2}=-\frac{5}{k}$,$b^{2}$=1.又c=2,所以$-\frac{5}{k}-1=4$,

所以 k=-1.

答案:-1



知识:椭圆的定义

难度:1

题目:已知椭圆的焦点是$F_{1}(-1,0),F_{2}(1,0)$,P是椭圆上的一点,则$|F_1F_2|$是$|PF_1|$和$|PF_2|$的等差中项,则该椭圆的方程是\_\_\_\_\_\_\_\_\_\_\_.

解析:由题意得$2|F_1F-2|=|PF_1|+|PF_2|$,

所以 4c=2a=4,所以 a=2.

又c=1,所以 $b^{2}=a^{2}-c^{2}=3$,

故椭圆方程为$\frac{x^2}{4}+\frac{y^2}{3}=1$.

答案:$\frac{x^2}{4}+\frac{y^2}{3}=1$



知识:椭圆的定义

难度:1

题目:若椭圆$\frac{x^2}{49}+\frac{y^2}{24}=1$上一点P与椭圆的两个焦点$F_{1},F_{2}$的连线互相垂直,则${\vartriangle} PF_{1}F_{2}$的面积为\_\_\_\_\_\_\_\_.

解析:设$|PF_{1}|=x$,则$|PF_2|=14-x$,又2c=10,

根据勾股定理,得$x^{2}+(14-x)^{2}=100$,

解得x=8或x=6,所以$S=\frac{1}{2}{\times}8{\times}6=24$.

答案:24



知识:椭圆的定义

难度:1

题目:已知椭圆的中心在原点,两焦点$F_{1},F_{2}$在x轴上,且过点A(-4,3).若$F_{1}A{\bot}F_{2}A$,求椭圆的标准方程.

解析:

解:设所求椭圆的标准方程为$\frac{x^2}{a^2}+\frac{y^2}{b^2}=1(a>b>0)$.

设焦点$F_{1}(-c,0),F_{2}(c,0)(c{>}0)$.

因为$F_{1}A{\bot}F_{2}A$,

所以$\overrightarrow{F_1A}·\overrightarrow{F_2A}=0$,

而$\overrightarrow{F_1A}=(-4+c,3)$,

$\overrightarrow{F_2A}=(-4-c,3)$,

所以$(-4+c)\cdot (-4-c)+3^{2}=0$,

所以$c^{2}=25$,即c=5.

所以$F_{1}(-5,0),F_{2}(5,0)$.

所以$2a=AF_{1}+AF_{2}$

$=\sqrt{(-4+5)^2+3^2}+\sqrt{(-4-5)^2+3^2}$

$=\sqrt{10}+\sqrt{90}$

$=4\sqrt{10}$.

所以$a=2\sqrt{10}$,

所以$b^{2}=a^{2}-c^{2}=(2\sqrt{10})^{2}-5^{2}=15$.

所以所求椭圆的标准方程为$\frac{x^2}{40}+\frac{y^2}{15}=1$.



知识:椭圆的定义

难度:1

题目:已知圆A:$(x+3)^{2}+y^{2}=100$,圆A内一定点B(3,0),圆P过B且与圆A内切,求圆心P的轨迹方程.

解析:

解:如图,设圆P的半径为r,又圆P过点B,所以$|PB|=r$.

\includegraphics*[width=1.18in, height=1.05in, keepaspectratio=false]{image4}

又因为圆P与圆A内切,圆A的半径为10,

所以 两圆的圆心距$|PA|=10-r$,

即$|PA|+|PB|=10$(大于$|AB|$).

所以 点P的轨迹是以A、B为焦点的椭圆.

所以 2a=10,2c=$|AB|$=6.

所以 a=5,c=3.

所以 $b^{2}=a^{2}-c^{2}=25-9=16$.

所以 点P的轨迹方程为$\frac{x^2}{25}+\frac{y^2}{16}=1$.



知识:椭圆的定义

难度:1

题目:平面内有两个定点A,B及动点P,设甲:$|PA|+|PB|$是定值,乙:点P的轨迹是以A,B为焦点的椭圆,则甲是乙的(  )

A.充分不必要条件   

B.必要不充分条件

C.充要条件   

D.既不充分也不必要条件

解析:点P的轨迹是以A,B为焦点的椭圆,则$|PA|+|PB|$是定值,由椭圆的定义,知反之不一定成立.

答案:B



知识:椭圆的定义

难度:1

题目:若椭圆$\frac{x^2}{m}+\frac{y^2}{15}=1$的焦距等于2,则m的值是\_\_\_\_\_\_\_\_.

解析:当椭圆的焦点在x轴上时,$a^{2}=m,b^{2}=15$,

所以 $c^{2}=m-15$,所以 $2c=2\sqrt{m-15}=2$,解得m=16;

当椭圆的焦点在y轴上时,同理有$2\sqrt{15-m}=2$,

所以 m=14.

答案:16或14



知识:椭圆的定义

难度:1

题目:已知P是椭圆$\frac{x^2}{4}+y^{2}=1$上的一点,$F_{1},F_{2}$是椭圆的两个焦点.

(1)当${\angle}F_{1}PF_{2}=60^{\circ}$时,求${\vartriangle}F_{1}PF_{2}$的面积;

(2)当${\angle}F_{1}PF_{2}$为钝角时,求点P横坐标的取值范围.

解析:

解:(1)由椭圆的定义,得$|PF_1|+|PF_2|=4$,①且$F_{1}(-\sqrt{3},0),F_{2}(\sqrt{3},0)$.在${\vartriangle}F_{1}PF_{2}$中,由余弦定理得$|F_1F_2|^2=|PF_1|^2+|PF_2|^2-2|PF_1|\cdot |PF_2|\cos 60^{\circ}$.②由①②得$|PF_1|\cdot |PF_2|=\frac{4}{3}$.

所以S${\vartriangle}F_{1}PF_{2}=\frac{1}{2}|PF_1|\cdot |PF_2|\sin \angle F_1PF_2=\frac{\sqrt{3}}{3}$.

(2)设点P(x,y),由已知${\angle}F_{1}PF_{2}$为钝角,得$\overrightarrow{F_1P}\cdot \overrightarrow{F_2P}<0$,即$(x+\sqrt{3},y)\cdot(x-\sqrt{3},y)<0$,

又$y^{2}=1-\frac{x^2}{4}$,所以$\frac{3}{4}x^{2}<2$,解得$-\frac{2\sqrt{6}}{3}<x<\frac{2\sqrt{6}}{3}$,

所以点P横坐标的取值范围是$-\frac{2\sqrt{6}}{3}<x<\frac{2\sqrt{6}}{3}$



知识:椭圆的性质

难度:1

题目:过原点作直线l交椭圆$x^{2}+2y^{2}$=6于A,B两点,若A(2,-1),则点B的坐标为(  )

A.(-1,2)     

B.(-2,-1)

C.(1,-2)   

D.(-2,1)

解析:依据椭圆的对称性知,A、B两点关于原点中心对称,故选D.

答案:D



知识:椭圆的性质

难度:1

题目:曲线$\frac{x^2}{25}+\frac{y^2}{9}=1$与曲线$\frac{x^2}{25-k}+\frac{y^2}{9-k}=1(k<9)$的(  )

A.长轴长相等   

B.短轴长相等

C.离心率相等   

D.焦距相等

解析:两方程都表示椭圆,由方程可知$c^{2}$都为16,所以焦距2c相等.

答案:D



知识:椭圆的性质

难度:1

题目:椭圆以两条坐标轴为对称轴,一个顶点是(0,13),另一个顶点是(-10,0),则焦点坐标为(  )

A.(${\pm}$13,0)   

B.(0,${\pm}$10)

C.(0,${\pm}$13)   

D.(0,${\pm}\sqrt{69}$)

解析:由题意知椭圆焦点在y轴上,且a=13,b=10,

则$c=\sqrt{a^2-b^2}=\sqrt{69}$,故焦点坐标为(0,${\pm}\sqrt{69}$).

答案:D
%%
%%
%%
知识:椭圆的性质

难度:1

题目:已知中心在原点的椭圆C的右焦点为F(1,0),离心率等于$\frac{1}{2}$,则椭圆C的方程是(  )

A.$\frac{x^2}{3}+\frac{y^2}{4}=1$   

B.$\frac{x^2}{4}+\frac{y^2}{\sqrt{3}}=1$  

C.$\frac{x^2}{4}+\frac{y^2}{2}=1$    

D.$\frac{x^2}{4}+\frac{y^2}{3}=1$  

解析:设椭圆C的方程为$\frac{x^2}{a^2}+\frac{y^2}{b^2}=1(a>b>0)$,

则$c=1,e=\frac{c}{a}=\frac{1}{2}$,所以$a=2,b=\sqrt{3}$,

所以 椭圆C的方程是$\frac{x^2}{4}+\frac{y^2}{3}=1$.

答案:D



知识:椭圆的性质

难度:1

题目:已知椭圆$x^{2}+my^{2}=1$的焦点在y轴上,且长轴长是短轴长的2倍,则m=(  )

A.$\frac{1}{4}$

B.$\frac{1}{2}$

C.2  

D.4

解析:将椭圆方程化为标准方程为$x^{2}+\frac{y^2}{\frac{1}{m}}=1$.

因为焦点在y轴上,所以$ \frac{1}{m}>1$,所以 $0<m<1$,

由方程得$a=\sqrt{\frac{1}{m}},b=1$.

因为a=2b,所以 $m=\frac{1}{4}$.

答案:A



知识:椭圆的第二定义

难度:1

题目:已知椭圆C:$x^{2}+3y^{2}=3$,则椭圆C的离心率为\_\_\_\_\_\_.

解析:椭圆C的标准方程为$\frac{x^2}{3}+y^{2}=1$,所以$a=\sqrt{3},b=1$,

$c=\sqrt{2}$,故$e=\frac{c}{a}=\frac{\sqrt{2}}{\sqrt{3}}=\frac{\sqrt{6}}{3}$.

答案:$\frac{\sqrt{6}}{3}$

%%
%%
知识:椭圆的第二定义

难度:1

题目:已知椭圆的短半轴长为1,离心率$0<e\le \frac{\sqrt{3}}{2}$.则长轴长的取值范围为\_\_\_\_\_\_\_\_.

解析:因为$0<e\le \frac{\sqrt{3}}{2}$,所以 $0<e^2\le \frac{3}{4}$.

又因为$e^{2}=1-\frac{b^2}{a^2},b=1$,而$0<1-\frac{1}{a^2}{\le}\frac{3}{4}$,

所以 $-\frac{3}{4}{\le}\frac{1}{a^2}-1<0$,

所以 $\frac{1}{4}{\le}\frac{1}{a^2}<1$,

所以 $1<a^{2}{\le}4$,而$1<a{\le}2$

所以 长轴长$2a{\in}(2,4]$.

答案:(2,4]

知识:椭圆的第二定义

难度:1

题目:若椭圆$\frac{x^2}{k+8}+\frac{y^2}{9}=1$的离心率$e=\frac{1}{2}$,则k的值等于\_\_\_\_.

解析:分两种情况进行讨论:

当焦点在x轴上时,$a^{2}=k+8,b^{2}=9$,得$c^{2}=k-1$,

又因为$e=\frac{1}{2}$,所以$\frac{\sqrt{k-1}}{\sqrt{k+8}}=\frac{1}{2}$,解得k=4。

当焦点在y轴上时,$a^{2}=9,b^{2}=k+8$,得$c^{2}=1-k$,

又因为$e=\frac{1}{2}$,所以$\frac{\sqrt{1-k}}{\sqrt{9}}=\frac{1}{2}$,解得$k=-\frac{5}{4}$.

所以 k=4或$k=-\frac{5}{4}$

答案:4或$-\frac{5}{4}$



知识:椭圆的第二定义

难度:1

题目:分别求适合下列条件的椭圆的标准方程:

(1)离心率是$\frac{2}{3}$,长轴长是6;

(2)在x轴上的一个焦点与短轴两个端点的连线互相垂直,且焦距为6.

解:(1)设椭圆的方程为

$\frac{x^2}{a^2}+\frac{y^2}{b^2}=1$(a>b>0)或$\frac{y^2}{a^2}+\frac{x^2}{b^2}=1$(a>b>0).

由已知得2a=6,$e=\frac{c}{a}=\frac{2}{3}$,所以 a=3,c=2.

所以 $b^{2}=a^{2}-c^{2}=9-4=5$.

所以 椭圆方程为$\frac{x^2}{9}+\frac{y^2}{5}=1$或$\frac{x^2}{5}+\frac{y^2}{9}=1$.

(2)设椭圆方程为$\frac{x^2}{a^2}+\frac{y^2}{b^2}=1$(a>b>0).

\includegraphics*[width=1.26in, height=1.02in, keepaspectratio=false]{image5}

如图所示,${\vartriangle}A_{1}FA_{2}$为一等腰直角三角形,OF为斜边$A_{1}A_{2}$上的中线(高),且$|OF|=c, |A_1A_2|=2b$,所以 c=b=3所以 $a^{2}=b^{2}+c^{2}=18$,故所求椭圆的方程为$\frac{x^2}{18}+\frac{y^2}{9}=1$.



知识:椭圆的第二定义

难度:1

题目:设椭圆方程$mx^{2}+4y^{2}=4m(m{>}0)$的离心率为$\frac{1}{2}$,试求椭圆的长轴长和短轴长、焦点坐标及顶点坐标.

解析:

解:(1)当0${<}$m${<}$4时,长轴长和短轴长分别是4,$2\sqrt{3}$,焦点坐标为$F_{1}(-1,0),F_{2}(1,0)$,顶点坐标为$A_{1}(-2,0),A_{2}(2,0),B_{1}(0,-\sqrt{3}),B_{2}(0,\sqrt{3})$.

(2)当m${>}$4时,长轴长和短轴长分别为$\frac{8\sqrt{3}}{3}$,4,焦点坐标为$F_{1}(0, -\frac{2\sqrt{3}}{3}),F_{2}(0, \frac{2\sqrt{3}}{3})$,顶点坐标为$A_{1}(0,- \frac{4\sqrt{3}}{3}),A_{2}(0,\frac{4\sqrt{3}}{3}),B_{1}(-2,0),B_{2}(2,0)$.



知识:椭圆的第二定义

难度:2

题目:设椭圆的两个焦点分别为$F_{1},F_{2}$,过$F_{2}$作椭圆长轴的垂线交椭圆于点P,若${\vartriangle}F_{1}PF_{2}$为等腰直角三角形,则椭圆的离心率为(  )

A.$\frac{\sqrt{2}}{2}$   

B.$\frac{\sqrt{2}-1}{2}$

C.$2-\sqrt{2}$

D.$\sqrt{2}-1$

解析:因为$|F_1F_2|$=2c,$|PF_2|$=2c,

所以$|PF_1|=\sqrt{2}|F_1F_2|=2\sqrt{2}c$.

所以$|PF_1|+|PF_2|=2c+2\sqrt{2}c$.

又$|PF_1|+|PF_2|=2a$,所以$2c+2\sqrt{2}c=2a$.

所以$\frac{c}{a}=\sqrt{2}-1$,即$e=\sqrt{2}-1$.

答案:D
%%
%%
%%
知识:椭圆的性质

难度:2

题目:已知AB为过椭圆$\frac{x^2}{a^2}+\frac{y^2}{b^2}=1$中心的弦,F(c,0)为椭圆的右焦点,则${\vartriangle}$AFB面积的最大值为(  )

A.$b^{2}$  

B.ab  

C.ac  

D.bc

解析:设A的坐标为(x,y),则根据对称性得B(-x,-y)

则${\vartriangle}AFB$面积$S=\frac{1}{2}\cdot |OF|\cdot |2y|=c|y|$

由椭圆图象知,当A点在椭圆的顶点时,其${\vartriangle}AFB$面积最大值为bc.

答案:D



知识:椭圆的性质

难度:2

题目:已知点P为椭圆$x^{2}+2y^{2}=98$上一个动点,点A的坐标为(0,5),求$|PA|$的最值.

解析:

解:设P(x,y),则$|PA|=\sqrt{x^2+(y-5)^2}=\sqrt{x^2+y^2-10y+25}$

因为点P为椭圆$x^{2}+2y^{2}=98$上一点,

所以$x^{2}=98-2y^{2},-7{\le}y{\le}7$,

则$|PA|=\sqrt{98-2y^2+y^2-10y+25}=\sqrt{-(y+5)^2+148}$

因为$-7{\le}y{\le}7$,

所以当y=-5时,$|PA|_{max}=\sqrt{148}=2\sqrt{37}$;

当y=7时,$|PA|_{min}$=2.

\noindent 

知识:椭圆的性质

难度:1

题目:点A(a,1)在椭圆$\frac{x^2}{4}+\frac{y^2}{2}=1$的内部, 则a的取值范围是(  )

A.$-\sqrt{2}<a<\sqrt{2}$   

B.$a<-\sqrt{2}$或$a>\sqrt{2}$

C.$-2<a<2$   

D.$-1<a<1$

解析:由A(a,1)在椭圆内部,则$\frac{a^2}{4}+\frac{1^2}{2}<1$,即$a^{2}<2$,则$-\sqrt{2}<a<\sqrt{2}$.

答案:A



知识:椭圆的性质

难度:1

题目:已知直线l过点(3,-1),且椭圆C:$\frac{x^2}{25}+\frac{y^2}{36}=1$,则直线l与椭圆C的公共点的个数为(  )

A.1  

B.1或2  

C.2  

D.0

解析:点(3,-1)满足$\frac{3^2}{25}+\frac{(-1)^2}{36}<1$,即点在椭圆内,过椭圆内部点作的直线与椭圆必有2个交点.

答案:C
%%
%%
%%
知识:椭圆的性质

难度:1

题目:若直线$kx-y+3=0$与椭圆$\frac{x^2}{16}+\frac{y^2}{4}=1$有两个公共点,则实数k的取值范围是(  )

A.$-\frac{5}{4}<k<\frac{\sqrt{5}}{4}$   

B.$k=\frac{\sqrt{5}}{4}$或$k=-\frac{\sqrt{5}}{4}$

C.$k>\frac{\sqrt{5}}{4}$或$k<-\frac{\sqrt{5}}{4}$   

D.$k<\frac{\sqrt{5}}{4}$且$k\ne -\frac{\sqrt{5}}{4}$

解析:由$\left\{
\begin{array}{l}
y=kx+3, \\
\frac{x^2}{16}+\frac{y^2}{4}=1
\end{array}
\right.$可得$(4k^2+1)x^2+24kx+20=0$,

当$\Delta=16(16k^{2}-5)>0$,即$k>\frac{\sqrt{5}}{4}$或$k<-\frac{\sqrt{5}}{4}$时,直线与椭圆有两个公共点.

答案:C



知识:椭圆的性质

难度:1

题目:过椭圆$\frac{x^2}{6}+\frac{y^2}{5}=1$内的一点P(2,-1)的弦,恰好被点P平分,则这条弦所在的直线方程是(  )

A.5x-3y-13=0   

B.5x+3y-13=0

C.5x-3y+13=0   

D.5x+3y+13=0

解析:设弦的端点为$A(x_{1},y_{1}),B(x_{2},y_{2})$,则

$\left\{
\begin{array}{l}
\frac{x_1^2}{6}+\frac{y_1^2}{5}=1, \\
\frac{x_2^2}{6}+\frac{y_1^2}{5}=1
\end{array}
\right.$

故$\frac{1}{6}\times \frac{x_1+x_2}{y_1+y_2}+\frac{1}{5}\times \frac{y_1-y_2}{x_1-x_2}=0$,

又$x_{1}+x_{2}=4,y_{1}+y_{2}=-2$,故斜率$k=\frac{5}{3}$.

故直线方程为$y+1=\frac{5}{3}(x-2)$,即5x-3y-13=0.

答案:A



知识:椭圆的性质

难度:1

题目:已知椭圆$\frac{x^2}{3}+\frac{y^2}{4}=1$的两个焦点为$F_{1},F_{2}$,M是椭圆上一点,且$|MF_1|-|MF_2|=1$,则${\vartriangle}MF_{1}F_{2}$是(  )

A.锐角三角形   

B.钝角三角形

C.直角三角形   

D.等边三角形

解析:由$\frac{x^2}{3}+\frac{y^2}{4}=1$知$a=2,b=\sqrt{3},c=1,e=\frac{1}{2}$,

则$|MF_1|+|MF_2|=4$,又$|MF_1|-|MF_2|=1$,

所以$|MF_1|=\frac{5}{2}, |MF_2|=\frac{3}{2}$.

又$|F_1F_2|=2$,所以$|MF_1|>|F_1F_2|>|MF_2|$.

因为$|F_1F_2|^2+|MF_2|^2=|MF_1|^2$,

所以 ${\vartriangle}MF_{1}F_{2}$是直角三角形.

答案:C

%%
%%
知识:椭圆的性质

难度:1

题目:椭圆$x^{2}+4y^{2}=16$被直线$y=\frac{1}{2}x+1$截得的弦长为\_\_\_\_\_\_\_\_.

解析:由$\left\{
\begin{array}{l}
x^2+4y^2=16, \\
y=\frac{1}{2}+1
\end{array}
\right.$

消去y并化简得$x^{2}+2x-6=0$.

设直线与椭圆的交点为$M(x_{1},y_{1}),N(x_{2},y_{2})$,

则$x_{1}+x_{2}=-2,x_{1}x_{2}=-6$.

所以弦长$|MN|=\sqrt{1+k^2}|x_1-x_2|=$

$\sqrt{\frac{5}{4}[(x_1+x_2)^2]-4x_1x_2}=\sqrt{\frac{5}{4}(4+24)}=\sqrt{35}$

答案:$\sqrt{35}$



知识:椭圆的性质

难度:1

题目:若A为椭圆$x^{2}+4y^{2}=4$的右顶点,以A为直角顶点作一个内接于椭圆的等腰直角三角形,则该三角形的面积为\_\_\_\_\_\_\_\_.

解析:由题意得,该三角形的两直角边关于x轴对称,且其中一边在过点A(2,0),斜率为1的直线上,此直线的方程为y=x-2,将y=x-2代入$x^{2}+4y^{2}=4$,得$5x^{2}-16x+12=0$,解得$x_{1}=2,x_{2}=\frac{6}{5}$.把$x=\frac{6}{5}$代入椭圆方程得$y={\pm}\frac{4}{5}$,所以三角形的面积$S=\frac{1}{2}{\times}\frac{8}{5}\times (2-\frac{6}{5})=\frac{16}{25}$.

答案:$\frac{16}{25}$

%%
%%
知识:椭圆的性质

难度:1

题目:已知动点P(x,y)在椭圆$\frac{x^2}{25}+\frac{y^2}{16}=1$上若A点坐标为(3,0),$\overrightarrow{|AM|}=1$,且$\overrightarrow{PM}\cdot \overrightarrow{AM}=0$,则$\overrightarrow{|PM|}$的最小值是\_\_\_\_\_\_\_\_.

解析:易知点A(3,0)是椭圆的右焦点.

因为$\overrightarrow{PM}\cdot \overrightarrow{AM}=0$,所以$\overrightarrow{PM}\bot \overrightarrow{AM}$.

所以$\overrightarrow{|PM|}^2=\overrightarrow{|AP|}^2-\overrightarrow{|AM|}^2=\overrightarrow{|AP|}^2-1$,

因为椭圆右顶点到右焦点A的距离最小,故$\overrightarrow{|AP|}_{min}=2$,

所以$\overrightarrow{|PM|}_{min}=\sqrt{3}$.

答案:$\sqrt{3}$
%%
%%
%%
知识:椭圆的性质

难度:1

题目:判断直线kx-y+3=0与椭圆$\frac{x^2}{16}+\frac{y^2}{4}=1$的位置关系.

解:由$\left\{
\begin{array}{l}
y=kx+3, \\
\frac{x^2}{16}+\frac{y^2}{4}=1
\end{array}
\right.$可得$(4k^{2}+1)x^{2}+24kx+20=0$,

所以 $\Delta=16(16k^{2}-5)$.

(1)当$\Delta=16(16k^{2}-5)>0$,即$k>\frac{\sqrt{5}}{4}$或$k<-\frac{\sqrt{5}}{4}$时,

直线kx-y+3=0与椭圆$\frac{x^2}{16}+\frac{y^2}{4}=1$相交.

(2)当$\Delta=16(16k^{2}-5)=0$,即$k=\frac{\sqrt{5}}{4}$或$k=-\frac{\sqrt{5}}{4}$时,

直线kx-y+3=0与椭圆$\frac{x^2}{16}+\frac{y^2}{4}=1$相切.

(3)当$\Delta=16(16k^{2}-5)<0$,即$-\frac{\sqrt{5}}{4}<k<\frac{\sqrt{5}}{4}$时,

直线kx-y+3=0与椭圆$\frac{x^2}{16}+\frac{y^2}{4}=1$相离.
%%
%%
%%
知识:椭圆的性质

难度:1

题目:设椭圆C:$\frac{x^2}{a^2}+\frac{y^2}{b^2}=1$(a>b>0)过点(0,4),离心率为$\frac{3}{5}$.

(1)求C的方程;

(2)求过点(3,0)且斜率为$\frac{4}{5}$的直线被C所截线段的中点坐标.

解析:

解:(1)将(0,4)代入C的方程得$\frac{16}{x^2}=1$,所以 b=4.

又$e=\frac{c}{a}=\frac{3}{5}$,得$\frac{a^2-b^2}{a^2}=\frac{9}{25}$,

则$1-\frac{16}{a^2}=\frac{9}{25}$,所以 a=5,

所以 C的方程为$\frac{x^2}{25}+\frac{y^2}{16}=1$.

(2)过点(3,0)且斜率为的直线方程为$y=\frac{4}{5}(x-3)$.设直线与C的交点为$A(x_{1},y_{1}),B(x_{2},y_{2})$,

将直线方程$y=\frac{4}{5}(x-3)$代入C的方程,

得$\frac{x^2}{25}+\frac{(x-3)^2}{25}=1$,即$x^{2}-3x-8=0$,解得$x_{1}+x_{2}=3$,

所以AB的中点坐标$x=\frac{x_1+x_2}{2}=\frac{3}{2},y=\frac{y_1+y_2}{2}=\frac{2}{5}(x_1+x_2-6)=-\frac{6}{5}$,即中点坐标为$(\frac{3}{2},-\frac{6}{5})$.



知识:椭圆的性质

难度:2

题目:若直线$y=x+t$与椭圆$\frac{x^2}{4}+y^{2}=1$相交于A,B两点,当t变化时,$|AB|$的最大值为(  )

A.2   

B.$\frac{4\sqrt{5}}{5}$

C.$\frac{4\sqrt{10}}{5}$

D.$\frac{8\sqrt{10}}{5}$

解析:将y=x+t代入$\frac{x^2}{4}+y^{2}=1$,得$5x^{2}+8tx+4t^{2}-4=0$,则$x_{1}+x_{2}=-\frac{8t}{5},x_{1}x_{2}=\frac{4t^2-4}{5}$.

由$|AB|=\sqrt{1+1^2}\times \sqrt{(x_1+x_2)^2-4x_1x_2}=\sqrt{2}\sqrt{\frac{80-16t^2}{25}}$,当t=0时$|AB|$最大,最大为$\sqrt{2}\times \frac{4\sqrt{5}}{5}=\frac{4\sqrt{10}}{5}$.

答案:C



知识:椭圆的性质

难度:2

题目:已知点P是椭圆$\frac{x^2}{5}+\frac{y^2}{4}=1$上一点,且以点P及焦点$F_{1},F_{2}$为顶点的三角形的面积等于1,则点P的坐标为\_\_\_\_\_\_\_\_.

解析:因为$\frac{x^2}{5}+\frac{y^2}{4}=1$,

所以$a^{2}=5,b^{2}=4$,

所以$c^{2}=a^{2}-b^{2}=1$,

所以$|F_1F_2|=2c=2$.

设点P的纵坐标$y_{p}$,

所以$S_{\vartriangle PF_{1}F_{1}}=\frac{1}{2}|F_1F_2||y_p|$,

所以$|y_p|=1$,

故$y_{p}={\pm}1$,

当$y_{p}=1$时,代入$\frac{x^2}{5}+\frac{y^2}{4}=1$中,可得$x={\pm}\frac{\sqrt{15}}{2}$;

当$y_{p}=-1$时,代入$\frac{x^2}{5}+\frac{y^2}{4}=1$中,可得$x={\pm}\frac{\sqrt{15}}{2}$.

所以点P的坐标为$(\frac{\sqrt{15}}{2},1)$或$(-\frac{\sqrt{15}}{2},1)$或$(\frac{\sqrt{15}}{2},-1)$或$(-\frac{\sqrt{15}}{2},-1)$.

答案:$(\frac{\sqrt{15}}{2},1)$或$(-\frac{\sqrt{15}}{2},1)$或$(\frac{\sqrt{15}}{2},-1)$或$(-\frac{\sqrt{15}}{2},-1)$
%%
%%
%%
知识:椭圆的性质

难度:2

题目:已知椭圆G:$\frac{x^2}{a^2}+\frac{y^2}{b^2}=1$(a${>}$b${>}$0)的离心率为$\frac{\sqrt{6}}{3}$,右焦点为(2,0).斜率为1的直线l与椭圆G交于A,B两点,以AB为底边作等腰三角形,顶点为P(-3,2).

(1)求椭圆G的方程;

(2)求${\vartriangle}$PAB的面积.

解析:

解:(1)由已知得$c=2\sqrt{2},\frac{c}{a}=\frac{\sqrt{6}}{3}$.

解得$a=2\sqrt{3}$.

又$b^{2}=a^{2}-c^{2}=4$,所以椭圆G的方程为$\frac{x^2}{12}+\frac{y^2}{4}=1$.

(2)设直线l的方程为y=x+m.

由$\left\{
\begin{array}{l}
y=x+m, \\
\frac{x^2}{12}+\frac{y^2}{4}=1
\end{array}
\right.$得$4x^{2}+6mx+3m^{2}-12=0$.①

设A,B的坐标分别为$(x_{1},y_{1}),(x_{2},y_{2})(x_{1}{<}x_{2})$,AB中点为$E(x_{0},y_{0})$,则$x_{0}=\frac{x_1+x_2}{2}=-\frac{3m}{4},y_{0}=x_{0}+m=\frac{m}{4}$.于是得$E(-\frac{3m}{4},\frac{m}{4})$.

因为AB是等腰${\vartriangle}$PAB的底边,E为中点,所以PE${\bot}$AB.

所以PE的斜率$k=\frac{2-\frac{m}{4}}{-3+\frac{3m}{4}}=-1$.

解得m=2.

所以直线l的方程为y=x+2.

此时方程①为$4x^{2}+12x=0$.

解得$x_{1}=-3,x_{2}=0$.

所以$y_{1}=-1,y_{2}=2$.所以$|AB|=3\sqrt{2}$.

此时,点P(-3,2)到直线AB:x-y+2=0的距离$d=\frac{|-3-2+2|}{\sqrt{2}}=\frac{3\sqrt{2}}{2}$,所以${\vartriangle}$PAB的面积$S=\frac{1}{2}|AB|\cdot d=\frac{9}{2}$.



知识:双曲线的定义

难度:1

题目:双曲线方程为$x^{2}-2y^{2}=1$,则它的右焦点坐标为(  )

A. $(\frac{\sqrt{2}}{2},0)$      

B. $(\frac{\sqrt{5}}{2},0)$

C. $(\frac{\sqrt{6}}{2},0)$

D.$(\sqrt{3},0)$

解析:将双曲线方程化成标准方程为$\frac{x^2}{1}-\frac{y^2}{\frac{1}{2}}=1$,

所以$a^{2}=1,b^{2}=\frac{1}{2}$,所以$c=\sqrt{a^2+b^2}=\frac{\sqrt{6}}{2}$,故其右焦点坐标为$(\frac{\sqrt{6}}{2},0)$.

答案:C

%%
%%
知识:双曲线的定义

难度:1

题目:若方程$\frac{x^2}{10-k}+\frac{y^2}{5-k}=1$表示双曲线,则k的取值范围是(  )

A.(5,10)   

B.(-${\infty}$,5)

C.(10,+${\infty}$)   

D.(-${\infty}$,5)${\cup}$(10,+${\infty}$)

解析:由题意得(10-k)(5-k)${<}$0,解得5${<}$k${<}$10.

答案:A



知识:双曲线的定义

难度:1

题目:已知双曲线C:$\frac{x^2}{a^2}-\frac{y^2}{b^2}=1$中$\frac{c}{a}=\frac{5}{4}$,且其右焦点为$F_{2}(5,0)$,则双曲线C的方程为(  )

A.$\frac{x^2}{4}-\frac{y^2}{3}=1$   

B.$\frac{x^2}{9}-\frac{y^2}{16}=1$

C.$\frac{x^2}{16}-\frac{y^2}{9}=1$   

D.$\frac{x^2}{3}-\frac{y^2}{4}=1$

解析:由题意得$c=5,\frac{c}{a}=\frac{5}{4}$,所以a=4,则$b^{2}=c^{2}-a^{2}=25-16=9$.所以双曲线的标准方程为$\frac{x^2}{16}-\frac{y^2}{9}=1$.

答案:C



知识:双曲线的定义

难度:1

题目:已知$F_{1}(-5,0),F_{2}(5,0)$为定点,动点P满足$|PF_1|-|PF_2|=2a$,当a=3和a=5时, P点的轨迹分别为(  )

A.双曲线和一条直线

B.双曲线的一支和一条直线

C.双曲线和一条射线

D.双曲线的一支和一条射线

解析:由题意知$|F_1F_2|=10$,因为$|PF_1|-|PF_2|=2a$,所以 当a=3时,$2a=6<|F_1F_2|$,为双曲线的一支,当a=5时,$2a=10=|F_1F_2|$,为一条射线.

答案:D
%%
%%
%%
知识:双曲线的定义

难度:1

题目:椭圆$\frac{x^2}{4}+\frac{y^2}{a^2}=1$与双曲线$\frac{x^2}{a}-\frac{y^2}{2}=1$有相同的焦点,则a的值是(  )

A.$\frac{1}{2}$  

B.1或-2

C.1或$\frac{1}{2}$

D.1

解析:依题意得$\left\{
\begin{array}{l}
a > 0, \\
0 < a^2 < 4, \\
4-a^2=a+2
\end{array}
\right.$解得a=1.

答案:D



知识:双曲线的定义

难度:1

题目:设m是大于0的常数,若点F(0,5)是双曲线$\frac{y^2}{m}-\frac{x^2}{9}=1$的一个焦点,则m=\_\_\_\_\_\_\_\_.

解析:由题意可知m+9=25,所以m=16.

答案:16



知识:双曲线的定义

难度:1

题目:双曲线$\frac{x^2}{25}-\frac{y^2}{9}=1$的两个焦点分别为$F_{1},F_{2}$,双曲线上的点P到$F_{1}$的距离为12,则点P到$F_{2}$的距离为\_\_\_\_\_\_\_\_.

解析:因为$||PF_1|-12|=2a=10$,

所以$|PF_2|=12\pm 10$,即$|PF_2|=2$或$|PF_2|=22$.

答案:2或22



知识:双曲线的定义

难度:1

题目:若双曲线$x^{2}-4y^{2}=4$的左、右焦点分别是$F_{1}、F_{2}$,过$F_{2}$的直线交右支于A、B两点,若$|AB|=5$,则${\vartriangle}AF_{1}B$的周长为\_\_\_\_\_\_\_\_.

解析:由双曲线定义可知$|AF_1|=2a+|AF_2|=4+|AF_2|;|BF_1|=2a+|BF_2|=4+|BF_2|$,

所以$|AF_1|+|BF_1|=8+|AF_2|+|BF_2|=8+|AB|=13$.

${\vartriangle}AF_{1}B$的周长为$|AF_1|+|BF_1|+|AB|=18$.

答案:18

%%
%%
知识:双曲线的定义

难度:1

题目:双曲线$\frac{x^2}{m}-\frac{y^2}{m-5}=1$的一个焦点到中心的距离为3,那么m的取值范围.

解析:

解:(1)当焦点在x轴上,有m>5,则$c^{2}=m+m-5=9$,

所以 m=7;

(2)当焦点在y轴上,有m<0,则$c^{2}=-m+5-m=9$,

所以 m=-2.

综上所述,m=7或m=-2.



知识:双曲线的定义

难度:1

题目:已知k为实常数,命题p:方程$(k-1)x^{2}+(2k-1)y^{2}=(2k-1)(k-1)$表示椭圆,命题q:方程$(k-3)x^{2}+4y^{2}=4(k-3)$表示双曲线.

(1)若命题p为真命题,求实数k的取值范围;

(2)若命题p,q中恰有一个为真命题,求实数k的取值范围.

解析:

解:(1)若命题p为真命题,则$\left\{
\begin{array}{l}
2k-1>0, \\
k-1>0, \\
2k-1\ne k-1
\end{array}
\right.$解得k>1,即实数k的取值范围是$(1,+{\infty})$.

(2)当p真q假时,$\left\{
\begin{array}{l}
k>1, \\
k\ge 3
\end{array}
\right.$解得k${\ge}$3,

当p假q真时,$\left\{
\begin{array}{l}
k\le 1, \\
k< 3
\end{array}
\right.$解得k${\le}$1,

故实数k的取值范围是(-${\infty}$,1]${\cup}$[3,+${\infty}$).



知识:双曲线的定义

难度:2

题目:k<2是方程$\frac{x^2}{4-k}+\frac{y^2}{k-2}=1$表示双曲线的(  )

A.充分不必要条件

B.必要不充分条件

C.充要条件

D.既不充分也不必要条件

解析:$k<2 \Rightarrow $方程$\frac{x^2}{4-k}+\frac{y^2}{k-2}=1$表示双曲线,而方程$\frac{x^2}{4-k}+\frac{y^2}{k-2}=1$表示双曲线$\Rightarrow (4-k)(k-2)<0 \Rightarrow k<2$或$k>4$,故$k<2$是方程$\frac{x^2}{4-k}+\frac{y^2}{k-2}=1$表示双曲线的充分不必要条件.

答案:A



知识:双曲线的定义

难度:2

题目:过点$P_{1}(2,1)$和$P_{2}(-3,2)$的双曲线的方程是\_\_\_\_\_\_\_\_.

解析:设方程为$ax^{2}+by^{2}=1(ab<0)$,则$\left\{
\begin{array}{l}
4a+b=1, \\
9a+4b=1
\end{array}
\right.$解方程组得$\left\{
\begin{array}{l}
a=\frac{3}{7}, \\
b=-\frac{5}{7}
\end{array}
\right.$所以双曲线的方程是$\frac{3x^2}{7}-\frac{5y^2}{7}=1$.

答案:$\frac{3x^2}{7}-\frac{5y^2}{7}=1$

%%
%%
知识:双曲线的定义

难度:2

题目:已知双曲线$16x^{2}-9y^{2}=144,F_{1},F_{2}$是左右两焦点,点P在双曲线上,且$|PF_1|\cdot |PF_2|=32$,求${\angle}F_{1}PF_{2}$.

解析:

解:由题意知$||PF_1|-|PF_2||=6$,

所以$(|PF_1|-|PF_2|)^2=|PF_1|^2+|PF_2|^2-2|PF_1|\cdot |PF_2|=36$.所以$|PF_1|^2+|PF_2|^2=36+2\times 32=100$.

又由题意知$|F_1F_2|=2c=10$,

所以$\cos \angle F_1PF_2 =\frac{|PF_1|^2+|PF_2|^2-|F_1F_2|^2}{2|PF_1|\cdot |PF_2|}=\frac{100-100}{2|PF_1|\cdot |PF_2|}=0$.

所以 ${\angle}F_{1}PF_{2}=90^{\circ}$.



知识:双曲线的性质

难度:1

题目:双曲线$2x^{2}-y^{2}=8$的实轴长是(  )

A.2  

B.$2\sqrt{2}$  

C.4  

D.$4\sqrt{2}$

解析:双曲线方程可变形为$\frac{x^2}{4}-\frac{y^2}{8}=1$,所以$a^{2}=4,a=2$,从而2a=4.

答案:C

知识:双曲线的性质

难度:1

题目:等轴双曲线的一个焦点是$F_{1}(-6,0)$,则其标准方程为(  )

A.$\frac{x^2}{9}-\frac{y^2}{9}=1$   

B.$\frac{y^2}{9}-\frac{x^2}{9}=1$

C.$\frac{y^2}{18}-\frac{x^2}{18}=1$   

D.$\frac{x^2}{18}-\frac{y^2}{18}=1$

解析:由已知可得c=6,所以$a=b=\frac{\sqrt{2}}{2}c=3\sqrt{2}$,

所以 双曲线的标准方程是$\frac{x^2}{18}-\frac{y^2}{18}=1$.

答案:D
%%
%%
%%
知识:双曲线的第二定义

难度:1

题目:已知双曲线$\frac{x^2}{3}-\frac{y^2}{b}=1$(b${>}$0)的焦点到其渐近线的距离为1,则该双曲线的离心率为(  )

A.$\sqrt{2}$  

B.$\sqrt{3}$

C.$\frac{2\sqrt{3}}{3}$

D.$\frac{3\sqrt{2}}{2}$

解析:由题意及对称性可知焦点$(\sqrt{b^2+3},0)$到$bx-\sqrt{3}y=0$的距离为1,即$\frac{|\sqrt{b^2+3}\cdot b|}{\sqrt{b^2+3}}=1$,所以b=1,所以c=2,又$a=\sqrt{3}$,所以双曲线的离心率为$\frac{2\sqrt{3}}{3}$.

答案:C



知识:双曲线的第二定义

难度:1

题目:已知双曲线C:$\frac{x^2}{a^2}-\frac{y^2}{b^2}=1$(a>0,b>0)的离心率为$\frac{\sqrt{5}}{2}$,则C的渐近线方程为(  )

A.y=${\pm}\frac{1}{4}$x   

B.y=${\pm}\frac{1}{3}$x

C.y=${\pm}\frac{1}{2}$x   

D.y=${\pm}$x

解析:因为双曲线$\frac{x^2}{a^2}-\frac{y^2}{b^2}=1$的焦点在x轴上,所以双曲线的渐近线方程为y=${\pm}\frac{b}{a}$x.

又离心率为$e=\frac{c}{a}=\frac{\sqrt{a^2+b^2}}{a}=\sqrt{1+(\frac{b}{a})^2}=\frac{\sqrt{5}}{2}$,

所以$\frac{b}{a}=\frac{1}{2}$,所以双曲线的渐近线方程为$y={\pm} \frac{1}{2}x$.

答案:C



知识:双曲线的第二定义

难度:1

题目:双曲线C:$\frac{x^2}{a^2}-\frac{y^2}{b^2}=1$(a>0,b>0)的离心率为2,焦点到渐近线的距离为$\sqrt{3}$,则C的焦距等于(  )

A.2  

B.$2\sqrt{2}$  

C.4  

D.$4\sqrt{2}$

解析:双曲线的一条渐近线方程为$\frac{x}{a}-\frac{y}{b}=0$,即bx-ay=0,焦点(c,0)到该渐近线的距离为$\frac{bc}{\sqrt{a^2+b^2}}=\frac{bc}{c}=\sqrt{3}$,故$b=\sqrt{3}$,结合$\frac{c}{a}=2,c^{2}=a^{2}+b^{2}$得c=2,则双曲线C的焦距为2c=4.

答案:C
%%
%%
%%
知识:双曲线的第二定义

难度:1

题目:已知双曲线$\frac{x^2}{n}-\frac{y^2}{12-n}=1$(0${<}$n${<}$12)的离心率为$\sqrt{3}$,则n的值为\_\_\_\_\_\_\_\_.

解析:因为0${<}$n${<}$12,所以$a^{2}=n,b^{2}=12-n$.

所以$c^{2}=a^{2}+b^{2}=12$.所以$e=\frac{c}{a}=\frac{\sqrt{12}}{\sqrt{n}}=\sqrt{3}$.

所以n=4.

答案:4



知识:双曲线的性质

难度:1

题目:(2016·北京卷)已知双曲线$\frac{x^2}{a^2}-\frac{y^2}{b^2}=1$(a${>}$0,b${>}$0)的一条渐近线为2x+y=0,一个焦点为$(\sqrt{5},0)$,则a=\_\_\_\_,b=\_\_\_\_\_\_\_\_.

解析:因为双曲线$\frac{x^2}{a^2}-\frac{y^2}{b^2}=1$(a${>}$0,b${>}$0)的一条渐近线为2x+y=0,即y=-2x,所以$\frac{b}{a}=2$.①

又双曲线的一个焦点为$(\sqrt{5},0)$,所以$a^{2}+b^{2}=5$.②

由①②得a=1,b=2.

答案:1 2



知识:双曲线的第二定义

难度:1

题目:双曲线$\frac{x^2}{4}+\frac{y^2}{k}=1$的离心率e${\in}$(1,2),则k的取值范围是\_\_\_\_\_\_\_\_.

解析:双曲线方程可变为$\frac{x^2}{4}-\frac{y^2}{-k}=1$,则$a^{2}=4,b^{2}=-k,c^{2}=4-k,e=\frac{c}{a}=\frac{\sqrt{4-k}}{2}$,又因为e${\in}$(1,2),则$1<\frac{\sqrt{4-k}}{2}<2$,解得-12<k<0

答案:(-12,0)



知识:双曲线的性质

难度:1

题目:求适合下列条件的双曲线的标准方程:

(1)过点$(3,-\sqrt{2})$,离心率$e=\frac{\sqrt{5}}{2}$;

(2)中心在原点,焦点$F_{1},F_{2}$在坐标轴上,实轴长和虚轴长相等,且过点$P(4,-\sqrt{10})$.

解析:

解:(1)若双曲线的焦点在x轴上,设其标准方程为$\frac{x^2}{a^2}-\frac{y^2}{b^2}=1$(a>0,b>0).

因为双曲线过点$(3,-\sqrt{2})$,则$\frac{9}{a^2}-\frac{2}{b^2}=1$.①

又$e=\frac{c}{a}=\sqrt{\frac{a^2+b^2}{a^2}}=\frac{\sqrt{5}}{2}$ 故$a^{2}=4b^{2}$.②

由①②得$a^{2}=1,b^{2}=\frac{1}{4}$,故所求双曲线的标准方程为$x^{2}-\frac{y^2}{\frac{1}{4}}=1$.

若双曲线的焦点在y轴上,设其标准方程为$\frac{y^2}{a^2}-\frac{x^2}{b^2}=1$(a>0,b>0).同理可得$b^{2}=-\frac{17}{2}$,不符合题意.

综上可知,所求双曲线的标准方程为$x^{2}-\frac{y^2}{\frac{1}{4}}=1$.

(2)由2a=2b得a=b,所以$e=\sqrt{1+\frac{b^2}{a^2}}=\sqrt{2}$,

所以可设双曲线方程为$x^{2}-y^{2}=\lambda$($\lambdaup$${\neq}$0).

因为双曲线过点P$(4,-\sqrt{10})$,

所以 16-10=$\lambda$,即$\lambda$=6.

所以 双曲线方程为$x^{2}-y^{2}$=6.

所以 双曲线的标准方程为$\frac{x^2}{6}-\frac{y^2}{6}=1$.

%%
%%
知识:双曲线的性质

难度:1

题目:设双曲线C:$\frac{x^2}{a^2}-y^{2}=1$(a>0)与直线l:x+y=1相交于两个不同的点A、B.

(1)求实数a的取值范围;

(2)设直线l与y轴的交点为P,若$\overrightarrow{PA}=\frac{5}{12}\overrightarrow{PB}$,求a的值.

解析:

解:(1)将y=-x+1代入双曲线方程$\frac{x^2}{a^2}-y^{2}=1$(a>0)中得$(1-a^{2})x^{2}+2a^{2}x-2a^{2}=0$.

依题意$\left\{
\begin{array}{l}
1-a^2\ne 0, \\
\Delta = 4a^4+8a^2(1-a^2)>0
\end{array}
\right.$

所以$0<a<\sqrt{2}$且a${\neq}$1.

(2)设$A(x_{1},y_{1}),B(x_{2},y_{2}),P(0,1)$,

因为$\overrightarrow{PA}=\frac{5}{12}\overrightarrow{PB}$,所以$(x_{1},y_{1}-1)=\frac{5}{12}(x_{2},y_{2}-1)$.

由此得$x_{1}=\frac{5}{12}x_{2}$.

由于$x_{1},x_{2}$是方程$(1-a^{2})x^{2}+2a^{2}x-2a^{2}=0$的两根,且$1-a^{2}\ne0$,所以$\frac{17}{12}x_2=-\frac{2a^2}{1-a^2},\frac{5}{12}x_2^2=-\frac{2a^2}{1-a^2}$

消去$x_{2}$得$-\frac{2a^2}{1-a^2}=\frac{289}{60}$.

由a>0,解得$a=\frac{17}{13}$.



知识:双曲线的性质

难度:2

题目:若$0<k<a^{2}$,则双曲线$\frac{x^2}{a^2-k}-\frac{y^2}{b^2+k}=1$与$\frac{x^2}{a^2}-\frac{y^2}{b^2}=1$有(  )

A.相同的虚线   

B.相同的实轴

C.相同的渐近线   

D.相同的焦点

解析:因为$0<k<a^{2}$,所以$a^{2}-k>0$.对于双曲线$\frac{x^2}{a^2-k}-\frac{y^2}{b^2+k}=1$,焦点在x轴上且$c^{2}=a^{2}-k+b^{2}+k=a^{2}+b^{2}$.同理双曲线$\frac{x^2}{a^2}-\frac{y^2}{b^2}=1$焦点在x轴上且$c^{2}=a^{2}+b^{2}$,故它们有共同的焦点.

答案:D



知识:双曲线的第二定义

难度:2

题目:已知$F_{1},F_{2}$是双曲线$\frac{x^2}{a^2}-\frac{y^2}{b^2}=1$(a>0,b>0)的两焦点,以线段$F_{1}F_{2}$为边作正三角形$MF_{1}F_{2}$,若边$MF_{1}$的中点P在双曲线上,则双曲线的离心率是\_\_\_\_\_\_\_\_.

解析:如图,连接$F_{2}P$,P是$MF_{1}$中点,则$PF_{2}{\bot}MF_{1}$,在正三角形$MF_{1}F_{2}$中,$|F_1F_2|=2c$,则$|PF_1|=c, |PF_2|=\sqrt{3}c$.

\includegraphics*[width=1.30in, height=1.20in, keepaspectratio=false]{image6}

因为P在双曲线上,

所以$|PF_2|-|PF_1|=2a$

而$\sqrt{3}c-c=2a$

所以$\frac{c}{a}=\frac{2}{\sqrt{3}-1}=\frac{2(\sqrt{3}+1)}{(\sqrt{3}-1)(\sqrt{3}+1)}=\sqrt{3}+1$.

答案:$\sqrt{3}+1$
%%
%%
%%
知识:双曲线的性质

难度:2

题目:已知直线kx-y+1=0与双曲线$\frac{x^2}{2}-y^{2}=1$相交于两个不同点A,B.

(1)求k的取值范围;

(2)若x轴上的点M(3,0)到A,B两点的距离相等,求k的值.

解析:

解:(1)由$\left\{
\begin{array}{l}
kx-y+1=0, \\
\frac{x^2}{2}-y^2=1
\end{array}
\right.$得$(1-2k^{2})x^{2}-4kx-4=0$.

所以$\left\{
\begin{array}{l}
1-2k^2\ne 0, \\
\Delta =16k^2+16(1-2k^2)=16(1-k^2)>0
\end{array}
\right.$

解得:-1${<}$k${<}$1,且k${\neq}$${\pm}\frac{\sqrt{2}}{2}$.

(2)设$A(x_{1},y_{1}),B(x_{2},y_{2})$,

则$x_{1}+x_{2}=\frac{4k}{1-2k^2}$,

设P为AB中点,则$P(\frac{x_1+x_2}{2},\frac{k(x_1+x_2)}{2}+1)$,

即$P(\frac{2k}{1-2k^2},\frac{1}{1-2k^2})$,

因为M(3,0)到A,B两点的距离相等,

所以$MP{\bot}AB$,所以$k_{MP}\cdot k_{AB}$=-1,

即$k\cdot \frac{\frac{1}{1-2k^2}}{\frac{2k}{1-2k^2}-3}=-1$,解得$k=\frac{1}{2}$或$k=-1$(舍去),

所以$k=\frac{1}{2}$.



知识:抛物线的定义

难度:1

题目:准线方程为$y=\frac{2}{3}$的抛物线的标准方程为(  )

A.$x^{2}=\frac{8}{3}y$   

B.$x^{2}=-\frac{8}{3}y$

C.$y^{2}=-\frac{8}{3}x$   

D.$y^{2}=\frac{8}{3}x$

解析:由准线方程为$y=\frac{2}{3}$,知抛物线焦点在y轴负半轴上,且$\frac{p}{2}=\frac{2}{3}$,则$p=\frac{4}{3}$.故所求抛物线的标准方程为$x^{2}=-\frac{8}{3}y$.

答案:B



知识:抛物线的定义

难度:1

题目:已知抛物线$y-2016x^{2}=0$,则它的焦点坐标是(  )

A.(504,0)   

B. $(\frac{1}{8064}, 0)$

C. $(0, \frac{1}{8064})$ 

D.$(0, \frac{1}{504})$

解析:抛物线的标准方程为$x^{2}=\frac{1}{2016}y$,故其焦点为$(0,\frac{1}{8064})$.

答案:C



知识:抛物线的定义

难度:1

题目:抛物线$y=12x^{2}$上的点到焦点的距离的最小值为(  )

A.3  

B.6  

C.$\frac{1}{48}$

D.$\frac{1}{24}$

解析:将方程化为标准形式是$x^{2}=\frac{1}{12}y$,因为$2p=\frac{1}{12}$,所以$p=\frac{1}{24}$.故到焦点的距离最小值为$\frac{1}{48}$.

答案:C
%%
%%
%%
知识:抛物线的定义

难度:1

题目:一动圆的圆心在抛物线$y^{2}=8x$上,且动圆恒与直线x+2=0相切,则动圆过定点(  )

A.(4,0)   

B.(2,0)

C.(0,2)   

D.(0,4)

解析:由题意易知直线x+2=0为抛物线$y^{2}=8x$的准线,由抛物线的定义知动圆一定过抛物线的焦点.

答案:B



知识:抛物线的定义

难度:1

题目:抛物线$y^{2}=2px$(p${>}$0)上有$A(x_{1},y_{1}),B(x_{2},y_{2}),C(x_{3},y_{3})$三点,F是焦点,$|AF|,|BF|,|CF|$成等差数列,则(  )

A.$x_{1}$,$x_{2}$,$x_{3}$成等差数列

B.$x_{1}$,$x_{3}$,$x_{2}$成等差数列

C.$y_{1}$,$y_{2}$,$y_{3}$成等差数列

D.$y_{1}$,$y_{3}$,$y_{2}$成等差数列

解析:由抛物线的定义知$|AF|=x_1+\frac{p}{2},|BF|=x_2+\frac{p}{2},|CF|=x_3+\frac{p}{2}$.

因为$|AF|,|BF|,|CF|$成等差数列,

所以$2(x_2+\frac{p}{2})=(x_1+\frac{p}{2})+(x_3+\frac{p}{2})$,即$2x_{2}=x_{1}+x_{3}$.故$x_{1},x_{2},x_{3}$成等差数列.故选A.

答案:A



知识:抛物线的定义

难度:1

题目:抛物线$y^{2}=2x$上的两点A,B到焦点的距离之和是5,则线段AB中点的横坐标是\_\_\_\_\_\_\_\_.

解析:由抛物线的定义知点A,B到准线的距离之和是5,则AB的中点到准线的距离为$\frac{5}{2}$,故AB中点的横坐标为$x=\frac{5}{2}-\frac{1}{2}=2$.

答案:2



知识:抛物线的定义

难度:1

题目:抛物线过原点,焦点在y轴上,其上一点P(m,1)到焦点的距离为5,则抛物线的标准方程是\_\_\_\_\_\_\_\_.

解析:由题意,知抛物线开口向上,且$1+\frac{p}{2}=5$,所以p=8,即抛物线的标准方程是$x^{2}=16y$.

答案:$x^{2}=16y$



知识:抛物线的定义

难度:1

题目:焦点为F的抛物线$y^{2}=2px$(p>0)上一点M在准线上的射影为N,若$|MN|=p$,则$|FN|$=\_\_\_\_\_\_\_\_.

解析:由条件知$|MF|=|MN|=p,MF\bot MN$,在$\vartriangle MNF$中,$\angle FMN=90^{\circ}$,得$|FN|=\sqrt{2}p$.

答案:p


%%
知识:抛物线的定义

难度:1

题目:求满足下列条件的抛物线的标准方程.

(1)焦点在坐标轴上,顶点在原点,且过点(-3,2);

(2)顶点在原点,以坐标轴为对称轴,焦点在直线x-2y-4=0上.

解析:

解:(1)当焦点在x轴上时,设抛物线的标准方程为$y^{2}=-2px$(p${>}$0).把(-3,2)代入,得$2^{2}=-2p\times (-3)$,解得$p=\frac{2}{3}$.

所以所求抛物线的标准方程为$y^{2}=-\frac{4}{3}x$.

当焦点在y轴上时,设抛物线的标准方程为$x^{2}=2py$(p${>}$0).

把(-3,2)代入,得$(-3)^{2}=4p$,解得$p=\frac{9}{4}$.

所以所求抛物线的标准方程为$x^{2}=\frac{9}{2}y$.

(2)直线x-2y-4=0与x轴的交点为(4,0),与y轴的交点为(0,-2),故抛物线的焦点为(4,0)或(0,-2).

当焦点为(4,0)时,设抛物线方程为$y^{2}=2px$(p${>}$0),

则$\frac{p}{2}=4$,所以$p=8$.所以抛物线方程为$y^{2}=16x$.

当焦点为(0,-2)时,设抛物线方程为$x^{2}=-2py$(p${>}$0),则$-\frac{p}{2}=-2$,所以p=4.所以抛物线方程为$x^{2}=-8y$.



知识:抛物线的定义

难度:1

题目:已知动圆M与直线y=2相切,且与定圆C:$x^{2}+(y+3)^{2}=1$外切,求动圆圆心M的轨迹方程.

解析:

解:设动圆圆心为M(x,y),半径为r,则由题意可得M到C(0,-3)的距离与到直线y=3的距离相等,

则动圆圆心的轨迹是以C(0,-3)为焦点,y=3为准线的一条抛物线,其方程为$x^{2}=-12y$.



知识:抛物线的定义

难度:2

题目:点M(5,3)到抛物线$y=ax^{2}$的准线的距离为6,那么抛物线的方程是(  )

A.$y=12x^{2}$

B.$y=12x^{2}$或$y=-36x^{2}$

C.$y=-36x^{2}$

D.$y=\frac{1}{12}x^{2}$或$y=-\frac{1}{36}x^{2}$

解析:当a>0时,抛物线开口向上,准线方程为$y=-\frac{1}{4a}$,则点M到准线的距离为$3+\frac{1}{4a}=6$,解得$a=\frac{1}{12}$,抛物线方程为$y=\frac{1}{12}x^{2}$.当a<0时,开口向下,准线方程为$y=-\frac{1}{4a}$,点M到准线的距离为$|3+\frac{1}{4a}|=6$,解得$a=-\frac{1}{36}$,抛物线方程为$y=-\frac{1}{36}x^{2}$.

答案:D

%%
%%
知识:抛物线的定义

难度:2

题目:已知直线$l_{1}:4x-3y+6=0$和直线$l_{2}:x=-1$,抛物线$y^{2}=4x$上一动点P到直线$l_{1}$和直线$l_{2}$的距离之和的最小值为\_\_\_\_\_\_\_\_.

解析:由已知得抛物线的焦点为$F(1,0)$,由抛物线的定义知:动点P到直线$l_{1}$和直线$l_{2}$的距离之和的最小值即为焦点$F(1,0)$到直线$l_{1}:4x-3y+6=0$的距离,由点到直线的距离公式得:$d=\frac{|4-0+6|}{\sqrt{4^2+(-3)^2}}=2$,所以动点P到直线$l_{1}$和直线$l_{2}$的距离之和的最小值是2.

答案:2



知识:抛物线的定义

难度:2

题目:抛物线$y^{2}=2px(p>0)$且一个内接直角三角形,直角顶点是原点,一条直角边所在直线方程为y=2x,斜边长为$5\sqrt{13}$,求此抛物线方程.

解析:

解:设抛物线$y^{2}=2px(p>0)$的内接直角三角形为AOB,直角边OA所在直线方程为y=2x,另一直角边所在直线方程为$y=-\frac{1}{2}x$.解方程组$\left\{
\begin{array}{l}
y=2x, \\
y^2=2px
\end{array}
\right.$可得点A的坐标为$(\frac{p}{2},p)$;解方程组$\left\{
\begin{array}{l}
y=-\frac{1}{2}x, \\
y^2=2px
\end{array}
\right.$可得点B的坐标为(8p,-4p).

因为$|OA|^2+|OB|^2=|AB|^2$,且$|AB|=5\sqrt{13}$,

所以$(\frac{p^2}{4}+p^2)+(64p^2+16p^2)=325$.

所以p=2,所以所求的抛物线方程为$y^{2}=4x$.



知识:抛物线的性质

难度:1

题目:已知抛物线的对称轴为x轴,顶点在原点,焦点在直线2x-4y+11=0上,则此抛物线的方程是(  )

A.$y^{2}=-11x$   

B.$y^{2}=11x$

C.$y^{2}=-22x$   

D.$y^{2}=22x$

解析:令y=0得$x=-\frac{11}{2}$,

所以 抛物线的焦点为F$(-\frac{11}{2},0)$,

即$\frac{p}{2}=\frac{11}{2}$,所以 p=11,

所以 抛物线的方程是$y^{2}=-22x$.

答案:C

知识:抛物线的性质

难度:1

题目:过抛物线$y^{2}=8x$的焦点作倾斜角为$45^{\circ}$的直线,则被抛物线截得的弦长为(  )

A.8    B.16    C.32   D.64

解析:由题可知抛物线$y^{2}=8x$的焦点为(2,0),直线的方程为y=x-2,代入$y^{2}=8x$,得$(x-2)^{2}=8x$,即$x^{2}-12x+4=0$,所以$x_{1}+x_{2}=12$,弦长=$x_{1}+x_{2}+p=12+4=16$.

答案:B
%%
%%
%%
知识:抛物线的性质

难度:1

题目:已知抛物线$y^{2}=2px$(p>0)的焦点为F,点$P_{1}(x_{1},y_{1}),P_{2}(x_{2},y_{2}),P_{3}(x_{3},y_{3})$在抛物线上,且$2x_{2}=x_{1}+x_{3}$,则有(  )

A.$|FP_1|+|FP_2|=|FP_3|$

B.$|FP_1|^2+|FP_2|^2=|FP_3|^2$

C.$|FP_1|+|FP_3|=2|FP_2|$

D.$|FP_1|\cdot|FP_3|=|FP_2|^2$

解析:由焦半径公式,知$|FP_1|=x_1+\frac{p}{2},|FP_2|=x_2+\frac{p}{2},|FP_3|=x_3+\frac{p}{2}$

因为$2x_{2}=x_{1}+x_{3}$,

所以$2(x_2+\frac{p}{2})=(x_1+\frac{p}{2})+(x_3+\frac{p}{2})$,

即$2|FP_2|=|FP_1|+|FP_3|$

答案:C
%%
知识:抛物线的性质

难度:1

题目:过抛物线$y^2=2px(p>0)$的焦点作一条直线交抛物线于点$A(x_1,y_1)$,$B(x_2,y_2)$,则$\frac{y_1y_2}{x_1x_2}$的值为(  )

A.4  

B.-4  

C.$p^2$  

D.$-p^2$

解析:法一(特例法):当直线垂直于x轴时,$A(\frac{p}{2},p)$,$B(\frac{p}{2},-p)$,则$\frac{y_1y_2}{x_1x_2}=\frac{-p^2}{\frac{p^2}{4}}=4$.

法二:由焦点弦所在直线方程与抛物线方程联立,可得$y_1y_2=-p^2$,则$\frac{y_1y_2}{x_1x_2}=\frac{y_1\cdot y_2}{\frac{y_1^2}{2p}\cdot\frac{y_2^2}{2p}}=\frac{4p^2}{y_1y_2}=\frac{4p^2}{-p^2}=-4$.

答案:B



知识:抛物线的性质

难度:1

题目:过抛物线$y^{2}=2px(p>0)$的焦点F的直线与抛物线交于A、B两点,若A、B在准线上的射影为$A_{1}$、$B_{1}$,则${\angle}A_{1}FB_{1}$等于(  )

A.$90{{}^\circ}$  

B.$45{{}^\circ}$  

C.$60{{}^\circ}$ 

D.$120{{}^\circ}$

解析:如图,由抛物线定义知$|AA_{1}|=|AF|$,$|BB_{1}|=|BF|$,所以${\angle}AA_{1}F={\angle}AFA_{1}$,又${\angle}AA_{1}F={\angle}A_{1}FO$,

\includegraphics*[width=1.18in, height=1.29in, keepaspectratio=false]{image7}

所以 ${\angle}AFA_{1}={\angle}A_{1}FO$,

同理${\angle}BFB_{1}={\angle}B_{1}FO$,

于是${\angle}AFA_{1}+{\angle}BFB_{1}={\angle}A_{1}FO+{\angle}B_{1}FO={\angle}A_{1}FB_{1}$.故${\angle}A_{1}FB_{1}=90{{}^\circ}$.

答案:A



知识:抛物线的性质

难度:1

题目:抛物线$y^{2}=4x$的弦AB垂直于x轴,若$|AB|=4$,则焦点到弦AB的距离为\_\_\_\_\_\_\_\_.

解析:由题意我们不妨设A(x,2),则$(2)^{2}=4x$,所以$x=3$,所以直线AB的方程为$x=3$,又抛物线的焦点为$(1,0)$,

所以焦点到弦AB的距离为2.

答案:2



知识:抛物线的性质

难度:1

题目:抛物线$y^{2}=4x$与直线$2x+y-4=0$交于两点A与B,F为抛物线的焦点,则$|FA|+|FB|=$\_\_\_\_\_\_\_\_.

解析:设$A(x_{1},y_{1}),B(x_{2},y_{2})$,

则$|FA|+|FB|=x_{1}+x_{2}+2$.

又$
\begin{cases}
y^2=4x,\\
2x+y-4=0,
\end{cases}
\Rightarrow x^{2}-5x+4=0$,

所以 $x_{1}+x_{2}=5$,$|FA|+|FB|=x_{1}+x_{2}+2=7$.

答案:7



知识:抛物线的性质

难度:1

题目:在抛物线$y^{2}=16x$内,过点(2,1)且被此点平分的弦AB所在直线的方程是\_\_\_\_\_\_\_\_.

解析:显然斜率不存在时的直线不符合题意.设直线斜率为k,则直线方程为$y-1=k(x-2)$,①

由
$
\begin{cases}
y-1=k(x-2),\\
y^2=16x,
\end{cases}$

消去$x$得$ky^{2}-16y+16(1-2k)=0$,

所以$y_{1}+y_{2}=\frac{16}{k}=2(y_{1},y_{2}$分别是A,B的纵坐标),

所以$k=8$.代入①得$y=8x-15$.

答案:$y=8x-15$



知识:抛物线的性质

难度:1

题目:已知过抛物线$y^{2}=4x$的焦点F的弦长为36,求弦所在的直线方程.

解析:

解:因为过焦点的弦长为36,

所以 弦所在的直线的斜率存在且不为零.

故可设弦所在直线的斜率为k,

且与抛物线交于$A(x_{1},y_{1})$、$B(x_{2},y_{2})$两点.

因为抛物线$y^{2}=4x$的焦点为$F(1,0)$.

所以 直线的方程为$y=k(x-1)$.

由$
\begin{cases}
y=k(x-1),\\
y^2=4x,
\end{cases}$整理得$k^{2}x^{2}-(2k^{2}+4)x+k^{2}=0(k\neq0)$.

所以 $x_{1}+x_{2}=\frac{2k^2+4}{k^2}$.

所以 $|AB|=|AF|+|BF|=x_{1}+x_{2}+2=\frac{2k^2+4}{k^2}+2$.

又$|AB|=36$,所以$\frac{2k^2+4}{k^2} +2=36$,所以 $k=\pm$.

所以 所求直线方程为$y=\frac{\sqrt{2}}{4}(x-1)$或$y=-\frac{\sqrt{2}}{4}(x-1)$.



知识:抛物线的性质

难度:1

题目:正三角形的一个顶点位于坐标原点,另外两个顶点在抛物线$y^{2}=2px(p>0)$上,求这个正三角形的边长.

解析:

解:如图所示:设正三角形OAB的顶点A,B在抛物线上,且坐标分别为$A(x_{1},y_{1})$,$B(x_{2},y_{2})$,

\includegraphics*[width=0.90in, height=0.92in, keepaspectratio=false]{image8}

则$y=2px_{1}$,$y=2px_{2}$.

又因为$|OA|=|OB|$,

所以 ${x_1}^2+{y_1}^2={x_2}^2+{y_2}^2$,即${x_1}^2-{x_2}^2+2px_1-2px_2$,

整理得$(x_{1}-x_{2})(x_{1}+x_{2}+2p)=0$.

因为$x_{1}>0,x_{2}>0$,$2p>0$,所以 $x_{1}=x_{2}$,

由此可得$|y_{1}|=|y_{2}|$,即点A,B关于x轴对称.

由此得${\angle}AOx=30{{}^\circ}$,

所以 $y_{1}=\frac{\sqrt{3}}{3}x_{1}$,与${y_1}^2=2px_{1}$联立,解得$y_{1}=2\sqrt{3}p$.

所以 $|AB|=2y_{1}=4\sqrt{3}p$.

\noindent 知识:抛物线的性质

难度:2

题目:在同一平面直线坐标系中,方程$a^{2}x^{2}+b^{2}y^{2}=1$与$ax+by^{2}=0(a>b>0)$的曲线大致为(  )

\includegraphics*[width=2.17in, height=1.90in, keepaspectratio=false]{image9}



解析:将方程$a^{2}x^{2}+b^{2}y^{2}=1$与$ax+by^{2}=0$转化为$\frac{x^2}{\frac{1}{a^2}}+\frac{y^2}{\frac{1}{b^2}}=1$与$y^{2}=-\frac{a}{b}x$.因为$a>b>0$,所以$\frac{1}{b}>\frac{1}{a}>0$,

所以椭圆的焦点在y轴上,抛物线的焦点在x轴上,且开口向左.

答案:D



\noindent 知识:抛物线的性质

难度:2

题目:设A,B是抛物线$x^{2}=4y$上两点,O为原点,若$|OA|=|OB|$,且${\vartriangle}AOB$的面积为16,则${\angle}AOB$等于\_\_\_\_\_\_\_\_.

解析:由$|OA|=|OB|$,知抛物线上点A,B关于y轴对称.

设$A(-a,\frac{a^2}{4})$,$B(a,\frac{a^2}{4})$,$a>0$,$S_{\triangle AOB}=\frac{1}{2}{\times}2a{\times}\frac{a^2}{4}=16$,解得$a=4$.所以 $\triangle AOB$为等腰直角三角形,$\angle AOB=90{{}^\circ}$.

答案:90${{}^\circ}$



\noindent 知识:抛物线的性质

难度:2

题目:已知过抛物线$y^{2}=2px(p{>}0)$的焦点,斜率为$2\sqrt{2}$ 的直线交抛物线于$A(x_{1},y_{1})$,$B(x_{2},y_{2})(x_{1}{<}x_{2})$两点,且$|AB|=9$.

(1)求该抛物线的方程;

(2)O为坐标原点,C为抛物线上一点,若$\vec{OC}=\vec{OA}=+\lambdaup\vec{OB}$,求$\lambdaup$的值.

解析:

解:(1)直线AB的方程是$y=2\sqrt{2}(x-\frac{p}{2})$,与$y^{2}=2px$联立,消去y得$4x^{2}-5px+p^{2}=0$,所以$x_{1}+x_{2}=\frac{5p}{4}$.

由抛物线的定义得$|AB|=x_{1}+x_{2}+p=\frac{5p}{4}+p=9$,所以$p=4$,从而抛物线方程是$y^{2}=8x$.

(2)由于$p=4$,所以$4x^{2}-5px+p^{2}=0$即为$x^{2}-5x+4=0$,从而$x_{1}=1$,$x_{2}=4$,于是$y_{1}=-2\sqrt{2}$,$y_{2}=4\sqrt{2}$,从而$A(1,-2\sqrt{2})$,$B(4,4\sqrt{2})$.

设$C(x_{3},y_{3})$,则$\vec{OC}=(x_{3},y_{3})=(1,-2\sqrt{2})+\lambdaup(4,4\sqrt{2})=(4\lambdaup+1,4\sqrt{2}\lambdaup-2\sqrt{2})$,又${y_3}^2=8x_{3}$,所以$[2\sqrt{2}(2\lambdaup-1)]^{2}=8(4\lambdaup+1)$,即$(2\lambdaup-1)^{2}=4\lambdaup+1$,解得$\lambdaup=0$或$\lambdaup=2$.

\noindent 

知识:变化率

难度:1

题目:设函数$y=f(x)$,当自变量由$x_{0}$变到$x_{0}+\Delta x$时,函数值的改变量$\Delta y$为(  )

A.$f(x_{0}+\Delta x)$   

B.$f(x_{0})+\Delta x$

C.$f(x_{0})\Delta x$   

D.$f(x_{0}+\Delta x)-f(x_{0})$

解析:函数值的改变量为$f(x_{0}+\Delta x)-f(x_{0})$,所以$\Delta y=f(x_{0}+\Delta x)-f(x_{0})$.

答案:D



知识:变化率

难度:1

题目:如果函数$y=ax+b$在区间$[1,2]$上的平均变化率为3,则a=(  )



A.-3  

B.2  

C.3  

D.-2

解析:根据平均变化率的定义,可知$\frac{\Delta y}{\Delta x}=\frac{(2a+b)-(a+b)}{2-1}=a=3$.

答案:C



知识:变化率

难度:1

题目:一直线运动的物体,从时间$t$到$t+\Delta t$时,物体的位移为$\Delta s$,则$\lim\limits_{\Delta x \rightarrow}\frac{\Delta s}{\Delta t}$ 为(  )

A. 从时间$t$到$t+\Delta t$一段时间内物体的平均速度

B.在$t$时刻时该物体的瞬时速度

C.当时间为$\Delta t$时物体的速度

D.在时间$t+\Delta t$时刻物体的瞬时速度

解析:由瞬时速度的求法可知,$\lim\limits_{\Delta x \rightarrow}\frac{\Delta s}{\Delta t}$表示在t时刻时该物体的瞬时速度.

答案:B



知识:导数的概念

难度:1

题目:函数$f(x)$在$x_{0}$处可导,则$\lim\limits_{\Delta x \rightarrow 0}\frac{f(x_0+h)-f(x_0)}{h}$(  )

A.与$x_{0}$、$h$都有关

B.仅与$x_{0}$有关,而与$h$无关

C.仅与$h$有关,而与$x_{0}$无关

D.与$x_{0}$、$h$均无关

解析:因为$f^{'}(x_{0})= \lim\limits_{\Delta x \rightarrow 0}\frac{f(x_0+h)-f(x_0)}{h}$,

所以 $f^{'}(x_{0})$仅与$x_{0}$有关,与$h$无关.

答案:B



知识:导数的概念

难度:1

题目:已知$f(x)=x^{2}-3x$,则$f^{'}(0)$=(  )

A.$\Delta x-3$

B.$(\Delta x)^{2}-3\Delta x$

C.-3   

D.0

解析:$f^{'}(0)=\lim\limits_{\Delta x \rightarrow 0}\frac{f(x_0+0)-f(0)}{\Delta x}
=\lim\limits_{\Delta x \rightarrow 0}\frac{(\Delta x)^{2}-3\Delta x}{\Delta x}
=\lim\limits_{\Delta x \rightarrow 0}(\Delta x-3)=-3$ .

答案:C



知识:变化率

难度:1

题目:如图,函数$f(x)$在A,B两点间的平均变化率是\_\_\_\_\_\_\_\_.

\includegraphics*[width=1.42in, height=1.13in, keepaspectratio=false]{image17}

解析:函数f(x)在A,B两点间的平均变化率是$\frac{\Delta y}{\Delta x}=\frac{f(3)-f(1)}{3-1}=\frac{1-3}{3-1}=-1$.

答案:-1



知识:导数的概念

难度:1

题目:设函数$y=x^{2}+2x$在点$x_{0}$处的导数等于3,则$x_{0}$=\_\_\_\_\_\_.

解析:$f^{'}(x)=\lim\limits_{\Delta x \rightarrow 0}\frac{{(x_0+\Delta x)}^2+2(x_0+\Delta x)-x_0^2-2x_0}{\Delta x} =2x_{0}+2$,又$2x_{0}+2=3$,所以$x_{0}=\frac{1}{2}$.

答案:$\frac{1}{2}$



知识:导数的概念

难度:1

题目:若函数$y=f(x)$在$x=x_{0}$处的导数为-2,则

$\lim\frac{f(x_0-\frac{1}{2}k)-f(x_0)}{k}$ =\_\_\_\_\_\_\_\_.

解析:
\begin{align}
\notag
\lim\limits_{k \rightarrow 0}\frac{f(x_0-\frac{1}{2}k)-f(x_0)}{k}
&=-\frac{1}{2}\lim\limits_{k \rightarrow 0}\frac{f(x_0-\frac{1}{2}k)-f(x_0)}{-\frac{1}{2}k}\\\notag
&=-\frac{1}{2}f'(x_0)\\\notag
&=-\frac{1}{2}\times(-2)\\\notag
&=1
\end{align}.

答案:1



知识:变化率

难度:1

题目:如图是函数y=f(x)的图象.

\includegraphics*[width=1.19in, height=1.10in, keepaspectratio=false]{image21}

(1)求函数f(x)在区间[-1,1]上的平均变化率;

(2)求函数f(x)在区间[0,2]上的平均变化率.

解析:

解:(1)函数f(x)在区间[-1,1]上的平均变化率为$$\frac{f(1)-f(-1)}{1-(-1)}=\frac{2-1}{2}=\frac{1}{2}$$.

(2)由函数f(x)的图象知,$
f(x)=
\begin{cases}
\frac{x+3}{2},-1\le x\le 1,\\
x+1,1<x\le3,
\end{cases}$
所以函数$f(x)$在区间[0,2]上的平均变化率为$\frac{f(2)-f(0)}{2-0}=\frac{3-\frac{3}{2}}{2}=\frac{3}{4}$.



知识:导数的概念

难度:1

题目:求函数$y=f(x)=2x^{2}+4x$在$x=3$处的导数.

解析:

解:$\Delta y=2(3+\Delta  x)^{2}+4(3+\Delta x)-(2{\times}3^{2}+4{\times}3)=
12\Delta x+2(\Delta x)^{2}+4\Delta x=2(\Delta x)^{2}+16\Delta x$,

所以 $\frac{\Delta y}{\Delta x}=\frac{2{\Delta x}^2+16\Delta x}{\Delta x}=2\Delta x+16$.

所以 $y'|_{x=3}=\lim\limits_{\Delta x\rightarrow 0}\frac{\Delta y}{\Delta x} =\lim\limits_{\Delta x\rightarrow 0} (2\Delta x+16)=16$.



知识:导数的概念

难度:2

题目:设函数$f(x)$在点$x_{0}$附近有定义,且有$f(x_{0}+\Delta x)-f(x_{0})=a\Delta x+b(\Delta x)^{2}$(a,b为常数),则(  )

A.$f^{'}(x)=a$   

B.$f^{'}(x)=b$

C.$f^{'}(x_{0})=a $

D.$f^{'}(x_{0})=b$

解析:$\frac{\Delta y}{\Delta x}=\frac{f(x_0+\Delta x)-f(x_0)}{\Delta x}=a+b\Delta x$.

所以 $f^{'}(x_{0})=\lim\limits_{\Delta x \rightarrow x} (a+b\Delta x)=a$.

答案:C



知识:导数的概念

难度:2

题目:将半径为R的球加热,若半径从$R=1$到$R=m$时球的体积膨胀率为$\frac{28\pi}{3}$,则m的值为\_\_\_\_\_\_\_\_.

解析:$\Delta V=\frac{4\pi}{3}m^{3}-\frac{4\pi}{3}{\times}1^{3}=\frac{4\pi}{3}(m^{3}-1)$,

所以 $\frac{\Delta V}{\Delta R}=\frac{\frac{4\pi}{3}(m^2-1)}{m-a}=\frac{28}{3}\piup$.

所以 $m^{2}+m+1=7$.

所以 $m=2$或$m=-3$(舍去).

答案:2



知识:变化率

难度:2

题目:若一物体的运动方程为$
s=
\begin{cases}
29+3{(t-3)}^2,0\le t<3,\\
3t^2+2,t\ge3
\end{cases}$(路程单位:m,时间单位:s).求:

(1)物体在$t=3s$到$t=5s$这段时间内的平均速度;

(2)物体在$t=1s$时的瞬时速度.

解:(1)因为$\Delta s=3\times5^{2}+2-(3\times3^{2}+2)=48(m)$,$\Delta t=2 s$,所以物体在$t=3 s$到$t=5 s$这段时间内的平均速度为$\frac{\Delta s}{\Delta t}=\frac{48}{2}=24(m/s)$.


(2)因为从1 s到$(1+\Delta t)s$的位移为$\Delta s=29+3[(1+\Delta t)-3]^{2}-29-3{\times}(1-3)^{2}=[3(\Delta t)^{2}-12\Delta t](m)$,所以平均速度为$\frac{\Delta s}{\Delta t}=\frac{3{(\Delta)}^2-12\Delta t}{\Delta t}=(3\Delta t-12)(m/s)$,则物体在$t=1 s$时的瞬时速度为$\lim\limits_{\Delta t\rightarrow 0}\frac{\Delta s}{\Delta t}=\lim\limits_{\Delta t\rightarrow 0} (3\Delta t-12)=-12(m/s)$.

\noindent 

\noindent 知识:导函数

难度:1

题目:下列说法正确的是(  )

A.曲线的切线和曲线有且只有一个公共点

B.过曲线上的一点作曲线的切线,这点一定是切点

C.若$f^{'}(x_{0})$不存在,则曲线$y=f(x)$在点$(x_{0},f(x_{0}))$处无切线

D.若$y=f(x)$在点$(x_{0},f(x))$处有切线,则$f^{'}(x_{0})$不一定存在

解析:曲线的切线和曲线除有一个公共切点外,还可能有其他的公共点,故A、B错误;$f^{'}(x_{0})$不存在,曲线$y=f(x)$在点$(x_{0},f(x))$的切线的斜率不存在,但切线可能存在,此时切线方程为$x=x_{0}$,故C错误,D正确.

答案:D



知识:导函数

难度:1

题目:曲线$y=f(x)$在点$(x_{0},f(x_{0}))$处的切线方程为$2x-y+1=0$,则(  )

A.$f^{'}(x_{0}){>}0$      

B.$f^{'}(x_{0}){<}0$

C.$f^{'}(x_{0})=0$  

D.$f^{'}(x_{0})$不存在

解析:因为函数$y=f(x)$在$x=x_{0}$处的导数就是曲线$y=f(x)$在$x=x_{0}$处的切线的斜率,又切线$2x-y+1=0$的斜率为2,所以$f^{'}(x_{0})=2{>}0$.

答案:A



知识:导函数

难度:1

题目:若曲线$f(x)=ax^{2}$在点$(1,a)$处的切线与直线$2x-y-6=0$平行,则a等于(  )

A.1  

B.  

C.-  

D.-1

解析:因为
\begin{align}
\notag
f^{'}(1)
&=\lim\limits_{x\rightarrow 0}\frac{a{(1+\Delta x)}^2-a\times 1^2}{\Delta x}\\\notag
&=\lim\limits_{x \rightarrow 0}\frac{2a\Delta x + a{(\Delta x)}^2}{\Delta x}\\\notag
&=\lim\limits_{x \rightarrow 0}(2a+a\Delta x)=2a,
\end{align}


所以 $2a=2$,所以 $a=1$.

答案:A



知识:导函数

难度:1

题目:$y=-\frac{1}{x}$在点$(\frac{1}{2},-2)$处的切线方程是(  )

A.$y=x-2 $

B.$y=x-\frac{1}{2}$

C.$y=4x-4$   

D.$y=4x-2$

解析:先求$y=-\frac{1}{x}$的导数:$\Delta y=-\frac{1}{x+\Delta x}+\frac{1}{x}=\frac{\Delta x}{x(x+\Delta x)}$,$\frac{\Delta y}{\Delta x}=\frac{1}{x(x+\Delta x)}$,$\frac{\Delta y}{\Delta x}=\frac{1}{x^2}$ ,即$y^{'}=\frac{1}{x^2}$,所以$y=-\frac{1}{x}$在点$(\frac{1}{2},-2)$处的切线斜率为$k=y^{'}|x=\frac{1}{2}=4$.所以切线方程是$y+2=4(x-\frac{1}{2})$,

即$y=4x-4$.

答案:C



知识:导函数

难度:1

题目:曲线$y=f(x)=x^{3}$在点P处切线的斜率为k,当$k=3$时点P的坐标为(  )

A.(-2,-8)   

B.(-1,-1)或(1,1)

C.(2,8)   

D. $(-\frac{1}{2},-\frac{1}{8})$

解析:设点P的坐标为$(x_{0},y_{0})$,

则
\begin{align}
\notag
k=f^{'}(x_{0})
&=\lim\limits_{x \rightarrow 0}\frac{f(x_0+\Delta x)-f(x_0)}{\Delta x}\\\notag
&=\lim\limits_{x \rightarrow 0}\frac{{(x_0+\Delta)}^2-x_0^2}{\Delta x}\\\notag
&=\lim\limits_{x \rightarrow 0}[(\Delta x)^{2}+3x_0^2+3x_{0}·\Delta x]\\\notag
&=3x_0^2
\end{align}

因为$k=3$,所以 $3x_0^2=3$,所以 $x_{0}=1$或$x_{0}=-1$,

所以 $y_{0}=1$或$y_{0}=-1$.

所以 点P的坐标为(-1,-1)或(1,1).

答案:B



知识:导函数

难度:1

题目:已知函数$y=f(x)$在点(2,1)处的切线与直线$3x-y-2=0$平行,则$y'|_{x=2}$等于\_\_\_\_\_\_\_\_.

解析:因为直线$3x-y-2=0$的斜率为3,所以由导数的几何意义可知$y'|_{x=2}=3$.

答案:3



知识:导函数

难度:1

题目:曲线$f(x)=\frac{1}{2}x^{2}$的平行于直线$x-y+1=0$的切线方程为\_\_\_\_\_\_\_\_.

解析:$f^{'}(x)=\lim\limits_{x \rightarrow 0}\frac{\frac{1}{2}{(x+\Delta x)}^2-\frac{1}{2}x^2}{\Delta x}=x$.因为直线$x-y+1=0$的斜率为1,所以$x=1$,所以$f(1)=\times 1=\frac{1}{2}$,切点为$(1,\frac{1}{2})$.故切线方程为$y-\frac{1}{2}=1·(x-1)$,即$x-y-\frac{1}{2}=0$.

答案:$x-y-\frac{1}{2}=0$



知识:导函数

难度:1

题目:已知函数$y=f(x)$的图象在点$M(1,f(1))$处的切线方程是$y=\frac{1}{2}x+2$,则$f(1)+f^{'}(1)$=\_\_\_\_\_\_\_\_.

解析:由导数的几何意义,得$f^{'}(1)=\frac{1}{2}$,又切点在切线上,故$f(1)=\frac{1}{2}{\times}1+2=\frac{5}{2}$,所以$f(1)+f^{'}(1)=3$.

答案:3



知识:导函数

难度:1

题目:在抛物线$y=x^{2}$上哪一点处的切线平行于直线$4x-y+1=0$?哪一点处的切线垂直于这条直线?

解析:

解:$y^{'}=\lim\limits_{x \rightarrow 0}\frac{{(x+\Delta x)}^2-x^2}{\Delta x}=\lim\limits_{x \rightarrow 0}(2x+\Delta x)=2x$.

设抛物线上点$P(x_{0},y_{0})$处的切线平行于直线$4x-y+1=0$,



则$y'|_{x=x_0}=2x_0=4$,解得$x_{0}=2$.

所以$y_{0}=x_0^2=4$,即$P(2,4)$.

设抛物线上点$Q(x_{1},y_{1})$处的切线垂直于直线$4x-y+1=0$,

则$y'|_{x=x_1}=2x_1=-\frac{1}{4}$,解得$x_{1}=-\frac{1}{8}$.

所以$y_{1}=x_1^2=\frac{1}{64}$,即$Q(-\frac{1}{8},\frac{1}{64})$.

故抛物线$y=x^{2}$在点(2,4)处的切线平行于直线$4x-y+1=0$,在点$(-\frac{1}{8},\frac{1}{64})$处的切线垂直于直线$4x-y+1=0$.



知识:导函数

难度:1

题目:已知曲线$y=\frac{1}{t-x}$上两点$P(2,-1)$,$Q(-1,\frac{1}{2})$.

(1)求曲线在点P,Q处的切线的斜率;

(2)求曲线在点P,Q处的切线方程.

解析:

解:将(2,-1)代入$y=\frac{1}{t-x}$,得$t=1$,

所以$y=\frac{1}{1-x}$.

\begin{align}
\notag
y'&=\lim\limits_{\Delta x \rightarrow 0}\frac{\Delta y}{\Delta x}\\\notag
&=\lim\limits_{\Delta x \rightarrow 0}\frac{\frac{1}{1-(x+\Delta x)}-\frac{1}{1-x}}{\Delta x}\\\notag
&=\lim\limits_{\Delta x \rightarrow 0}\frac{\Delta x}{[1-(x+\Delta x)](1-x)\Delta x}\\\notag
&=\lim\limits_{\Delta x \rightarrow 0}\frac{1}{(1-x-\Delta x)(1-x)}\\\notag
&=\frac{1}{{(1-x)}^2}
\end{align}

(1)曲线在点P处的切线斜率为$y'|_{x=2}=\frac{1}{{(1-2)}^2}=1$;曲线在点Q处的切线斜率为$y'|_{x=-1}=\frac{1}{4}$.

(2)曲线在点P处的切线方程为$y-(-1)=x-2$,即$x-y-3=0$,曲线在点Q处的切线方程为$y-\frac{1}{2}=\frac{1}{4}[x-(-1)]$,即$x-4y+3=0$.



知识:导函数

难度:2

题目:已知直线$y=kx+1$与曲线$y=x^{3}+ax+b$相切于点(1,3),则b的值为(  )

A.3  

B.-3  

C.5  

D.-5

解析:点(1,3)既在直线上,又在曲线上.由于
$y'=\lim\limits_{\Delta x\rightarrow 0}\frac{{(x+\Delta x)}^2+a(x+\Delta x)+b-(x^2+ax+b)}{\Delta x}=3x^2+a$
,所以$y'|_{x=1}=3+a=k$,将(1,3)代入$y=kx+1$,得$k=2$,所以$a=-1$,又点(1,3)在曲线$y=x^{3}+ax+b$上,故$1+a+b=3$,又由$a=-1$,可得$b=3$.

答案:A



知识:导函数

难度:2

题目:曲线$f(x)=\frac{9}{x}$在点(3,3)处的切线的倾斜角等于\_\_\_\_\_\_\_\_.

解析:
\begin{align}
\notag
f'(x)&=\lim\limits_{\Delta x\rightarrow 0}\frac{f(x+\Delta x)-f(x)}{\Delta x}\\\notag
&=9\lim\limits_{\Delta x\rightarrow 0}\frac{\frac{1}{x+\Delta x}-\frac{1}{x}}{\Delta x}\\\notag
&=-9\lim\limits_{\Delta x\rightarrow 0}\frac{1}{(x+\Delta x)x}=-\frac{9}{x^2}
\end{align}
,所以 $f'(3)=-\frac{9}{9}=-1$,又因为直线的倾斜角范围是$[0{{}^\circ},180{{}^\circ})$,所以 倾斜角为$135{{}^\circ}$.

答案:135${{}^\circ}$



知识:导函数

难度:2

题目:设函数$f(x)=x^{3}+ax^{2}-9x-1(a<0)$,若曲线$y=f(x)$的斜率最小的切线与直线$12x+y=6$平行,求a的值.

解析:

解:因为
\begin{align}
\notag
\Delta y&=f(x_{0}+\Delta x)-f(x_{0})\\\notag
&=(x_{0}+\Delta x)^{3}+a(x_{0}+\Delta x)^{2}-9(x_{0}+\Delta x)-1-(x+ax-9x_{0}-1)\\\notag
&=(3x+2ax_{0}-9)\Delta x+(3x_{0}+a)(\Delta x)^{2}+(\Delta x)^{2}+(\Delta x)^{3}
\end{align}
,

所以$\frac{\Delta y}{\Delta x}=3x_0^2+2ax_{0}-9+(3x_{0}+a)\Delta x+(\Delta x)^{2}$.

当$\Delta x$无限趋近于0时,

无限趋近于$3x_0^2+2ax_{0}-9$,

即$f'(x_{0})=3x_0^2+2ax_{0}-9$,

所以$f'(x_{0})=3(x_{0}+\frac{a}{3})^{2}-9-\frac{a^2}{3}$.

当$x_{0}=-\frac{a}{3}$时,

$f'(x_{0})$取最小值$-9-\frac{a^2}{3}$.

因为斜率最小的切线与直线$12x+y=6$平行,

所以该切线斜率为-12.

所以$-9-\frac{a^2}{3}=-12$.

解得$a=\pm 3$.又a<0,

所以$a=-3$.



知识:常用函数的导数

难度:1

题目:某物体运动方程为$y=4.9t^{2}$(其中y的单位为米,t的单位为秒),则该物体在1秒末的瞬时速度为(  )

A.4.9米/秒    

B.9.8米/秒

C.49米/秒   

D.2.45米/秒

解析:由题意知$y'=9.8t$,则$y'|_{t=1}=9.8$,故选B.

答案:B



知识:常用函数的导数

难度:1

题目:$f(x)=x^{3}$,$f'(x_{0})=6$,则$x_{0}$等于(  )

A.$\sqrt{2}$  B.-$\sqrt{2}$  C.$\pm\sqrt{2}$  D.$\pm1$

解析:$f'(x)=3x^{2}$,由$f'(x_{0})=6$,知$3x_0^2=6$,所以 $x_{0}=\pm\sqrt{2}$.

答案:C

知识:常用函数的导数

难度:1

题目:若指数函数$f(x)=a^{x}(a>0,a\neq1)$满足$f'(1)=\ln 27$,则$f'(-1)$=(  )

A.2   

B.$\ln 3$

C. $\frac{\ln 3}{3}$  

D.$-\ln 3$

解析:$f'(x)=a^{x}\ln a$,则$f'(1)=a\ln a=\ln 27$,

解得$a=3$,所以$f'(x)=3^{x}\ln 3$.

故$f'(-1)=3^{-1}\ln 3=\frac{\ln 3}{3}$.

答案:C



知识:常用函数的导数

难度:1

题目:曲线$y=e^{x}$在点$(2,e^{2})$处的切线与坐标轴所围成的三角形的面积为(  )

A.$\frac{9}{4}e^{2}$  

B.$2e^{2}$  

C.$e^{2}$  

D.$\frac{e^2}{2}$

解析:因为$y=e^{x}$,所以 $y'=e^{x}$,所以 $y'|_{x=2}=e^{2}=k$,所以 切线方程为$y-e^{2}=e^{2}(x-2)$,即$y=e^{2}x-e^{2}$.在切线方程中,令$x=0$,得$y=e^2x-e^{2}$,令$y=0$,得$x=1$,所以$ S_{\text{三角形}}=\frac{1}{2}\times |-e^{2}|\times 1=\frac{e^2}{2}$.

答案:D



知识:常用函数的导数

难度:1

题目:若$f_{0}(x)=\sin x,f_{1}(x)=f'_{0}(x)$,$f_{2}(x)=f'_{1}(x),{\dots},f_{n+1}(x)=f'_{n}(x)$,$n\in N,则f_{2\ 013}(x)$=(  )

A.$\sin x   $

B.$-\sin x$

C.$\cos x   $

D.$-\cos x$

解析:因为$f_{1}(x)=(\sin x)'=\cos x,f_{2}(x)=(\cos x)'=-\sin x,f_{3}(x)=(-\sin x)'=-\cos x,f_{4}(x)=(-\cos x)'=\sin x,f_{5}(x)=(\sin x)'=\cos x$,所以循环周期为4,因此$f_{2\ 013}(x)=f_{1}(x)=\cos x$.

答案:C



知识:常用函数的导数

难度:1

题目:已知点P在曲线$f(x)=x^{4}-x$上,曲线在点P处的切线平行于直线$3x-y=0$,则点P的坐标为\_\_\_\_\_\_\_\_.

解析:设点P的坐标为$(x_{0},y_{0})$,

因为$f'(x)=4x^{3}-1$,所以 $4x_0^3-1=3$,所以 $x_{0}=1$.

所以 $y_{0}=1^{4}-1=0$,所以 即得P(1,0).

答案:(1,0)

知识:常用函数的导数

难度:1

题目:已知$f(x)=\frac{1}{3}x^{3}+3xf'(0)$,则$f'(1)=$\_\_\_\_\_\_\_\_.

解析:由于$f'(0)$是一常数,所以$f'(x)=x^{2}+3f'(0)$,令$x=0$,则$f'(0)=0$,所以 $f'(1)=1^{2}+3f'(0)=1$.

答案:1



知识:常用函数的导数

难度:1

题目:曲线$y=\frac{x}{x-2}$在点(1,-1)处的切线方程为\_\_\_\_\_\_\_\_.

解析:因为$y'=\frac{x-2-x}{{(x-2)}^2}=\frac{-2}{(x-2)}$,所以曲线在点(1,-1)处的切线的斜率$k=\frac{-2}{{(1-2)}^2}=-2$,故所求切线方程为$y+1=-2(x-1)$,即$2x+y-1=0$.

答案:$2x+y-1=0$



知识:常用函数的导数

难度:1

题目:求下列函数的导数:

(1)$y=(2x^{2}+3)(3x-1)$;

(2)$y=(\sqrt{x}-2)^{2}$;

(3)$y=x-\sin\frac{x}{2}\cdot \cos\frac{x}{2}$ .

解析:

解:(1)法一:$y'=(2x^{2}+3)'(3x-1)+(2x^{2}+3)(3x-1)'=4x(3x-1)+3(2x^{2}+3)=18x^{2}-4x+9$.

法二:因为$y=(2x^{2}+3)(3x-1)=6x^{3}-2x^{2}+9x-3$,

所以 $y'=(6x^{3}-2x^{2}+9x-3)'=18x^{2}-4x+9$.

(2)因为$y=(\sqrt{2}-2)^{2}=x-4\sqrt{4}+4$,

所以 $y'=x'-(4\sqrt{x})'+4'=1-4\times x-\frac{1}{2}=1-2x-\frac{1}{2}$.

(3)因为$y=x-\sin\frac{x}{2}\cos\frac{x}{2} =x-\frac{1}{2}\sin x$,

所以 $y'=x'-(\frac{1}{2}\sin x)'=1-\frac{1}{2}\cos x$.


知识:常用函数的导数

难度:1

题目:设函数$f(x)=\frac{1}{3}x^{3}-\frac{a}{2}x^{2}+bx+c$,其中$a>0$,曲线$y=f(x)$在点$P(0,f(0))$处的切线方程为$y=1$,确定b,c的值.

解:由题意得$f(0)=c$,$f'(x)=x^{2}-ax+b$,

由切点P(0,f(0))既在曲线$f(x)=\frac{1}{3}x^{3}-\frac{a}{2}x^{2}+bx+c$上又在切线$y=1$上知
$
\begin{cases}
f'(0)=0,\\
f(0)=1,
\end{cases}$
即
$
\begin{cases}
0^2-a\times 0+b=0,\\
\frac{1}{3}\times 0^2-\frac{a}{2}\times 0^2+b\times 0+c=1,
\end{cases}$
故$b=0$,$c=1$.



知识:常用函数的导数

难度:2

题目:已知点P在曲线$y=\frac{4}{e^x+1}$上,$\alpha$为曲线在点P处的切线的倾斜角,则$\alpha$的取值范围是(  )

A.$[0,\frac{\pi}{4}) $  

B.$[\frac{\pi}{2},\frac{\pi}{4})$

C.$(\frac{\pi}{2},\frac{3\pi}{4}]$   

D.$[\frac{3\pi}{4},\piup)$

解析:$y'=-\frac{4e^x}{{(e^x+1)}^2}=-\frac{4e^x}{e^{2x}+2e^x+1}$,

设$t=e^{x}\in (0,+\infty)$,则
$y'=-\frac{4t}{t^2+2t+1}=-\frac{4}{t+\frac{1}{t}+2}$,

因为$t+\frac{1}{t} \ge2$,所以 $y'\in [-1,0)$,$\alpha \in[\frac{3\pi}{4},\pi)$.

答案:D



知识:常用函数的导数

难度:2

题目:点P是曲线$y=ex$上任意一点,则点P到直线$y=x$的最小距离为\_\_\_\_\_\_\_\_.

解析:根据题意设平行于直线$y=x$的直线与曲线$y=e^{x}$相切于点$(x_{0},y_{0})$,该切点即为与y=x距离最近的点,如图,则在点$(x_{0},y_{0})$处的切线斜率为1,即$y'|_{x=x_0}=1$.

\includegraphics*[width=1.02in, height=1.03in, keepaspectratio=false]{image49}

因为$y'=(e^{x})'=e^{x}$,所以 $ex_{0}=1$,

得$x_{0}=0$,代入$y=e^{x}$,得$y_{0}=1$,即P(0,1).

利用点到直线的距离公式得距离为$\frac{\sqrt{2}}{2}$.

答案:



知识:常用函数的导数

难度:2

题目:设函数$f(x)=ax-\frac{b}{x}$,曲线$y=f(x)$在点$(2,f(2))$处的切线方程为$7x-4y-12=0$.

(1)求$f(x)$的解析式;

(2)证明:曲线$y=f(x)$上任意一点处的切线与直线$x=0$和直线$y=x$所围成的三角形的面积为定值,并求此定值.

解析:

(1)解:$f'(x)=a+\frac{b}{x^2}$.

因为点$(2,f(2))$在切线$7x-4y-12=0$上,

所以 $f(2)=\frac{2\times7-12}{4}=\frac{1}{2}$.

又曲线$y=f(x)$在点$(2,f(2))$处的切线方程为$7x-4y-12=0$,

所以 $
\begin{cases}
f'(2)=\frac{7}{4},\\
f(2)=\frac{1}{2},
\end{cases}
\Rightarrow 
\begin{cases}
a+\frac{b}{4}=\frac{7}{4},\\
2a-\frac{b}{2}=\frac{1}{2},
\end{cases}
\Rightarrow 
\begin{cases}
a=1,\\
b=3,
\end{cases}
$

所以 $f(x)$的解析式为$f(x)=x-\frac{3}{x}$.

(2)证明:设$(x_0,x_0\frac{3}{x_0})$为曲线$y=f(x)$上任意一点,则切线斜率$k=1+\frac{3}{x_0^2}$,切线方程为$y-(x_0-\frac{3}{x_0})=(1+\frac{3}{x_0^2})(x-x_{0})$,令$x=0$,得$y=-\frac{6}{x_0}$.

由
$
\begin{cases}
y-(x_0-\frac{3}{x_0^2})=(1+\frac{3}{x_0^2})(x-x_0),\\
y=x,
\end{cases}$
得
$
\begin{cases}
x=2x_0,\\
y=2x_0.
\end{cases}$
所以 曲线$y=f(x)$上任意一点处的切线与直线$x=0$和直线$y=x$所围成的三角形的面积$S=\frac{1}{2}|2x_{0}||-\frac{6}{x_0}|=6$,为定值.

\noindent 

知识:函数的单调性与导数

难度:1

题目:函数$y=\frac{1}{2}x^{2}-\ln x$的单调减区间是(  )

A.$(0,1) $  

B.$(0,1)\cup (-\infty,-1)$

C.$(-\infty,1)   $

D.$(-\infty,+\infty)$

解析:因为$y=\frac{1}{2}x^{2}-\ln x$的定义域为 $(0,+\infty)$,

所以 $y'=x-\frac{1}{x}$,令$y'<0$,即$x-\frac{1}{x}<0$,

解得:$0<x<1$或$x<-1$.

又因为$x>0$,所以 $0<x<1$.

答案:A



知识:函数的单调性与导数

难度:1

题目:下列函数中,在(0,$+\infty$)内为增函数的是(  )

A.$y=\sin x   $

B.$y=xe^{2}$

C.$y=x^{3}-x  $ 

D.$y=\ln x-x$

解析:显然$y=sin x$在(0,$+\infty$)上既有增又有减,故排除A;对于函数$y=xe^{2}$,因$e^{2}$为大于零的常数,不用求导就知$y=xe^{2}$在(0,$+\infty$)内为增函数;

对于C,$y'=3x^{2}-1=3(x+\frac{\sqrt{3}}{3})(x-\frac{\sqrt{3}}{3})$,

故函数在$(-\infty,\frac{\sqrt{3}}{3})$和$(\frac{\sqrt{3}}{3},+\infty)$上为增函数,

在$(-\frac{\sqrt{3}}{3},\frac{\sqrt{3}}{3})$上为减函数;对于D,$y'=\frac{1}{x}-1(x>0)$.

故函数在(1,$+\infty$)上为减函数,在(0,1)上为增函数.

答案:B

知识:函数的单调性与导数

难度:1

题目:函数$f(x)=x^{3}+ax^{2}+bx+c$,其中a,b,c为实数,当$a^{2}-3b<0$时,f(x)是(  )

A.增函数

B.减函数

C.常数

D.既不是增函数也不是减函数

解析:求函数的导函数$f'(x)=3x^{2}+2ax+b$,导函数对应方程$f'(x)=0$的$\Delta=4(a^{2}-3b)<0$,所以$f'(x)>0$恒成立,故f(x)是增函数.

答案:A



知识:函数的单调性与导数

难度:1

题目:设函数f(x)在定义域内可导,y=f(x)的图象如图所示,则y=f'(x)的图象可能为(  )

\includegraphics*[width=0.95in, height=0.85in, keepaspectratio=false]{image50}

\includegraphics*[width=2.21in, height=0.77in, keepaspectratio=false]{image51}

A       B

\includegraphics*[width=2.21in, height=0.71in, keepaspectratio=false]{image52}

C      D

解析:由题图找出函数f(x)的增(减)区间,则其导函数f'(x)在相应区间上的函数值为正(负),即导函数在相应区间上的图象在x轴的上(下)方,易知D正确.

答案:D

知识:函数的单调性与导数

难度:1

题目:若函数$f(x)=kx-\ln x$在区间(1,$+\infty$)上单调递增,则k的取值范围是(  )

A.(-$\infty$,-2]   

B.(-$\infty$,-1]

C.[2,$+\infty$)   

D.[1,$+\infty$)

解析:依题意得$f'(x)=k-\frac{1}{x}\ge0$在(1,$+\infty$)上恒成立,即$k\ge\frac{1}{x}$在(1,$+\infty$)上恒成立,因为$x>1$,所以$0<\frac{1}{x}<1$,

所以$k\ge1$,故选D.

答案:D



知识:函数的单调性与导数

难度:1

题目:函数$f(x)=x-2\sin x$在(0,$\piup$)上的单调递增区间为\_\_\_\_\_\_\_\_.

解析:令$f'(x)=1-2\cos x>0$,得$\cos x<\frac{1}{2}$,又$x\in(0,\piup)$,所以 $\frac{\pi}{3}<x<\pi$
.

答案:$(\frac{\pi}{3},\pi)$

\noindent 

知识:函数的单调性与导数

难度:1

题目:已知函数$f(x)=\sqrt{x}+\ln x$,则f(2),f(3),f(e)按从小到大排列应为\_\_\_\_\_\_\_\_ .

解析:因为在定义域(0,$+\infty$)上$f'(x)=\frac{1}{2\sqrt{2}}+\frac{1}{x}>0$,

所以f(x)在(0,$+\infty$)上是增函数,所以有f(2)<f(e)<f(3).

答案:f(2)<f(e)<f(3)



知识:函数的单调性与导数

难度:1

题目:函数$f(x)=x^{3}+x^{2}+mx+1$是R上的单调递增函数,则m的取值范围为\_\_\_\_\_\_\_\_.

解析:因为$f(x)=x^{3}+x^{2}+mx+1$,所以$f'(x)=3x^{2}+2x+m$,由题意可知$f'(x)\ge0$在R上恒成立,所以$\Delta=4-12m\le0$,即$m\ge\frac{1}{3}$.

答案:$[\frac{1}{3},=\infty)$



知识:函数的单调性与导数

难度:1

题目:证明:函数$f(x)=\frac{\ln x}{x}$在区间(0,2)内是增函数.

解析:

证明:$f'(x)=\frac{\frac{1}{x}\cdot \ln x}{x^2}=\frac{1-\ln x}{x^2}$.

因为0<x<2,所以$\ln x<\ln 2<1$,故$1-\ln x>0$.

所以$f'(x)=\frac{1-\ln x}{x^2}>0$.

根据导数与函数单调性的关系,

得函数f(x)=在区间(0,2)内是增函数.



知识:函数的单调性与导数

难度:1

题目:已知函数$f(x)=x^{3}+bx^{2}+cx+d$的图象经过点P(0,2),且在点M(-1,f(-1))处的切线方程为$6x-y+7=0$.

(1)求函数y=f(x)的解析式;

(2)求函数y=f(x)的单调区间.

解析:

解:(1)由y=f(x)的图象经过点P(0,2),知d=2,

所以 $f(x)=x^{3}+bx^{2}+cx+2$,$f'(x)=3x^{2}+2bx+c$.

由在点M(-1,f(-1))处的切线方程为$6x-y+7=0$,

如$-6-f(-1)+7=0$,即f(-1)=1,f'(-1)=6.

所以 即

解得b=c=-3.

故所求的解析式是$f(x)=x^{3}-3x^{2}-3x+2$.

(2)$f'(x)=3x^{2}-6x-3$,令f'(x)>0,得$x<1-\sqrt{2}$或$x>1+\sqrt{2}$;

令f'(x)<0,得$1-\sqrt{2}<x<1+\sqrt{2}$.

故$f(x)=x^{3}-3x^{2}-3x+2$的单调递增区间为$(-\infty,1-\sqrt{2})$,$(1+\sqrt{2},+\infty)$,单调递减区间为$(1-\sqrt{2},1+\sqrt{2})$.



知识:函数的单调性与导数

难度:2

题目:设f(x),g(x)在[a,b]上可导,且f'(x)>g'(x),则当a<x<b时,有(  )

A.f(x)>g(x)

B.f(x)<g(x)

C.f(x)+g(a)>g(x)+f(a)

D.f(x)+g(b)>g(x)+f(b)

解析:因为f'(x)-g'(x)>0,所以 '>0,所以 f(x)-g(x)在[a,b]上是增函数,

所以 当a<x<b时f(x)-g(x)>f(a)-g(a),

所以 f(x)+g(a)>g(x)+f(a).

答案:C



知识:函数的单调性与导数

难度:2

题目:若函数$f(x)=x^{3}+bx^{2}+cx+d$的单调递减区间为(-1,2),则b=\_\_\_\_\_\_\_\_,c=\_\_\_\_\_\_\_\_.

解析:$f'(x)=3x^{2}+2bx+c$,由题意知-1<x<2是不等式f'(x)<0的解,即-1,2是方程$3x^{2}+2bx+c=0$的两个根,把-1,2分别代入方程,联立解得$b=-\frac{3}{2}$,$c=-6$.

答案:$-\frac{3}{2}$ \ -6



知识:函数的单调性与导数

难度:2

题目:已知函数$f(x)=ax^{3}+3x^{2}-x+1$在R上是减函数,求a的取值范围.

解析:

解:$f'(x)=3ax^{2}+6x-1$.

由已知得$f'(x)\le0$在R上恒成立,即$3ax^{2}+6x-1\le0$在R上恒成立,当$a\ge0$时,不满足题意,所以a${<}$0且$\Delta=36+12a\le0\Leftrightarrow a\le-3$.

当a=-3时,$f(x)=-3x^{3}+3x^{2}-x+1=-3{(x-\frac{1}{3})}^3+\frac{8}{9}$,由函数$y=-x^{3}$在R上的单调性可知当a=-3时,f(x)在R上是减函数.

综上,a的取值范围是(-$\infty$,-3].

\noindent 

知识:函数的最值与导数

难度:1

题目:可导``函数y=f(x)在一点的导数值为0''是``函数y=f(x)在这点取得极值''的(  )

A.充分不必要条件

B.必要不充分条件

C.充要条件

D.既不充分也不必要条件

解析:对于$f(x)=x^{3}$,$f'(x)=3x^{2}$,$f'(0)=0$,不能推出f(x)在x=0处取极值,反之成立.

答案:B



知识:函数的最值与导数

难度:1

题目:已知可导函数f(x),x${\in}$R,且仅在x=1处,f(x)存在极小值,则(  )

A.当x${\in}$(-$\infty$,1)时,f'(x)>0;当x${\in}$(1,$+\infty$)时,f'(x)<0

B.当x${\in}$(-$\infty$,1)时,f'(x)>0;当x${\in}$(1,$+\infty$)时,f'(x)>0

C.当x${\in}$(-$\infty$,1)时,f'(x)<0;当x${\in}$(1,$+\infty$)时,f'(x)>0

D.当x${\in}$(-$\infty$,1)时,f'(x)<0;当x${\in}$(1,$+\infty$)时,f'(x)<0

解析:因为f(x)在x=1处存在极小值,

所以 x<1时,f'(x)<0,x>1时,f'(x)>0.

答案:C



知识:函数的最值与导数

难度:1

题目:函数$y=x^{3}-3x^{2}-9x(-2<x<2)$有(  )

A.极大值5,极小值-27

B.极大值5,极小值-11

C.极大值5,无极小值

D.极小值-27,无极大值

解析:由$y'=3x^{2}-6x-9=0$,得x=-1或x=3,

当x<-1或x>3时,y'>0;当-1<x<3时,y'<0.

故当x=-1时,函数有极大值5;x取不到3,故无极小值.

答案:C



知识:函数的最值与导数

难度:1

题目:已知$f(x)=x^{3}+ax^{2}+(a+6)x+1$有极大值和极小值,则a的取值范围为(  )

A.-1<a<2   

B.-3<a<6

C.a<-1或a>2   

D.a<-3或a>6

解析:$f'(x)=3x^{2}+2ax+(a+6)$,因为f(x)既有极大值又有极小值,那么$\Delta=(2a)^{2}-4\times3\times(a+6)>0$,解得a>6或a<-3.

答案:D



知识:函数的最值与导数

难度:1

题目:设a${\in}$R,若函数$y=e^{x}+ax$,x${\in}$R有大于零的极值点,则(  )

A.a<-1   

B.a>-1

C.a>-   

D.a<-

解析:$y'=e^{x}+a=0$,$e^{x}=-a$,

因为x>0,所以 $e^{x}$>1,即-a>1,所以 a<-1.

答案:A



知识:函数的最值与导数

难度:1

题目:函数$f(x)=x^{3}-6x+a$的极大值为\_\_\_\_\_\_\_\_,极小值为\_\_\_\_\_\_\_\_.

解析:$f'(x)=x^{2}-6$

令f'(x)=0,得$x=-\sqrt{2}$或$x=\sqrt{2}$,

所以$f(x)_{\textrm{极}\textrm{大}\textrm{值}}=f(-\sqrt{2})=a+4\sqrt{2}$,

$f(x)_{\textrm{极}\textrm{小}\textrm{值}}=f(\sqrt{2})=a-4\sqrt{2}$.

答案:$a+4\sqrt{2}$,$a-4\sqrt{2}$.



知识:函数的最值与导数

难度:1

题目:已知函数$y=x^{3}+ax^{2}+bx+27$在x=-1处取极大值,在x=3处取极小值,则a=\_\_\_\_\_\_\_\_,b=\_\_\_\_\_\_\_\_.

解析:$y'=3x^{2}+2ax+b$,根据题意知,-1和3是方程$3x^{2}+2ax+b=0$的两根,由根与系数的关系可求得a=-3,b=-9.经检验,符合题意.

答案:-3 -9



知识:函数的最值与导数

难度:1

题目:已知函数$f(x)=ax^{3}+bx^{2}+cx$,其导函数y=f'(x)的图象经过点(1,0),(2,0),如图所示.

\includegraphics*[width=0.92in, height=0.67in, keepaspectratio=false]{image53}

则下列说法中不正确的是\_\_\_\_\_\_\_\_.

①当$x=\frac{3}{2}$时,函数取得极小值;

②f(x)有两个极值点;

③当x=2时,函数取得极小值;

④当x=1时,函数取得极大值.

解析:由图象可知当x${\in}$(-$\infty$,1)时,$f'(x)>0$;当x${\in}$(1,2)时,f'(x)${<}$0;当x${\in}$(2,$+\infty$)时,f'(x)${>}$0,所以f(x)有两个极值点1和2,且当x=2时,函数取得极小值,当x=1时,函数取得极大值.故只有①不正确.

答案:①



知识:函数的最值与导数

难度:1

题目:已知$f(x)=\frac{1}{3}x^{3}-\frac{1}{2}x^{2}-2x$,求f(x)的极大值与极小值.

解析:

解:由已知得f(x)的定义域为R.

$f'(x)=x^{2}-x-2=(x+1)(x-2)$.

令f'(x)=0,得x=-1或x=2.

当x变化时,f'(x)与f(x)的变化情况如下表:



\begin{tabular}{|p{0.5in}|p{0.8in}|p{0.5in}|p{0.7in}|p{0.5in}|p{0.8in}|} \hline 
	x & (-$\infty$,-1) & -1 & (-1,2) & 2 & (2,$+\infty$) \\ \hline 
	f'(x) & + & 0 & - & 0 & + \\ \hline 
	f'(x) & ${\nearrow}$ & 极大值 & ${\searrow}$ & 极小值 & ${\nearrow}$ \\ \hline 
\end{tabular}

因此,当x=-1时,f(x)取得极大值,且极大值为$f(-1)=\times(-1)^{3}-\times(-1)^{2}-2\times(-1)=\frac{7}{6}$;

当x=2时,f(x)取得极小值,且极小值为$f(2)=\frac{1}{3}\times2^{3}-\frac{1}{2}\times2^{2}-2\times2=-\frac{10}{3}$.

从而f(x)的极大值为$\frac{10}{3}$,极小值为$-\frac{10}{3}$.



知识:函数的最值与导数

难度:1

题目:已知函数$f(x)=x^{3}+ax^{2}+bx+a^{2}$在x=1处取极值10,求f(2)的值.

解析:

解:$f'(x)=3x^{2}+2ax+b$.

由题意得
$
\begin{cases}
f(1)=10,\\
f'(1)=0,
\end{cases}$
即
$
\begin{cases}
a^2+a+b+1=10,\\
2a+b+3=0,
\end{cases}$
解得
$
\begin{cases}
a=4,\\
b=-11,
\end{cases}$或
$
\begin{cases}
a=-3,\\
b=3.
\end{cases}$
当a=4,b=-11时,令f'(x)=0,得$x_{1}=1$,$x_{2}=-\frac{11}{3}$.

当x变化时,f'(x),f(x)的变化情况如下表:



\begin{tabular}{|p{0.5in}|p{0.9in}|p{0.5in}|p{0.9in}|p{0.5in}|p{0.8in}|} \hline 
	x &  & - & (-,1) & 1 & (1,$+\infty$) \\ \hline 
	f'(x) & + & 0 & - & 0 & + \\ \hline 
	f'(x) & ${\nearrow}$ & 极大值 & ${\searrow}$ & 极小值 & ${\nearrow}$ \\ \hline 
\end{tabular}

显然函数f(x)在x=1处取极小值,符合题意,此时f(2)=18.

当a=-3,b=3时,$f'(x)=3x^{2}-6x+3=3(x-1)^{2}\ge0$,

所以 f(x)在x=1处没有极值,不合题意.

综上可知f(2)=18.



知识:函数的最值与导数

难度:2

题目:等差数列${a_{n}}$中的$a_{1}$,$a_{4\ 031}$是函数$f(x)=\frac{1}{3}x^{3}-4x^{2}+6x-1$的极值点,则$\log_{2}a_{2\ 016}$的值为(  )

A.2    

B.3    

C.4    

D.5

解析:因为$f'(x)=x^{2}-8x+6$,且$a_{1}$,$a_{4\ 031}$是函数$f(x)=\frac{1}{3}x^{3}-4x^{2}+6x-1$的极值点,所以$a_{1}$,$a_{4\ 031}$是方程$x^{2}-8x+6=0$的两个实数根,则$a_{1}+a_{4\ 031}=8$.而${a_{n}}$为等差数列,所以$a_{1}+a_{4\ 031}=2a_{2\ 016}$,即$a_{2\ 016}=4$,从而$\log_{2}a_{2\ 016}=\log_{2}4=2$.故选A.

答案:A

\end{document}

